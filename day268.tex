\section{ദിവസം 268}

\slokam{
ഭോഗാഭോഗതിരസ്കാരൈ: കാര്യം നേയം ശനൈര്‍മന:\\
രസാപഹാരൈസ്തജ്ഞേന കാലേനാജീര്‍ണ്ണപര്‍ണ്ണവത്  (5/50/56)\\
}

വസിഷ്ഠന്‍ തുടര്‍ന്നു: ക്ഷമയും അവധാനതയും കൊണ്ട് സമ്പന്നവും  അപരിമേയവുമായ ബുദ്ധിശക്തിയാല്‍ അശാപാശങ്ങളുടെ, പ്രത്യാശകളുടെ കെട്ടുകള്‍ പൊട്ടിച്ച് ധര്‍മ്മാധര്‍മ്മങ്ങള്‍ക്കുമപ്പുറം ഉയരുക. ആത്മജ്ഞാനത്തില്‍ വേരുറച്ചവനെ സംബന്ധിച്ചിടത്തോളം കൊടിയ വിഷംപോലും അമൃതസമാനമാണ്. അനന്താവബോധത്തെ മൂടി മറയ്ക്കുന്ന അജ്ഞാനാന്ധകാരം മൂലമാണ് ലോകമെന്ന ഈ ഭ്രമക്കാഴ്ച മനസ്സില്‍ സംജാതമാവുന്നത്.  

എന്നാല്‍ അനന്തവും അപരിമേയവും ഉപാധികളില്ലാത്തതുമായ ആത്മജ്ഞാനത്തില്‍ അടിയുറച്ചാല്‍പ്പിന്നെ ലോകമെന്ന ഭ്രമക്കാഴ്ച നല്‍കുന്ന മായയ്ക്കവസാനമായി. പിന്നെ നാനാ ദിശകളില്‍നിന്നും ജ്ഞാനത്തിന്റെ വിവേകവെളിച്ചം ലോകം മുഴുവന്‍ പ്രസരിക്കുകയായി. ആത്മജ്ഞാനത്തില്‍ അഭിരമിച്ച് ആ അമൃതാസ്വദിക്കുന്നവര്‍ക്ക് ഇന്ദ്രിയസുഖങ്ങള്‍ തുലോം വേദനാജനകമത്രേ. ആത്മജ്ഞാനികളുമായി മാത്രമേ നമുക്ക് സംഗം അഭികാമ്യമായുള്ളു. കാരണം മറ്റുള്ളവര്‍ മനുഷ്യവേഷം കെട്ടിയ കഴുതകളത്രേ!. ഗജകേസരികള്‍ നീണ്ട കാലടിവെച്ചു നടക്കുന്നതുപോലെ ബോധതലത്തിന്റെ ഉത്തുംഗസീമകളിലെത്തിയ ഋഷിവര്യന്മാര്‍ , വീണ്ടും ആ പാതയില്‍ ഉയര്‍ന്നുയര്‍ന്നു പോവുന്നു.

അവര്‍ക്ക് ബാഹ്യമായ സഹായങ്ങള്‍ ഒന്നുമില്ല. അവരുടെ പാതയെ ഭാസുരമാക്കാന്‍ പോന്നത്ര പ്രഭയുള്ള മറ്റൊരു സൂര്യന്മാരുമില്ല. ആത്മജ്ഞാനം മാത്രമാണവരുടെ പാതയിലെ വെളിച്ചം. വാസ്തവത്തില്‍ സൂര്യചന്ദ്രന്മാര്‍ അവരുടെ പ്രജ്ഞയ്ക്ക് വിഷയമല്ല. കാരണം അവര്‍ വിഷയധാരണകള്‍ക്കപ്പുറം, അറിവുകള്‍ക്കപ്പുറം,  എത്തിയിരിക്കുന്നു. മധ്യാഹ്നസൂര്യന്റെ വെളിച്ചത്തില്‍ ചെറിയൊരു വിളക്കിനെന്താണ് പ്രസക്തി? ഔന്നത്യത്തിന്റെ ഉച്ചകോടിയിലുള്ള എല്ലാറ്റിന്റെയും മുകളിലാണ് ആത്മജ്ഞാനി. പ്രഭ, മാഹാത്മ്യം, ശക്തി, എന്നുവേണ്ട എല്ലാ സല്‍ഗുണങ്ങളുടെയും പൂര്‍ണ്ണത അയാളില്‍ കാണാം. 
 
ഈ മഹര്‍ഷിമാര്‍ സൂര്യനെപ്പോലെ, ചന്ദ്രനെപ്പോലെ, അഗ്നിയെപ്പോലെ, നക്ഷത്രങ്ങള്‍ ഒന്നുചേര്‍ന്നതുപോലെ, പ്രഭാപൂരിതരാണ്. അതുപോലെ ആത്മജ്ഞാനം ലഭിക്കാത്തവര്‍ കീടാണുക്കളേക്കാള്‍ നികൃഷ്ടരത്രേ. ഭ്രമക്കാഴ്ച എന്ന ദുര്‍ഭൂതം ബാധിക്കുന്നത് ആത്മജ്ഞാനമില്ലാത്തവരെ മാത്രമാണ്. ആജ്ഞാനി എപ്പോഴും ദു:ഖാകുലനാണ്. ആ ദു:ഖങ്ങളെ ഇല്ലായ്മ ചെയ്യാന്‍ അയാള്‍ ഉഴറി നടക്കുന്നുണ്ടെങ്കിലും അവ അയാളെ വിട്ടുപോകുന്നില്ല. അയാള്‍ നടക്കുന്നൊരു ശവത്തിനു തുല്യം!. ആത്മജ്ഞാനം ലഭിച്ച ഋഷിയെ മാത്രമേ ജീവിക്കുന്ന, ചേതനയുറ്റ, ഒരു ജീവിയെന്ന് വിളിക്കാനാവൂ.  

നിബിഢമായ കാര്‍മേഘങ്ങള്‍ സൂര്യനെ മറയ്ക്കുന്നതുപോലെ മനസ്സ് മാലിന്യം നിറഞ്ഞ് അജ്ഞതയിലാണ്ടിരിക്കുമ്പോള്‍ ആത്മജ്ഞാനം മൂടിമറയ്ക്കപ്പെടുന്നു. “അതിനാല്‍ ഇതുവരെ അനുഭവിക്കാത്ത സുഖങ്ങള്‍ക്കായും പണ്ടനുഭവിച്ച സുഖാനുഭവത്തിന്റെ ആവര്‍ത്തനത്തിനായുമുള്ള ത്വരകള്‍ നാമുപേക്ഷിക്കണം.  സുഖാന്വേഷണമെന്ന വ്യര്‍ഥകര്‍മ്മങ്ങള്‍ ഉപേക്ഷിക്കുമ്പോള്‍ മനസ്സില്‍ അവകളോടുള്ള പ്രതിപത്തി താനേ കുറഞ്ഞു വരും” എന്നാല്‍ ശരീരം, ഭാര്യ, വീട്, മക്കള്‍ , കുടുംബം തുടങ്ങിയ അനാത്മവസ്തുക്കളോടുള്ള ബന്ധം മനസ്സിനെ അധോഗതിയില്‍ നയിക്കുന്നു.

‘ഞാന്‍’, ‘നീ’, എന്നീ ധാരണകള്‍ മനസ്സിനെ മന്ദവും ജഢിലവുമാക്കുന്നു. ഇത് വാര്‍ദ്ധക്യമെത്തുമ്പോള്‍ കൂടുതല്‍ വഷളാവുന്നു. ദു:ഖം, ആഗ്രഹം, മനോവിഷമങ്ങള്‍ , ഗ്രാഹ്യ-ത്യാജ്യ വിഷയങ്ങള്‍ , അത്യാര്‍ത്തി, ധനത്തിനും ലൈംഗികാസ്വാദനത്തിനുമായുള്ള ത്വര, മറ്റ്‌ ഇന്ദ്രിയ സുഖങ്ങള്‍ക്കായുള്ള തൃഷ്ണ തുടങ്ങിയ എല്ലാം അജ്ഞാനാവരണത്തിന്റെ ഫലമായുണ്ടാവുന്ന ഭ്രമാത്മകത തന്നെയാണ്.