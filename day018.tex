\newpage
\section{ദിവസം 018}

\slokam{
ഇതി മേ ദോഷദാവാഗ്നിദഗ്ധേ മഹതി ചേതസി\\
പ്രസ്ഫുരന്തി ന ഭോഗാശാ മൃഗതൃഷ്ണാ:  സര:സ്വിവ (1/29/1)\\
}

രാമന്‍ തുടര്‍ന്നു: ഈ ക്ഷണഭംഗുരമായ ലോകത്ത്‌ കാണപ്പെടുന്നതെല്ലാം ഒരു സ്വപ്നം പോലെ തോന്നുന്നു. പണ്ട്‌ വലിയൊരു പര്‍വ്വതമുണ്ടായിരുന്നിടത്ത്‌ ഇന്ന് വലിയൊരു ഗര്‍ത്തമാണു കാണുന്നത്‌.. അതുപോലെ പണ്ടുണ്ടായിരുന്ന കിടങ്ങിന്റെ സ്ഥാനത്ത്‌ ഇന്നൊരു മലയാണ്‌. . ഘോരവനം ഉണ്ടായിരുന്നിടങ്ങളില്‍ ഇപ്പോള്‍ വന്‍ നഗരങ്ങളാണുള്ളത്‌.. ഫലഭൂയിഷ്ടമായ കൃഷിയിടങ്ങള്‍ മരുപ്പറമ്പുകളായി മാറിപ്പോയി. അതുപോലെയാണ്‌ മനുഷ്യന്റെ ശാരീരികമായ മാറ്റങ്ങളും, ജീവിതശൈലിയിലും ഭാഗ്യത്തിലുമുള്ള വ്യതിയാനങ്ങളും. ജനമരണചക്രമെന്നത്‌ പ്രാപ്തിയും വൈദഗ്ധ്യവുമുള്ളൊരു നര്‍ത്തകിയാണ്‌. അവളുടെ ഉടയാടയുടെ ഞൊറികള്‍ ജീവാത്മാക്കള്‍. അവരെ സ്വര്‍ഗ്ഗത്തിലേയ്ക്കുയര്‍ത്തുന്നതും നരകത്തിലേയ്ക്കു താഴ്ത്തുന്നതും ഭൂമിയിലേയ്ക്ക്‌ തിരികെ കൊണ്ടുവരുന്നതും അവളുടെ അംഗവിക്ഷേപങ്ങള്‍. ഏറ്റവും മഹത്തരമായ പ്രവര്‍ത്തനങ്ങളും മതകര്‍മ്മങ്ങളും പോലും ഓര്‍മ്മയിലേയ്ക്ക്‌ മാത്രമായി ഒതുങ്ങിപ്പോകുന്നു.

മനുഷ്യന്‍ മൃഗങ്ങളായി പിറക്കുന്നു. മൃഗങ്ങള്‍ മനുഷ്യരായും. ദേവതകള്‍ക്കുപോലും അവരുടെ ദിവ്യത്വം നഷ്ടമാവുന്നു. ഇവിടെ ശാശ്വതമായി എന്തുണ്ട്‌? സൃഷ്ടി കര്‍ത്താവായ ബ്രഹ്മാവ്‌, സംരക്ഷകനായ വിഷ്ണു, സംഹാരകനായ രുദ്രന്‍, എല്ലാവരും വിനാശത്തിലേയ്ക്ക്‌ നീങ്ങുന്നു. ഇഹലോകത്ത്‌ ഇന്ദ്രിയവസ്തുക്കള്‍ സുഖപ്രദമാണെന്നുള്ള തോന്നല്‍ ഈ അനിവാര്യമായ വിനാശത്തെപ്പറ്റി ഓര്‍മ്മയുണ്ടാകുമ്പോള്‍ ഇല്ലാതാവുന്നു. കളിമണ്ണുകൊണ്ട്‌ കുട്ടി പലവിധ രൂപങ്ങളും ഉണ്ടാക്കുന്നതുപോലെ വിശ്വനിയന്താവ്‌ നവംനവങ്ങളായ പല സൃഷ്ടികളും ഉണ്ടാക്കി നിലനിര്‍ത്തി നശിപ്പിച്ചുകൊണ്ടേയിരിക്കുന്നു.

"ലോകത്തിന്റെ ഈദൃശമായ കുറവുകളെപ്പറ്റി ചിന്തിക്കാനിടയായതുകൊണ്ട്‌ എന്റെ മനസ്സില്‍ അനഭികാമ്യമായ വാസനകള്‍ ഉണ്ടാവുന്നില്ല. ജലനിരപ്പിനു മുകളില്‍ മരീചിക കാണപ്പെടാത്തതുപോലെ എന്റെയുള്ളില്‍ ഇന്ദ്രിയസുഖാസക്തി ഉയരുന്നുമില്ല."

ഇഹലോകദൃശ്യങ്ങളും രമണീയതയും എന്നില്‍ കയ്പ്പുളവാക്കുന്നു. സുഖോദ്യാനങ്ങളില്‍ അലയാന്‍ എനിയ്ക്കു താല്‍പ്പര്യമില്ല. പെണ്‍കുട്ടികളുടെ സാമീപ്യം എനിയ്ക്കു ഹൃദ്യമല്ല. ധനസമ്പാദനത്തില്‍ എനിക്ക്‌ മതിപ്പില്ല. മന:ശാന്തിയോടെയിരിക്കാന്‍ ഞാന്‍ ഇഷ്ടപ്പെടുന്നു. നിരന്തരം ചഞ്ചലമായ ഈ മായക്കാഴ്ച്ചയില്‍ നിന്നും ഹൃദയത്തെ എങ്ങിനെ മോചിപ്പിക്കാമെന്നാണ്‌ ഞാന്‍ എപ്പോഴും ചിന്തിക്കുന്നത്‌. എനിയ്ക്കു മരണത്തോടോ ജീവിതത്തോടോ ആസക്തിയില്ല. കാമപ്പനിയുടെ പിടിയില്‍പ്പെടാതെ ഞാന്‍ ഞാനായി നിലകൊള്ളുന്നു. രാജപദവികൊണ്ടോ ധനസമൃദ്ധികൊണ്ടോ സുഖഭോഗങ്ങളെക്കൊണ്ടോ എനിയ്ക്കെന്തുപ്രയോജനം? എല്ലാം അഹംകാരത്തെ പരിപോഷിപ്പിക്കാനേ ഉതകൂ. എന്നില്‍ അഹംകാരം ഇല്ല. ഇപ്പോള്‍ വിവേകവിജ്ഞത്തില്‍ അടിയുറച്ചില്ലെങ്കില്‍ ഇത്തരം ഒരവസരം എനിക്കിനിയെന്നാണു കിട്ടുക? ഇന്ദ്രിയസുഖങ്ങളാകുന്ന വിഷത്തിന്റെ വാസനാമാലിന്യം പല ജന്മങ്ങള്‍ നിലനില്‍ക്കുന്നു.   ആത്മജ്ഞാനം ലഭിച്ചവനു മാത്രമേ ഇതില്‍ നിന്നു മോചനമുള്ളു. 

അതിനാല്‍ മഹര്‍ഷേ, ഭയത്തില്‍നിന്നും, ദുരിതത്തില്‍നിന്നും, ആശങ്കകളില്‍ നിന്നും, മോചനം കിട്ടാനുള്ള മാര്‍ഗ്ഗം എനിയ്ക്കുപദേശിച്ചുതന്നാലും. അങ്ങയുടെ ഉപദേശത്തിന്റെ വെളിച്ചം എന്നിലെ അജ്ഞാനാധകാരത്തെ അകറ്റുമാറാകട്ടെ.

