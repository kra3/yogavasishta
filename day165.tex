\section{ദിവസം 165}

\slokam{
അജ്ഞസ്യേയമനന്താനാം ദു:ഖാനാം കോശമാലികാ\\
ജ്ഞസ്യ ത്വിയമനന്താനാം സുഖാനാം കോശമാലികാ (4/23/18)\\
}

വസിഷ്ഠൻ തുടർന്നു: പരമപദത്തിലേയ്ക്കുന്നം വച്ചു ചലിക്കുന്നവർ, ശരീരാവസ്ഥയിൽ ഉള്ളപ്പോൾ കുശവന്റെ ചക്രത്തിന്റെ ആയംകൊണ്ടുള്ള തുടർച്ചുറ്റൽ പോലെയാണ്‌ വർത്തിക്കുന്നത്. അവർ ചെയ്യുന്ന പ്രവർത്തികൾ കർമ്മങ്ങളാകുന്നില്ല. അവ വാസനകളെയുണ്ടാക്കുന്നുമില്ല. അവർക്ക് ശരീരം സ്വാഭീഷ്ടനിവൃത്തിക്കും മുക്തിസമ്പാദനത്തിനുമുള്ള ഒരുപാധി മാത്രമത്രേ. ശരീരംകൊണ്ട് അയാൾക്ക് യാതൊരുവിധ ദു:ഖവും അനുഭവിക്കേണ്ടിവരുന്നില്ല.

“അജ്ഞാനിക്ക്, ശരീരം ദു:ഖസ്രോതസ്സാകുമ്പോൾ പ്രബുദ്ധനത് അനന്തമായ ആനന്ദത്തിന്റെ ഒടുങ്ങാത്ത ഉറവയാണ്‌.” നിലനില്‍ക്കുന്നിടത്തോളം കാലം ജ്ഞാനി സ്വശരീരത്തിൽത്തന്നെ ആഹ്ളാദം കണ്ടെത്തുന്നു. പ്രബുദ്ധതയിൽ അഭിരമിക്കാനുള്ള ഉപാധിയായി അതിനെ ഉപയോഗിക്കുന്നു. ദേഹത്തിനു മരണമുണ്ടാകുമ്പോൾ അതൊരു നഷ്ടമായി അയാൾ കണക്കാക്കുന്നതേയില്ല. പ്രബുദ്ധതയിൽ വിരാജിക്കുന്നവന്‌ ശരീരം ആഹ്ളാദവാഹിനിയാണ്‌... കാരണം ഈ ലോകത്ത് സർവ്വസ്വതന്ത്രമായി വിഹരിക്കുവാൻ അയാൾക്ക് സഹായകമായുള്ളത് ഈ ദേഹം തന്നെയാണല്ലോ. ശരീരമെന്നത് വിജ്ഞാനത്തിന്റെ വാഹനമാണ്‌... ജ്ഞാനിക്ക് ശരീരം പലവിധത്തിലുള്ള അനുഭവങ്ങൾ നല്‍കുന്നു; അവരെ മറ്റുള്ളവരുടെ സ്നേഹാദരങ്ങൾക്ക് പാത്രമാക്കാന്‍ ഇടവരുത്തുകയും ചെയ്യുന്നു. ആകയാൽ ശരീരം അയാൾക്ക് ഒരു നേട്ടം തന്നെയാണ്‌... ജ്ഞാനി ശരീരമെന്ന നഗരത്തിൽ സ്വർഗ്ഗം ഭരിക്കുന്ന ഇന്ദ്രനെന്നപോലെ പൂർണ്ണ സ്വതന്ത്രനാണ്‌... ഈ ശരീരം അയാളെ കാമത്തിനോ ലോഭത്തിനോ വശംവദനാക്കുന്നില്ല. അജ്ഞതയെയോ ഭയത്തെയോ തീണ്ടാനിടവരുത്തുന്നുമില്ല.

ജ്ഞാനിയുടെ ശരീരം നയിക്കപ്പെടുന്നത് ഇന്ദ്രിയസുഖങ്ങളുടെ വികാരപ്രചോദനങ്ങളാലല്ല, മറിച്ച് ധ്യാനാവസ്ഥയുടെ പ്രശാന്തതയാലാണ്‌.. ശരിരമെടുത്ത ജീവൻ ദേഹമുള്ളിടത്തോളം അതുമായി സഹവാസത്തിലാണെങ്കിലും അതു പോയിക്കഴിഞ്ഞാൽപ്പിന്നെ ആ ബന്ധം നിലനിർത്തുന്നില്ല. ഒരു കുടമുള്ളിടത്തോളം അതിനുള്ളിൽ വായുവുണ്ട്. അതുടഞ്ഞാൽ, ആ വായുവിന്‌ കുടവുമായുള്ള ബന്ധം എങ്ങിനെയാണോ അതുപോലെയാണ്‌ ദേഹവുമായി ദേഹിക്കുള്ള ചാർച്ച. പരമശിവനെ ഏറ്റവും കൊടിയ വിഷം കുടിച്ചിട്ടുപോലും അതു ബാധിക്കുകയുണ്ടായില്ല. മറിച്ച് അദ്ദേഹത്തിന്റെ ആകർഷണീയത വർദ്ധിക്കുകയാണല്ലോ ചെയ്തത് (നീലകണ്ഠന്‍ ). ശരീരത്തിൽ അനുഭവവേദ്യമാകുന്ന സുഖദു:ഖാദികൾ ജ്ഞാനിയെ ബന്ധിക്കുന്നില്ല. അവ അദ്ദേഹത്തിനെ ജനന മരണചക്രത്തിലേയ്ക്ക് തിരികെ വലിച്ചിഴക്കുന്നുമില്ല. ഒരുവനെ അയാൾ കള്ളനെന്നറിഞ്ഞുകൊണ്ട് തന്നെ അയാളുമായി സൌഹൃദഭാവത്തില്‍ ഇടപെടുന്നതായാൽ അയാളെ സുഹൃത്താക്കാം. അതുപോലെ പദാർത്ഥങ്ങളെ അവയുടെ യാഥാർത്ഥ്യമറിഞ്ഞ് അറിവോടെ അനുഭവിച്ചാൽ അവ ആഹ്ളാദദായകമാണ്‌.. ജ്ഞാനി സംശയങ്ങളെല്ലാം ഒഴിഞ്ഞവനാണ്‌.. സ്വയമുണ്ടാക്കിയ ഒരു പ്രതിച്ഛായയാൽ അയാൾ ബന്ധിതനുമല്ല. അങ്ങിനെയുള്ളവൻ തന്റെ ശരീരത്തിന്റെ ചക്രവർത്തിയാണ്‌..

അതിനാൽ എല്ലാവിധ സുഖാസക്തികളും ഉപേക്ഷിച്ച് വിജ്ഞാനപദം പ്രാപിക്കുക. ശരിയായി നിയന്ത്രണംവന്ന മനസ്സുള്ളപ്പോഴേ സന്തോഷം ആസ്വദിക്കാനാവൂ. ബന്ധനസ്ഥനായ ഒരു രാജാവ് സ്വതന്ത്രനായ ശേഷം കിട്ടുന്ന ഒരു കഷണം അപ്പം എത്ര സന്തോഷത്തോടെയാണ്  ആസ്വദിക്കുന്നത്? ഒരിക്കലും ബന്ധനത്തിന്റെ അനുഭവമുണ്ടായിട്ടില്ലാത്ത രാജാവിന്‌ അടുത്തുള്ളൊരു രാജ്യത്തെ കീഴടക്കി തന്റെ വരുതിയിലാക്കിയാൽപ്പോലും ഇത്ര സൗഖ്യം ഉണ്ടാവുകയില്ല. ആയതിനാൽ ജ്ഞാനി മനസ്സേന്ദ്രിയങ്ങളെ കീഴടക്കി നിയന്ത്രിക്കാൻ തീവ്രമായി പരിശ്രമിക്കുന്നു. കാരണം അത്തരം കീഴടക്കൽ പുറമേയുള്ള ശത്രുക്കളെ വെല്ലുന്നതിനേക്കാൾ ഉന്നതമായ വിജയമത്രേ. 

