\section{ദിവസം 290}

\slokam{
ഏഷൈവ രാമാ സൌഷുപ്തി സ്ഥിതിരഭ്യാസയോഗത:\\
പ്രൌഢാ സതീ തുര്യമിതി കഥിതാ തത്വകോവിദൈ:     (5/70/26)\\
}

വസിഷ്ഠന്‍ തുടര്‍ന്നു: രാമാ, എല്ലായ്പ്പോഴും ഉചിതമായ കാര്യങ്ങള്‍ ചെയ്തുകൊണ്ടിരിക്കണം . എന്നാല്‍ ചിന്തകളിലോ വസ്തുക്കളിലോ മനസ്സിനു സംഗമുണ്ടാകാതെയായിരിക്കണം കര്‍മ്മങ്ങളിലേര്‍പ്പെടേണ്ടത്. അവ മുകളിലുള്ള  സ്വര്‍ഗ്ഗത്തെയോ താഴെയുള്ള നരകങ്ങളെയോ മറ്റു ദിശകളെയോ ലക്ഷ്യമാക്കി ആകരുത്. അവ ബാഹ്യബന്ധങ്ങളെയോ അന്തരേന്ദ്രിയങ്ങളുടെ സ്വാഭാവിക ചലനത്തെയോ പ്രാണശക്തിയേയോ ആശ്രയിച്ചാകരുത്. മനസ്സ് ശിരസ്സിലോ, അണ്ണാക്കിലോ പുരികങ്ങള്‍ക്കിടയിലോ മൂക്കിന്‍തുമ്പിലോ വായിലോ കണ്ണിലോ ഒന്നും കേന്ദ്രീകൃതമാകരുത്.       

അത് അന്ധകാരത്തിലോ വെളിച്ചത്തിലോ ഹൃദയഗുഹയിലോ അഭിരമിക്കരുത്. അത് ജാഗ്രദവസ്ഥയിലോ, സ്വപ്നത്തിലോ സുഷുപ്തിയിലോ പിടിച്ചുനിര്‍ത്തരുത്. വിടര്‍ന്നു വികസിച്ച ശുദ്ധമായ ആകാശത്തും അതിനെ തളച്ചിടരുത്. നിറഭേദങ്ങളിലോ, അവയുടെ വിന്യാസങ്ങളിലോ ചലനത്തിലോ, സുസ്ഥിരതയിലോ, ആദിയിലോ മദ്ധ്യത്തിലോ അന്ത്യത്തിലോ മറ്റെവിടെയെങ്കിലുമോ മനസ്സ് ഉറയ്ക്കരുത്. മുന്നിലോ പിന്നിലോ അടുത്തോ അകലത്തോ വസ്തുക്കളിലോ ആത്മാവിലോ പോലും മനസ്സുടക്കി നില്‍ക്കരുത്.

ഇന്ദ്രിയാനുഭവങ്ങള്‍ , ഭ്രമാത്മകമായ സന്തോഷനിമിഷങ്ങള്‍ , ധാരണകള്‍ , പ്രതീതികള്‍ , എന്നിവയ്ക്കൊന്നും മനസ്സിനുമേല്‍ സ്വാധീനം ഉണ്ടാവാന്‍ ഇടകൊടുക്കരുത്. മനസ്സ് ശുദ്ധാവബോധത്തില്‍ , ശുദ്ധമായ വെറും ബോധമായി മാത്രം നിലകൊള്ളട്ടെ. അതിന്, ലോകത്തിന്റെ വ്യര്‍ത്ഥതയെപ്പറ്റി അറിഞ്ഞുവെക്കാനായി മാത്രം ബാഹ്യമായ ചെറിയൊരു ചിന്താശകലം വേണമെങ്കിലാവാം. അങ്ങിനെ എല്ലാ സംഗവും ഇല്ലാതെയാകുമ്പോള്‍ ജീവന്‍, ‘അജീവന്‍’ ആകും. അതിനുശേഷം എന്തു സംഭവിച്ചാലും അത് വെറും സംഭവം മാത്രം. കര്‍ത്താവില്ലാത്ത കര്‍മ്മം.  കര്‍മ്മനിരതമാണെങ്കിലും അല്ലെങ്കിലും അവയ്ക്ക് ബന്ധിക്കാന്‍ ആരാണുള്ളത്? അങ്ങിനെ നിര്‍മമമായ ജീവന് കര്‍മ്മഫലങ്ങളനുഭവിക്കേണ്ടി വരുന്നില്ല.  

ഒടുവില്‍ അല്‍പം ബാക്കിനില്‍ക്കുന്ന ശേഷവസ്തുബോധത്തെയും വേണ്ടെന്നു വെച്ച് ജീവന്‍ പരമപ്രശാന്തതയില്‍ വിലീനമാവട്ടെ. അങ്ങിനെ മുക്തനായ ഒരാള്‍ മറ്റുള്ളവരുടെ ദൃഷ്ടിയില്‍ കര്‍മ്മനിരതനായിരുന്നാലുമില്ലെങ്കിലും അയാള്‍ പിന്നെ ദു:ഖത്തിനും ഭയത്തിനും വശംവദനാവുകയില്ല. എല്ലാവരും അദ്ദേഹത്തെ സ്നേഹിക്കുന്നു, ആദരിക്കുന്നു. മറ്റുള്ളവരുടെ കണ്ണില്‍ അയാള്‍ ചഞ്ചലനായി കാണപ്പെട്ടാലും ഉള്ളാലെ അയാള്‍ ജ്ഞാനത്തില്‍ സമാരൂഢനാണ്. സുഖദു:ഖങ്ങള്‍ അയാളുടെ ബോധത്തിന് നിറഭേദം ഉണ്ടാക്കുകയില്ല. ലോകത്തിന്‍റെ വര്‍ണ്ണത്തിളക്കം അയാളുടെ കണ്ണഞ്ചിപ്പിക്കുന്നില്ല. 

ആത്മവിദ്യയുടെ നിറവില്‍ സദാ ധ്യാനനിരതനെപ്പോലെ, അതുകൊണ്ടുതന്നെ ലോകത്തിലെ ഒന്നിനോടും മമതയില്ലാതെ ജ്ഞാനി ജീവിക്കുന്നു. പരസ്പര വിരുദ്ധങ്ങളായ ദ്വന്ദശക്തികളുടെ സ്വാധീനത്തില്‍ അല്ലാത്തതുകൊണ്ട് ജാഗ്രദിലും അയാള്‍ ദീര്‍ഘസുഷുപ്തിയിലത്രേ. ഈയവസ്ഥയില്‍ മനസ്സില്‍ ചിന്തകളില്ല. ഉള്ളത് പ്രശാന്തതയുടെ നേരനുഭവം മാത്രം. ‘ജാഗ്രദിലെ ദീര്‍ഘസുഷുപ്തി’ എന്നാണീ അവസ്ഥയ്ക്ക് പറയുന്നത്. അതില്‍ അഭിരമിക്കുന്നവരുടെ ജീവിതം സ്വേച്ഛയാലല്ല മുന്നോട്ടു പോവുന്നത്. എല്ലാ മനോവികലതകളും അയാളില്‍ അസ്തമിച്ചിരിക്കുന്നു. ആയുസ്സ് ദീര്‍ഘമോ ഹൃസ്വമോ എന്നയാള്‍ക്ക് ആശങ്കയില്ല.  

"രാമാ, ഇങ്ങിനെയുള്ള ‘ജാഗ്രദില്‍ ദീര്‍ഘസുഷുപ്തി’ എന്ന അവസ്ഥ പക്വമാവുമ്പോള്‍ അത് ‘തുരീയം’ അല്ലെങ്കില്‍ നാലാമത്തെ അവസ്ഥ എന്നറിയപ്പെടുന്നു. ആ സ്ഥിതിയില്‍ ഉറച്ചുനിന്നുകൊണ്ട് ഋഷികള്‍ വിശ്വപ്രപഞ്ചത്തെ ബോധത്തിന്റെ വിശ്വലീലയ്ക്കുള്ള  കളിത്തട്ടായും  ജീവിതത്തെ പ്രപഞ്ചനൃത്തമായും അറിയുന്നു." ഭയദു:ഖാദികളില്‍ നിന്നും പൂര്‍ണ്ണമായും മുക്തരായി തുരീയാവസ്ഥയെ പ്രാപിച്ചവര്‍ വീണ്ടും തെറ്റുകളിലേയ്ക്ക് വീഴാന്‍ ഇടവരുന്നില്ല. അയാള്‍ ആനന്ദത്തില്‍ എന്നെന്നേയ്ക്കുമായി ആമഗ്നനാണ്. ഈയവസ്ഥ്യ്ക്കുമപ്പുറം പരമമായ ആനന്ദത്തിന്റെ വിവരണാതീതമായ മറ്റൊരു തലത്തിലേയ്ക്കാണ് അയാളുടെ ഗമനം. ആ അഭൌമതലത്തെ നമ്മുടെ അറിവിന്റെ പരിധിക്കുള്ളില്‍ നിന്നുകൊണ്ട് വിവരിക്കാനോ അറിയാനോ ആവില്ല.

