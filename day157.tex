\section{ദിവസം 157}

\slokam{
പ്രതിഭാസവശാദസ്തി  നാസ്തി വസ്ത്വവലോകനാത്\\
ദീർഘസ്വപ്നോ ജഗജ്ജാലമാലാനം ചിത്തദന്തിന: (4/17/18)\\
}

രാമൻ ചോദിച്ചു: മഹർഷേ, ശുക്രൻ സ്വർഗ്ഗത്തിൽ പോകണമെന്നാഗ്രഹിച്ചത് അപ്രകാരം തന്നെ നിറവേറിയല്ലോ. എന്നാൽ മറ്റുള്ളവരുടെ ആഗ്രഹങ്ങൾ എന്തുകൊണ്ടാണ്‌ അതുപോലെ  നടപ്പിലാവാത്തത്?

വസിഷ്ഠൻ പറഞ്ഞു: ശുക്രന്റെ മനസ്സ് അതീവ നിർമ്മലമായിരുന്നു. അത് അദ്ദേഹത്തിന്റെ ആദ്യജന്മമായിരുന്നതിനാൽ പൂർവ്വജന്മവാസനകളാകുന്ന കളങ്കങ്ങളൊന്നും അദ്ദേഹത്തിന്റെ മനസ്സിനു ഭാരമുണ്ടാക്കിയിരുന്നില്ല. എല്ലാ ആശകളും അടങ്ങിയ മനസ്സ് നിർമ്മലമത്രേ. അങ്ങിനെയുള്ള ശുദ്ധമനസ്സ് ഇച്ഛിക്കുന്നതെന്തും സാധിതമാകുന്നു. ശുക്രനാൽ സാധിച്ച ഇക്കാര്യം ആർക്കും നേടാവുന്നതേയുള്ളു.

ഒരോ ജീവനിലും ഈ ലോകമൊരു വിത്തുപോലെ എല്ലാ സാദ്ധ്യതകളും ഉള്ളിലടങ്ങി നിലനിൽക്കുന്നു. പിന്നീട് അതിനൊരു മുളപൊട്ടി വൃക്ഷമായി പടർന്നു പന്തലിച്ചു  പ്രകടമാവുന്നു. ഇപ്രകാരം ഓരോരുത്തരം അവരവരുടെ ഭാവനയ്ക്കനുസരിച്ച് ലോകത്തെ സൃഷ്ടിക്കുന്നു. വാസ്തവത്തിൽ ലോകം ഉദിക്കുകയോ അസ്തമിക്കുകയോ ചെയ്യുന്നില്ല. അതെല്ലാം വിഭ്രാന്തമനസ്സിന്റെ ഭാവന മാത്രം. നമ്മിലെല്ലാം ഭാവനയുടെ ഒരു ലോകമുണ്ട്. ഒരാളുടെ സ്വപ്നം അന്യർക്ക് അറിയാത്തതുപോലെ അയാളുടെ ലോകവും മറ്റുള്ളവർക്ക് അറിയാനാവില്ല.

ദേവതമാർ, അസുരന്മാർ എന്നുവേണ്ട ഭൂതപിശാചുക്കളുമെല്ലാം മനസ്സിന്റെ മോഹവിഭ്രാന്തികൾ ഉടലെടുത്തുണ്ടാവുന്നവയാണ്. നമ്മളും അങ്ങിനെ ജനിച്ചതാണു രാമാ. ശുദ്ധചിത്തത്തിൽ നിന്നും ഉണ്ടായതെങ്കിലും ജീവികൾ അസത്തിനെ സത്തായി ധരിക്കുന്നു. അനന്താവബോധത്തിലെ സൃഷ്ടിയുടെ ഉദ്ഭവം ഇങ്ങിനെയാണ്‌.. വസ്തുക്കൾ യാഥാർത്ഥ്യമല്ലെങ്കിലും ശൂന്യതയിൽ അത് യാഥാർത്ഥ്യമാണെന്നു കരുതപ്പെടുന്നു. എല്ലാവരും അവരവരുടെ ലോകം ഭാവനയിൽ കാണുന്നു. എന്നാൽ സത്യസാക്ഷാത്കാരത്തോടെ ഈ ലോകമെല്ലാം അവസാനിക്കും.

“ഈ ലോകം പ്രകടമായി നിലകൊള്ളുന്നത് പദാർത്ഥങ്ങളെ, വസ്തുക്കളെ നാം  കാണുന്നതുകൊണ്ടല്ല. ഭാവനയിലും കാഴ്ച്ചയിലും മാത്രമാണ്‌. അതുള്ളത്. അത് നീണ്ടൊരുസ്വപ്നമോ ജാലവിദ്യക്കാരന്റെ കൺകെട്ടുവിദ്യയോ പോലെയാണ്‌... മനസ്സാകുന്ന ആനയെ കെട്ടിയിട്ട സ്തംഭമാണത്.” മനസ്സാണ്‌ ലോകം. ലോകമാണു മനസ്സ്. അവയിൽ ഒന്നിന്റെ സത്യാവസ്ഥ- അയാഥാർത്ഥ്യമാണെന്ന സത്യം- അറിഞ്ഞാൽ മറ്റേതിന്റെ സത്യവും അറിഞ്ഞു. അതായത് രണ്ടും ഇല്ലാതാവുന്നു. മനസ്സ് ശുദ്ധമാവുമ്പോൾ അതിൽ സത്യം പ്രതിഫലിക്കുന്നു. അയാഥാർത്ഥ്യമായ ലോകമെന്ന ഈ കാഴ്ച്ച അങ്ങിനെ ഇല്ലാതാകുന്നു. നിരന്തരമായ ധ്യാനംകൊണ്ട് മനസ്സിനെ ശുദ്ധമാക്കാം.

രാമൻ ചോദിച്ചു: എങ്ങിനെയാണ്‌ ശുക്രന്റെ മനസ്സിൽ തുടർച്ചയായുള്ള ജന്മങ്ങൾ ഭാവനയായുദിച്ചത്?

വസിഷ്ഠൻ പറഞ്ഞു: ശുക്രന്റെ അച്ഛൻ ഭൃഗുമഹർഷി അദ്ദേഹത്തെ ജീവന്റെ പുനർജന്മങ്ങളെപ്പറ്റി പഠിപ്പിച്ചിരുന്നു. ആ പഠനം ശുക്രന്റെ മനസ്സിൽപ്പതിഞ്ഞ് പിന്നീടത് വാസനകളായും ഉപാധികളായും പരിണമിച്ചു. എല്ല ഉപാധികളുമൊഴിഞ്ഞു ശുദ്ധീകരിച്ച മനസ്സിലേ പരിപൂർണ്ണമായ പരിശുദ്ധി വീണ്ടും സംജാതമാവുകയുള്ളു. ഈ ശുദ്ധമനസ്സാണ്‌ മുക്തിപദമനുഭവിക്കുന്നത്. 

