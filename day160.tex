\section{ദിവസം 160}

\slokam{
ചിത്രാമൃതം നാമൃതമേവ വിദ്ധി ചിത്രാനലം നാനലമേവ വിദ്ധി\\
ചിത്രാംഗനാ നൂനമനംഗനേതി വാചാ വിവേകസ്ത്വവിവേക ഏവ (4/18/69)\\
}

വസിഷ്ഠൻ തുടർന്നു: ഓരോ ജീവനും സ്വപ്രാണബലത്തിന്റെ സഹായത്താൽ എന്തെന്തെല്ലാം, എങ്ങിനെയെല്ലാം സ്വയം സങ്കല്പ്പിച്ചുണ്ടാക്കുന്നുവോ അവ അനുഭവിക്കുന്നു. രാമ: നിന്റെ ഉൾക്കണ്ണിലെ വിവേകവെളിച്ചത്താൽ ഓരോ അണുവിലും എണ്ണമറ്റ ലോകങ്ങൾ പ്രകടമായിക്കൊണ്ടിരിക്കുന്നതു ദർശിച്ചാലും. എല്ലാവരുടെയും മനസ്സിൽ, ആകാശത്തിൽ, കല്ലുകളിൽ, അഗ്നിജ്വാലയിൽ, ജലത്തിൽ എല്ലാം എള്ളിൽ എണ്ണയെന്നപോലെ എണ്ണമറ്റ ലോകങ്ങൾ നിലകൊള്ളുന്നു എന്നറിയുക. മനസ്സ് തികച്ചും പരിശുദ്ധമാവുമ്പോഴാണ്‌ അത് ശുദ്ധ അവബോധമായി, അനന്താവബോധവുമായി വിലയിക്കുന്നത്.

പ്രത്യക്ഷലോകമെന്നത് എല്ലായിടത്തും പ്രകടമാവുന്ന ഒരു നീണ്ട സ്വപ്നമത്രേ. അത് സൃഷ്ടികർത്താവായ ബ്രഹ്മാവിന്റെയും മറ്റെല്ലാവരുടേയും സങ്കൽപ്പമാണ്‌.. അങ്ങിനെ സൃഷ്ടാവിന്റെ സങ്കൽപ്പത്തിൽ നിന്നുത്ഭവിച്ച് ഒരു സ്വപ്നത്തിൽ നിന്നു മറ്റൊന്നിലേയ്ക്ക് പ്രവേശിച്ച്, ഒരു ശരീരത്തിൽ നിന്ന് മറ്റൊന്നിലേയ്ക്ക് പരിക്രമിച്ച് പ്രത്യക്ഷ ലോകത്തിന്‌ മായികമായ ഒരുണ്മ തോന്നിപ്പിക്കുന്നു. സ്വപ്നസമയത്ത് ഈ മായക്കാഴ്ച്ച സത്യമത്രേ. ഓരോ അണുവിലും സർവ്വവിധത്തിലുമുള്ള അനുഭവസാദ്ധ്യതകൾ, വിത്തിൽ പൂക്കൾ, കായ്കൾ, ഇലകൾ എന്നിവപോലെ നിലകൊള്ളുകയാണ്‌. ഓരോ അണുവിലും അവിച്ഛിന്നമായ അനന്താവബോധമാണുള്ളത്. അതിനാൽ നിന്റെയുള്ളിലെ നാനാത്വം, ഏകത്വം എന്നിങ്ങനെയുള്ള ധാരണകളെല്ലാം ദൂരെക്കളയുക.

സമയം, ദൂരം, കർമ്മം (ചലനം), വസ്തുക്കൾ എന്നിവയെല്ലാം ആ അനന്താവബോധത്തിന്റെ വൈവിദ്ധ്യമാർന്ന ഭാവങ്ങളത്രേ. ബോധം അവയെ സ്വയം അനുഭവിക്കുന്നു. അത് ബ്രഹ്മാവിന്റെ ശരീരത്തിലോ ഒരു കീടത്തിന്റെ ശരീരത്തിലോ ആയാലും അനുഭവിക്കുന്നത് ബോധം തന്നെയാണ്‌.. ബോധത്തിന്റെ കണം പൂർണ്ണവളർച്ചയെത്തിയ ശരീരമായി സ്വയം അനുഭവങ്ങളെ സാക്ഷാത്കരിക്കുന്നു. ചിലർക്ക് വസ്തുക്കൾ അവനവനു പുറമേയുള്ളതായി തോന്നുന്നത് അനന്താവബോധം സർവ്വവ്യാപിയായതിനാലാണ്‌.. മറ്റു ചിലർ എല്ലാറ്റിനേയും ഉള്ളിൽക്കണ്ട് വികാസം പ്രാപിച്ച്, വിലയം പ്രാപിച്ച് നിലകൊള്ളുന്നു. ഇനിയും ചിലർ ഒരു സ്വപ്നാനുഭവത്തിൽ നിന്നും മറ്റൊന്നിലേയ്ക്ക്- ഈ പ്രത്യക്ഷലോകത്തിൽത്തന്നെ അലയുന്നു. എന്നാൽ തുലോം ചെറിയൊരു വിഭാഗം ആളുകൾ ഈ ലോക കാഴ്ച്ചയെന്നത് വെറും മായയാണെന്ന തിരിച്ചറിവോടെ, അനന്താവബോധം മാത്രമാണുണ്മയെന്നറിയുന്നു.

ഈ ബോധത്തിന്റെ പ്രഭാവത്താൽ ജീവനിൽ ലോകം പ്രകടമാവുന്നു. ജീവനുള്ളിൽ അനേകം ജീവനുകൾ അന്തമില്ലാതെ അന്തർലീനമായിരിക്കുന്നു. എപ്പോഴാണോ ഒരുവൻ ഈ സത്യത്തെ സാക്ഷാത്കരിക്കുന്നത് അപ്പോഴാണ്‌ അവൻ മുക്തനാകുന്നത്. അപ്പോൾ അവനിലെ സുഖാസക്തിക്കും ശമനമുണ്ടാകുന്നു. അതുമാത്രമാണ്‌ ജ്ഞാനത്തിന്റെ ലക്ഷണം. "ഭംഗിയായി വരച്ചു ചായമടിച്ച അമൃതിന്റെ കുംഭം അമൃതല്ല; ചിത്രപടത്തിലെ ജ്വാല അഗ്നിയല്ല; ചിത്രത്തിലെ സ്ത്രീ സ്ത്രീയല്ല; ഉന്നതമൂല്യങ്ങളുള്ള, വിജ്ഞാനദായിയായ വാക്കുകൾ വെറും വാക്കുകൾ മാത്രം- അവ വിജ്ഞാനമല്ല. ഈ വാക്കുകൾ മൂല്യവത്താവുന്നത് ആശകളുടേയും ക്രോധത്തിന്റെയും അഭാവത്താലാണ്‌.." 

