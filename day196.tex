\section{ദിവസം 196}

\slokam{
കിം കരോമി ക്വ ഗച്ഛാമി കിം ഗൃഹ്ണാമി ത്യജാമി കിം\\
ആത്മനാ പൂരിതം വിശ്വം മഹാകൽപ്പാംബുനാ യഥാ (4/58/5)\\
ദു:ഖമാത്മാ സുഖം ചൈവ ഖമാശാസുമഹത്തയാ\\
സർവമാത്മമയം ജ്ഞാതം നഷ്ടകഷ്ടോഹമാത്മനാ (6)\\
സബാഹ്യാഭ്യന്തരേ ദേഹേ അധശ്ചോർദ്ധ്വം ച ദിക്ഷു ച\\
ഇത ആത്മാ തതശ്ചാത്മാ നാസ്ത്യനാത്മമയം ക്വചിത് (7)\\
സർവത്രൈവ സ്ഥിതോ ഹ്യാത്മാ സർവമാത്മമയം സ്ഥിതം\\
സർവമേവേദ മാത്മൈവ മാത്മന്യേവ ഭവാമ്യഹം (8)\\
യന്നാമ നമ തത്കിംചിത് സർവമേവാഹമാന്തര:\\
അപൂരീതാ പാരനഭാ: സർവത്ര സന്മയ: സ്ഥിത: (9)\\
പൂർണസ്തിഷ്ഠാമി മോദാത്മാ സുഖമേകാർണവോപമ:\\
ഇത്യേവം ഭാവയംസ്തത്ര കനകാചലകുഞ്ചകേ (10)\\
ഉച്ചാരയന്നോംകാരം ച ഘണ്ടാസ്വനമിവ ക്രമാത്\\
ഓംകാരസ്യ കലാമാത്രം പാശ്ചാത്യം ബാലകോമളം\\
നാന്തരസ്ഥോ ന ബഹ്യാസ്ഥോ ഭാവയൻപരമേ ഹൃദി (11)\\
}

വസിഷ്ഠൻ തുടർന്നു: ഇതോടനുബന്ധിച്ച് ദേവഗുരുവിന്റെ പുത്രൻ കചൻ ആലപിച്ച പ്രചോദനാത്മകമായ ഒരു ഗാനം ഞാനോർക്കുന്നു. കചൻ ആത്മവിദ്യയിൽ അടിയുറച്ച ജ്ഞാനിയായിരുന്നു. മേരു പർവ്വതത്തിലെ ഒരു ഗുഹയിലാണദ്ദേഹം വസിച്ചിരുന്നത്. മനസ്സ് പരമവിജ്ഞാനത്താൽ സാന്ദ്രമായതിനാൽ പഞ്ചഭൂതാത്മകമായ യാതൊരു ലൗകീകവസ്തുവിനും അദ്ദേഹത്തെ ബാധിക്കാൻ കഴിഞ്ഞില്ല.

വിഷാദഭാവം നടിച്ച്, അർത്ഥവത്തായ ആ ഗാനം,  കചൻ ഇങ്ങിനെ പാടി. കേട്ടാലും:

\slokam{
ഞാനെന്തുചെയ്യുവാൻ, എങ്ങു ഞാൻ പോകുവാൻ?\\
എന്തിനെ ഞാൻ വീണ്ടുമാശ്രയിക്കാൻ?\\
എന്താണെനിക്കു ത്യജിക്കുവാനായുള്ളു?\\
എന്തുമെല്ലാടവും ഞാനല്ലയോ?\\
വിശ്വം നിറഞ്ഞു വിളങ്ങുന്നതാത്മാവ്\\
സന്തോഷ സന്താപ ദ്വന്ദങ്ങളും\\
സങ്കൽപ്പമിഹലോക സുഖദുഖ മിഥ്യകൾ\\
ആശകളോ വെറും നിശ്ശൂന്യശൂന്യത\\
അകത്തും പുറത്തും താഴെയും മുകളിലും\\
അവിടേയുമിവിടേയുമെല്ലാടവും\\
ദേഹഗേഹത്തിലും സാർവ്വ ഭൗമത്തിലും\\
സർവ്വത്തിലും ഉണ്മയാത്മാവു താൻ\\
അനാത്മ വസ്തുക്കളായൊന്നുമില്ലെവിടെയും\\
ജഗജാലവൃന്ദമൊരേയാത്മാവു താന്‍\\
‘ആത്മാവിതെല്ലാം’ ഇതറിവായുണർന്നാൽ\\
സർവ്വസ്വതന്ത്രൻ സ്വയംപ്രഭൻ ഞാൻ\\
ആത്മാവിലെല്ലാമടങ്ങുന്നു നിൽക്കുന്നു\\
ആത്മാവിലാത്മസ്വരൂപനാം ഞാൻ\\
എല്ലാടവും ഘനസാന്ദ്രമായ് നിറതിങ്ങുമാത്മാവു\\
ഞാൻ പരിപൂർണ്ണനല്ലോ,\\
അനശ്വരൻ ഞാൻ പരമാനന്ദ കേവലൻ\\
അദ്വയൻ ആദിമദ്ധ്യാന്തരഹിതൻ\\
വിശ്വപ്രപഞ്ച സമുദ്രമായാമഗ്ന-\\
മദ്വൈതമുണ്മ ഞാൻ ആത്മാവുതാൻ\\
}


ഞാൻ എന്തു ചെയ്യട്ടെ? ഞാനെങ്ങോട്ടാണു പൊവേണ്ടത്? ഞാൻ എന്തിനെയാണ്‌ സമാശ്രയിക്കേണ്ടത്? എന്താണു ഞാൻ ന്യസിക്കേണ്ടത്? ഈ വിശ്വം മുഴുവൻ ഒരേയൊരാത്മാവിനാൽ വ്യാപൃതം. ദു:ഖവും അസന്തുഷ്ടിയും ആത്മാവുതന്നെ. സന്തോഷമെന്നതും ആത്മാവ്. എല്ലാ അശകളും പൊള്ളയായ നിശ്ശൂന്യത. ഇതെല്ലാം ആത്മാവെന്നറിഞ്ഞാൽ ഞാൻ സ്വതന്ത്രൻ. ഈ ശരീരത്തിൽ, അകത്തും പുറത്തും, താഴെയും മുകളിലും, അവിടെയുമിവിടെയും എല്ലാടത്തും ഒരേയൊരാത്മാവു മാത്രം. അനാത്മാവായി ഒന്നുമില്ല. ആത്മാവാണെല്ലാടവും. എല്ലാം ആത്മാവിൽ നിലകൊള്ളുന്നു. സത്യമായും ഇതെല്ലാം ആത്മാവു തന്നെ. ഞാൻ ആത്മാവിലാത്മാവായി നിലകൊള്ളുന്നു.
ഞാൻ അനശ്വരനായി, എന്നെന്നും എല്ലായിടവും ഉണ്മയായുണ്ട്. ഞാൻ പരിപൂർണ്ണൻ. ഞാൻ സ്വയം ആനന്ദമാണ് . വിശ്വം മുഴുവൻ ഞാൻ പ്രപഞ്ചസമുദ്രമെന്നപോലെ നിറഞ്ഞിരിക്കുന്നു.

ഇത്രയും പാടിയിട്ട് അദ്ദേഹം ഓം എന്ന മന്ത്രം ഒരു മണിനാദം പോലെ മന്ത്രിച്ചു. ജീവസത്തയെമുഴുവൻ ആ ഓംകാരത്തിലദ്ദേഹം വിലയിപ്പിച്ചു. അദ്ദേഹം ഒന്നിന്റേയുമുള്ളിലോ പുറത്തോ ആയിരുന്നില്ല. അദ്ദേഹം പരിപൂർണ്ണമായും ആത്മാഭിരാമനായി അവിടെ വിരാജിച്ചു. 
