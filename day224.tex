\section{ദിവസം 224}

\slokam{
സുബന്ധു: കസ്യചിത്ക: സ്യാദിഹ നോ കശ്ചിദപ്യരി:\\
സദാ സർവേ ച സർവസ്യ സർവം സർവേശ്വരേച്ഛയാ (5/18/49)\\
}

വസിഷ്ഠൻ തുടർന്നു: രാമ, നീ ജ്ഞാനിയാണ്‌.  അഹംകാരരഹിതനായി, ആകാശം പോലെ പരിശുദ്ധനായി നിലകൊണ്ടാലും. അഹമെന്നൊരു ധാരണതന്നെയില്ലെങ്കിൽ “ഇതെന്റെ ബന്ധുക്കളാണ്‌” എന്ന ചിന്ത എവിടെനിന്നുവരാനാണ്‌? ആത്മസ്വരൂപത്തിൽ അത്തരം ധാരണകൾ ഇല്ല. സുഖദു:ഖങ്ങളോ നന്മ-തിന്മകളോ അതിലില്ല. ഈ പ്രത്യക്ഷജഗത്തുണ്ടാക്കുന്ന ഭയവും വിഭ്രമവും നിന്നെ ബാധിക്കതിരിക്കട്ടെ. ഒരിക്കലും ‘ജനിച്ചിട്ടില്ലാത്തവന്‌’ (അജൻ) ബന്ധുക്കളെവിടെ? അവർ മൂലമുണ്ടാകുന്ന ദു:ഖങ്ങളെവിടെ?

നീ പണ്ട് ആരോ ഒരാളായിരുന്നു; ഇപ്പോഴും നീ ആരോ ആണ്‌.  നാളേയും അങ്ങിനെതന്നെയായിരിക്കും. ഇക്കാര്യങ്ങളെല്ലാം നിന്റെ ബന്ധുക്കളെസംബന്ധിച്ചും ശരിയാണെന്നു നീ തിരിച്ചറിഞ്ഞാൽപ്പിന്നെ ഭ്രമകൽപ്പനകളിൽ നിന്നും നിനക്കു മോചനമായി. പണ്ടു നീയുണ്ടായിരുന്നു; ഇപ്പോഴുമുണ്ട്, എന്നാൽ ഇനി മുതൽ നീയില്ല എന്നാണു നിനക്കു തോന്നുന്നതെങ്കിലും ദു:ഖിക്കാനൊന്നുമില്ല. കാരണം  ലോകമെന്ന ഈ പ്രകടനം അവസാനിച്ചു എന്നാണല്ലോ അതിനർത്ഥം. അതിനാൽ ഈ ലോകത്ത് എന്തിനെയെങ്കിലും പറ്റി വ്യകുലപ്പെടുന്നത് മൂഢത്വമാണ്‌..  സമുചിതമായ കർമ്മങ്ങളിലേർപ്പെട്ട് എപ്പോഴും സന്തോഷമായിരിക്കുകയാണ്‌ വിവേകം.

എങ്കിലും രാമ, അമിതാഹ്ലാദത്തിലും വിഷാദത്തിലും ആമഗ്നനാവരുത്. സമതാഭാവം കൈക്കൊണ്ടാലും. അതീവ സൂക്ഷ്മവും നിത്യശുദ്ധവുമായ, അനന്തശാശ്വതമായ പ്രകാശമാണു നീ. ഈ പ്രത്യക്ഷലോകം ഉണ്ടായി, നിലനിന്ന്, ഇല്ലാതാവുന്നത് അജ്ഞാനിയെ സംബന്ധിച്ചിടത്തോളം മാത്രമേ സത്യമായുള്ളു. ജ്ഞാനിയ്ക്ക് അതെല്ലാം മായയാണ്‌..  ജഗത്തിന്റെ സഹജഭാവമാണ്‌ ദു:ഖം. അജ്ഞാനം അതിനെ വികസിപ്പിച്ച് വലുതാക്കി വഷളാക്കുന്നു. പക്ഷേ നീ ബുദ്ധിമാനാണു രാമ. സന്തോഷമായിരിക്കൂ. മായക്കാഴ്ച്ചയെന്നാൽ മായ തന്നെ. സ്വപ്നം എന്നത് മറ്റൊരു സ്വപ്നം മാത്രം. ഇതെല്ലാം സർവ്വശക്തന്റെ പ്രാഭവം. പ്രകടിതലോകമെന്നത് വെറും ബാഹ്യപ്രകടനം മാത്രം.

“ഇവിടെ ആര്‌ ആർക്കാണൊരു ബന്ധുവായുള്ളത്? ആര്‌ ആർക്കാണൊരു ശത്രു? ജീവജാലങ്ങളുടെയെല്ലാം നാഥനായ ജഗദീശ്വരന്റെ ഇച്ഛയ്ക്കൊത്ത് എല്ലാവരും എല്ലാവർക്കും എല്ലാക്കാലവും എല്ലാമെല്ലാമായി വർത്തിക്കുന്നു.” ബന്ധുതയുടെ പുഴയൊഴുക്ക് അനവരതം തുടർന്നുകൊണ്ടിരിക്കുന്നു. ഒരുരഥചക്രത്തിലെന്നപോലെ താഴെയുള്ളവ മുകളിലേയ്ക്കും മുകളിലുള്ളവ താഴേയ്ക്കും പോയിക്കൊണ്ടേയിരിക്കുന്നു. സ്വർഗ്ഗത്തിലുള്ളവർ ആ വാസം മതിയാക്കി നരകത്തിൽപ്പോവുന്നു. നരകവാസികൾ സ്വർഗ്ഗത്തിലേയ്ക്കും പോവുന്നു.അവർ ഒരു ജീവിവർഗ്ഗത്തിൽ ജനിച്ചുമരിച്ച് പിന്നീട് മറ്റൊരു വർഗ്ഗത്തിൽ ജന്മമെടുക്കുന്നു. ലോകത്തിന്റെ ഒരു കോണിൽ നിന്നും മറ്റൊരു കോണിലേയ്ക്ക് വാസം മാറിപ്പോകുന്നു. ധീരൻ ഭീരുവും ഭീരു ധീരനുമാവുന്നു. കുറച്ചുകാലം ബന്ധുക്കളായിരുന്നവർ പിന്നീട് അകന്നുപോകുന്നു. മാറ്റമില്ലാത്തതായി ഈ വിശ്വത്തിൽ യാതൊന്നുമില്ല, രാമ.

സുഹൃത്ത്, ശത്രു, ബന്ധു, അപരിചിതൻ, ഞാൻ, നീ എന്നീ വാക്കുകൾക്കൊന്നും കാതലായ, ശാശ്വതമായ അർത്ഥങ്ങൾ ഒന്നുമില്ല. അവ വെറും വാക്കുകൾ മാത്രം. സങ്കുചിതമനസ്കന്റെ ഉള്ളിൽ ‘അയാളെന്റെ സുഹൃത്താണ്‌’, 'ഇയാളെന്റെ ബന്ധുവല്ല', തുടങ്ങിയ ചിന്തകൾ ഉണ്ടാവുമ്പോൾ വിശാലമനസ്കന്‌ ഇത്തരം ഭിന്നചിന്തകളില്ല. രാമ, എല്ലാ ജീവജാലങ്ങളും നിന്റെ ബന്ധുക്കളാണ്‌..  ഈ പ്രപഞ്ചത്തിൽ പരസ്പര ബന്ധമില്ലാത്ത ഒന്നുമില്ല. ആത്യന്തികമായി യാതൊന്നു തമ്മിലും 'അബന്ധുത്വം' എന്ന ഒന്ന് ഇല്ലേയില്ല. എല്ലാം പരസ്പര പൂരകങ്ങള്‍ . “ഞാനില്ലാത്ത ഒരിടവും ഇല്ല” എന്നും “എന്റേതല്ലാത്ത യാതൊന്നും ഇല്ല” എന്നുമുള്ള അറിവിൽ ജ്ഞാനികള്‍ അഭിരമിക്കുന്നു. അങ്ങിനെ അവർ പരിമിതികൾക്കും ഉപാധികൾക്കും അതീതരായി വർത്തിക്കുന്നു. 
