\section{ദിവസം 291}

\slokam{
ചിദാത്മോ നിര്‍മലോ നിത്യ: സ്വാവഭാസോ നിരാമയ:\\
ദേഹസ്ത്വനിത്യോ മലവാംസ്തേന സംബദ്ധ്യതേ കഥം  (5/71/24)\\
}

വസിഷ്ഠന്‍ തുടര്‍ന്നു: നാലാമത്തെ തലമായ തുരീയാവസ്ഥയില്‍ വിരാജിക്കുന്ന ജ്ഞാനിയെപ്പറ്റി, അയാളനുഭവിക്കുന്ന പരിപൂര്‍ണ്ണ സ്വാതന്ത്ര്യത്തെപ്പറ്റി ഏകദേശമായി നമുക്ക് അനുമാനിച്ചു പറയാന്‍ സാധിക്കും. എന്നാല്‍ തുരീയത്തിനുമതീതമായി ദേഹബോധംതന്നെ അതീന്ദ്രിയമായിരിക്കുന്ന പരമാവസ്ഥയെപ്പറ്റി ഒന്നും പറയുവാനാവില്ല. അതാണ്‌ തുരീയാതീത അവസ്ഥ. രാമാ ആ തലത്തിലെത്താന്‍ ശ്രമിക്കൂ. 

എന്നാല്‍ ആദ്യം ‘ജാഗ്രദില്‍ ദീര്‍ഘസുഷുപ്തി’ എന്ന അവസ്ഥയെ പ്രാപിക്കണം. ശരീരത്തെപ്പറ്റി യാതൊരാശങ്കകളും വേണ്ട. കാരണം അതൊരു ഭ്രമം മാത്രമാണല്ലോ. രാമാ നിനക്ക് ജ്ഞാനമുണ്ട്. നിന്റെയുള്ളില്‍ നിനക്ക് പ്രശാന്തി സഹജമായുണ്ടുതാനും. ആത്മജ്ഞാനിയുടെ പാത ഒരിക്കലും അപചയത്തിലേയ്ക്കാവുകയില്ല. ശുദ്ധാവബോധം മാത്രമാണുണ്മ. അതുകൊണ്ട് ‘ഞാന്‍ ഇന്നയാളാണ്’, ‘ഇതെന്റേതാണ്’ തുടങ്ങിയ ധാരണകളെ നിന്റെയുള്ളില്‍ ഉയരാന്‍ അനുവദിക്കരുത്. ആത്മാവ് എന്നതുപോലും പറയാന്‍ സൌകര്യപ്രദമായ വെറുമൊരു വാക്കെന്നു മാത്രം കരുതുക. സത്യം അതിനുമപ്പുറമാണ്.

സത്യത്തില്‍ ദ്വന്ദത എന്നത് ഇല്ലാത്ത ഒന്നാണ്. വാസ്തവത്തില്‍ വൈവിധ്യമാര്‍ന്ന ശരീരങ്ങള്‍ ‘ഇല്ലാ’ത്തതാണ്. സൂര്യനില്‍ നിഴലുണ്ടാവുന്നതെങ്ങിനെ? ഇല്ലാത്തവയ്ക്ക്‌ എങ്ങിനെ ബന്ധങ്ങള്‍ ഉണ്ടാവാനാണ്? ഇരുട്ടും വെളിച്ചവും തമ്മില്‍ എങ്ങിനെ ബന്ധമുണ്ടാവാനാണ്? ദേഹത്തിനും ദേഹരൂപമെടുത്ത ദേഹിക്കും തമ്മില്‍ യാതൊരു ബന്ധവുമില്ല. സത്യമറിയുമ്പോള്‍പ്പിന്നെ അസത്യത്തിനു നിലനില്‍പ്പുണ്ടോ?

ആത്മാവ് ബോധമാണ്. നിത്യശുദ്ധശാശ്വതമായ അതിന് മാറ്റങ്ങളില്ല. എന്നാല്‍ ശരീരമോ മാറിക്കൊണ്ടിരിക്കുന്നതും അശുദ്ധവും കേവലം താല്‍ക്കാലികവുമാണ്. അവ തമ്മില്‍ എങ്ങിനെയാണ് ബന്ധമുണ്ടാവുക? ശരീരത്തിന് ജീവന്‍ വയ്ക്കുന്നത് പ്രാണവായുവിനാലും മറ്റു ഘടകപദാര്‍ത്ഥങ്ങളാലുമാണ്. ഈ ദേഹത്തിനു ആത്മാവുമായി യാതൊരുവിധ ബന്ധവും സാദ്ധ്യമല്ല. ശരീരം, ആത്മാവ് എന്നീ രണ്ടുകാര്യങ്ങളെയും രണ്ടു സ്വതന്ത്ര സത്തകളായി കണക്കാക്കിയാല്‍ അവ തമ്മില്‍ യാതൊരു ബന്ധവും ഇല്ല എന്ന് കണ്ടു. അതേസമയം ദ്വന്ദത എന്നത് ഇല്ലാത്ത ഒരു കാര്യമാണെന്നുവരികില്‍ പിന്നെ ബന്ധം എന്ന ഈ ചിന്തയ്ക്കുപോലും സാധുതയില്ല. ഈ സത്യം നിന്നില്‍ രൂഢമൂലമാകട്ടെ. വാസ്തവത്തില്‍ എവിടെയും യാതൊരു തരത്തിലുമുള്ള ബന്ധനങ്ങളും ഇല്ല. അതുകൊണ്ട് തന്നെ ആര്‍ക്കും ഒരിടത്തും മുക്തിയുമില്ല. ഇതെല്ലാം ഏകമായ, അദ്വയമായ അനന്താവബോധം മാത്രമാണ്. ‘ഞാന്‍ സന്തുഷ്ടന്‍’, ‘ഞാന്‍ അജ്ഞാനി’, തുടങ്ങിയ ധാരണകള്‍ക്ക് നീ ചെവികൊടുക്കുകയാണെങ്കില്‍ അവ നിനക്ക് അന്തമില്ലാത്ത ദു:ഖമാണ് നല്‍കുക. 

ശരീരം ഉരുത്തിരിഞ്ഞത് പ്രാണവായു കാരണമാണ്. അതിലാണു ശരീരം നിലകൊള്ളുന്നത്. ശരീരത്തിന് സംസാരശേഷി നല്‍കുന്നതും ഈ വായുവാണ്. എന്നാല്‍ അതിലെ ചൈതന്യം, പ്രജ്ഞ, അവിച്ഛിന്നബോധമാണ്. ആകാശവും മറ്റുമായി വ്യാപരിച്ചിരിക്കുന്നത് അനന്താവബോധമാണ്. അകാശാദികള്‍ പ്രതിഫലിക്കുന്നതും ഈ ബോധത്തിലാണ്. ഈ പ്രതിഫലനം തന്നെയാണ് മനസ്സായത്. മനസ്സ് ദേഹമെന്ന ഈ കൂടുപേക്ഷിച്ചു പറന്നുപോകുമ്പോള്‍ ആത്മാവ് അനുഭവവേദ്യമാകുന്നു.  അത് ബോധമല്ലാതെ മറ്റൊന്നുമല്ല.

പൂമണം എവിടെയുണ്ടോ അവിടെ പൂവുണ്ട്. മനസ്സെവിടെയുണ്ടോ അവിടെ ബോധമുണ്ട്.

എന്നാല്‍ ഒന്നറിയുക: മനസ്സാണ് ലോകമെന്ന ഈ പ്രകടനത്തിന് ഹേതുവാകുന്നത്. ബോധം സര്‍വ്വവ്യാപിയും അനന്തവും അന്തിമവിശകലനത്തില്‍ എല്ലാറ്റിന്റെയും കാരണവുമാണെങ്കിലും പ്രകടിതപ്രപഞ്ചത്തിന്റെ കാരണം മനസ്സാണ്. ബോധമല്ല. അതുകൊണ്ട് പ്രത്യക്ഷലോകത്തിന്റെ കാരണം സത്യാന്വേഷണത്തിന്റെ അഭാവമത്രേ. അതായത് അജ്ഞാനമാണതിന്റെ ഹേതു. പക്ഷേ, വിളക്കിന്റെ വെളിച്ചം ഇരുട്ടിനെ ഇല്ലാതാക്കുന്നതുപോലെ ആത്മജ്ഞാനം അജ്ഞാനാന്ധകാരത്തെ ക്ഷണത്തില്‍ ഇല്ലാതാക്കുന്നു. അതിനാല്‍ എന്താണ് ജീവന്‍, എന്താണ് മനസ്സ്, തുടങ്ങിയ ചോദ്യങ്ങള്‍ക്കുള്ള ഉത്തരം തേടുകയാണ് സാധകന്‍ ചെയ്യേണ്ടത്.

