\section{ദിവസം 175}

\slokam{
ശ്രൂയതാം ജ്ഞാനസർവസ്വം ശ്രുത്വാ ചൈവാവധാര്യതാം\\
ഭോഗേഛാമാത്രകോ ബന്ധസ്ത ത്യാഗോ മോക്ഷ ഉച്യതേ (4/35/3)\\
}

വസിഷ്ഠൻ തുടർന്നു: രാമാ, അജ്ഞതയും മോഹവും നിറഞ്ഞ മനസ്സിനെ കീഴടക്കി നിയന്ത്രിച്ചവർ തന്നെയാണ്‌ യഥാർത്ഥത്തിൽ വീരനായകന്മാർ. ഈ പ്രത്യക്ഷലോകമെന്ന ജനന മരണചക്രങ്ങളോടുകൂടിയ പ്രകടനത്തിൽ അനുഭവവേദ്യമായ ദു:ഖദുരിതങ്ങൾക്കുള്ള ഏക ഔഷധം മന:സംയമനം മാത്രമാണ്‌. “ഞാൻ എല്ലാ വിജ്ഞാനങ്ങളുടെയും സാരസത്തയായ കാര്യം നിനക്കു പറഞ്ഞു തരാം. അതുകേട്ട് നിന്റെ ജീവിതത്തെ മുഴുവൻ സുഗന്ധപൂരിതമാക്കിയാലും. ബന്ധനം എന്നത് സുഖാസ്വാദനത്തിനുള്ള ആസക്തിയാണ്‌.. അതിന്റെ നിരാസമാണ്‌ മുക്തി.” അതുകൊണ്ട് ലോകത്തിലെ എല്ലാ സുഖഭോഗശാലകളേയും വിഷം വമിക്കുന്ന ഇടങ്ങളായി കരുതുക. എന്നാൽ അന്ധമായ സംത്യജിക്കൽ അഭികാമ്യമല്ല. തീവ്രമായി, ആഴത്തിൽ ഇന്ദ്രിയസുഖങ്ങളുടെ സ്വഭാവത്തെപ്പറ്റി അന്വേഷിക്കൂ, എന്നിട്ട് അവയോടുള്ള ആസക്തികളെല്ലാം ഉപേക്ഷിച്ചാലും. അപ്പോൾ നിനക്ക് ആനന്ദത്തോടെ ജീവിക്കാം.

പവിത്രഗുണങ്ങളെ വളർത്തിയെടുക്കുന്നതിലൂടെ എല്ലാ തെറ്റിദ്ധാരണകളും നീങ്ങുന്നു. മനസ്സ് ആഗ്രഹമുക്തമാവുന്നു. എല്ലാവിധ ദ്വന്ദങ്ങളുടേയും ശല്യമില്ലാതെയാവുന്നു. അശാന്തിയും ഭയവും മോഹവുമകന്ന് മനസ്സ് പ്രശാന്തമാവുന്നു. ആ മനസ്സ് നിർമ്മലമാണ്‌.. ദുഷ്ചിന്തകളോ വികാരങ്ങളോ ആശാപാശങ്ങളോ ദു:ഖമോ അവിടെയില്ല. ആ മനസ്സ്, തന്റെ സംശയം എന്നുപേരായ ദുര്‍പ്പുത്രനേയും അവന്റെ ഭാര്യയായ ആസക്തിയേയും പുറത്താക്കുന്നു. വിരോധാഭാസമെന്നു പറയട്ടേ, പ്രബുദ്ധമായ മനസ്സ് അതിന്റെ വളർച്ചയ്ക്കു കാരണമായിരുന്ന ആഗ്രഹങ്ങളേയും ചിന്തകളെയും (പ്രബുദ്ധതയ്ക്കായുള്ള അഭിവാഞ്ഛ പോലും) ഇല്ലാതാക്കുന്നു.

സ്വരൂപത്തെപ്പറ്റി അന്വേഷിക്കുന്നതിലൂടെ മനസ്സ് ദേഹാഭിമാനം ഉപേക്ഷിക്കുന്നു. അജ്ഞതനിറഞ്ഞ മനസ്സ് വളരുന്നു, വലുതാവുന്നു. എന്നാൽ വിവേകമുദിക്കുമ്പോൾ മനസ്സില്ലാതെയാകുന്നു. മനസ്സു മാത്രമാണീ വിശ്വം; പർവ്വതശിഖരങ്ങളും, ആകാശവും ഈശ്വരനും, സുഹൃത്തും ശത്രുവും എല്ലാം മനസ്സു തന്നെ.

അനന്താവബോധം സ്വയം മറന്ന് മാറ്റങ്ങൾക്കു വിധേയമായി ഉപാധികൾക്കു വശംവദമായി ‘മനസ്സ്’ ആവുന്നു. അതാണ്‌ ജനന-മരണങ്ങൾക്ക് ഹേതുവാകുന്നത്. ഇതാണ്‌ ജീവൻ എന്നറിയപ്പെടുന്നത്. അതായത് അനന്താവബോധത്തിന്റെ ഒരു ഭാഗം ബോധവസ്തുവിന്റെ സ്വഭാവം കൈക്കൊണ്ട്, മാനസീകോപാധികൾ കൊണ്ട് പൊതിഞ്ഞ് സംജാതമായി. ഈ ജീവനാണ്‌, അനന്താവബോധമെന്ന സത്യത്തിൽനിന്നു അകന്നകന്ന് ഉപാധികൾക്കു വിധേയമായി ലോകമെന്ന ഈ പ്രകടനത്തിൽ ആമഗ്നമാവുന്നത്. സത്യത്തിൽ ആത്മാവെന്നത് ഈ ജീവനോ, ദേഹമോ അതിന്റെ ഘടകങ്ങളോ ഒന്നുമല്ല. ആത്മാവ് ആകാശം  പോലെയാണ്‌... ഇവയിൽ നിന്നുമെല്ലാം സർവ്വസ്വതന്ത്രം. 

