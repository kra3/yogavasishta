\section{ദിവസം 272}

\slokam{
പാദാംഗുഷ്ടാച്ഛിരോ യാവത്കണശ: പ്രവിവിചാരിതം\\
ന ലബ്ധോഽസാവാഹം നാമ ക: സ്യദഹമിതി സ്ഥിത: (5/52/36)\\
}

ഉദ്ദാലകന്‍ തന്റെ മനനം തുടര്‍ന്നു. അനന്താവബോധത്തെ മനസ്സിനുള്ളില്‍ ഉള്‍ക്കൊള്ളുക എന്നത് ആനയെ കൂവളക്കായ്ക്കുള്ളില്‍ കുത്തിനിറയ്ക്കാമെന്നു വ്യാമോഹിക്കുന്നതുപോലെയാണ്. സ്വയം പരിമിതപ്പെടുത്തിയ ബോധമണ്ഡലം ധാരണാകല്‍പ്പനകളുടെ ചെറിയ പരിധിയ്ക്കുള്ളില്‍ നിലകൊള്ളുന്നതാണ്‌ മനസ്സ്. അജ്ഞാനജന്യമായതിനാല്‍ അതെനിയ്ക്ക് സ്വീകാര്യമല്ല. അഹംഭാവം എന്നത് വെറും ബാലിശമായ ധാരണയാണ്. സത്യമന്വേഷിക്കാന്‍ പരിശ്രമിക്കാത്തവരാണ് അതിനെ വിശ്വസിക്കുന്നത്.  

“എന്റെ കാല്‍വിരല്‍ത്തുമ്പ്‌ മുതല്‍ ഉച്ചിവരെ ഞാന്‍ വിശദമായി പരിശോധിച്ചു പഠിച്ചിരിക്കുന്നു. എന്നാല്‍ ‘ഇത് ഞാനാണ്’ എന്ന് നിശ്ചയിച്ചു പറയാവുന്ന ഒന്നുമെനിക്ക് കണ്ടെത്താനായിട്ടില്ല. അരാണീ ‘ഞാന്‍’?” സര്‍വ്വവ്യാപിയായ ഞാന്‍ അറിവിനുള്ള ഒരു വിഷയമാവുക അസാദ്ധ്യം. അറിയുക എന്ന പ്രക്രിയയാവുകയും സാദ്ധ്യമല്ല. അതില്‍ പരിമിതമായ വ്യക്തിത്വം ഇല്ല. ഞാന്‍ നാമരൂപങ്ങളില്ലാത്ത അവിച്ഛിന്നസ്വരൂപമാണ്. മാറ്റങ്ങള്‍ക്കും ഏകത്വം, നാനാത്വം തുടങ്ങിയ ധാരണകള്‍ക്കും, ചെറുതും വലുതുമായ എല്ലാ  അളവുകള്‍ക്കും അതീതമാണത്. അതല്ലാതെ വേറൊന്നുമില്ല.

ദു:ഖത്തിന്റെ അവസാനിക്കാത്ത സ്രോതസ്സായതിനാല്‍ മനസ്സേ നിന്നെ ഞാനിതാ ഉപേക്ഷിക്കുന്നു. രക്താസ്ഥിമാംസ നിബദ്ധമായ ഈ ശരീരത്തില്‍ ‘ഇതാണ് ഞാന്‍’ എന്നുപറയുന്നതാരാണ്? ചലനം ചൈതന്യത്തിന്റെ സഹജഭാവം; ചിന്തകള്‍ ബോധത്തിനും ജരാനരകള്‍ ദേഹത്തിനും സഹജം. ‘ഇതാണ് ഞാന്‍’ എന്നാരാണുല്‍ഘോഷിക്കുന്നത്? ഇതാണ് നാവ്, ഇതാണ് ചെവികള്‍ , ഇതാണ് മൂക്ക്, ഇതാണ് ചലനം, ഇതാണ് കണ്ണുകള്‍ , എന്നൊക്കെ നമുക്ക് ചൂണ്ടിക്കാണിക്കാം. എന്നാല്‍ ‘ഇതാണ് ഞാന്‍’ എന്ന് പറയുന്നതാര് ? ഞാനിതൊന്നുമല്ല. ഞാന്‍ നീയുമല്ല, മനസ്സേ. അത് ധാരണാ കല്‍പ്പനകളല്ല. ശുദ്ധസ്വതന്ത്രമായ അനന്താവബോധം മാത്രമാണ് ഞാന്‍... എന്നാല്‍ ‘ഞാനാണിതെല്ലാം’ എന്നുപറഞ്ഞാലും അത് ശരിതന്നെ. കാരണം ഒരേ സത്യത്തിന്റെ രണ്ടുതരം വ്യാഖ്യാനങ്ങളത്രേ ഇവ.

കഷ്ടം! ഇത്രകാലം ഞാന്‍ ഈ അജ്ഞാനത്തിന്റെ ഇരയായിരുന്നു. എന്നാല്‍ ഭാഗ്യവശാല്‍ എന്റെ ആത്മജ്ഞാനത്തെ മോഷ്ടിച്ച കള്ളനെ ഞാന്‍ പിടിച്ചു കഴിഞ്ഞിരിക്കുന്നു. ഇനിയൊരിക്കലും ഞാന്‍ അവന്  കീഴ്പ്പെടുകയില്ല. മലമുകളില്‍ കാണപ്പെടുന്ന കാര്‍മേഘം മലയുടേതല്ലാത്തതുപോലെ എന്നെ ചൂഴ്ന്നുനില്‍ക്കുന്ന ദു:ഖങ്ങളും ഞാനും തമ്മില്‍ യാതൊരു ബന്ധവുമില്ല. ആത്മജ്ഞാനമില്ലാത്തപ്പോള്‍ അഹംഭാവം ഉണ്ട്. ഇപ്പോള്‍ ഞാന്‍ അതില്‍ നിന്നും മുക്തനായിരിക്കുന്നു.

ഇന്ദ്രിയങ്ങളും ശരീരാദികളും മറ്റും നശിച്ചുകൊള്ളട്ടെ. എനിക്കവയുമായി ബന്ധമൊന്നുമില്ല. കണ്ണ് മുതലായ ഇന്ദ്രിയങ്ങള്‍ അവയുമായി ബന്ധമുള്ള വിഷയങ്ങളെ തേടി അലഞ്ഞുകൊള്ളട്ടെ. ‘ഞാനാണിത്’, ‘ഞാന്‍ കാണുന്നു’ എന്നെല്ലാം അവകാശപ്പെടുന്നതാരാണ്? ഇന്ദ്രിയങ്ങള്‍ സ്വയം പ്രവര്‍ത്തിക്കുന്നത്, അല്ലെങ്കില്‍ അനുഭവിക്കുന്നത് വാസനകളുടെ സമ്മര്‍ദ്ദത്താലല്ല. അതുകൊണ്ട് മനോവസനകളുടെ ഉപാധികളില്ലാതെ സഹജമായി ചെയ്യുന്ന കര്‍മ്മങ്ങള്‍ നല്‍കുന്ന അനുഭവങ്ങള്‍ ശുദ്ധമായിരിക്കും. അവ പൂര്‍വ്വാനുഭവസന്തുഷ്ടികളും അസുന്തുഷ്ടികളുമായി സങ്കലനം ചെയ്തു കൂടുതല്‍ വാസനാ മാലിന്യങ്ങളെ ഉണ്ടാക്കുന്നില്ല. അതുകൊണ്ട് ഇന്ദ്രിയങ്ങളേ, നിങ്ങള്‍ സ്വകര്‍മ്മങ്ങള്‍ മനോവാസനകളുടെ അകമ്പടികൂടാതെ നിര്‍വ്വഹിച്ചാലും.

വാസനകള്‍ എന്നത് വാസ്തവത്തില്‍ വസ്തുതാപരമായി നിലനില്‍ക്കുന്ന ഒന്നല്ല. അനന്താവബോധത്തില്‍ നിന്നും വിഭിന്നമായി, സ്വതന്ത്രമായി നിലകൊള്ളുന്ന ഒന്നല്ല അത്. ബോധതലത്തില്‍ വീണ്ടും ഉയര്‍ത്തിക്കൊണ്ടു വരാതിരുന്നാല്‍ അവയെ നിഷ്പ്രയാസം ഇല്ലായ്മചെയ്യാം. അതിനാല്‍ മനസ്സേ, വ്യത്യസ്ഥത, നാനാത്വം എന്നീ തെറ്റിദ്ധാരണകളെ അകറ്റിയാലും. അനന്താവബോധത്തില്‍ നിന്നും ഭിന്നമായി നിനക്ക് സ്വതന്ത്രമായ ഒരസ്ഥിത്വമില്ലെന്നു നീയറിയുക. അതാണ്‌ മുക്തി.

