\newpage
\section{ദിവസം 029}

\slokam{
പ്രസന്നേ ചിത്തത്വേ ഹൃദി ശമഭവേ  വൽഗതിപരേ\\
ശമാഭോഗി ഭൂതാസ്വഖില കലനാദൃഷ്ടിഷു നര:\\
സമം യാതി സ്വാന്ത:ക്കരണ ഘടനാസ്വാദിത രസം\\
ധിയാ ദൃഷ്ടേ തത്ത്വേ രമണമഥനാം ജാ ഗതമിദം (2/12/21)\\
}

വസിഷ്ഠന്‍ തുടര്‍ന്നു: രാമ: പ്രശാന്തമായ, സംശയലേശം പോലുമില്ലാത്ത, തുറന്ന മനസ്സോടെ നിര്‍മ്മലഹൃദയനായി ഈ മുക്തിസാധനവിദ്യയെക്കുറിച്ച്‌ കേട്ടാലും. കാരണം പരമസത്യം സാക്ഷാത്കരിക്കുംവരെ ജനനമരണ ദുരിതങ്ങള്‍ക്ക്‌ അറുതി വരികയില്ല. ഈ അവിദ്യയെന്ന ഭയങ്കരസര്‍പ്പത്തെ ഈ ജന്മത്തില്‍ത്തന്നെ വെന്നില്ലെങ്കില്‍ അത്‌ ഈ ജീവിതത്തില്‍ മാത്രമല്ല, വരും ജന്മങ്ങളിലും അനന്തമായ ദുരിതങ്ങള്‍ക്ക്‌ കാരണമാവും. ആര്‍ക്കും ഈ ദു:ഖാനുഭവങ്ങളെ അവഗണിക്കാന്‍ സാധിക്കില്ല. എന്നാല്‍ ഞാനുപദേശിക്കാന്‍ പോകുന്ന ജ്ഞാനത്തിന്റെ വെളിച്ചത്തില്‍ ഇതിനെ മറികടക്കാന്‍ കഴിയുന്നതാണ്‌. രാമ: അങ്ങിനെ നീ 'സംസാരം' എന്ന ആവര്‍ത്തന ചരിത്രത്തിന്റെ മറുകരകടന്നാല്‍ നിനക്കും ബ്രഹ്മാ-വിഷ്ണു-മഹേശ്വരന്മാരേപ്പോലെ, ഈ ലോകത്തില്‍ത്തന്നെ ഈശ്വരനായി ജീവിക്കാം. ആത്മാന്വേഷണത്തിലൂടെ മോഹമില്ലാതായാല്‍ സത്യം അകതാരില്‍ തെളിഞ്ഞുവിളങ്ങും. 

"മനസ്സ്‌ പ്രശാന്തവും ഹൃദയം പരമസത്യത്തിലേയ്ക്കുന്മുഖവും ചിന്താസരണികള്‍ അലകളടങ്ങിയ കടലുപോലെ ശാന്തവും, ഇടമുറിയാത്ത ശാന്തി അനുഭവവേദ്യവും ആവുമ്പോള്‍ ഹൃദയത്തില്‍ പരമാനന്ദം നിറയുന്നു. അങ്ങിനെ പരം പൊരുളറിഞ്ഞവന്‌ ഇഹലോകജീവിതം തന്നെ പരമാനന്ദപ്രദമാണ്‌." അങ്ങിനെയുള്ള ഒരുവന്‌ നേടാനും നഷ്ടപ്പെടാനും ഒന്നുമല്ല. അവനില്‍ ജീവിതത്തിന്റെ കുറ്റങ്ങളും കുറവുകളും ഒരു കളങ്കമായിത്തീരുന്നില്ല. ദു:ഖങ്ങള്‍ അവനെ ബാധിക്കുന്നില്ല. അവന്‍ ജന്മമെടുക്കുകയോ മരിക്കുകയോ ചെയ്യുന്നില്ല - മറ്റുള്ളവരുടെ ദൃഷ്ടിയില്‍ അതു പ്രത്യക്ഷമല്ലെങ്കിലും അതാണു സത്യം. ധാര്‍മ്മീകമോ മതപരമോ ആയ കാര്യങ്ങള്‍ പോലും അവന്‍ ചെയ്യേണ്ടതായില്ല.

അവനെ പൊയ്പ്പോയജന്മങ്ങളിലെ വാസനകള്‍ പ്രചോദിപ്പിക്കുന്നില്ല. കാരണം അവയുടെ ചാലകശക്തി നഷ്ടമായിരിക്കുന്നു. പ്രക്ഷുബ്ധമായ മനസ്സ്‌ അവിനിനിയില്ല. തന്റെ സ്വരൂപമായ പരമാനന്ദത്തിലാണ്‌ അവന്‍ അഭിരമിക്കുന്നത്‌. അത്തരം ആനന്ദം ആത്മജ്ഞാനം കൊണ്ടു മാത്രമേ ലഭിക്കൂ. അതുകൊണ്ട്‌ മനുഷ്യന്‍ ആത്മജ്ഞാനമാര്‍ഗ്ഗത്തിലൂടെ മാത്രമേ സഞ്ചരിക്കാവൂ. അത്മാന്വോഷണം മാത്രമാണ്‌ ഒരുവന്റെ ധര്‍മ്മം. പുണ്യഗ്രന്ഥങ്ങളേയും മഹാത്മാക്കളേയും നിന്ദിക്കുന്നവന്‌ ആത്മജ്ഞാനം ലഭ്യമല്ല. ഇഹലോകദുരിതങ്ങളേക്കാളെല്ലാം കൊടിയ ദുരന്തമാണ്‌ അത്തരം വിഡ്ഢിത്തങ്ങള്‍ ഒരുവനു സമ്മാനിക്കുന്നത്‌. അതുകൊണ്ട്‌ ശ്രദ്ധാഭക്തിബഹുമാനങ്ങളോടെ വേദശാസ്ത്രങ്ങളെ പഠിച്ച്‌ ആത്മാന്വേഷണത്തില്‍ മുഴുകുക. ഈ വേദഗ്രന്ഥം പഠിച്ചവന്‍ പിന്നെ അജ്ഞാനത്തിന്റെ അന്ധകൂപത്തില്‍ വീഴുകയില്ല. രാമ: നിനക്ക്‌ സംസാരമെന്ന ഈ തുടര്‍ക്കഥയില്‍ നിന്നുമുക്തിവേണമെങ്കില്‍ എന്നേപ്പോലുള്ള മഹര്‍ഷിമാരില്‍ നിന്നും ഈ വിജ്ഞാനം പഠിക്കാം. ഇതു തികച്ചും സൌജന്യമാണെന്നുമറിയുക!
