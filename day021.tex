 
\section{ദിവസം 021}

ഭാഗം 2. മുമുക്ഷു പ്രകരണം ആരംഭം 

\slokam{
യഥായം സ്വവികല്പോത്ഥ:  സ്വവികല്പാ പരിക്ഷയാത് \\
ക്ഷീയതേ ദഗ്ദ്ധ സംസാരോ നി:സ്സാര ഇതി നിശ്ചയ: (2/1/33)\\
}

വിശ്വാമിത്രന്‍ പറഞ്ഞു: അല്ലയോ രാമ! അങ്ങ്‌ ജ്ഞാനികളില്‍ അഗ്രഗണ്യനും അറിവിന്റെ നിറകുടവുമാണ്‌. അങ്ങേയ്ക്ക്‌ കൂടുതലായി ഒന്നും  ഇനി അറിയാനില്ല. എന്നാല്‍ അങ്ങയുടെ അറിവിനെ ദൃഢീകരിക്കേണ്ടതുണ്ട്‌. ശ്രീ ശുകമഹര്‍ഷി പോലും തന്റെ ആത്മജ്ഞാനം ജനകന്റെ അടുക്കല്‍പ്പോയി ഉറപ്പിച്ചതിനുശേഷമാണല്ലോ എല്ലാ അറിവുകള്‍ക്കുമതീതമായ പരമശാന്തിയെ പ്രാപിച്ചത്‌. 

രാമന്‍ ചോദിച്ചു: "മാഹത്മന്‍, എങ്ങിനെയാണ്‌ ശുകന്‌ ആത്മജ്ഞനായിരുന്നിട്ടുകൂടി അശാന്തിയേര്‍പ്പെട്ടത്‌? എങ്ങിനെയാണ്‌ അദ്ദേഹം പിന്നെ പ്രശാന്തനായത്‌?"

വിശ്വാമിത്രന്‍ പറഞ്ഞു: കേട്ടാലും രാമ: അങ്ങയുടെ പിതാവിനൊപ്പം ഇവിടെ ആസനസ്ഥനായിരിക്കുന്ന വേദവ്യാസന്റെ പുത്രന്‍ ശ്രീശുകന്റെ ആത്മോദ്ധാരണകാരിയായ ചരിതം ഞാന്‍ പറഞ്ഞു തരാം. അങ്ങയേപ്പോലെ ശുകനും തീവ്രധ്യാനത്താല്‍ ഈ ലോകത്തിന്റെ ക്ഷണഭംഗുരമായ അവസ്ഥയെപ്പറ്റിയുള്ള സത്യം  മനസ്സിലാക്കി. എന്നാല്‍ അത്‌ സ്വയം ആര്‍ജ്ജിച്ചതും അനുഭവത്താല്‍ ഉറപ്പിക്കാത്തതുമായ കേവല ജ്ഞാനമായിരുന്നു. അദ്ദേഹം അനാസക്തിനിരതനായിരുന്നു എന്നു നിശ്ചയ  മാണെങ്കിലും "ഇതാണുണ്മ" എന്ന് സ്വയം ഉറപ്പിക്കാന്‍ അദ്ദേഹത്തിനും സാധിച്ചില്ല. ഒരുദിവസം തന്റെ അച്ഛനായ വേദവ്യാസനോട്‌ അദ്ദേഹം ഇങ്ങിനെ ചോദിച്ചു: "ഈ ലോകസൃഷ്ടിയില്‍ എങ്ങിനെയാണ്‌ ഇത്രയേറെ വൈവിദ്ധ്യം വന്നത്‌? എന്നാണിനി ഇതെല്ലാം അവസാനിക്കുക?" വേദവ്യാസന്‍ ഇതിനുത്തരം വളരെ വിശദമായിത്തന്നെ നല്‍കി. പക്ഷേ ശുകന്‍ സംതൃപ്തനായില്ല. "ഇതെനിയ്ക്ക്‌ നേരത്തേ അറിയാവുന്ന കാര്യങ്ങളാണ്‌. പുതുതായി ഒന്നുമില്ലിതില്‍"

"ഇതിലധികമായി എനിയ്ക്കൊന്നുമറിയില്ല. അതിനാല്‍ നീ ഭൂമിയിലെ രാജര്‍ഷിയായ ജനകനെച്ചെന്നു കണ്ടാലും. അദ്ദേഹം  എന്നിലുപരി ജ്ഞാനമുള്ളയാളാണ്“.  ശുകന്‍ ജനകന്റെ കൊട്ടാരത്തില്‍ച്ചെന്നു. ശുകന്റെ ആഗമനം സേവകരില്‍ നിന്നറിഞ്ഞ ജനകന്‍ ഒരാഴ്ച്ചക്കാലം അദ്ദേഹത്തെ അവഗണിച്ചു. ശുകനാകട്ടെ ക്ഷമയോടെ കൊട്ടാരവാതില്‍ക്കല്‍ കാത്തുനില്‍ ക്കുകയും ചെയ്തു. അടുത്തയാഴ്ച്ച  ജനകന്‍ കൊട്ടാര നര്‍ത്തകരാലും ഗായകരാലും ശുകന്‌ സ്വീകരണം ഏര്‍പ്പെടുത്തി. ഇതിലൊന്നും ചഞ്ചലചിത്തനാവാതെയിരുന്ന ശുകനെ രാജാവ്‌ വിളിപ്പിച്ചു. "നീ സത്യമറിഞ്ഞവനാണ്‌. ഇനിയെന്താണ്‌ ഞാന്‍ നിനക്കു പറഞ്ഞു തരേണ്ടത്‌?" ശുകന്‍ പിതാവിനോട്‌ ചോദിച്ച ചോദ്യം ആവര്‍ത്തിച്ചു. ജനകനും വിശദമായി പറഞ്ഞ മറുപടി തന്റെ അച്ഛന്‍ പറഞ്ഞതു തന്നെയായിരുന്നു. 

ശുകന്‍ പറഞ്ഞു: ഇതെനിയ്ക്ക്‌ അച്ഛന്‍ പറഞ്ഞു തന്നിരുന്നു; വേദഗ്രന്ഥങ്ങളും ഇതാണു പ്രസ്താവിക്കുന്നത്‌. "ഈ നാനാത്വം ഉണ്ടാവുന്നത്‌ മനസ്സിന്റെ ചാഞ്ചല്യം കൊണ്ടാണെന്നും ഈ ചഞ്ചലത്വം ഇല്ലാതാവുന്നതോടെ ഈ കാണപ്പെടുന്ന നാനാത്വവും ഇല്ലാതാവുമെന്ന സത്യം എന്നിലുറപ്പിച്ചു തന്നാലും" അങ്ങിനെ ജനകസന്നിധിപൂകി ആത്മജ്ഞാനമുറച്ച ശുകന്‍ പരമശാന്തിയോടെ നിര്‍വ്വികല്‍പ്പസമാധിയില്‍ സ്ഥിരപ്രതിഷ്ഠിതനായി.

