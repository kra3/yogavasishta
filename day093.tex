 
\section{ദിവസം 093}

\slokam{
സ്വാര്‍ത്ഥക്രിയോഗ്രസാമര്‍ദ്ധ്യാതി ഭാവനയാന്യതാം\\
പദാര്‍ത്ഥോഽഭിമതാം ശാഠൃോ നി:ശ്വാസേനേവ ദര്‍പ്പണ: (3/70/19)\\
}

വസിഷ്ഠന്‍ തുടര്‍ന്നു: രാമ: ആ പര്‍വ്വതാകാരയായിരുന്ന രാക്ഷസി ചുരുങ്ങിച്ചുരുങ്ങി ഒരു സൂചിയുടെയത്ര (സൂചിക) ചെറുതായി. അവളുടെ രൂപം ഭാവനയില്‍ മാത്രം കാണാവുന്ന തരത്തില്‍ അതി സൂക്ഷ്മമായിരുന്നു. നട്ടെല്ലിന്റെ അടിമുതല്‍ ശിരസ്സിന്റെ മുകളറ്റംവരെ ബന്ധിപ്പിക്കുന്ന സുഷുമ്നാ നാഡിപോലെ അവള്‍ അതിസൂക്ഷ്മമായി നിലകൊണ്ടു. അവള്‍ ബുദ്ധമതക്കാര്‍ പറയുമ്പോലെ 'അലയ' ബോധത്തിലായിരുന്നു. അവളുടെ മറ്റേ രൂപമായ വിഷൂചിക (കോളറ) അനവരതം അവളെ പിന്തുടര്‍ന്നുകൊണ്ടിരുന്നു. അവള്‍ അതിസൂക്ഷ്മവും അദൃശ്യയും ആയിരുന്നുവെങ്കിലും അവളിലെ രാക്ഷസീയതയ്ക്ക്‌ മാറ്റമൊന്നും ഉണ്ടായിരുന്നില്ല. അവള്‍ ആവശ്യപ്പെട്ട വരം ലഭിച്ചുവെങ്കിലും എല്ലാ ജീവികളേയും ആഹരിക്കണമെന്ന അവളുടെ ആഗ്രഹം നടപ്പായില്ല. കാരണം അവള്‍ രൂപത്തില്‍ സൂചിയുടെയത്ര ചുരുങ്ങിപ്പോയിരുന്നല്ലോ. എത്ര വിചിത്രം! മോഹവിഭ്രമത്തില്‍പ്പെട്ടവര്‍ക്ക്‌ ദീര്‍ഘവീക്ഷണം എങ്ങിനെയുണ്ടാവാനാണ്‌?

"സ്വാര്‍ത്ഥന്റെ, തന്‍കാര്യം നേടാനുള്ള ഉഗ്രകര്‍മ്മങ്ങള്‍ അവനെ മറ്റുഫലങ്ങളിലേയ്ക്കു നയിക്കുന്നു. ഒരാള്‍ ഏറെദൂരം ഓടിക്കിതച്ചുവന്ന് മുഖക്കണ്ണാടി നോക്കിയാല്‍ തന്റെ മുഖം വ്യക്തമായി കാണാന്‍ കഴിയില്ല. കാരണം അവന്റെ ഉഛ്വാസനീരാവി കണ്ണാടിയില്‍ മങ്ങലുണ്ടാക്കുമല്ലോ." ഈ രാക്ഷസിക്ക്‌ വലിയൊരു രൂപമുണ്ടായിരുന്നു. അവളാ രൂപമുപേക്ഷിച്ചത്‌ സൂക്ഷ്മരൂപിയാകാനാണ്‌. അതിനായി അവള്‍ മരിച്ചു!. സ്വാര്‍ത്ഥലാഭത്തില്‍ ഇത്തരം അത്യാസക്തിയുള്ളപ്പോള്‍ മരണം പോലും അഭിലഷണീയമായിത്തോന്നും. വിഷൂചിക പ്രഭായോലുന്നതും പുഷ്പഗന്ധം പോലെ സൂക്ഷ്മവുമായിരുന്നു. മറ്റുള്ള ജീവനുകള്‍ കൊണ്ട്‌ അവള്‍ കര്‍മ്മനിഷ്ഠയായി വാണു. സൂചികയും വിഷൂചികയുമായി അവള്‍ എല്ലാവരേയും ബാധിച്ചുകൊണ്ട്‌ ലോകം മുഴുവന്‍ ചുറ്റിനടന്നു. അവളുടെ സ്വേഛപ്രകാരമാണ്‌ അവള്‍ സൂക്ഷ്മരൂപിയായത്‌. എന്താവണമെന്ന് തീവ്രമായി ആഗ്രഹിക്കുന്നുവോ, ഒരുവന്‍ അതായിത്തീരുന്നു. ഹീനചിന്തയുള്ളവര്‍ തുലോം തുച്ഛമായ കാര്യങ്ങള്‍ക്കായി പ്രാര്‍ത്ഥിക്കുന്നു. ഈ രാക്ഷസി ആഗ്രഹിച്ചത്‌ ക്രൂരയായ സൂചിയാകാനാണല്ലോ. ഒരാളുടെ ജന്മസ്വഭാവം തപസ്സുകൊണ്ടുപോലും മാറ്റുക ദുഷ്കരം!

നേരത്തേതന്നെ സ്ഥൂലശരീരമുള്ളവരുടേയും രോഗങ്ങള്‍ക്കടിപ്പെട്ടവരുടേയും ദേഹത്ത്‌ സൂചിക പ്രവേശിച്ച്‌ സ്വയം വിഷൂചികയായി പരിണമിക്കുന്നു. സൂചിക ഏത്‌ അരോഗദൃഢഗാത്രന്റേയും ഹൃദയത്തില്‍ പ്രവേശിച്ച്‌ അവന്റെ ബുദ്ധിയെ വഴിതെറ്റിക്കുന്നു. ചിലരുടെ കാര്യത്തില്‍ , ചികില്‍സകാരണംകൊണ്ടോ മന്ത്രബലംകൊണ്ടോ മരുന്നുകൊണ്ടോ അവള്‍ രോഗിയെ ഉപേക്ഷിക്കാന്‍ നിര്‍ബ്ബന്ധിതയാവുന്നു. അങ്ങിനെ അവള്‍ ഏറെക്കാലം ഭൂമിയില്‍ ചുറ്റി നടന്നു.

(കോളറാ വൈറസ്സിനെപറ്റിയാവണം ഈ വിവരണം. ഭക്ഷണം, ജീവിതരീതി എന്നിവയുമായി ഇതിനെ ബന്ധിപ്പിച്ചിരിക്കുന്നത്‌ രസകരമാണ്‌.)
