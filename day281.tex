\section{ദിവസം 281}

\slokam{
യാവത്സര്‍വ്വം ന സംത്യക്തം താവദാത്മാ ന ലഭ്യതേ\\
സര്‍വ്വാവസ്ഥാ പരിത്യാഗേ ശേഷ ആത്മോതി കഥ്യതേ (5/58/44)   \\
}

വസിഷ്ഠന്‍ തുടര്‍ന്നു: ഇതോടനുബന്ധിച്ച് രസകരമായ ഒരു കഥ പറയാം ശ്രദ്ധിച്ചു കേട്ടാലും. ഹിമാലയപര്‍വ്വതനിരകളില്‍ കൈലാസം എന്നൊരു ശിഖരമുണ്ട്. അതിന്‍റെ അടിവാരത്ത് മഞ്ഞനിറത്തില്‍ മുടികളോടുകൂടി ഹേമജാതര്‍ എന്നൊരു ഗോത്രവര്‍ഗ്ഗം താമസിക്കുന്നുണ്ടായിരുന്നു. സുരാഗു ആയിരുന്നു അവരുടെ രാജാവ്. ശക്തിമാനും ബുദ്ധിമതിയും ജ്ഞാനിയുമായിരുന്നു സുരാഗു. ആത്മജ്ഞാനം ആര്‍ജ്ജിച്ചിരുന്ന അദേഹം കവിയും സാഹിത്യ കുശലനുമായിരുന്നു.

ക്ഷീണമെന്തെന്ന് അദ്ദേഹത്തിനറിഞ്ഞുകൂടായിരുന്നു. ഭരണത്തില്‍ നിപുണനായിരുന്ന അദ്ദേഹം ശിക്ഷിക്കേണ്ടവര്‍ക്കും രക്ഷിക്കേണ്ടവര്‍ക്കും ഉചിതമായ നീതിന്യായം നടപ്പിലാക്കിയിരുന്നു. എന്നാല്‍ ഈ തിരക്ക് പിടിച്ച കര്‍മ്മപരിപാടിക്കിടയില്‍ അദ്ദേഹത്തിന്‍റെ ആത്മീയദര്‍ശനത്തില്‍ മങ്ങലുണ്ടായി. പക്ഷേ അദ്ദേഹം സ്വയം അക്കാര്യം തിരിച്ചറിഞ്ഞു, ആ ച്യുതിയെക്കുറിച്ചാലോചിച്ചു: ഞാന്‍ മൂലം ആളുകള്‍ പലവിധ ദു:ഖങ്ങളും അനുഭവിക്കേണ്ടിവരുന്നു. എന്നാല്‍പ്പിന്നെ അവരെയെല്ലാം സമ്പന്നരാക്കിയാലോ?. എനിക്കും ധനികനായാല്‍ സന്തോഷമുണ്ടാകുമല്ലോ, അതുപോലെ  അവരും ആഹ്ലാദിക്കട്ടെ. അവരുടെ സന്തോഷമാണെന്റെ സന്തോഷം. എന്നാലും പ്രജകളെ രക്ഷിച്ചും ശിക്ഷിച്ചും എനിക്ക് സന്തോഷസന്താപങ്ങള്‍ മാറിമാറി അനുഭവിക്കാനാണല്ലോ ഇതുവരെ കഴിഞ്ഞിരുന്നത്.  

ഇങ്ങിനെയെല്ലാം ആലോചിച്ച് രാജാവ് വിഷണ്ണനായിരിക്കെ ഒരുദിവസം മാണ്ഡവ്യമുനി രാജാവിനെ കാണാന്‍ വന്നു. രാജാവ് മഹര്‍ഷിയെ ഉപചാരപൂര്‍വ്വം ആദരിച്ചാനയിച്ചു. എന്നിട്ടു ചോദിച്ചു: ഞാന്‍ എന്റെ പ്രജകളില്‍ നടപ്പിലാക്കുന്ന അനുഗൃഹങ്ങളെപ്പറ്റിയും ശിക്ഷകളെപ്പറ്റിയും ഓര്‍ത്ത്‌ ആശങ്കാകുലനാണ്. അവ എന്തായാണ് എങ്ങിനെയാണ് എന്നിലേയ്ക്ക് കര്‍മ്മഫലങ്ങളായി തിരിച്ചു വരികയെന്നാണെന്റെ ആശങ്ക. ദയവായി എന്നെ മുന്‍വിധികളിലും പക്ഷപാതങ്ങളിലും വീഴ്ത്താന്‍ ഇടവരാതെ സമതാദര്‍ശനത്തിലേയ്ക്ക് നയിച്ചാലും.       

മാണ്ഡവ്യന്‍ പറഞ്ഞു: ആത്മഞാനത്തില്‍ അടിയുറച്ചവന്‍റെ വിവേകത്തെ മുന്‍നിര്‍ത്തി ഒരുവന്‍ പ്രയത്നിക്കുന്നതായാല്‍ അവന്‍റെ എല്ലാ മനോവൈകല്യങ്ങളും ഇല്ലാതെയാകും. ആത്മാവിന്റെ സഹജസ്വഭാവത്തെപ്പറ്റി അന്വേഷിക്കുന്നതിലൂടെ മാനോദു:ഖങ്ങള്‍ക്ക് അറുതി വരുത്താം. സ്വയം മനസ്സിനോട് ചോദിക്കൂ – എന്തൊക്കെ വ്യത്യസ്ഥമായ മനോഭാവങ്ങളും വ്യാപാരങ്ങളുമാണെന്നിലുണ്ടാവുന്നത്? ഇങ്ങിനെയുള്ള ചിന്തകള്‍ മനസ്സിനെ വലുതാക്കും. ക്രമേണ നിന്‍റെ സ്വരൂപത്തെപ്പറ്റി ശരിയായ അറിവുറയ്ക്കുമ്പോള്‍ ആഹ്ലാദദു:ഖങ്ങള്‍ക്ക് നിന്നെ അലട്ടുവാന്‍ കഴിയുകയില്ല. മനസ്സ് ഭൂത-ഭാവികളെയും തല്‍ഫലമായ വൈകല്യങ്ങളെയും ഉപേക്ഷിക്കുന്നതോടെ പരമശാന്തിയായി. ആ പ്രാശാന്തതയില്‍ സമ്പത്തിലും പ്രതാപത്തിലും അഭിമാനിക്കുന്നവരെപ്പറ്റി നിനക്ക് കഷ്ടം തോന്നും.   

നിനക്ക് ആത്മജ്ഞാനം സിദ്ധിക്കുന്നതോടെ, നിന്‍റെ ബോധം അനന്തമായി വികാസം പ്രാപിക്കും. പിന്നെ മനസ്സ്‌ ലോകമെന്ന ചെളിക്കുഴിയില്‍ വീഴുകയില്ല. ആന നടക്കുമ്പോള്‍ അത് ചെളിയില്‍ ചവിട്ടാറില്ലല്ലോ. ചെറിയ മനസ്സുകളാണ് ചെറിയ സുഖങ്ങളും പ്രതാപങ്ങളും തേടി നടക്കുന്നത്. ഉന്നതമായ ഒരു സമതാ ദര്‍ശനമുണ്ടാവുമ്പോള്‍ മനസ്സ്‌ എല്ലാം ഉപേക്ഷിക്കുന്നു. അതുകൊണ്ട് സ്വരൂപത്തെ സാക്ഷാത്ക്കരിക്കുംവരെ എല്ലാത്തിനെയും സംന്യസിക്കുക തന്നെ വേണം. “എല്ലാത്തിനെയും ഉപേക്ഷിച്ചാലല്ലാതെ ആത്മജ്ഞാനമുണ്ടാവുകയില്ല. എല്ലാ ധാരണാസങ്കല്‍പ്പങ്ങളും ഇല്ലാതാവുമ്പോള്‍ ശേഷമുള്ളതെന്തോ അതാണ്‌ ആത്മാവ്.” ഒരുവന്‍ ആഗ്രഹിക്കുന്നത് നേടണമെങ്കില്‍ അതിനു തടസ്സം നില്‍ക്കുന്ന പ്രശ്നങ്ങളെയെല്ലാം നീക്കം ചെയ്യണമല്ലോ. ലോകജീവിതത്തിലും ആത്മീയതയിലും ഇതുതന്നെ സത്യം. ആത്മജ്ഞാനത്തിന്റെ കാര്യത്തില്‍ ഇത് കൂടുതല്‍ പ്രസക്തമാണ് താനും.