\newpage
\section{ദിവസം 015}

\slokam{
യുഗവത്സര കല്പാഖൈ: കിഞ്ചിത്പ്രകടതാം ഗത:\\
രൂപൈരലക്ഷ്യ രൂപാത്മാ സർവ്വമാക്രമ്യ തിഷ്ടതി (1/23/7)\\
}

രാമന്‍ തുടര്‍ന്നു: എല്ലാ രസാനുഭവങ്ങളും വാസ്തവത്തില്‍ മിഥ്യയാണ്‌..  കണ്ണാടിയിലെ നിഴലിലൂടെ പഴങ്ങളുടെ സ്വാദനുഭവിക്കുന്നതുപോലെ ഭ്രാന്തമായ ഒരു രസമാണത്‌..  മനുഷ്യന്റെ പ്രത്യാശകളെല്ലാം കാലം സ്ഥിരമായി നശിപ്പിച്ചുകൊണ്ടിരിക്കുന്നു. കാലം എല്ലാത്തിനേയും ക്ഷയിപ്പിച്ചുകളയുന്നു. സൃഷ്ടികളില്‍ ഒന്നിനും കാലത്തിന്റെപിടിയില്‍ നിന്നും രക്ഷപ്പെടാന്‍ കഴിയില്ല. കാലമാണ്‌ എണ്ണമില്ലാത്ത ബ്രഹ്മാണ്ഡങ്ങളെ സൃഷ്ടിക്കുന്നതും ക്ഷണനേരം കൊണ്ട്‌ അവയെ ഇല്ലാതാക്കുന്നതും. 

"വര്‍ഷം, പ്രായം, യുഗം എന്നിവയിലൂടെ തന്റെ പ്രഭാവത്തിന്റെ ചെറിയൊരംശം കാലം നമുക്ക്‌ അനുഭവവേദ്യമാക്കുന്നു. എന്നാല്‍ അതിന്റെ ശരിയായ സ്വഭാവം നമുക്കറിയില്ല തന്നെ. കാലം എല്ലാത്തിനേയും കീഴടക്കുന്നു."

കാലം ദയവില്ലാത്തതും, വശപ്പെടുത്താനാവാത്തതും ക്രൂരവും, തൃപ്തിപ്പെടുത്താനാവാത്തതും അത്യാര്‍ത്തിപൂണ്ടതുമത്രേ. നമ്മെ മോഹവിഭ്രാന്തിയിലാക്കുന്ന അനേകം കൌശലങ്ങളുള്ള മഹാമാന്ത്രികനാണ്‌ കാലം. കാലത്തെ വിശകലനം ചെയ്യാന്‍ സാദ്ധ്യമല്ല. കാരണം എത്രചെറുതായി ഖണ്ഡിച്ചാലും അത്‌ അനശ്വരമായി ശേഷിക്കുന്നു. അതിന്‌ അടങ്ങാത്ത വിശപ്പാണ്‌. കൃമികീടങ്ങളും മാമലകളും സ്വര്‍ഗ്ഗത്തിന്റെ ചക്രവര്‍ത്തിയും എല്ലാം കാലത്തിന്റെ വരുതിയിലാണ്‌. നേരമ്പോക്കിന്‌ പന്തു തട്ടിക്കളിക്കുന്ന ബാലനേപ്പോലെ സൂര്യചന്ദ്രന്മാര്‍ എന്ന രണ്ടു പന്തുകളുമായി കാലം വിളയാടുന്നു. ബ്രഹ്മാണ്ഡങ്ങളെ നശിപ്പിക്കുന്ന രുദ്രനായതും, ദേവരാജാവായ ഇന്ദ്രനായതും, സമ്പത്തിന്റെ അധിദേവതയായ കുബേരനായതും വിശ്വം ലയിച്ചില്ലാതാവുന്ന ശൂന്യതയും എല്ലാം കാലം തന്നെയാണ്‌..  തീര്‍ച്ചയായും കാലമാണ്‌ തുടര്‍ച്ചയായി അണ്ഡകടാഹങ്ങളെ സൃഷ്ടിച്ച്‌ സംഹരിച്ചുകൊണ്ടേയിരിക്കുന്നത്‌..

മഹത്തും ബൃഹത്തുമായ പര്‍വ്വതങ്ങള്‍ ഭൂമിയില്‍ വേരുറപ്പിച്ചിരിക്കുന്നതുപോലെ പ്രതാപിയായ കാലം പരബ്രഹ്മത്തില്‍ സ്ഥാപിതമത്രേ. കാലം എണ്ണമില്ലാത്ത അണ്ഡകടാഹങ്ങളെ സൃഷ്ടിക്കുന്നുവെങ്കിലും അതിന്‌ അപചയമില്ല, അതു സന്തോഷിക്കുന്നില്ല, വരുന്നും പോകുന്നുമില്ല, ഉദിക്കുകയും അസ്തമിക്കുകയും ചെയ്യുന്നില്ല. കാലമെന്ന പാചകവിദഗ്ധന്‍ ലൌകീകവസ്തുക്കളെ സൂര്യന്റെ തീയില്‍ പാകപ്പെടുത്തി പാകമാവുന്നമുറയ്ക്ക്‌ ആഹരിക്കുന്നു. കാലം നിറപ്പകിട്ടുള്ള ജീവജാലങ്ങളാകുന്ന രത്നക്കല്ലുകള്‍ കൊണ്ട്‌ യുഗാന്തരങ്ങളായി അണിഞ്ഞൊരുങ്ങിയിട്ട്‌ കളിയായി അവയെല്ലാം നശിപ്പിച്ചുകളയുന്നു.

യൌവ്വനമെന്ന താമരയ്ക്ക്‌ കാലം രാത്രിയാണ്‌..  ആയുസ്സെന്ന ആനയ്ക്ക്‌ കാലം സിംഹമാണ്‌.. കാലം നശിപ്പിക്കാത്തതായി ഇഹലോകത്ത്‌ ഉയര്‍ന്നതോ താഴ്ന്നതോ ആയ ഒന്നുമില്ല. ഈ നശീകരണങ്ങള്‍ക്കെല്ലാമിടയിലും കാലം നാശരഹിതമായി അവശേഷിക്കുന്നു. ദിവസത്തെ മുഴുവന്‍ കഠിനാദ്ധ്വാനത്തിനുശേഷം ഒരുവന്‍ വിശ്രമിക്കുന്നതുപോലെ അജ്ഞാനത്തിലെന്നവണ്ണം വിശ്വപ്രളയം കഴിയവേ കാലം തന്റെ സര്‍ഗ്ഗശക്തി സംഭരിച്ചുറങ്ങുന്നു. കാലം എന്തെന്ന് ആര്‍ക്കും അറിയില്ല.
