 
\section{ദിവസം 133}

\slokam{
മാ വാകര്‍ത്താ ഭവ പ്രാജ്ഞ കിമകര്‍തൃതയേഹിതേ\\
സാദ്ധ്യം സാദ്ധ്യം ഉപാദേയം തസ്മാത്‌ സ്വസ്ഥോ ഭവാനഘാ (3/113/7)\\
}

വസിഷ്ഠന്‍ തുടര്‍ന്നു: മാനസീകമായ ഉപാധികള്‍ , അഥവാ വാസന, യാഥാര്‍ത്ഥ്യമല്ലെങ്കിലും അത്‌ മനസ്സില്‍ ഉയര്‍ന്നുവരുന്നുണ്ട്‌. ഒരേവസ്തുവിനെ (ചന്ദ്രനെ) രണ്ടായിക്കാണുന്ന ഒരുതരം മാനസീക രോഗത്തോട്‌ ഇതിനെ ഉപമിക്കാം. അതുകൊണ്ട്‌ ഈ വാസനകളെ വെറും മതിഭ്രമമെന്നു മനസ്സിലാക്കി തീര്‍ത്തും ത്യജിക്കണം. അവിദ്യയുടെ പരിണിതഫലം അജ്ഞാനിയെ മാത്രമേ ബാധിക്കൂ. ജ്ഞാനിക്കോ, അത്‌ വെറും വാക്കുകള്‍ മാത്രം. വന്ധ്യയുടെ പുത്രന്‍ എന്നു പറയുന്നതുപോലെ അസംബന്ധം.

രാമാ, അജ്ഞതയില്‍ തുടരാതിരിക്കൂ. ജ്ഞാനിയാവാന്‍ ശ്രമിക്കൂ. രണ്ടാമതൊരു ചന്ദ്രനുണ്ടെന്ന് ആരെങ്കിലും പറഞ്ഞാല്‍ അതു തള്ളിക്കളയുന്നതുപോലെ മനസ്സിന്റെ ഉപാധികളെ ഉപേക്ഷിച്ചാലും. ഇവിടെ യാതൊരു കര്‍മ്മത്തിന്റേയും കര്‍ത്താവ്‌ നീയല്ല. രാമ: പിന്നെ നീയെന്തിന്‌ കര്‍ത്തൃത്വ ഭാവം കൈക്കൊള്ളുന്നു? 'ഒന്നു' മാത്രം ഉള്ളപ്പോള്‍ ആര്‌ എന്തു കര്‍മ്മം എങ്ങിനെ ചെയ്യാനാണ്‌? "നിഷ്ക്രിയനാകരുത്‌; ഒന്നും ചെയ്യാതിരുന്നിട്ട്‌ എന്തു കിട്ടാനാണ്‌? ചെയ്യേണ്ട കാര്യങ്ങള്‍ ചെയ്യുകതന്നെ വേണം. അതുകൊണ്ട്‌ ആത്മാവില്‍ അഭിരമിക്കൂ." സ്വയം സഹജമായ കര്‍മ്മങ്ങള്‍ ചെയ്യുമ്പോഴും അവയോട്‌ നിനക്ക്‌ മമതയില്ലെങ്കില്‍ നീ കര്‍ത്താവല്ല. നീയൊന്നും ചെയ്യുന്നില്ലെങ്കിലും, ആ 'നിഷ്ക്രിയത്തോട്‌' മമതവെച്ചുകൊണ്ടിരിക്കുന്നുവെങ്കില്‍ നീ കര്‍ത്താവാണ്‌. .

പ്രത്യക്ഷ ലോകത്തിന്റെ ചാലകശക്തി വാസനകളാണ്‌.. അത്‌ മണ്‍കുടമുണ്ടാക്കുന്ന കുശവന്റെ ചക്രം പോലെ ചുറ്റിക്കൊണ്ടിരിക്കുന്നു. പാഴ്മുളത്തണ്ടുപോലെ അകം പൊള്ളയായ ഒന്നാണു വാസന. നദിയിലെ ഓളങ്ങളെപ്പോലെ, മുറിച്ചുമാറ്റിയാലും അതൊടുങ്ങുന്നില്ല. അതീവ സൂക്ഷ്മവും മൃദുവുമാണെങ്കിലും അതിന്‌ വാളിനേക്കാള്‍ മൂര്‍ച്ചയുണ്ട്‌.. അതിനെ ഗ്രഹിക്കുക വയ്യ. അതിനെ അറിയുന്നത്‌ അതിന്റെ പ്രതിഫലനങ്ങളില്‍നിന്നുമാണ്‌.. എന്നാല്‍ ഒരുവന്റെ സത്യാന്വേഷണത്തില്‍ അതുകൊണ്ട്‌ പ്രയോജനങ്ങളൊന്നുമില്ല. സൃഷ്ടിജാലങ്ങളില്‍ നാനാത്വം കാണപ്പെടുന്നത്‌ ഈ ഉപാധികളാലാണ്‌..  അതിന്‌ പ്രത്യേകിച്ച്‌ വാസസ്ഥലങ്ങളൊന്നുമില്ല. എല്ലാടവും അതുണ്ട്‌..  മനോപാധികള്‍ മേധാശക്തിയുടെ പ്രകടനമല്ലെങ്കിലും അത്‌ ബുദ്ധിയില്‍ അധിഷ്ഠിതമായതിനാല്‍ അങ്ങിനെ തോന്നുന്നു. എപ്പോഴും മാറ്റത്തിനു വിധേയമായികൊണ്ടിരിക്കുമ്പോഴും അത്‌ ശാശ്വതമാണെന്ന തോന്നലുളവാക്കുന്നു. അനന്താവബോധവുമായുള്ള ഇടപഴകല്‍ നിമിത്തം അത്‌ കര്‍മ്മോന്മുഖമായും അനുഭവപ്പെടുന്നു. അനന്താവബോധം സാക്ഷാത്കരിക്കുമ്പോള്‍ ഈ മനോപാധികള്‍ക്ക്‌ അവസാനമായി. വിഷയങ്ങളോടുള്ള ആസക്തിയാണ്‌ വാസനകളെ ഊട്ടി വളര്‍ത്തുന്നത്‌.. ഈ ആസക്തിയുടെ അഭാവത്തില്‍പ്പോലും വാസന ഒരു സാദ്ധ്യതയായി അവശേഷിക്കുന്നു. 

