\newpage
\section{ദിവസം 046}

\slokam{
വിവർത്തമേവ ധാവന്തി നിർവിവർത്താനി സന്തി ച\\
ചിദ്വേധിതാനി സർവാണി ക്ഷണാത് പിംഡീഭവന്തി ച (3/12/30)\\
}

വസിഷ്ഠന്‍ പറഞ്ഞു: ഉറങ്ങിക്കിടക്കുന്നവന്റെയുള്ളില്‍ സ്വപ്നങ്ങള്‍ പ്രത്യക്ഷമാവുന്നതുപോലെ അവിച്ഛിന്നമായ വിശ്വസത്ത്വത്തില്‍ സൃഷ്ടി പ്രകടിതമായതെങ്ങിനെയെന്ന് ഞാന്‍ വിശദമാക്കാം. ശാശ്വതഭാസുരമായ അനന്താവബോധത്തില്‍ നിന്നു സ്വയം ഉദ്ഭൂതമാണീ പ്രപഞ്ചം. ആ സത്തയില്‍ സ്വയം 'അറിയുന്ന വസ്തു' (ആത്മാവ്‌) അതിന്റെ 'സ്വരൂപത്തെ'ക്കുറിച്ച്‌ ആരാഞ്ഞതിന്റെ ഫലമായി ആകാശം എന്ന 'ഇടം' സംജാതമായി. കാലമേറെക്കഴിഞ്ഞ്‌ സൃഷ്ടിബോധം അനന്ത സത്തയില്‍ ശക്തിപ്രാപിക്കേ ജീവനായി വിശ്വാത്മാവ്‌ - ഹിരണ്യഗര്‍ഭം- അതില്‍ നിന്നുദ്ഭൂതമായി. അനന്തത അതിന്റെ പൂര്‍ണ്ണതയെ വെടിഞ്ഞ്‌ സ്വയം ജീവാത്മാവായി പരിമിതപ്പെട്ടതുപോലെ ആയിത്തീര്‍ന്നു.

അപ്പോഴും ബ്രഹ്മം മാറ്റമേതുമില്ലാതെ, അനന്തമായിത്തന്നെ നിലകൊണ്ടു. അകാശത്തില്‍ ശബ്ദം സ്വയമേവ മാറ്റൊലിക്കൊണ്ടു. പിന്നീട്‌ ജീവനില്‍ തുടര്‍സൃഷ്ടിക്കനിവാര്യമായ 'അഹങ്കാരം' ഉണര്‍ന്നു. അതേസമയം, 'കാല'വും ഉണ്ടായി. ഇതെല്ലാം ഉണ്ടാവുന്നത്‌ അനന്തതയില്‍ യാഥാർഥ്യത്തിലുണ്ടായ വ്യതിയാനങ്ങളായല്ല. ഹിരണ്യഗര്‍ഭത്തിലെ സൃഷ്ട്യുന്മുഖചിന്തയില്‍ മാത്രമാണീ സൃഷ്ടി. അതേപോലെ സൃഷ്ടിവിചാരത്താല്‍ വായുവുണ്ടായി; വേദങ്ങളും പിന്നീടുണ്ടായി. ഇതിനെയെല്ലാം വലയംചെയ്യുന്ന ബോധം ജീവാത്മാവായി മറ്റെല്ലാവിധ ഭൂതങ്ങള്‍ക്കും കാരണമായി. നിവാസികളോടുകൂടിയ പതിന്നാലുതരം ജീവിതാസ്തിത്വങ്ങളുണ്ട്‌. അവയും ബോധത്തിന്റെ സര്‍ഗ്ഗസൃഷ്ടികളാണ്‌. ഈ ബോധം 'ഞാന്‍ വെളിച്ചം' എന്നാലോചിക്കേ പ്രകാശസ്രോതസ്സുകളായ സൂര്യനും മറ്റും ഞൊടിയിടയില്‍ സൃഷ്ടമായി. അതുപോലെ ജലവും ഭൂമിയും സൃഷ്ടിക്കപ്പെട്ടു. 

ഈ അടിസ്ഥാന ഘടകങ്ങള്‍ (പഞ്ചഭൂതങ്ങള്‍ ) ഒന്നൊന്നിന്മേലും തിരിച്ചും പ്രവര്‍ത്തനം തുടങ്ങി അവ ക്രമത്തിൽ അനുഭവവങ്ങളും അനുഭവിക്കുന്നവരുമായി  മാറി. സമുദ്രോപരി കാണുന്ന വര്‍ത്തുളമായ അലകള്‍ പോലെ സകലമാന സൃഷ്ടികളും  അങ്ങിനെ  ഉണ്ടായി. വിശ്വപ്രളയത്തിനുമുന്‍പ്‌ , വേര്‍തിരിച്ചു വിടര്‍ത്തിയെടുക്കാനാവാത്തവണ്ണം ആ ഘടകങ്ങള്‍ പരസ്പരം നൂലിഴപാകിയിരുന്നു. "ഈ പ്രകടിത വസ്തുക്കള്‍ (പഞ്ചഭൂതങ്ങള്‍ ) എപ്പോഴും മാറ്റങ്ങള്‍ക്കു വിധേയമാവുമ്പോഴും ആ പരമസത്തയ്ക്കു മാറ്റമൊന്നുമില്ല. ബോധവുമായി നിരന്തരബന്ധം കൊണ്ട്‌ ഘടകവസ്തുക്കള്‍ ക്ഷണത്തില്‍ കൂടുതല്‍ പ്രത്യക്ഷസ്വഭാവമുള്ള ഭൌതീകവസ്തുക്കളായിത്തീരുന്നു." എങ്കിലും ഇതെല്ലാം ആ അനന്തബോധം തന്നെയാണ്‌. അത്‌ മാറ്റങ്ങള്‍ക്ക്‌ വിധേയമല്ല. 

