\newpage
\section{ദിവസം 059}

\slokam{
തച്ചിന്തനം തത്കഥനമന്യോന്യം തത്പ്രബോധനം\\
ഏതദേകപരത്വം ച തദഭ്യാസം വിദുർബുധാ: (3/22/24)\\
}

സരസ്വതി പറഞ്ഞു: സ്വപ്നത്തില്‍ സ്വപ്നശരീരം തികച്ചും ഉള്ളതായി കാണപ്പെടുന്നു. എന്നാല്‍ അത്‌ സ്വപ്നമായിരുന്നു എന്ന അറിവുണര്‍ന്നുകഴിഞ്ഞാല്‍പ്പിന്നെ ആശരീരം ഉണ്മയായിരുന്നുവെന്ന തോന്നലേ ഇല്ലാതാവുന്നു. അതുപോലെ ഓര്‍മ്മകളിലും വാസനകളിലും നിലനില്‍ ക്കുന്നതായ ഈ ശരീരവും അയാഥാര്‍ത്ഥ്യമാണെന്നു 'കണ്ടാല്‍ ' അവ അങ്ങിനെതന്നെയാണ്‌. സ്വപ്നാന്ത്യത്തില്‍ നാം നമ്മുടെ ഭൌതീകശരീരത്തെപറ്റി ബോധവാനാവുന്നു. അതുപോലെ വാസനാന്ത്യത്തില്‍ നാം നമ്മുടെ സൂക്ഷ്മശരീരത്തെപ്പറ്റി ബോധവാനാവുന്നു. സ്വപ്നം ഇല്ലാത്തപ്പോള്‍ ദീര്‍ഘനിദ്രയായി. ചിന്തകളുടെ വിത്തു നശിക്കുമ്പോള്‍ മുക്തിയായി. അവിടെ ചിന്തകളില്ല. മുക്തനായ ഒരു മഹാത്മാവ്‌ ജീവിക്കുന്നതായും ചിന്തിക്കുന്നതായും എല്ലാം നാം കണ്ടുവെന്നിരിക്കട്ടെ; അതെല്ലാം തറയില്‍ കിടക്കുന്ന ഒരു കരിഞ്ഞതുണിപോലെ ഒരു പുറംകാഴ്ച്ച മാത്രം. എന്നാല്‍ ഇത്‌ ദീര്‍ഘനിദ്രപോലെയോ അബോധാവസ്ഥയോ അല്ല. കാരണം അവ രണ്ടിലും ചിന്തകള്‍ മറഞ്ഞിരിക്കുന്നുണ്ടല്ലോ. (ആ അവസ്ഥകളുടെ അന്ത്യത്തില്‍ ചിന്തകള്‍ തിരികേ വരുന്നുണ്ട്) 

നിസ്തന്ദ്രമായ പരിശ്രമംകൊണ്ട്‌ (അഭ്യാസം) അഹംകാരത്തെ നിശ്ശബ്ദമാക്കാം. അപ്പോൾ നീ സ്വാഭാവികമായും അവബോധത്തിലാണ്‌ നിലകൊള്ളുക. കാണപ്പെടുന്ന പ്രപഞ്ചം അദൃശ്യമായി മറയുന്നതിന്റെ ഉത്തുംഗതയിലെത്തും (പരിപൂര്‍ണ്ണമായി മറയുന്നില്ല എന്നര്‍ത്ഥം.) എന്താണീ അഭ്യാസം? "സദാ 'അതി'നെപ്പറ്റി ചിന്തിക്കുക, പറയുക, മറ്റുള്ളവരുമായി സംവദിക്കുക, 'അതില്‍ ' മാത്രം പരിപൂര്‍ണ്ണ സമര്‍പ്പണം, എന്നിവയെയാണ്‌ അഭ്യാസം എന്നതുകൊണ്ട്‌ ജ്ഞാനികള്‍ ഉദ്ദേശിക്കുന്നത്‌". എപ്പോള്‍ ഒരുവന്റെ ബുദ്ധിയില്‍ സൌന്ദര്യവും ആനന്ദവും നിറഞ്ഞുനില്‍ ക്കുന്നുവോ അവന്റെ വീക്ഷണം വിശാലമാവുന്നുവോ, ഇന്ദ്രിയ സുഖാനുഭവങ്ങള്‍ക്കായി ആസക്തിയില്ലാതിരിക്കുന്നുവോ അപ്പോളവന്‍ അഭ്യാസിയാണ്‌. ഈ വിശ്വം ഒരിക്കലും സൃഷ്ടിക്കപ്പെട്ടിട്ടില്ല എന്ന ഉറച്ചബോധമുള്ളതിനാല്‍ അതിനു നിലനില്‍ പ്പില്ലെന്നറിയുന്നവനില്‍ 'ഇതു ലോകമാണ്‌, ഇതു ഞാനാണ്‌' തുടങ്ങിയ ചിന്തകളുണ്ടാവുകയില്ല. അവന്‍ അഭ്യാസിയാണ്‌. അവനില്‍ ഒന്നിനോടും ആസക്തിയോ വിരക്തിയോ ഉണ്ടാവുകയില്ല. ഇഷ്ടാനിഷ്ടങ്ങളെ മറികടക്കാന്‍ ഇടയാക്കുന്ന ആത്മബലമാണ്‌ തപസ്സ്‌. വിജ്ഞാനമല്ല.

ഈ സമയം സന്ധ്യയായതിനാല്‍ സഭപിരിഞ്ഞു. പിറ്റേന്ന് അതിരാവിലെ വസിഷ്ഠമഹര്‍ഷി സഭയില്‍ തന്റെ പ്രഭാഷണം തുടര്‍ന്നു: രാമ:, സരസ്വതീ ദേവിയും ലീല രാജ്ഞിയും ഉടനേതന്നെ തീവ്രമായ ധ്യാനത്തിലാണ്ടു. നിര്‍വ്വികല്‍പ്പ സമാധി ! അവര്‍ ശരീരബോധത്തിന്റെ തലങ്ങളില്‍ നിന്നുയര്‍ന്നു പോയിരുന്നു. പ്രപഞ്ചത്തെപറ്റിയുള്ള എല്ലാ ധാരണകളും ഉപേക്ഷിച്ചിരുന്നതിനാല്‍ അവരുടെ ബോധമണ്ഡലത്തില്‍ ലോകമാകെ മറഞ്ഞുപോയിരുന്നു. അവര്‍ തങ്ങളുടെ ജ്ഞാനശരീരവുമായി സര്‍വ്വതന്ത്രസ്വതന്ത്രരായി പാറി നടന്നു. അനേകലക്ഷം യോജന സഞ്ചരിച്ചതായി തോന്നിയെങ്കിലും അവര്‍ അതേ 'മുറിയില്‍ ' ത്തന്നെയായിരുന്നു. പക്ഷേ അവര്‍ മറ്റൊരു ബോധമണ്ഡലത്തിലാണു വിഹരിച്ചിരുന്നത്‌.
