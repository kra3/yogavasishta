\newpage
\section{ദിവസം 072}

\slokam{
തപോ വാ ദേവതാ വാപി ഭൂത്വാ സ്വൈവ ചിദന്യഥാ\\
ഫലം ദദാത്യഥ സ്വൈരം നാഭ: ഫലനിപാതവത് (3/45/19)\\
}

രണ്ടാമത്തെ ലീല സരസ്വതീ ദേവിയോടു പറഞ്ഞു: ദേവീ, ഞാന്‍ സരസ്വതീദേവിയെ പൂജിക്കാറുണ്ട്‌. ദേവി എന്റെ സ്വപ്നങ്ങളില്‍ പ്രത്യക്ഷപ്പെടാറുമുണ്ട്‌. അവിടുന്ന് ആ ദേവിയുടെ തല്‍സ്വരൂപം പോലെയുണ്ട്‌. അവിടുന്ന് സരസ്വതീദേവി തന്നെയാണെന്നു ഞാന്‍ കരുതുന്നു. എനിക്കൊരു വരം തരണമെന്ന് ഞാന്‍ അഭ്യര്‍ത്ഥിക്കുന്നു. എന്റെ പ്രിയതമന്‍ യുദ്ധക്കളത്തില്‍ വീരചരമം പ്രാപിക്കുമ്പോള്‍ എനിക്കും അദ്ദേഹത്തോടൊപ്പം - ഏതൊരിടത്തേക്കാണദ്ദേഹം പോവുന്നതെങ്കിലും അങ്ങോട്ട്‌- എന്റെ ഈ ശരീരത്തോടെ തന്നെ പോവാന്‍ സാധിക്കുമാറാകണം.

സരസ്വതി പറഞ്ഞു: പ്രിയപ്പെട്ടവളേ, നീയെന്നെ ഏറെക്കാലമായി തീവ്രഭക്തിയോടെ പൂജിക്കുന്നുവല്ലോ. ആയതിനാല്‍ നീയാവശ്യപ്പെട്ടവരം ഇതാ ഞാന്‍ നല്‍കുന്നു. അപ്പോള്‍ ആദ്യത്തെ ലീല സരസ്വതിയോടു പറഞ്ഞു: ശരിയാണ്‌. അവിടുത്തെ വാക്കുകള്‍ പാഴാവുകയില്ല. അതെല്ലാം സത്യമായി ഭവിക്കുന്നു. എനിക്ക്‌ ഒരു ബോധതലത്തില്‍നിന്നും മറ്റൊന്നിലേയ്ക്ക്‌ ഒരേ ശരീരവുമായി സഞ്ചരിക്കാന്‍ അവിടുന്ന് എന്തുകൊണ്ട്‌ അനുവദിച്ചില്ല എന്നു ദയവായി പറഞ്ഞു തരൂ.

സരസ്വതി പറഞ്ഞു: ഞാന്‍ ആര്‍ക്കുംവേണ്ടി ഒന്നും ചെയ്യുന്നില്ല. എല്ലാ ജീവനും അവരവരുടെ പ്രയത്നത്തിന്റെ ഫലമായി അവസ്ഥകളെ പ്രാപിക്കുകയാണ്‌. ഞാന്‍ എല്ലാവരുടെയും ബുദ്ധിയെ പ്രദ്യോതിപ്പിക്കുന്ന ദേവതമാത്രമാണ്‌. അവരുടെ ജീവശക്തിയും ബോധബലവുമാണ്‌ ഞാന്‍. ഓരോ ജീവനും ഏതൊരുതരം ചൈതന്യമാണോ (ഊര്‍ജ്ജം) ഉള്ളില്‍ ആവഹിക്കുന്നത്‌ അതാണ്‌ ഫലപ്രാപ്തിയിലെത്തുന്നത്‌. നീ മുക്തിയാണാഗ്രഹിച്ചത്‌. നിനക്കതു സ്വായത്തമാവുകയും ചെയ്തു. "നിനക്കത്‌ നിന്റെ തപസ്സിന്റെ ഫലമെന്നോ, ഇഷ്ടദേവതയുടെ അനുഗ്രഹമെന്നോ പറയാം. എന്നാല്‍ ബോധം മാത്രമാണത്‌. അതാണ്‌ നിന്റെ ഫലദാതാവ്‌. അകാശത്തുനിന്നും ഒരു ഫലം വീഴുന്നുവെന്നു തോന്നിയെന്നാലും ഫലം വീഴുന്നത്‌ വൃക്ഷത്തില്‍ നിന്നും തന്നെയാണ്‌."

വസിഷ്ഠന്‍ തുടര്‍ന്നു: അവരിങ്ങിനെ സംസാരിച്ചുനില്‍ക്കേ വിദുരഥ രാജാവ്‌ തന്റെ തിളക്കമേറിയ രഥത്തില്‍ പടക്കളത്തിലേയ്ക്കു പുറപ്പെട്ടു. നിര്‍ഭാഗ്യമെന്നുപറയട്ടേ, ശത്രുക്കളുടെയും തന്റെയും ബലാബലങ്ങള്‍ നോക്കുന്നതില്‍ ഉണ്ടായ പിഴവുമൂലം ശത്രുപാളയത്തില്‍ എത്തുംവരെ രാജാവിന്‌ അവരുടെ യഥാര്‍ഥബലം അറിയാന്‍ കഴിഞ്ഞില്ല. രണ്ടു ലീലമാരും സരസ്വതിദേവിയും, ദേവിയുടെ അനുഗൃഹത്തിനു പാത്രമായ രാജകുമാരിയും കൊട്ടാരമുകളില്‍ നിന്നും ഭയാനകമായ ഈ യുദ്ധം വീക്ഷിച്ചുകൊണ്ടിരുന്നു. ഇരുസൈന്യങ്ങളില്‍നിന്നും വര്‍ഷിച്ച ആയുധങ്ങള്‍ ആകാശത്തെ മേഘാവൃതമാക്കി. സൈന്യങ്ങളുടെ മുറവിളി എങ്ങും കേള്‍ക്കായി. നഗരം മുഴുവന്‍ പൊടിയും കട്ടപിടിച്ച പുകയും നിറഞ്ഞു. വിദുരഥന്‍ ശത്രുപാളയത്തില്‍ പ്രവേശിച്ചതും വലിയ ടക്‌ ടക്‌ ശബ്ദം കേട്ടു. യുദ്ധം തീവ്രമായി തുടര്‍ ന്നു. പറക്കുന്ന വ്യോമായുധങ്ങള്‍ തമ്മിലിടിക്കവേ ഖടഖടാരവവും ഛുണുഛുണു നാദവും അവിടെ മുഴങ്ങി.
