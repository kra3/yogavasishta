\section{ദിവസം 201}

\slokam{
യദ്യദ് രാഘവ സംയാതി മഹാജനസപര്യയാ\\
ദിനം തദിഹ സാലോകം ശേഷാസ്ത്വന്ധാ ദിനാലയ: (5/4/12)\\
}

വാല്‍മീകി തുടർന്നു: നേരം പുലർന്നപ്പോൾ രാമനും മറ്റുള്ളവരും എഴുന്നേറ്റ് പ്രഭാതകർമ്മങ്ങൾ ചെയ്ത ശേഷം വസിഷ്ഠമുനിയുടെ  ഗൃഹത്തിലേക്കുചെന്നു. മഹർഷി ഗാഢമായ ധ്യാനത്തിലായിരുന്നു. അദ്ദേഹം ധ്യാനമുണർന്നപ്പോൾ എല്ലാവരുംകൂടി ഒരു  തേരിൽ കയറി ദശരഥന്റെ രാജധാനിയിലേക്കുപോയി. രാജാവവരെ സ്വീകരിക്കാൻ ഉപചാരപൂർവ്വം മൂന്ന് അടിവെച്ചു. അതുകഴിഞ്ഞ് സഭാവാസികളോരോരുത്തരായി ആസനസ്ഥരായി. അവരിൽ ദേവന്മാരും ഉപദേവതമാരും മഹർഷിമാരും ഉണ്ടായിരുന്നു.

ദിനാരംഭത്തിനു നാന്ദികുറിച്ച് ദശരഥരാജൻ ഇങ്ങിനെ പറഞ്ഞു: ഭഗവൻ, ഇന്നലത്തെ തുടർച്ചയായ പ്രഭാഷണപരമ്പരയ്ക്കുശേഷം അങ്ങേക്ക് ഉചിതമായ വിശ്രമം ലഭിച്ചുവെന്നു കരുതുന്നു. അങ്ങയുടെ പരമവിജ്ഞാനപ്രദമായ പ്രഭാഷണം ഞങ്ങളെ അറിവിന്റെ ഉന്നതശിഖരങ്ങളിലെത്തിച്ചിരിക്കുന്നു. തീർച്ചയായും പ്രബുദ്ധരായ മാമുനിമാരുടെ വാക്കുകൾ ജീവജാലങ്ങളുടെ ദു:ഖനിവാരണം വരുത്തുവാനുതകുന്നതും അവരിൽ ആനന്ദാനുഗ്രഹം ചൊരിയുന്നവയുമാണ്‌. നമ്മുടെ സ്വയംകൃതാനർത്ഥങ്ങളെ നീക്കി മനസ്സിനെ നിർമ്മലമാക്കാൻ അവയ്ക്കു കഴിയുന്നു. ലൗകീകാസക്തി, അത്യാഗ്രഹം തുടങ്ങിയ ദുഷ്ച്ചിന്തകളെ ക്ഷീണിപ്പിക്കാൻ ആ വാക്കുകൾക്കാവുന്നു. ഞങ്ങൾ മോഹത്തിനടിപ്പെട്ട് ഈ പ്രത്യക്ഷലോകത്തെ യാഥാർത്ഥ്യമെന്നു കരുതിയാണിരുന്നത്. ആ വിശ്വാസത്തിനൊരു വെല്ലുവിളിയാണവിടുത്തെ വാക്കുകൾ.  “രാമാ, ജ്ഞാനികളായ മാമുനിമാരെ പൂജിച്ചുബഹുമാനിക്കുന്ന ദിനങ്ങളെ മാത്രമേ ഫലപ്രദമായി നമുക്കു കണക്കാക്കുവാനാകൂ. മറ്റു ദിനങ്ങൾ വെറും അന്ധകാരം മാത്രം”. ഇതു നിനക്കുള്ള സുവർണ്ണാവസരമാണ്‌.. ഈ മഹർഷിയിൽ നിന്നും നിനക്ക് വേണ്ട ജ്ഞാന വിജ്ഞാനാദികള്‍ ആരാഞ്ഞറിയൂ. പഠനയോഗ്യമായ കാര്യങ്ങൾ എന്തെന്നറിയൂ.

വസിഷ്ഠൻ പറഞ്ഞു: രാമാ, ഞാൻ നിനക്കുപദേശിച്ച കാര്യങ്ങൾ നീ ഗാഢമായി  ധ്യാനിക്കുകയുണ്ടായോ? രാത്രിയിൽ നീ അവയെക്കുറിച്ച് ചിന്തിച്ചുവോ? നിന്റെ ഹൃദയഭിത്തിയിൽ അവ തെളിഞ്ഞുകിടക്കുന്ന രേഖാചിത്രങ്ങളായിത്തീർന്നുവോ? നീയോർക്കുന്നുവോ മനുഷ്യൻ മനസ്സാണെന്നു ഞാൻ പറഞ്ഞത്? ഈ ലോകസൃഷ്ടികളെപ്പറ്റി ഞാൻ വിശദമായിപ്പറഞ്ഞത് നീ ഓർത്തുവോ? തുടർച്ചയായി ഇവയെല്ലാം ആവർത്തിച്ചോർക്കുന്നതുകൊണ്ടു മാത്രമേ ഈദൃശമായ ആശയങ്ങൾക്കു തെളിമയുണ്ടാവൂ.

രാമൻ പറഞ്ഞു: ഭഗവൻ, ഞാൻ അങ്ങുപറഞ്ഞതുമാത്രമാണ്‌ രാത്രിയിൽ ചെയ്തത്. നിദ്രയുപേക്ഷിച്ച് രാത്രിമുഴുവൻ അങ്ങയുടെ പ്രബുദ്ധമായ ഉപദേശങ്ങളെ ഞാൻ ധ്യാനിക്കുകയായിരുന്നു. അവയിലൂടെ അങ്ങു കാണിക്കാൻ ശ്രമിച്ച സത്യത്തെ കാണാൻ ഞാൻ ശ്രമിച്ചു. ആ സത്യമിപ്പോൾ എന്റെയുള്ളിൽ ചിരപ്രതിഷ്ഠമാണ്‌. ഏറ്റവും ഉയർന്ന തലത്തിലുള്ള ആനന്ദത്തെ പ്രദാനം ചെയ്യുമെന്നറിഞ്ഞാൽപ്പിന്നെ ആരെങ്കിലും അങ്ങയുടെ വാക്കുകളെ ശിരസാവഹിക്കാതിരിക്കുമോ? അവ കേൾക്കാൻ അതീവ മധുരിമയാർന്നവയത്രേ. അവ നമ്മിലുണർത്തുന്നത് അതിപാവനമായ ഒരു പവിത്രതാ ഭാവമാണ്. അത്  നമ്മെ നയിക്കുന്നതോ, സമാനതകളില്ലാത്ത ഒരനുഭവമേഘലയിലേയ്ക്കുമാണ്‌. ഭഗവൻ, അങ്ങ്  പ്രഭാഷണം തുടർന്നാലും. 

