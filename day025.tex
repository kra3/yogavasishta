 
\section{ദിവസം 025}

\slokam{
അശുഭേഷു സമാവിഷ്ടം  ശുഭേഷ്വേവാവതാരയേത്\\
പ്രയത്നാചിത്തമിത്യേഷ സർവശാസ്ത്രാര്‍ത്ഥസംഗ്രഹ: (2/7/12)\\
}

വസിഷ്ഠന്‍ തുടര്‍ന്നു: രാമ: വീണ്ടും വീണ്ടും ജന്മമെടുക്കാതിരിക്കാന്‍ ഒരുവന്‍ ശാരീരികമായ അസുഖങ്ങളോ മാനസീകമായ അസ്വസ്ഥതകളോ ഇല്ലാത്തപ്പോള്‍ ആത്മജ്ഞാനത്തിനായി പ്രയത്നിക്കണം. അങ്ങിനെയുള്ള ഉദ്യമങ്ങള്‍ക്ക്‌ മൂന്നുവേരുകളാണുള്ളത്‌ - ആന്തരീകമായി ബോധത്തിലെ ഉണര്‍വ്വ്‌, മനസ്സിലെ നിശ്ചയദാര്‍ഢ്യം, ശാരീരികമായ പ്രവൃത്തി.

സ്വപ്രയത്നം ഇനിപ്പറയുന്ന മൂന്നുകാര്യങ്ങളെ ആസ്പദിച്ചാണു നിലകൊള്ളുന്നത്‌. വേദശാസ്ത്രങ്ങളിലുള്ള അറിവ്‌, ഗുരുവിന്റെ നിര്‍ദ്ദേശങ്ങള്‍, പിന്നെ സ്വന്തം അദ്ധ്വാനം. ദിവ്യനിയോഗം അല്ലെങ്കില്‍ വിധി എന്നതിന്‌ ഇവിടെ സ്ഥാനമില്ല. "അതുകൊണ്ട്‌ മുക്തിയാഗ്രഹിക്കുന്നവന്‍ അനവരതമായ സ്വപ്രയത്നം കൊണ്ട്‌ മലിനമായ മനസ്സിനെ പരിശുദ്ധിയിലേയ്ക്കു വഴിതിരിച്ചുവിടാനുള്ള പരിശ്രമം ചെയ്യണം. വേദശാസ്ത്രങ്ങളുടെയെല്ലാം സാരാംശം ഇതത്രേ" 

അനശ്വരമായ നന്മയുടെ പാതയില്‍ സ്ഥിരോല്‍സാഹത്തോടെ അടിവച്ചു മുന്നേറാനാണ്‌ മഹാത്മാക്കള്‍ ഏവരും ആഹ്വാനം ചെയ്യുന്നത്‌. തന്റെ പരിശ്രമത്തിന്റെ തീവ്രതയ്ക്കനുസരിച്ചാണ്‌ ഫലാനുഭവങ്ങളുണ്ടാവുക എന്നും നിയതിക്കോ ഈശ്വരനോ ഇതിന്റെമേല്‍ നിയന്ത്രണമൊന്നുമില്ലെന്നും ജ്ഞാനമുറച്ചവനറിയാം. അങ്ങിനെയുള്ള പ്രയത്നങ്ങളാണ്‌ മനുഷ്യന്‌ ഈ ജീവിതത്തില്‍ കിട്ടുന്ന ഏതനുഭവങ്ങള്‍ ക്കും ഉത്തരവാദി.

ഒരുവന്‍ ദു:ഖത്തില്‍ മുഴുകുമ്പോള്‍ അവനെ സമാശ്വസിപ്പിക്കാന്‍ ആളുകള്‍ പറയും 'ഇത്‌ വിധിയെന്നു സമാധാനിക്കുക'. ഒരുവന്‍ വിദേശത്ത്‌ എത്തുന്നു- യാത്രചെയ്തിട്ട്‌; ഒരുവന്‍ വിശപ്പടക്കുന്നു- ഭക്ഷണം കഴിച്ചിട്ട്‌. ഇതെല്ലാം വളരെ സ്പഷ്ടമായ കാര്യങ്ങളാണ്‌. വിധിക്കിതില്‍ പങ്കൊന്നുമില്ല. ആരും ഈ വിധിയേയോ, ഈശ്വരനേയൊ കണ്ടിട്ടില്ല. എന്നാല്‍ നല്ലതോ ചീത്തയോ ആയ കര്‍മ്മങ്ങള്‍ക്കു ഫലമായി നന്മയോ തിന്മയോ ഉണ്ടാവുന്നത്‌ നാം കണ്ടിട്ടുണ്ട്‌. അതുകൊണ്ട്‌ ചെറുപ്പം മുതലേ തന്നെ ഒരുവന്‍ തന്റെ സത്യാന്യോഷണവും മുക്തിസാധനയും തുടങ്ങണം. വേദശാസ്ത്രങ്ങളെ ബുദ്ധിപൂര്‍വ്വം പഠിച്ച്‌ മഹാത്മാക്കളുടെ സംസര്‍ഗ്ഗത്തോടെ സ്വപ്രയത്നത്താല്‍ ഇതു സാധിക്കാം. വിധി, ദൈവനിയോഗം എന്നതൊക്കെ പലതവണ ആവര്‍ത്തിച്ചു പറയുന്നതുകൊണ്ടുമാത്രം സത്യമെന്ന തോന്നലുളവാക്കുന്ന സൌകര്യപ്രദമായ ഒരു സങ്കല്‍പ്പമാണ്‌. ഈ ദൈവനിയോഗം സത്യമായും എല്ലാത്തിന്റേയും നിയന്താവാണെങ്കില്‍ എന്തിനാണ്‌ നാം പ്രയത്നങ്ങള്‍ ചെയ്യുന്നത്‌? ആരാരെയാണ്‌ കര്‍മ്മം ചെയ്യാന്‍ പഠിപ്പിക്കേണ്ടത്‌? ഈ ലോകത്ത്‌ ശവങ്ങളൊഴിച്ച്‌  മറ്റെല്ലാവരും കര്‍മ്മോന്മുഖരായി വര്‍ത്തിക്കുന്നു. അവര്‍ക്കെല്ലാം ഉചിതമായ ഫലങ്ങള്‍ ലഭിക്കുന്നുമുണ്ട്‌. ആരും വിധിയേയോ ദൈവനിയോഗത്തെയോ സാക്ഷാത്കരിച്ചിട്ടില്ല.

ചിലര്‍ പറഞ്ഞേക്കാം "വിധിയുടെ പ്രേരണയാലാണ്‌ അല്ലെങ്കില്‍ ദൈവനിശ്ചയത്താലാണ്‌ ഞാനിതുചെയ്തത്‌" എന്നെല്ലാം. പക്ഷേ അതു സത്യമല്ല. ഒരു യുവാവ്‌ അതിവിദ്വാനായ ഒരു പണ്ഡിതനാവുമെന്ന് ജ്യോതിഷി പ്രസ്താവിക്കുന്നു എന്നുകരുതുക. അയാള്‍ ഒന്നും പഠിക്കാതെ അങ്ങിനെയായിത്തീരുമോ? ഇല്ല. അപ്പോള്‍പ്പിന്നെ ഈദിവ്യനിയോഗങ്ങളില്‍ നാമെന്തിനു വിശ്വസിക്കണം? "രാമ: ഈ വിശ്വാമിത്രന്‍ ബ്രഹ്മര്‍ഷിയായത്‌ സ്വപ്രയത്നത്താലാണ്‌. നാമെല്ലാം ആത്മജ്ഞാനത്തിനര്‍ഹരായതും അതുകൊണ്ടുമാത്രം. അതുകൊണ്ട്‌ ഈ ഗ്രഹപ്പിഴ, വിധി എന്നുള്ള വിശ്വാസങ്ങളെല്ലാം ഉപേക്ഷിച്ച്‌ ആത്മജ്ഞാനത്തിനായി പ്രവര്‍ത്തനനിരതനായാലും."
