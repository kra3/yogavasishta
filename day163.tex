\section{ദിവസം 163}

\slokam{
വിചാരണാ പരിജ്ഞാതസ്വഭാവസ്യോദിതാത്മന:\\
അനുകമ്പ്യാ ഭവന്തീഹ ബ്രഹ്മ വിഷ്ണ്വിന്ദ്രശങ്കരാ: (4/22/15)\\
}

വസിഷ്ഠൻ തുടർന്നു: കിണറിനുള്ളിലിട്ട ഒരുകഷണം തുരിശ് (ആലം) അതിലെ ജലത്തെ  ശുദ്ധീകരിക്കുന്നതുപോലെ ആത്മാന്വേഷണത്തിലൂടെ ജ്ഞാനമാർജ്ജിച്ച ഒരുവൻ തികഞ്ഞ തെളിമയോടെ മനസ്സുണർന്ന്‍ ആത്മജ്ഞാനിയാവുന്നു. അവന്റെ മനസ്സിൽ മാറ്റങ്ങൾകൊണ്ട് അല്ലലൊന്നും ഉണ്ടാവുന്നില്ല. അവനിലെ സ്വത്വബോധം, ആത്മജ്ഞാനമായതിനാൽ അവനിൽ വിഷയ-വസ്തു ധാരണകൾ ഇല്ലേയില്ല. ദൃക്ക്, ദൃശ്യം, ദൃഷ്ടാവ് എന്നീ ഭേദചിന്തകൾ അവനിലില്ല. അവൻ പരംപൊരുളിൽ, സത്യത്തിൽ, പൂർണ്ണമായി ഉണർന്നിരിക്കുന്നു. പ്രകടമായ ലോകത്തെ സംബന്ധിച്ചിടത്തോളം അവൻ സമ്പൂർണ്ണ സുഷുപ്തിയിലാണ്ടിരിക്കുന്നു എന്നു പറയാം. അവന്റെ നിർമമതയുടെ സർവ്വവ്യാപിത്വം കാരണം അവന്‌ സുഖത്തിലോ ദു:ഖത്തിലോ ആസക്തിയില്ല. അവനിലെ ആർത്തികളെല്ലാം അടങ്ങിയിരിക്കുന്നു. നദിയുടെ കുത്തൊഴുക്ക് സമുദ്രത്തിലെത്തും വരെ മാത്രമാണല്ലോ.

കാണപ്പെടുന്ന ഈ ലോകമെന്ന വല, എലി കയർ കരണ്ടുമുറിക്കുന്നതുപോലെ അവൻ മുറിച്ചു കളഞ്ഞിരിക്കുന്നു. മനസ്സ് ഭ്രമകല്‍പ്പനകളുടെ ബന്ധനത്തിൽ നിന്നും മോചിക്കപ്പെടണമെങ്കിൽ അത് എല്ലാ ആസക്തികളിൽനിന്നും മുക്തമായിരിക്കണം. പരസ്പരവിരുദ്ധമായ ദ്വന്ദഭാവങ്ങളുടെ പിടിയിൽപ്പെടാത്തതായിരിക്കണം. വസ്തു-വിഷയങ്ങളോട് ആകർഷണം ഇല്ലാത്തതായിരിക്കണം. ഉപാധികളെ ആശ്രയിക്കാതെ നിലകൊള്ളാനും മനസ്സിനാകണം. സംശയങ്ങളൊഴിഞ്ഞ് ശാന്തമായ മനസ്സിൽ ആഹ്ളാദമോ വിഷാദമോ ചലനമുണ്ടാക്കുന്നില്ല. അപ്പോൾ പൂർണ്ണചന്ദ്രനെപ്പോലെ മനസ്സു പ്രശോഭിക്കുന്നു. നിർമ്മലമായ മനസ്സ് ഹൃദയത്തിൽ ഏകാത്മതാദർശനം സാദ്ധ്യമാക്കുന്നു. അവിടെ ശുഭോദർക്കമായ ഗുണഗണങ്ങൾ ഉയരുന്നു. സൂര്യപ്രകാശത്തിൽ ഇല്ലാതാവുന്ന ഇരുട്ടെന്നപോലെ മായാപ്രപഞ്ചം അനന്താവബോധത്തിന്റെ സാന്നിദ്ധ്യംകൊണ്ട് ഇല്ലാതാവുന്നു. അത്തരം വിജ്ഞാനവിവേകം ജീവജാലങ്ങളുടെ ഹൃദയങ്ങളെ ആഹ്ളാദനിരതമാക്കുന്നു. ചുരുക്കത്തിൽ അറിയാൻ അനുയോജ്യമായ അറിവെന്താണോ അതറിഞ്ഞവൻ എല്ലാ മാറ്റങ്ങൾക്കും അതീതനാണ്‌.. അവൻ ജനന-മരണങ്ങൾ എന്ന മാറ്റങ്ങൾക്കുമപ്പുറം കണ്ടവനാണ്‌..

ബ്രഹ്മാവിഷ്ണുമഹേശ്വരന്മാർ പോലും ആത്മജ്ഞാനം ലഭിച്ച മഹാത്മാക്കളോട് അനുഭാവമുള്ളവരാണ്‌.. ആത്മാന്വേഷണത്തിലൂടെയും നേരറിവിലൂടെയും ആത്മജ്ഞാനമാർജ്ജിച്ചവർ ഈ ത്രിമൂർത്തികൾക്കുപോലും സഹായികളായി വർത്തിക്കുന്നു. അഹങ്കാരം ഇല്ലാത്തപ്പോൾ ചിന്താക്കുഴപ്പങ്ങൾ ഒന്നുമില്ലാതെ മനസ്സ് സഹജഭാവമാർജ്ജിക്കുന്നു. കടലിൽ അലകൾ ഉണ്ടായി മറയുന്നതുപോലെ ലോകം ഉണ്ടായി മറയുന്നു. ഈ മാറ്റങ്ങൾ അജ്ഞാനിയെ ഭ്രമിപ്പിക്കുന്നു. ജ്ഞാനിയെ ഇതു ബാധിക്കുന്നതേയില്ല.

ഒരു മൺകുടത്തിനുള്ളിലെ ആകാശം, കുടം ഉണ്ടാക്കുന്നതിന്റെയൊപ്പം ഉണ്ടാവുന്നതോ ഉണ്ടാക്കുന്നതോ അല്ല. കുടമുടയ്ക്കുമ്പോൾ ഈ ആകാശം നശിക്കുന്നതുമല്ല. കുടത്തിന്റെയും ആകാശത്തിന്റെയും (ശരീരത്തിന്റെയും ആത്മാവിന്റെയും) ബന്ധമെങ്ങിനെയെന്ന് അറിയാവുന്നവരെ നിന്ദാസ്തുതികൾ ബാധിക്കുകയില്ല. ഈ വർണ്ണാഭമായ ലോകം ആത്മാന്വേഷണത്തിലേർപ്പെടാത്തവരെ മാത്രമേ വിടാതെ പിന്തുടർന്നു പിടികൂടി ബാധിക്കുകയുള്ളു. ജ്ഞാനമുദിക്കുന്നതോടെ ഒരുവന്റെ ഭ്രമകൽപ്പനകൾ എല്ലാം അസ്തമിക്കുന്നു. 

