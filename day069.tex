\newpage
\section{ദിവസം 069}

\slokam{
ദീർഘസ്വപ്നമിദം വിശ്വം വിദ്വയഹന്താദിസംയുതം\\
അത്രാന്യേ സ്വപ്നപുരുഷാ യഥാ സത്യാസ്തഥാ ശൃണു. (3/42/8)\\
}

സരസ്വതി തുടര്‍ന്നു: അപക്വമതിയായ ഒരാള്‍ , ഈ കാണപ്പെടുന്ന ലോകം യഥാര്‍ത്ഥ്യം തന്നെയെന്നുറച്ചു വിശ്വസിക്കുന്നുവെങ്കില്‍ അത്‌ അങ്ങിനെ തന്നെ തുടരുന്നു. ഭൂതപിശാചുക്കൾ തന്നെ പിന്തുടര്‍ന്നുകൊണ്ടിരിക്കുന്നു എന്നുകരുതുന്ന കുട്ടിയുടെ അവസ്ഥയാണത്‌. ഒരാള്‍ തന്റെ കൈവളയുടെ ഭംഗിയില്‍ സമാകൃഷ്ടനായാല്‍ അതിന്റെ മൂല്യം സ്വര്‍ണ്ണമാണെന്നു കാണാതെപോവുന്നു. കൊട്ടാരങ്ങളുടേയും, നഗരങ്ങളുടേയും ആനകളുടേയും പകിട്ടിലും പ്രൌഢിയിലും മതിമറന്നയാള്‍ അവയുടെ എല്ലാം അടിസ്ഥാനതത്വമായ അനന്താവബോധത്തെ തിരിച്ചറിയുന്നില്ല.

"ഈ വിശ്വം ഒരു സുദീര്‍ഘസ്വപ്നമത്രേ. അഹംകാരവും 'മറ്റുള്ളവര്‍ ' എന്ന ഭാവവും, എല്ലാം സ്വപ്നവസ്തുക്കള്‍ എന്നപോലെ അയാഥാര്‍ഥ്യമാണ്‌." ഒരേയൊരുണ്മ അനന്താവബോധം മാത്രം. അത്‌ സര്‍വ്വവ്യാപിയും നിര്‍മ്മലവും, പ്രശാന്തവും, സര്‍വ്വശക്തിമാനുമാകുന്നു. അതിന്റെ ശരീരം 'അറിയപ്പെടവുന്ന' ഒരു 'പദാര്‍ത്ഥ'മല്ല. അതു ശുദ്ധപ്രജ്ഞയാകുന്നു. ബോധം, എവിടെ എങ്ങിനെ  പ്രകടമാവുന്നുവോ അതെല്ലാം ബോധം തന്നെ. അടിസ്ഥാനം ശാശ്വതവും സത്യവുമായതിനാല്‍ അതിനെ കാരണമാക്കി പ്രകടമാവുന്നതിനെല്ലാം യാഥാര്‍ഥ്യഭാവം സഹജമായും ഉണ്ടെങ്കിലും സത്തായിട്ടുള്ളത്‌ ആ അടിസ്ഥാനം മാത്രം. ഈ വിശ്വപ്രപഞ്ചവും അതിലുള്ളതെല്ലാം ഒരു നീണ്ട സ്വപ്നമാണ്‌. അങ്ങെനിക്ക്‌ സത്തായിതോന്നുന്നു. അങ്ങേയ്ക്കു ഞാനും ഉണ്മയാണ്‌. നമുക്ക്‌ മറ്റുള്ളവരും സത്യമാണ്‌. ഈ ആപേക്ഷിക യഥാര്‍ഥ്യം സ്വപ്നവസ്തുക്കളെപ്പോലെയാണ്‌. അയാഥാർത്ഥ്യം.

രാമന്‍ ചോദിച്ചു: മഹാത്മന്‍, ഒരാള്‍ സ്വപ്നത്തില്‍ ദര്‍ശിക്കുന്ന നഗരം യഥാര്‍ഥത്തില്‍ ഉള്ളതാണെങ്കില്‍ അതൊരു നഗരമായിത്തന്നെ തുടരുന്നു എന്നല്ലേ അങ്ങുദ്ദേശിച്ചത്‌?

വസിഷ്ഠന്‍ പറഞ്ഞു: ശരിയാണ്‌ രാമ: സ്വപ്നനഗരം എന്നത്‌ അനന്താവബോധത്തിന്റെ അടിസ്ഥാനത്തിലുള്ളതാകയാല്‍ സ്വപ്നവസ്തുക്കള്‍ സത്യമാണെന്നു പ്രത്യക്ഷത്തില്‍ തോന്നുന്നു. എന്നാല്‍ ഉണര്‍ന്നിരിക്കുന്ന അവസ്ഥയും സ്വപ്നാവസ്ഥയും തമ്മില്‍ യാതൊരു വ്യത്യാസവും ഇല്ല. കാരണം, ഒരാളുടെ യാഥാര്‍ഥ്യം മറ്റൊരാള്‍ക്ക്‌ അങ്ങിനെയാവണമെന്നില്ല. അതുകൊണ്ട്‌ ഈ രണ്ടവസ്ഥകളും ഒരുപോലെയാണ്‌. അതുകൊണ്ട്‌ സ്വപ്നാവസ്ഥയും ജാഗ്രതവസ്ഥയും അയാഥാര്‍ഥമാണ്‌. അനന്താവബോധത്തിന്റെ  അടിസ്ഥാനത്തിലാണ്‌ അവയെല്ലാം ആരോപിതമാകുന്നത്‌. ഇത്രയും കാര്യങ്ങള്‍ രാജാവിനെ ഉദ്ബോധിപ്പിച്ച്‌ അനുഗ്രഹിച്ചശേഷം സരസ്വതീ ദേവി പറഞ്ഞു: താങ്കള്‍ക്ക്‌ എല്ലാവിധ ഐശ്വരങ്ങളും തുണയാകട്ടെ. കാണേണ്ടതെല്ലാം അങ്ങു കണ്ടുകഴിഞ്ഞു. ഞങ്ങൾ മടങ്ങിപ്പോവട്ടെ.

വിദുരഥന്‍ പറഞ്ഞു: ദേവീ, ഞാനുടനേതന്നെ ഇവിടെനിന്നും പുറപ്പെടുകയായി. ഒരു സ്വപ്നത്തില്‍ നിന്നും മറ്റൊരു നിദ്രയിലേയ്ക്ക്‌. എന്റെ മന്ത്രിമാരേയും കന്യകയായ എന്റെ മകളേയും കൂടെ കൊണ്ടുപോവാന്‍ അനുമതി നല്‍കിയാലും. ദേവി അദ്ദേഹത്തിന്റെ ആഗ്രഹം നിറവേറ്റി.
