\section{ദിവസം 161}

\slokam{
ദേവാന്‍ ദേവയജോ യാന്തി യക്ഷ യക്ഷാന്‍ വ്രജന്തി ഹി\\
ബ്രഹ്മ ബ്രഹ്മയജോ യാന്തി യദതുച്ഛം തദാശ്രയേത് (4/19/5)\\
}

വസിഷ്ഠൻ തുടർന്നു: എല്ലാ ജീവനുകളുടേയും ബീജം പരബ്രഹ്മമാണ്‌...  അത് എല്ലായിടവും നിറഞ്ഞു നിൽക്കുന്നു. മാത്രമല്ല, ജീവനുകൾക്കുള്ളിൽ മറ്റ് എണ്ണമില്ലാത്തത്ര ജീവനുകൾ ഉണ്ട് താനും. അനന്താവബോധം പരിപൂർണ്ണമായും ഈ വിശ്വത്തെ മുഴുവൻ വിലയനം ചെയ്തിരിക്കുന്നതുകൊണ്ടാണിതു സാധിക്കുന്നത്. ജീവനുകളായി പ്രത്യക്ഷമായാൽപ്പിന്നെ അതത് ജീവൻ അവലംബിക്കുന്ന ധ്യാനാവസ്ഥയ്ക്കനുസൃതമായി വൈവിദ്ധ്യമാര്‍ന്ന സഹജഭാവം ആർജ്ജിക്കുന്നു. "ഈശ്വരഭക്തിയുള്ളവർ ഈശ്വരനെ പ്രാപിക്കുന്നു. ഉപദേവതമാരെ പൂജിക്കുന്നവർ അവരെ പ്രാപിക്കുന്നു. പരബ്രഹ്മത്തെ ധ്യാനിക്കുന്നവർ ബ്രഹ്മം തന്നെയാവുന്നു. അതിനാൽ ഒരുവൻ നിയതമായതിനെ, പരിമിതമായതിനെ അവലംബിക്കാതെ അപരിമേയമായതിനെ ധ്യാനിക്കുന്നതാണ്‌ അഭികാമ്യം."  അപ്സരസ്സിനെ ധ്യാനിച്ച് ശുക്രൻ ബന്ധിതനായി. എന്നാൽ തന്റെ ആത്മശുദ്ധി, അതായത് അനന്താവബോധം സാക്ഷാത്കരിച്ചപ്പോൾ അദ്ദേഹം ബന്ധവിമുക്തനുമായി.

രാമൻ ചോദിച്ചു: മഹാത്മൻ, ദയവായി ജാഗ്രത്ത്, സ്വപ്നാവസ്ഥകളുടെ സത്യമെന്തെന്ന്‌ പറഞ്ഞു തരൂ. എന്താണ്‌ ജാഗ്രത്തായി അനുഭവവേദ്യമാകുന്നത്? എങ്ങിനെയാണ്‌ സ്വപ്നവും ഭ്രമങ്ങളും ജാഗ്രദവസ്ഥയിൽ ഉയരുന്നത്?

വസിഷ്ഠൻ പറഞ്ഞു: നീണ്ടു നിലനിൽക്കുന്ന അവസ്ഥയാണ്‌ ജാഗ്രദവസ്ഥ. സ്വപ്നമോ താൽ ക്കാലികമാണ്‌..  എന്നാൽ സ്വപ്നാവസ്ഥയിൽ അത് ജാഗ്രദവസ്ഥയുടെ സ്വഭാവസവിശേഷതകൾ പ്രദർശിപ്പിക്കുന്നു. ജാഗ്രദവസ്ഥയിൽ, അതിന്റെ ക്ഷണഭംഗുരത മൂലം അത്‌ സ്വപ്നസ്വഭാവം ആർജ്ജിക്കുന്നു. അതായത് രണ്ടും വാസ്തവത്തിൽ ഒരേ അവസ്ഥ തന്നെ. ഒരുശരീരത്തിൽ ജീവശക്തി ചലിക്കുമ്പോൾ ചിന്തകളുടെ, വാക്കുകളുടെ, കർമ്മങ്ങളുടെ എല്ലാം അവയവങ്ങൾ പ്രവർത്തനോന്മുഖമാകുന്നു. അവ മനസ്സിലുണ്ടായ ഭ്രമധാരണകളുടെ അടിസ്ഥാനത്തിൽ അതതിന്റെ വിഷയങ്ങളിലേയ്ക്ക് ഓടിയണയുകയാണ്‌..  ജീവശക്തി ആത്മാവിനുള്ളിൽ വൈവിദ്ധ്യമാർന്ന രൂപങ്ങൾ ധരിക്കുന്നു. ഈ ധാരണകൾക്ക് ദൃഢതയുള്ളതായി തോന്നുമ്പോൾ അത് ജാഗ്രദവസ്ഥ. എന്നാൽ ഈ ജീവശക്തിയെ (ജീവചേതന) മനസ്സും ശരീരവും ഇപ്രകാരത്തിൽ വഴി തെറ്റിച്ചു വിടാതിരുന്നാൽ അത് ഹൃദയത്തിൽ പ്രശാന്തതയായിത്തീരുന്നു. ശരീരത്തിലെ നാഡീവ്യൂഹങ്ങളിലൂടെ ബോധം ചലിക്കുന്നില്ല. ജീവശക്തി ഇന്ദ്രിയങ്ങളെ ചലിപ്പിക്കുന്നുമില്ല.  എന്നാൽ ദീർഘനിദ്രയിലും ഉണർന്നിരിക്കുന്ന, സ്വപ്നത്തിലും ജാഗ്രത്തിലും വെളിച്ചമായി പരിലസിക്കുന്ന ആ ബോധം, അതീന്ദ്രിയമാണ്‌..  അതിന്‌ തുരീയം എന്നു പറയുന്നു. എന്നാൽ പിന്നീട് അജ്ഞതയുടെയും ഭ്രമത്തിന്റെയും ബീജങ്ങൾ വളർന്ന് 'ആദ്യചിന്ത' ഉദയം ചെയ്യുന്നു. ‘ഞാൻ’ ഉണ്ട് എന്ന ചിന്ത. എന്നിട്ട് ചിന്താരൂപങ്ങളെ മനസ്സിനുള്ളിൽ സ്വപ്നങ്ങളായി കാണുന്നു. ഈ സമയം ബാഹ്യേന്ദ്രിയങ്ങൾ പ്രവർത്തനോന്മുഖമല്ല. അന്തരേന്ദ്രിയങ്ങൾ പ്രവർത്തിക്കുകയും ഒരുവനുള്ളിൽ ധാരണകൾ സംജാതമാവുകയും ചെയ്യുന്നു. ഇത് സ്വപ്നാവസ്ഥ. ജീവശക്തി വീണ്ടും ഇന്ദ്രിയങ്ങളെ സംചലിപ്പിക്കുമ്പോൾ വീണ്ടും ജാഗ്രദവസ്ഥയായി. 

