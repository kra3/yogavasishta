\newpage
\section{ദിവസം 143}

\slokam{
സംബന്ധേ ദൃശ്യദൃഷ്ടീനാം മദ്ധ്യേ ദൃഷ്ടുർ ഹി യദ്വപു:\\
ദൃഷ്ടുർദർശന ദൃശ്യാദിവർജിതം തദിതം പരം (3/121/53)\\
}

വസിഷ്ഠൻ തുടർന്നു: എല്ലാ ബന്ധങ്ങളും മുൻപേ നിലവിലുണ്ടായിരുന്ന ചാര്‍ച്ചയുടെ  സാക്ഷാത്കാരമാണ്‌....   മുന്‍പു തന്നെയുള്ള  വിഷയം, വിഷയി മുതലായ  മോഹനിബദ്ധമായ അനുമാനങ്ങൾ മൂലമാണ്‌ ഈ ‘ബാന്ധവം’ എന്ന ധാരണതന്നെയുണ്ടാവുന്നത്. വാസ്തവത്തിൽ അനന്താവബോധം മാത്രമേയുള്ളു. അതുകൊണ്ട് രാമാ, ഈ വിശ്വത്തെ അനന്താവബോധമായി സാക്ഷാത്കരിച്ചാലും. ആ ബോധത്തിന്റെ പ്രഭാവമാണ്‌ ഇക്കാണായ മായജാലങ്ങളെല്ലാം. എങ്കിലും യാതൊന്നും 'സംഭവിച്ചിട്ടില്ല' എന്നതാണു സത്യം. സർവ്വവും നിറഞ്ഞുനിൽക്കുന്ന ആ ‘ഒന്നിൽ’ കൂടുതലായി മറ്റ് ഒന്നും ചേർക്കാൻ കഴിയില്ലല്ലോ. ആകാശത്ത് ഭാവനയിലുണ്ടാക്കിയ ഒരു നഗരം പണിതുയർത്തുംപോലെ മാത്രമേ അനന്തതയിലേയ്ക്ക് എന്തെങ്കിലും കൂട്ടിച്ചേർക്കുവാനാകൂ. സ്വർണ്ണത്തിന്റെ സാന്നിദ്ധ്യം മറന്നാലേ കൈവള കാണുകയുള്ളൂ. കൈവള സ്വർണ്ണത്തിന്റെ തന്നെ ഒരു മായക്കാഴ്ച്ചയാണെന്നതുപോലെ മിഥ്യയാണ്‌ രാജ്യം, ലോകം, ആവർത്തിച്ചുള്ള ജനനമരണങ്ങൾ, തുടങ്ങിയ എല്ലാ ധാരണകളും.

ആഭരണത്തിനെക്കുറിച്ചുള്ള തെറ്റിദ്ധാരണ നീങ്ങുമ്പോൾ സ്വർണ്ണമെന്ന സത്യം കാണാകും. അതുപോലെ വിഷയം-വിഷയി (ദൃക്ക്-ദൃശ്യം) എന്ന 'തെറ്റിദ്ധാരണ' ഉപേക്ഷിച്ചാൽപ്പിന്നെ ‘ഒന്നിനെ’ പലതാക്കി വിഭജനം ചെയ്യുന്ന അവിദ്യയ്ക്ക് അവിടെ ഇടമില്ല. ചിന്തകളാണ്‌ എല്ലാ വൈവിദ്ധ്യങ്ങള്‍ക്കും മോഹദൃശ്യങ്ങൾക്കും കാരണം. ചിന്തകളവസാനിക്കുമ്പോൾ സൃഷ്ടിയും അവസാനിച്ചു. അപ്പോൾ കടലിലെ തിരകൾ കടൽ തന്നെയെന്നു തിരിച്ചറിയുന്നു. മരപ്പാവകൾ മരമായും, മൺകുടങ്ങൾ മണ്ണായും, ത്രിലോകങ്ങൾ എല്ലാം പരബ്രഹ്മമായും വെളിപ്പെടുന്നു.

ദൃശ്യത്തിനും ദർശനം എന്ന പ്രക്രിയക്കുമിടയിൽ മൂന്നാമതായി ദൃഷ്ടാവുണ്ട്. ഈ മൂന്നും (ദൃശ്യം, ദർശനം, ദൃഷ്ടാവ്) തമ്മിലുള്ള അന്തരമവസാനിപ്പിക്കുമ്പോൾ പരമ്പൊരുളായി. മനസ്സ് ഒരു രാജ്യത്തു നിന്ന്‌ മറ്റൊരിടത്തേയ്ക്കു പോകുമ്പോൾ വിശ്വപ്രജ്ഞ കൂടെയുണ്ട്. അതങ്ങിനെയായിരിക്കട്ടെ. നിന്റെ ഉണ്മയായ അവസ്ഥ പരിമിതമായ ജാഗ്രത്-സ്വപ്ന-സുഷുപ്തികളല്ല. പിന്നെയോ, അത് ശാശ്വതമായതും, അറിവിനപ്പുറമുള്ളതും, ജഢത തൊട്ടുതീണ്ടാത്തതുമാണ്‌.   അതങ്ങിനെ തന്നെയായിരിക്കട്ടെ. നിന്റെ ഉൽസാഹമില്ലായ്മയെല്ലാം മാറ്റി ഹൃദയത്തിൽ സത്യത്തെ ഉറപ്പിക്കൂ. പിന്നെ ധ്യാനനിരതനായാലും, കർമ്മനിരതനായാലും വേണ്ടില്ല, ആസക്തിയും, വെറുപ്പും, ശരീരാഭിമാനവും വെടിഞ്ഞ് അങ്ങിനെതന്നെ തുടരുകയാണ് വേണ്ടത്. ഭാവിയിൽ ഉണ്ടായേക്കാന്‍ പോവുന്ന ഒരു ഗ്രാമത്തിന്റെ കാര്യങ്ങളെക്കുറിച്ച് ഇപ്പോൾ നീ വ്യഗ്രതപ്പെടാത്തതുപോലെ, മനസ്സിന്റെ ഗതിമാറ്റത്തിനനുസരിച്ച് ചാഞ്ചാടാതിരിക്കുക. സത്യത്തിൽ മാത്രം  ചിത്തമുറപ്പിക്കുക.

മനസ്സിനെ ഒരു വിദേശിയായോ, ഒരു കഷണം കല്ലോ മരമായോ കണക്കാക്കുക. അനന്താവബോധത്തിൽ മനസ്സില്ല. അതുകൊണ്ട് ഈ അയഥാർത്ഥമായ മനസ്സുണ്ടാക്കുന്നതൊന്നും യാഥാർഥ്യമല്ലെന്ന സത്യത്തിൽ അടിയുറപ്പിക്കുക. മനസ്സിന്‌ അസ്തിത്വമില്ല. അഥവാ താല്‍ക്കാലികമായി ഉണ്ടായിരുന്നുവെങ്കിൽത്തന്നെ അതിന്‌ അന്ത്യമായിരിക്കുന്നു. മരിച്ചുവെങ്കിലും ഈ മനസ്സ് എല്ലാം കാണുന്നു.  അതുകൊണ്ട് അതു നൽകുന്നത് തെറ്റായ ധരണകളാണ്‌...  ഈ അറിവ് ഉറപ്പാക്കുക. അസ്തിത്വമില്ലാത്ത ഈ മനസ്സിന്റെ വരുതിയിൽ കാലം കഴിച്ചുകൂട്ടുന്നവൻ ഭ്രാന്തനത്രേ. അവന്‌ ഇടിമിന്നൽപ്പിണരുകള്‍  ചന്ദ്രനിൽ നിന്നിറങ്ങി വരുന്നതായി തോന്നും. അതുകൊണ്ട്  മനസ്സ് സത്യമാണെന്ന മിഥ്യാധാരണയെ നീ  ദൂരെ ഉപേക്ഷിക്കുക. എന്നിട്ട് ഉചിതമായി ചിന്തിക്കുകയും ധ്യാനം പരിശീലിക്കുകയും ചെയ്യുക. രാമ: ഞാൻ മനസ്സിനെ ഏറെക്കാലം അന്വേഷിച്ചിരിക്കുന്നു. ഒരിടത്തും അതെനിക്ക് കണ്ടെത്തനായിട്ടില്ല. കാരണം അനന്താവബോധം മാത്രമേ സത്തായി നിലനിൽക്കുന്നുള്ളു. 

