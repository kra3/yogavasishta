\newpage
\section{ദിവസം 064}

\slokam{
ഉത്പധ്യോത്പധ്യതേ തത്ര സ്വയം സംവിത്സ്വഭാവത:\\
സ്വസങ്കല്പ്പൈ: ശമം യാതി ബാലസങ്കൽപ്പജാലവത് (3/30/8)\\
}

വസിഷ്ഠന്‍ തുടര്‍ന്നു: സരസ്വതീ ദേവിയോടൊപ്പം ലീല ആകാശത്തേക്കുയര്‍ന്നു. അവര്‍ ധ്രുവ നക്ഷത്രങ്ങള്‍ ക്കും, ഉത്തമരായ മാമുനിമാരിരിക്കുന്നിടത്തിനും, ദേവതകളുടെ ആസ്ഥാനങ്ങള്‍ക്കും ബ്രഹ്മലോകത്തിനും, ഗോലോകത്തിനും ശിവലോകത്തിനും പിതൃലോകത്തിനുമെല്ലാമപ്പുറത്തേയ്ക്ക്‌ പോയി. അവിടെനില്‍ക്കുമ്പോള്‍ സൂര്യചന്ദ്രന്മാരെ, കഷ്ടിച്ചു കാണാമെങ്കിലും അവര്‍ വളരെ താഴത്താണെന്നു കണ്ടു. സരസ്വതി പറഞ്ഞു: കുഞ്ഞേ ഇതിനുമപ്പുറം, സൃഷ്ടിയുടെ ഏറ്റവും ഉന്നത പീഠത്തിലേയ്ക്കു പോകൂ. നീ കണ്ടതൊക്കെ അവിടെനിന്നുയര്‍ന്ന വെറും പൊടിപടലങ്ങള്‍ മാത്രം. ഉടനേതന്നെ അവര്‍ ആ കൊടുമുടിയിലെത്തിച്ചേര്‍ ന്നു. അവിടെയെത്തുന്നവരുടെ ഇച്ഛാശക്തി വജ്രം പോലെ കഠിനവും മറ്യെല്ലാം നീക്കിയ ബോധം നിര്‍മലവുമാവുന്നു. ലീല അവിടെ ജലം, അഗ്നി, വായു, ആകാശം എന്നീ അടിസ്ഥാനഘടകങ്ങളുടെ പടലങ്ങളായി സൃഷ്ടിയെ ദര്‍ശിച്ചു. അതിനുമപ്പുറം ശുദ്ധബോധം മാത്രം. സ്വമഹിമയില്‍ സുസ്ഥാപിതമായ ആ അനന്താവബോധം നിര്‍മ്മലവും, ഭ്രമരഹിതവും പ്രശാന്തവുമാണ്‌. അതിലാണ്‌ ഒഴുകിനടക്കുന്ന പൊടിപടലങ്ങളായി അനേകം സൃഷ്ടിജാലങ്ങളെ ലീല ദര്‍ശിച്ചത്‌. ആ ലോകങ്ങളില്‍ ജീവിക്കുന്നവരുടെ സ്വവിക്ഷേപങ്ങളാണ്‌ അവയ്ക്ക്‌ രൂപഭാവങ്ങള്‍ നല്‍കുന്നത്‌. "ഒരു കുട്ടി യദൃച്ഛയാ കളികളിലേര്‍പ്പെടുന്നപോലെ അനന്താവബോധത്തിന്റെ തല്‍സ്വഭാവം കാരണം ഇതെല്ലാം ഉയര്‍ന്നുണര്‍ന്നുണ്ടായി അതതിന്റെ സ്വന്തം ചിന്താബലം കൊണ്ട്‌ തിരികെ പ്രശാന്തിയടയുകയാണ്‌."

രാമന്‍ ചോദിച്ചു: അനന്തത മാത്രം ഉണ്മയായിരിക്കേ എന്താണാളുകള്‍ 'ഉയരെ', 'താഴെ' എന്നതുകൊണ്ടുദ്ദേശിക്കുന്നത്‌?

വസിഷ്ഠന്‍ പറഞ്ഞു: രാമ: ചെറിയ ഉറുമ്പുകള്‍ ഉരുണ്ട പാറമേല്‍ അരിച്ചുനടക്കുമ്പോള്‍ അവരുടെ കാലിനടിയില്‍ ഉള്ള ഇടങ്ങളെല്ലാം 'താഴെ'യും അവരുടെ പിന്‍ ഭാഗത്തിനു പിറകേയുള്ളവയെല്ലാം 'ഉയരെ'യും ആയിരിക്കുമല്ലോ. അതുപോലെയാണ്‌ ആളുകള്‍ ദിക്കുകളെക്കുറിച്ചു പറയുന്നത്‌. ഈ എണ്ണമറ്റ ലോകങ്ങളില്‍ ചിലതില്‍ സസ്യങ്ങള്‍ മാത്രമേയുള്ളു. ചിലതില്‍ ബ്രഹ്മാവിഷ്ണുമഹേശ്വരന്മാരാണ്‌ അദ്ധ്യക്ഷദേവതകള്‍ . ചിലതില്‍ ജീവജാലങ്ങളൊന്നുമില്ല. മറ്റുചിലതില്‍ പക്ഷിമൃഗാദികള്‍ മാത്രമേയുള്ളു. ചിലതില്‍ കടല്‍ മാത്രം, മറ്റുചിലതില്‍ കരിമ്പാറക്കെട്ടുകള്‍ മാത്രം. ചിലതില്‍ കൃമികീടങ്ങള്‍ മാത്രം, മറ്റുചിലതില്‍ ഘനസാന്ദ്രമായ ഇരുട്ടു മാത്രം. ചിലതില്‍ ദേവതകള്‍ വസിക്കുന്നു. ചിലത്‌ സദാ ഭാസുരപ്രദീപ്തിയില്‍ തിളങ്ങുന്നു. ചിലത്‌ പ്രളയത്തിലേയ്ക്കുള്ള പ്രയാണത്തിലാണ്‌. മറ്റുചിലത്‌ നാശത്തിലേയ്ക്കും നീങ്ങുന്നു. ബോധം എല്ലായിടത്തും നിലനില്‍ ക്കുന്നതുകൊണ്ട്‌ സൃഷ്ടിപ്രളയങ്ങള്‍ എല്ലായ്പ്പോഴും അനുസ്യൂതം തുടര്‍ന്നുകൊണ്ടേയിരിക്കുന്നു. ഇതിനെയെല്ലാം ഒരുക്കിയൊതുക്കിനടത്തുന്നത്‌ അജ്ഞ്യേയമായ ഏതോ ഒരു നിഗൂഢശക്തിയാണ്‌. രാമാ, എല്ലാം നിലനില്‍ക്കുന്നത്‌ ഒരേയൊരു അനന്താവബോധത്തിലാണ്‌. അതിലാണെല്ലാം ഉയര്‍ന്നുണരുന്നത്‌. അതുമാത്രമേ എല്ലാറ്റിന്റേയും ഉണ്മയായുള്ളു.

