\section{ദിവസം 152}

\slokam{
നാസ്തി ബന്ധോ ന മോക്ഷോസ്തി തന്മയസ്ത്വിവ ലക്ഷ്യതേ\\
ഗ്രസ്തം നിത്യമനിത്യേന മായാമയമഹോ ജഗത് (4/11/63)\\
}

കാലം തുടർന്നു: ഞാൻ ദുർബ്ബലൻ, അസന്തുഷ്ടൻ, മൂഢൻ തുടങ്ങിയ വിവിധങ്ങളായ ധാരണകളാൽ മനസ്സ് പ്രത്യക്ഷലോകത്തിൽ നിമഗ്നമാവുന്നു. എന്നാൽ ജ്ഞാനമുദിക്കുമ്പോൾ ഇവയെല്ലാം മനസ്സിന്റെ വ്യാജസൃഷ്ടികളായിരുന്നുവെന്ന ബോധമുദിക്കുന്നു. ഞാൻ, ഞാൻ മാത്രമെന്നറിയുമ്പോൾ ബോധത്തിൽ പരമശാന്തി വിളയാടുന്നു. മനസ്സ് അസംഖ്യം വിവിധജന്തുക്കൾ നിവസിക്കുന്ന ഒരു വലിയ പാരാവാരം തന്നെ. ഈ സമുദ്രോപരി അനേകം ഓളങ്ങളും അലകളും, പല തരത്തിലും വലുപ്പത്തിലും ഉണ്ടായി മറയുന്നു. ചെറിയ അല സ്വയം ചെറുതെന്നു ചിന്തിക്കുന്നു. വലുതോ സ്വയം അതിന്റെ വലുപ്പം ചിന്തിക്കുന്നു. കാറ്റിൽ തകർന്നടിഞ്ഞ അല, താൻ നശിച്ചുവെന്നും കരുതുന്നു. ചിലവ തണുപ്പെന്നും മറ്റുചിലവ ചൂടെന്നും സ്വയം വിലയിരുത്തുന്നു. എന്നാൽ എല്ലാ അലകളും സമുദ്രജലം മാത്രം. സമുദ്രത്തിലെ അലകൾ സത്യമല്ല എന്നുപറയാം. കാരണം സമുദ്രമില്ലാതെ അലകൾക്ക് അസ്തിത്വം ഇല്ലല്ലോ. അലകൾ ഉള്ളതാണെന്നും പറയാം. ബ്രഹ്മം മാത്രമേ നിലകൊള്ളുന്നതായുള്ളു എന്നുപറയുന്നത് ഇപ്രകാരമാണ്‌... സർവ്വശക്തമായതിനാൽ അതിന്റെ സ്വാഭാവീകമായ ആവിഷ്കാരമാണ്‌ അനന്തമായ വൈവിദ്ധ്യങ്ങളായി പ്രപഞ്ചത്തിൽ പ്രകടമാകുന്നത്. ഒരുവന്റെ ഭാവനയിലല്ലാതെ മറ്റെങ്ങും വൈവിദ്ധ്യം എന്ന ധാരണയ്ക്ക്‌ നിലനിൽപ്പില്ല.

‘ഇതെല്ലാം പരബ്രഹ്മം’ എന്നതാണു സത്യം. മറ്റു ധാരണകളെല്ലാം ഉപേക്ഷിക്കൂ. അലകൾ സമുദ്രത്തിൽനിന്നും വിഭിന്നമല്ല. പ്രപഞ്ചവസ്തുക്കൾ ബ്രഹ്മത്തിൽ നിന്നും വിഭിന്നമല്ല. വിത്തിൽ മുഴുവൻ വന്മരവും സാദ്ധ്യതയായി നില കൊള്ളുന്നതുപോലെ വിശ്വം മുഴുവൻ എക്കാലവും ബ്രഹ്മത്തിൽ നിലകൊള്ളുന്നു. വർണ്ണവൈവിദ്ധ്യതയോടെ കാണുന്ന മഴവില്ല് സൂര്യന്റെ 'സൃഷ്ടി'യാണ്‌.. വെറും ജഢമായചിലന്തിവല ജീവനുള്ള എട്ടുകാലിയിൽനിന്നും പുറത്തുവരുമ്പോലെ അനന്താവബോധത്തിൽനിന്നും ജഢപ്രപഞ്ചം പ്രകടമാവുന്നു. പട്ടുനൂൽപ്പുഴു സ്വയം ചുറ്റിവരിഞ്ഞൊരു പുഴുക്കൂടുണ്ടാക്കുന്നു. അതിൽ ബന്ധിതമാവുന്നു. അനന്താവബോധം വിശ്വത്തെ സങ്കൽപ്പിച്ചുണ്ടാക്കുന്നു. എന്നിട്ടതിൽ ആമഗ്നമാവുന്നു. ആന, അതിനെ തളച്ചിരിക്കുന്ന വിളക്കുകാലിൽ നിന്നും ആയാസരഹിതമായി ചങ്ങലപൊട്ടിച്ചു വരുമ്പോലെ ആത്മാവ് സ്വയം ബന്ധവിമോചിതമാവുന്നു. അത്മാവു മാത്രമേ നിജമായുള്ളു. ഈശ്വരന്‌ ബന്ധനമോ ബന്ധമോചനമോ ഇല്ല. ഈ ബന്ധനം, മോക്ഷം എന്ന ധാരണകൾ എവിടെനിന്നുവന്നുവെന്ന് എനിക്കറിയില്ല.

“ബന്ധനമോ മോക്ഷമോ ഇല്ല. ഉള്ളത് അനന്തമായ പരമപുരുഷൻ മാത്രം. എങ്കിലും ശാശ്വതമായ സത്യത്തെ ക്ഷണികമായ കാഴ്ച്ചകൾ എന്ന മായ, മറയിട്ടുമൂടുന്നു. ഇത് എത്ര വിചിത്രമായ അത്ഭുതം!” മനസ്സ് അനന്താവബോധത്തിൽ അങ്കുരിച്ചപ്പോൾ മുതൽ വൈവിദ്ധ്യം - നാനാത്വം- എന്ന ധാരണയും ഉടലെടുത്തു. ഈ ധാരണകൾ അനന്താവബോധത്തിൽ സ്ഥിതിചെയ്യുന്നു. അതിനാൽത്തന്നെ ഈ പ്രപഞ്ചത്തിൽ പലദേവതകളും അസംഖ്യം ജീവജാലങ്ങളും ഉള്ളതായി കാണുന്നു. ചിലവയ്ക്ക് ദീർഘായുസ്സുണ്ട്, ചിലവയ്ക്ക് ക്ഷണികജീവിതമാണുള്ളത്. ചിലത് ഭീമാകരം, മറ്റു ചിലതിന്‌ സൂക്ഷ്മരൂപം. ചിലർക്ക് സന്തോഷം ചിലർക്ക് ദു:ഖം. ഈ ജീവജാലങ്ങളെല്ലാം അനന്താവബോധത്തിലെ ധാരണകൾ മാത്രം. ചിലർ സ്വയം അജ്ഞാനിയെന്നും, ബന്ധനസ്ഥനെന്നും കരുതുന്നു. മറ്റുള്ളവർ ജ്ഞാനികളും ബന്ധവിമുക്തരുമാണെന്നും.

