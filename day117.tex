\newpage
\section{ദിവസം 117}

\slokam{
കര്‍മ നാശോ  മനോനാശോ മനോനാശോ ഹ്യകര്‍മതാ\\
മുക്തസ്യൈവ ഭവത്യേവ നാമുക്തസ്യ കദാചന (3/95/36)\\
}

വസിഷ്ഠന്‍ തുടര്‍ന്നു: കര്‍മ്മവും അതിന്റെ കര്‍ത്താവും പരമപുരുഷനില്‍ സംജാതമായത്‌ പൂവും അതിന്റെ സുഗന്ധവുമെന്നപോലേ ഒരേ സമയത്താണ്‌.. എങ്കിലും ഒന്നറിയുക, അജ്ഞാനിയുടെ ദൃഷ്ടിയില്‍ മാത്രമേ ജീവന്റെ സൃഷ്ടി എന്നത്‌ ഉണ്മയായുള്ളു. നീലാകാശത്തിന്റെ നീലിമ സത്യമെന്നു കരുതുന്നത്‌ അജ്ഞാനികള്‍ മാത്രമാണല്ലോ. ജ്ഞാനികളെ സംബന്ധിച്ചിടത്തോളം ജീവന്‍ ബ്രഹ്മത്തില്‍ നിന്നുണ്ടായി എന്നു പറയുന്നതും ജീവന്‍ ബ്രഹ്മത്തില്‍നിന്നല്ല ഉണ്ടായത്‌ എന്നു പറയുന്നതും ഒരേപോലെ അസംബന്ധമാണ്‌. പഠനസൌകര്യത്തിനുവേണ്ടി ഈ ദ്വന്ദതാ സങ്കല്‍പ്പം താല്‍ക്കാലികമായെങ്കിലും എടുത്തില്ലെങ്കില്‍ ഇക്കാര്യം പറഞ്ഞുമനസ്സിലാക്കല്‍ അസാദ്ധ്യമാണ്‌..

ജീവന്‍ ബ്രഹ്മത്തില്‍നിന്നും ഉത്ഭവിച്ചു എന്നുപറഞ്ഞു. കാര്യം കാരണത്തില്‍നിന്നു വിഭിന്നമല്ല എന്ന തത്വമനുസരിച്ച്‌ ജീവന്‍ ബ്രഹ്മത്തില്‍ നിന്നും വിഭിന്നമല്ല എന്നും ഗുരു നമ്മെ ഉപദേശിക്കുന്നു. ഇക്കാണപ്പെടുന്നതെല്ലാം പൂവില്‍ നിന്നു സുഗന്ധം എന്നപോലെ ബ്രഹ്മത്തില്‍നിന്നു സംജാതമത്രെ. ഒരു ഋതു മറ്റൊന്നിലേയ്ക്ക്‌ കടക്കുമ്പോലെ ജീവാത്മാക്കള്‍ തിരികെ ബ്രഹ്മത്തില്‍ ലയിക്കുന്നു. പ്രപഞ്ചത്തിലെ ഓരോ ജീവിയും പ്രകടമായി മൂര്‍ത്തീകരിക്കുന്നതിനോടൊപ്പം അതാതിന്റെ സ്വഭാവഗുണങ്ങളും സംജാതമാകുന്നു. സ്വരൂപത്തെപ്പറ്റിയുള്ള അജ്ഞാനം കൊണ്ടാണ്‌ പുനര്‍ജന്മങ്ങളിലേയ്ക്കു നയിക്കുന്ന കര്‍മ്മങ്ങളിലും പ്രവര്‍ത്തനങ്ങളിലും ആ ജീവനുകള്‍ ആമഗ്നരാകുന്നത്‌. .

രാമന്‍ പറഞ്ഞു: മഹാത്മന്‍, വേദഗ്രന്ഥങ്ങള്‍ തീര്‍ച്ചയായും മാമുനിമാരുടെ നിറഭേദമൊന്നുമില്ലാത്ത മനസ്സില്‍ നിന്നും നിര്‍ഗ്ഗളിച്ച പ്രസ്താവനകളാണല്ലോ. നിര്‍മ്മലഹൃദയരും ഭേദരഹിതമായ ദര്‍ശനത്തിന്റെ വക്താക്കളുമാണ്‌ ഋഷികള്‍ .  അജ്ഞാനിയായ ഒരുവന്‌ ഈദൃശ വേദങ്ങളുടെ സഹായത്താലും പ്രബുദ്ധനായ ഒരു ജ്ഞാനിയുടെ സ്വഭാവസവിശേഷതകളെന്തെന്ന ഉദാഹരണത്തിലൂടെയും മാത്രമേ പരമസത്യപ്രകാശം എന്തെന്ന്, എങ്ങിനെയെന്ന്, പ്രതീക്ഷിക്കാന്‍പോലുമാകൂ.

മഹര്‍ഷേ, ഈ ലോകത്ത്‌ വിത്ത്‌ മരത്തില്‍നിന്നുണ്ടാവുന്നതായി നാം കാണുന്നു. അതുപോലെ വിത്തില്‍നിന്നു മരവും. അപ്പോള്‍ പൂര്‍വ്വാര്‍ജ്ജിത കര്‍മ്മങ്ങളുടെ അഭാവത്തില്‍ നാനാവിധ ജീവജാലങ്ങള്‍ ആദ്യമായുണ്ടായത്‌ പരബ്രഹ്മത്തില്‍നിന്നാണെന്നു പറയുന്നത്‌ ഉചിതമാകുമോ?

വസിഷ്ഠന്‍ പറഞ്ഞു: രാമാ, സൂക്ഷ്മമായി നോക്കിയാല്‍ നിനക്കറിയാന്‍ കഴിയും കര്‍മ്മത്തില്‍ മനസ്സിന്റെ ഇടപെടല്‍ ഉള്ളപ്പോള്‍ മാത്രമേ കര്‍മ്മത്തിന്റെ തുടര്‍ച്ചയായി കര്‍മ്മഫലങ്ങള്‍ ഉണ്ടാവുകയുള്ളു എന്ന്‍ . അതുകൊണ്ട്‌ മനസ്സാണ്‌ കര്‍മ്മത്തിനു വിത്ത്‌.. വിശ്വമനസ്സ്‌ പരബ്രഹ്മത്തില്‍ പ്രകടമായ അതേ ക്ഷണത്തില്‍ നാനാജീവ-നിര്‍ജ്ജീവജാലങ്ങള്‍ അവയുടെ സ്വഭാവസവിശേഷതകളോടുകൂടി ഉണ്ടായി. അവയില്‍ ശരീരങ്ങളോടുകൂടിയവ ജീവാത്മാക്കളായി. മനസ്സും കര്‍മ്മവും തമ്മില്‍ ഭേദമില്ല. പുറമേ പ്രാവര്‍ത്തികമാവുന്നതിനുമുന്‍പ്‌ കര്‍മ്മങ്ങള്‍ മനസ്സിലുയരുന്നു. കര്‍മ്മം എന്നത്‌ ബോധമണ്ഡലത്തിലെ ഊര്‍ജ്ജത്തിന്റെ ചലനമാണ്‌.. അവ നിയതമായ ഫലപ്രാപ്തിക്കു കാരണമാവുകയും ചെയ്യുന്നു.

"അത്തരം കര്‍മ്മങ്ങള്‍ അവസാനിക്കുമ്പോള്‍ മനസ്സും അവസാനിക്കുന്നു. മനസ്സൊടുങ്ങുമ്പോള്‍ കര്‍മ്മങ്ങളില്ല. മുക്തിപദം പ്രാപിച്ച മഹര്‍ഷിമാര്‍ക്കു മാത്രം ബാധകമാണിത്‌.. മറ്റുള്ളവര്‍ക്കല്ല."
