\section{ദിവസം 271}

\slokam{
കുരംഗാലിപതംഗേഭമീനാസ്ത്വേകൈകശോ ഹതാ:\\
സര്‍വ്വൈര്‍യുക്തൈരനര്‍ത്ഥേസ്തു വ്യാപ്തസ്യാജ്ഞ കുത: സുഖം (5/52/21)\\
}

വസിഷ്ഠന്‍ തുടര്‍ന്നു: ഉദ്ദാലകന്‍ രമണീയമായ ആ ഗുഹയില്‍ക്കയറി ധ്യാനനിരതനായി ഇരിപ്പുറപ്പിച്ചു. ചിന്തകളൊഴിഞ്ഞ അചഞ്ചലമായ മനസ്സുണ്ടാവണമെന്ന ദൃഢ നിശ്ചയത്തോടെ അദ്ദേഹം തന്റെ ശ്രദ്ധയെ മനസ്സിലെ ലീനവാസനകളിലേയ്ക്ക് സംക്രമിപ്പിച്ചു. അദ്ദേഹം ആലോചിച്ചു: മനസ്സേ ഈ പ്രത്യക്ഷലോകത്തില്‍ നിനക്കെന്താണ് ജോലി? ജ്ഞാനികള്‍ സുഖാനുഭാവങ്ങളുമായി സമ്പര്‍ക്കം പുലര്‍ത്തുന്നില്ല. കാരണം അവ പിന്നീട് ദു:ഖാനുഭവങ്ങളാവുന്നതായാണ് എല്ലാവരുടെയും അനുഭവം. അകമേയുള്ള പരമാനന്ദത്തെ അവഗണിച്ചുകൊണ്ട് ഇന്ദ്രിയസുഖങ്ങളുടെ പിന്നാലെ പോകുന്നവര്‍ രമണീയമായൊരു നന്ദനോദ്യാനം വേണ്ടെന്നുവെച്ച് വിഷച്ചെടികള്‍ വളരുന്ന കുറ്റിക്കാട്ടില്‍ ആശ്രയം തേടുന്നവരത്രേ.  

നിനക്കും  എങ്ങോട്ട് വേണമെങ്കില്‍ പോവാം. എന്നാല്‍ മനസ്സിനെ പരമപ്രശാന്തമാക്കിയാലല്ലാതെ പരമാനന്ദം അറിയുകയില്ല. അതുകൊണ്ട് എല്ലാ ആശകളും പ്രത്യാശകളും അവസാനിപ്പിച്ചാലും. കാരണം ഇപ്പോള്‍ക്കാണുന്ന, അത്ഭുതകരങ്ങളായി കാണപ്പെടുന്ന, ഭാവാഭാവപ്രകൃതി ദൃശ്യങ്ങള്‍ പോലും നിനക്ക് സുഖം നല്‍കുകയില്ല. മാധുര്യമേറിയ മണിനാദവും സംഗീതവുമാണ് മാന്‍പേടയെ കുടുക്കുന്നത്. പെണ്ണാനയെക്കൊണ്ടാണ് (സ്പര്‍ശനസുഖാസക്തിയാല്‍ ) കൊമ്പനാനയെ ചതിക്കുഴിയില്‍ വീഴ്ത്തുന്നത്. സ്വാദിഷ്ടമായ ഇര വിഴുങ്ങാനാര്‍ത്തി പിടിക്കുമ്പോഴാണ് മത്സ്യം ചൂണ്ടയില്‍ കുരുങ്ങുന്നത്. ദീപപ്രഭയില്‍ തലകറങ്ങി ആകര്‍ഷിക്കപ്പെട്ടാണ് ഈയാംപാറ്റകള്‍ വിളക്കിലെ ദീപനാളത്തിലേയ്ക്ക് പറന്നുചെന്ന് മരണം വരിക്കുന്നത്. മധുവുണ്ണാന്‍ വരുന്ന തേനീച്ച മണംപിടിച്ചാണ് പൂവിനുള്ളില്‍ കയറുന്നത്. എന്നാല്‍ അതേപൂവുതന്നെ കൂമ്പിയടയുമ്പോള്‍ അതൊരു മരണക്കെണിയാകുന്നു.      

“അല്ലയോ മൂഢമനസ്സേ, ഇപ്പറഞ്ഞ ജീവികളെല്ലാം നശിക്കുന്നത് കേവലം ഒരേയൊരിന്ദ്രിയത്തിന്റെ ആകര്‍ഷണവലയത്തില്‍പ്പെട്ടാണ്. (മാനുകള്‍  കേള്‍വിസുഖത്താല്‍ , തേനീച്ചകള്‍ മണത്താല്‍ , ഈയാംപാറ്റകള്‍  ദൃഷ്ടിസുഖത്താല്‍ , ആനകള്‍ സ്പര്‍ശനസുഖത്താല്‍ , മത്സ്യങ്ങള്‍ രസനാസുഖത്താല്‍ . എന്നാല്‍ നീ അഞ്ച് ഇന്ദ്രിയങ്ങളുടേയും പ്രലോഭനങ്ങള്‍ക്കടിമയാണ്. അപ്പോള്‍പ്പിന്നെ നിനക്കെങ്ങിനെ ശാശ്വതസുഖം ലഭിക്കാനാണ്?” പട്ടുനൂല്‍പ്പുഴു സ്വയമൊരു കൂടുണ്ടാക്കി അതില്‍പ്പെട്ടിരിക്കുമ്പോലെ നീ ധാരണകളുടേയും സങ്കല്‍പ്പങ്ങളുടേയും വല സ്വയം നെയ്ത് അതില്‍ക്കുടുങ്ങിയിരിക്കുന്നു. 

എന്നാല്‍ അതില്‍ നിന്നും രക്ഷപ്പെടാമെങ്കില്‍ നിനക്ക് പരമശുദ്ധിയും ജനന മരണഭയങ്ങളില്‍ നിന്ന് മോചനവും പ്രാപ്തമാവും. അങ്ങിനെ സര്‍വ്വസമത്വ ഭാവവും ഉന്നതവിജയവും നിന്റെ വരുതിയിലാവും. മറിച്ച് ഈ മാറിക്കൊണ്ടിരിക്കുന്ന ലോകത്തിന്റെ സുഖസാദ്ധ്യതകളിലാണ് നീ കടിച്ചുതൂങ്ങിക്കിടക്കുന്നതെങ്കില്‍ നിനക്ക് കൊടിയദു:ഖവും അതിലൂടെ നാശവുമുറപ്പാണ്.
കഷ്ടം! ഞാനെന്തിനാണ് മനസ്സേ നിന്നോടിതൊക്കെ പറഞ്ഞു തരുന്നത്! മനസ്സെന്നൊരു വസ്തു തന്നെയില്ലല്ലോ! മനസ്സ് അജ്ഞാനത്തിന്റെ സന്തതിയാണ്. അജ്ഞാനമില്ലാതാവുമ്പോള്‍ മനസ്സും ഇല്ലാതാവുന്നു. അതിനാല്‍ നീ ഇല്ലാതായിക്കൊണ്ടിരിക്കുന്നു എന്നറിഞ്ഞാലും. അപചയത്തിലാണ്ട് പോയ്‌ക്കൊണ്ടിരിക്കുന്ന ഒരുവനോടുപദേശം നടത്തുന്നത് മൂഢത്വമാണ്!. ദിനംതോറും അവശനായിക്കൊണ്ടിരിക്കുന്ന, ഇല്ലാതായിക്കൊണ്ടിരിക്കുന്ന  നിന്നെ ഞാനിതാ സംത്യജിക്കുന്നു. ഉപേക്ഷിക്കാന്‍പോവുന്നവനെ വിവേകശാലികള്‍ വീണ്ടും ഉപദേശിക്കുകയില്ലല്ലോ.

മനസ്സേ, ഞാന്‍ അഹംകാരരഹിതമായ, അവിച്ഛിന്നമായ, അനന്താവബോധമാണ്. അഹത്തിനു കാരണമായ നീയുമായി എനിക്കിനി യാതൊരു ബന്ധവുമില്ല.