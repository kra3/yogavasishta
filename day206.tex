\section{ദിവസം 206}

\slokam{
അരജ്ജുരേവ ബദ്ധോഽഹമപങ്കോഽസ്മി കളങ്കിത:\\
പതിതോസ്മ്യുപരിസ്ഥോപി ഹാ മമാത്മൻഹതാ സ്ഥിതി: (5/9/16) \\
}

വസിഷ്ഠൻ തുടർന്നു: മാമുനിമാരുടെ വാക്കുകൾ കേട്ട് ജനകൻ ചിന്താകുലനായി. അദ്ദേഹം തന്റെ ഉല്ലാസനടത്തം മതിയാക്കി കൊട്ടാരത്തിലേയ്ക്കു തിരിച്ചു പോയി. തന്റെ സേവകരെയെല്ലാം പറഞ്ഞുവിട്ട് പള്ളിയറയിൽ ഏകാന്തനായിരുന്ന് ആശങ്കയോടെ സ്വയം ഇങ്ങിനെ  പറഞ്ഞു: ഇഹലോകമെന്ന ദുരിതത്തിൽ ഞാനൊരു കല്ലുപോലെ ഉരുളുകയാണ്‌.. ഈ ജീവിതത്തിൽ ആയുസ്സെത്രനാളുണ്ട്? എന്നിട്ടും എനിക്കതിനോടൊരു മമതയുണ്ടായിരിക്കുന്നു! ഛീ! എന്തൊരു കഷ്ടമാണീ മനസ്സിന്റെ കാര്യം!

ജീവിതകാലം മുഴുവൻ ലോകത്തിന്റെ പരമാധികാരിയായിരുന്നിട്ടും എന്തു കാര്യം? വ്യര്‍ത്ഥമെന്നറിഞ്ഞിട്ടും  ഒരു മൂഢനെപ്പോലെ ലോകം എനിക്കനിവാര്യമാണെന്നു ഞാൻ കരുതുന്നു. എന്റെ ആയുസ്സ്  തുലോം ചെറിയൊരു കാലയളവു മാത്രം. അനശ്വരത എന്നത് എന്റെ ആയുസ്സിനു മുൻപും പിൻപുമായങ്ങിനെ നീണ്ടു പരന്നു കിടക്കുന്നു. അതിനെ  എങ്ങിനെയിപ്പോൾ ഞാൻ പരിപോഷിപ്പിക്കും? ആരാണ്‌ ലോകമെന്ന ഈ മായാപ്രപഞ്ചത്തെ വിക്ഷേപിച്ച് പ്രത്യക്ഷമാക്കിയത്? ലോകമെന്ന കാഴ്ച്ചയിൽ ഞാനിത്ര ഭ്രമിക്കാൻ കാരണമെന്താണ്‌? അടുത്തും അകലത്തും എല്ലാമുള്ളത് എന്റെ മനസ്സിന്റെയുള്ളിൽത്തന്നെയാണെന്നറിഞ്ഞ് ബാഹ്യവസ്തുക്കളിലുള്ള എല്ലാ ആശങ്കകളും ഞാനുപേക്ഷിക്കും. ഇഹലോകത്തിലെ എല്ലാ ധൃതിപിടിച്ച പ്രവർത്തനങ്ങളും അന്തമില്ലാത്ത ദു:ഖത്തിനു കാരണമാകുന്നു എന്നറിയുമ്പോൾ സന്തോഷത്തിനായി ഞാനെന്തിനെ ആശ്രയിക്കുവാനാണ്‌?

ദിനംതോറും, മാസംതോറും, വർഷംതോറും, നിമിഷങ്ങൾതോറും കാണുന്ന സന്തോഷങ്ങൾ സന്താപങ്ങളേയും കൂട്ടിക്കൊണ്ടാണു വരുന്നത്. എന്നാൽ ദു:ഖങ്ങളോ അനവരതം വന്നുകൊണ്ടേയിരിക്കുന്നു. ഇവിടെ കാണുന്നതും അനുഭവിക്കുന്നതുമെല്ലാം മാറ്റങ്ങൾക്കും നാശത്തിനും വിധേയമാണ്‌.. വിവേകശാലിക്ക് അവലംബമായി ഇഹലോകത്തിൽ യാതൊന്നുമില്ല. ഇന്ന് പ്രശസ്തിയും സ്ഥാനമാനങ്ങളും കിട്ടി പുകഴ്ത്തപ്പെട്ടവർ നാളെ ചവിട്ടിത്താഴ്ത്തപ്പെടുന്നു. മൂഢമനസ്സേ, ഈ ലോകത്തിൽ എന്തിനെയാണു നാം വിശ്വസിക്കുക?

“കഷ്ടം! ഞാൻ കയറില്ലാതെയുള്ള ഒരു ബന്ധനത്തിലാണ്‌.. അശുദ്ധനല്ലെങ്കിലും ഞാൻ കളങ്കപ്പെട്ടിരിക്കുന്നു! ഉയർന്നൊരു സ്ഥാനത്തിലാണെങ്കിലും ഞാൻ പതിതൻ. ഞാൻ എന്നത് തന്നെ എന്തൊരു സമസ്യയാണ് !” എപ്പോഴും പ്രോജ്ജ്വലിക്കുന്ന സൂര്യനെ ഒരു തുണ്ട് മേഘം മറയ്ക്കുന്നു. ഈ മായീകവിഭ്രമം എന്നെ വലയംചെയ്തിരിക്കുന്നു. ആരാണീ ബന്ധുക്കളും സുഹൃത്തുക്കളും? എന്താണീ സുഖം? ഇരുട്ടത്ത് ഭൂതപിശാചുക്കളെക്കണ്ടു പേടിക്കുന്ന ബാലനെപ്പോലെ ഈ വിചിത്രരായ ബന്ധുക്കളെന്നിൽ ഭീതിയുളവാക്കുന്നു. അവരാണല്ലോ എന്നെ വാർദ്ധക്യവുമായും മരണവുമായും ബന്ധിപ്പിക്കുന്നത്? ഇതറിഞ്ഞിട്ടും ഞാനവരെ വിടാതെ ഒട്ടിപ്പിടിച്ചിരിക്കുന്നു. ഈ ബന്ധുക്കൾ ജീവിച്ചാലും നശിച്ചാലും എനിക്കെന്ത്?

ഈ ലോകത്ത് മഹാത്മാക്കൾ എത്രയോപേർ ജനിച്ചുമരിച്ചു? മഹത്സംഭവങ്ങളും എത്രയുണ്ടായി? അവയെല്ലാം നമ്മില്‍  ഒരോർമ്മ മാത്രം അവശേഷിപ്പിക്കുന്നു. എന്താണു നമുക്കൊരവലംബം? ദേവതമാരും ത്രിമൂർത്തികൾ പോലും കോടിക്കണക്കിനു വന്നുപോയിരിക്കുന്നു. എന്താണീ പ്രപഞ്ചത്തിൽ ശാശ്വതമായുള്ളത്?  പ്രത്യക്ഷലോകമെന്ന ഈ പേടിസ്വപ്നത്തിൽ പ്രത്യാശയെന്ന ഒരു കയർ മാത്രമാണീ ബന്ധനത്തിനെല്ലാം കാരണം. ഛെ! എത്ര നികൃഷ്ടമാണീ അവസ്ഥ!

