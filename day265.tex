\section{ദിവസം 265}

\slokam{
തഥാഹി ബഹവ: സ്വപ്നമേകം പശ്യന്തി മാനവാ:\\
സ്വാപഭ്രമദമൈരേയമദമന്ഥരചിത്തവത് (5/49/11)\\
}

വസിഷ്ഠന്‍ തുടര്‍ന്നു: തന്റെ മായാദര്‍ശനത്തെപ്പറ്റി ഉറപ്പുവരുത്താനായി ഗാധി ഭൂതമണ്ഡലത്തിലും കീരരാജ്യത്തും മറ്റും വീണ്ടും യാത്രപോയി. വീണ്ടും അന്നാട്ടുകാരില്‍ നിന്നും ആദ്യം കേട്ടപോലെയുള്ള കഥകള്‍ അങ്ങിനെതന്നെ വീണ്ടും കേട്ടു. വീണ്ടും അദ്ദേഹം വിഷ്ണുപൂജ ചെയ്തു. വിഷ്ണുഭഗവാന്‍ മുന്നില്‍ പ്രത്യക്ഷപ്പെട്ടു. ഗാധി ഭഗവാനോട് ചോദിച്ചു.: ഭഗവാനേ, കഴിഞ്ഞ ആറുമാസക്കാലം ഞാന്‍ മായാനുഭവത്തില്‍ അറിഞ്ഞ രണ്ടിടങ്ങളിലും പോയി അവിടുത്തുകാരോട് ചോദിച്ചതില്‍നിന്നും അവരാ കഥകളൊക്കെ സത്യമായി വിശ്വസിക്കുന്നു എന്നാണു മനസ്സിലാക്കിയത്. ഇതെന്നെ ചിന്താക്കുഴപ്പത്തിലാക്കുന്നു. ദയവായി ഇതിന്റെ പൊരുളെനിക്ക് പറഞ്ഞു തന്നാലും.

ഭഗവാന്‍ പറഞ്ഞു: ഗാധീ ഈ സംഭവങ്ങളെല്ലാം നിനക്ക് നേരിട്ട് ബന്ധമുള്ള സ്ഥലത്തല്ല നടന്നതെങ്കിലും അവ നിന്റെ മനസ്സിലെ പ്രതിഫലനങ്ങളായാണ് നീയറിയുന്നത്. നീയും അവയും തമ്മിലുള്ള ബന്ധം ‘കാക്കയും പനമ്പഴവും’ എന്ന പോലെ തികച്ചും ആകസ്മികം എന്നേ പറയാവൂ. അതിനാലാണ് നിന്റെയടുക്കല്‍ , നീ സത്യമെന്ന് വിശ്വസിക്കുന്ന കഥ അങ്ങിനെതന്നെ അവര്‍ പറഞ്ഞുതരുന്നത്. അത്തരം യാദൃശ്ചികത അപൂര്‍വ്വമൊന്നുമല്ല. ചിലപ്പോള്‍ ഒരേ ഭ്രമദൃശ്യം തന്നെ പലര്‍ക്കും ഒരേപോലെ അനുഭവവേദ്യമാവാറുണ്ട്.

“ചിലപ്പോള്‍ കുറെയേറെപ്പേര്‍ക്ക് ഒരേ സ്വപ്നമുണ്ടാവാറുണ്ട്. ഒരേ ഭ്രമാത്മക ദര്‍ശനം! മദിര കുടിച്ചു മദോന്മാത്തരായ കുറെപ്പേര്‍ക്ക് ഒന്നിച്ച് ഒരേസമയം അവര്‍ക്കുചുറ്റും ഭൂമി വട്ടംകറങ്ങുന്നതുപോലെ അനുഭവപ്പെടാമല്ലോ.” അനേകം കുട്ടികള്‍ ഒരേ കളികളിലേര്‍പ്പെടുന്നു എന്ന്പോലെയാണത്. സമയത്തെപ്പറ്റിയും ആളുകള്‍ക്ക് ഇത്തരം ചിന്താക്കുഴപ്പമുണ്ടാകാറുണ്ട്. സമയമെന്നത് മനസ്സിലെ ഒരു ധാരണ മാത്രമാണല്ലോ. പരസ്പരമുള്ള കാരണ-ബന്ധ പ്രതിഭാസവുമായി സമയമെന്ന ധാരണ കെട്ടുപിണഞ്ഞു കിടക്കുന്നു.

ഭഗവാന്‍ അവിടെനിന്നും അപ്രത്യക്ഷനായിട്ടും ഗാധി തന്റെ ധ്യാനം തുടര്‍ന്നു. വീണ്ടുമേറെക്കാലം കഴിഞ്ഞപ്പോള്‍ വിഷ്ണുഭഗവാന്‍ പ്രത്യക്ഷപ്പെട്ടു. ഗാധി പ്രാര്‍ത്ഥിച്ചു: ഭഗവാനെ, അങ്ങയുടെ മായാശക്തിയാല്‍ ഞാന്‍ കഷ്ടപ്പെടുന്നു. ഞാനാകെ ചിന്താക്കുഴപ്പത്തിലാണ്. ഇതില്‍നിന്ന് കരകേറാന്‍ ഉചിതമായ മാര്‍ഗ്ഗമെന്തെന്നരുളിയാലും.
   
ഭഗവാന്‍ പറഞ്ഞു: നീ ഭൂതമണ്ഡലത്തിലും കേരരാജ്യത്തും കണ്ടതെല്ലാം ഒരു പക്ഷെ സത്യമായിരിക്കാം. കടഞ്ചന്‍ എന്ന പേരിലൊരു കാട്ടുജാതിക്കാരന്‍ അവിടെ വസിച്ചിരുന്നിരിക്കാം. അയാളുടെ ബന്ധുക്കള്‍ മരിച്ചശേഷം അയാള്‍ കീരരാജ്യം ഭരിച്ചുവെന്നതും അങ്ങിനെതന്നെയിരിക്കട്ടെ. എന്നാലിതെല്ലാം പ്രതിഫലിച്ചത് നിന്റെ ബോധതലത്തിലാണ്.  
ചിലപ്പോള്‍ മനസ്സ്, താനെന്താണ്‌ ശരിക്കും അനുഭവിച്ചതെന്നു മറന്നുപോകുന്നു. അതുപോലെ ചിലപ്പോള്‍ ഒരിക്കലും കണ്ടിട്ടോ അനുഭവിച്ചിട്ടോ ഇല്ലാത്തതിനെ അനുഭവിച്ചുവെന്നു സങ്കല്‍പ്പിച്ച് ചിന്തിച്ചുറപ്പിക്കുകയും ചെയ്യും. സ്വപ്നത്തില്‍ പലവിധ ദൃശ്യങ്ങള്‍ കാണുന്നതുപോലെ ജാഗ്രദവസ്ഥയിലും ഭ്രമകല്‍പ്പനകള്‍ അനുഭവവേദ്യമാവാം. കടഞ്ചന്‍ ജീവിച്ചിരുന്നത് ഏറെക്കാലം മുന്‍പായിരുന്നുവെങ്കിലും അതിപ്പോള്‍ നിന്റെയുള്ളില്‍ , ബോധതലത്തില്‍ പ്രകടമായി എന്ന് മാത്രം.

‘ഇത് ഞാന്‍’ എന്ന ഒരു തോന്നല്‍ ആത്മജ്ഞാനം ലഭിച്ചവന്റെയുള്ളില്‍ അങ്കുരിക്കയില്ല. അതുണ്ടാവുന്നത് അജ്ഞാനിയുടെ മനസ്സിലാണ്. എന്നാല്‍ ‘ഞാന്‍ എല്ലാമെല്ലാമാണ്’ എന്നറിയുന്നവന്* ദു:ഖത്തില്‍ ഉഴറേണ്ടതായി വരികയില്ല. കാരണം ദു:ഖദായികളായ ഭൌതീകവസ്തുക്കളെ അയാള്‍ സമാശ്രയിക്കുന്നില്ല. അയാള്‍ സന്തോഷസന്താപങ്ങളാല്‍ ചാഞ്ചാടപ്പെടുന്നില്ല. നിനക്ക് പൂര്‍ണ്ണ പ്രബുദ്ധതയിനിയും കൈവന്നിട്ടില്ലാത്തതിനാല്‍ നിന്റെ മനസ്സ് വിഷയധാരണകളിലും ഭ്രമദൃശ്യങ്ങളിലും ഉള്ള ബന്ധം തുടരുകയാണ്. ഈ മായ എല്ലായിടത്തേയ്ക്കും, എല്ലാദിശയിലേയ്ക്കും വ്യാപരിച്ചിരിക്കുന്നു. എന്നാല്‍ ഏകാത്മകമായ ആ മദ്ധ്യബിന്ദുവില്‍ ദൃഢമായി ദൃഷ്ടിയുറപ്പിച്ചവന്‍ ഭ്രമകല്‍പ്പനകളില്‍ നിന്നും മുക്തനാണ്. എഴുന്നേല്‍ക്കൂ. ഇനിയും ഒരു പത്തുകൊല്ലം ഗാഢമായി തപസ്സുചെയ്താലും.

ഗാധി പിന്നീട് തീവ്രസാധനയില്‍ മുഴുകി ഒടുവില്‍ ആത്മസാക്ഷാത്കാരം നേടി. എന്നിട്ട് ജീവന്മുക്തനായ ഒരു മഹാമുനിയായി, ദു:ഖഭയാദികളില്ലാതെ  ശിഷ്ടകാലം ജീവിച്ചു. 
   
\textit{തോന്നുന്നതാകില്‍ അഖിലം ഞാനിതെന്നവഴി തോന്നേണമേ വരദ നാരായണായ നമ}

