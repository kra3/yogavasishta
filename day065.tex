 
\section{ദിവസം 065}

\slokam{
പ്രജോപദ്രവനിഷ്ടസ്യ രാജ്ഞോരാജ്ഞോഥ വാ പ്രഭോ:\\
അർത്ഥേന യേ മൃതാ യുദ്ധേ തേ വൈ നിരയഗാമിന: (3/31/30)\\
}

വസിഷ്ഠന്‍ തുടര്‍ന്നു: ഇതെല്ലാം കണ്ടുകഴിഞ്ഞ്‌ ലീല അന്ത:പ്പുരത്തിനുള്ളില്‍ തന്റെ ഭര്‍ത്താവായ രാജാവിന്റെ മൃതദേഹം ഒരു പുഷ്പക്കൂമ്പാരത്തിനടിയില്‍ കിടത്തിയിരിക്കുന്നതു കണ്ടു. അതവളില്‍ തന്റെ ഭര്‍ത്താവിന്റെ മറുജന്മം കാണാനുള്ള അഭിവാഞ്ഛയുളവാക്കി. ക്ഷണത്തില്‍ അവള്‍ വിശ്വസൃഷ്ടിയുടെ കൊടുമുടിയില്‍നിന്നു മടങ്ങി തന്റെ ഭര്‍ത്താവു ഭരിച്ചിരുന്ന രാജ്യത്തിലേയ്ക്കു കുതിച്ചു. അതേസമയം സിന്ധു പ്രവിശ്യയിലെ രാജാവ്‌ ലീലയുടെ ഭര്‍ത്താവിന്റെ രാജ്യത്തെ ഉപരോധിച്ച്‌ കീഴടക്കാനുള്ള ശ്രമത്തിലായിരുന്നു. ഈ വനിതകള്‍ അകാശമാര്‍ഗ്ഗേ സഞ്ചരിക്കുമ്പോള്‍ യുദ്ധക്കളത്തിനുമുകളിലായി അനേകം ആകാശചാരികള്‍ , യക്ഷകിന്നരന്മാര്‍ , യുദ്ധം കാണാനും വീരയോദ്ധാക്കളുടെ പ്രകടനം ദര്‍ശിക്കാനുമായി നിരന്നു നിന്നിരുന്നു.

രാമന്‍ ചോദിച്ചു: മഹാത്മന്‍, യുദ്ധത്തില്‍ ആരാണ്‌ വീരന്‍? ആരാണ്‌ ക്രൂരന്‍ അല്ലെങ്കില്‍ യുദ്ധക്കുറ്റവാളി?

വസിഷ്ഠന്‍ പറഞ്ഞു: രാമ: ശാസ്ത്രസംഗതമായ രീതിയില്‍ കളങ്കമില്ലാത്തവനും ധര്‍മ്മിഷ്ഠനുമായ ഒരു രാജാവിനുവേണ്ടി യുദ്ധം ചെയ്യുന്നവന്‍ വിജയിയായാലും യുദ്ധത്തില്‍ മരണപ്പെട്ടാലും വീരയോദ്ധാവാണ്‌. അധര്‍മ്മിയായ ഒരേകാധിപതിക്കുവേണ്ടി ആളുകളെ ഉപദ്രവിക്കുകയും അവരുടെ ശരീരം വികലമാക്കി അംഗഭംഗപ്പെടുത്തുകയും ചെയ്യുന്നവനാണ്‌ യുദ്ധക്കുറ്റവാളി. അവന്‍ നരകത്തില്‍പ്പോവുന്നു. ആരൊരുവന്‍ പശുക്കളേയും സാധുക്കളേയും പരിരക്ഷിക്കുന്നുവോ ആരുടെയടുക്കല്‍ സദ്ജനങ്ങള്‍ക്ക്‌ അഭയം ലഭിക്കുമോ അവന്‍ സ്വര്‍ഗ്ഗത്തിനുപോലും ഭൂഷണമത്രേ. മറിച്ച്‌ "ആരൊരുവന്‍ ജനദ്രോഹിയായ, ജനങ്ങളുടെ ദു:ഖത്തില്‍ സന്തോഷിക്കുന്ന ഒരു രാജാവിനുവേണ്ടിയോ ജന്മിക്കുവേണ്ടിയോ യുദ്ധംചെയ്യുന്നുവോ അവന്‍ നരകത്തിലേയ്ക്കു പോവുന്നു." വീരചരമം പ്രാപിച്ചവനുള്ളതാണ്‌ സ്വര്‍ഗ്ഗം. അധാര്‍മ്മികമായി യുദ്ധത്തിലേര്‍പ്പെട്ടവന്‍ അതില്‍ മരിച്ചാലും അവന്‌ സ്വര്‍ഗ്ഗം അപ്രാപ്യം.

രാമാ, ആകാശത്തുനിന്നുകൊണ്ടുതന്നെ രണ്ടു പ്രബല സൈന്യങ്ങള്‍ യുദ്ധോത്സുകരായി അടുത്തടുത്തുവരുന്നത്‌ ലീല കണ്ടു. (ഇവിടെ യുദ്ധത്തെപറ്റി വലിയൊരു വിവരണമുണ്ട്‌. അതിലെ നാശങ്ങളുടെ വിശദവും ബീഭല്‍സവുമായ ചിത്രം ഗ്രന്ഥത്തിലുണ്ട്‌). വൈകുന്നേരമായപ്പോള്‍  രാജാവ്‌ (ലീലയുടെ ഭർത്താവ്) സഭ വിളിച്ചുകൂട്ടി അന്നത്തെ സംഭവങ്ങള്‍ വിലയിരുത്തി. എന്നിട്ട്‌ പള്ളിയറയിലേയ്ക്ക്‌ വിശ്രമത്തിനായി പോയി. വനിതകള്‍ രണ്ടാളും ഒരു ചെറിയ കാറ്റുപോലെ അയത്നലളിതമായി ആകാശത്തുനിന്നും പറന്ന് രാജാവുറങ്ങുന്ന പള്ളിയറയിലെത്തി.
