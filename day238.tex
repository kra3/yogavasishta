\section{ദിവസം 238}

\slokam{
അവിഷ്ണു: പൂജയന്‍വിഷ്ണും ന പൂജാഫലഭാഗ്ഭവേത്\\
വിഷ്ണുര്‍ഭൂത്വാ യജേദ്  വിഷ്ണുമയം വിഷ്ണുരഹം സ്ഥിത:   (5/31/40)\\
}

പ്രഹ്ലാദന്‍ തന്റെ ഗാഢചിന്ത തുടര്‍ന്നു: മഞ്ഞണിഞ്ഞ ഹിമാലയപര്‍വ്വതശിഖരങ്ങള്‍  സൂര്യതാപത്താല്‍ ഉരുകി ഇല്ലാതാകുന്നില്ല. അതുപോലെ ഭഗവാന്‍ വിഷ്ണുവിന്റെ അഭയം ലഭിച്ചവരെ ദുരിതങ്ങള്‍ ബാധിക്കുന്നില്ല. മരക്കൊമ്പിലിരിക്കുന്ന ചെറിയൊരു കുരങ്ങനുപോലും താഴെ നില്‍ക്കുന്ന വലിയൊരു നായയെ അലോസരപ്പെടുത്തുവാനാകും. അതുപോലെ ദേവന്മാരിപ്പോള്‍ വിഷ്ണുവിന്റെ സംരക്ഷയുടെ ബലത്തിലാണ് അസുരന്മാരെ ഉപദ്രവിക്കുന്നത്. 

ഭഗവാന്‍ വിഷ്ണുവാണല്ലോ വിശ്വത്തെ മുഴുവന്‍ സംരക്ഷിച്ചു നിലനിര്‍ത്തുന്നത്. വിഷ്ണുഭഗവാന്‍ ആയുധമുപേക്ഷിച്ചാല്‍ക്കൂടി ആര്‍ക്കും അദ്ദേഹത്തെ ചെറുക്കാനാവില്ല. നരസിംഹം സ്വന്തം കൈനഖങ്ങളല്ലാതെ ആയുധങ്ങള്‍ ഒന്നും ഉപയോഗിച്ചില്ല. ആ ഭഗവാന്‍ മാത്രമാണ് എല്ലാ ജീവജാലങ്ങള്‍ക്കും ഏകാശ്രയം. അതുകൊണ്ട് ആ സവിധത്തില്‍ അഭയം തേടുക മാത്രമാണൊരു വഴി. ആരും അദ്ദേഹത്തിനു മുകളിലില്ല. സൃഷ്ടി-സ്ഥിതി-സംഹാരങ്ങള്‍ എല്ലാം നിയന്ത്രിക്കുന്നതദ്ദേഹമാണല്ലോ. ഈ നിമിഷം മുതല്‍ ഞാനും ഭഗവാന്‍ വിഷ്ണുവിനെ ശരണം പണിഞ്ഞ് ആ സാന്നിദ്ധ്യം ഉള്ളില്‍ നിറച്ച് ജീവിക്കാന്‍ പോവുന്നു. ‘ഓം നമോ നാരായണായ’ എന്ന മന്ത്രം ഭക്തനില്‍ സകലവിധ അനുഗൃഹങ്ങളും ചൊരിയുന്ന ദിവ്യൌഷധമത്രേ. അതെന്റെ മനസ്സില്‍ നിന്നും ഒരിക്കലും പിരിയാതിരിക്കട്ടെ.

“എന്നാല്‍ സ്വയം വിഷ്ണുവല്ലാത്ത ഒരുവന് വിഷ്ണുവില്‍ നിന്നും യാതൊരനുഗ്രഹവും ലഭ്യമല്ല. ഒരുവന്‍ വിഷ്ണുവിനെ പൂജിക്കേണ്ടത് വിഷ്ണുവായിത്തന്നെയാണ്. അതുകൊണ്ട് ഞാന്‍ വിഷ്ണുവാണ്.” പ്രഹ്ലാദന്‍ എന്നറിയപ്പെടുന്ന ആള്‍ വിഷ്ണു തന്നെയാണ്. അവിടെ ദ്വന്ദമില്ല. വിഷ്ണുവിന്റെ ഗരുഡവാഹനം എന്റേതാണ്. എന്റെ അവയവങ്ങളില്‍ വിഷ്ണുഛിഹ്നങ്ങളുണ്ട്. വിഷ്ണുപത്നിയായ ലക്ഷ്മീദേവി എന്റെയടുത്തുണ്ട്. വിഷ്ണുവിന്റെ ദിവ്യപ്രഭ ഇപ്പോഴെന്റേതാണ്. വിഷ്ണുവിനെ അലങ്കരിച്ചിരുന്ന ശംഖം, ചക്രം, ഗദ, പങ്കജം. വാള്‍ തുടങ്ങിയ ദിവ്യായുധങ്ങളിപ്പോള്‍ എന്നെയാണലങ്കരിക്കുന്നത്.

വിഷ്ണുനാഭിയില്‍ എന്നപോലെ എന്റെ നാഭിയില്‍നിന്നുമുയര്‍ന്നുവന്ന താമരയില്‍ ബ്രഹ്മാവ്‌ നിലകൊള്ളുന്നു. എന്റെ വയറ്റില്‍ അനവധി അണ്ഡകഠാഹങ്ങള്‍ ഉണ്ടായി നിലനിന്നു നശിച്ചു കൊണ്ടിരിക്കുന്നു. എന്റെ നിറമിപ്പോള്‍ വിഷ്ണുവര്‍ണ്ണമായ നീലയാണ്. വിഷ്ണുവിനേപ്പോലെ പീതാംബാരധാരിയാണ് ഞാന്‍. ഞാന്‍ വിഷ്ണുവാണ്.

എനിക്കാരുണ്ട് ശത്രുവായി? ആരുണ്ടെനിക്കൊരെതിരാളി? ഞാന്‍ വിഷ്ണുവാകയാല്‍ എന്റെ എതിരാളികളെല്ലാം അങ്ങേലോകമെത്തിക്കഴിഞ്ഞിരിക്കുന്നു. എന്നില്‍ നിന്നുള്ള പ്രഭാപൂരം താങ്ങാനാവാതെ അസുരന്മാര്‍ കുഴങ്ങുന്നു. ഞാന്‍ വിഷ്ണുവാകയാല്‍ ദേവന്മാര്‍ എന്നെ പ്രകീര്‍ത്തിച്ചു പാടുന്നു. ഞാന്‍ എന്നിലെ എല്ലാ ദ്വന്ദഭാവങ്ങളും നീക്കി വിഷ്ണുവായിരിക്കുന്നു. മൂലോകങ്ങള്‍ കുക്ഷിയിലടക്കി, വിശ്വത്തിലെ എല്ലാ തിന്മകളെയും ഒതുക്കി, എല്ലാ ജീവജാലങ്ങള്‍ക്കും അഭയം നല്‍കി, അവരിലെ ആശങ്കകള്‍ നീക്കി വിരാജിക്കുന്ന വിഷ്ണുവാണ് ഞാന്‍. ആ മഹാവിഷ്ണുവിനെ ഞാന്‍ നമസ്കരിക്കുന്നു.

