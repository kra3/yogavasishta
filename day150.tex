\section{ദിവസം 150}

\slokam{
കർത്തവ്യമേവ നിയതം കേവലം കാര്യകോവിദൈ:\\
സുഷുപ്തിവൃത്തിമാശ്രിത്യ കദാചിത്വം ന നാശയ (4/10/39)\\
}

കാലം (കാലന്‍ ) തുടർന്നു: ക്രോധത്തിനു വശംവദനാവാതിരിക്കൂ മഹർഷേ, അത് നാശത്തിലേയ്ക്കുള്ള പാതയാണെന്നു നിശ്ചയം. എന്താണു നടക്കേണ്ടതെന്നുവെച്ചാൽ നടന്നിരിക്കും. ഈ സത്യം മനസ്സിലാക്കൂ. ഞങ്ങൾ മായാമോഹത്തിനടിമകളല്ല. സ്വാഭാവികമായുള്ള പ്രവർത്തനങ്ങളിലേർപ്പെട്ട് അവയെ ഞങ്ങൾ പൂർത്തീകരിക്കുന്നു എന്നു മാത്രം. ജ്ഞാനികൾ അങ്ങിനെയാണ്‌.. “എന്താണു ചെയ്തുതീർക്കേണ്ടതെന്നുവെച്ചാൽ അത് അഹംകാരം കൂടാതെ, സ്വാർത്ഥലാഭേച്ഛകൂടാതെ, ദീർഘനിദ്രയിലെന്നപോലെ  നിർമമതയോടെ  നിവഹിക്കണം. ഈ സ്വാഭാവികതയ്ക്ക് ഭംഗം വരുത്തരുത്.” അങ്ങയുടെ വിവേകവിജ്ഞാനാദികളും ധർമ്മബോധവും എവിടെപ്പോയി? സച്ചിദാനന്ദത്തിലേയ്ക്കുള്ള പാതയെന്തെന്ന് അറിയാമായിരുന്നിട്ടുകൂടി അങ്ങെന്താണ്‌ ഒരു വിഡ്ഢിയേപ്പോലെ പെരുമാറുന്നത്? പാകം വന്ന പഴം താഴെ വീഴുമെന്ന് അങ്ങേയ്ക്കറിയായ്കയല്ല. പിന്നെന്തിനാണ്‌ എന്നെ ശപിക്കാനൊരുങ്ങുന്നത്?

എല്ലാവർക്കും രണ്ടു ശരീരങ്ങളുണ്ട്. ഒന്ന് ഭൗതീകം, മറ്റേത് മാനസീകം. ഭൗതീകശരീരം ചൈതന്യരഹിതമാകയാൽ സ്വയം നശിക്കുന്നു. മനസ്സും പരിമിതമാണെങ്കിലും അതിന്‌ ചിട്ടവട്ടങ്ങളും ക്രമവുമുണ്ട്. എന്നാൽ അങ്ങയുടെ മനസ്സിപ്പോൾ കലുഷമായിരിക്കുന്നു. മനസ്സ് ശരീരത്തെ അതിന്റെ താളത്തിനൊത്ത് തുള്ളിക്കുന്നു. കുട്ടികൾ മണ്ണെടുത്ത് കളിക്കുന്നതുപോലെ മനസ്സ് ശരീരത്തെ മാറ്റങ്ങൾക്കു വിധേയമാക്കുന്നു. മാനസീകമായ കർമ്മങ്ങളാണു കർമ്മങ്ങൾ. മനസ്സിലെ ചിന്തകളാണ്‌ ബന്ധനം. നിർമ്മലവും പ്രശാന്തവുമായ മനസ്സാണ്‌ മോക്ഷം. അവയവങ്ങളോടുകൂടിയ ശരീരത്തെ ഉണ്ടാക്കുന്നതും മനസ്സാണ്‌.. മനസ്സാണ് ചൈതന്യമുള്ളതും അല്ലാത്തതുമായ എല്ലാം. ഈ എണ്ണമില്ലാത്ത വൈവിദ്ധ്യങ്ങളെല്ലാം മനസ്സല്ലാതെ മറ്റൊന്നല്ല.

മനസ്സ് നിർണ്ണയങ്ങൾ നടത്തുമ്പോൾ അതു ബുദ്ധിയാണ്‌.. വിഷയവസ്തുക്കളുമായി തദാത്മ്യം പ്രാപിക്കുമ്പോൾ അത് അഹംകാരം. ശരീരം വെറും ജഢപദാർത്ഥങ്ങളുടെ സംഘാതമാണെങ്കിലും മനസ്സ് അതിനെ സ്വന്തമെന്നു കരുതുന്നു. എന്നാൽ മനസ്സ് സത്യാന്വേഷണോന്മുഖമാവുമ്പോൾ ശരീരാഭിമാനം വെടിഞ്ഞ് പരമപദം പ്രാപിക്കുന്നു.

മഹർഷേ അങ്ങ് ധ്യാനത്തിലാണ്ടിരുന്നപ്പോൾ അങ്ങയുടെ പുത്രൻ സ്വന്തം മായക്കാഴ്ച്ചയിൽ നിന്നൊക്കെ ഏറെ അകന്നുപോയിരുന്നു. ‘ഭൃഗുവിന്റെ മകൻ’ എന്ന മേല്‍വിലാസത്തോടു കൂടിയ ശരീരം അദ്ദേഹമിവിടെ ഉപേക്ഷിച്ചിട്ട് സ്വർഗ്ഗത്തിൽ ഉയിർത്തെഴുന്നേറ്റു. അവിടെ അപ്സരസ്സുകളുമായി അദ്ദേഹം സുഖിച്ചു വാണു. എന്നാൽ കാലം കുറേക്കഴിഞ്ഞപ്പോൾ, പുണ്യമെല്ലാം തീർന്നപ്പോൾ, അദ്ദേഹം പാകമെത്തിയ ഒരു പഴം വീഴുമ്പോലെ താഴേയ്ക്ക്, ഭൂമിയിലേയ്ക്ക് തിരിച്ചു വന്നു. കൂടെ അപ്സരസ്സും ഭൂമിയിലെത്തി. എന്നാൽ തന്റെ സ്വർഗ്ഗീയ ശരീരം അദ്ദേഹത്തിനവിടെത്തന്നെ ഉപേക്ഷിക്കേണ്ടിയും വന്നു.

മറ്റൊരു ഭൗതീകശരീരമെടുക്കാനാണദ്ദേഹം ഭൂമിയിലെത്തിയത്. ഇവിടെ അദ്ദേഹം അനേകം ജന്മങ്ങളെടുത്തു. ബ്രാഹ്മണ കുമാരൻ, രാജാവ്, മുക്കുവൻ, അരയന്നം, വീണ്ടും രാജാവ്, സിദ്ധികളുള്ള ഒരു മുനിവര്യൻ, സ്വർഗ്ഗവാസിയായ ദേവത, മുനികുമാരൻ , മറ്റൊരു മുനികുമാരൻ, വീണ്ടും രാജാവ്, വീണ്ടും മുനികുമാരൻ, നായാട്ടുകാരൻ, രാജാവ്, പുഴുക്കൾ, ചെടികൾ, കഴുത, മുളംകാട്, മാൻ കിടാവ്, പാമ്പ്, വീണ്ടും ദേവത, എന്നീ ജന്മങ്ങൾക്കുശേഷം ഇപ്പോൾ വാസുദേവൻ എന്നപേരിൽ ഒരു ബ്രാഹ്മണകുമാരനായി അദ്ദേഹം ജനിച്ചിരിക്കയാണ്‌. വേദശാസ്ത്രനിപുണനായ അദ്ദേഹമിപ്പോൾ പുണ്യനദിയായ സാമംഗയുടെ തീരത്ത് തപസ്സിലേർപ്പെട്ടു കഴിയുന്നു. 
