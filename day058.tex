\newpage
\section{ദിവസം 058}

\slokam{
മഹച്ചിദ്രൂപമേവത്വം സ്മരണം വിദ്ധി വേദനം\\
കാര്യകാരണതാ തേന സ ശബ്ദോ ന ച വാസ്തവ: (3/21/23)\\
}

ലീല ചോദിച്ചു: മുന്‍പേ തന്നെയുള്ള മതിവിഭ്രമം (പൂർവ്വാര്‍ജ്ജിതവാസനകൾ) ഇല്ലാതെ ആ മഹാത്മാവും ഭാര്യയും ഉണ്ടായതെങ്ങിനെയാണ്‌?

സരസ്വതി പറഞ്ഞു: തീര്‍ച്ചയായും അതിന്റെ ഹേതു സൃഷ്ടികര്‍ത്താവായ ബ്രഹ്മാവിന്റെ ചിന്തയാണ്‌. പ്രളയത്തിനുമുന്‍പ്‌ സമ്പൂര്‍ണ്ണ മുക്തിപ്രാപിച്ചിരുന്നതിനാല്‍ ബ്രഹ്മാവില്‍ ചിന്തകളോ ഓര്‍മ്മകളോ ഒന്നുമുണ്ടായിരുന്നില്ല. ഈ യുഗാരംഭത്തില്‍ ആരോ ഒരാളിൽ സ്വയം ഇങ്ങനെയൊരു ചിന്ത ഉദിച്ചു: 'ഞാനാണ്‌ ഇനി ബ്രഹ്മാവ്‌'. ഇതു തികച്ചും യാദൃശ്ചികം എന്നേ പറയാവൂ. കാക്കയും പനമ്പഴവും പോലെ - കാക്കവന്നിരുന്നതും ഒരു പഴം തിന്റെ തലയില്‍ വീണു- രണ്ടും തികച്ചും സ്വതന്ത്രസംഭവങ്ങളാണെങ്കിലും ഒരേ നിമിഷം അതു സംഭവിച്ചു. പക്ഷേ ഒന്നു മറക്കരുത്‌- ഇതൊക്കെ സംഭവിക്കുന്നതായി തോന്നുന്നുണ്ടെങ്കിലും സൃഷ്ടിയെന്നത്‌ ഉണ്മയല്ല. "അനന്തമായ അവബോധം തന്നെയാണ്‌ ചിന്തകള്‍ , അല്ലെങ്കില്‍ അനുഭവങ്ങള്‍ . കാര്യവും കാരണവും തമ്മില്‍ ബന്ധമൊന്നുമില്ല; രണ്ടും വെറും വാക്കുകള്‍ മാത്രം; വസ്തുതകളല്ല." അനന്താവബോധം എപ്പോഴും മാറ്റമേതുമില്ലതെ നിലകൊള്ളുന്നു.

ലീല പറഞ്ഞു: ദേവീ, അവിടുത്തെ വാക്കുകള്‍ വിജ്ഞാനപ്രദം തന്നെ. എങ്കിലും ഇതുവരെ ഞാന്‍ കേട്ടിട്ടേയില്ലാത്ത അറിവായതിനാല്‍ അതെന്നില്‍ ശരിക്കുറച്ചിട്ടില്ല. അതിനാല്‍ ദിവ്യനായ വസിഷ്ഠന്റെ ആദിമ ഗൃഹം കാണാന്‍ എനിക്കാഗ്രഹമുണ്ട്‌. 

സരസ്വതി പറഞ്ഞു: ലീലേ, നീ നിന്റെ രൂപം ഉപേക്ഷിച്ച്‌ നിര്‍മ്മലമായ ഉള്‍ക്കാഴ്ച്ചയിലേക്കുയര്‍ന്നാലും. കാരണം ബ്രഹ്മത്തിനു മാത്രമേ ബ്രഹ്മത്തെ സാക്ഷാത്കരിക്കാനാവൂ. എന്റെ ശരീരം ശുദ്ധപ്രകാശമാണ്‌; ശുദ്ധബോധമാണ്‌. നിന്റെ ശരീരം അങ്ങിനെയല്ല. ഈ ശരീരംകൊണ്ട്‌ നിന്റെ സ്വന്തം സങ്കല്‍പ്പലോകങ്ങളിലേയ്ക്കുപോലും നിനക്ക്‌ സഞ്ചരിക്കാനാവില്ല. പിന്നെയെങ്ങിനെ മറ്റുള്ളവരുടെ സങ്കല്‍പ്പലോകത്തില്‍ സഞ്ചരിക്കുവാനാകും? എന്നാല്‍ നീ ബോധശരീരിയായാല്‍ നിനക്കുടനേ തന്നെ ആ ദിവ്യാത്മാവിന്റെ ഗൃഹം കാണാം. നീ സ്വയം ഇങ്ങിനെ ധ്യാനിച്ചുറപ്പിച്ചാലും: 'ഞാന്‍ ഈ ശരീരമിവിടെയുപേക്ഷിച്ച്‌ പ്രകാശഗാത്രം സ്വീകരിക്കും. ആ ശരീരവുമായി സാമ്പ്രാണിയിലെ സുഗന്ധമെന്നപോലെ ഞാന്‍ മഹാത്മാവിന്റെ ഗൃഹത്തില്‍ പോവും'. അപ്പോള്‍ ജലം ജലത്തിലലിയും പോലെ നീ ബോധമണ്ഡലത്തില്‍ വിലീനയായി ഒന്നാവും. ഇതുപോലെ നിസ്തന്ദ്രമായി ധ്യാനിച്ചാല്‍ നിന്റെ ശരീരം പോലും സൂക്ഷ്മതപ്രാപിച്ച്‌ ശുദ്ധബോധസ്വരൂപമായിത്തീരും. ഞാന്‍ എന്റെ ശരീരത്തെ ശുദ്ധബോധമായാണ്‌ ദര്‍ശിക്കുന്നത്‌. നിനക്കതിനാവുന്നില്ല. കയ്യിലിരിക്കുന്ന രത്നത്തെ വിലകുറഞ്ഞ വെള്ളാരം കല്ലെന്നു കരുതുന്ന അജ്ഞാനിയേപ്പോലെയാണു നീ. അത്തരം അജ്ഞാനം ഉണ്ടാവുന്നതും സ്വമേധയാ ആണ്‌. എന്നാല്‍ ഈ അജ്ഞത അന്വേഷണം കൊണ്ടും അറിവുകൊണ്ടും ഇല്ലാതാക്കാം. ഈ അജ്ഞതപോലും വാസ്തവമല്ല. അവിദ്യ ഇല്ല; അറിവില്ലായ്മ ഇല്ല; ബന്ധനം ഇല്ല; മുക്തിയും ഇല്ല. എന്നാല്‍ ഒരേയൊരു നിര്‍മ്മലാവബോധം മാത്രം ഉണ്ട്‌.
