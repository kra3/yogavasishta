\section{ദിവസം 267}

\slokam{
ചേതനം ചിത്തരിക്തം ഹി പ്രത്യക്ചേതന മുച്യതേ\\
നിര്‍മനസ്കസ്വഭാവം തന്ന തത്ര കലനാമല: (5/50/21)\\
}

വസിഷ്ഠന്‍ തുടര്‍ന്നു: “മനസ്സിന്റെ പരിമിതികള്‍ ബാധിക്കാത്ത ബോധമാണ് ചേതന, അല്ലെങ്കില്‍ ബുദ്ധിശക്തി. അതാണ്‌ ‘അമനസ്സിന്റെ’ സഹജഭാവം. അതില്‍ ധാരണകള്‍ , ആശയസങ്കല്‍പ്പങ്ങള്‍ എന്നീ മലിനതകളില്ല.” അതാണ്‌ പരമമായ അവബോധം. പരമാത്മാവിന്റെ തലമാണത്. അത് സര്‍വ്വവ്യാപിയാണ്. ദുഷ്ടമനസ്സു വര്‍ത്തിക്കുമ്പോള്‍ ആ ദിവ്യദര്‍ശനം സാദ്ധ്യമല്ല. മനസ്സുള്ളിടത്തു ആശകളുണ്ട്, പ്രത്യാശകളുണ്ട്, സുഖദു:ഖങ്ങളുമുണ്ട്.      

സത്യത്തിലേയ്ക്ക് ഉണര്‍ന്ന ബോധം ധാരണാപരിമിതികളുടെ കുഴിയില്‍ വീഴുന്നില്ല. അതുകൊണ്ട് മാനസീകമായി പലേ ഭാവങ്ങളിലൂടെ, അനുഭവങ്ങളിലൂടെ, കടന്നുപോകുമ്പോഴും ആ ബോധം ലോകമെന്ന മായക്കാഴ്ചയോ  പ്രത്യക്ഷലോകമെന്ന ആവര്‍ത്തനമോ ഉണ്ടാക്കുന്നില്ല.

വേദശാസ്ത്രപഠനത്തിലൂടെ അറിവുണര്‍ന്നവര്‍ , മഹാത്മാക്കളുടെ സത്സംഗം ഹേതുവായും ജാഗരൂകമായ സത്യാന്വേഷണത്താലും അഭ്യാസത്താലും വിഷയാനുബദ്ധമല്ലാത്ത ഒരവസ്ഥയെ പ്രാപിച്ചവരാണ്. അതിനാല്‍ ഒരു സാധകന്‍ തന്റെ മനസ്സിനെ അജ്ഞതയില്‍നിന്നും ചാഞ്ചല്യത്തില്‍നിന്നും ബലമായിത്തന്നെ നീക്കി ശാസ്ത്രപഠനത്തിലേയ്ക്കും സത്സംഗത്തിലേയ്ക്കും ഉന്‍മുഖമാക്കണം. അനന്താവബോധത്തെ, പരമാത്മാവിനെ സാക്ഷാത്ക്കരിക്കാന്‍ ഒരേയൊരു വഴി ആത്മജ്ഞാനം മാത്രമേയുള്ളു. ഒരുവന്റെ ദു:ഖനിവാരണത്തിനായുള്ള ഏക മാര്‍ഗ്ഗം ആത്മാവിനെ സ്വയമറിയുക എന്നത് മാത്രമാണ്.
        
അതുകൊണ്ട് രാമാ നീയീ ലോകത്ത് കര്‍മ്മനിരതനായിരിക്കുമ്പോഴും മനസ്സിന് പിടികൊടുക്കാതെ സ്വയം ശുദ്ധബോധമാണെന്നറിയുക. അതനുഭവമാക്കുക. ‘ഇതെന്റേത്, അത് അയാള്‍ , ഇത് ഞാന്‍’ തുടങ്ങിയ ഭേദചിന്തകള്‍ ഉപേക്ഷിച്ച് അവിച്ഛിന്നമായ ആ ഏകതയില്‍ അഭിരമിക്കൂ. ദേഹമുള്ളിടത്തോളം കാലം വര്‍ത്തമാനവും ഭാവിയും ഒരേബോധത്തോടെ സമതാഭാവത്തില്‍ കാണുക. എപ്പോഴും – സുഖദു:ഖങ്ങളിലും, യൌവനത്തിലും, വാര്‍ദ്ധക്യത്തിലും, നടക്കുമ്പോഴും, സ്വപ്നം കാണുമ്പോഴും, ദീര്‍ഘസുഷുപ്തിയിലും എല്ലാം ആ അനന്താവബോധത്തില്‍ വിലീനനാവൂ. വിഷയധാരണകള്‍ , ആശാസങ്കല്‍പ്പങ്ങള്‍ , എന്നീ മാലിന്യങ്ങളാല്‍ കളങ്കപ്പെടാതെ ആത്മജ്ഞാനത്തില്‍ അഭിരമിക്കൂ.
          
പാവനമെന്നും അല്ലെന്നും സംഭവങ്ങളെ തരംതിരിക്കാതെ ഇഷ്ടാനിഷ്ടങ്ങളെന്ന വേര്‍തിരിവ് മാറ്റിവച്ച് നീ ബോധസ്വരൂപമാണെന്നു തിരിച്ചറിയുക. വിഷയങ്ങളോ വിഷയിയോ കര്‍മ്മങ്ങളോ നിന്നെ ബാധിക്കയില്ലെന്നു തിരിച്ചറിയുക. അങ്ങിനെ നീ ശുദ്ധബോധസ്വരൂപമായി നിലകൊണ്ടാലും.  ജാഗ്രദവസ്ഥയില്‍ ‘ഞാനാണല്ലാമെല്ലാം’ എന്ന അറിവുറപ്പിച്ച് ദീര്‍ഘനിദ്രയിലെന്നപോലെ നിര്‍മമനായി വാഴുക. ദ്വൈതാദ്വൈതഭാവങ്ങളുടെ തരം തിരിവുകളെപ്പറ്റി വ്യാകുലപ്പെടാതെ സമതുലിതാവസ്ഥയെ പ്രാപിച്ചാലും. ശുദ്ധബോധത്തിന്റെയും പരിപൂര്‍ണ്ണ സ്വാതന്ത്ര്യത്തിന്റെയും ഉന്നതതലമാണത്.

അനന്തമായ വിശ്വാവബോധത്തെ ‘ഞാന്‍’ ‘മറ്റുള്ളവര്‍ ’ എന്നിങ്ങനെ ദ്വൈതഭാവത്തില്‍ തരംതിരിക്കാന്‍ സാധിക്കുകയില്ല എന്ന പരമസത്യം നിന്റെ മനസ്സില്‍ രൂഢമൂലമാകട്ടെ.