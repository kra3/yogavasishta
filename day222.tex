\section{ദിവസം 222}

\slokam{
ഭാവാദ്വൈതമുപാശ്രിത്യ സത്താദ്വൈതമയാത്മക:\\
കർമാദ്വൈതമനാദൃത്യ ദ്വൈതാദ്വൈതമയോ ഭവ (5/17/29)\\
}

വസിഷ്ഠൻ തുടർന്നു: അല്ലയോ രാമാ, ദേഹബോധത്തിന്റെ പരിമിതികൾക്കതീതരായി വർത്തിക്കുന്നവർ വിവരണങ്ങള്‍ക്കെല്ലാം അതീതരാണ്‌.. അതിനാൽ ജീവന്മുക്തരായവരുടെ ലക്ഷണങ്ങൾ എന്തെന്നു ഞാൻ പറയാം. സ്വാഭാവികമായി ഒരുവനിൽ നടക്കുന്ന ധർമ്മ കർമ്മങ്ങൾക്കായുള്ള ത്വരകള്‍ -ശരീരധർമ്മങ്ങൾപോലെയുള്ളവ - ആസക്തികളെയും വാസനകളെയും ഉണ്ടാക്കുന്നില്ല. ജീവന്മുക്തന്റെ കർമ്മങ്ങൾ അപ്രകാരമുള്ളവയാണ്‌.. ആർത്തിത്വരകളുണ്ടാക്കുന്ന കർമ്മങ്ങൾ ബന്ധനങ്ങളാണല്ലോ. മുക്തിപദം പ്രാപിച്ച ഋഷിയിൽ അഹംകാരധാരണകൾ ഇല്ല. അങ്ങിനെയുള്ള മുക്തനിലുളവാകുന്ന സഹജഭാവങ്ങൾ എന്താണോ അതുതന്നെയാണയാളുടെ ലക്ഷ്ണങ്ങൾ.

ബാഹ്യവസ്തുക്കളുമായുള്ള സംസർഗ്ഗം ബന്ധനത്തിനു കാരണമാകുന്നു. എന്നാൽ മനപ്പൂർവ്വമല്ലാത്ത, സഹജവും നൈസര്‍ഗികവുമായ ഇച്ഛ യാതൊന്നിനേയും ആശ്രയിച്ചല്ല ഉണ്ടാകുന്നത്. വസ്തുക്കളുമായി സമ്പർക്കമുണ്ടാവുന്നതിനു മുൻപേയുണ്ടായിരുന്ന ഇച്ഛ തന്നെയാണ്‌ ഇപ്പോഴും, എപ്പോഴും ഉള്ളത്. അതു സഹജവും ദു:ഖവിമുക്തവും നിർമ്മലവുമത്രേ. അത്തരം ഇച്ഛ ബന്ധനങ്ങളെ ഉണ്ടാക്കുന്നില്ല എന്ന് ജ്ഞാനികൾ കരുതുന്നു.

‘ഇതെനിയ്ക്കു സ്വന്തമാക്കണം’ എന്ന ഒരു ത്വര ഒരുവനിലുദിക്കുമ്പോൾ അതവന്റെ ഹൃദയത്തെ മലിനമാക്കുന്നു. ജ്ഞാനി അത്തരം ആശകളെ ഏതുവിധേനെയും ഇല്ലായ്മചെയ്യണം. ബന്ധനങ്ങളുണ്ടാക്കുന്ന എല്ലാ ആഗ്രഹങ്ങളേയും മോക്ഷം ലഭിക്കണമെന്ന ആശയെപ്പോലും ഉപേക്ഷിക്കൂ. ഒരു സമുദ്രംപോലെ സ്ഥിതപ്രജ്ഞനാകൂ. ആത്മാവ് ജരാനരകൾക്കും മരണത്തിനും അതീതമാണെന്ന അറിവിൽ ഇവയെക്കുറിച്ചുള്ള ആശങ്കകൾ ഉപേക്ഷിക്കൂ. ഈ പ്രപഞ്ചം മുഴുവൻ ഒരു മായക്കാഴ്ച്ചയാണെന്ന തിരിച്ചറിവിൽ ആഗ്രഹങ്ങളുടെ വ്യർത്ഥത മനസ്സിലാക്കൂ.

മനുഷ്യ ഹൃദയത്തിൽ നാലുതരം ഭാവങ്ങളുണ്ടാവുന്നുണ്ട്. അവ ഇങ്ങിനെയാണ്‌.:: :
1. ഞാനെന്റെ മാതാപിതാക്കളിൽ നിന്നുണ്ടായ ഈ ദേഹമാണ്‌...
2. ഞാൻ ഈ ദേഹത്തിൽ നിന്നു വിഭിന്നമായ, സുസൂക്ഷ്മമായ ഒരണുതത്വമാണ്‌..
3. ലോകത്തിൽ നിറഞ്ഞിരിക്കുന്ന വിഭിന്നങ്ങളായ വസ്തുക്കളുടെയെല്ലാം സനാതനമായ തത്വം ഞാൻ തന്നെയാണ്‌. .
4, ഞാനും ലോകവും ആകാശം പോലെ ശുദ്ധമായ നിശ്ശൂന്യതയാണ്‌..

ഇതിൽ ആദ്യത്തേത് മനുഷ്യനെ ബന്ധനത്തിൽ നിർത്തുമ്പോൾ, മറ്റുള്ളവ അവനെ മുക്തിയിലേയ്ക്കു നയിക്കുന്നു. ആദ്യത്തേതുമായി ബന്ധപ്പെട്ട ആശകൾ മനുഷ്യനെ കൂടുതൽ ബന്ധനങ്ങളിലേയ്ക്കു നയിക്കുമ്പോള്‍ മറ്റു മൂന്നുമായി ബന്ധപ്പെട്ട ആശകളും പ്രതിജ്ഞാബദ്ധതയും അവനെ പരാധീനനാക്കുന്നില്ല.

ഒരിക്കൽ ‘ഞാൻ എല്ലാറ്റിന്റേയും ആത്മാവാണ്‌’ എന്നു തിരിച്ചറിഞ്ഞാൽപ്പിന്നെ ഒരുവൻ അധ:പ്പതിക്കുകയോ വ്യാകുലചിത്തനാവുകയോ ഇല്ല. ഈ ആത്മാവിനെയാണ്‌ ശാസ്ത്രങ്ങളിൽ ശൂന്യത, പ്രകൃതി, മായ, ബ്രഹ്മം, ബോധം, ശിവൻ, പുരുഷൻ എന്നെല്ലാം ഉള്ള വാക്കുകളാൽ വിവരിക്കുന്നത്. അതുമാത്രമാണുണ്മ. മറ്റൊന്നിനും സത്തയില്ല.

“ഈ അദ്വൈത - രണ്ടില്ലാത്ത- സത്യാവസ്ഥയെ അറിയുക. എന്നാൽ കർമ്മങ്ങള്‍ ദ്വൈതതലത്തിലാണുള്ളത് എന്നതുകൊണ്ട് ആപേക്ഷികമായി ദ്വൈതാവസ്ഥയെയും അറിയുക. അങ്ങിനെ നിന്റെ സ്വഭാവം ദ്വൈതാദ്വൈതങ്ങളിൽ ഉചിതമായി വിഹരിക്കട്ടെ.” ദ്വൈതവും ഏകാത്മകതയും ഒന്നും യദാർത്ഥത്തിൽ ഇല്ല. ഏകത എന്നത് ദ്വൈതത്തിന്റെ വിപരീത ആശയമാണ്‌. ഈ രണ്ടു ധാരണകളും മനസ്സിന്റെ സൃഷ്ടികളാണു താനും. ഈ ധാരണകളെല്ലാം അവസാനിക്കുമ്പോൾ അനന്താവബോധം മാത്രം ഒരേയൊരുണ്മയായി സാക്ഷാത്കരിക്കപ്പെടും. 
