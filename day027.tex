 
\section{ദിവസം 027}

\slokam{
ഇമാം മോക്ഷ കഥാ  ശ്രുത്വാ സഹസർവൈർവിവേകിഭി:\\
പരം യസ്യസി നിർദു:ഖം നാശോ യത്ര ന വിദ്ധ്യതേ (2/10/8)\\
}

വസിഷ്ഠന്‍ തുടര്‍ന്നു: മനുഷ്യര്‍ നിയതിയെന്നു പറയുന്ന പ്രപഞ്ചനീതിയാണ്‌ എല്ലാ പ്രവര്‍ത്തനങ്ങള്‍ക്കും ഉചിതമായ ഫലപൂര്‍ത്തി അനുഗ്രഹിച്ചുറപ്പാക്കുന്നത്‌. അതു നടപ്പിലാവുന്നത്‌ സര്‍വ്വശക്തവും സര്‍വ്വവ്യാപിയുമായ ബ്രഹ്മം മൂലമാണ്‌. അതുകൊണ്ട്‌ മനസ്സേന്ദ്രിയങ്ങളെ സംയമനം ചെയ്ത്‌ മനസ്സിനെ ഏകാഗ്രമാക്കി ഞാന്‍ പറയാന്‍ പോവുന്ന കാര്യം ശ്രദ്ധയോടെ കേട്ടാലും. "ഇത്‌ മുക്തിയെപ്പറ്റിയുള്ള ആഖ്യാനമാണ്‌. ഇവിടെ സന്നിഹിതരായ മറ്റു സത്യാന്യോഷികളോടൊത്ത്‌ ഇതു ശ്രവിച്ചാല്‍ നിനക്ക്‌ പരംപൊരുളിനെ സാക്ഷാത്കരിക്കാനാവും. അതു ദു:ഖമോ വിനാശമോ ഇല്ലാത്തൊരു തലമത്രേ." 

കഴിഞ്ഞ യുഗത്തില്‍ ബ്രഹ്മദേവന്‍ എനിക്കു വെളിപ്പെടുത്തിയതാണിത്‌. രാമ: എല്ലാ ജീവജാലങ്ങളിലും സര്‍വ്വശക്തനായ സര്‍വ്വവ്യാപി, വിശ്വനാഥന്‍, നിതാന്തഭാസുര സാന്നിദ്ധ്യമായി വിരാജിക്കുന്നു. ഉണ്മയില്‍ ഉണ്ടായ ആദ്യപ്രകമ്പനത്തില്‍നിന്നും ഭഗവാന്‍ വിഷ്ണു സംജാതനായി. സമുദ്രോപരി കാറ്റടിച്ചും മറ്റുമുണ്ടാവുന്ന അസ്വസ്ഥതകള്‍ തിരകളുണ്ടാവാന്‍ കാരണമാവുന്നതുപോലെയാണത്. വിഷ്ണുവില്‍ നിന്നും സൃഷ്ടികര്‍ത്താവായ ബ്രഹ്മാവു ജനിച്ചു. ബ്രഹ്മാവ്‌ അസംഖ്യം തരത്തിലുള്ള ജീവ-നിര്‍ജ്ജീവജാലങ്ങളെയും ബോധമുള്ളവയും ഇല്ലാത്തതുമായ സത്വങ്ങളേയും സൃഷ്ടിക്കാന്‍ തുടങ്ങി. പ്രളയത്തിനുമുന്‍പുണ്ടായിരുന്നതുപോലെ വിശ്വം വീണ്ടും ഉണര്‍ന്നു. ജീവജാലങ്ങള്‍ രോഗത്തിനും വേദനകള്‍ക്കും ദുരിതങ്ങള്‍ക്കും മരണത്തിനും അടിപ്പെട്ടിരിക്കുന്നതു കണ്ട്‌ ബ്രഹ്മാവിന്റെ മനസ്സലിഞ്ഞു. ഈ ദുരിതസഞ്ചയങ്ങളില്‍ നിന്നും അവയെ രക്ഷിക്കാനുള്ള മാര്‍ഗ്ഗമെന്തുള്ളു എന്ന് അദ്ദേഹം ആരാഞ്ഞു. അങ്ങിനെ തീര്‍ത്ഥാടന സങ്കേതങ്ങളും; തപസ്സ്‌, ദാനം, ധര്‍മ്മം, സത്യം എന്നീ ഉത്തമഗുണങ്ങളും സ്ഥാപിച്ചു. എന്നാല്‍ ഇവ അപര്യാപ്തമായിരുന്നു. കാരണം അവയ്ക്ക്‌ ദുരിതങ്ങളില്‍ നിന്നും താല്‍ ക്കാലികമായ ശമനം നല്‍കാനേ കഴിഞ്ഞുള്ളു. അന്തിമമുക്തി അപ്രാപ്യമായി നിലനിന്നു. ഇതേപറ്റി ധ്യാനിച്ച്‌ സൃഷ്ടാവെന്നെ മാനസപുത്രനായി ജനിപ്പിച്ചു.

എന്നെ അദ്ദേഹം ചേര്‍ത്തുപിടിച്ചു നിര്‍ത്തി ആദ്യം തന്നെ അജ്ഞാനത്തിന്റെ മുഖപടം അണിയിച്ചു. അപ്പോള്‍ത്തന്നെ എന്റെ സ്വത്ത്വബോധം മറഞ്ഞു.  ഞാന്‍ സ്വയം ആരെന്ന അറിവേതുമില്ലാതെ വലയുകയായിരുന്നു. ഈ കഷ്ടപ്പാടില്‍നിന്നും രക്ഷകിട്ടാനുള്ള മാര്‍ഗ്ഗമുപദേശിക്കാന്‍ ഞാന്‍ ബ്രഹ്മാവിനോട്‌ യാചിച്ചു.    ദുരവസ്ഥയില്‍ മുങ്ങിയ  ഞാന്‍   ഒന്നും ചെയ്യാനരുതാതെ മടിയനും കര്‍മ്മവിമുഖനുമായിത്തീര്‍ന്നിരുന്നു. എന്റെ പ്രാര്‍ത്ഥനയില്‍ സം പ്രീതനായ പിതാവ്‌ എനിക്കായി ആ സത്യജ്ഞാനം വെളിപ്പെടുത്തി. ആ ക്ഷണത്തില്‍ അദ്ദേഹം എന്നെ മൂടിയ അജ്ഞാനാവരണം നീങ്ങി. അപ്പോള്‍ ബ്രഹ്മാവ്‌ പറഞ്ഞു: "മകനേ ഞാന്‍ നിന്നെ ആദ്യം അജ്ഞാനാവരണം കൊണ്ട് മൂടിയിട്ട്‌ പിന്നീട്‌ അതുമാറ്റി നിനക്ക്‌ സത്യം വെളിപ്പെടുത്തിയത്‌ എന്റെ മഹിമ നിനക്ക്‌ അനുഭവവേദ്യമാക്കാനാണ്‌. അങ്ങിനെമാത്രമേ ജീവജാലങ്ങള്‍ കടന്നുപോകുന്ന കഷ്ടപ്പാടുകളെപ്പറ്റി മനസ്സിലാക്കി അവരെ സഹായിക്കാന്‍ നിനക്കു സാദ്ധ്യമാവൂ."

രാമ: ഈ വിദ്യയുമായി സൃഷ്ടി അവസാനിക്കുവോളം ഞാന്‍ സേവനനിരതനായി ഇവിടെയുണ്ടാവും. 
