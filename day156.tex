\section{ദിവസം 156}

\slokam{
മത്പുത്രോയമിതി സ്നേഹോ ഭൃഗുമപ്യഹരത്തദാ\\
പരമാത്മീയതാ ദേഹേ യാവദാകൃതിഭാവിനീ (4/16/18)\\
}

വസിഷ്ഠൻ തുടർന്നു: യുവമുനിയായ വാസുദേവൻ തന്റെ പൂർവ്വജന്മശരീരത്തിന്റെ ഇപ്പോഴത്തെ സ്ഥിതികണ്ട് വിലപിക്കവേ കാലദേവൻ (യമൻ) വസുദേവശരീരത്തിലുള്ള ശുക്രനോടായി ഇങ്ങിനെ പറഞ്ഞു: അല്ലയോ ഭൃഗുപുത്രാ ഒരു രാജാവ് തന്റെ സാമ്രാജ്യത്തിലേയ്ക്ക് പുന:പ്രവേശനംചെയ്യുന്നതുപോലെ നിന്റെ ഈ ശരീരമുപേക്ഷിച്ച് മറ്റേ ശരീരത്തിൽ കയറിയാലും. ശുക്രന്റെ ആ ശരീരമുപയോഗിച്ച് വീണ്ടും തപസ്സനുഷ്ഠിക്കണം. അങ്ങിനെ അസുരവംശത്തിന്റെ അത്മീയ ഗുരുസ്ഥാനമേറ്റെടുക്കുകയും വേണം. ഈ യുഗമവസാനിക്കുമ്പോൾ അങ്ങേയ്ക്ക് ഈ ശരീരം പോലും ഉപേക്ഷിക്കാം. പിന്നീട് ഒരിക്കലും ശരീരമെടുക്കേണ്ടിവരികയില്ല. ഇത്രയും പറഞ്ഞ് യമൻ അവിടെനിന്നും അപ്രത്യക്ഷമായി.

ശുക്രൻ സാമംഗാ നദിക്കരയിൽ തീവ്രതപസ്സനുഷ്ഠിച്ചിരുന്ന തന്റെ വാസുദേവനായുള്ള ശരീരത്തെ ഉപേക്ഷിച്ച് ശുക്രന്റെ ജീർണ്ണിച്ച ദേഹത്തിലേയ്ക്ക്, ഭൃഗുപുത്രനായി കൂടുമാറി. ആ ക്ഷണത്തിൽ വാസുദേവന്റെ ശരീരം വെട്ടിയിട്ട മരം പോലെ ശവമായി നിലത്തു വീണു. ഭൃഗു മഹർഷി തന്റെ കമണ്ഡലുവിൽ നിന്നും ദിവ്യജലമെടുത്ത് മന്ത്രജപങ്ങളോടെ ശുക്രന്റെ ജീർണ്ണദേഹത്തിൽ തളിച്ചു. ശരീരത്തെ മാംസാദികളായ വസ്ത്രങ്ങളുടുപ്പിച്ച് പുനരുദ്ധരിക്കാൻ ശക്തിയുള്ള മന്ത്രങ്ങളായിരുന്നു അദ്ദേഹമുച്ചരിച്ചത്. ആ ശരീരത്തിന്‌ പഴയപോലെ യൗവ്വനവും തേജസ്സും തിരികെ കിട്ടി. ധ്യാനാസനത്തിൽ നിന്നും എഴുന്നേറ്റ ശുക്രൻ മുന്നിൽ നിൽക്കുന്ന പിതാവിനെ സാഷ്ടാംഗം നമസ്കരിച്ചു. ഭൃഗുമുനി ആളാദത്തോടെ പുഞ്ചിരിതൂകി മരണത്തിൽ നിന്നും തിരികെവന്ന മകനെ ഗാഢമായി ആലിംഗനം ചെയ്തു.

“ഇതാ എന്റെ മകൻ എന്ന സ്നേഹഭാവം ഭൃഗുവിൽ തീവ്രമായി അങ്കുരിച്ചു. ശരീരബോധമുള്ളിടത്തോളം ഇതു സഹജമാണ്‌.” രണ്ടാളും ഈ പുന:സമാഗമത്തിൽ സന്തോഷചിത്തരായി. പിന്നീടവർ 'വാസുദേവൻ എന്ന ബ്രാഹ്മണകുമാരന്റെ' ശരീരത്തെ ദഹിപ്പിച്ച് വേണ്ടരീതിയിൽ കർമ്മങ്ങൾ അനുഷ്ഠിച്ചു. ജ്ഞാനികൾ സമൂഹത്തിലെ നിയതകർമ്മങ്ങളേയും പാരമ്പര്യങ്ങളേയും ബഹുമാനിക്കുന്നവരത്രേ. സൂര്യചന്ദ്രന്മാരെപ്പോലെ അവര്‍ രണ്ടുപേരും ഭാസുരപ്രഭയാർന്നു നിലകൊണ്ടു. ലോകത്തിന്റെ മുഴുവൻ ആത്മീയഗുരുക്കളായി അവർ വിശ്വം മുഴുവൻ സഞ്ചരിച്ചു. ആത്മവിദ്യയിൽ അടിയുറച്ചിരുന്നതിനാൽ അവരെ സ്ഥലകാലവ്യതിയാനങ്ങൾ ബാധിച്ചതേയില്ല. കാലക്രമത്തിൽ ശുക്രൻ അസുരവംശത്തിന്റെ ഗുരുവായി. ഭൃഗുമുനി പരമവിജ്ഞാനത്തിന്റെ ഉത്തുംഗത്തിൽ വിരാജിക്കുന്ന ഋഷിവര്യനായി.

ഇതാണ്‌ ഒരപ്സരസ്സിനെ കണ്ടു മോഹിച്ചതിന്റെ ഫലമായി അനേകം യോനികളിൽ ജനിച്ച് അലയേണ്ടിവന്ന ശുക്രന്റെ കഥ. 
