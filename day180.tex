\section{ദിവസം 180}

\slokam{
അജ്ഞസ്യാർധപ്രബുദ്ധസ്യ സർവം ബ്രഹ്മേതി യോ വദേത്\\
മഹാനരകജാലേഷു സ തേന വിനിയോജിത: (4/39/24)\\
}

വസിഷ്ഠൻ തുടർന്നു: രാമാ, പരബ്രഹ്മം സർവ്വശക്തമായതിനാൽ അതിലെ അനന്തമായ സാദ്ധ്യതകളാണ്‌ ഈ കാണപ്പെടുന്ന പ്രപഞ്ചമായി പ്രത്യക്ഷമാകുന്നത്. എല്ലാ വൈവിദ്ധ്യങ്ങളും - സത്തായതും അസത്തായതും, ഏകത്വവും നാനാത്വവും, തുടക്കവും ഒടുക്കവും എല്ലാം ബ്രഹ്മത്തിലാണു സ്ഥിതിചെയ്യുന്നത്. സമുദ്രോപരി കാണപ്പെടുന്ന തിരകളെന്നപോലെ വ്യക്തിഗതമായി പരിമിതപ്പെട്ട അവബോധവുമായി ജീവൻ പ്രത്യക്ഷമാവുന്നതും ഈ ബ്രഹ്മത്തിലത്രേ. ഈ ജീവനാകട്ടേ ഉത്തരോത്തരം കൂടുതൽ ആഴത്തിലുള്ള ഉപാധികളുടെ സ്വാധീനത്താൽ തദനുസാരിയായ കർമ്മങ്ങളിലേർപ്പെട്ട് അവയുടെ ഫലങ്ങൾ അനുഭവിക്കുകയും ചെയ്യുന്നു.

രാമൻ ചോദിച്ചു: ഭഗവൻ, ബ്രഹ്മം ദു:ഖരഹിതമാണെന്നു പറഞ്ഞു. എന്നാൽ ഒരു ദീപത്തിൽ നിന്നു കൊളുത്തിയ മറ്റൊരു ദീപം പോലെ, ബ്രഹ്മത്തിൽ നിന്നുദ്ഭൂതമായ ലോകം എന്തുകൊണ്ടാണ്‌ ദു:ഖപൂരിതമായിരിക്കുന്നത്? അതെങ്ങിനെ പറ്റി?

വാൽമീകി പറഞ്ഞു: രാമന്റെ ഈ ചോദ്യംകേട്ട് വസിഷ്ഠൻ കുറച്ചുനേരം ചിന്താമഗ്നനായിരുന്നു: രാമന്റെ മനസ്സിലെ മാലിന്യം മുഴുവനും ഇല്ലാതായിട്ടില്ല. അതിനാലാണീ സംശയം. എന്നാൽ ഇതിന്റെ ഉത്തരം കണ്ടെത്തുംവരെ ആ മനസ്സിൽ സമാധാനം ഉണ്ടാവുകയില്ലല്ലോ. മനസ്സ് സുഖസന്തോഷചിന്തകളിൽ ആന്ദോളനംചെയ്യുന്നിടത്തോളം സത്യമെന്തെന്നറിയാൻ അതിനു കഴിയുകയില്ല. നിർമ്മലമായ മനസ്സിൽ ഈ അറിവ് ക്ഷണനേരത്തിൽ ഉണർവ്വാവും. അതുകൊണ്ടാണ്‌ "ആരൊരുവൻ അജ്ഞാനികളെ ‘ഇതെല്ലാം ബ്രഹ്മമാണ്‌’ എന്നു പഠിപ്പിക്കുന്നുവോ അവൻ നരകത്തിൽപ്പോവും എന്നിങ്ങിനെ ഒരു ചൊല്ലുള്ളത്.” ഉത്തമനായ ഗുരു ആദ്യം തന്നെ തന്റെ ശിഷ്യനെ ആത്മസംയമനവിദ്യയില്‍ നിരതനാക്കി പ്രശാന്തമനസ്സിനുടമയാക്കുന്നു. എന്നിട്ട് ശിഷ്യന്റെ ജാഗ്രതയെ പരീക്ഷിച്ചറിഞ്ഞതിനുശേഷം മാത്രമേ സത്യജ്ഞാനം പകർന്നു നൽകൂ.

വസിഷ്ഠൻ പറഞ്ഞു: പരബ്രഹ്മം ദു:ഖമുക്തമോ അല്ലയോ എന്ന് നിനക്ക് സ്വയം കണ്ടെത്തുവാനാകും. അല്ലെങ്കിൽ കാലക്രമത്തിൽ നിന്നെ ഞാനതിനു സഹായിക്കാം. ഇപ്പോള്‍  ഇത്രയും മനസ്സിലാക്കുക: പരബ്രഹ്മം സർവ്വശക്തവും സർവ്വവ്യാപിയും എല്ലാവരുടെയുള്ളിലും കുടികൊള്ളുന്ന ചൈതന്യവുമാണ്‌... വിശദീകരണങ്ങൾക്കതീതമായ മായാശക്തിയിലൂടെയാണ്‌ പരബ്രഹ്മം സൃഷ്ടികളെ സാക്ഷാത്കരിച്ചത്. ഈ മായയുടെ പ്രഭാവത്തിലാണ്‌ അസത്ത് സത്തായും, തിരിച്ചും പ്രത്യക്ഷമാകുന്നത്. നിശ്ശൂന്യമായ ആകാശത്തിനു നീലനിറം നൽകുന്നതും ഈ മായാശക്തിയത്രേ.

നോക്കൂ രാമാ, വിപുലമായ വൈവിദ്ധ്യത ഈ ലോകത്തിലെ ജീവജാലങ്ങളിൽത്തന്നെ നിനക്കു ദർശിക്കാമല്ലോ. അതാണ്‌ ഭഗവാന്റെ അനന്തമായ ശക്തിവിശേഷം. പ്രശാന്തതയെ കൈവരിക്കൂ. ആന്തരീകമായി ശാന്തതയുള്ളവർക്ക് സത്യം വെളിപ്പെടുന്നു. മനസ്സ് അശാന്തമാണെങ്കിലോ ലോകം വൈവിദ്ധ്യങ്ങൾ നിറഞ്ഞതും, ചിന്താക്കുഴപ്പം ഉണ്ടാക്കുന്നതും ആണ്‌.... എന്നാൽ ഈ ലോകമെന്നത് ഭഗവാന്റെ അനന്തസാദ്ധ്യതകളുടെ പ്രത്യക്ഷപ്രകടനമായ വിക്ഷേപം മാത്രമാണ്. പ്രകാശത്തിന്റെ സാന്നിദ്ധ്യത്തിൽ വസ്തുക്കളുടെ കാഴ്ച്ച സ്വാഭാവികമാണെന്നതുപോലെ സർവ്വശക്തനായ ഭഗവദ്സാന്നിദ്ധ്യത്തിൽ ലോകമെന്ന കാഴ്ചയും സഹജമായി ആ ഭഗവാനിൽ നിന്നും ഉദ്ഭൂതമായതാണ്‌.. ലോകം പ്രത്യക്ഷമായതിനൊപ്പം തന്നെ അജ്ഞാനവും ഉണ്ടായി. അതാണ്‌ ശോകത്തിനു കാരണം. അജ്ഞാനത്തെ ഉപേക്ഷിക്കൂ, അങ്ങിനെ സ്വതന്ത്രനാവൂ. 

