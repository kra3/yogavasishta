\section{ദിവസം 264}

\slokam{
ഗാധേ സ്വാധി വിധുതസ്യ സ്വരൂപസ്യൈതദാത്മകം\\
ചേതസോഽദൃഷ്ടത്വസ്യ  യത്പശ്യത്യുരുവിഭ്രമം   (5/48/48)\\
}

വസിഷ്ഠന്‍ തുടര്‍ന്നു: ആ ഗ്രാമത്തില്‍ താനുമായി, തന്റെ ‘ജീവിതവുമായി’ ബന്ധപ്പെട്ടിരുന്ന പല സാധനസാമഗ്രികളെയും ഗാധി തിരിച്ചറിഞ്ഞു. താനെവിടെയാണ് കുടിച്ചു മദോന്മത്തനായി മയങ്ങിക്കിടന്നത്, ഏതു വേഷമാണ് താന്‍ ധരിച്ചിരുന്നത്, എന്താണ് ആഹാരമായി കഴിച്ചിരുന്നത് എന്നെല്ലാം ഗാധിയുടെ ചിന്താമണ്ഡലത്തില്‍ പൊങ്ങിവന്നു. അവിടെനിന്നും ഗാധി കേര രാജ്യത്തേയ്ക്കാണു പോയത്. അവിടെ തലസ്ഥാനനഗരിയില്‍ ചെന്ന് നാട്ടുകാരോടു കാര്യങ്ങള്‍ തിരക്കി.

‘ഈ നാട് കുറേകാലം മുന്‍പ് ഒരു ചണ്ഡാളന്‍  ഭരിച്ചിരുന്നുവോ?’ അവര്‍ ആവേശത്തോടെ പറഞ്ഞു: തീര്‍ച്ചയായും. അയാള്‍ എട്ടുവര്‍ഷം രാജ്യം ഭരിച്ചു. ഇവിടുത്തെ രീതിയനുസരിച്ച് കൊട്ടാരത്തിലെ ആനയാണ് അദ്ദേഹത്തെ രാജാവായി തിരഞ്ഞെടുത്തത്. പക്ഷെ അയാളുടെ തനിനിറം പുറത്തുവന്നപ്പോള്‍ രക്ഷയില്ലാതെ അയാള്‍ ആത്മഹത്യചെയ്തു.! അതൊരു പന്ത്രണ്ടുകൊല്ലം മുന്‍പാണ്.’ ഇത് പറഞ്ഞു നില്‍ക്കുമ്പോഴേയ്ക്ക് അതാ രാജാവ് പരിവാരസമേതം കൊട്ടാരത്തില്‍ നിന്ന് പുറത്തേയ്ക്കെഴുന്നള്ളുന്നു. അത്ഭുതം! അത് രാജാവിന്റെ വേഷത്തില്‍ സാക്ഷാല്‍ വിഷ്ണുഭഗവാന്‍ തന്നെയായിരുന്നു.

ഇതെല്ലാം കണ്ട ഗാധി വിസ്മയംകൂറി ഇങ്ങിനെ ചിന്തിച്ചു: ഞാന്‍ അടുത്തകാലത്ത് ഭരിച്ചിരുന്ന കീരരാജ്യമാണല്ലോ. എന്നാല്‍ അതെന്റെ പോയ ജന്മത്തിലെപ്പൊഴോ സംഭവിച്ചതുപോലെയാണ് ഞാനിപ്പോള്‍ കാണുന്നത്. അതൊരു സ്വപ്നമായിരുന്നു എന്നാല്‍ ജാഗ്രദവസ്ഥയില്‍ എന്റെ കണ്‍മുന്നില്‍ നടന്നതുപോലെയാണെല്ലാം. അഹോ കഷ്ടം ഞാനേതോ മായാവിദ്യയ്ക്കടിമയായിരിക്കുന്നു. ശരിയാണ്! എനിയ്ക്ക് മായാദര്‍ശനം ലഭിക്കുമെന്നാണല്ലോ വിഷ്ണുഭഗവാന്‍ വരം തന്നത്! അതിതു തന്നെയാണ്!  

അദ്ദേഹം ആ നിമിഷം നഗരം വിട്ടുപോയി ഒരു മലമുകളിലെ ഗുഹയ്ക്കുള്ളില്‍ തീവ്രമായ തപസ്സനുഷ്ഠിച്ചു. താമസംവിനാ വിഷ്ണുഭഗവാന്‍ പ്രത്യക്ഷപ്പെട്ട് ഇഷ്ടവരം എന്തുവേണമെന്നാരാഞ്ഞു. ഗാധി ഭഗവാനോട് ചോദിച്ചു: ഭഗവാനേ, ഞാന്‍ സ്വപ്നത്തില്‍ കണ്ട, അല്ലെങ്കില്‍ മായക്കാഴ്ചയില്‍ അറിഞ്ഞ കാര്യങ്ങള്‍ ജാഗ്രദിലും അങ്ങിനെതന്നെ സംഭവിക്കാന്‍ എന്താണ് കാരണം?

ഭഗവാന്‍ പറഞ്ഞു: “ഗാധി! നീയിപ്പോള്‍ കാണുന്നതും ഒരു മായക്കാഴ്ച തന്നെ. എന്നാല്‍ സത്യത്തില്‍ അത് ആത്മാവുതന്നെയാണ്. മറ്റൊന്നല്ല. എന്നാല്‍ മനസ്സതിനെ മറ്റേതോ 'വസ്തുക്കളായി' കാണുന്നത് ജീവന് പരിശുദ്ധിയില്ലാത്തതിനാലും അതിനിതുവരെ സത്യദര്‍ശനം ലഭിച്ചിട്ടില്ലാത്തതിനാലുമാണ്.”  

ആത്മാവില്‍ നിന്നും ബാഹ്യമായി മറ്റൊന്നുമില്ല. ഒരു ചെറുവിത്തിനുള്ളില്‍ വന്‍മരമുള്ളതുപോലെ ഇതെല്ലാം മനസ്സില്‍ നേരത്തെതന്നെ ഉള്ളതാണ്. എന്നാല്‍ മനസ്സ് അതെല്ലാം ബാഹ്യമെന്ന മട്ടില്‍ കാണുന്നു എന്നുമാത്രം. മനസ്സാണിതെല്ലാം. ഇപ്പോള്‍കാണുന്നതും ഭാവിയിലേയ്ക്കുള്ളതെന്നമട്ടില്‍  സങ്കല്‍പ്പിക്കുന്നതും കഴിഞ്ഞകാര്യങ്ങളെന്ന മട്ടില്‍ ഓര്‍ത്തുവയ്ക്കുന്നതും എല്ലാം. മനസ്സാണ് സ്വപ്നമായും ഭ്രമകല്‍പ്പനയായും, അസുഖങ്ങളായുമെല്ലാം പ്രകടമാവുന്നത്.
   
മനസ്സില്‍ എണ്ണമൊടുങ്ങാത്ത ‘സംഭവങ്ങള്‍ ’ മരത്തില്‍ പൂത്തുലഞ്ഞു നില്‍ക്കുന്ന പൂക്കളെപ്പോലെ നിറതിങ്ങി നില്‍ക്കുന്നു. എന്നാല്‍ വെരറ്റുപോയ മരത്തില്‍ പൂക്കളുണ്ടാകാത്തതുപോലെ ധാരണകളും ആശാസങ്കല്‍പ്പങ്ങളുമൊഴിഞ്ഞ മനസ്സ് വീണ്ടും വീണ്ടും ജനിക്കേണ്ടതായി വരുന്നില്ല. എണ്ണമറ്റ ചിന്താരൂപങ്ങള്‍ നിറഞ്ഞ മനസ്സിനു ‘ഞാന്‍ ചണ്ഡാളനാണ് ’ എന്നൊരു തോന്നലുണ്ടാക്കാന്‍ എത്ര എളുപ്പം! അതുപോലെ ‘ഞാന്‍’ തന്നെയാണ് നിന്റെയടുത്തു വന്നു കഥ പറഞ്ഞ ആ അതിഥി ബ്രാഹ്മണനും. ‘ഞാന്‍ ഭൂതമണ്ഡലത്തില്‍ പോകുന്നു' എന്നും ‘ഞാനിപ്പോള്‍ കീരരാജ്യത്തിലാണെന്നും’ എല്ലാം സങ്കല്‍പ്പിച്ചുണ്ടാക്കുന്നത് ആ മനസ്സാണ്. ഇതൊക്കെ വെറും ഭ്രമചിന്തകളായിരുന്നു! 

അങ്ങ് രണ്ടു തരത്തിലുമുള്ള ഭ്രമകല്‍പ്പനകള്‍ കണ്ടു കഴിഞ്ഞു. ഒന്ന്, അങ്ങ് സ്വയം ഭ്രമകല്‍പ്പനയെന്നു നിശ്ചയിച്ചു കണ്ട കാഴ്ചകള്‍ . മറ്റേത്, യാഥാര്‍ത്ഥ്യമെന്നു സ്വയം കല്‍പ്പിച്ചുകണ്ട ഭ്രമക്കാഴ്ചകള്‍ . രണ്ടും വാസ്‌തവത്തില്‍ വെറും ഭ്രമം മാത്രമായിരുന്നു! അങ്ങ് ആ ബ്രാഹ്മണനായി അതിഥി സല്‍ക്കാരമൊന്നും നടത്തിയിട്ടില്ല. അങ്ങെങ്ങും യാത്ര പോയുമില്ല. എല്ലാം ഭ്രമകല്‍പ്പനയിലെ മായക്കാഴ്ചകള്‍ മാത്രം. അങ്ങ് ഭൂതമണ്ഡത്തിലോ കീരരാജ്യത്തിലോ പോയിട്ടില്ല. അവയും വെറും മായ. ഉണരൂ മഹര്‍ഷേ! അങ്ങ് ഇവിടെയിപ്പോള്‍ ഉചിതമായ കര്‍മ്മങ്ങളില്‍ ഏര്‍പ്പെട്ടാലും. കാരണം കര്‍മ്മങ്ങളെക്കൂടാതെ ആരും ഈ ജീവിതത്തില്‍ മൂല്യവത്തായി ഒന്നും നേടിയിട്ടില്ല എന്നറിയുക.