\section{ദിവസം 147}

\slokam{
മന: സർവ്വമിദം രാമ: തസ്മിന്നന്തശ്ചികിത്സിതേ\\
ചികിത്സിതോ വൈ സകലോ ജഗജ്ജാലാമയോ ഭവേത് (4/4/5)\\
}

വസിഷ്ഠൻ തുടർന്നു: രാമ: ഈ ദുർഘടമായ സംസാരസാഗരത്തെ കടക്കാൻ ഇന്ദ്രിയങ്ങളെ വിജയകരമായി നിയന്ത്രിക്കുക എന്നൊരു മാർഗ്ഗം മാത്രമേയുള്ളു. മറ്റൊരു പ്രയത്നംകൊണ്ടും പ്രയോജനമില്ല. ശാസ്ത്രപഠനങ്ങളിലൂടെയും മഹർഷിമാരുമായുള്ള സത്സംഗം വഴിയും ജ്ഞാനമാർജ്ജിച്ച് ഇന്ദ്രിയങ്ങളെ തന്റെ വരുതിയിൽ നിർത്തിയവന്‌ കാണപ്പെടുന്ന വസ്തുക്കൾ - ദൃശ്യപ്രപഞ്ചം- മിഥ്യയാണെന്ന ധാരണ ദൃഢീകരിക്കുന്നു. “രാമ: ഇതെല്ലാം മനസ്സു മാത്രമാണ്‌.. ഈ മനസ്സ് ശമിച്ചാൽ ദൃശ്യപ്രപഞ്ചമെന്ന ഈ മായക്കാഴ്ച്ചയും ശമിക്കും.“

മനസ്സാണ്‌ ശരീരത്തെ ചിന്തിച്ചുണ്ടാക്കുന്നത്. മനസ്സു പ്രവർത്തിക്കാത്ത ഒരിടവും ആരും കണ്ടിട്ടില്ല. അതിനാൽ വിഷയവസ്തുക്കളെക്കുറിച്ചുള്ള തെറ്റിദ്ധാരണയെന്ന മാനസീകരോഗം മാറാനുള്ള ചികിൽസയാണ് എറ്റവും ഉത്തമമായ ചികിൽസ. മനസ്സ് മോഹവിഭ്രാന്തിയുണ്ടാക്കുന്നു; ജനന മരണാദി ആശയങ്ങളുണ്ടാക്കുന്നു. മാത്രമല്ല, സ്വയം ചിന്തിച്ചുണ്ടാക്കിയ ബന്ധനങ്ങളിൽപ്പെട്ടുഴന്ന് അവസാനം മുക്തിയും പ്രാപിക്കുന്നു.

രാമൻ ചോദിച്ചു: മഹർഷേ, ഇത്ര മഹത്തായ ലോകം മനസ്സിലെങ്ങിനെ നിലകൊള്ളുന്നു? ദയവായി വിശദീകരിച്ചാലും.

വസിഷ്ഠൻ പറഞ്ഞു: അത് ആ ബ്രാഹ്മണകുമാരന്മാർ ഉണ്ടാക്കിയ വിശ്വങ്ങൾ പോലെയാണ്‌.. ലവണരാജാവിന്റെ മായകാഴ്ച്ചകൾ പോലെയാണ്‌.. മറ്റൊരുദാഹരണംകൂടിയുണ്ട്. അത് ശുക്രമുനിയുടെ കഥയാണ്‌.. ഞാനത് വിശദമാക്കാം.

പണ്ട് ഭൃഗുമഹർഷി മലമുകളിൽ തീവ്രമായ ഒരു തപസ്സിലേർപ്പെട്ടിരുന്നു. അദ്ദേഹത്തിന്റെ പുത്രൻ ശുക്രൻ അന്ന് ചെറുപ്പം. അച്ഛൻ ധ്യാനനിമഗ്നനായിരുന്നപ്പോൾ മകൻ അദ്ദേഹത്തിനുവേണ്ട ശുശ്രൂഷകൾ ചെയ്തുവന്നു. യുവാവായ ശുക്രൻ ഒരുദിവസം ആകാശചാരിയായ ഒരപ്സരസ്സിനെ കണ്ടു. അദ്ദേഹത്തിന്റെ മനസ്സ് അവളോടുള്ള ആശയാൽ അസ്വസ്ഥമായി. അപസരസ്സിന്റെ മനസ്സിലും അപ്രകാരമൊരു വികാരം തേജസ്വിയായ മുനികുമാരനെക്കണ്ടപ്പോഴുണ്ടായി. അവളോടുണ്ടായ ഉൾക്കടമായ ആശയാൽ ശുക്രൻ കണ്ണടച്ച് ധ്യാനിച്ച് അവളെ പിന്തുടർന്ന് സ്വർഗ്ഗത്തിലെത്തി. അവിടെ സ്വർഗ്ഗവാസികളായ ദേവഗന്ധർവ്വാദികളെ അദ്ദേഹം കണ്ടു. സ്വർഗ്ഗത്തിലെ ആനകളെയും കുതിരകളെയും കണ്ടു. ബ്രഹ്മാവിനെയും പ്രപഞ്ചത്തെ ഭരിക്കുന്ന മറ്റ് ദേവതമാരെയും അദ്ദേഹമവിടെക്കണ്ടു. സിദ്ധന്മാരെക്കണ്ടു. സ്വർഗീയസംഗീതമാസ്വദിച്ചു. സ്വർഗ്ഗത്തിലെ നന്ദനോദ്യാനങ്ങൾ കണ്ടു. അവസാനം സ്വർഗ്ഗാധിപനായ ഇന്ദ്രനെയും കണ്ടു. അനേകം അതിസുന്ദരികളായ അപ്സരസ്സുകളാൽ പരിലാളിതനായി അദ്ദേഹം ഉന്നതമായൊരു സിംഹാസനത്തിലിരിക്കുന്നു. ശുക്രൻ ഇന്ദ്രനെ അഭിവാദ്യം ചെയ്തു. ഇന്ദ്രൻ സിംഹാസനത്തിൽ നിന്നെണീറ്റ് മുനികുമാരനെ ഉപചാരപൂർവ്വം സീകരിച്ച് കുറച്ചധികം നാളുകൾ സ്വർഗ്ഗത്തിലെ അഥിതിയായി താമസിക്കാൻ അദ്ദേഹത്തെ ക്ഷണിച്ചു. ശുക്രൻ ക്ഷണം സ്വീകരിച്ചു. 

