 
\section{ദിവസം 100}

\slokam{
ഏകസ്യാനേക സംഖ്യസ്യ കസ്യാണോരംബുധേരിവ\\
അന്തര്‍ബ്രഹ്മാണ്ഡലക്ഷാണി ലീയന്തേ ബുദ്ബുദാ ഇവ (3/79/2)\\
}

രാക്ഷസി ചോദിച്ചു:"രാജാവേ, എന്താണ്‌ ഏകമായിരിക്കുമ്പോഴും പലതായിരിക്കുന്നത്‌? ഏതൊന്നിലാണ്‌ കോടിക്കണക്കിനു പ്രപഞ്ചങ്ങള്‍ തിരമാലകള്‍ സമുദ്രജലത്തില്‍ വിലീനമാകുമ്പോലെ വിലീനമാവുന്നത്‌?" കാഴ്ച്ചയില്‍ അങ്ങിനെയല്ലെങ്കിലും, എന്താണ്‌ ശുദ്ധാകാശം?,  എന്നില്‍ ഉള്ള നീയും നിന്നിലുള്ള ഞാനും എന്താണ്‌? എന്താണ്‌ ചലിക്കുന്നുണ്ടെങ്കിലും അചലമായിരിക്കുന്നത്‌? എന്താണ്‌ ബോധത്തില്‍ പാറപോലെ ഉറച്ചിരിക്കുന്നതും എന്നാല്‍ ആകാശത്ത്‌ മാസ്മരിക വിദ്യകള്‍ ലീലയായി ആടുന്നതും? സൂര്യചന്ദ്രന്മാരും അഗ്നിയുമല്ലെങ്കിലും എപ്പോഴും പ്രഭാപൂരമായുള്ളതെന്താണ്‌? ഏറെ അകലെയെന്നുതോന്നുമെങ്കിലും ഏറ്റവും അടുത്തുള്ള ആ അണു എന്താണ്‌? എന്താണ്‌ ബോധ സ്വഭാവമുള്ളതെങ്കിലും അറിയാന്‍ കഴിയാത്ത വസ്തു? എന്താണ്‌ എല്ലാമായിരിക്കുമ്പോഴും യാതൊന്നും അല്ലാതിരിക്കുന്നത്‌? എന്താണ്‌ എല്ലാറ്റിന്റേയും ആത്മസത്തയാണെങ്കിലും അവിദ്യയാല്‍ മൂടിയതും പലേ ജന്മങ്ങളിലെ നിരന്തരവും തീവ്രവുമായ സാധനയിലൂടെ മാത്രം ലഭ്യമാവുന്നത്‌? എന്താണ്‌ അണുരൂപത്തിലിരിക്കുമ്പോഴും മലകളെ ഉള്‍ക്കൊള്ളാന്‍ പോന്ന വസ്തു? എന്തിനാണ്‌ ത്രിലോകങ്ങളേയും വെറും പുല്‍ക്കൊടിയാക്കാന്‍ കഴിയുക? എന്താണ്‌ അപരിമേയമെങ്കിലും അണുസമാനമായുള്ളത്‌? എന്താണ്‌ അണുരൂപത്തിലിരിക്കുമ്പോഴും ഏറ്റവും വലുപ്പമാര്‍ജ്ജിച്ചിരിക്കുന്നത്‌? 

വിശ്വപ്രളയ സമയത്ത്‌ വിത്തില്‍ ചെടിയെന്നപോലെ വിശ്വത്തെമുഴുവന്‍ ഉള്‍ക്കൊള്ളുന്ന പരമാണു എന്താണ്‌? പ്രപഞ്ച ഘടകങ്ങളുടെ പ്രവര്‍ത്തനങ്ങള്‍ക്കെല്ലാം അടിസ്ഥാനമായിരിക്കുമ്പോഴും യാതൊരു കര്‍മ്മവും ചെയ്യാതിരിക്കുന്നതാരാണ്‌? സ്വര്‍ണ്ണാഭരണങ്ങളില്‍ സ്വര്‍ണ്ണമെന്നപോലെ, കാണുന്നവന്‍, കാഴ്ച്ച, കാണല്‍ എന്നീ ത്രിപുടികളുടെ അടിസ്ഥാനമെന്താണ്‌? ഈ ത്രിപുടികളുടെ പ്രത്യക്ഷഭാവത്തെ മൂടിമറച്ച്‌ പിന്നീട്‌ വെളിപ്പെടുത്തുന്ന പ്രതിഭാസമെന്താണ്‌? വിത്തില്‍ മരമെന്നപോലെ എന്തിലാണ്‌ ഭൂതം, വര്‍ത്തമാനം, ഭാവി എന്നീ മൂന്നു കാലങ്ങള്‍ ഉറപ്പിച്ചിരിക്കുന്നത്‌? വിത്തില്‍ നിന്നു മരവും മരത്തില്‍നിന്നു വിത്തും ഇടവിട്ടിടവിട്ട്‌ സംജാതമാകുന്നതുപോലെ എന്താണ്‌ പ്രകടീകൃതമായി മറഞ്ഞ്‌ വീണ്ടും പ്രത്യക്ഷമാവുന്നത്‌? 

രാജാവേ ഈ പ്രപഞ്ചസൃഷ്ടാവെന്താണ്‌? ആരുടെ ശക്തിയിലാണങ്ങു ദുഷ്ടജന ശിക്ഷയും ശിഷ്ടജന പരിരക്ഷയും നടത്തി രാജാവായി വിലസുന്നത്‌? ആരുടെ ദര്‍ശനത്താലാണ്‌ അങ്ങയുടെ ദൃഷ്ടി നിര്‍മ്മലമാവുന്നത്‌? ആരാണ്‌ അങ്ങയുടെ അവിച്ഛിന്നമായ നിലനില്‍പ്പിനു കാരണമായുള്ളത്‌? ജീവനില്‍ കൊതിയുണ്ടെങ്കില്‍ രാജാവേ ഈ ചോദ്യങ്ങള്‍ക്ക്‌ ഉചിതമായ മറുപടിയേകിയാലും. അങ്ങയുടെ ജ്ഞാനത്തിന്റെ പ്രഭയില്‍ എന്റെ സംശയത്തിന്റെ ഇരുട്ട്‌ നീങ്ങട്ടെ. അവിദ്യയുടേയും സംശയത്തിന്റേയും വേരറുക്കാനുതകുന്ന ഉത്തരം നല്‍കാത്തവന്‍ ജ്ഞാനിയല്ല. അങ്ങേയ്ക്കെന്റെ സംശയങ്ങള്‍ ദൂരീകരിക്കാനായില്ലെങ്കില്‍ എന്റെ വിശപ്പടക്കാനാവും.
