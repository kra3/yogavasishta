\newpage
\section{ദിവസം 079}

\slokam{
ഇതി സര്‍വ്വ ശരീരേണ ജംഗമത്വേന  ജംഗമം സ്ഥാവരം\\
സ്ഥാവരത്വേന സര്‍വാത്മാ ഭാവയന്‍ സ്ഥിത: (3/5/54)\\
}

സരസ്വതി തുടര്‍ന്നു: ഈ പരേതാത്മാക്കളെല്ലാം അവരുടെയുള്ളില്‍ പൂര്‍വ്വകര്‍മ്മങ്ങളുടെ ഫലങ്ങള്‍ അനുഭവിക്കുന്നു. ആദ്യം 'ഞാന്‍ മരിച്ചു' എന്നും പിന്നീട്‌ 'എന്നെ യമദൂതന്മാര്‍ കൂട്ടിക്കൊണ്ടുപോവുന്നു' എന്നുമുള്ള തോന്നലുകള്‍ അവര്‍ക്കുണ്ടാവുന്നു. അവരില്‍ ധര്‍മ്മിഷ്ഠരായവര്‍ കരുതുന്നു തങ്ങളെ സ്വര്‍ഗ്ഗത്തിലെയ്ക്കാണു കൊണ്ടുപൊവുന്നതെന്ന്. സാധാരണക്കാരും പാപഭീതിയുള്ളവരും തങ്ങളെ വിചാരണയ്ക്കായി ചിത്രഗുപ്തനുമുന്നിലേയ്ക്കാണു നയിക്കുന്നതെന്ന് അനുമാനിക്കുന്നു. ചിത്രഗുപ്തന്റെ കയ്യില്‍ പരേതന്റെ പൂര്‍വ്വകര്‍മ്മങ്ങളുടെ നിഗൂഢചരിത്രമെഴുതിയ പുസ്തകമുണ്ട്‌. ജീവാത്മാവ്‌ എന്തുകാണുന്നുവോ അതനുഭവമാകുന്നു. 

ഈനിശ്ശൂന്യമായ അനന്താവബോധത്തില്‍ കാലം, കര്‍മ്മമെന്നിങ്ങനെ യാതൊന്നുമില്ല. എന്നാല്‍ ജീവന്‍ ഇങ്ങിനെ ചിന്തിക്കുന്നു:'മരണദേവന്‍ എന്നെ സ്വര്‍ഗ്ഗത്തിലെയ്ക്ക്‌, അല്ലെങ്കില്‍ നരകത്തിലേയ്ക്കയച്ചു. അവിടെ ഞാന്‍ സുഖം അല്ലെങ്കില്‍ ദുരിതം അനുഭവിച്ചു.യമന്റെ ആജ്ഞപ്പടി ഞാന്‍ ഒരു മൃഗമായി ജനിച്ചു.' ആ ക്ഷണത്തില്‍ ജീവന്‍ പുരുഷശരീരത്തില്‍ അവന്‍ കഴിക്കുന്ന ഭക്ഷണത്തിലൂടെ പ്രവേശിക്കുന്നു.അതുപിന്നെ സ്ത്രീശരീരത്തിലേയ്ക്ക്‌ കടത്തിവിട്ട്‌ കാലക്രമത്തില്‍ മറ്റൊരു ശരീരമായി പുറത്തുവരുന്നു. അതു വീണ്ടും തന്റെ പൂര്‍വ്വാര്‍ജ്ജിതകര്‍മ്മഫലങ്ങളനുസരിച്ച്‌ ജീവിതം നയിക്കുന്നു. അവിടെയവന്‍ ചന്ദ്രന്റെ വൃദ്ധിക്ഷയങ്ങള്‍ പോലെ വര്‍ത്തിച്ച്‌ ഒരിക്കല്‍കൂടി ജരാനരകള്‍ക്കു വശംവദനായി മരണത്തെ പ്രാപിക്കുന്നു. ആത്മജ്ഞാനപ്രബുദ്ധത നേടുന്നതുവരെ ജീവാത്മാവ്‌ ഈ യാത്ര തുടരുന്നു.

പ്രബുദ്ധയായ ലീല ചോദിച്ചു: ദേവീ, ഇതെല്ലാം ആദ്യമെങ്ങിനെ തുടങ്ങിയെന്ന് ദയവായി പറഞ്ഞുതന്നാലും.

സരസ്വതി പറഞ്ഞു: മലകളുംകാടുകളും ഭൂമിയും അകാശവുമെല്ലാം അനന്ത ബോധമല്ലാതെ മറ്റൊന്നുമല്ല. അതുമാത്രമാണ്‌ ഉണ്മയായുള്ളത്‌. എല്ലാറ്റിന്റേയും സത്ത അതാണ്‌. എന്നാല്‍ ആ സത്ത സ്വയം പ്രകടമായപ്പോള്‍ അതു സ്വാംശീകരിച്ച രൂപഭാവങ്ങള്‍ അങ്ങിനെതന്നെ പ്രത്യക്ഷമായി കാണപ്പെട്ടു.അതിപ്പോഴും തുടരുകയും ചെയ്യുന്നു. ശരീരങ്ങളില്‍ (പദാര്‍ത്ഥസംഘാതങ്ങളില്‍ ) പ്രാണവായു പ്രവേശിച്ച്‌ വിവിധ ഭാഗങ്ങളില്‍ സ്പന്ദനങ്ങള്‍  തുടങ്ങുമ്പോള്‍ ആ ശരീരങ്ങള്‍ക്ക്‌ 'ജീവനുണ്ട്‌ ' എന്നു പറയുന്നു. അത്തരം ജീവികള്‍ സൃഷ്ടിയുടെ സമാരംഭം മുതല്‍ നിലവിലുണ്ട്‌. പ്രാണവായുവിന്റെ സാന്നിദ്ധ്യത്തിലും വേണ്ടത്ര തീവ്രമായി സ്പന്ദനമുണ്ടാകാതിരുന്ന ശരീരങ്ങള്‍ മരങ്ങളും ചെടികളുമായി. അവയിലും ഒരു ചെറിയ ബോധതലം ലീനമായി മരുവുന്നുണ്ട്‌. അതാണ്‌ ആ ശരീരങ്ങളുടെ പ്രജ്ഞക്കടിസ്ഥാനം. ഈ പ്രജ്ഞ ശരീരത്തില്‍ പ്രവേശിച്ച്‌ കണ്ണ് മുതലായ അവയവങ്ങള്‍ക്ക്‌ ചൈതന്യമേകുന്നു. ഈ ബോധത്തിന്റെ സങ്കല്‍പ്പമനുസരിച്ച്‌ അപ്രകാരം ഓരോന്നും ആയിത്തീരുകയാണ്‌. "അങ്ങിനെ ഈ ആത്മാവ്‌ എല്ലാ 'ശരീരങ്ങളിലും' നിലകൊള്ളുന്നു. അത്‌ ജംഗമവസ്തുക്കളില്‍ അവയുടെ ചലനാത്മകതയായും സ്ഥാവരവസ്തുക്കളില്‍ അചലാവസ്ഥയായും സ്വഭാവം പ്രകടിപ്പിക്കുന്നു." അങ്ങിനെ എല്ലാ ശരീരങ്ങളുമിപ്പോഴും ഇതു തുടരുന്നു.
