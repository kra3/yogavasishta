\section{ദിവസം 260}

\slokam{
രാമാ പര്യവസാനേയം മായാ സംസൃതിനാമികാ\\
ആത്മചിത്തജയേനൈവ ക്ഷയമായാതി നാന്യഥാ (5/44/1)\\
}

വസിഷ്ഠന്‍ തുടര്‍ന്നു: “രാമാ, ഈ ജനനമരണചക്രം അവസാനമില്ലാത്ത ഒന്നാണ്. മനസ്സിനെ പൂര്‍ണ്ണമായും വരുതിയില്‍ നിര്‍ത്തിയാല്‍ മാത്രമേ ഈ മായക്കാഴ്ച്ചയ്ക്ക് അവസാനമുണ്ടാവുകയുള്ളു.” 

ഇത് വ്യക്തമാക്കാന്‍ ഞാന്‍ ഒരു കഥ പറയാം. കോസലം എന്നൊരു രാജ്യത്ത് ഗാധി എന്നുപേരായ ഒരു ബ്രാഹ്മണന്‍ ഉണ്ടായിരുന്നു. അതീവ ധര്‍മ്മിഷ്ഠനും പണ്ഡിതനുമായിരുന്നു ഗാധി. ചെറുപ്പത്തിലെ തന്നെ നിര്‍മമത, സന്യാസത്തോടുള്ള അഭിനിവേശം തുടങ്ങിയ ഗുണങ്ങള്‍ അദ്ദേഹം പ്രകടിപ്പിച്ചു. തപസ്സനുഷ്ഠിക്കാനായി അദ്ദേഹം ഒരിക്കല്‍ കാട്ടില്‍പ്പോയി. വിഷ്ണുപ്രീതിയ്ക്കായി അവിടെയൊരു നദിയിലിറങ്ങിനിന്ന് മന്തോച്ചാരണങ്ങളോടെ അദ്ദേഹം തപം ചെയ്ത് മനസ്സിനെ പരിശുദ്ധമാക്കി.

എട്ടു മാസങ്ങളങ്ങിനെ കഴിഞ്ഞപ്പോള്‍ വിഷ്ണു ഭഗവാന്‍ പ്രത്യക്ഷപ്പെട്ടു. ‘നിനക്കെന്തു വരമാണ് വേണ്ടത്?’ ഭഗവാന്‍ ചോദിച്ചു.

ഗാധി പറഞ്ഞു: ഭഗവാനെ, എല്ലാ ജീവജാലങ്ങളെയും അവിദ്യയില്‍ അടക്കി നിര്‍ത്തുന്ന മായാശക്തിയെ എനിക്കൊന്നു നേരില്‍ക്കാണണം. ഭഗവാന്‍ പറഞ്ഞു: നീ മായയെ കണ്ടുകഴിഞ്ഞാല്‍പ്പിന്നെ നിന്റെയുള്ളിലെ മായക്കാഴ്ചകളും വിഷയ-വസ്തുബോധവും ധാരണകളുമെല്ലാം നീ ക്ഷണത്തിലുപേക്ഷിക്കും. ഭഗവാന്‍ അപ്രത്യക്ഷനായി. ഗാധി വെള്ളത്തില്‍ നിന്നും എഴുന്നേറ്റു. അദേഹം സംപ്രീതനും സന്തുഷ്ടനുമായിരുന്നു.

പിന്നീട് കുറേക്കാലം ഗാധി പലവിധത്തിലുള്ള പവിത്ര കര്‍മ്മങ്ങളിലേര്‍പ്പെട്ടിരുന്നു. വിഷ്ണുഭഗവാന്റെ ദര്‍ശനത്തില്‍ നിന്നുണ്ടായ ആനന്ദാതിരേകം അദ്ദേഹത്തില്‍ നിറഞ്ഞു നിന്നിരുന്നു. ഭഗവാന്റെ വാക്കുകളെ മനനം ചെയ്തുകൊണ്ട് ഒരുദിനം അദ്ദേഹം കുളിക്കാനായി നദിക്കരയില്‍ച്ചെന്നു. ജലത്തില്‍ മുങ്ങവേ അദ്ദേഹം താന്‍ മരിച്ചുപോയതായും ചുറ്റും ആളുകള്‍ ശോകാര്‍ത്തരായി കരയുന്നതായും കണ്ടു. ശരീരം ചേതനയറ്റ് വിളറിവീണു പോയിരുന്നു. അനേകം ബന്ധുമിത്രാദികള്‍ ചുറ്റും നിന്നു വിതുമ്പുന്നു. ഭാര്യയുടെ കണ്ണീര്‍ അണപൊട്ടിയതുപോലെ ഒഴുകുന്നു. അവള്‍ കാലില്‍ കെട്ടിപ്പിടിച്ചു കരയുമ്പോള്‍ അമ്മ തന്റെ മുഖം ചേര്‍ത്തു പിടിച്ചു തേങ്ങുന്നു. അനേകം ബന്ധുക്കള്‍ ചുറ്റിലും ദു:ഖാര്‍ത്തരായി വിലപിക്കുന്നു. താനവരുടെ മദ്ധ്യത്തില്‍ ദീര്‍ഘ നിദ്രയിലോ ധ്യാനത്തിലോ എന്നവണ്ണം പ്രശാന്തനായി കിടക്കുന്നതും അദ്ദേഹം കണ്ടു. വിലാപങ്ങളെല്ലാം കേട്ട് എന്താണിതെന്റെ അര്‍ത്ഥം എന്നദേഹം അത്ഭുതപ്പെട്ടു. ബന്ധുതയുടെയും മിത്രഭാവത്തിന്റെയും സവിശേഷതകളെപ്പറ്റി അദ്ദേഹം ആലോചിച്ചു. താമസംവിനാ ശരീരം ശ്മശാനത്തിലേയ്ക്ക് ദാഹിപ്പിക്കാനായി കൊണ്ടുപോയി. ചടങ്ങുകള്‍ കഴിഞ്ഞപ്പോള്‍ ശരീരം ചിതയിലേയ്ക്ക് വച്ചു. ചിതയ്ക്ക് തീകൊളുത്തിയതോടെ അഗ്നി അതിനെ ആര്‍ത്തിയോടെ വിഴുങ്ങി. ഗാധിയുടെ ശരീരത്തിനന്ത്യമായി.