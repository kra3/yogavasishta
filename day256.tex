\section{ദിവസം 256}

\slokam{
ഇദം സുഖ മിദം ദു:ഖമിദം നാസ്തീദമസ്തി മേ\\
ഇതി ദോളായിതം ചേതോ മുഢമേവ ന പണ്ഡിതം (5/41/12)\\
}

പ്രഹ്ലാദന്‍ പറഞ്ഞു: ഭഗവാനേ തളര്‍ച്ച കൊണ്ട് ഞാന്‍ ഒന്ന് മയങ്ങിപ്പോയി. അങ്ങയുടെ കൃപയാല്‍ ധ്യാനത്തിനും ധ്യാനത്തിലല്ലാത്ത അവസ്ഥയ്ക്കും തമ്മില്‍ അന്തരമൊന്നുമില്ലെന്നു എനിക്കറിവുണര്‍ന്നിരിക്കുന്നു.

അങ്ങയെ എന്റെയുള്ളില്‍ ഞാന്‍ ഏറെക്കാലമായി കണ്ടുകൊണ്ടിരിക്കുന്നു. ഭാഗ്യാതിരേകമെന്നു പറയട്ടെ, ഇപ്പോള്‍ ഇതാ അങ്ങെന്റെ മുന്നിലും വന്നിരിക്കുന്നു. ഈ പ്രത്യക്ഷലോകത്തെ ഭയപ്പെടാതെ, ശരീരമുപേക്ഷിക്കാന്‍ ആശയൊന്നുമില്ലാതെ, അനാസക്തിയെപ്പറ്റി ആകുലതകളില്ലാതെ, ഭ്രമചിന്തകളോ ദു:ഖങ്ങളോ അലട്ടാതെ, ഞാനാ അനന്താവബോധം എന്ന സത്യത്തെ അനുഭവിച്ചറിഞ്ഞിരിക്കുന്നു. ആ എകാത്മകതയെ അറിഞ്ഞാല്‍പ്പിന്നെ ദു:ഖമെവിടെ? നാശമെവിടെ? ശരീരവും കാണപ്പെടുന്ന ഈ ലോകവുമെവിടെ? അവ നഷ്ടപ്പെട്ടാലുണ്ടാവുന്ന ഭയമെവിടെ? പൊടുന്നനെ എന്നിലുളവായ ആ അഭൗമബോധതലത്തിലായിരുന്നു ഞാനിത്രയും നാള്‍ .

‘ഈ ലോകമെത്ര നികൃഷ്ടം! ഞാനിതുപേക്ഷിക്കാന്‍ പോവുന്നു’ എന്ന് ചിന്തിക്കുന്നത് അജ്ഞാനികള്‍ മാത്രം. ശരീരമുള്ളപ്പോള്‍ ദു:ഖമുണ്ടെന്നും അതില്ലാത്തപ്പോള്‍ ദു:ഖനിവൃത്തിയായെന്നും ഉള്ള ചിന്തയും അജ്ഞത തന്നെയാണ്. “ഇത് സുഖം, ഇത് ദു:ഖം, ഇതതാണ്, ഇതതല്ല, എന്നെല്ലാമുള്ള ചാഞ്ചാട്ടം അജ്ഞാനിയുടെ ഉള്ളില്‍ മാത്രമേയുള്ളു. ജ്ഞാനിക്കതില്ല.” 

‘ഞാന്‍’, ‘മറ്റുള്ളവര്‍’ എന്ന ധാരണകള്‍  വിവേകം ഇല്ലാതെ അജ്ഞാനികളായി തുടരുന്നവരില്‍ മാത്രമേയുള്ളു. അതുപോലെ ‘ഇതെനിക്ക് നേടണം’, ഇതെനിക്ക് ത്യജിക്കണം’ എന്നീ ചിന്തകളും അജ്ഞാനിക്കുണ്ട്. എല്ലാടവും നീ നിറഞ്ഞു വിളങ്ങുമ്പോള്‍ ‘മറ്റെന്താണ്’ നേടാനും കളയാനുമുള്ളത്? വിശ്വം മുഴുവനും നിറഞ്ഞുനില്‍ക്കുന്നത് ബോധമാണ്. അപ്പോള്‍പ്പിന്നെ എന്താണു 'നേടാനും കളയാനു'മുള്ളത്? ഞാന്‍ എന്നില്‍ത്തന്നെ ഇപ്രകാരമുള്ള മനനത്തോടെ ധ്യാനത്തില്‍  അഭിരമിക്കുകയായിരുന്നു. ഭാവാഭാവങ്ങളില്ലാതെ ത്യാജ്യ-ഗ്രാഹ്യ ധാരണകളില്ലാതെ ഞാനല്‍പ്പനേരം വിശ്രമത്തിലായിരുന്നു. എന്നില്‍ ആത്മജ്ഞാനം ഉണര്‍ന്നിരിക്കുന്നു. അങ്ങയെ പ്രസാദിപ്പിക്കാനായി എന്തുംചെയ്യാന്‍ ഞാനിതാ ഒരുങ്ങിനില്‍ക്കുന്നു. എന്റെ പ്രാര്‍ത്ഥന കൈക്കൊണ്ടാലും.

പ്രഹ്ലാദന്റെ പൂജയ്ക്ക് ശേഷം ഭഗവാന്‍ ഇങ്ങിനെ അരുളി: എഴുന്നേല്‍ക്കൂ പ്രഹ്ലാദാ, ദേവന്മാരും മാമുനിമാരും നിന്നെ വാഴ്ത്തുന്ന ഈ അവസരത്തില്‍ നിന്നെ പാതളചക്രവര്‍ത്തിയായി ഞാന്‍ അവരോധിക്കുന്നു. സൂര്യച്ന്ദ്രന്മാരുള്ളിടത്തോളം കാലം നീയീ പദവി അലങ്കരിച്ചാലും. കാമക്രോധലോഭാദികളുടെ വരുതിയില്‍പ്പെടാതെ സമതാഭാവത്തോടെ ഇവിടം സംരക്ഷിച്ചു വസിച്ചാലും. ദേവലോകത്തും മനുഷ്യലോകത്തും ആവശ്യമില്ലാത്ത ആശങ്കകളൊന്നുമുണ്ടാക്കാതെ ഐശ്വര്യരാജഭോഗങ്ങളെല്ലാം ആസ്വദിച്ചു ജീവിച്ചാലും.

ആലോചനകളിലും ഉദ്ദേശലക്ഷ്യങ്ങളിലും കാലം കളയാതെ ഉചിതമായ കര്‍മ്മങ്ങളിലേര്‍പ്പെടുമ്പോള്‍ അവ ‘കര്‍മ്മ’ങ്ങളല്ല. അവ നിന്നെ ബന്ധിക്കുകയില്ല. നിനക്കെല്ലാം അറിയാവുന്നതാണ്. ഞാനിനി പറഞ്ഞുതരേണ്ടതായി ഒന്നുമില്ല. ഇനിമുതല്‍ ദേവന്മാരും അസുരന്മാരും സൌഹൃദത്തില്‍ കഴിയുക. ദേവത്വവും അസുരത്വവും രമ്യതയിലാവട്ടെ. രാജന്‍, അജ്ഞാനത്തെ അടുപ്പിക്കാതെ അകലത്തു നിര്‍ത്തുക. എന്നിട്ട് പ്രബുദ്ധതയോടെ ഏറെക്കാലം ഈ ലോകം ഭരിച്ചാലും. 
