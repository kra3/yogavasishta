\section{ദിവസം 202}

\slokam{
ഹേ ജനാ അപരിജ്ഞാത ആത്മാ വോ ദു:ഖസിദ്ധയേ\\
പരിജ്ഞാതസ്ത്വനന്തായ സുഖായോപശമായ ച (5/5/23)\\
}

വസിഷ്ഠൻ പറഞ്ഞു: രാമാ, വിശ്വപ്രളയത്തെക്കുറിച്ചും പരമശാന്തിയടയുന്നതിനെക്കുറിച്ചുമുള്ള ഈ പ്രഭാഷണം കേട്ടാലും. ഈ പ്രത്യക്ഷലോകത്തെ താങ്ങി നിലനിർത്തുന്നത് രാജസീകതയും (മലിനത) താമസീകതയും (മന്ദത) ആണ്‌...  ഒരു കെട്ടിടത്തെ താങ്ങി നിർത്തുന്ന സ്തംഭങ്ങൾ പോലെയാണവ. ഒരു പാമ്പ് പടം പൊഴിക്കുന്നതുപോലെ നിർമ്മലരായവർ ഈ ലോകത്തെ നിഷ് പ്രയാസം  ഉപേക്ഷിക്കുന്നു. സാത്വീകതയും (നിർമ്മലത) കർമ്മപടുത്വവും (രാജസീയത) ഉള്ളവർ യാന്ത്രികമായി വെറുതേ ജീവിക്കുന്നവരല്ല. അവർ ഈ ലോകത്തിന്റെ ഉത്ഭവത്തെപ്പറ്റിയും പ്രകടനാത്മകതയെപ്പറ്റിയും കൂലങ്കഷമായി ആലോചിച്ച് അന്വേഷിക്കുന്നവരാണ്‌..  ഈ അന്വേഷണങ്ങൾ ശരിയായ ശാസ്ത്രപഠനങ്ങളിലൂടെയും മഹത്തുക്കളായ ഗുരുവരന്മാരുടെ സഹായത്താലും നടത്തുന്ന പക്ഷം അവരിൽ സത്യസാക്ഷാത്കാരമുണ്ടാവും. സത്യം വിളക്കിന്റെ വെളിച്ചത്തിലെന്നവണ്ണം അവർക്കു സ്വരൂപമായി തെളിഞ്ഞു കാണാകുന്നു. സ്വാത്മാവിൽ ആത്മപ്രയത്നത്താൽ സ്വയം അന്വേഷിച്ചു കണ്ടെത്തിയാലല്ലാതെ ഈ സത്യം തെളിയുകയില്ല.

രാമാ, നീ തീർച്ചയായും ശുദ്ധനൈർമ്മല്യം തന്നെയാണ്‌..  അതിനാൽ നീ സത്യാന്വേഷണം ചെയ്താലും. സത്യവും മിഥ്യയുമെന്തെന്നാരാഞ്ഞ് സത്യത്തിനായി സ്വയം സമർപ്പിക്കുക. ആദ്യമേ തന്നെ ഇല്ലാത്തതും കുറേക്കാലത്തിനപ്പുറം നിലനിൽക്കാത്തതുമായതിനെ എങ്ങിനെയാണു സത്യമെന്നെണ്ണുക? എന്നെന്നും ഉണ്ടായിരുന്ന, നിലനില്‍ക്കുന്ന ഒന്നിനെ മാത്രമേ സത്യമെന്നു കരുതാനാവൂ. ജനനം എന്നതും, വളർച്ചയെന്നതും മനസ്സിനാണു രാമാ. ഈ സത്യം തെളിഞ്ഞുറച്ചാല്‍പ്പിന്നെ മനസ്സാണ്‌ സ്വയമുണ്ടാക്കിയ പരിമിതിയായ അജ്ഞാനത്തിൽ നിന്നും സ്വതന്ത്രമാവുന്നത്. അതുകൊണ്ട് മനസ്സിനെ ശാസ്ത്രപഠനങ്ങളാകുന്ന തയ്യാറെടുപ്പോടെ  ധർമ്മത്തിന്റെ പാതയിലേയ്ക്കു നയിക്കുക. സദ്സംഗം, അനാസക്തി പരിശീലനം എന്നിവയും ഇതിനുള്ള തയ്യാറെടുപ്പുതന്നെയാണ്‌..  ഇങ്ങിനെ സ്വയം പക്വതയാർജ്ജിച്ച ശേഷം പരിപൂർണ്ണ ജ്ഞാനിയായ ഒരു ഗുരുവിന്റെ പാദങ്ങളെ ശരണം പ്രാപിക്കുക. ഗുരു അരുളിത്തരുന്ന ശാസ്ത്രപാഠങ്ങൾ ശ്രദ്ധാഭക്തിവിശ്വാസങ്ങളോടെ പിന്തുടരുന്നതിലൂടെ കാലക്രമത്തിൽ ശിഷ്യൻ പരിപൂർണ്ണ പരിശുദ്ധിയെ പ്രാപിക്കുന്നു.

രാമാ, ഈ നിർമ്മലതയിൽ ആത്മാവിനെ ആത്മാവുകൊണ്ടറിയുക. ചന്ദ്രന്റെ ശീതളിമയാര്‍ന്ന പ്രഭ  ആകാശത്തെ മുഴുവൻ കാണിച്ചുതരുന്നുവല്ലോ. മായികമായ സംസാരസമുദ്രത്തിൽ ഒരു വയ്ക്കോൽത്തുരുമ്പുപോലെ നമ്മുടെ ജീവിതം തട്ടിക്കളിച്ചുകൊണ്ടിരിക്കുന്നു. അതിനൊരവസാനം വരണമെങ്കിൽ ആത്മാന്വേഷണമെന്ന വഞ്ചിയിൽ കയറുക എന്ന ഒരൊറ്റ പോംവഴിയേയുള്ളു. നദികളിൽ പൊങ്ങിയൊഴുകി നീങ്ങുന്ന മണൽത്തരികൾ അടിയുന്നത് പ്രശാന്തമായ ജലത്തിലാണല്ലോ. ആത്മവിദ്യാനിരതനായ ഒരുവന്റെ മനസ്സിൽ സത്യം പ്രശാന്തതയോടെ അറിവായുറയ്ക്കുന്നു. ഒരിക്കലുറച്ചാൽ നശിക്കാത്ത അറിവാണത്. ചാരം മൂടിക്കിടക്കുന്നുവെങ്കിലും ഒരുകഷണം സ്വർണ്ണം കണ്ടെത്താൻ സ്വർണ്ണപ്പണിക്കാരന്‌ വിഷമങ്ങളേതുമില്ല. സത്യം കണ്ടെത്തുംവരെ മാത്രമേ ചിന്താക്കുഴപ്പങ്ങളുള്ളു. സത്യത്തിന്റെ അറിവുറച്ചാൽപ്പിന്നെ എല്ലാം വ്യക്തമായി തെളിഞ്ഞുകാണാം. “ആത്മാവിനെക്കുറിച്ചുള്ള അജ്ഞതയാണ്‌ നിന്റെ ദു:ഖത്തിനു കാരണം. എന്നാൽ ആത്മാവിനെക്കുറിച്ചുള്ള ജ്ഞാനം നിനക്കു പ്രശാന്തിയും പ്രഹർഷവുമേകും.” 

