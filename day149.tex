\section{ദിവസം 149}

\slokam{
സ്വയമൂർധ്വം പ്രയാത്യഗ്നി: സ്വയം യാന്തി പയാംസ്യധ:\\
ഭോക്താരം ഭോജനം യാതി സൃഷ്ടിം ചാപ്യന്തക: സ്വയം (4/10/29)\\
}

വസിഷ്ഠൻ തുടർന്നു: ഒരുനൂറു ദിവ്യവർഷങ്ങൾ തപസ്സിലിരുന്നശേഷം ഭൃഗുമഹർഷി തന്റെ ആസനത്തിൽ നിന്നും എഴുന്നേറ്റു. തന്റെ മുന്നിലിരുന്ന മകൻ ശുക്രനെ അദ്ദേഹം കണ്ടില്ല. എന്നാൽ അവിടെ ഉണങ്ങിവരണ്ട് ഒരു ശരീരം കണ്ടു. അതിനികൃഷ്ടമായ ഒരു ദേഹമായിരുന്നു അത്. കൺകുഴികളിൽ പുഴുക്കൾ ഞുളച്ചുപെരുകുന്നു. ഈ കാഴ്ച്ചയിൽ അതിവികാരാധീനനായി ക്രുദ്ധനായ മഹർഷി കാര്യവിചാരം ഏറെയൊന്നും ചെയ്യാതെ തന്റെ മകന്റെ അകാലമരണത്തിനിടയാക്കിയ കാലനെ ശപിക്കാൻ തീരുമാനിച്ചു.

കാലൻ ഉടനേതന്നെ മഹർഷിക്കുമുന്നിൽ പ്രത്യക്ഷപ്പെട്ടു. ഒരുകയ്യിൽ വാളും മറ്റേക്കയ്യിൽ കയറുമായാണ്‌ കാലന്റെ വരവ്. പ്രതിരോധിക്കാനാവാത്ത കവചമാണദ്ദേഹത്തിന്റെ വസ്ത്രം. ആറുകരങ്ങളും ആറുമുഖങ്ങളുമാണദ്ദേഹത്തിന്‌.. ചുറ്റും സേവകരും ദൂതന്മാരുമുണ്ട്. കയ്യിലെ ആയുധങ്ങളിൽ നിന്നും ദേഹത്തുനിന്നും നാശത്തിന്റെ തീനാളങ്ങൾ പൊങ്ങുന്നു.

ശാന്തനായി, ദൃഢസ്വരത്തിൽ കാലൻ ഭൃഗുവിനോടു പറഞ്ഞു: മഹർഷേ, അങ്ങയേപ്പോലെ ഉന്നതനായ ഒരു ഋഷി ഇങ്ങിനെ അനുചിതമായൊരു  ശാപത്തിനൊരുങ്ങാനെന്തേ?  ജ്ഞാനികൾ അവരെ അപമാനിച്ചാൽപ്പോലും ചഞ്ചലചിത്തരാകയില്ല. അങ്ങയെ ആരും അപമാനിക്കാതെ തന്നെ അങ്ങയുടെ സമനില തെറ്റിയിരിക്കുന്നു. അങ്ങ് ബഹുമാന്യനായ ഒരാളാണ്‌.. ഞാനാണെങ്കില്‍  ഉചിതമായും അനുയോജ്യമായും മാത്രം പ്രവർത്തിക്കുന്നയാളുമാണ്‌.. ആയതിനാൽ ഞാനങ്ങയെ നമസ്കരിക്കുന്നു - മറ്റൊരുദ്ദേശത്തിലുമല്ല ഈ നമസ്കാരം. ഈ വ്യർത്ഥമായ ശാപത്തിലൂടെ അങ്ങാർജ്ജിച്ച തപോബലം നശിപ്പിക്കാതിരിക്കൂ. വിശ്വപ്രളയാഗ്നിക്കുപോലും എന്നെ നശിപ്പിക്കാനാവില്ല. എന്നെ ശപിച്ചില്ലാതാക്കാമെന്ന ആഗ്രഹം എത്ര ബാലിശം!

ഞാൻ കാലമാണ്‌.. വിശ്വപാലകന്മാരായ ദേവതമാരെയടക്കം എത്രയോപേരെ ഞാൻ ഇല്ലാതാക്കിയിരിക്കുന്നു. നിങ്ങളെല്ലാം എനിയ്ക്കുള്ള ഭക്ഷണം മാത്രം. ഇത് പ്രകൃതിനിയമമാണെന്നറിയുക. പരസ്പരമുള്ള ഇഷ്ടാനിഷ്ടങ്ങളെ ആസ്പദമാക്കിയുള്ളതല്ല ഈ ബന്ധം. ”തീയിന്റെ സഹജസ്വഭാവമാണ്‌ അതിന്റെ നാളം മേലോട്ടുയരുകയെന്നത്. അതുപോലെ ജലപ്രവാഹം തഴോട്ട്. അഹാരവസ്തുക്കൾ ആഹരിക്കുന്നവനെ തേടുന്നു. ഉണ്ടായ/ഉണ്ടാക്കിയ എല്ലാ വസ്തുക്കൾക്കും അവസാനവുമുണ്ട്.“ ഈശ്വരനിശ്ചയമാണിത്. എല്ലാറ്റിന്റേയും ആത്മസത്തയുടെ, സ്വയം നിലകൊള്ളുന്ന ആത്മാവിന്റെ സഹജഭാവം. ശുദ്ധദൃഷ്ടിയിൽ കർത്താവോ ഭോക്താവോ ഇല്ല. കളങ്കമുള്ള  ദൃഷ്ടിയിലാണ്‌ ഇത്തരം വിഭജനങ്ങളുള്ളത്. അങ്ങ് ജ്ഞാനിയല്ലേ? സത്യമറിയുന്നയാളല്ലേ? കർത്തൃത്വവും ഭോക്ത്തൃത്വവും മിഥ്യയാണെന്ന് ഞാനങ്ങേയ്ക്കു പറഞ്ഞു തരേണ്ടതില്ലല്ലോ.

ജീവികൾ, മരങ്ങളിലെ പൂക്കളും കായ്കളും പോലെ, വന്നും പോയുമിരിക്കും. അവയുടെയെല്ലാം കാരണം അഭ്യൂഹങ്ങൾ മാത്രമാണ്‌.. എല്ലാം കാലനിബദ്ധം. അവ ഉണ്മയാണെന്നും അല്ലെന്നും പറയാം. തടാകജലത്തിൽ പ്രതിബിംബിക്കുന്ന ചന്ദ്രന്‌ ഇളക്കമുള്ളതായി തോന്നുന്നത് ജലോപരിയുണ്ടാവുന്ന ചലനം മൂലമാണല്ലോ. അത് സത്യമെന്നും അല്ലെന്നും വാദിക്കാം. 

