\section{ദിവസം 170}

\slokam{
ചിദാകാശോഹമിത്യേവ രജസാ രഞ്ജിതപ്രഭ:\\
സ്വരൂപമത്യജന്നേവ വിരൂപമപി ബുദ്ധ്യതേ (4/32/31)\\
}

രാമൻ ചോദിച്ചു: മഹാമുനേ ഈ മൂന്നു രാക്ഷന്മാർക്ക് എപ്പോൾ എവിടെവെച്ചാണ്‌ മോക്ഷം ലഭിക്കുക?

വസിഷ്ഠൻ പറഞ്ഞു: അവർ അവരുടെ തന്നെ പൂർവ്വ കഥ കേൾക്കാനിടവരികയും തങ്ങളുടെ സ്വരൂപമെന്തെന്ന ഓർമ്മ - അനന്താവബോധമാണെന്ന സത്യം- അവരിൽ അങ്കുരിപ്പിക്കുകയും ചെയ്യുമ്പോൾ അവർക്കു മോക്ഷമാവും. കാശ്മീരത്തിൽ അധിഷ്ഠാന എന്നുപേരായ ഒരു നഗരം വളർന്നുയരും. അതിനുമധ്യത്തിലായി പ്രദ്യുമ്ന എന്നുപേരായ കൊടുമുടിയുള്ള ഒരു മലയുമുണ്ടാവും. അതിനും മുകളിൽ അംബരചുംബിയായ ഒരു കെട്ടിടം. അതിലെ ഒരു മൂലയ്ക്ക് വ്യാളാസുരൻ ഒരു ചെറുകിളിയായി ജന്മമെടുക്കും. ആ കൊട്ടാരത്തിലാണ്‌ യശസ്കരൻ എന്ന രാജാവു താമസിക്കുന്നത്. ആ കൊട്ടാരത്തിന്റെ തൂണുകളിലൊന്നിലെ ഒരു പൊത്തിൽ കൊതുകായിട്ടാണ്‌ ദാമാസുരൻ ജനിക്കാൻ പോകുന്നത്. നഗരത്തിലെ മറ്റൊരിടത്ത് രത്നാവലിവിഹാരം എന്നു പേരായ കൊട്ടാരത്തിൽ മുഖ്യമന്ത്രിയായ നരസിംഹൻ വസിക്കുന്നു. മൂന്നാമത്തെ അസുരൻ - കടാസുരൻ ആ കൊട്ടാരത്തിൽ ഒരു മൈനയായി ജനിക്കും.

ഒരു ദിവസം നരസിംഹൻ ഈ മൂന്നു രാക്ഷസന്മാരുടെ - ദാമൻ, വ്യാളൻ, കടൻ എന്നിവരുടെ കഥ പറയുന്നതു കേൾക്കുമ്പോൾ മൈനയ്ക്ക് ബോധോദയമുണ്ടാവും. തന്റെ പൂർവ്വരൂപം ശംഭരൻ മായാവിദ്യകൊണ്ടുണ്ടാക്കിയതാണെന്ന് തിരിച്ചറിയുന്ന നിമിഷം അവൻ ആ മായാവലയത്തിൽ നിന്നും വിമുക്തനാവും. അങ്ങിനെ കടാസുരനു മോക്ഷമാവും. മറ്റുള്ളവർ ഈ കഥ പറയുന്നതുകേട്ട് ചെറുകിളിയും (വ്യാളാസുരൻ) നിർവ്വാണപദം പ്രാപിക്കും. ദാമാസുരനും (കൊതുക്) കഥാശ്രവണംകൊണ്ട് മുക്തിയെ പ്രാപിക്കും. ഇതാണു രാമാ ദാമൻ മുതലായ മൂന്നു രാക്ഷസന്മാരുടെ കഥ. അവരുടെ അഹംഭാവം എങ്ങിനെ അവരെ നരകഗർത്തങ്ങളിൽ ചാടിച്ചു എന്നു നാം കണ്ടു. ഇവയെല്ലാം അവിദ്യയുടേയും മോഹവിഭ്രാന്തിയുടേയും ലീലകളാണ്‌..

“വാസ്തവത്തിൽ ശുദ്ധാവബോധം തന്നെയാണ്‌ ‘ഇതു ഞാൻ’ എന്ന ധാരണയെ വച്ചുപുലർത്തുന്നത് എന്നു തോന്നുന്നു. ഒരു ലീലപോലെ, ഒരിക്കലും സഹജസ്വരൂപമായ അനന്താവബോധത്തെ ഉപേക്ഷിക്കാതെ തന്നെ അതു സ്വയം വികൃതമായ ദൃശ്യങ്ങളെ ഉള്ളിൽക്കണ്ട് അനുഭവമാക്കുന്നു എന്നാണു തോന്നുന്നത്.” ഈ വൈകൃത ദൃശ്യങ്ങൾ തികച്ചും അയാഥാർത്ഥ്യമാണെങ്കിലും അഹംഭാവം അവയെ ഉണ്മയാണെന്നു കരുതി സ്വയം മോഹത്തിനടിമയാകുന്നു. 
