\section{ദിവസം 257}

\slokam{
ആത്മാവലോകനേനാശു മാധവ: പരിദൃശ്യതേ\\
മാധവാരാധനേനാശു സ്വയമാത്മാവലോക്യതേ (5/42/21)\\
}

വസിഷ്ഠന്‍ തുടര്‍ന്നു: അത്രയും പറഞ്ഞ് വിഷ്ണുഭഗവാന്‍ അസുരന്മാരുടെ അടുത്തുനിന്നും വിടവാങ്ങി. ഭഗവല്‍പ്രസാദത്താല്‍ ദേവന്മാരും അസുരന്മാരും മനുഷ്യരുമെല്ലാം ബുദ്ധിമുട്ടുകളൊന്നുമില്ലാതെ സന്തോഷമായി കഴിഞ്ഞുവന്നു. ഇതാണ് പ്രഹ്ലാദ ചരിതം. രാമാ, ഹൃദയത്തിലെ ആശുദ്ധികളെ ഇല്ലായ്മ ചെയ്യാന്‍ പര്യാപ്തമാണിക്കഥ. പാപികളായാലും ദുഷ്ടരായാലും ഈ കഥ കേട്ട് അതിനെക്കുറിച്ചു മനനം ചെയ്‌താല്‍ അവരുടെ ബോധം ഉയര്‍ന്ന തലങ്ങളിലെത്തിച്ചേരും. ഈ കഥയിലേയ്ക്ക് അല്‍പ്പമാത്രം ശ്രദ്ധ ചെലുത്തിയാല്‍ത്തന്നെ ഒരുവന്റെ പാപമെല്ലാം നശിക്കുന്നു. എന്നാലാ ശ്രദ്ധ യോഗമാര്‍ഗ്ഗത്തിലാണെങ്കിലോ, അവന്‍ ഉയര്‍ന്ന ബോധതലത്തെ പ്രാപിക്കും, നിശ്ചയം. പാപം എന്നത് അജ്ഞാനമാണ്. അന്വേഷണത്തില്‍ അത് പാടേ നശിക്കുന്നു. അതുകൊണ്ട് നാം അന്വേഷണം ഒരിക്കലും ഉപേക്ഷിക്കരുത്.

രാമന്‍ ചോദിച്ചു: മഹാത്മന്‍, അതീന്ദ്രിയ ധ്യാനാവസ്ഥയിലിരുന്ന പ്രഹ്ലാദന്‍ എങ്ങിനെയാണ് ശംഖുനാദം കേട്ടപ്പോള്‍ ഉണരാനിടയായത്?

വസിഷ്ഠന്‍ പറഞ്ഞു: മുക്തിയെന്നത് രണ്ടുതരമാണ് രാമാ. ഒന്ന്‍ ദേഹത്തോട് കൂടിയതും (ജീവന്മുക്തി) മറ്റേത് ദേഹമില്ലാതെയും (വിദേഹമുക്തി). ജീവന്മുക്തനില്‍ മനസ്സ് പരിപൂര്‍ണ്ണമായും അനാസക്തമാണ്. യാതൊന്നിനോടും, നേടാനോ, ത്യജിക്കാനോ ഉള്ള കര്‍മ്മങ്ങളോടുപോലും ബന്ധമില്ലാത്ത അവസ്ഥ. എന്നാല്‍ ശരീരം വീണുകഴിഞ്ഞുള്ള അവസ്ഥയാണ് വിദേഹമുക്തി. ജീവന്മുക്തനില്‍ എല്ലാ വാസനകളും മനോപാധികളും വറുത്തെടുത്ത വിത്തുപോലെ ഇനിയുമൊരിക്കലും മുളപൊട്ടാത്തവയാണ്. ഇനിയൊരു ജന്മമെടുക്കാന്‍ അവ അവസരമുണ്ടാക്കുന്നില്ല. എന്നാല്‍ നിര്‍മലത, ആത്മജ്ഞാനം, വിശാലബുദ്ധി, തുടങ്ങിയ മനോപാധികള്‍ അയാളില്‍ നിലനില്‍ക്കുന്നുണ്ട്. അവ ദീര്‍ഘസുഷുപ്തിയിലുള്ളവന്റെ മനോവ്യാപാരങ്ങള്‍പോലെ നിര്‍ലീനമാണ്.      

ഈ മനോപാധികളുടെ അംശം ജീവന്മുക്തയോഗികളില്‍ ഉള്ളിടത്തോളം കാലം, നൂറിലേറെ കൊല്ലങ്ങള്‍ കഴിഞ്ഞാല്‍പ്പോലും  അവരെ ലോകബോധത്തിലേയ്ക്ക് തിരികെ ഉണര്‍ത്താന്‍ കഴിയും. ശംഖനാദം മുഴങ്ങിയപ്പോള്‍  പ്രഹ്ലാദന്റെ അവസ്ഥ അതായിരുന്നു. മാത്രമല്ല, ഭഗവാന്‍ വിഷ്ണു എല്ലാവരുടെയും ആത്മാവ് തന്നെയായതുകൊണ്ട് അവിടുത്തെ ഇച്ഛ നടപ്പാവാന്‍ കാലതാമസം ഉണ്ടാവുകയില്ലല്ലോ. അവിടുത്തെ അവതാരത്തിന് കാരണമൊന്നും വേണ്ട. എന്നാലാ അവതാരങ്ങള്‍ ഈ പ്രപഞ്ചത്തില്‍ അനന്തമായ മറ്റ്‌ ജീവജാലങ്ങളെ സൃഷ്ടിക്കാനായി ഉടലെടുക്കുന്നവയാണ്.     

“ആത്മജ്ഞാനമാര്‍ജ്ജിച്ചു കഴിഞ്ഞാല്‍ ഭഗവാന്‍ വിഷ്ണുവിനെ പ്രാപിച്ചു എന്നര്‍ത്ഥം. വിഷ്ണുപൂജ കൊണ്ട് ആത്മസാക്ഷാത്കാരം പ്രാപ്യമാണ് .” രാമാ, നിസ്തന്ദ്രമായ അന്വേഷണത്തിലൂടെ പ്രഹ്ലാദന്റെ സ്ഥിതിയെ പ്രാപിക്കൂ. അങ്ങിനെ നിനക്കും പരംപൊരുളിനെ സാക്ഷാത്കരിക്കാം. ഹൃദയത്തില്‍ ആത്മാന്വേഷണത്തിന്റെ സൂര്യന്‍ ഉദിക്കാതിരുന്നാല്‍ മാത്രമേ ലോകം നിന്നെ ഭ്രമിപ്പിക്കുകയുള്ളു. വിഷ്ണുകൃപയും ആത്മജ്ഞാനവും ഉള്ളവനെ പ്രത്യക്ഷലോകമെന്ന സത്വത്തിന് ഭ്രമിപ്പിക്കാന്‍ കഴിയില്ല.