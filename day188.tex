\section{ദിവസം 188}

\slokam{
നിദർശനാർഥം സൃഷ്ടേസ്തു മയൈകസ്യ പ്രജാപതേ:\\
ഭവതേ കഥിതോത്പത്തിർ ന തത്ര നിയമ: ക്വചിത് (4/47/47)\\
}

വസിഷ്ഠൻ തുടർന്നു: അല്ലയോ രാമ, ബ്രഹ്മാ-വിഷ്ണു-ശിവന്മാരും ഇന്ദ്രനും മറ്റും ലക്ഷക്കണക്കിനുണ്ടായിട്ടുണ്ട്. എന്നാലും ദേവതമാരുടെ സൃഷ്ടികളും എല്ലാം മായയുടെ വെറും കളിയാണെന്നറിയുക. ചിലപ്പോൾ സൃഷ്ടികളുണ്ടാവുന്നത് ബ്രഹ്മാവിൽ നിന്നുമാണ്‌.; മറ്റു ചിലപ്പോൾ ശിവനിൽ നിന്നും നാരായണനിൽ നിന്നും മാമുനിമാരിൽ നിന്നുമാകാം. ബ്രഹ്മാവുണ്ടായതോ, ചിലപ്പോൾ താമരപ്പൂവിൽ നിന്നും, ചിലപ്പോൾ ജലത്തിൽ നിന്നും മറ്റുചിലപ്പോൾ ഒരണ്ഡത്തിൽ നിന്നോ ആകാശത്തുനിന്നോ ആകാം. ചില ലോകങ്ങളിൽ ബ്രഹ്മാവാണ്‌ പരമദൈവതം. മറ്റിടങ്ങളിൽ സൂര്യൻ, ഇന്ദ്രൻ, നാരായണൻ, ശിവൻ എന്നിങ്ങനെ പലരെയാണ്‌ പരമപൂജ്യരായി കണക്കാക്കുന്നത്. ചില ലോകങ്ങളിൽ ഭൂമി മുഴുവൻ വൃക്ഷനിബദ്ധമാണ്‌..  മറ്റുലോകങ്ങളിൽ ഉള്ളത് മനുഷ്യരും മലകളുമാവാം. ചിലയിടങ്ങളിലെ മണ്ണ്‌ പശിമയുള്ള കളിമണ്ണാണ്‌. മറ്റിടങ്ങളിൽ ചെമ്പുനിറവും സ്വർണ്ണവർണ്ണമുള്ള പാറക്കല്ലുകളുണ്ട്. ഒരുപക്ഷേ സൂര്യകിരണങ്ങളെത്രയെന്ന് എണ്ണാൻ കഴിഞ്ഞാലും എത്ര ലോകങ്ങളാണുള്ളതെന്ന് എണ്ണി തിട്ടപ്പെടുത്താൻ കഴിയില്ല. ഈ സൃഷ്ടിയ്ക്ക് ആദിയില്ല. അന്തവും.

ബ്രഹ്മം എന്ന നഗരത്തിൽ അതായത് അനന്താവബോധത്തിൽ, ഹൃദയാകാശത്തിലെ ബോധത്തിൽ, ഈ ലോകങ്ങൾ ഉണ്ടായി മറിഞ്ഞുകൊണ്ടേയിരിക്കുന്നു. അനന്തവാബോധത്തിൽ നിന്നും ഇവ വ്യത്യാസപ്പെട്ടിരിക്കുന്നു. അനന്താവബോധത്തിൽ നിന്നുദ്ഭൂതമായ സൂക്ഷ്മ ഘടകങ്ങളാൽ (ഭൂതങ്ങളാൽ) കൊരുക്കപ്പെട്ട മാലകളാണ്‌ സ്തൂലവും സൂക്ഷ്മവുമായ എല്ലാ സൃഷ്ടിവിശേഷങ്ങളും. ചിലപ്പോൾ ആകാശമാണാദ്യം ഉണ്ടാവുക- അപ്പോൾ സൃഷ്ടികർത്താവ് ആകാശത്തുനിന്നും ജനിച്ചു എന്നു പറയുന്നു. ചിലപ്പോൾ വായുവാകാം ആദ്യമുണ്ടാവുന്നത്. മറ്റുചിലപ്പോൾ ജലം, അഗ്നി, ഭൂമി എന്നിങ്ങനെ ആദ്യമുണ്ടാകുന്ന ഘടകത്തിനനുസരിച്ച് സൃഷ്ടാവിന്‌ 'നാമരൂപ' വ്യക്തിത്വമുണ്ടാവുന്നു. സൃഷ്ടാവിന്റെ ശരീരത്തിൽ നിന്നും ‘ബ്രാഹ്മണൻ’, ‘പൂജാരി’, തുടങ്ങിയ വാക്കുകളുണ്ടായി അവ ജീവനുള്ള സത്വങ്ങളായി പരിണമിക്കുന്നു. ഇതെല്ലാം സ്വപ്നസദൃശമായ അയാഥാർത്ഥ്യമാണെന്ന് തീർച്ചയായും മറക്കാതിരിക്കുക. അതുകൊണ്ട് 'ഇക്കാണായതെല്ലാം അനന്താവബോധത്തിൽ എങ്ങിനെ ഉദ്ഭൂതമായി?' എന്ന ചോദ്യം തന്നെ ബാലിശവും അപക്വവുമാണ്‌.. മനസ്സിന്റെ ഉദ്ദേശങ്ങളും സങ്കൽപ്പങ്ങളുനുസരിച്ചാണ്‌ സൃഷ്ടി നടക്കുന്നതായി തോന്നുന്നത്. ഇത് തീർച്ചയായും വിസ്മയകരം തന്നെ.

”സത്യത്തിനെ ഉദാഹരിക്കാനുമായുള്ള ഒരുപാധിയായാണ്‌ ഞാനിതു നിനക്കു വിവരിച്ചു തന്നത്. യഥാർത്ഥത്തിൽ ഇപ്പറഞ്ഞപോലുള്ള യാതൊരു ക്രമവും സൃഷ്ടിക്കില്ല.“ സൃഷ്ടിയെന്നാൽ മനസ്സിന്റെ സൃഷ്ടിയെന്നർത്ഥം. ഇതാണു സത്യം. മറ്റെല്ലാം വെറും വിചിത്രമായ ഭാവനകൾ കൊണ്ടു മെനഞ്ഞ വിവരണങ്ങൾ മാത്രം. തുടർച്ചയായുള്ള സൃഷ്ടിയും വിലയനവും കാരണം അണുവിടമുതൽ യുഗപര്യന്തമുള്ള കാലയളവുകൾക്കായി സമയം എന്ന സങ്കൽപ്പം ഉണ്ടായി. എന്നാൽ ഈ വിശ്വം നിലകൊള്ളുന്നത് അനന്താവബോധത്തിൽ മാത്രമാണ്‌.. ചുട്ടുപഴുത്ത ലോഹത്തിലടങ്ങിയ തീപ്പൊരിപോലെ ഒരു സാദ്ധ്യതാ സാന്നിദ്ധ്യമാണത്. ശുദ്ധദൃഷ്ടിയുള്ള ജ്ഞാനിക്ക് എല്ലാം ബ്രഹ്മം മാത്രം. ലോകമെന്ന ഒരു വ്യതിരിക്ത സത്ത അയാളെ സംബന്ധിച്ചിടത്തോളം ഇല്ല. എണ്ണമറ്റ ലോകങ്ങളുടെ അനന്തമായ സൃഷ്ടിയും വിലയനവും, അവയ്ക്കുളിലെ സൃഷ്ടികർത്താക്കളുമെല്ലാം അജ്ഞാനിയുടെ മനസ്സിലെ വിചിത്രസങ്കൽപ്പങ്ങൾ മാത്രം. അവ കേവലം മിഥ്യയത്രേ.

