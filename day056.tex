 
\section{ദിവസം 056}

\slokam{
പ്രാക്തനി സാ സ്മൃതിർ ലുപ്താ യുവയോരുദിതാന്യഥാ\\
സ്വപ്നേജാഗ്രത്സ്മൃതിര്യദ്വ ദേതന്മരണമംഗനേ (3/20/16)\\
}

സരസ്വതി തുടര്‍ന്നു: ആ മഹാത്മാവിന്റെ പേര്‍ വസിഷ്ഠന്‍ ; പത്നി അരുന്ധതി (ലോക പ്രശസ്തരായ വസിഷ്ഠനും അരുന്ധതിയുമല്ല ഇവര്‍ ). ഒരിക്കല്‍ അദ്ദേഹം ഒരു മലമുകളില്‍ ഇരിക്കുമ്പോള്‍ താഴ്വാരത്തുകൂടി വര്‍ണ്ണാഭമായ ഒരു ഘോഷയാത്ര പോവുന്നതു കണ്ടു. ഒരു രാജാവ്‌ ആനപ്പുറത്ത്‌ തന്റെ സൈന്യത്തോടും രാജപരിവാരങ്ങളോടും കൂടി എഴുന്നുള്ളുകയാണ്‌. അതുകണ്ട്‌ മഹാത്മാവിന്റെ ഉള്ളില്‍ ഒരാഗ്രഹം ഉണര്‍ന്നു: 'തീര്‍ച്ചയായും ഒരു രാജാവിന്റെ സമ്പന്നജീവിതം മഹിമയും സന്തോഷവും ഒത്തുചേര്‍ന്നതാണ്‌. എനിക്കിനിയെന്നാണ്‌ അതുപോലെ, രാജാവിനേപ്പോലെ ആനപ്പുറത്തെഴുന്നള്ളാനും സൈന്യസന്നാഹങ്ങളോടെ നീങ്ങാനും കഴിയുക?' താമസംവിനാ അദ്ദേഹത്തെ വാര്‍ദ്ധക്യം പിടികൂടി അവസാനം മരണം വന്നു കൂട്ടിക്കൊണ്ടു പോവുകയും ചെയ്തു. അദ്ദേഹത്തിന്റെ ഭാര്യ, തന്റെ ഭര്‍ത്താവിനോട്‌ അങ്ങേയറ്റം പ്രിയമുണ്ടായിരുന്നതുകൊണ്ട്‌ നീ എന്നോടാവശ്യപ്പെട്ട വരങ്ങള്‍ തന്നെ ആദരവോടെ എന്നെ സം പ്രീതയാക്കി ചോദിച്ചുവാങ്ങി. അതായത്‌ അദ്ദേഹത്തിന്റെ ജീവന്‍ അവളുടെ വീടു വിട്ട്‌ പോവരുതെന്ന്. ഞാന്‍ ആ വരം നല്‍കി. 

സൂക്ഷ്മശരീരിയായിരുന്നുവെങ്കിലും ആ മഹാത്മാവ്‌ കഴിഞ്ഞ ജന്മത്തിലെ തന്റെ നിരന്തരമായ ആഗ്രഹം ഹേതുവായി പ്രഗത്ഭനായ ഒരു രാജാവായിത്തീര്‍ന്നു. അദ്ദേഹം തന്റെ സാമ്രാജ്യം, ഭൂമിയെ സ്വര്‍ഗ്ഗസമാനമാക്കി ഭരിച്ചു. അദ്ദേഹം ശത്രുക്കള്‍ ക്കു ഭയമുണ്ടാക്കി; സ്ത്രീകള്‍ക്ക്‌ അദ്ദേഹം കാമദേവനായി; പ്രലോഭനങ്ങള്‍ക്കെതിരേ പര്‍വ്വതം പോലെ ഉറച്ചനിലപാടെടുത്തു അദ്ദേഹം. വേദശാസ്ത്രങ്ങള്‍ കണ്ണാടിയിലെന്നവണ്ണം അദ്ദേഹത്തില്‍ പ്രതിഫലിച്ചു. പ്രജകളുടെ എല്ലാ ആവശ്യങ്ങളും അദ്ദേഹം നടത്തിക്കൊടുത്തു. സദ്പുരുഷന്മാര്‍ക്ക്‌ അദ്ദേഹം ഒരത്താണിയായി. ധാര്‍മ്മീകജീവിതത്തിന്റെ പൂര്‍ണ്ണചന്ദ്രനായി അദ്ദേഹം തിളങ്ങി. അരുന്ധതിയും ഭര്‍ത്താവിനെ പിന്തുടര്‍ന്ന് മരണപ്പെട്ടിരുന്നു. ഇതെല്ലാം നടന്നിട്ട്‌ എട്ടു ദിവസങ്ങളായിരിക്കുന്നു.

ലീലേ, ഈ മഹാത്മാവാണ്‌ നിന്റെ പ്രിയതമനായ രാജാവ്‌. നീ തന്നെയാണദ്ദേഹത്തിന്റെ ഭാര്യ അരുന്ധതി. അവിദ്യയും മോഹവിഭ്രാന്തിയും കാരണം ഇതെല്ലാം അനന്താവബോധത്തില്‍ സംഭവിക്കുന്നതായി തോന്നുന്നു. ഇതെല്ലാം ശരിയെന്നോ തെറ്റെന്നോ നിനക്കു ബോധിച്ചപോലെ കണക്കാക്കാം. 

ലീല പറഞ്ഞു: ദേവീ, ഇതെല്ലാം അവിശ്വസനീയവും വിചിത്രവുമായെനിക്കു തോന്നുന്നു. വലിയൊരാന ചെറിയൊരു കടുകുമണിയില്‍ ബന്ധിതനാണെന്നു പറയുമ്പോലെ വിചിത്രം! ഒരണുവില്‍ ഒരുകൊതുക്‌ സിംഹത്തോട്‌ യുദ്ധംചെയ്തു എന്നു പറയുമ്പോലെ അവിശ്വസനീയം! ഒരുതാമരമൊട്ടില്‍ പര്‍വ്വതമിരിക്കുന്നു എന്നപോലെ അസംബന്ധം!

സരസ്വതി പറഞ്ഞു: ഞാന്‍ കള്ളം പറയുകയില്ല. സത്യം അവിശ്വസനീയം തന്നെ. എന്നാല്‍ ഞാന്‍ പറഞ്ഞ രാജ്യം പ്രത്യക്ഷമായത്‌ ഈ കുടിലില്‍ മാത്രമാണ്‌. അതിനുകാരണമോ ആ മഹാത്മാവിന്റെ രാജാവാകാനുള്ള ആഗ്രഹവുമാണ്‌. "ഭൂതകാലസ്മരണകള്‍ നമ്മില്‍ നിന്നും മറഞ്ഞിരിക്കുന്നു. നിങ്ങള്‍ രണ്ടാളും വീണ്ടും ആവിര്‍ഭവിച്ചിരിക്കുന്നു. മരണം എന്നത്‌ സ്വപ്നത്തില്‍നിന്നുമുള്ള ഉണരല്‍ മാത്രം." ആഗ്രഹത്തില്‍നിന്നുദ്ഭൂതമവുന്ന ജനനത്തിന്‌ ആഗ്രഹത്തിനേക്കാള്‍ യാഥര്‍ത്ഥ്യതയൊന്നുമല്ല. മരുപ്പച്ചയിലെ അലകള്‍പോലെയാണത്‌.
