\newpage
\section{ദിവസം 139}

\slokam{
ജ്ഞാനഭൂമി: ശുഭേച്ഛാഖ്യാ പ്രഥമാ സമുദാഹൃത:\\
വിചാരണാ  ദ്വിതീയാ തു തൃതീയാ തനുമാനസാ (3/118/5)\\
സത്ത്വാപത്തിശ്ചതുർത്ഥാ സ്യാത്തതോസംസക്തിനാമികാ\\
പദാർത്ഥാഭാവനീ  ഷഷ്ഠീ  സപ്തമീ തുര്യഗാ  സ്മൃതാ (3/118/6)\\
}

വസിഷ്ഠൻ തുടർന്നു: രാമാ, ഞാനിനി വിജ്ഞാനതലങ്ങളുടെ ഏഴുപടികൾ എന്തെന്നു വിശദമാക്കാം. അതറിഞ്ഞുകഴിഞ്ഞാൽ നീ മോഹവലയത്തിലകപ്പെടുകയില്ല. “ശുദ്ധമായ ഇച്ഛയാണ്‌ ഒന്നാമത്. രണ്ടാമത് അന്വേഷണം. മൂന്നാമത് മനസ് സൂക്ഷ്മമാവുന്ന അവസ്ഥ. നാലാമത് സത്യത്തിൽ മനസ്സുറച്ച അവസ്ഥ . കെട്ടുപാടുകളിൽ നിന്നുമുള്ള മോചനമാണ്‌ അഞ്ചാമത്. വസ്തുനിഷ്ഠസമീപനത്തിന്റെ (വിഷയ-വിഷയി കാഴ്ച്ച ) അവസാനമാണ്‌ ആറാമത്. ഏഴാമത് ഇവയ്ക്കെല്ലാമുപരിയുള്ള മറൊരു തലമത്രേ.”

‘ഞാനിങ്ങിനെ മൂഢനായി തുടരുന്നതെന്തേ? ഇതില്‍നിന്നും എനിക്കൊരു മോചനം വേണം. മഹാത്മാക്കളേയും വേദഗ്രന്ഥങ്ങളേയും സമാശ്രയിച്ച് എനിക്ക് അനാസക്തി വളർത്തിയെടുക്കണം’ ഇങ്ങിനെയുള്ള ഇച്ഛയാണ്‌ ആദ്യത്തെ പടി. അതിനുശേഷം നേരിട്ടുള്ള അന്വേഷണമാണടുത്ത പടി. ഇതുമൂലം ആസക്തിയിൽ കുറവുണ്ടായി, മനസ്സ് സൂക്ഷ്മവും സുതാര്യവുമാവുന്നതാണ്‌ മൂന്നാമത്തെ അവസ്ഥ. ഈ മൂന്നവസ്ഥകളും അഭ്യസിച്ചുവരുമ്പോൾ ഇന്ദ്രിയസുഖങ്ങളിൽ നിന്നും സ്വാഭാവികമായി ഒരകൽച്ചയും അതുമൂലം സത്യവസ്തുവിൽ അഭിരമിക്കാനുള്ള സഹജഭാവവും ഉണ്ടാവുന്നു. ഇവയെല്ലാം നന്നായി അഭ്യസിച്ച് തികഞ്ഞ അനാസക്തിയാവുമ്പോൾ സത്യത്തിൽ മാത്രം മനസ്സുറയ്ക്കുന്നതാണ്‌ അഞ്ചാമത്തെ അവസ്ഥ. അങ്ങിനെ ആത്മാവിൽ അഭിരമിച്ച് അകത്തും പുറത്തുമുള്ള ദ്വന്ദഭാവനകളും വൈവിദ്ധ്യമാര്‍ന്ന  കാഴ്ച്ചകളും അവസാനിക്കുന്നു. മഹാത്മാക്കളുടെ പ്രചോദനം മൂലം തുടങ്ങിയ ആത്മാന്വേഷണത്തിന്‌ സ്വയം 'നേരനുഭവം' എന്ന ഫലം കാണുന്നു. അതിനുശേഷം മറ്റൊരുപാധിയും, വിഭാഗീയതയും ഇല്ലാതെ ആത്മജ്ഞാനം സഹജവും ഇടമുറിയാത്തതുമായ അനുഭവമാകുന്നു. ഇതാണ്‌ ഏഴാമത്തെ പടിയായ അതീന്ദ്രിയതലം. ഇത് ജീവന്മുക്താവസ്ഥയാണ്‌.. അതിനുമപ്പുറമാണ്‌ ശരീരബോധാതീതമായ തുരീയമെന്ന അവസ്ഥ. രാമാ, ഇപ്പറഞ്ഞ വിജ്ഞാനത്തിന്റെ ഏഴുപടികൾ കയറിയവർ മഹാത്മാക്കളത്രേ. അവർ മുക്തരാണ്‌.. സന്തോഷദു:ഖങ്ങൾ അവരെ ബാധിക്കയില്ല. അവർ കർമ്മനിരതരോ കർമ്മമുപേക്ഷിച്ചവരോ ആകാം. ആത്മാരാമന്മാരായ അവർക്ക് സന്തോഷം കണ്ടെത്താൻ മറ്റുള്ളവരുടെ സഹായം ആവശ്യമില്ല.

പരമോന്നതമായ ബോധാവസ്ഥയെ പ്രാപിക്കാൻ എല്ലാവർക്കുമാവും. ശരീരമുള്ള മനുഷ്യർക്കും, മൃഗങ്ങൾക്കും, ശരീരമെടുക്കാത്ത സത്ത്വങ്ങൾക്കുപോലും ഇതു സാദ്ധ്യമാണ്‌.. കാരണം, വിജ്ഞാനത്തിന്റെ ഉദയം മാത്രമാണല്ലോ ഇതിനുവേണ്ടത്. ഇങ്ങിനെ പരമോന്നതബോധതലത്തിലെത്തിയവർ മഹാത്മാക്കൾ തന്നെയാണ്‌.. അവർ ഏവര്‍ക്കും ഏറെ പ്രിയപ്പെട്ടവരാണ്‌.. അവർക്കു മുന്നിൽ ചക്രവർത്തിമാർ പോലും തൃണതുല്യരത്രേ. അവർ, ഇവിടെ ഇപ്പോൾ, മുക്തരായി ജീവിക്കുന്നു. 

