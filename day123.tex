\newpage
\section{ദിവസം 123}

\slokam{
സങ്കല്‍പ്പജാല കലനൈവ ജഗത്സമഗ്രം\\
സങ്കല്‍പ്പമേവ നനു വിദ്ധി വിലാസചേത്യം\\
സങ്കല്‍പമാത്രമലമുത്സൃജ നിര്‍വികല്‍പ്പ-\\
മാശ്രിത്യ നിശ്ചയമവാപ്നുഹി രാമ ശാന്തിം (3/101/39)\\
}

വസിഷ്ഠന്‍ തുടര്‍ന്നു: ഇതിനെപ്പറ്റി രസകരമായ ഒരു കഥയുണ്ട്‌. ശ്രദ്ധിച്ചു കേട്ടാലും. ചെറിയൊരാണ്‍കുട്ടി തന്റെ ആയയോട്‌ ഒരു കഥപറഞ്ഞു കൊടുക്കാന്‍ ആവശ്യപ്പെട്ടു. അവര്‍ പറഞ്ഞ കഥ,കുട്ടി ശ്രദ്ധയോടെ കേട്ടിരുന്നു.

"ഒരിക്കല്‍ ഒരിടത്ത്‌ ഒരു നഗരത്തില്‍ - ഇല്ലാത്ത നഗരമാണ്‌ കെട്ടോ-, മൂന്നു രാജകുമാരന്മാര്‍ ഉണ്ടായിരുന്നു. അവര്‍ വീരന്മാരും സന്തുഷ്ടരുമായിരുന്നു. ഈ മൂന്നുപേരില്‍ രണ്ടുപേര്‍ ജനിച്ചിട്ടില്ല. മൂന്നാമനെ ഗര്‍ഭം ധരിച്ചിട്ടുപോലുമില്ല. കഷ്ടം അവരുടെ ബന്ധുജനങ്ങള്‍ എല്ലാവരും മരിച്ചുപോയി. സ്വന്തം നഗരം വിട്ട്‌ കുമാരന്മാര്‍ വേറൊരിടത്തേയ്ക്കു പുറപ്പെട്ടു. സൂര്യതാപം സഹിക്കാഞ്ഞ്‌ അവര്‍ മൂര്‍ഛിച്ചു വിണു. ചുട്ടുപഴുത്തമണലില്‍ അവരുടെ കാല്‍പ്പാദങ്ങള്‍ പൊള്ളി. പുല്‍ക്കൊടിമ്പുകള്‍ അവരെ കുത്തിനോവിച്ചു. അവര്‍ മൂന്നു മരങ്ങളുടെ തണലില്‍ എത്തി. അവയില്‍ രണ്ടു മരങ്ങള്‍ അവിടെ ഉണ്ടായിരുന്നില്ല. മൂന്നാമത്തെ മരം നട്ടിട്ടുകൂടിയുണ്ടായിരുന്നില്ല. കുറച്ചുനേരം വിശ്രമിച്ച്‌ ആ മരങ്ങളില്‍ നിന്നുള്ള ഫലങ്ങളും ഭക്ഷിച്ച്‌ അവര്‍ വീണ്ടും യാത്ര തുടര്‍ന്നു. അവര്‍ മൂന്നു നദികളുടെ കരയില്‍ എത്തിച്ചേര്‍ന്നു. അവയില്‍ രണ്ടു നദികള്‍ ഉണങ്ങിവരണ്ടിരുന്നു. മൂന്നാമത്തെതില്‍ ജലമുണ്ടായിരുന്നുമില്ല. കുമാരന്മാര്‍ മുങ്ങിക്കുളിച്ചു, ദാഹവും തീര്‍ത്തു.  അവര്‍ പിന്നീട്‌ വലിയൊരു നഗരത്തിലെത്തി. നഗരം ഇനിയും പണിയാന്‍ പോവുന്നതേയുണ്ടായിരുന്നുള്ളു. അവര്‍ ആ നഗരത്തില്‍ അതിസുന്ദരങ്ങളായ മൂന്നു കൊട്ടാരങ്ങള്‍ കണ്ടു. അതില്‍ രണ്ടെണ്ണം നിര്‍മ്മിച്ചിട്ടില്ല. മൂന്നാമത്തേതിന്‌ ചുവരുകളുമില്ല. കൊട്ടരങ്ങളില്‍ കടക്കവേ അവര്‍ മൂന്നു സ്വര്‍ണ്ണത്തളികകള്‍ കണ്ടു. അതില്‍ രണ്ടെണ്ണം രണ്ടു ഭാഗങ്ങളായി പൊട്ടിയിരുന്നു. മൂന്നാമത്തേത്‌ പൊട്ടി പൊടിഞ്ഞുമിരുന്നു. പൊട്ടിപൊടിഞ്ഞ ആ തളിക അവര്‍ എടുത്തു. അവര്‍ തൊണ്ണൂറ്റിയൊന്‍പതു ഗ്രാം അരിയെടുത്ത്‌ അതില്‍ നിന്നു നൂറുഗ്രാം അരി മാറ്റി, അതു വേവിച്ചു. അവര്‍ മൂന്നു മഹാത്മാക്കളെ ഭിക്ഷയ്ക്കായി ക്ഷണിച്ചു. അവരില്‍ രണ്ടുപേര്‍ക്ക്‌ ദേഹമുണ്ടായിരുന്നില്ല. മൂന്നാമന്‌ വായയും. മഹാത്മാക്കള്‍ ആഹരിച്ചശേഷം ബാക്കിയായവ കുമാരന്മാര്‍ ഉണ്ടു സംതൃപ്തരായി. അങ്ങിനെയവര്‍ ആ നഗരിയില്‍ സന്തോഷത്തോടും സമാധാനത്തോടും കൂടി കഴിഞ്ഞുവന്നു. കുഞ്ഞേ ഇതൊരു മനോഹരമായ കഥയാണ്‌. എപ്പോഴും ഇതോര്‍മ്മിക്കൂ, നിനക്ക്‌ ജ്ഞാനിയായി വളര്‍ന്നു വലിയൊരാളാകാം."

രാമാ, ഈ കഥകേട്ട്‌ ആ ബാലന്‍ ആഹ്ലാദിച്ചു. ഈ കുട്ടിക്കു പറഞ്ഞുകൊടുത്ത കഥയിലെ യാഥാര്‍ഥ്യത്തേക്കാള്‍ കൂടുതല്‍ ഉണ്മയൊന്നും ഈ ലോകസൃഷ്ടിയ്ക്കില്ല. ലോകമെന്നത്‌ വെറും ഭ്രമം മാത്രമാണ്‌. അതൊരാശയത്തിനപ്പുറം, അല്ലെങ്കില്‍ ഒരു സാദ്ധ്യതയ്ക്കപ്പുറം  ഒന്നുമല്ല. അനന്താവബോധത്തില്‍ സൃഷ്ടിയെന്നൊരാശയം ഉദയം ചെയ്തു. അത്രതന്നെ.

"രാമ: ലോകമെന്നത്‌ ഒരാശയം അല്ലെങ്കില്‍ സങ്കല്‍പ്പം  എന്നതിനപ്പുറം ഒന്നുമല്ല. ബോധത്തില്‍ അറിയപ്പെടുന്ന പദാര്‍ത്ഥങ്ങള്‍ എല്ലാം ആശയം മാത്രം. സങ്കല്‍പ്പമെന്ന മാലിന്യത്തെ നീക്കി, ആശയവിമുക്തനായി സത്യത്തില്‍ രൂഢമൂലനായി പരമശാന്തിയെ പ്രാപിച്ചാലും." 
