\newpage
\section{ദിവസം 089}

\slokam{
ബ്രാഹ്മണ സ്പുരണം കിഞ്ചിദ്യദവാതാംബുധേരിവ\\
ദീപസ്യേവാപ്യവാതസ്യ തം ജീവം വിദ്ധി രാഘവം  (3/64/8)\\
}

രാമന്‍ ചോദിച്ചു: അനന്തമായ അവബോധവും അതിന്റെ ചാലക ഊര്‍ജ്ജവും മാത്രം സത്യമായിരിക്കേ ഈ 'ഏകാദ്വയ'ത്തില്‍നിന്നും ജീവന്‍ എങ്ങിനെയാണ്‌ താന്‍ യാഥാര്‍ഥ്യമാണെന്ന ധാരണ ഉരുത്തിരിക്കുന്നത്‌?

വസിഷ്ഠന്‍ പറഞ്ഞു: അജ്ഞാനിയുടെ മനസ്സില്‍ മാത്രമേ ജീവന്‍ എന്ന ഈ ഭൂതം ഒരു പ്രതിഫലനമായി സ്ഫുടീകരിക്കുന്നുള്ളു. ഇതെന്താണെന്നു കൃത്യമായിപ്പറയാന്‍ വിവേകവിജ്ഞാന വിചക്ഷണരായ മഹാത്മാക്കള്‍ക്കുപോലും കഴിയില്ല. കാരണം അതിനു സ്വഭാവസൂചനകള്‍ ഒന്നുമില്ല. അനന്താവബോധമെന്ന കണ്ണാടിയിലെ പ്രതിബിംബങ്ങളാണ്‌ ലോകമായി കാണപ്പെടുന്നത്‌..  "ജീവന്‍ എന്നത്‌ ബ്രഹ്മസമുദ്രോപരി മന്ദസ്പന്ദനം പോലെയുള്ള അലകളാണ്‌.. അല്ലെങ്കില്‍ കാറ്റില്ലാത്തൊരു മുറിയില്‍ വച്ചിട്ടുള്ള ദീപനാളത്തിന്റെ ലോലമായ ചലനമാണ്‌." അനന്തതയിലെ ആ മന്ദചലനം. അനന്താവബോധത്തെ ആഛാദനം ചെയ്യുമ്പോള്‍ ബോധത്തിനു പരിമിതിയുള്ളതായി തോന്നുകയാണ്‌.. ഇതുപോലും അനന്തതയില്‍ സഹജം. അങ്ങിനെ പരിമിതപ്പെട്ട ബോധമാണ്‌ ജീവന്‍.. ഒരു ദീപനാളത്തില്‍ നിന്നു തെറിക്കുന്ന തീപ്പൊരി സ്വതന്ത്രമായ മറ്റൊരു നാളമാവും പോലെ ഈ പരിമിതബോധം നിര്‍ലീനവാസനകളാലും ഓര്‍മ്മകളാലും പ്രേരിതമായി, അഹംകാരമായി, - ഞാന്‍ - ആയി സ്ഫുടീകരിക്കുന്നു. ഈ 'ഞാന്‍' എന്ന ഭാവം ദൃഢമായ യാഥാര്‍ഥ്യമൊന്നുമല്ല. പക്ഷേ ജീവന്‍ അതിനെ സത്തായിക്കാണുന്നു. ആകാശത്തിന്റെ നീലനിറം പോലെയാണത്‌..

അഹംകാരം സ്വയം ധാരണകളുണ്ടാക്കി മനസ്സ്‌ എന്ന 'വസ്തു' സംജാതമാവുന്നു. വ്യക്തിഗത ജീവന്‍, മനസ്സ്‌, മായ, പ്രപഞ്ചമെന്ന ഭ്രമം, അവയുടെ സ്വഭാവസവിശേഷതകള്‍ തുടങ്ങിയ ആശയങ്ങള്‍ ഭാവിതമാവുന്നു. ഈ ധാരണകള്‍ വെച്ചുപുലര്‍ത്തുന്ന മേധാശക്തി പഞ്ചഭൂതങ്ങളെ (ഭൂമി,ജലം,അഗ്നി,വായു,ആകാശം) ആവാഹനം ചെയ്തുണ്ടാക്കുന്നു. ഇതോടുചേര്‍ന്ന് ആ ബുദ്ധി സ്വയം പ്രകാശകണമായിത്തിരുന്നു. ഇതും വിശ്വദീപ്തിതന്നെയാണ്‌. അതു പിന്നീട്‌ എണ്ണമറ്റ രൂപഭാവങ്ങളെ പ്രാപിക്കുന്നു. ചിലയിടത്ത്‌ വൃക്ഷാദികള്‍ , ചിലയിടത്ത്‌ പക്ഷിമൃഗാദികള്‍ , ചിലയിടത്ത്‌ ഭൂതപിശാചുക്കള്‍ , മറ്റുചിലയിടത്ത്‌ ദേവതകള്‍ ഇങ്ങിനെയെല്ലാം ആയിത്തീരുന്നു.

ഇതിലെ ആദ്യത്തെ രൂപീകൃതസത്വം സൃഷ്ടികര്‍ത്താവായ ബ്രഹ്മാവാണ്‌. ഈ ബ്രഹ്മാവാണ്‌ ചിന്തകളും ഇച്ഛാശക്തിയുംകൊണ്ട്‌ മറ്റു ജീവജാലങ്ങളെ സൃഷ്ടിക്കുന്നത്‌. . ബോധത്തിലെ സ്പന്ദനം മാത്രമാണു ജീവന്‍.. കര്‍മ്മവും, ദൈവവും, മറ്റ്‌ എല്ലാം സ്പന്ദനമാണ്‌. . മനസ്സിന്റെ സൃഷ്ടിയെന്നത്‌ ബോധത്തിന്റെ ചലനമാണ്‌. . ലോകം നിലനില്‍ക്കുന്നതോ, മനസ്സിലാണ്‌. . ദര്‍ശനത്തിലെയും അറിവിലേയും അപൂര്‍ണ്ണതയാണ്‌ മനസ്സിന്റെ നിലനില്‍പ്പിനു കാരണം. അതൊരു ദീര്‍ഘനിദ്രയേക്കാള്‍ സാംഗത്യമുള്ള ഒന്നുമല്ല. ഈ അറിവുറച്ചാല്‍ എല്ലാ ദ്വന്ദഭാവവും ഇല്ലാതാവും. ബ്രഹ്മം, ജീവന്‍, മനസ്സ്‌, മായ, കര്‍മ്മി, കര്‍മ്മം, ലോകം എന്നിവയെല്ലാം ആ അവിച്ഛിന്നവും അദ്വയവുമായ അനന്താവബോധത്തിന്റെ പര്യായങ്ങളാണെന്ന സാക്ഷാത്കാരവും  ഉണ്ടാവും.

