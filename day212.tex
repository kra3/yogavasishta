\section{ദിവസം 212}

\slokam{
അയമേവാഹമിത്യസ്മിൻ സങ്കോചേ വിലയം ഗതേ\\
അനന്തഭുവനവ്യാപീ വിസ്താര ഉപജായതേ (5/13/15)\\
}

വസിഷ്ഠൻ തുടർന്നു: അതുകൊണ്ട് രാമാ, ജനകമഹാരാജാവ് ചെയ്തതുപോലെ ആത്മവിചാരംചെയ്യൂ. അങ്ങിനെ നിനക്ക് അറിയപ്പെടേണ്ടതായി എന്തുണ്ടോ അതറിഞ്ഞ ജ്ഞാനികളുടെ അവസ്ഥയെ നിഷ് പ്രയാസം പ്രാപിക്കാം. തുടർച്ചയായി ഇന്ദ്രിയങ്ങളാകുന്ന ശത്രുക്കളെ നേരിടേണ്ടി വന്നേക്കാമെങ്കിലും സ്വപരിശ്രമത്തിൽ നിന്നുണ്ടാകുന്ന ആത്മസംതൃപ്തി വലുതാണല്ലോ. അങ്ങിനെ അനന്തമായ ആത്മാവിനെ സാക്ഷാത്കരിക്കുമ്പോൾ ദു:ഖങ്ങൾക്ക് അറുതിയായി. ഭ്രമാത്മകതയുടെ വിത്തുകളും ദൗർഭാഗ്യങ്ങളും നശിക്കുമ്പോള്‍ ദുഷ് പ്രവണതകൾ ഇല്ലാതാവുന്നു. അതുകൊണ്ട് ജനകനേപ്പോലെയാവൂ. സ്വയംപ്രഭമായ ആത്മാവിന്റെ വെളിച്ചത്തിൽ ആത്മാവിനെ സാക്ഷാത്കരിച്ചാലും.

രാമാ, ഉത്തമനായിത്തീരൂ. ജനകനേപ്പോലെ നിരന്തരം ആത്മവിചാരംചെയ്യുന്ന,  നിരന്തരം മാറ്റങ്ങൾക്കുവിധേയമായിക്കൊണ്ടിരിക്കുന്ന ജഗത്തിനെ സാക്ഷിഭാവത്തില്‍ വീക്ഷിക്കുന്ന, ഏതൊരുവനും കാലക്രമത്തിൽ ആത്മജ്ഞാനം ലഭ്യമത്രേ. ഈശ്വരനോ യാഗകർമ്മങ്ങളോ, സമ്പത്തോ, ബന്ധുക്കളോ ഒന്നും ഇതിൽ നമുക്കുതകുകയില്ല. ലോകമെന്ന മായക്കാഴ്ച്ചയിൽ ഭീതിയുള്ളവർക്ക് ആത്മജ്ഞാനത്തിനുള്ള മാർഗ്ഗം ആത്മാന്വേഷണം എന്ന സ്വപ്രയത്നം തന്നെയാണ്‌... നിന്നെപ്പോലുള്ളവർ ദൈവങ്ങളിലോ, യാഗകർമ്മങ്ങളിലോ, എന്നുവേണ്ട വിചിത്രങ്ങളായ ആചാരധർമ്മങ്ങളെ ആശ്രയിക്കുന്ന ഭ്രമചിത്തന്മാരായ ഗുരുക്കന്മാരുടെ പാതകളിലോ ഒന്നും പോയി വീഴരുതേ എന്നു ഞാൻ പ്രാർത്ഥിക്കുന്നു.

സംസാരസാഗരം തരണംചെയ്യാൻ പരമാർത്ഥജ്ഞാനംകൂടിയേ തീരൂ. അതായത് മേധാശക്തിയെ ഛിന്നഭിന്നമായിപ്പോകാതെ ഏകാത്മകമായും ഇന്ദ്രിയങ്ങളുടെ നിറഭേദങ്ങൾക്കടിമപ്പെടാതെ ജാഗ്രതയോടെയും ഉചിതമായി വിനിയോഗിച്ച് സ്വയം ആത്മാവിനെ ആത്മാന്വേഷണത്തിലൂടെ ആത്മാവെന്നു തിരിച്ചറിഞ്ഞു കണ്ടാലാണിതു സാദ്ധ്യമാവുക. ജനകമഹാരാജാവിന്‌ ആത്മജ്ഞാനമുണ്ടായ കഥ ഞാൻ പറഞ്ഞു. എങ്ങുനിന്നോ ലഭിച്ച കൃപാലേശം മൂലം സ്വർഗ്ഗത്തിൽ നിന്നും ജ്ഞാനത്തിന്റെ അമൃതബിന്ദു താഴെവീണുകിട്ടിയതുപോലെയായിരുന്നു അത്. ജനകനെപ്പോലെ വിവേകശാലികളായവർക്ക് അവരുടെയുള്ളിലങ്കുരിക്കുന്ന ആത്മപ്രകാശംകൊണ്ട് ലോകമെന്ന മായക്കാഴ്ച്ച ക്ഷണത്തിൽ ഇല്ലാതാകുന്നതാണനുഭവം.

“ഞാൻ ഇന്നയാളാണ്‌ എന്ന പരിമിതവും ഉപാധിസ്ഥവുമായ തോന്നൽ ഇല്ലാതാവുമ്പോൾ സർവ്വവ്യാപിത്വമാർന്ന അനന്താവബോധത്തിന്റെ ഉദയമായി” അതുകൊണ്ട് രാമാ, നീയും ജനകൻ ചെയ്തതുപൊലെ മനസ്സിലുദിക്കുന്ന ഭാവനാത്മകവും അസത്തുമായ അഹം ഭാവത്തെ ഉപേക്ഷിച്ചാലും. അഹംഭാവം നീങ്ങിയാൽ ആത്മജ്ഞാനത്തിന്റെ വെളിച്ചം നിന്നിലുദിക്കും, തീർച്ച. അഹമാണ്‌ ഏറ്റവും കൊടിയ അന്ധകാരം. അതില്ലാതായാൽ അകമ്പൊരുൾ പ്രകാശം സ്വയം പ്രഭാസിക്കുന്നു.

ആരൊരുവൻ സ്വയം ‘ഞാൻ ഇല്ല’ എന്നറിയുന്നുവോ അവന്‌ ‘മറ്റുള്ളവരും ഇല്ലെന്ന്’അറിയാം. അയാൾക്ക് ‘ഇല്ലായ്മപോലും ശരിക്കും ഉള്ളതല്ല’ എന്നുമറിയാം. ഇങ്ങിനെ മാനസീകവ്യാപാരങ്ങളൊഴിഞ്ഞയാൾക്ക് 'നേടേണ്ട'തായി യാതൊന്നുമില്ല. രാമാ, ബന്ധനം എന്നത് എന്തെങ്കിലും നേടാനുള്ള ആർത്തിയും മറ്റുള്ളവർ അനഭികാമ്യമെന്നുകരുതുന്നതിനെ വരിക്കാതിരിക്കാനുള്ള ആശങ്കാകുലതയുമാണ്‌.. അത്തരം ആശങ്കയ്ക്കടിമയാകാതിരിക്കൂ. അഭിമതങ്ങളായവയെ നേടുക എന്നത് നിന്റെ ലക്ഷ്യവുമാകാതെയുമിരിക്കട്ടെ. ഈ രണ്ടു മനോഭാവങ്ങളേയുമുപേക്ഷിച്ച് ബാക്കി എന്താണോ ഉള്ളത്, അതിൽ അഭിരമിച്ചാലും.

