 
\section{ദിവസം 041}

\slokam{
ദൃഷ്ടുദൃശ്യ ക്രമോ യത്ര സ്ഥിതോപ്യസ്തമയം ഗത:\\
യദനാകാശമാകാശം തദ്രൂപം പരമാത്മന: (3/7/21)\\
}

രാമന്‍ ചോദിച്ചു: ഈ ഈശ്വരന്‍ അധിവസിക്കുന്നതെവിടെയാണ്‌? എനിക്കെങ്ങിനെ അദ്ദേഹത്തെ പ്രാപിക്കാന്‍ കഴിയും?

വസിഷ്ഠന്‍ പറഞ്ഞു: ഈശ്വരന്‍ എന്നു പറയപ്പെടുന്നയാള്‍ ദൂരെയൊന്നുമല്ല. ഈ ശരീരത്തില്‍ കുടികൊള്ളുന്ന ബോധം തന്നെയാണീശ്വരന്‍ . വിശ്വമാണദ്ദേഹം എന്നാല്‍ വിശ്വം അദ്ദേഹമല്ല. ശുദ്ധബോധമാണത്‌. 

രാമന്‍ പറഞ്ഞു: ഏതൊരു കുഞ്ഞിനുപോലും ഈശ്വരന്‍ ബോധസ്വരൂപനാണെന്നറിയാം. ഇതിനായി പ്രത്യേക പഠനത്തിന്റെ ആവശ്യമെന്താണ്‌?

വസിഷ്ഠന്‍ പറഞ്ഞു: ശുദ്ധബോധമാണ്‌ വസ്തുപ്രപഞ്ചമെന്ന അറിവുള്ളവന്‌ വാസ്തവത്തില്‍ ഒന്നും അറിയില്ല. പ്രപഞ്ചവും ജീവാത്മാവും ചേതനയുള്ളതാണ്‌. ഈ ചേതനയാണ്‌ അറിയപ്പെടുന്നവയെ സൃഷ്ടിച്ച്‌ ദു:ഖങ്ങളിലാമഗ്നമാവുന്നത്‌. ഈ 'അറിയപ്പെടുന്നവകള്‍ ' ഇല്ലാതായി ശ്രദ്ധ ശുദ്ധബോധത്തിലേയ്ക്കുന്മുഖമാവുമ്പോള്‍ (അറിയപ്പെടാനാവാത്തതിലേയ്ക്ക്‌) ഒരുവന്‌ ദു:ഖങ്ങള്‍ക്കതീതമായ സാഫല്യം കൈവരുന്നു.

അറിയപ്പെടുന്നവ ഇല്ലാതായാല്‍ മാത്രമേ അവയില്‍ നിന്നും ശ്രദ്ധ തിരിക്കാന്‍ പറ്റുകയുള്ളു. ജീവാത്മാവ്‌ സംസാരത്തില്‍ മുഴുകിയിരിക്കുന്നു എന്ന കേവലജ്ഞാനം മാത്രം ഉണ്ടായതുകൊണ്ടു കാര്യമില്ല. എന്നാല്‍ പരം പൊരുളിനെ അറിഞ്ഞാല്‍ ദു:ഖങ്ങള്‍ക്ക്‌ അറുതിയായി. 

രാമന്‍ പറഞ്ഞു: ഭഗവന്‍ , ഈശ്വരനെപ്പറ്റി വിശദമായി പറഞ്ഞുതന്നാലും

വസിഷ്ഠന്‍ മറുപടിയായി പറഞ്ഞു: ഈ പ്രപഞ്ചം തന്നെ ഇല്ലാതായിത്തീരുന്ന ആ വിശ്വാവബോധം എന്താണോ അതാണീശ്വരന്‍ . "അവനില്‍ വിഷയവും വിഷയിയും തമ്മിലുള്ള ബന്ധം നിലച്ചതായി കാണപ്പെടുന്നു. പ്രത്യക്ഷമായ ഈ വിശ്വപ്രപഞ്ചം സ്ഥിതിചെയ്യുന്നതായി തോന്നുന്നത്‌ ഈശ്വരനെന്ന ശൂന്യതയിലാണ്‌. അവനില്‍ വിശ്വാവബോധം മഹാമേരുവിനേപ്പോലെ അചലമാണ്‌."

രാമന്‍ വീണ്ടും ചോദിച്ചു: നാം സത്തെന്നു ചിന്തിച്ചു വിശ്വസിച്ച പോരുന്ന ഈ വിശ്വം ഉണ്മയല്ലെന്ന സത്യം നമുക്കെങ്ങിനെ അനുഭവസിദ്ധമാക്കാം?

വസിഷ്ഠന്‍ അതിനുത്തരമായിപ്പറഞ്ഞു: ഈശ്വരനെ സാക്ഷാത്കരിക്കണമെങ്കില്‍ വിശ്വം ഉണ്മയല്ലെന്നുള്ള കാര്യം ഉള്ളില്‍ ദൃഢീകരിച്ചാല്‍ മാത്രമേ സാധിക്കൂ. ആകാശത്തിന്റെ നീലിമ സത്യമല്ലെന്നുള്ള അറിവുപോലെ അതുള്ളില്‍ സുവിദിതമാകണം. ദ്വന്ദത എന്ന ധാരണ ഏകതയെക്കുറിച്ചുള്ള മുന്‍ ധാരണയില്‍ അധിഷ്ഠിതമാണ്‌. അതുപോലെ അദ്വൈതം (രണ്ടില്ല) എന്ന ധാരണ ദ്വൈതസങ്കല്‍പ്പത്തെ അടിസ്ഥാനമാക്കിയുള്ളതാണ്‌. സൃഷ്ടിയെന്നത്‌ ഉണ്മയേ അല്ലെന്ന് നിശ്ശേഷം ബോദ്ധ്യമായാലേ ഈശ്വരനെ സാക്ഷാത്കരിക്കുവാന്‍ ആവൂ.
