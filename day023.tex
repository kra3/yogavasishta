\newpage
\section{ദിവസം 023}

\slokam{
പരം പൗരുഷമാശ്രിത്യ ദന്തൈർദന്താനവിചൂർണ്ണയൻ\\
ശുഭേനാശുഭം ഉദ്യുക്തം പ്രാക്ത്തനം പൗരുഷം ജയേത് (2/5/9)\\
}

വസിഷ്ഠന്‍ തുടര്‍ന്നു: രാമ: ജലപ്പരപ്പില്‍ ഓളങ്ങള്‍ ഉള്ളപ്പോഴും ഇല്ലാത്തപ്പോഴും ജലത്തിന്റെ സ്വഭാവത്തിന്‌ മാറ്റമേതുമില്ലാത്തതുപോലെ മുക്തപുരുഷന്‍ കാഴ്ച്ചയില്‍ എങ്ങിനെയിരുന്നാലും തന്നിലുറച്ച വിജ്ഞാനത്തിനു മാറ്റമൊന്നും ഉണ്ടാവുന്നില്ല. കാണുന്നവന്റെ അജ്ഞതയുടെ നിലവാരമനുസരിച്ചാണ്‌ ഭേദങ്ങളെല്ലാം തോന്നുന്നത്‌. അതിനാല്‍ അജ്ഞാനാന്ധകാരത്തെ ഇല്ലാതാക്കുന്നതിനായി എന്റെ വാക്കുകള്‍ ശ്രദ്ധിച്ചു കേട്ടാലും. ഇഹലോകത്തിലെ നേട്ടങ്ങള്‍ എല്ലാം സ്വപ്രയത്നത്താല്‍ മാത്രമേ ഉണ്ടാവുന്നുള്ളു. പരാജയങ്ങള്‍ എല്ലാം പ്രയത്നത്തിന്റെ പോരായ്മകൊണ്ടാണെന്നറിയുക. ഇത്‌ എല്ലാവര്‍ക്കും അറിയാവുന്നതാണ്‌. എന്നാല്‍ വിധിയെന്നത്‌ സാങ്കല്‍പ്പികമാണ്‌. സ്വപ്രയത്നം എന്നാല്‍ മനസാ വാചാ കര്‍മ്മണാ, ശാസ്ത്രാനുസാരിയായും, മഹാത്മാക്കളുടെ നിര്‍ദ്ദേശപ്രകാരവും നാം ചെയ്യുന്ന പ്രവൃത്തികളാണ്‌. അത്തരം പ്രവര്‍ത്തനങ്ങളുടെ ഫലമായാണ്‌ ദേവരാജാവായ ഇന്ദ്രനും സൃഷ്ടികര്‍ത്താവായ ബ്രഹ്മാവിനും മറ്റു ദേവതമാര്‍ക്കും അതതു പദവികള്‍ ലഭിച്ചത്‌. 

സ്വപ്രയത്നം രണ്ടു തരത്തിലാണ്‌. ഈ ജന്മത്തിലേയും കഴിഞ്ഞ ജന്മങ്ങളിലേതും. ഈ ജന്മത്തിലെ പ്രയത്നങ്ങള്‍ കഴിഞ്ഞ ജന്മങ്ങളിലെ പ്രവര്‍ത്തനങ്ങള്‍ക്ക്‌ വിപരീതഫലമുളവാക്കുന്നു. വിധി എന്നതും സ്വപരിശ്രമം തന്നെ. കഴിഞ്ഞ ജന്മങ്ങളുടേതാണെന്നുമാത്രം. ഈ ജന്മത്തില്‍ രണ്ടും തമ്മില്‍ സംഘര്‍ഷങ്ങളുണ്ടാവുക സഹജവും തമ്മില്‍ ശക്തികൂടിയ പ്രയത്നത്തിനു വിജയം നിശ്ചിതവുമാണ്‌. 

ശാസ്ത്രാനുസാരിയല്ലാത്ത ഉദ്യമങ്ങള്‍ വ്യാമോഹത്താല്‍ പ്രചോദിതമാണ്‌. പ്രയത്നങ്ങള്‍ക്ക്‌ സദ്ഫലം ലഭിക്കാതെവരുമ്പോള്‍ ആ പ്രവര്‍ത്തനങ്ങള്‍ക്ക്‌ പ്രചോദനമായിരുന്നത്‌ വ്യാമോഹങ്ങളാണോ എന്നു പരിശോധിക്കുക. എന്നിട്ട്‌ ഉചിതമായ മാറ്റങ്ങള്‍ ഉടനേ തന്നെ സ്വാംശീകരിക്കുക. വര്‍ത്തമാനകാലത്തു ചെയ്യുന്ന ഉചിതമായ പ്രയത്നത്തേക്കാള്‍ പ്രബലവും പ്രസക്തവുമായി മറ്റൊന്നില്ല.

"അതുകൊണ്ട്‌ പല്ലുഞരിച്ച്‌ ചെയ്യുന്ന കഠിനമായ സ്വപ്രയത്നത്താല്‍ ഒരുവന്‌ അവന്റെ നല്ലതും ചീത്തയുമായ എല്ലാ വിധികളേയും ഇപ്പോഴത്തെ പ്രവര്‍ത്തികള്‍കൊണ്ട്‌ തരണം ചെയ്യാം" മടിയന്‍ കഴുതയേക്കാള്‍ നികൃഷ്ടനത്രേ. ആയുസ്സ്‌ അനുനിമിഷം കുറഞ്ഞുകൊണ്ടിരിക്കുന്നു എന്ന ബോധത്തില്‍ മോക്ഷപ്രാപ്തിക്കായി കഠിനയത്നത്തില്‍ നാം മടികൂടാതെ മുഴുകുക തന്നെവേണം. മലിനജലത്തിലും ചലത്തിലും കിടന്നുല്ലസിച്ചു പുളയ്ക്കുന്ന പുഴുക്കളെപ്പോലെ ഇന്ദ്രിയസുഖമെന്ന ചെളിക്കുണ്ടില്‍ വീഴാതെ ജാഗരൂകരായിരിക്കണം.

"വിധിയാണ്‌ എന്നേക്കൊണ്ടിങ്ങനെ ചെയ്യിക്കുന്നത്‌" എന്നു പറയുന്നവന്‍ ബുദ്ധിഹീനനാണ്‌. അവനെ ഭാഗ്യദേവത തിരിഞ്ഞു നോക്കുകപോലുമില്ല. അതുകൊണ്ട്‌ സ്വപ്രയത്നം കൊണ്ട്‌ ആത്മജ്ഞാനം നേടി ഈ പ്രയത്നങ്ങള്‍ക്കെല്ലാം സത്യസാക്ഷാത്കാരമാകുന്ന ഫലപ്രാപ്തിയുണ്ടെന്നു തിരിച്ചറിയുക. ദുഷ്ടതയുടെ ഉറവയായ 'മടി' ലോകത്തിലില്ലെങ്കില്‍ ദാരിദ്ര്യവും നിരക്ഷരതയും ആര്‍ക്കുണ്ടാവും? മനുഷ്യന്‍ മൃഗങ്ങളേപ്പോലെ കഷ്ടപ്പെട്ട്‌ ദുരിതത്തിലും ദാരിദ്ര്യത്തിലും കഴിയാന്‍, ഈ മടിതന്നെയാണ് കാരണം. 

വാല്‍മീകി പറഞ്ഞു: അനന്തരം സായാഹ്ന പ്രാത്ഥനകള്‍ക്കു സമയമാകയാല്‍ സഭ അന്നേയ്ക്കു പിരിഞ്ഞു. 
