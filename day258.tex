\section{ദിവസം 258}

\slokam{
ആരാദ്ധ്യാത്മനാത്മാനമാത്മനാത്മാനമര്‍ച്ചയാ\\
ആത്മനാത്മനമാലോക്യ സംതിഷ്ഠസ്വാത്മനാത്മനി (5/43/19)\\
}

രാമന്‍ ചോദിച്ചു: മഹാത്മന്‍, അങ്ങു പറഞ്ഞു വിഷ്ണുഭഗവാന്റെ കൃപയാലാണ് പ്രഹ്ലാദന് പ്രബുദ്ധത കൈവന്നതെന്ന്. എല്ലാക്കാര്യങ്ങളും സ്വപരിശ്രമങ്ങളാലാണ് സാധിക്കുകയെങ്കില്‍ വിഷ്ണു കൃപ കൂടാതെ തന്നെ അദ്ദേഹത്തിനു പ്രബുദ്ധനാവാന്‍ കഴിയുമായിരുന്നില്ലേ?

വസിഷ്ഠന്‍ പറഞ്ഞു: തീര്‍ച്ചയായും പ്രഹ്ലാദന്‍ എന്തെല്ലാം നേടിയോ അതെല്ലാം സ്വപ്രയത്നഫലമായാണ്. കാരണം വിഷ്ണു ആത്മാവാണ്. ആത്മാവ് വിഷ്ണുവും. വാക്കുകള്‍ അവയുടെ സ്വതവേയുള്ള പരിമിതികളോടെ ഉപയോഗിക്കുമ്പോള്‍ ഉണ്ടാവുന്ന ധാരണാ വ്യത്യാസമാണ് ഈ ചിന്താക്കുഴപ്പത്തിന് കാരണം. പ്രഹ്ലാദനിലെ ആത്മാവാണല്ലോ അദ്ദേഹത്തില്‍ വിഷ്ണുഭക്തി ഉണ്ടാക്കിയത്. പ്രഹ്ലാദന്‍ വിഷ്ണുഭാഗവാനില്‍ നിന്നും, അതായത് തന്റെ തന്നെ ആത്മാവില്‍ നിന്നും ആത്മാന്വേഷണത്വര ഉണ്ടാകണം എന്നൊരു വരമാണ്  വാങ്ങിയത്. കാരണം അപ്രകാരമുള്ള അന്വേഷണമാണല്ലോ ആത്മജ്ഞാനത്തില്‍ കലാശിക്കുക.

ചിലപ്പോള്‍ ഇപ്രകാരം ഉള്ള സ്വപ്രയത്നത്താല്‍ ജ്ഞാനോദയമുണ്ടാവും. മറ്റു ചിലപ്പോള്‍ ഈ പ്രയത്നം വിഷ്ണുവിനോടുള്ള തീവ്രഭക്തിയായും പ്രകടമാവും അങ്ങിനെയും പ്രബുദ്ധത കൈവരിക്കാം. ഏറെക്കാലം വിഷ്ണുഭക്തിയോടെ ജീവിച്ചാലും ആത്മജ്ഞാനനിരതനല്ലാത്ത ഒരാള്‍ക്ക് പ്രബുദ്ധതയെന്ന വരം അദ്ദേഹം കൊടുക്കുകയില്ല. അതുകൊണ്ട് ആത്മജ്ഞാനത്തിനു ആദ്യമായി വേണ്ടത് ആത്മാന്വേഷണം തന്നെയാണ്. ആനുഗ്രഹം, കൃപ തുടങ്ങിയ കാര്യങ്ങളെല്ലാം രണ്ടാമത് വരുന്നവയാണ്. അതുകൊണ്ട് ഇന്ദ്രിയങ്ങളെ നിയന്ത്രിച്ച് ആത്മീയപാതയില്‍ സഞ്ചരിച്ച് മനസ്സിനെ അത്മാന്വേഷണത്തിലേയ്ക്ക് ഉന്മുഖമാക്കിയാലും.

സ്വപ്രയത്നത്തെ ആശ്രയിച്ച് സംസാരസാഗരത്തെ കടന്നു മറുകരയെത്തുക. സ്വപ്രയത്നം കൂടാതെ തന്നെ വിഷ്ണുദര്‍ശനം കിട്ടുമെന്നാണ് നീ കരുതുന്നതെങ്കില്‍ പക്ഷിമൃഗാദികള്‍ക്ക് ആ ദര്‍ശനം അദ്ദേഹം നല്‍കാത്തതെന്തുകൊണ്ടാണ്? ഗുരുകൃപ കൊണ്ട് മാത്രം ഒരുവനെ ഉയര്‍ത്തിക്കൊണ്ടുവരാം എന്നുണ്ടെങ്കില്‍ ആ ഗുരു എന്തുകൊണ്ടാണ് ഒരു കാളയ്ക്കോ ഒട്ടകത്തിനോ ഈ സൌഭാഗ്യം നല്‍കാത്തത്? മനസ്സിനെ പൂര്‍ണ്ണമായി അടക്കി സ്വപ്രയത്നം കൊണ്ട് മാത്രമേ എന്തും നേടുവാനാവൂ. ഈശ്വരന്‍, ഗുരു, ധനം എന്നല്ല, മറ്റൊന്നിനും സ്വപ്രയത്നത്തിനു പകരം നില്‍ക്കാനാവില്ല. എല്ലാ മാനസീകോപാധികളുടേയും നിറഭേദങ്ങളൊഴിഞ്ഞ മനസ്സും ഉറച്ച ആത്മനിയന്ത്രണവും വഴി നേടാന്‍ കഴിയാത്ത ആ തലം മറ്റൊരു മാര്‍ഗ്ഗം വഴിയും സ്വായത്തമാക്കാന്‍ കഴിയുകയില്ല തന്നെ.

"അതുകൊണ്ട് ആത്മാവിനെ ആത്മാവില്‍ത്തന്നെ ആദരിക്കുക. പൂജിക്കുക. അങ്ങിനെ ആത്മാവില്‍ത്തന്നെ ആത്മാവായി സ്വയം അടിയുറപ്പിക്കുക.” ആത്മാന്വേഷണ പാതയില്‍ നിന്നും വ്യതിചലിച്ചും, ശാസ്ത്രപഠനത്തില്‍ വിമുഖരായും നില്‍ക്കുന്നവരെ നന്മയിലേയ്ക്ക് തിരിക്കാനാവണം വിഷ്ണു ഭഗവാനോടുള്ള ഭക്തിയും അതിനുള്ള പൂജകളും ഒരു പ്രസ്ഥാനമായി മഹാത്മാക്കള്‍ സ്ഥാപിച്ചിരിക്കുന്നത്. നിസ്തന്ദ്രമായ, മനസ്സുറപ്പുള്ള, ആത്മാന്വേഷണമാണ് ഉത്തമം. അതില്ലാത്ത പക്ഷം മറ്റു മാര്‍ഗ്ഗങ്ങളാവാം എന്ന് മാത്രം.  

ഇന്ദ്രിയങ്ങള്‍ പൂര്‍ണ്ണമായും വരുതിയിലായാല്‍പ്പിന്നെ  പൂജാദികള്‍കൊണ്ടെന്തു കാര്യം? ആത്മാന്വേഷണം വഴിയുണ്ടാവുന്ന പ്രശാന്തത കൂടാതെ വിഷ്ണുഭക്തിയോ ആത്മജ്ഞാനമോ ഉണ്ടാവുക അസാദ്ധ്യം. അതുകൊണ്ട് ആത്മാന്വേഷണം തുടരുക. അനാസക്തി പരിശീലിക്കുക. അങ്ങിനെ ആത്മാവിനെ പൂജിക്കുക. ഇതില്‍ എത്രത്തോളം വിജയിക്കുന്നുവോ നിനക്ക് അത്രതന്നെ പൂര്‍ണ്ണത്വം പ്രാപിക്കാം. അതില്ലെങ്കില്‍ നിന്റെ ജീവിതം ഒരു കാട്ടുകഴുതയുടേതില്‍ നിന്നും വിഭിന്നമല്ല.

