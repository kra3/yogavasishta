\newpage
\secstar{അവതാരിക}

വിദ്വാന്മാര്‍ ഈ ബൃഹദ്ഗ്രന്ഥത്തിന്റെ രചയിതാവിനെപ്പറ്റി പല അനുമാനങ്ങളും ഊഹിച്ചു പറയുന്നുണ്ട്‌.. അവയെല്ലാം ഗവേഷണവിഷയങ്ങളായതുകൊണ്ട്‌ എല്ലാ ഗവേഷണകുതുകികള്‍ ക്കും ഞാന്‍ വിജയമാശംസിക്കുന്നു.

യോഗവാസിഷ്ഠം ആത്മസാക്ഷാത്കാരത്തിന്‌ ഏറ്റവും സഹായകരമായ സത്യത്തിന്റെ നേരനുഭവമത്രേ. അതാണ്‌ നിങ്ങള്‍ തേടുന്നതെങ്കില്‍ യോഗവാസിഷ്ഠത്തിലേയ്ക്കു സുസ്വാഗതം. ഇതില്‍ പലകാര്യങ്ങളും ആവര്‍ത്തിച്ചു പറഞ്ഞിട്ടുണ്ടെങ്കിലും അവ ആവര്‍ത്തനവിരസങ്ങളല്ല. നിങ്ങള്‍ക്ക്‌ ഇങ്ങിനെയുള്ള   ആവര്‍ത്തനം പ്രിയമല്ല എന്നുണ്ടെങ്കില്‍ ഈ ഒരു ശ്ലോക സംഗ്രഹം  മാത്രം വേണ്ടപോലെ പഠിച്ചാല്‍ മതി. " ആകാശത്തിന്റെ നീലിമ ഒരു ദൃശ്യസംഭ്രമം മാത്രമാണെന്നതുപോലെ ഈ കാണപ്പെടുന്ന ലോകം ആകെ ചിന്താക്കുഴപ്പം പിടിച്ചതാണ്‌ - അതിനാല്‍ എനിക്കു തോന്നുന്നത്‌ ഈ ലോകത്തിനെപ്പറ്റി അധികം വിചിന്തനം ചെയ്ത്‌ മനസ്സു ഭ്രമിപ്പിക്കുന്നതിനു പകരം അതിനെ നമ്മുടെ ചിന്തയില്‍ പ്പെടുത്താതിരിക്കുകയാണു നല്ലത്‌ എന്നാണ്‌.".  ഈ ശ്ലോകം പലയിടത്തും ആവര്‍ത്തിച്ചു കൊടുത്തിട്ടുണ്ട്‌. കാരണം യോഗവാസിഷ്ഠത്തിന്റെ കാതലായ സന്ദേശം ഇതാണെന്നു തോന്നുന്നു. ഇത്‌ കൃത്യമായും മനസ്സിലായില്ലെങ്കില്‍ ഗ്രന്ഥം മുഴുവനും പഠിക്കുക. പലരീതികളില്‍ ഈ സത്യത്തെ വെളിപ്പെടുത്തിയിരിക്കുന്നത്‌ നിങ്ങളുടെ മനസ്സു തുറക്കാനുതകും. ദിവസവും ഈ ഗ്രന്ഥത്തിന്റെ ഒരു പുറം വീതം വായിച്ചാല്‍ മതി. പഠനവിഷയം അതിഗഹനമായതുകൊണ്ട്‌ മുന്‍വിധികളുള്ള മനസ്സില്‍ ഈ  വിഷയം സ്വീകരിക്കപ്പെടുകയില്ല. നിത്യവും പാരായണശേഷം ധ്യാനിക്കൂ. അതിലെ സന്ദേശം ഉള്ളിലാഴ്ന്നിറങ്ങട്ടെ.

"കാകതാലീയം" എന്നൊരു കല്‍പന ഇതില്‍ പലയിടത്തും കാണാം. ഒരു കാക്ക തെങ്ങോലയില്‍ ഇരിക്കുന്ന അതേമാത്രയില്‍ത്തന്നെ ഒരു കൊട്ടത്തേങ്ങ കാക്കയുടെ തലയില്‍ വീഴുന്നു. രണ്ടു സംഭവങ്ങള്‍ക്കും തമ്മില്‍ കാലദേശാനുസാരിയായോ കാരണപരമായോ യാതൊരു ബന്ധവുമില്ല, എന്നാല്‍ അവ തമ്മില്‍ ബന്ധപ്പെട്ടിരിക്കുന്നതായി നമുക്കനുഭവപ്പെടുന്നു. ജീവിതവും സൃഷ്ടിയും അപ്രകാരമത്രേ. എന്നാല്‍ മനസ്സ്‌ സ്വയംകൃതമായ യുക്തിയിലും ചോദ്യങ്ങളിലും കുടുങ്ങി 'എന്തുകൊണ്ട്‌?', 'എവിടെനിന്ന്' എന്നെല്ലാം അന്വേഷിച്ച്  സ്വയം സംതൃപ്തിപ്പെടാന്‍ വേണ്ടി ചില ഉത്തരങ്ങളൂം കണ്ടെത്തുന്നു. എന്നാല്‍ ഉള്ളില്‍ അപ്പോഴും ബുദ്ധിക്ക്‌ അലോസരമുണ്ടാക്കുന്ന ചോദ്യങ്ങള്‍ ബാധപോലെ അവശേഷിക്കുന്നു. വസിഷ്ഠമുനി ആവശ്യപ്പെടുന്നത്‌ മനസ്സിനെ അതിന്റെ വ്യാപാരങ്ങളോടെ നേരേ നിരീക്ഷിക്കാനാണ്‌. അതിന്റെ ഗതിവിഗതികള്‍ , അനുമാനങ്ങള്‍ , നിഗമനങ്ങള്‍ , ഫലങ്ങള്‍ എന്നിവ മാത്രമല്ല നിരീക്ഷിക്കുക എന്ന പ്രവര്‍ത്തിയേപ്പോലും നിരീക്ഷിക്കുക എന്നാണ്‌ ആഹ്വാനം. അങ്ങിനെ അനന്തവും അവിഛിന്നവുമായ ബോധസ്വരൂപത്തെ സാക്ഷാത്കരിക്കാം.

ഈ വേദഗ്രന്ഥം സ്വയം അതിന്റെ പരമോത്കൃഷ്ടതയെപ്പറ്റി ഉദ്ഘോഷിക്കുന്നത്‌ തികച്ചും അതുല്യമായ രീതിയിലത്രേ. "ഈ ഗ്രന്ഥത്തിലൂടെയല്ലാതെ ഒരുവന്‌ സദ്‌വസ്തുബോധം ഒരുകാലത്തും ഉണ്ടാവുകയില്ല. അതുകൊണ്ട്‌ പരമസാക്ഷാത്കാരത്തിനായി ഇതിലെ പാഠങ്ങള്‍ ആവേശത്തോടെ വിചിന്തനം ചെയ്യേണ്ടതാണ്‌.. യാതൊരു ഗ്രന്ഥവും മഹര്‍ഷിയും പഠനവിഷയ ത്തേക്കാള്‍ മഹത്തരമല്ല. അതിനാല്‍ വസിഷ്ഠമുനി സധൈര്യം പറയുന്നു: "ഇതു മനുഷ്യന്റെ സൃഷ്ടിയായതുകൊണ്ട്‌ ആധികാരികമല്ല എന്ന് കരുതുന്ന ആള്‍ക്ക്‌ ആത്മജ്ഞാനത്തെപ്പറ്റിയും പരമസാക്ഷാത്കാരമായ മുക്തിയെപ്പറ്റിയും പ്രതിപാദിക്കുന്ന മറ്റേതു ഗ്രന്ഥങ്ങളേയും ആശ്രയിക്കാവുന്നതാണ്‌." (6.2.175.

ഏതു വേദഗ്രന്ഥം പഠിച്ചാലും ആരു പഠിപ്പിച്ചാലും ഏതു പാത സ്വീകരിച്ചാലും മാനസീകോപാധികള്‍ പരിപൂര്‍ണ്ണമായി അവസാനിക്കുംവരെ  അന്വേഷണം   നിര്‍ത്തരുത്. അതുകൊണ്ട്‌ വസിഷ്ഠമുനി  പറയുന്നു: 'ഒരുവന്‍ ഈ വേദശാസ്ത്രത്തിന്റെ ചെറിയൊരു ഭാഗമെങ്കിലും ദിവസവും പഠിക്കണം. ഇതിന്റെ പ്രത്യേകത എന്തെന്നാല്‍ പഠിതാവിനെ പാതിവഴിയില്‍ ആശയക്കുഴപ്പത്തോടെ ഇതുപേക്ഷിക്കുന്നില്ല. ആദ്യവായനയില്‍ ചില ആശയങ്ങള്‍ വ്യക്തമായില്ലെങ്കില്‍പ്പോലും തുടര്‍ന്നു പഠിക്കുന്നതിലൂടെ അവയുടെ ആന്തരാര്‍ത്ഥം ഉള്ളില്‍ തെളിഞ്ഞുവരുന്നതാണ്‌.' (6.2.175)
