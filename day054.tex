 
\section{ദിവസം 054}

\slokam{
ചിത്താകാശം ചിദാകാശം അകാശം ച തൃതീയകം\\
ദ്വാഭ്യാം ശൂന്യതരം വിദ്ധി ചിദാകാശം വരാനനേ (3/17/10)\\
}

അശരീരി, സരസ്വതീ ദേവിയുടെ സ്വരത്തില്‍ മൊഴിഞ്ഞു: "കുഞ്ഞേ രാജാവിന്റെ ശരീരം പൂക്കള്‍കൊണ്ടു മൂടുക. അതു ജീര്‍ണ്ണിക്കുകയില്ല. അദ്ദേഹം കൊട്ടാരം വിട്ടു പോവുകയില്ല"

ലീല അപ്രകാരം ചെയ്തു. എങ്കിലും അവള്‍ക്കു തൃപ്തിയായില്ല. ധനികനായ ഒരാളെ ധനം കയ്യിലുണ്ടായിട്ടും ആരോ തന്ത്രം പ്രയോഗിച്ച്‌ ദരിദ്രനേപ്പോലെ കഴിയാനിടയാക്കിയതുപോലെ അവള്‍ക്കു ആകെയൊരു പോരായ്മ തോന്നി. അവള്‍ സരസ്വതീ ദേവിയെ സ്മരിച്ചു വരുത്തി. ദേവി ചോദിച്ചു: 'കുഞ്ഞേ നിന്റെ ദു:ഖത്തിനു കാരണമെന്താണ്‌? ദു:ഖം മരീചികയിലെ ജലം പോലെ വെറും ഭ്രമം മാത്രമാണ്‌.'
ലീല ചോദിച്ചു. 'ദയവായി പറയൂ എന്റെ പ്രിയതമന്‍ എവിടെയാണ്‌?'
'കുഞ്ഞേ മൂന്നു തരം തലങ്ങളുണ്ട്‌- മാനസീകതലം, ഭൌതീകതലം പിന്നെ അനന്തമായ ബോധമെന്ന മറ്റൊന്ന്. ഇവയില്‍ ഏറ്റവും സൂക്ഷ്മമായത്‌ അനന്തമായ ബോധതലമാണ്‌.'  ഈ ബോധതലത്തിനെ തീവ്രമായ ധ്യാനംകൊണ്ട്‌ നിനക്ക്‌ കാണാനും അനുഭവിക്കാനുമാവും. എന്നാല്‍ അവിടെ നിന്റെ പ്രിയനേപ്പോലെയുള്ളവരുടെ ദേഹം അനന്തതയില്‍ വിലീനമായിരിക്കുന്നതിനാല്‍ അദ്ദേഹത്തെ ഒരു വ്യക്തിയായി കാണാന്‍ നിനക്കു കഴിയില്ല. അത്‌ അനന്തതയാണ്‌. പരിമിതികളുള്ള മേധാശക്തി ഒരിടത്തുനിന്ന് മറ്റൊരിടത്തേയ്ക്ക്‌ സഞ്ചരിക്കുമ്പോള്‍ അനന്തതയിലായതിനാല്‍ അത്‌ മദ്ധ്യത്തിലായി നിലകൊള്ളുന്നു. നീ എല്ലാ ചിന്തകളും ഉപേക്ഷിച്ചാല്‍ ഇപ്പോള്‍ ഇവിടെ വച്ച്‌ നിനക്കും ആ ഏകാത്മകത്വം അനുഭവിക്കാം. വിശ്വനിര്‍മ്മിതിയുടെ അയാഥാര്‍ത്ഥ്യത്തെപ്പറ്റി തികഞ്ഞ അറിവുറച്ചവര്‍ക്കുമാത്രമേ ഈ ദര്‍ശനം സാക്ഷാത്കരിക്കുവാനാവൂ. എങ്കിലും എന്റെ കാരുണ്യത്താല്‍ നിനക്കീ ദര്‍ശനം സാദ്ധ്യമാണ്‌.'

വസിഷ്ഠന്‍ തുടര്‍ന്നു: ലീല ധ്യാനനിരതയായി. എല്ലാ ആകര്‍ഷണങ്ങളില്‍ നിന്നും വിമുക്തയായി അവള്‍ പെട്ടെന്നുതന്നെ നിര്‍വികല്‍പ്പസമാധിയില്‍ , ബോധത്തിന്റെ ഉന്നത തലത്തില്‍ എത്തിച്ചേര്‍ ന്നു. അവള്‍ അനന്തബോധത്തില്‍ ആമഗ്നയായി. അവിടെ അവള്‍ രജാവിനെ വീണ്ടും കണ്ടു. സിംഹാസനനസ്ഥനായി വിരാജിക്കുന്ന അദ്ദേഹത്തിനുചുറ്റും മറ്റു രാജാക്കന്മാര്‍ അദ്ദേഹത്തെ ആരാധനയോടെ നോക്കുന്നു. മഹര്‍ഷികളും മാമുനിമാരും വേദാലാപനം ചെയ്യുന്നു. ആയുധമേന്തിയ സൈന്യങ്ങളും സ്ത്രീകളും അദ്ദേഹത്തെ പരിചരിക്കുന്നു. അവള്‍ അവരെ കണ്ടെങ്കിലും അവരാരും അവളെ കണ്ടില്ല. കാരണം ഒരാളുടെ ചിന്താകല്‍പ്പനകള്‍ അയാള്‍ക്കു മാത്രമാണ്‌ ഗോചരമായിട്ടുള്ളത്‌. രാജാവിന്‌ യൌവ്വനയുക്തമായ ശരീരമാണുള്ളതെന്ന് അവള്‍ കണ്ടു. പദ്മരാജാവിന്റെ സഭയിലെ പലരേയും അവളവിടെ കണ്ടു. അവള്‍ അത്ഭുതം കൊണ്ടു: 'അവരും മരണത്തിനടിപ്പെട്ടുകഴിഞ്ഞോ!?' 

പിന്നെ സരസ്വതീദേവിയുടെ കൃപകൊണ്ട്‌ അവള്‍ വീണ്ടും കൊട്ടാരത്തിലെത്തിയപ്പോള്‍ കൊട്ടാരസേവകരെല്ലാം ഉറക്കത്തിലാണെന്നു കണ്ടു. അവരെ ഉണര്‍ത്തി കൊട്ടാരത്തിലെ രാജസഭ ഉടനേ വിളിച്ചുകൂട്ടാന്‍ രാജ്ഞി കല്‍പ്പിച്ചു. ദൂതന്മാര്‍ എല്ലാവരേയും വിളിക്കാനായി പോയി. താമസം വിനാ പദ്മരാജാവിന്റെ സഭ മന്ത്രിമാരാലും മഹര്‍ഷിമാരാലും ഉദ്യോഗസ്ഥരാലും ബന്ധുമിത്രാദികളാലും നിറഞ്ഞതുകണ്ട്‌ രാജ്ഞി സന്തോഷം പ്രകടിപ്പിച്ചു.
