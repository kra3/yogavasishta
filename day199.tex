\section{ദിവസം 199}

\slokam{
തവ തുല്യമതിര്യ: സ്യാത്സുജന: സമദർശന:\\
യോഗ്യൗഽസൗ ജ്ഞാനദൃഷ്ടീനാം മയോക്താനാം സുദൃഷ്ടിമാൻ (4/62/9)\\
}

വസിഷ്ഠൻ തുടർന്നു: രാമ, ബുദ്ധിയും വിവേകവുമുള്ള ഒരുവൻ സത്യാന്യോഷണത്തിനുള്ള പ്രാപ്തിയുള്ളപക്ഷം സദ് വൃത്തനും ജ്ഞാനിയുമായ ഒരാളെ സമീപിച്ച് വേദശാസ്ത്രപഠനം ചെയ്യണം. സത്യസാക്ഷാത്കാരത്തിന്റെ നേരനുഭവം കിട്ടിയിട്ടുള്ള ഈ ഗുരു ലൗകീകസുഖാസക്തികളെ വിജയിച്ചയാളായിരിക്കണം. അദ്ദേഹത്തിന്റെ സഹായത്തോടെയാണ്‌ ശാസ്ത്രം പഠിക്കേണ്ടത്. മഹത്തായ യോഗാഭ്യാസസാധനയിലൂടെ ഒരാൾക്ക് പരമപദപ്രാപ്തി സാധ്യമാണ്‌. രാമാ, നിനക്ക് ആത്മീയകാര്യങ്ങളിൽ നെടുനായകസ്ഥാനമുണ്ടല്ലോ. നീയാണെങ്കില്‍ നന്മയുടെ ഇരിപ്പിടവുമാണ്‌. ശോകത്തിൽ നിന്നും നിനക്ക് മുക്തിയും ലഭിച്ചിരിക്കുന്നു. സമതാ ദർശനം നിനക്കു സ്വായത്തം. ബോധത്തിന്റെ, മേധാശക്തിയുടെ, ഏറ്റവും ഉയർന്നതലത്തിൽ വിരാജിക്കുന്ന നിനക്ക് മായാമോഹങ്ങളെ ഉപേക്ഷിക്കാനാവും. ലോകവസ്തുക്കളെപ്പറ്റിയുള്ള എല്ലാ ആശങ്കകളും അവസാനിക്കുമ്പോൾ നിനക്ക് അദ്വൈതമായ അനന്താവബോധത്തെ സാക്ഷാത്കരിക്കാനാവും. അതാണ്‌ മുക്തിപദം. ഇതിനു സംശയമേതുമില്ലെന്നറിയുക. ആത്മവിദ്യാനിരതരായ മാമുനിമാർപോലും നിന്നെ അനുകരിക്കും. നീ കാണിച്ച പാത പിന്തുടരും.

“രാമാ, നിന്നെപ്പോലെ ബുദ്ധിമാനും, സമദർശനം പരിശീലിച്ചവനും, നന്മയെ വ്യതിരിക്തമായി അറിഞ്ഞു ദർശിക്കുന്നവനും മാത്രമേ ഞാനീപ്പറഞ്ഞ വിവേകദർശനവും വിജ്ഞാനവും ലഭിക്കുകയുള്ളു. "  രാമാ, ഈ ശരീരമുള്ളിടത്തോളം കാലം ഇഷ്ടാനിഷ്ടങ്ങളുടെ ചാഞ്ചല്യത്തിനടിമയാവാതെ, ആകർഷണവികർഷണങ്ങൾക്ക് വശംവദനാകാതെ, നീ ജീവിക്കുന്ന സമൂഹത്തിന്റെ പെരുമാറ്റരീതിക്കനുസൃതമായി ജീവിക്കൂ. എന്നാൽ ആശകളും ആസക്തികളും നിന്നെ ബാധിക്കാതെ നോക്കുക. പരമസത്യമെന്തെന്നറിയാൻ നിരന്തരം പരിശ്രമിക്കുക. ദിവ്യരായ മഹർഷിമാരങ്ങിനെയാണ്‌..

മഹാത്മാക്കളുടെ പെരുമാറ്റരീറ്റി പിന്തുടരുന്നത് പരമപദത്തിലേയ്ക്കുള്ള പാത സുഗമമാക്കും. അങ്ങിനെയാണ്‌ വിവേകികൾ തങ്ങളുടെ ലക്ഷ്യപ്രപ്തിയിലേക്കുള്ള പുരോഗതി ഉറപ്പുവരുത്തുന്നത്. ഈ ജീവിതത്തിലെ സ്വഭാവങ്ങളാണ്‌ ഈ ജീവിതകാലം കഴിഞ്ഞാലും ഒരു ജീവനു ‘സമ്പാദ്യ’മായി ഉണ്ടാവുക. എന്നാൽ ഇത്തരം പ്രവണതകളിൽ നിന്നും വാസനകളില്‍ നിന്നും രക്ഷപ്പെടാൻ ആരോണോ കഠിനമായി പരിശ്രമിക്കുന്നത്, അതിനു ഫലമുണ്ടാവുകതന്നെ ചെയ്യും. തമസ്സിൽ നിന്നും മനോമാന്ദ്യത്തിൽനിന്നും കരേറി ശുദ്ധമാവാനായി അയാൾ പരിശ്രമിക്കണം. വിവേകവിജ്ഞാനങ്ങളോടെയുള്ള ഉചിതമായ പരിശ്രമങ്ങൾ ഒരുവനെ നിർമ്മലതയുടെയും പ്രബുദ്ധതയുടെയും ഉന്നതശൃംഗങ്ങളിലേയ്ക്ക് നയിക്കുന്നു.

കഠിനപരിശ്രമംകൊണ്ടേ നല്ലൊരു ശരീരം ലഭിക്കൂ. തീവ്ര പരിശ്രമംകൊണ്ടു സാധിക്കാത്തതായി ഒന്നുമില്ല. ബ്രഹ്മചര്യമനുഷ്ഠിച്ച്, ധീരതയോടെ, സഹനശക്തിയോടെ, നിർമമതയോടെ, സാമാന്യബുദ്ധിയോടെയുള്ള അഭ്യാസങ്ങളിലൂടെ ഒരാൾക്ക് താൻ അന്വേഷിക്കുന്ന ആത്മവിദ്യ സാക്ഷാത്കരിക്കാം. രാമാ, നീ ഇപ്പോഴെ മുക്തിയെപ്രാപിച്ചവനാണ്‌.. അതിനുയോജിച്ചരീതിയിൽ ജീവിച്ചാലും!

സ്ഥിതി പ്രകരണം എന്ന നാലാം ഭാഗം അവസാനിച്ചു .

