\section{ദിവസം 250}

\slokam{
ഹംസി പാസി ദദാസി ത്വമവസ്ഫുര്‍ജസി വല്‍ഗസി\\
അനഹംകൃതിരൂപോഽപി ചിത്രേയം തവ മായിതാ (5/36/36)   \\
}

പ്രഹ്ലാദന്റെ ധ്യാനം തുടര്‍ന്നു: ആത്മാവേ നിന്റെ പരിശുദ്ധിയല്ലേ സൂര്യനില്‍ തിളങ്ങുന്നത്? നിന്നിലുള്ള അമൃതസമാനമായ ശീതളിമ ചന്ദ്രനില്‍ പരിലസിക്കുന്നു. പര്‍വ്വതങ്ങളുടെ ദാര്‍ഢ്യം, കാറ്റിന്റെ ഗതിവേഗം എല്ലാം നിന്നില്‍ നിന്നുദ്ഭവിച്ചത്. നീയുള്ളതിനാല്‍ ഭൂമിക്ക് ഉറപ്പും ആകാശത്തിനു ശൂന്യതയും സഹജം. ഭാഗ്യം കൊണ്ട് ഞാന്‍ നിന്നെ കണ്ടെത്തി. ഞാന്‍ നിന്റേതായി. ഭാഗ്യം കൊണ്ട് നാം തമ്മില്‍ യാതൊരന്തരവുമില്ല. ഞാന്‍ ആത്മാവാണ്. ആത്മാവ് ഞാനാണ്. ഞാനെന്നു പറയപ്പെടുന്നതും നീയെന്നു വിവക്ഷിക്കപ്പെടുന്നതും ഒന്നാണ്. അവയില്‍  വിത്തേതായാലും വൃക്ഷമേതായാലും ഞാനതിനെ വീണ്ടും വീണ്ടും നമസ്കരിക്കുന്നു.
     
അഹംകാരരഹിതവും അനന്തവുമായ ആത്മാവിനു നമോവാകം. രൂപരഹിതമായ ആത്മാവിനു നമസ്കാരം. ആത്മാവായി നീയെന്നില്‍ നിലകൊള്ളുമ്പോള്‍ ആകെ സമതുലനമാകുന്നു. അവിടെ കാലദേശാനുബന്ധമില്ലാത്ത, പരിമിതികളില്ലാത്ത ശുദ്ധമായ സാക്ഷീബോധമുദിക്കുന്നു. പിന്നെ മനസ്സ് ചഞ്ചലപ്പെട്ട് ഇന്ദ്രിയങ്ങള്‍ക്കിളക്കമുണ്ടാവുന്നു. അങ്ങിനെ ഊര്‍ജ്ജപ്രവാഹമുണ്ടായിട്ട്  പ്രാണന്‍, അപാനന്‍ എന്നീ രണ്ടുജീവശക്തികള്‍ ചലനം തുടങ്ങുന്നു. ആശകളുടെ ആകര്‍ഷണത്തില്‍പ്പെട്ട് മനസ്സെന്ന സാരഥി മാംസാസ്ഥിരക്തചര്‍മ്മ നിര്‍മ്മിതമായ ശരീരത്തെ മൂര്‍ത്തീകരിക്കുന്നു.
    
ഞാനേതായാലും ശുദ്ധ അവബോധമാണ്. ഞാന്‍ ശരീരമായോ മറ്റെന്തെങ്കിലുമായോ ബന്ധനത്തിലും ബന്ധത്തിലുമല്ല. ആഗ്രഹങ്ങളുടെ ചഞ്ചലതയ്ക്കനുസരിച്ച് ദേഹം വളരുകയോ തളരുകയോ ചെയ്യട്ടെ. വിശ്വസൃഷ്ടിയും പ്രളയവും എന്നപോലെ കാലക്രമത്തില്‍ അഹംഭാവം ഉയര്‍ന്നുപൊങ്ങി ക്രമേണ അതിനു നാശവും വന്നുകൊള്ളട്ടെ. എന്നാല്‍ ഏറെത്തവണ ആവര്‍ത്തിച്ച ജനനമരണ ചക്രത്തിനൊടുവിലാണെനിക്ക് ഈ പ്രശാന്തി കൈവന്നത്. ഈ പ്രപഞ്ചവിശ്വവും സ്വന്തം ചാക്രികപഥങ്ങള്‍ പലതവണ താണ്ടിയശേഷം അതിനു വിരാമമായി വിശ്രമിക്കുന്നുണ്ടല്ലോ. എല്ലാത്തിനും  അതീതമായി വര്‍ത്തിക്കുന്ന, എല്ലാമെല്ലാമായ, എനിക്കും നിനക്കും നമസ്കാരങ്ങള്‍. നമ്മെക്കുറിച്ചു സംസാരിക്കുന്ന എല്ലാവര്ക്കും നമസ്കാരം.

സാക്ഷീഭാവത്തിലുള്ള പരമപുരുഷനെ അതുമായി ബന്ധപ്പെട്ടെന്ന് പറയപ്പെടുന്ന  അനുഭവങ്ങളിലെ ഏറ്റക്കുറച്ചിലുകള്‍ തീരെ ബാധിക്കുന്നില്ല. പൂക്കളില്‍ സുഗന്ധംപോലെ എള്ളുമണിയില്‍ എണ്ണപോലെ, ആത്മാവ് എല്ലാടവും എപ്പോഴും എല്ലാറ്റിലും നിലകൊള്ളുന്നു.

“അല്ലയോ ആത്മാവേ, അഹംഭാവരഹിതമെങ്കിലും നീ സര്‍വ്വ സ്വാതന്ത്ര്യത്തോടെ വിരാജിക്കുന്നു. നീ നല്‍കുകയും, നശിപ്പിക്കുകയും, അലറിയുറയുകയും, സംരക്ഷിക്കുകയും ചെയ്യുന്നു. തീര്‍ച്ചയായും ഇതൊരത്യദ്ഭുതം തന്നെ.” ആത്മാവിന്റെ വെളിച്ചമായതിനാല്‍ ഞാന്‍ കണ്ണുതുറക്കുമ്പോള്‍ പ്രപഞ്ചമുണരുന്നു. കണ്ണടയ്ക്കുമ്പോള്‍ ലോകം ഇല്ലാതാവുകയും ചെയ്യുന്നു. വലിയൊരാല്‍മരത്തിന്റെ വിത്തില്‍ ആ വന്മരം നേരത്തേതന്നെ ഒരു സാദ്ധ്യതയായി ഉള്ളതുപോലെ   ആത്മാവേ, നീയെന്ന അണുവില്‍ ലോകംമുഴുവനും നിലകൊള്ളുന്നു. ആകാശത്തിലെ മേഘവിതാനങ്ങളില്‍ ആന, കുതിര, മറ്റ് മൃഗങ്ങള്‍ എന്നിവയെ കാണാനാകുമെന്നതുപോലെ നീ അനന്തവിഹായസ്സില്‍ എണ്ണമില്ലാത്ത ചരാചരവസ്തുക്കളായി, ഒന്നില്‍ നിന്നും മറ്റൊന്ന് തികച്ചും വ്യത്യസ്ഥമായി, വൈവിദ്ധ്യപൂര്‍ണ്ണമായി  കാണപ്പെടുന്നു. നീ സ്വയം ചരമോ അചരമോ അല്ലെങ്കിലും അങ്ങിനെയൊക്കെയാണ് കാണപ്പെടുന്നത്.
