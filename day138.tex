 
\section{ദിവസം 138}

\slokam{
ബീജജാഗ്രത്തഥാ ജാഗ്രൻ മഹജാഗ്രത് തഥൈവ ച\\
ജാഗ്രത്സ്വപ്നസ്തഥാ സ്വപ്ന: സ്വപ്നജാഗ്രത്സുഷുപ്തകം\\
ഇതി സപ്തവിധോ മോഹ: പുനരേവ പരസ്പരം (3/117/12)\\
}

വസിഷ്ഠൻ തുടർന്നു: ലവണ രാജാവിന്റെ കൊട്ടാരത്തിൽ ആ ജാലവിദ്യക്കാരൻ വന്നപ്പോൾ ഞാനും അവിടെയുണ്ടായിരുന്നു എന്നു പറഞ്ഞല്ലോ. സഭയിൽ നിന്നും പെട്ടെന്ന് അപ്രത്യക്ഷനായ അയാൾ ആരാണെന്നറിയാൻ സഭാവാസികൾക്ക് ആകാംക്ഷയായി. അയാൾ ഒരു ദേവദൂതനാണെന്ന് എന്റെ ദിവ്യദൃഷ്ടിയാൽ ഞാനറിഞ്ഞു . ഉന്നതമായ യാഗകർമ്മങ്ങൾ ചെയുന്നവരെ പരീക്ഷിക്കാനായി അവർക്കു പലതരത്തിലും ഉപദ്രവം നല്കുക ഇന്ദ്രന്റെ പതിവാണ്‌.. ലവണരാജാവ്‌ മാനസീകമായി യാഗമനുഷ്ഠിക്കുകയായിരുന്നല്ലോ. അദ്ദേഹത്തില്‍ മോഹവിഭ്രാന്തിയുണ്ടായത്  ഇങ്ങിനെയാണ്. യാഗകർമ്മങ്ങൾ നടത്തിയതും മനസ്സ്; ദുരിതം സഹിച്ചതും മനസ്സ്. ഇതേ മനസ്സ് നിർമ്മലമാവുമ്പോൾ അതുണ്ടാക്കിയ ദ്വന്ദഭാവവും നാനാത്വവും സ്വയം അപ്രത്യക്ഷമാവും. രാമാ, ചാക്രികമായി നടക്കുന്ന സൃഷ്ടിസർഗ്ഗത്തെപ്പറ്റി, അതായത്  കഴിഞ്ഞ വിശ്വപ്രളയശേഷം ഉള്ള കാര്യങ്ങളെപ്പറ്റി, ഞാൻ പറഞ്ഞുകഴിഞ്ഞു. ‘ഞാൻ’, ‘എന്റേത്’ എന്നീ തെറ്റിധാരണകൾ ഒരുവനിൽ അങ്കുരിക്കുന്നതെങ്ങിനെയെന്നും പറഞ്ഞു. ജ്ഞാനത്തിന്റെ വെളിച്ചത്തിൽ ആരൊരുവൻ ക്രമമായി യോഗമാർഗ്ഗത്തിലെ എഴു പടികൾതാണ്ടി പരിപൂർണ്ണതയിലെത്തുന്നുവോ അവന്‌ മുക്തിയായി.

രാമൻ ചോദിച്ചു : ഭഗവൻ, ഏതാണീ ഏഴുപടികൾ?

വസിഷ്ഠൻ പറഞ്ഞു: രാമ: അജ്ഞാനത്തിൽ താഴോട്ട് ഏഴും; വിജ്ഞാനത്തിൽ മുകളിലേയ്ക്ക് ഏഴും പടികളാണുള്ളത്. അതെപ്പറ്റി ഞാനിനി പറയാം. മനസ്സ് ആത്മജ്ഞാനത്തിൽ ദൃഢീകൃതമായാൽ മോക്ഷമായി. എന്നാല്‍ അതിലുണ്ടാവുന്ന ക്ഷോഭം അഹംകാരത്തിനും ബന്ധനത്തിനും കാരണമാവുന്നു. ആത്മജ്ഞാനാവസ്ഥയിൽ മനസ്സ് ക്ഷോഭങ്ങളൊഴിഞ്ഞ് പ്രശാന്തമാണ്‌.. മനസ്സിൽ മന്ദതയോ അസ്വസ്ഥതയോ, അഹംകാരമോ നാനാത്വധാരണകളോ അപ്പോഴില്ല. “ആത്മജ്ഞാനത്തെ മൂടിമറയ്ക്കുന്ന മോഹവിഭ്രാന്തികൾ ഏഴുതരമാണ്‌.. ബീജാവസ്ഥയിലിരിക്കുന്ന ജാഗ്രത്ത്, ജാഗ്രതാവസ്ഥ, മഹാ ജാഗ്രത്ത്, ജാഗ്രത്തായ സ്വപ്നാവസ്ഥ, സ്വപ്നാവസ്ഥ, സ്വപ്നാവസ്ഥയിലുള്ള ജാഗ്രത്ത്, സുഷുപ്തി എന്നിവയാണവ.”

നിർമ്മലബോധത്തിൽ മനസ്സും ജീവനും നാമമാത്രമായി ഉള്ളപ്പോൾ, അതിന്‌ ബീജാവസ്ഥയിലുള്ള ജാഗ്രത്ത് എന്നു പറയും. ‘ഞാൻ’.‘എന്റെ’ തുടങ്ങിയ ധാരണകൾ ഉണരുമ്പോൾ അത് ജാഗ്രതാവസ്ഥയായി. ഈ ധാരണകൾ പൂർവ്വജന്മങ്ങളിലെ ഓർമ്മകളുമായിച്ചേർന്ന് പ്രബലമാവുമ്പോൾ അത് മഹാജാഗ്രത്തായി. മനസ്സ് പൂർണ്ണമായും ഉണർന്നിരിക്കുമ്പോൾ സ്വന്തം ഭാവനകളാലും മോഹങ്ങളാലും നിറഞ്ഞു നില്‍ക്കുന്ന അവസ്ഥയാണ്‌ ജാഗ്രത്തായ സ്വപ്നാവസ്ഥ. ഉറക്കത്തിലെ അനുഭവങ്ങളെന്ന തെറ്റിദ്ധാരണ- ഇനിയും യാഥാർത്ഥ്യമാവാത്ത അനുഭവങ്ങൾ- സ്വപ്നാവസ്ഥയാണ്‌.. സ്വപ്നത്തിൽ പൂർവ്വാനുഭവങ്ങൾ അനുഭവവേദ്യമാകുന്ന അവസ്ഥയാണ്‌ സ്വപ്നാവസ്ഥയിലെ ജാഗത്ത്. ഇതെല്ലാം പരിപൂർണ്ണമായ മാന്ദ്യത്തിനു വഴിമാറിക്കൊടുക്കുമ്പോൾ സുഷുപ്തിയായി. ഈ ഏഴു പടികൾക്കുള്ളിൽ അനേകം ഉപവിഭാഗങ്ങളുണ്ട്.

