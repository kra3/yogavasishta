\section{ദിവസം 169}

\slokam{
അഹംകാരമതോ രാമ മാർജയാന്ത: പ്രയത്നത:\\
അഹം ന കിംചിദേവേതി ഭാവയിത്വാ സുഖീ ഭവ (4/31/7)\\
}

വസിഷ്ഠൻ തുടർന്നു: അങ്ങിനെ വിജ്ഞാനമില്ലാത്തതിന്റെ ദാരുണഫലങ്ങൾ എന്തെന്നു നാം കണ്ടു. അജയ്യരായിരുന്ന രാക്ഷസപ്രമുഖർ അവരിലെ അഹംഭാവത്തിന്റെ ഫലമായുണ്ടായ ഭയത്താൽ നിശ്ശേഷം പരാജിതരായി അധ:പ്പതിച്ചതും നാം കണ്ടുവല്ലോ. കൊടും വിപത്തായൊരു കിനാവള്ളിയാണീ ലൗകീകത. അതിന്റെ വിത്താണ്‌ അഹം ഭാവം. "അതുകൊണ്ട് രാമാ, അഹംഭാവത്തെ അതിനുള്ളിലെ എല്ലാ ശക്തിവിശേഷങ്ങളടക്കം വർജ്ജിക്കൂ. സ്വയം ’ഞാനൊന്നുമല്ല‘ എന്നുറച്ച് സന്തോഷവാനായിരിക്കൂ". ഏകവും അദ്വിതീയവുമായ അനന്താവബോധം ശുദ്ധമായ ആനന്ദസ്വരൂപമാണ്‌., എന്നാൽ അതിനെ അഹംഭാവം എന്ന ഗ്രഹണം ബാധിക്കുന്നു.

ദാമൻ, വ്യാളൻ, കടൻ എന്നിവർ ജനനമരണങ്ങൾക്കതീതരായിരുന്നു. എന്നാൽ അവരുടെ അഹംഭാവം കാരണം അവർ സംസാരചക്രത്തിൽ വീണുപോയി. ദേവന്മാർ പോലും ഭയപ്പെട്ടിരുന്ന അവരിപ്പോൾ മൽസ്യങ്ങളായി കാശ്മീരത്തിലെ തടാകത്തിൽ ജീവിക്കുന്നു.

രാമൻ ചോദിച്ചു: മഹാത്മാവേ, ദാമൻ, വ്യാളൻ, കടൻ എന്നീ മൂവർ ശംഭരന്റെ മായാ സൃഷ്ടികൾ മാത്രമായിരുന്നുവല്ലോ. അവർ എങ്ങിനെയാണ്‌ നമ്മെപ്പോലെ ഉണ്മയെ പ്രാപിച്ചത്?

വസിഷ്ഠൻ പറഞ്ഞു: രാമാ, ദാമൻ മുതലായ രാക്ഷസന്മാർ മായാസൃഷ്ടികളും അയാഥാർത്ഥ്യവുമാണെന്നതുപോലെ തന്നെ നാമും ദേവന്മാരുമെല്ലാം മിഥ്യ മാത്രമാണ്‌. 'ഞാൻ', 'നീ' തുടങ്ങിയ എല്ലാ ധാരണകളും അയാഥാർത്ഥ്യമാണ്‌. ഞാനും നീയും ഇപ്പോൾ ‘ഉള്ളതുപോലെ’ കാണപ്പെടുന്നുവെങ്കിലും സത്യം മാറുകില്ലല്ലോ. മരിച്ചവർ ഇപ്പോൾ നിന്റെമുന്നിൽ പ്രത്യക്ഷപ്പെട്ടുവെന്നിരിക്കട്ടെ. അവർ മരണപ്പെട്ടവരാണെന്ന സത്യത്തിനു മാറ്റമുണ്ടാകയില്ലല്ലോ. എങ്കിലും ഒന്നു പറയാം “ബ്രഹ്മം മാത്രമേ ഉണ്മയായുള്ളു (ബ്രഹ്മസത്യം ജഗത് മിഥ്യ)” എന്ന സത്യം അജ്ഞാനിക്ക് ഉപദേശിക്കരുത്. കാരണം ലോകത്തിന്റെ ഉണ്മയെപ്പറ്റി അവനിൽ രൂഢമൂലമായിരിക്കുന്ന തെറ്റിദ്ധാരണകൾ തീവ്രമായ സാധനയാലും ശാസ്ത്രഗ്രന്ഥങ്ങളിൽ നിന്നുമാർജ്ജിക്കുന്ന അറിവിന്റെ വെളിച്ചത്താലും മാത്രമേ മാറുകയുള്ളു.

‘ബ്രഹ്മം മാത്രമാണു സത്യം; ലോകം മിഥ്യയാണ്‌’ എന്നു പ്രസ്താവിക്കുന്നവനെ അജ്ഞാനികൾ പരിഹസിക്കും. എത്ര വിവരിച്ചു പറഞ്ഞാലും അവർക്കിതു മനസ്സിലാവുകയില്ല. ഒരു ശവശരീരത്തിനെ നടക്കാൻ പഠിപ്പിക്കുക അസാദ്ധ്യം. സത്യം വിവേകശാലിക്ക് അനുഭവവേദ്യമാകുന്ന ഒന്നാണ്‌.. രാമാ, നാമും ഇപ്പറഞ്ഞ രാക്ഷസന്മാരും ഒന്നും ഉണ്മയല്ല. സത്യമായുള്ളത് മാറ്റങ്ങൾക്കു വിധേയമല്ലാത്ത അനന്താവബോധം മാത്രം. അനന്താവബോധത്തിൽ ‘നീ’, ‘ഞാൻ’, ‘അവൻ’, ‘അസുരന്മാർ’ എന്നീ ധാരണകൾ ഉയർന്നുവന്ന് അവയിൽ യാഥാർത്ഥ്യബോധം അങ്കുരിക്കുന്നു. കാരണം ഈ ‘അറിവ്’ തെളിയുന്ന ബോധം സത്യമാണല്ലോ. ഈ ബോധം ഉണർന്നിരിക്കുമ്പോൾ മേൽപ്പറഞ്ഞ ധാരണകൾ ഉയരുന്നു. ആ ബോധമുറങ്ങുമ്പോൾ ഈ ധാരണകൾ ഇല്ലാതെയാവുന്നു. എങ്കിലും അനന്താവബോധത്തിൽ സുഷുപ്തിയും ജാഗ്രത്തുമൊന്നും ഇല്ലെന്നറിയുക. അതു നിർമ്മലമായ അവബോധം മാത്രമാണ്‌. ഈ സത്യമറിഞ്ഞ് ഭിന്നതയാലുണ്ടാവുന്ന ദു:ഖത്തിൽ നിന്നും ഭയത്തിൽ നിന്നും സ്വതന്ത്രനാവൂ. 
