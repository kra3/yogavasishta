\section{ദിവസം 233}

\slokam{
ഭവ്യോസി ചേത്തദേതസ്മാത്സര്‍വമാപ്നോഷി നിശ്ചയാത്\\
നോ ചേത്തദ്വഹ്വപി പ്രോക്തം ത്വയി ഭസ്മനി ഹൂയതേ  (5/26/12)\\
}

ബലി സ്വയം ഇങ്ങിനെ പറഞ്ഞു: ഭാഗ്യാതിരേകം കൊണ്ട് അച്ഛന്‍ പറഞ്ഞുതന്ന കാര്യങ്ങള്‍ എനിക്കോര്‍മ്മ വന്നു. ഇപ്പോള്‍ സുഖാനുഭാവങ്ങള്‍ക്കായുള്ള ആസക്തി എന്നില്‍ ഇല്ലാതായിരിക്കുന്നു. എനിക്ക് അമൃതസമാനമായ ആ  പ്രശാന്ത സ്ഥിതിയെ പ്രാപിക്കണം. ആവര്‍ത്തിച്ചുള്ള ധനസമ്പാദനത്തിലും ആഗ്രഹപൂര്‍ത്തീകരണത്തിലും ലൈംഗീകസുഖത്തിലും  എനിക്ക് താത്പ്പര്യമറ്റിരിക്കുന്നു. ഈ പ്രശാന്താവസ്ഥ തികച്ചും ആഹ്ലാദദായകം തന്നെ. ആന്തരീകമായ പ്രശാന്തിയില്‍ എല്ലാ സുഖദു:ഖങ്ങളും മൂല്യമറ്റവയായിത്തീരുന്നു. ജീവിതം എന്നത് അനുഭവങ്ങളുടെ തുടര്‍ച്ചയായ ആവര്‍ത്തനങ്ങളാണ്. യാതൊരു നവീനാനുഭവങ്ങളും വാസ്തവത്തില്‍ നമുക്കുണ്ടാവുന്നില്ല.

ഞാന്‍ എല്ലാമുപേക്ഷിച്ച് മനസാ സുഖാസക്തികളില്‍ നിന്നും പിന്മാറി ആത്മസ്വരൂപത്തില്‍ അഭിരമിക്കാന്‍ തീരുമാനിക്കുന്നു. ഈ വിശ്വം എന്നത് മനസ്സിന്റെ സൃഷ്ടിയാണല്ലോ. അതുപേക്ഷിച്ചതുകൊണ്ട് നമുക്ക് നഷ്ടപ്പെടാന്‍ എന്തുണ്ട്? മാനസാന്തരത്തെപ്പറ്റിയുള്ള ഈ ചര്‍ച്ചയും എനിക്ക് മതിയായി. കാരണം ഒരു രോഗത്തിന് അവശ്യം വേണ്ടത് ഉടനെയുള്ള ചികിത്സയാണ്. അവസാനിക്കാതെ നീളുന്ന ചര്‍ച്ചകളല്ല. ‘ഞാന്‍ ആരാണ്?’; ‘ഇതെല്ലാം എന്താണ്?’ ഈ ചോദ്യങ്ങള്‍ ശുക്രാചാര്യനോടു തന്നെ ചോദിക്കാം. 

വസിഷ്ഠന്‍ തുടര്‍ന്നു: ഇങ്ങിനെ തീരുമാനിച്ചുറച്ച് ബലി ശുക്രാചാര്യരെ ധ്യാനിച്ചു. അനന്താവബോധത്തില്‍ സ്വയം അഭിരമിച്ചിരുന്നതുകൊണ്ട് സര്‍വ്വവ്യാപിയായ ശുക്രാചാര്യര്‍ തന്റെ ശിഷ്യന് തന്റെ സാന്നിദ്ധ്യം ആവശ്യമുണ്ടെന്നറിഞ്ഞു. ക്ഷണത്തില്‍ അദ്ദേഹം ബലിയുടെ സവിധം എത്തിച്ചേര്‍ന്നു. ഗുരുസമക്ഷം ബാലിയില്‍ ഒരു ദിവ്യപ്രഭ കളിയാടി. ബലി അതീവഭക്തിയോടെ ഗുരുവിനെ ഉചിതമായി സ്വീകരിച്ച് പാദപൂജ ചെയ്തു നമസ്കരിച്ചു.

ബലി ശുക്രനോടു ചോദിച്ചു: ഭഗവന്‍, അങ്ങയുടെ ദിവ്യപ്രഭ തന്നെയാണ് അവിടുത്തെ മുന്നില്‍ ഈ പ്രശ്നം അവതരിപ്പിക്കാന്‍ എന്നെ പ്രേരിപ്പിക്കുന്നത്. എനിക്ക് സുഖങ്ങളില്‍ ആസക്തി ഒന്നുമില്ല. പക്ഷേ, എനിക്ക് സത്യമറിയാന്‍ ആഗ്രഹമുണ്ട്. ഞാന്‍ ആരാണ്? അങ്ങ് ആരാണ്? ഈ ലോകം എന്താണ്? ഇതെല്ലാം എനിക്ക് പറഞ്ഞു തന്നാലും. 

ശുകമുനി പറഞ്ഞു: അല്ലയോ ബലീ ഞാന്‍ മറ്റൊരു ലോകത്തേയ്ക്ക് പോകുന്ന വഴിയാണ്. അതുകൊണ്ട് ചുരുങ്ങിയ വാക്കുകള്‍കൊണ്ട് ജ്ഞാനത്തിന്റെ സാരാംശം ഞാന്‍ പറയാം. 'ബോധം മാത്രമേ നിലനില്‍ക്കുന്നതായുളളു. ഇക്കാണുന്നതും കാണാത്തതുമെല്ലാം ബോധം മാത്രം. എല്ലാടവും നിറഞ്ഞു നില്‍ക്കുന്നതും ബോധം. നീയും ഞാനും ഈ ലോകവുമെല്ലാം ബോധമല്ലാതെ മറ്റൊന്നുമല്ല.'   

“നിനക്ക് വിനയവും ആത്മാര്‍ത്ഥതയും ഉള്ളപക്ഷം ഞാനീ പറഞ്ഞതില്‍ നിന്നുതന്നെ നിനക്കെല്ലാം നേടാം. അതില്ലെങ്കില്‍ എത്ര വിവരിച്ചു പറഞ്ഞാലും അഗ്നികുണ്ഡത്തിലവശേഷിച്ച ചാരക്കൂമ്പാരത്തില്‍ അര്‍ഘ്യം അര്‍പ്പിക്കുന്നതുപോലെ വ്യര്‍ത്ഥമാവുമത്.” ആളുന്ന അഗ്നിയില്‍ അര്‍പ്പിച്ചാല്‍ മാത്രമേ അര്‍ഘ്യം സ്വീകരിക്കപ്പെടുകയുള്ളു എന്ന് നിനക്കറിയാം.

ബോധത്തില്‍ ഉണ്ടാവുന്ന വിഷയത്വം, ധാരണാകല്‍പ്പനകള്‍ എന്നിവയാണ് ബന്ധനങ്ങള്‍.  അവയുടെ നിരാസമാണു മുക്തി. അവബോധത്തില്‍ നിന്നും ധാരണകള്‍ (വസ്തുബോധം) നീക്കിയാല്‍ എല്ലാറ്റിന്റെയും സത്യസ്ഥിതിയായി. എല്ലാ തത്വചിന്തകളുടെയും അടിസ്ത്ഥാനമിതാണ്. ഈ ദര്‍ശനത്തില്‍ സ്വയം സ്ഥിരപ്രതിഷ്ഠനായാല്‍ നിനക്ക് അനന്താവബോധത്തിലെത്താം. ഇനിയെനിക്ക് ദേവന്മാര്‍ക്കായുള്ള കുറച്ചു ജോലികള്‍ ചെയ്യാനുണ്ട്. ഞാനിപ്പോള്‍ പോകുന്നു. ദേഹമെടുത്താല്‍പ്പിന്നെ അതിനനുയോജ്യമായ ഉചിതകര്‍മ്മങ്ങളെ ഉപേക്ഷിക്കുന്നത് ശരിയല്ലല്ലോ.
