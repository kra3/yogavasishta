\newpage
\section{ദിവസം 090}

\slokam{
ചിദ്ഘനേനൈകതാമേത്യ യദാ തിഷ്ഠതി നിശ്ചല:\\
ശാമ്യന്വയവഹരന്വാപി തദാ സംശാന്ത ഉച്യതെ (3/66/12)\\
}

വസിഷ്ഠന്‍ തുടര്‍ന്നു: രാമാ ആ 'ഒന്ന്' ഒരിക്കലും പലതായിട്ടില്ല. ഒരു തിരിയില്‍നിന്നും മറ്റുതിരികളിലേയ്ക്ക്‌ ദീപം കൊളുത്തുമ്പോള്‍ അതെല്ലാം ഒരേ ദീപനാളം തന്നെ. ഒരേ ബ്രഹ്മം പലതായി കാണപ്പെടുന്നു. ഈ നാനാത്വത്തിന്റെ അയാഥാര്‍ഥ്യത്തെപ്പറ്റി ബോധ്യമാവുമ്പോള്‍ അവന്‌ ദു:ഖനിവൃത്തിയുണ്ടാവുന്നു. ജീവനെന്നത്‌ പരിമിതപ്പെട്ട ബോധം മാത്രം. പരിമിതികള്‍ ഇല്ലാതാവുമ്പോള്‍ ശാന്തി അനുഭവിക്കുമാറാകുന്നു. പാദരക്ഷ ധരിച്ചവന്‌ ഭൂമി മുഴുവന്‍ തോലുപൊതിഞ്ഞപോലെയാണല്ലോ.

എന്താണു ലോകം? അതൊരു പ്രകടനം മാത്രം. വാഴത്തട എന്നുപറയുന്നത്‌ ഇലകള്‍ അല്ലാതെ മറ്റൊന്നുമല്ലല്ലോ. മദ്യപാനം ഒരുവനെ ശൂന്യാകാശത്ത്‌ മായക്കാഴ്ചകള്‍ കാണുമാറാക്കുന്നതുപോലെ മനസ്സിന്‌ ഒന്നിനെ പലതായി കാണാന്‍ കഴിയുന്നു. കുത്തനെ നില്‍ക്കുന്ന ഒരു സ്തംഭം ചലിക്കുന്നതായി മദ്യപനു തോന്നുന്നതു പോലെ അജ്ഞാനിക്കു ലോകത്തില്‍ ചലനങ്ങള്‍ ദൃശ്യമാകുന്നു. മനസ്സ്‌ ദ്വന്ദഭാവം കൈക്കൊള്ളൂമ്പോള്‍ ദ്വന്ദതയും അതിന്റെ പ്രതിരൂപമായ ഏകതയും ഉണ്ട്‌. . മനസ്സില്‍നിന്നും ഈ ധാരണ മാറുമ്പോള്‍ ദ്വന്ദതയോ ഏകതയോ ശേഷിക്കുന്നില്ല. "അനന്താവബോധത്തിന്റെ ഏകാത്മകതയില്‍ ദൃഢീകരിച്ചുകഴിഞ്ഞാല്‍ ഒരുവന്‍ നിശ്ശബ്ദനായി, നിഷ്ക്രിയനായിരുന്നാലും കര്‍മ്മങ്ങളില്‍ സജീവമായി മുഴുകിയാലും അയാള്‍ സ്വയം പ്രശാന്തനാണ്‌." ഇപ്രകാരം പരമപദത്തെ പ്രാപിച്ചവന്‍ അനാത്മാവസ്ഥയിലാണെന്നു പറയപ്പെടുന്നു. അത്‌ നിശ്ശൂന്യതയെക്കുറിച്ച്‌ അറിയുന്ന ഒരു തലമാണ്‌..

മനസ്സിലെ പ്രക്ഷോഭങ്ങളാണ്‌ ബോധത്തെ അറിവിനു വിധേയമായ വസ്തുവാണെന്ന തോന്നലുണ്ടാക്കുന്നത്‌.. പിന്നീട്‌, മനസ്സില്‍ 'ഞാന്‍ ജനിച്ചു' മുതലായ തെറ്റിദ്ധാരണകള്‍ ഉദിക്കുന്നു. ആ അറിവും മനസ്സുതന്നെയാണ്‌.. അതുകൊണ്ട്‌ അതിനെ 'അവിദ്യ' അല്ലെങ്കില്‍ ഭ്രമം എന്നു വിളിക്കുന്നു. ഒരുവന്‌ സംസാരമെന്ന ഈ രോഗത്തില്‍നിന്നു മുക്തിയേകാന്‍ ആത്മജ്ഞാനമല്ലാതെ മറ്റൊരു മാര്‍ഗ്ഗമില്ല. കയറില്‍ പാമ്പിനെ കണ്ടതിനു പ്രതിവിധി ശരിയായ അറിവുണ്ടാവുകമാത്രമാണ്‌.. അത്തരം അറിവുറയ്ക്കുമ്പോള്‍ മനസ്സില്‍ ഇന്ദ്രിയസുഖങ്ങളോടുള്ള ആസക്തി ഇല്ലാതാവുന്നു. അതാണല്ലോ അജ്ഞതയെ പെരുപ്പിക്കുന്നത്‌.. അതുകൊണ്ട്‌ ആസക്തികളെ, ആര്‍ത്തികളെ പ്രീണിപ്പിക്കാതിരിക്കുക. എന്താണതിനു ബുദ്ധിമുട്ട്‌?

മനസ്സ്‌ വിഷയ (പദാര്‍ത്ഥ) ധാരണയിലിരിക്കുമ്പോള്‍ പ്രക്ഷുബ്ധമാണ്‌.. എന്നാല്‍ വിഷയങ്ങളോ ആശയങ്ങളോ മഥിക്കാത്ത മനസ്സില്‍ ചിന്തകളില്ല. ചലനമില്ല. ലോകമെന്ന ദൃശ്യവിക്ഷേപവും അവിടെയില്ല. ചിന്തകളുടെ സഞ്ചാരമാണ്‌ ജീവന്‍.. കാരണവും കര്‍മ്മവും. അതാണ്‌ ലോകദൃശ്യത്തിന്റെ വിത്ത്‌.. പിന്നീടുള്ളത്‌ ശരീരസൃഷ്ടിയാണ്‌.

