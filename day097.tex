 
\section{ദിവസം 097}

\slokam{
വിദിതപരമകാരണാദ്ധ്യ ജാതാ\\
സ്വയമനുചേതന സംവിദം വിചാര്യ\\
സ്വമനനകലനാനുസാര ഏക\\
സ്ത്വിഹ ഹി ഗുരു: പരമോ ന രാഘവാന്യ:   (3/74/28)\\
}

വസിഷ്ഠന്‍ തുടര്‍ന്നു: ഹിമാലയത്തില്‍ സൂചിക മറ്റൊരു പര്‍വ്വതശിഖരം പോലെ നില്‍ക്കുന്നത്‌ വായു കണ്ടു. അവള്‍ ആഹാരമൊന്നും കഴിക്കാതിരുന്നതുകൊണ്ട്‌ ദേഹം ഏതാണ്ടു മുഴുവനായിത്തന്നെ ഉണങ്ങി വരണ്ടിരുന്നു. വായു അവളുടെ വായില്‍ക്കയറിയപ്പോഴൊക്കെ അവളതിനെ പുറത്തേയ്ക്കു കളഞ്ഞു. പ്രാണവായുവിനെ ശിരസ്സിലേയ്ക്കു വലിച്ച്‌ ഉത്തമയായ ഒരു യോഗിനിയായി അവള്‍ നിലകൊണ്ടു. അവളെക്കണ്ട്‌ വായു അത്ഭുതസ്തബ്ധനായിപ്പോയി. അവളോടൊന്നു സംസാരിക്കാന്‍ പോലും വായുവിനായില്ല. തീവ്രതപസ്സില്‍ നിന്നവളെ ഇളക്കുക അസാദ്ധ്യമെന്നു കണ്ട്‌ വായു സ്വര്‍ഗ്ഗത്തിലേയ്ക്ക്‌ തിരിച്ചു ചെന്ന് ഇന്ദ്രനെ വിവരങ്ങള്‍ അറിയിച്ചു. "ജംബുദ്വീപ ഭൂഖണ്ഡത്തില്‍ സൂചിക അനിതരസാധാരണമായ ഒരു തപസ്സിലാണ്‌. . അവള്‍ വായിലേയ്ക്ക്‌ കാറ്റിനെപ്പോലും കടത്തിവിടുന്നില്ല. വിശപ്പിനെ അതിജീവിക്കാന്‍ അവള്‍ തന്റെ വയറിനെ കട്ടിലോഹമാക്കി മാറ്റി. ദയവായി അങ്ങെഴുന്നേറ്റ്‌ പോയി ബ്രഹ്മാവിനെ സമീപിച്ച്‌ അവള്‍ക്കുവേണ്ട വരം കൊടുത്ത്‌ ഈ കൊടുംതപം അവസാനിപ്പിക്കണം. അല്ലെങ്കില്‍ ആ താപത്തില്‍ നാമെല്ലാം എരിഞ്ഞടങ്ങും." 

ഇന്ദ്രന്റെ അഭ്യര്‍ത്ഥനമാനിച്ച്‌ ബ്രഹ്മാവ്‌ സൂചിക തപസ്സനുഷ്ഠിച്ചിരുന്നയിടത്തു ചെന്നു. അപ്പോഴേയ്ക്ക്‌ സൂചികയാകട്ടെ പരിപൂര്‍ണ്ണ നിര്‍മ്മലയായിക്കഴിഞ്ഞിരുന്നു. സ്വന്തം നിഴലും തപശ്ചര്യകളുടെ അഗ്നിയും അവള്‍ക്ക്‌ സാക്ഷികളായിരുന്നു. സാമീപ്യംകൊണ്ട്‌ അവള്‍ക്കു ചുറ്റുമുള്ള പൊടിപടലങ്ങളും കാറ്റും പരമമോക്ഷപദം പൂകി. "അപ്പോളവളുടെ മേധാശക്തിയുടെ പ്രഭാവത്തില്‍ എല്ലാത്തിന്റേയും അഹൈതുകഹേതുവിനെപ്പറ്റി അവള്‍ക്ക്‌ നേരറിവുണ്ടായി. തീര്‍ച്ചയായും ബോധതലത്തില്‍ അവനവന്റെ ചിന്താഗമനങ്ങളെ ആത്മാന്വേഷണത്തിലൂടെ അറിയുന്ന പ്രക്രിയ മഹാഗുരു തന്നെ. രാമ, മറ്റാരുമല്ല അതുനേടിത്തരുന്നത്‌". .

ബ്രഹ്മാവ്‌ അവളോടു പറഞ്ഞു: നിനക്കെന്തു വരം വേണം, ചോദിച്ചാലും. (അവള്‍ക്ക്‌ ഇന്ദ്രിയങ്ങള്‍ ഇല്ലായിരുന്നുവെങ്കിലും ഇതവള്‍ സ്വയം അനുഭവിച്ചറിയുകയാണുണ്ടായത്‌).. ഇതിനു മറുപടിപറയാന്‍ അവള്‍ കുറച്ചാലോചിച്ചു: ഞാന്‍ പരം പൊരുളിനെ സാക്ഷാത്കരിച്ചിരിക്കുന്നു. എന്നില്‍ യാതൊരു സംശയങ്ങളും ആവശ്യങ്ങളും ഇല്ല. വരം കിട്ടിയിട്ട്‌ ഞാന്‍ എന്തു ചെയ്യാനാണ്‌? ഞാന്‍ അജ്ഞാനം നിറഞ്ഞ ഒരു പെണ്‍കുട്ടിയായിരുന്നപ്പോള്‍ ആശകള്‍ എന്ന പിശാചെന്നെ വേട്ടയാടിക്കൊണ്ടിരുന്നു. ഇപ്പോള്‍ ആത്മജ്ഞാനത്തിലൂടെ ആശാപിശാചിനെ ഞാന്‍ അടക്കം ചെയ്തു കഴിഞ്ഞു. 

ബ്രഹ്മാവ്‌ പറഞ്ഞു: അല്ലയോ താപസശ്രേഷ്ഠേ, ശാശ്വതമായ ലോകനിയമം മാറ്റാന്‍ കഴിയില്ല. അതിനാല്‍ നീ പഴയ രൂപം ധരിക്കേണ്ടതുണ്ട്‌. . ഏറെക്കാലം സന്തോഷത്തോടെ ജീവിച്ച്‌ മുക്തിപദം പ്രാപിച്ചാലും. നീ പ്രബുദ്ധമായ ജീവിതമാണിനി നയിക്കുക. ദുഷ്ടരേയും പാപികളേയും മാത്രം ബാധിച്ച്‌, വളരെക്കുറച്ചുമാത്രം ദുരിതം വിതച്ച്‌ - അതും എന്റെ വിശപ്പടാക്കാന്‍ മാത്രം - നിനക്കു കഴിയാം. സൂചിക ബ്രഹ്മാവിന്റെ അനുഗ്രഹമായി ഇതു സ്വീകരിച്ചു. ഉടനേ തന്നെ സൂചികയുടെ ദേഹം വളര്‍ന്നുപൊന്തി പര്‍വ്വതാകാരമായി. 

