\section{ദിവസം 251}

\slokam{
ഭാവാനയമയം ചാഹംത്വം ശബ്ദൈരേവമാദിഭി:\\
സ്വയമേവാത്മനാത്മാനം ലീലാര്‍ത്ഥം സ്തൌഷി വക്ഷി ച  (5/36/56)\\
}

പ്രഹ്ലാദന്‍ ധ്യാനം തുടര്‍ന്നു: ക്രോധവും ലോഭവും, പൊങ്ങച്ചവും അക്രമവാസനയും പോലുള്ള അധമഗുണങ്ങള്‍ മഹാത്മാക്കള്‍ക്ക് പറഞ്ഞിട്ടുള്ളതല്ല. അവയെ ഉപെക്ഷിക്കൂ. പണ്ടത്തെ ദു:ഖാനുഭവങ്ങള്‍ വീണ്ടും വീണ്ടുമോര്‍ത്ത് ‘ഞാന്‍ ആരാണ്?’; ‘ഇതെല്ലാം എങ്ങിനെ ഉണ്ടായി?’ എന്ന് പ്രസന്നഭാവത്തില്‍ മനനം ചെയ്ത് അവകളില്‍ നിന്നും നിവൃത്തി നേടുക. കഴിഞ്ഞതെല്ലാം കഴിഞ്ഞു. നിന്നെ തപിപ്പിച്ചിരുന്ന ദു:ഖാശങ്കകളെല്ലാം ശമിച്ചിരിക്കുന്നു.

ഇന്ന് നീ ദേഹമെന്ന നഗരത്തിന്റെ സര്‍വ്വാധിപത്യം നേടിയിരിക്കുന്നു. ആകാശത്തെ ആര്‍ക്കും കൈക്കുടന്നയിലൊതുക്കാന്‍ ആവാത്തതുപോലെ ദു:ഖങ്ങള്‍ക്ക് നിന്നില്‍ കൈവെക്കാനാവില്ല. ഇപ്പോള്‍ നീ നിന്നിലെ  ഇന്ദ്രിയങ്ങളുടെയും മനസ്സിന്റെയും നാഥനായി ആഹ്ലാദിക്കുന്നു. ഭഗവാനേ  അങ്ങ് സുഷുപ്തിയിലെന്നവണ്ണമിരിക്കുന്നു. എന്നാല്‍ അനുഭവങ്ങളെ ജാഗ്രതയോടെ അറിയാന്‍ അങ്ങ് സ്വന്തം ചൈതന്യവിശേഷം മൂലം ഉണരുകയും ചെയ്യുന്നതായി തോന്നുന്നു.   

വാസ്തവത്തില്‍ ഈ ചൈതന്യമാണ് അനുഭവങ്ങളുമായി സംവദിക്കുന്നതെങ്കിലും നീ സ്വയം ആ അനുഭവങ്ങളെ സ്വാംശീകരിക്കുന്നു. പ്രാണായാമത്തിലൂടെ ശിരോമകുടത്തിലെ ബ്രഹ്മപദമെത്തിയവര്‍ അവരുടെ ഭൂതഭാവികാലങ്ങളിലെ ഓരോ നിമിഷവും ബ്രഹ്മാവിന്റെ സവിധത്തിലാണ് കഴിയുന്നത്. ആത്മാവേ, ശരീരമെന്നറിയപ്പെടുന്ന പൂവിലെ സുഗന്ധമാണ് നീ. ശരീരമാകുന്ന ചന്ദ്രബിംബത്തിലെ അമൃതാണ് നീ. ദേഹമാകുന്ന ഔഷധച്ചെടിയുടെ മൂല്യവസ്തു നീയാണ്. ശരീരമെന്ന മഞ്ഞുകട്ടയുടെ തണുപ്പ് നീയാണ്.       

പാലില്‍ വെണ്ണയെന്നതുപോലെ ദേഹത്തില്‍ സൌഹൃദവും ആസക്തിയും സഹജമായുണ്ട്. വിറകില്‍ തീയെന്നപോലെ നീ ദേഹത്തില്‍ കുടികൊള്ളുന്നു. പ്രോജ്വലങ്ങളായ എല്ലാ വസ്തുക്കളിലെയും തിളക്കം നീയാണ്. 

വിഷയവസ്തുക്കളെക്കുറിച്ചുള്ള അറിവ് നല്‍കുന്ന ഉള്‍വെളിച്ചം നീയാണ്. മനോമത്തേഭത്തിന്റെ ശക്തി നീയാണ്. ആത്മജ്ഞാനത്തിന്റെ അഗ്നിയും പ്രകാശവും നീയല്ലേ? എല്ലാ വാക്കുകളും നിന്നില്‍ അവസാനിക്കുന്നു. എന്നിട്ടത് മറ്റെവിടെയോ വീണ്ടും പ്രത്യക്ഷമാവുന്നു.    വൈവിദ്ധ്യമാര്‍ന്ന ആഭരണങ്ങള്‍ സ്വര്‍ണ്ണത്തില്‍ പണിഞ്ഞുണ്ടാക്കിയതുപോലെ, എണ്ണമില്ലാത്ത സൃഷ്ടികള്‍ ഉണ്ടായത്, ഉണ്ടാക്കിയത്, നിന്നില്‍ നിന്നാണ്. അവകളിലെ വ്യത്യാസങ്ങള്‍ നാമരൂപങ്ങളില്‍ മാത്രമാണ്.

“ഇത്ഞാന്‍, ഇത് നീ, എന്നിങ്ങനെയുള്ള പ്രയോഗങ്ങള്‍ നിന്റെ പ്രഭാവത്തെ പ്രകീര്‍ത്തിക്കാനും ആദരിക്കാനും മാത്രമായുള്ളതത്രേ.” വലിയൊരു കാട്ടുതീ ക്ഷണനേരത്തെയ്ക്ക് പലപല രൂപങ്ങളും കൈക്കൊള്ളുമെങ്കിലും അതോരൊറ്റ അഗ്നിനാളമാണ്. അതുപോലെ നിന്റെ അദ്വിതീയസത്വം പലതായി വിശ്വത്തില്‍ കാണപ്പെടുന്നു എന്നുമാത്രം. ലോകങ്ങള്‍ മുഴുവന്‍ കോര്‍ത്തിണക്കുന്ന നാരാണു നീ. നീയാണ് സത്യത്തിന്റെ അടിത്തറ. അതിന്മേലാണല്ലോ ലോകമിരിക്കുന്നത്. ലോകങ്ങളെല്ലാം നിന്നില്‍ സാദ്ധ്യതകളായി കുടികൊള്ളുന്നു. പാചകക്കാരന്‍ വൈവിദ്ധ്യമാര്‍ന്ന രുചിഭേദങ്ങളോടെ ഭക്ഷ്യവസ്തുക്കളുണ്ടാക്കുംപോലെ ആ സാദ്ധ്യതകളുടെ ഒഴിയാക്കലവറയില്‍ നിന്നും നീ ലോകങ്ങളെ സൃഷ്ടിച്ചു പ്രകടമാക്കുന്നു. എന്നാലീ  ലോകങ്ങളുടെ നിലനില്‍പ്പ്‌ നീയുള്ളപ്പോള്‍ മാത്രമേയുള്ളു. കാരണം അവകളിലെ ഉണ്മ നീയാണല്ലോ. ഈ ദേഹമൊരു പാഴ്ത്തടിപോലെ ജീവനറ്റുവീഴാനുള്ളതാണ്. വെളിച്ചമടുത്തെത്തുമ്പോള്‍  ഇരുട്ടെന്നപോലെ സുഖദു:ഖങ്ങള്‍ നിന്റെ സവിധത്തില്‍ ഇല്ലാതാവുന്നു. എന്നാല്‍ ആഹ്ലാദം മുതലായ അനുഭവങ്ങള്‍ സാദ്ധ്യമാവുന്നത് നിന്നില്‍ നിന്നുമാര്‍ജ്ജിക്കുന്ന ഉണര്‍വ്വിന്റെ വെളിച്ചത്തില്‍ മാത്രമാണ്. 