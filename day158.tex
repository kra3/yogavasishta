\section{ദിവസം 158}

\slokam{
നകാരണേ കരണാദിപരേ വാസ്തവാദികാരണേ
വിചാരണീയ: സാരോ ഹി കിമസാരവിചാരണൈ: (4/18/23)
}

വസിഷ്ഠൻ തുടർന്നു: രാമാ, ഈ സൃഷ്ടിയിൽ കാണുന്ന വൈജാത്യവും നാനാത്വവും വെറുമൊരു കാഴ്ച്ച മാത്രമാണ്‌.. പരിണാമങ്ങൾക്കെല്ലാം ഒരേയൊരു അനന്താവബോധം മാത്രമേ അതിന്റെ കാരണവും ലക്ഷ്യവുമായി ഉള്ളൂ. അനന്താവബോധത്തിൽ പ്രകടമാവുന്ന ധാരണകൾക്കനുസരിച്ച്, പരിണാമസമയത്ത് ആ ബോധത്തിൽ നാനാത്വം ഉള്ളതായി തോന്നുന്നു. ഈ ധാരണകളിൽ ചിലവ കൂട്ടുചേർന്ന് അസംഖ്യം വൈവിദ്ധ്യമായ കാഴ്ച്ചകളെ പ്രകടമാക്കുന്നു. മറ്റു ചിലവ കൂട്ടുചേരാതെ നിലകൊള്ളുന്നു. വാസ്തവത്തിൽ ഓരോ അണുക്കളിലും ഇത്തരം ധാരണകൾ ഉണ്ട്. ഈ അണുക്കളാകട്ടെ ഒന്നിൽനിന്നു മറ്റൊന്ന് സ്വതന്ത്രവുമാണ്‌..

എല്ലാറ്റിന്റേയും ആകെത്തുകയാണ്‌ ബ്രഹ്മം. ഒരു വ്യക്തി അവന്റെ മനസ്സിനുള്ളിൽ രൂഢമൂലമായിരിക്കുന്ന വസ്തുക്കൾ മാത്രമേ കാണുന്നുള്ളു. മനസ്സിലെ ആശയങ്ങള്‍ക്ക് ഫലപ്രാപ്തിയില്ലാതാകുമ്പോൾ മനസ്സിൽ മാറ്റങ്ങളുണ്ടാവുന്നു. ഈ മാനസീക വ്യതിയാനങ്ങൾക്കനുസൃതമായി  ജീവന്‍ തുടർച്ചയായി ജന്മങ്ങളെടുക്കുന്നു. ഈ മാനസീക ബന്ധമാണ്‌ ജനന മരണങ്ങളും ശരീരങ്ങളും യാഥാർത്ഥ്യമാണെന്ന മിഥ്യാധാരണകൾക്കടിസ്ഥാനം. ഈ തെറ്റിദ്ധാരണകൾ നീങ്ങിയാൽപ്പിന്നെ ശരീരമെടുത്ത് ജീവന്‌ അനുഭവം നേടേണ്ടതില്ല. മറവിയാണ്‌ അസത്തിനെ സത്തായി ധരിക്കാനിടയാക്കുന്ന ചിന്താക്കുഴപ്പങ്ങൾക്കു കാരണം. പ്രാണശക്തിയെ സംശുദ്ധമാക്കുന്നതിലൂടെയും അതിനുമപ്പുറമുള്ള അറിവിനെ സ്വായത്തമാക്കുന്നതിലൂടെയും ഒരുവൻ താൻ അറിയേണ്ടതായ എല്ലാ അറിവുകളും ആർജ്ജിക്കുന്നു. മനോവ്യാപാരങ്ങളെപ്പറ്റിയും പുനർജന്മങ്ങളുടെ ഗതിവിഗതികളെപ്പറ്റിയും ഇങ്ങിനെ അറിവുറയ്ക്കുന്നു.

എല്ലാ ജീവികളുടെ ആത്മാക്കളും മൂന്നവസ്ഥകൾ കടന്നുപോവുന്നു. ജാഗ്രത്ത്, സ്വപ്നം, സുഷുപ്തി. ഇവയ്ക്ക് ശരീരവുമായി ബന്ധമൊന്നുമില്ല. ജീവജാലങ്ങൾ ഒരേയൊരാത്മാവിൽ നിലകൊള്ളുന്നു എന്ന തെറ്റായ അനുമാനങ്ങൾക്കനുസൃതമായാണിത് പറയുന്നത്. ഇതു പരമസത്യമല്ല. ജ്ഞാനി സുഷുപ്തി അവസ്ഥയ്ക്കുമപ്പുറമുള്ള ശുദ്ധബോധത്തിൽ എത്തുന്നു. അതായത് എല്ലാറ്റിന്റേയും ഉറവയായ അനന്താവബോധത്തിൽ വിരാജിക്കുന്നു. അജ്ഞാനിയോ, ജീവചക്രത്തിൽ ഉഴന്നുകൊണ്ടേയിരിക്കുന്നു.

ബോധം അനന്തമായതുകൊണ്ട് ഒരു ജീവചക്രത്തിൽ നിന്നും മറ്റൊന്നിലേയ്ക്ക് ഒരുവൻ നയിക്കപ്പെടുന്നു. അതിന്‌ ലോകമെന്ന അതിർത്തിയില്ല. ഇഹലോകത്തിനപ്പുറത്തേയ്ക്കും ഈ യാത്ര തുടരുന്നു. ഇങ്ങിനെയുള്ള സൃഷ്ടികൾക്ക് അവസാനമില്ല. വാഴത്തണ്ടിൽനിന്നും മറ്റൊരു വാഴക്കന്നു കിളിർക്കുമ്പോലെ ഒന്നിനുള്ളിൽ ചിലപ്പോൾ മറ്റൊന്നുണ്ടാവുന്നു. എന്നാൽ പരബ്രഹ്മത്തെ മറ്റൊന്നുമായി താരതമ്യപ്പെടുത്തുന്നത് ജ്ഞാനികൾക്കു ചേർന്നതല്ല.

എല്ലാ പ്രപഞ്ചപദാർത്ഥങ്ങൾക്കും ‘അകാരണ കാരണമായ’ ആ ഒന്നിനെപ്പറ്റി ഒരുവൻ അന്വേഷിക്കണം. അത് എല്ലാ കാരണാന്വേഷണങ്ങൾക്കും അതീതമാണെങ്കിലും അതുമാത്രമേ ധ്യാനയോഗ്യമായുള്ളു. അതുമാത്രമാണ്‌ ആവശ്യമുള്ളതായ ഒരേ ഒരു വസ്തു. അനാവശ്യമായ മറ്റ് വസ്തുക്കളെപ്പറ്റി അന്വേഷിച്ച് വിലപ്പെട്ട സമയം നാം വ്യർത്ഥമാക്കുന്നതെന്തിന്‌? 
