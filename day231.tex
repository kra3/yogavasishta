\section{ദിവസം 231}

\slokam{
അവശ്യം ഭാവിതവ്യാഖ്യാ സ്വേഹയാ നിയതിശ്ച യാ  \\
ഉച്യതേ ദൈവശബ്ദേന സാ നരൈരേവ നേതരൈ: (5/24/27)\\
}

വിരോചനന്‍ തുടര്‍ന്നു: ഉചിതമായ പ്രയത്നം ഒന്നുകൊണ്ടു മാത്രമേ നാം നിര്‍മമതയില്‍ എത്തുകയുള്ളൂ. ആളുകള്‍ ദൈവകൃപ, വിധി എന്നിവയെപ്പറ്റി സംസാരിക്കുന്നു. എന്നാല്‍ നാം ഈ ലോകത്ത് ശരീരത്തെ മാത്രമാണ് കാണുന്നത്. ഒരു ദൈവത്തെയല്ല. “ആളുകള്‍ ദൈവത്തെപ്പറ്റി പറയുമ്പോള്‍ അവരുദ്ദേശിക്കുന്നത് അനിവാര്യമായ കാര്യങ്ങളെയാണ്. അവരുടെ പിടിയിലൊതുങ്ങാത്ത, നിയന്ത്രണത്തില്‍ നില്‍ക്കാത്ത കാര്യങ്ങള്‍ , പ്രകൃതി നിയമമനുസരിച്ചുള്ള സഹജവും സ്വാഭാവികവുമായ സംഭവപരമ്പരകള്‍ , എന്നിവയെപ്പറ്റിയാണവര്‍ വിവക്ഷിക്കുന്നത്.”  എന്നാല്‍ സമ്പൂര്‍ണ്ണമായ സമതാഭാവം, സുഖദു:ഖദ്വന്ദങ്ങളുടെ അന്ത്യം എന്നിവയ്ക്കും ദൈവകൃപ എന്നു തന്നെയാണ് പറയുന്നത്. 

ദൈവകാരുണ്യം, പ്രകൃതിനിയമം, ശരിയായ സ്വപ്രയത്നം എന്നെല്ലാം പറയുന്നവ ഒരേ സത്യത്തെക്കുറിക്കുന്നു. തെറ്റായ കാഴ്ച്ചപ്പാട് അല്ലെങ്കില്‍ ഭ്രമമാണ് അവ തമ്മില്‍ വ്യത്യാസമുണ്ടെന്നു തോന്നിപ്പിക്കുന്നത്. സ്വപ്രയത്നത്താല്‍ മനസ്സ് മെനഞ്ഞുണ്ടാക്കുന്നവ ഫലപ്രദമാവുമ്പോള്‍ , മനസ്സ് ആ ഫലസംപാദനം അറിയുമ്പോള്‍ , സന്തോഷമെന്ന അനുഭവമുണ്ടാകുന്നു.

മനസ്സാണ് കര്‍മ്മം ചെയ്യുന്നത്. നിയതിക്കനുസൃതമായി, പ്രകൃതി നിയമങ്ങള്‍ക്കനുസരിച്ച് മന്സ്സോരോന്നു ചിന്തിക്കുകയും സൃഷ്ടിക്കുകയും അവയെ പ്രകടമാക്കുകയും ചെയ്യുന്നു. മനസ്സിന് പ്രകൃതിനിയമങ്ങള്‍തിരായും പ്രവര്‍ത്തിക്കുവാനാകും. അതിനാല്‍ മനസ്സാണ് പ്രകൃതിയെ നിയന്ത്രിക്കുന്നതെന്നും പറയാം. ആകാശത്ത് കാറ്റു സഞ്ചരിക്കും പോലെ വ്യക്തിജീവന്‍ ഈ ലോകത്ത് വര്‍ത്തിക്കുന്നു. അത്തരം പ്രവര്‍ത്തനങ്ങള്‍ സ്വാര്‍ത്ഥപരവും അഹങ്കാരവും ആണെന്ന് തോന്നുന്നുവെങ്കിലും പ്രകൃതി നിയമങ്ങള്‍ക്കുള്ളില്‍ നിന്നുകൊണ്ട് ചെയ്യേണ്ട കര്‍മ്മങ്ങള്‍ ജീവന്‍ ചെയ്തുകൂട്ടുന്നു. പ്രകൃതിയുടെ വിളിക്കനുസരിച്ച് ചലിച്ചോ ഒരിടത്ത് ചലനമില്ലാതെ നിന്നോ ജീവികള്‍ തങ്ങളുടെ ഭാഗം അഭിനയിക്കുന്നു. ഇതെല്ലാം വെറും തെറ്റിദ്ധാരണമാത്രമാണല്ലോ. മലമുകളിലെ മരച്ചില്ലകള്‍ ചാഞ്ചാടുമ്പോള്‍ മലയാണ് ചലിക്കുന്നതെന്ന് തോന്നുംപോലെ അസംബന്ധമത്രേ ഇത്. മനസ്സുള്ളപ്പോള്‍ ദൈവവുമില്ല പ്രകൃതി നിയമവുമില്ല. മനസ്സോടുങ്ങുമ്പോഴോ? അവിടെ എന്തെങ്കിലുമായിക്കൊള്ളട്ടെ.! 


ബലി ചോദിച്ചു: ഭഗവന്‍, സുഖാസക്തിയുടെ അന്ത്യം എന്റെ ഹൃദയത്തില്‍ എങ്ങിനെയാണ് ദൃഢീകരിക്കുക?

വിരോചനന്‍ പറഞ്ഞു: മകനേ ആത്മവിദ്യയാകുന്ന വളളിച്ചെടിയിലാണ് സുഖാസക്തിയുടെ അന്ത്യമെന്ന ഫലം കായ്ക്കുന്നത്. അതുണ്ടാവുന്നതോ? ആത്മദര്‍ശനം ഉണ്ടായിക്കഴിഞ്ഞുമാത്രം. അപ്പോള്‍ ഹൃദയത്തില്‍ അനാസക്തി രൂഢമൂലമാവും. ഒരേ സമയം ബുദ്ധിപൂര്‍വ്വമായ അന്വേഷണത്തിലൂടെ ആത്മാവിനെ തേടുകയും അങ്ങിനെ ആസക്തികളെ ഉപേക്ഷിക്കാന്‍ ആവുകയും വേണം.

ഒരുവന്റെ ബുദ്ധി ഉണര്‍ന്നിട്ടില്ലെങ്കില്‍ തന്റെ മനസ്സിന്റെ  പകുതി സുഖാനുഭവങ്ങളിലും, കാല്‍ഭാഗം ശാസ്ത്രപഠനങ്ങളിലും, ബാക്കി ഗുരുപൂജയിലും ആമഗ്നമാക്കണം. കുറച്ചൊന്നുണര്‍ന്ന ബുദ്ധിയുള്ളവര്‍ മനസ്സിന്റെ പകുതി ഗുരുപൂജയ്ക്കും കാല്‍ഭാഗം വീതം മറ്റുരണ്ടിനായും മാറ്റി വയ്ക്കണം. ബുദ്ധി മുഴുവനായി ഉണര്‍ന്നവര്‍  നിതാന്തജാഗ്രമായ നിര്‍മമതയോടെ മനസ്സിന്റെ പകുതിഭാഗം ഗുരുപൂജയിലും മറ്റേപകുതി ശാസ്ത്രപഠനത്തിലും മുഴുകിവേണം കഴിയാന്‍......
