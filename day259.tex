\section{ദിവസം 259}

\slokam{
ഇതാദപ്യാത്മനൈവാത്മാ ഫലമാപ്നോതി ഭാഷിതം\\
ഹരിപൂജാക്രമാഖ്യേന നിമിത്തേനാരിസൂദന  (5/43/33)\\
}

വസിഷ്ഠന്‍ തുടര്‍ന്നു: ഭഗവാന്‍ വിഷ്ണുവിനെയും മറ്റു ദേവതകളെയും പൂജിക്കുന്നതുപോലെ നിനക്ക് ആത്മപൂജയും ചെയ്യാമല്ലോ? വിഷ്ണു എല്ലാവരുടെയും ഉള്ളിന്റെയുള്ളില്‍ സ്ഥിതി ചെയ്യുന്നു. അകത്തുള്ള ഈ വിഷ്ണുവിനെ ഉപേക്ഷിച്ച് പുറത്തുള്ള മറ്റേതോ വിഷ്ണുവിനെ തിരഞ്ഞുപോകുന്നത് കേവലം നികൃഷ്ടരായവര്‍ മാത്രമാണെന്ന് പറയാം. എല്ലാ ജീവികളുടെയും ഹൃദയ ഗുഹയിലാണ് ഭഗവാന്റെ വാസം. അതാണ്‌ ഭഗവാന്റെ ശാശ്വതമായ ദേഹവും ഗേഹവും. ശംഖുചക്രഗദാധാരിയായി കാണപ്പെടുന്ന ആ രൂപം ആപേക്ഷികമായിപ്പറഞ്ഞാല്‍ വെറും രണ്ടാംതരമത്രേ. ആത്യന്തികമായ സത്യവസ്തുവിനെ ഉപേക്ഷിച്ച്, താഴെക്കിടയിലുള്ള വസ്തുക്കളെ സമാശ്രയിക്കുന്നത് തികച്ചും ഫലപ്രദമായ ഒരൌഷധത്തെ വേണ്ടെന്നു വെച്ച് മറ്റു വിഫല ചികിത്സാമാര്‍ഗ്ഗങ്ങള്‍ക്ക് പിറകെ ഓടുന്നതുപോലെ അപഹാസ്യമാണ് . 
  
സ്വയം എകാഗ്രചിത്തത്തോടെ അന്തര്യാമിയായ ആത്മാവിനെ ധ്യാനിക്കാനും അങ്ങിനെ ആത്മജ്ഞാനമാര്‍ജ്ജിക്കുവാനും കഴിയാത്തവര്‍ വിഷ്ണുഭഗവാന്റെ ബാഹ്യരൂപത്തെ പൂജിച്ചു കഴിയുന്നതു നല്ലതാണ്. കാരണം ഈ ആചാരങ്ങള്‍ ഉചിതമായി അനുഷ്ഠിക്കുമ്പോള്‍ മനസ്സ് ക്രമേണ പരിശുദ്ധി പ്രാപിച്ച് നിറഭേദങ്ങളൊടുങ്ങി പ്രശാന്തമാവുന്നു. അങ്ങിനെ മനസ്സ് പാകപ്പെട്ട് ആനന്ദത്തോടെ ആത്മജ്ഞാനത്തിനു യോഗ്യമാവുന്നു. 
 
“വാസ്തവത്തില്‍ അന്തിമ ഫലപ്രാപ്തിയുണ്ടാവുന്നത് ഞാന്‍ പറഞ്ഞതുപോലെ  ആത്മനിഷ്ഠമായിത്തന്നെയാണ്. വിഷ്ണുപൂജയെന്നൊക്കെ പറയുന്നത് അതിനുള്ള രൊഴികഴിവു  മാത്രമാണ്.” വിഷ്ണുപൂജ വഴി ലഭിക്കുന്ന വരങ്ങളും അനുഗ്രഹങ്ങളുമെല്ലാം വാസ്തവത്തില്‍ ഒരുവന്റെ ആത്മാന്വേഷണത്തിന്റെ ഫലമായി ആത്മാവില്‍നിന്നുതന്നെ ഉപലബ്ധമാവുന്നതത്രേ. എല്ലാവിധത്തിലുള്ള ഭക്ഷണസാമഗ്രികളുടെയും അടിസ്ഥാനം ഭൂമിയാണെന്നതുപോലെ വിവിധ പൂജാപദ്ധതികളും അവയുടെ പരിണിതഫലങ്ങളും ഒരുവന്റെ മനോനിയന്ത്രണത്തെയും അറിവിനെയും അടിസ്ഥാനമാക്കിയാണ് നിര്‍ണ്ണയിക്കപ്പെടുന്നത്. മണ്ണുഴുതുമറിക്കുന്നതിനും കല്ലുകള്‍ മാറ്റി നിലം പാകമാക്കുന്നതിനും പോലും മനോനിയന്ത്രണം കൂടിയേ തീരൂ എന്നതും ഒരനുബന്ധമായിപ്പറയാം.          

പരിപൂര്‍ണ്ണ മനോനിയന്ത്രണം വന്നു മനസ്സില്‍ പരമപ്രശാന്തിയും സമതാബുദ്ധിയും കളിയാടിയാലല്ലാതെ ജനനമരണചക്രത്തില്‍ ആയിരം തവണ ചുറ്റിയാലും ഈ ആവര്‍ത്തനത്തിനൊരവസാനമുണ്ടാകയില്ല. വഴിവിട്ടുവികലമായ മനസ്സിന്നുടമയായ ഒരുവനെ അതുണ്ടാക്കുന്ന ദുരിതാനുഭവങ്ങളില്‍ നിന്നും രക്ഷിക്കാന്‍ ദേവന്മാര്‍ക്കോ ത്രിമൂര്‍ത്തികള്‍ക്ക് പോലുമോ ഇത്രിലോകത്തിലും സാധിക്കുകയില്ല. അതുകൊണ്ട് രാമാ, അകത്തുള്ളതായാലും പുറത്തുള്ളതായാലും വിഷയവസ്തുക്കളുടെ സമൂര്‍ത്തഭാവമെന്ന പ്രതിഭാസത്തില്‍ ഭ്രമിക്കാതിരിക്കൂ. നിരന്തരവും നിത്യസത്യവുമായ ബോധത്തെക്കുറിച്ചുള്ള ധ്യാനം ആവര്‍ത്തിച്ചു ചുറ്റുന്ന ജനനമരണചക്രം നിര്‍ത്താന്‍ അനിവാര്യമാണ്.  
  
ആ നിത്യശുദ്ധബോധത്തെ അനുഭവിച്ചറിയൂ. സത്യത്തില്‍ അതുമാത്രമാണെല്ലാടവും നിലകൊള്ളുന്നത്. എല്ലാവിധ വിഷയബോധവും, ധാരണകളും ആശാസങ്കല്‍പ്പങ്ങളും ഉറച്ച മനസ്സോടെ നിരാകരിച്ച് അനന്തവും അചലവും മാറ്റങ്ങള്‍ക്ക് വശംവദമാവാത്തതുമായ ആ അനന്താവബോധത്തെ ധ്യാനിക്കൂ. നിനക്കങ്ങിനെ തീര്‍ച്ചയായും പ്രത്യക്ഷലോകമെന്ന ഈ നദിയുടെ മറുകരയണയാന്‍ കഴിയും. പുനര്‍ജന്മങ്ങളില്‍നിന്ന് മുക്തിയും സാദ്ധ്യമാവും.
