 
\section{ദിവസം 067}

\slokam{
സുകൃതം ദുഷ്കൃതം ചേദം മമേതി കൃതകല്പനം\\
ബലോഭുവമഹം ത്വധ്യ യുവേതി വിലസദ്ധൃദി (3/40/50)\\
}

വസിഷ്ഠന്‍ തുടര്‍ന്നു: മരണത്തിനുശേഷം  ഉടനെ 'അവിടേയും ഇവിടേയും' അല്ലാത്ത ഒരു സ്ഥിതിയിലായിരിക്കും ജീവൻ. പ്രജ്ഞ, തന്റെ കണ്ണൊന്നു തുറന്നു തുറന്നില്ല എന്ന മട്ടിലുള്ള അവസ്ഥയിലാണപ്പോള്‍ - അതിന്‌ 'പ്രധാന' എന്നു പറയും. അതായത്‌ പദാര്‍ത്ഥ ബോധം, അല്ലെങ്കിൽ ജഢ ബോധം എന്ന അവസ്ഥ. അതിനെ സ്വര്‍ഗ്ഗീയപ്രകൃതി എന്നും അപ്രത്യക്ഷമായ പ്രകൃതി (സൂക്ഷ്മപ്രകൃതി) എന്നും പറയാം. അത്‌ സചേതനമാണെന്നും അല്ലെന്നും പറയപ്പെടുന്നു. അതാണ്‌ ഓര്‍മ്മകള്‍ക്കും ഓര്‍മ്മകളില്ലാതിരിക്കുന്നതിനും കാരണമാവുന്നത്‌. അതായത്‌ അടുത്ത ജന്മത്തിനുത്തരവാദി അതാണ്‌. പ്രധാനയില്‍ ഉണര്‍വ്വുയരുമ്പോള്‍ അതിലെ ബോധം, അഹംകാരം, സ്വയം പഞ്ചഭൂതങ്ങളായും (ഭൂമി, ജലം, വായു, ആകാശം, അഗ്നി), സമയ-ദൂരമെന്ന അവിച്ഛന്ന പ്രതിഭാസമായും ജനനജീവിതങ്ങള്‍ക്കു വേണ്ട മറ്റ് എല്ലാ വസ്തുക്കളുമായും പ്രകടമാവുന്നു. ഇതെല്ലാം പിന്നെ അതാതിന്റെ പദാര്‍ത്ഥ പ്രതിരൂപങ്ങളായി സാന്ദ്രീഭവിക്കുന്നു. സ്വപ്നാവസ്ഥയിലും ഉണര്‍ന്നിരിക്കുമ്പോഴും അവ ശരീരമെന്ന വികാരമുണ്ടാക്കുന്നു.

യഥാര്‍ത്ഥത്തില്‍ ഇതെല്ലാം ചേരുന്നതാണ്‌ ജീവന്റെ സൂക്ഷ്മശരീരം. ഇതില്‍ 'ഞാന്‍ ശരീരമാണ്‌' എന്ന തോന്നല്‍ രൂഢമൂലമാവുമ്പോള്‍ ഈ സൂക്ഷ്മശരീരം സ്വയം ഭൌതികശരീരത്തിന്റെ സ്വഭാവസവിശേഷതകള്‍ വികസിപ്പിച്ച്‌ കണ്ണ്‌  മുതലായ അവയവങ്ങള്‍ ഉണ്ടാവുന്നു. ഇതെല്ലാം നടക്കുന്നത്‌ ചെറിയൊരു കമ്പനം പോലെയോ വായുവിന്റെ മന്ദഗമനം പോലെയോ ആണ്‌. ഇതെല്ലാം യാഥാര്‍ത്ഥ്യമാണെന്നു തോന്നുമെങ്കിലും അവ സ്വപ്നത്തിലെ ലൈംഗീകസുഖാനുഭവം പോലെ അയാഥാര്‍ത്ഥ്യമാണ്‌. ഒരുവന്റെ മരണസമയത്ത്‌ ജീവന്‍ ഇതെല്ലാം കാണുന്നു. അവിടെ ആ ബോധതലത്തില്‍ത്തന്നെ 'ഇതാണ്‌ ലോകം, ഇതു ഞാന്‍' എന്നു സങ്കല്‍പ്പിച്ച്‌, സ്വയം ജനിച്ചുവെന്നു വിശ്വസിച്ച്‌ ജീവന്‍ ലോകമെന്ന ആകാശത്തെ അനുഭവിക്കുന്നു. അയാള്‍ , ജീവന്‍, സ്വയമേവ ആകാശം തന്നെയാണുതാനും! 

അയാള്‍ 'ഇതെന്റെ അച്ഛന്‍, ഇതെന്റെ അമ്മ, ഇതെന്റെ ധനം' എന്നിങ്ങനെ ചിന്തിക്കുന്നു. "ഞാന്‍ ഈ അത്ഭുതങ്ങള്‍ ചെയ്തുവെന്നും, അയ്യോ! കഷ്ടം! ഞാന്‍ പാപം ചെയ്തുവെന്നും, ഞാന്‍ ചെറിയൊരു കുട്ടിയായെന്നും, ഞാന്‍ യുവാവായി എന്നുമെല്ലാം സങ്കല്‍പ്പിച്ച്‌ അയാള്‍ തന്റെ ഹൃദയത്തില്‍ ഇവയെല്ലാം ദര്‍ശിക്കുന്നു." സൃഷ്ടിയെന്ന ഈ കാനനം എല്ലാ ജീവഹൃദയങ്ങളിലും അങ്കുരിക്കുന്നു. ഒരാള്‍ മരിക്കുമ്പോള്‍ എവിടെയാണോ അവിടെ, അപ്പോള്‍ത്തന്നെ ജീവന്‍ ഈ കാട്‌ കാണുന്നു. ഈ രീതിയില്‍ വ്യക്തിജീവന്റെ ബോധത്തില്‍ എണ്ണമില്ലാത്ത ലോകങ്ങള്‍ ഉണ്ടായി നശിച്ചിട്ടുണ്ട്‌. അതുപോലെ എണ്ണമറ്റ ബ്രഹ്മാക്കളും, രുദ്രന്മാരും വിഷ്ണുമാരും, സൂര്യന്മാരും ഉണ്ടായി മറഞ്ഞിരിക്കുന്നു. സൃഷ്ടിയെന്ന ഈ മായാപ്രതിഭാസം അനവധിതവണ ഉണ്ടായിട്ടുണ്ട്‌, ഇപ്പോഴും ഉണ്ടാകുന്നു, ഇനിയും ഉണ്ടാവുകയും ചെയ്യും. കാരണം ഇതൊന്നും ചിന്തകളില്‍ നിന്നും വിഭിന്നമല്ല; അനന്താവബോധത്തില്‍ നിന്നും സ്വതന്ത്രവുമല്ല. മാനസീകവ്യാപാരം എന്നാല്‍ ബോധം തന്നെ. അതു തന്നെ പരമ സത്യവും.

