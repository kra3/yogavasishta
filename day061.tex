 
\section{ദിവസം 061}

\slokam{
ബ്രഹ്മാത്മൈക ചിദാകാശമാത്ര ബോധവതോ മുനേ:\\
പുത്രമിത്രകളത്രാണി കഥം കാനി കദാ കുത: (3/26/54)\\
}

വസിഷ്ഠന്‍ തുടര്‍ന്നു: എന്നിട്ട്‌ ഈ മഹിളകള്‍ മുന്‍പേ പറഞ്ഞ ആ മഹാത്മാവിന്റെ ഗൃഹത്തില്‍ പ്രവേശിച്ചു. കുടുംബം മുഴുവന്‍ വിലപിക്കുകയായിരുന്നു. അവരുടെ ദു:ഖത്തില്‍ അന്തരീക്ഷം പോലും വിഷണ്ണമായിരുന്നു. ശുദ്ധപ്രജ്ഞയുടെ യോഗബലത്താല്‍ ലീലയ്ക്ക്‌ അവളുടെ ചിന്തകള്‍ അപ്പോള്‍ത്തന്നെ പ്രാവര്‍ത്തികമാക്കാന്‍ കഴിഞ്ഞു. 'എന്റെ ബന്ധുക്കളായ ഇവര്‍ക്ക്‌ സരസ്വതീദേവിയേയും എന്നെയും സാധാരണക്കാരായ സ്ത്രീകളായി കാണാന്‍ സാധിക്കട്ടെ' എന്നു സ്മരിച്ചപ്പോഴേയ്ക്ക്‌ അവരാ കുടുംബത്തിലെ ദു:ഖത്തില്‍ പങ്കെടുക്കാനെത്തിയവരായി. എന്നാല്‍ ഈ അഭൌമവനിതകളുടെ തേജസ്സ്‌ ദു:ഖപൂരിതമായ ആ ഗൃഹത്തിന്റെ വിഷാദഭാവം മാറ്റി. മരിച്ചയാളിന്റെ മൂത്തപുത്രന്‍ സ്തീകളെ സ്വാഗതം ചെയ്തു. അയാള്‍ വിചാരിച്ചത്‌ ഇവര്‍ വനദേവതകളാണെന്നാണ്‌. 

അയാള്‍ പറഞ്ഞു: വനദേവതമാരായ നിങ്ങള്‍ വന്നിട്ടുള്ളത്‌ ഞങ്ങളുടെ ദു:ഖത്തെ ദൂരീകരിക്കാനാണെന്നു നിശ്ചയം. അതാണ്‌ ദിവ്യാത്മാക്കളുടെ മഹത്വം. അവര്‍ മറ്റുള്ളവരുടെ സങ്കടനിവൃത്തിക്കായി സദാ ജാഗരൂകരാണ്‌. സ്ത്രീകള്‍ ചോദിച്ചു: ഈ കുടുംബത്തെ മുഴുവന്‍ ബാധിച്ചിട്ടുള്ള ദു:ഖത്തിന്റെ ഹേതു എന്താണ്‌? അയാള്‍ പറഞ്ഞു: മഹിളാമണികളേ ഈ ഗൃഹത്തില്‍ അതീവ ദിവ്യനായ ഒരാള്‍ തന്റെ സാത്വിയായ ഭാര്യയോടൊപ്പം ധാര്‍മ്മീകജീവിതം നയിച്ചു വന്നു. ഈയിടെ അവര്‍ മക്കളേയും അവരുടെ കുട്ടികളേയും ഉപേക്ഷിച്ച്‌, വീടിനേയും പശുക്കളേയും ഇവിടെ ബാക്കിവെച്ച്‌ സ്വര്‍ഗ്ഗാരോഹണം ചെയ്തു. അതുകൊണ്ട്‌ ഞങ്ങള്‍ക്ക്‌ ഈ ലോകം മുഴുവന്‍ ശൂന്യമായി അനുഭവപ്പെടുന്നു. പറവകള്‍പോലും മരിച്ചവര്‍ക്കുവേണ്ടി കേഴുകയാണ്‌. ദേവന്മാരും സങ്കടക്കണ്ണീര്‍ മഴയായി പൊഴിക്കുന്നു. മരങ്ങള്‍ പ്രഭാതത്തിലെ ഹിമകണങ്ങളാകുന്ന കണ്ണീരിറ്റി നില്‍ക്കുന്നു. ഈ ലോകമുപേക്ഷിച്ച്‌ എന്റെ അച്ഛനമ്മമാര്‍ അനശ്വരരായ പിതൃക്കളുടെ ലോകത്തിലേയ്ക്കു പോയിരിക്കുന്നു.

ഇതുകേട്ട്‌ ലീല അയാളുടെ ശിരസ്സില്‍ തന്റെ കൈവയ്ച്ചു. ആ ക്ഷണത്തില്‍ അയാളുടെ ദു:ഖം ഇല്ലാതായി. ഇതുകണ്ട്‌ മറ്റുള്ളവരും ദു:ഖമുക്തരായി.

രാമന്‍ ചോദിച്ചു: മഹാത്മന്‍ , എന്തുകൊണ്ടാണ്‌ ലീല തന്റെ പുത്രനുമുന്നില്‍ സ്വന്തം അമ്മയായി നില്‍ക്കാതിരുന്നത്‌?

വസിഷ്ഠമുനി പറഞ്ഞു: രാമ: വിഷയവസ്തുക്കള്‍ സത്തല്ല എന്നു ശരിയായുറച്ചവര്‍ വിശ്വത്തെ മുഴുവനും ഒരൊറ്റ അവിച്ഛന്നബോധസ്വരൂപമായേ ദര്‍ശിക്കൂ. സ്വപ്നം കാണുന്നവന്‍ ലോകം കാണുന്നില്ല. എന്നാല്‍ തീരെ ബോധംകെട്ടുറങ്ങുന്നവന്‍ ചിലപ്പോള്‍ മറ്റുലോകങ്ങള്‍ കണ്ടേക്കാം. ലീല സത്യസാക്ഷാത്കാരം നേടിയിരുന്നു. "ആരൊരാള്‍ ബ്രഹ്മം, ആത്മന്‍ , എന്നിങ്ങനെ അറിയപ്പെടുന്ന സത്യവസ്തുവിനെ ബോധസ്വരൂപമായി സാക്ഷാത്കരിച്ചുവോ അവന്‌ മകനാര്‌? സുഹൃത്താര്‌? ഭാര്യയയും ബന്ധൂക്കളുമാര്‌?" ലീല ചെറുപ്പക്കാരന്റെ ശിരസ്സില്‍ കൈവെച്ചത്‌ ബ്രഹ്മത്തിന്റെ യാദൃശ്ചികമായ ഒരു കാരുണ്യപ്രകടനം എന്നേ പറയാവൂ.
