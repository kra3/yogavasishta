 
\section{ദിവസം 102}

\slokam{
യാവത്കടകസംവിത്തിസ്താവന്നാസ്തീവ ഹേമതാ\\
യാവച്ച ദൃശ്യാതാപത്തിസ്താവന്നാസ്തീവ സാ കലാ (3/80/48)\\
}

മന്ത്രി തുടര്‍ന്നു: ആത്മാവ്‌ നിര്‍മ്മല ബോധമാണെങ്കിലും ജഢതയോടു ചേര്‍ന്നു നില്‍ക്കുമ്പോള്‍ അത്‌ ചൈതന്യരഹിതമായും ജഢമായും തോന്നുന്നു. അനന്തമായ ആകാശത്തില്‍ ഈ അനന്താവബോധം എണ്ണമറ്റ പദാര്‍ത്ഥങ്ങളെ പ്രകടമാക്കുന്നു. ഇങ്ങിനെയെല്ലാം വൈവിദ്ധ്യമാര്‍ന്ന നിര്‍മ്മിതികള്‍ സത്യമാണെന്നു തോന്നുന്നുവെങ്കിലും സത്യത്തില്‍ അതെല്ലാം വെറും ഭ്രമം മാത്രമാണ്‌.. ആത്മാവ്‌ ഒന്നും ചെയ്യുന്നില്ല. അതുകൊണ്ട്‌ ആത്മാവ്‌ ബോധവും ജഢവുമാണ്‌..  കര്‍ത്താവും അകര്‍ത്താവുമാണ്‌. .  അഗ്നിയിലെ ഉണ്മ ആത്മാവാണ്‌..  എന്നാല്‍ അത്‌ ഒന്നിനേയും എരിക്കുന്നില്ല.സ്വയം എരിയുന്നുമില്ല. അത്‌ എല്ലാറ്റിന്റേയും അന്ത:സത്തയാണ്‌..  സൂര്യചന്ദ്രന്മാര്‍ക്കും അഗ്നിക്കും ദീപ്തിയേകുന്ന ശാശ്വതപ്രഭയായിരിക്കുമ്പോഴും അത്‌ അവയില്‍നിന്നും സ്വതന്ത്രമാണ്‌..  ഈ പ്രകാശ വസ്തുക്കള്‍ അണഞ്ഞാലും ആത്മാവിന്റെ പ്രഭ അണയുന്നതേയില്ല. അത്‌ എല്ലാറ്റിന്റേയും ഉള്ളില്‍നിന്ന് പ്രഭചൊരിഞ്ഞുകൊണ്ടേയിരിക്കുന്നു.

മരങ്ങളിലും വള്ളിച്ചെടികളിലും സ്ഥിതിചെയ്ത്‌ അവയെ പരിരക്ഷിക്കുന്ന പ്രജ്ഞ അതുതന്നെയാണ്‌..  ആ അനന്താവബോധം സാധാരണ വ്യാവഹാരിക തലത്തില്‍ നിന്നുനോക്കുമ്പോള്‍ സൃഷ്ടി, സ്ഥിതി, സംഹാരങ്ങള്‍ ചെയ്യുന്നയാളാണ്‌..  എന്നാല്‍ പാരമാര്‍ത്ഥിക തലത്തില്‍ , എല്ലാറ്റിന്റേയും ആത്മാവായതിനാല്‍ അതിന്‌ സൃഷ്ടിസ്ഥിതിസംഹാരം മുതലായ പരിമിത ജോലികള്‍ ഒന്നുമില്ല. ഈ ബോധത്തില്‍നിന്നും സ്വതന്ത്രമായി ലോകമില്ല. മാമലകള്‍പോലും അണുമാത്രമായ ആത്മാവിലാണു നിലകൊള്ളുന്നത്‌. സ്വപ്നാവസ്ഥയില്‍ കാണുന്ന വസ്തുക്കള്‍ സത്യമായിത്തോന്നുന്നപോലെ ആത്മാവില്‍ ക്ഷണനേരത്തേയ്ക്ക്‌ കാണപ്പെടുന്ന ഭ്രമം യുഗദൈര്‍ഘ്യം ഉള്ളതായും ഉണ്മയായും തോന്നുന്നു. ഇമചിമ്മുന്ന നേരത്തിനിടയില്‍ ഒരുയുഗം കടന്നുപോകുന്നു. വലിയൊരു നഗരം കണ്ണാടിയില്‍ പ്രതിഫലിക്കുന്നു. കാര്യങ്ങള്‍ അങ്ങിനെയിരിക്കേ എങ്ങിനെയാണ്‌ അദ്വൈതത്തെയോ ദ്വൈതത്തേയോ പറ്റി ഉറപ്പിച്ചുപറയാനാവുക? അടുത്തായും അകലെയായും ഞൊടിയിടയായും യുഗമായും കാണപ്പെടുന്നത്‌ ആ അനന്താവബോധം തന്നെയാണ്‌. ഒന്നും അതില്‍നിന്നു വേറിട്ടല്ല നിലകൊള്ളുന്നത്‌. ഇതില്‍ വിരോധാഭാസം ഒന്നുമില്ല. ഒരുകൈവളയെ കൈവള മാത്രമായി കാണുമ്പോള്‍ സ്വര്‍ണ്ണമായി ആരും അതിനെ  പരിഗണിക്കുന്നില്ല. എന്നാല്‍ കൈവള എന്നത്‌ വെറുമൊരു നാമരൂപ'വസ്തുവായി' കാണുമ്പോള്‍ സ്വര്‍ണ്ണം കാണാകുന്നു. "അതുപോലെ ഈ ലോകം സത്യമെന്നു കരുതുമ്പോള്‍ ആത്മാവിനെ കാണുന്നില്ല. എന്നാലീ വിചാരം മാറുമ്പോള്‍ ബോധം വെളിപ്പെടുന്നു." 

എല്ലാം അതാണ്‌..  അതിനാല്‍ അതുമാത്രമാണുണ്മ. അതനുഭവവേദ്യമല്ലാത്തതിനാല്‍ അത്  ഉണ്മയുമല്ല എന്നും പറയാം. വിഷയം, വിഷയി എന്നിങ്ങനെയുള്ള വേര്‍തിരിവ്‌ മായയുടെ ജാലവിദ്യയെന്നപോലെ ബോധത്തെ വിഭജിക്കുന്നതായി തോന്നുകയാണ്‌. സ്വപ്നനഗരം എന്നപോലെ അയാഥാര്‍ത്ഥമാണത്‌..  അത്‌ സത്തോ അസത്തോ അല്ല. ദീര്‍ഘമായ ഒരു ഭ്രമദൃശ്യമാണത്‌..  ഈ വിഭജനമെന്ന തോന്നലാണ്‌ ബ്രഹ്മാവു മുതല്‍ കീടങ്ങള്‍വരെയുള്ള സൃഷ്ടികളെ ഉണ്ടാക്കുന്നത്‌..  ഒരൊറ്റവിത്തില്‍ വന്മരത്തിന്റെ എല്ലാ സ്വഭാവസവിശേഷതകളും നിര്‍ലീനമായിരിക്കുന്നതുപോലെ ബോധമായി ഈ ആത്മാവില്‍ നാനാത്വം നിലകൊള്ളുന്നു. 

