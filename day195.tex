\section{ദിവസം 195}

\slokam{
യദിത്വമാത്മനാത്മാനം അവിഗച്ഛസി തം സ്വയം\\
എതത്പ്രശ്നോത്തരം സാധു ജനാസ്യത്ര ന സംശയ: (4/57/15)\\
}

രാമൻ ചോദിച്ചു: മഹാമുനേ എങ്ങിനെയാണ്‌ ഈ അയാഥാർത്ഥ്യമായ ലോകം പരമ്പൊരുളായ ബ്രഹ്മത്തിൽ സ്ഥിതിചെയ്യുന്നത്? സൂര്യതാപത്തിൽ മഞ്ഞിനെങ്ങിനെ നിലനിൽക്കാനാകും?

വസിഷ്ഠൻ പറഞ്ഞു: ഈ ചോദ്യമുന്നയിക്കാനുള്ള അവസരമിതല്ല. നിനക്കതിന്റെ ഉത്തരം ഇപ്പോൾ മനസ്സിലാവുകയില്ല. പ്രണയചാപല്യത്തിന്റെ കഥകൾ ചെറിയൊരു ബാലന്‌ താൽപ്പര്യപ്രദമാവുകയില്ലല്ലോ. എല്ലാ മരങ്ങളും അതതിന്റെ കാലത്ത് പൂവിട്ടു കായ്ക്കും. ഫലം തരും. അതുപോലെ ഞാൻ നിനക്കുപദേശിക്കുന്ന ഈ കാര്യങ്ങളും ഫലപ്രദമാവുകതന്നെ ചെയ്യും. “നീ സ്വയം സ്വപ്രയത്നത്താൽ നിന്റെ ആത്മാവിനെ ആത്മാവിന്റെ സഹായത്തോടെ അന്വേഷിക്കുകയാണെങ്കിൽ നിനക്കാ ചോദ്യത്തിനുള്ള ഉത്തരം ലഭിക്കും”.

ഞാൻ കർത്തൃത്വത്തെപ്പറ്റിയും അകർത്തൃത്വത്തെപ്പറ്റിയും ചർച്ചചെയ്തു വിവരിച്ചത് മനോപാധികളെപ്പറ്റി, ധാരണകളെപ്പറ്റി, ആശയങ്ങളെപ്പറ്റി നിനക്കു വ്യക്തമാക്കി മനസ്സിലാക്കിത്തരാനാണ്‌.. ബന്ധനം എന്നത് ഈ ധാരണകളും ഉപാധികളുമായുള്ള ബന്ധനമാണ്‌.. സ്വാതന്ത്ര്യം എന്നത് അവയിൽ നിന്നെല്ലാമുള്ള സ്വാതന്ത്ര്യവും. എല്ലാ ധാരണകളേയും, സ്വാതന്ത്ര്യമെന്ന ആശയത്തെപ്പോലും ഉപേക്ഷിക്കൂ.

ആദ്യം തന്നെ സൗഹൃദം മുതലായ സദ് ബന്ധങ്ങളുണ്ടാക്കി വിഷയങ്ങളോടും ആസക്തികളോടുമുള്ള താൽപ്പര്യം ഇല്ലാതാക്കുക. പിന്നീട്. സൗഹൃദത്തിൽ വർത്തിക്കുമ്പോഴും സൗഹൃദം പോലുള്ള ധാരണകൾ കൂടി അവസാനിപ്പിക്കുക. എല്ലാ ആഗ്രഹങ്ങളും ഇല്ലാതാക്കുക. എന്നിട്ട് അനന്താവബോധമെന്ന ആശയത്തെ ധ്യാനിക്കുക. അതുപോലും മാനസീകോപാധികളുടെ ഭാഗമാണെന്നറിയുക. കാലക്രമത്തിൽ ഈ സൂക്ഷ്മധാരണകള്‍ പോലും ഉപേക്ഷിക്കുക. അവസാനം ഇതെല്ലാമുപേക്ഷിച്ചശേഷവും എന്തു നിലനില്ക്കുന്നുവോ അതിൽ അഭിരമിക്കുക. എന്നിട്ട് ഇപ്പറഞ്ഞവയെ എല്ലാം സംത്യജിച്ച 'ആളെ'യും ത്യജിക്കുക. അങ്ങിനെ അഹംകാരമെന്ന ധാരണപോലും ഇല്ലാതായാൽ നീ അനന്താകാശം പോലെയാണ്‌.. അങ്ങിനെ ഹൃദയാകാശത്തിൽ നിന്നെല്ലാം സംത്യജിച്ചവൻ സ്വയം പരംപൊരുൾ തന്നെയാണ്‌.. അയാൾ കർമ്മനിരതമായ ഒരു ജീവിതം നയിച്ചാലും സദാ സമയം ധ്യാനനിമഗ്നനായാലും വ്യത്യാസമൊന്നുമില്ല. കർമ്മവും അകർമ്മവും അയാൾക്കൊരുപോലെ നിഷ്പ്രയോജനമാണ്‌..

രാമാ, ഞാൻ എല്ലാ ശാസ്ത്രങ്ങളും പഠിച്ച് സത്യമെന്തെന്ന് ഗവേഷണം നടത്തിയിട്ടുണ്ട്. എല്ലാ ധാരണകളുടേയും ഉപാധികളുടേയും സമ്പൂർണ്ണമായ ന്യാസത്തിലൂടെ മാത്രമല്ലാതെ മുക്തിയില്ല. ഈ നാമരൂപനിർമ്മിതമായ ലോകം ഹിതവും അഹിതവുമായ പദാർത്ഥങ്ങളുടെ സമ്മിശ്രമാണ്‌.. ഈ വസ്തുക്കൾക്കായി ആളുകൾ കഷ്ടപ്പെട്ടു പണിയുന്നു. എന്നാൽ ആത്മജ്ഞാനത്തിനായി ആരും പരിശ്രമിക്കുന്നില്ല. ആത്മജ്ഞാനമാർജ്ജിച്ച ഋഷികൾ മൂന്നുലോകങ്ങളിലും വിരളമാണ്‌.. ഒരാൾ ലോകചക്രവർത്തിയോ സ്വർഗ്ഗത്തിലെ രാജാവോ ആയിരിക്കാം. എന്നാൽ ഇവയെല്ലാം പഞ്ചഭൂതനിർമ്മിതികൾ മാത്രം! കഷ്ടം! ആളുകൾ ഈ നിസ്സാര വസ്തുക്കൾക്കായി ജീവിതത്തെ നശിപ്പിക്കുന്നു.എത്ര ലജ്ജാകരം! അവരൊന്നും ഉത്തമനും  ആത്മവിദ്യാ സമ്പന്നനുമായ ഒരു ഋഷിയെ സമീപിക്കുന്നില്ല. സൂര്യചന്ദ്രന്മാർക്കുപോലും പ്രവേശനമില്ലാത്ത പരമപദത്തിലാണല്ലോ ആത്മജ്ഞാനി വിരാജിക്കുന്നത് (സുഷുമ്നയാണോ ഇത്?) . അതുകൊണ്ട് ആത്മജ്ഞാനിയെ വശീകരിക്കാൻ ഈ വിശ്വം മുഴുവനും നൽകുന്ന ലൗകീകസുഖങ്ങൾക്കും നേട്ടങ്ങൾക്കും കഴിയുകയില്ല. 

