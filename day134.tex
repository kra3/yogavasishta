\newpage
\section{ദിവസം 134}

\slokam{
സനിതംബസ്തനീ  ചിത്രേ ന സ്ത്രീ സ്ത്രിധര്‍മ്മിണീ  യഥാ\\
തഥൈവാകാരചിന്തേയം കര്‍ത്തും യോഗ്യാ ന കിഞ്ചന (3/113/32)\\
}

വസിഷ്ഠന്‍ തുടര്‍ന്നു: ഈ വാസനകളെ, അജ്ഞതയെ, മനുഷ്യന്‍ നിഷ്പ്രയാസം ആര്‍ജ്ജിച്ചു കൂട്ടിവയ്ക്കുന്നു. സുഖദായകമാണെന്നുതോന്നുമെങ്കിലും അവ ദു:ഖത്തിനു കാരണമാകുന്നു. ആത്മജ്ഞാനത്തെ മറയ്ക്കുന്നതുകൊണ്ടാണ്‌ ഇത്‌ സുഖം തരുന്നതായി തോന്നുന്നത്‌..  ലവണ രജാവിന്‌ ഒരേയൊരുമണിക്കൂറിനെ ഒരായുസ്സുകാലമാക്കാന്‍ കഴിഞ്ഞുവല്ലോ. ഈ മനോപാധി, വാസന, എന്തെങ്കിലും ചെയ്യാന്‍ സ്വയം അശക്തമാണ്‌.. എന്നാല്‍ കണ്ണാടിയില്‍ക്കാണുന്ന ദീപനാളത്തിന്റെ ഇളക്കം പോലെ വാസനകള്‍ വളരെ ഊര്‍ജ്ജസ്വലമായി കാണപ്പെടുന്നു.

"മനോഹരമായി വരച്ചുവച്ച ചിത്രപടത്തിലെ സ്ത്രീയ്ക്ക്‌ ഒരു സ്തീയുടെ കര്‍ത്തവ്യങ്ങള്‍ ചെയ്യാനാകാത്തതുപോലെ വാസനകള്‍ പ്രബലങ്ങളാണെന്നു തോന്നുമെങ്കിലും സ്വയമായി എന്തെങ്കിലും ചെയ്യാന്‍ അവ പര്യാപ്തമല്ല." ജ്ഞാനിയെ ഭ്രമിപ്പിക്കാന്‍ അതിനാവില്ല. എന്നാല്‍ മൂഢനെ അതു കീഴടക്കുന്നു. മരുഭൂമിയിലെ കാനല്‍ ജലം കണ്ട്‌ മൃഗങ്ങള്‍ വിഡ്ഢികളാവുന്നു. എന്നാല്‍  ബുദ്ധിയുള്ള മനുഷ്യന്‌ അതിന്റെ സത്യമറിയാം. മനോപാധികള്‍ക്ക്‌ നൈമിഷികമായ അസ്തിത്വമേയുള്ളു; എങ്കിലും ഒഴുക്കുള്ള നദിപോലെ അതിനു സ്ഥിരതയുള്ളതായി തോന്നുന്നു. അതിന്‌ സത്യത്തെ മൂടാന്‍ കഴിയുന്നതുകൊണ്ട്‌ സ്വയം ഉണ്മയാണെന്നു ഭാവിക്കുന്നു. എന്നാല്‍ അതിനെ അറിയാന്‍ തുനിഞ്ഞാലോ, അതിന്റെ 'കഥയില്ലായ്മ' വെളിപ്പെടുന്നു. തുലോം ദുര്‍ബ്ബലമായ ചെറുനാരുകള്‍ പിരിച്ചു കൂട്ടി ശക്തിയുള്ള കയറുണ്ടാക്കുന്നതുപോലെ വാസനകള്‍ ശക്തിയും ദൃഢതയും ആര്‍ജ്ജിക്കുന്നത്‌ പ്രത്യക്ഷലോകത്തിലെ മുമ്പു പറഞ്ഞ ഗുണഗണങ്ങളാലാണ്‌. . 

വാസനകള്‍ പെരുകുന്നതായി തോന്നുന്നുവെങ്കിലും അതു സത്യമല്ല. തീനാളത്തിന്റെ തുമ്പിനെ പിടിക്കാനാവാത്തതുപോലെ അന്വേഷണത്തില്‍ അതപ്രത്യക്ഷമാവുന്നു. എങ്കിലും ആകാശത്തിന്റെ നീലനിറം പോലെ വാസനകള്‍ക്ക്‌ അസ്തിത്വമുണ്ടെന്നു തോന്നുന്നു. 'രണ്ടാമതൊരു' ചന്ദ്രന്റെ ദൃശ്യം പോലെ, സ്വപ്നത്തിലെ വസ്തുക്കളെപ്പോലെ, അതു ചിന്താക്കുഴപ്പമുണ്ടാക്കുന്നു. ശാന്തമായി വഞ്ചിയില്‍ പോകുന്നവര്‍ക്ക്‌ നദീതീരമാണ്‌ നീങ്ങുന്നതെന്ന് തോന്നുമ്പോലെയാണിത്‌..  കര്‍മ്മോന്മുഖമാവുമ്പോള്‍ പ്രത്യക്ഷലോകമെന്ന ഒരു നീണ്ടസ്വപ്നത്തെ അതു കാണിക്കുന്നു. എല്ലാ ബന്ധങ്ങളേയും അനുഭവങ്ങളെയും അതു വികടമാക്കുന്നു.

ഈ അവിദ്യ, അല്ലെങ്കില്‍ വാസനകളാണ്‌ ദ്വന്ദതയെ ഉണ്ടാക്കുന്നതും ആ ഭാവത്തെ വളര്‍ത്തുന്നതും. അനുഭവങ്ങളും ധാരണകളും തമ്മിലുള്ള അന്തരത്തിനും വിഭാഗീയചിന്തയ്ക്കും അതുതന്നെ കാരണം. ഈ അയാഥാര്‍ത്ഥ്യ സ്ഥിതിയെപ്പറ്റി അവബോധമുദിക്കുമ്പോള്‍ മനസ്സു നിലയ്ക്കുന്നു. വെള്ളത്തിന്റെ ഒഴുക്കു നിലയ്ക്കുമ്പോള്‍ നദിയില്ല.

രാമന്‍ ചോദിച്ചു: മഹാത്മന്‍, മരീചികയില്‍ കാണുന്ന നദി ഒരിക്കലും അവസാനിക്കുന്നില്ലല്ലോ! ലോകത്തെ മുഴുവന്‍ അന്ധമാക്കുന്ന ഈ അജ്ഞാനം എത്ര വിസ്മയകരം! ആശ, വെറുപ്പ്‌, എന്നീ രണ്ടു ചാലകശക്തികളാണ്‌ അജ്ഞാനത്തെ, വാസനകളെ, പരിപോഷിപ്പിക്കുന്നത്‌..  ഭഗവന്‍,   ഈ മനോപാധികള്‍ , അല്ലെങ്കില്‍ അജ്ഞാനം, വീണ്ടുമൊരിക്കലും ഉദിച്ചുയരാതിരിക്കുവാന്‍ എന്താണ്‌ ഏറ്റവും ഉചിതമായ മാര്‍ഗ്ഗം?

