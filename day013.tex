 
\section{ദിവസം 013}

\slokam{
ഉദ്ബോധയതി ദോഷാലിം നികൃന്തതി ഗുണാവലിം\\
നരാണാം യൗവ്വനോല്ലാസോ വിലാസോ ദുഷ്കൃതശ്രിയാം (1/20/29)\\
}

രാമന്‍ തുടര്‍ന്നു: ബാല്യം കടന്ന് മനുഷ്യന്‍ യൌവ്വനത്തില്‍ എത്തുന്നു. എന്നാല്‍ അവന്റെ നിര്‍ഭാഗ്യത്തെ അവന്‌ പിരിയാനാവുന്നില്ല. യൌവ്വനാവസ്ഥയില്‍ അവന്‍ പലേവിധ മാനസീക വ്യതിയാനങ്ങള്‍ക്കും വിധേയനായി ദുരിതങ്ങളില്‍ നിന്നും കൂടിയ ദുരിതങ്ങളിലേയ്ക്ക്‌ നീങ്ങുന്നു. അവന്‍ വിവേകബുദ്ധി ഉപേക്ഷിച്ച്‌ തന്റെ ഹൃദയത്തില്‍ നിവസിക്കുന്ന കാമം എന്ന പിശാചിനെ ആലിംഗനം ചെയ്യുന്നു. അവനില്‍ ആശകളും ആശങ്കകളും നിറയുന്നു. യൌവ്വനത്തില്‍ വിവേകനഷ്ടം വന്നിട്ടില്ലാത്തവന്‌ ഏതൊരനുഭവവും നിഷ്പ്രയാസം നേരിടാനാവും. എനിക്ക്‌ ഈ ക്ഷണികമായുള്ള യൌവ്വനാവസ്ഥ പ്രിയമേയല്ല. ഇക്കാലത്ത്‌ അല്‍പ്പായുസ്സായ സുഖാനുഭവത്തിനു പിറകേ നീണ്ടുനില്‍ക്കുന്ന ദുരിതാനുഭങ്ങള്‍ ഉണ്ടാകുമ്പോഴും മനുഷ്യന്‍ അസ്ഥിരമായതിനെ സ്ഥിരമെന്നു കണക്കാക്കി അതിനു പിറകേ ഓടിക്കൊണ്ടേയിരിക്കുന്നു. കഷ്ടം! മറ്റുള്ളവരെക്കൂടി ദുരിതത്തിലാഴ്ത്തുന്ന പല പ്രവര്‍ത്തനങ്ങളിലും അവന്‍ ഏര്‍പ്പെടുകയും ചെയ്യുന്നു.

പ്രിയപ്പെട്ടയാള്‍ പിരിഞ്ഞുപോവുമ്പോള്‍ യുവാവിന്റെ ഹൃദയം കാട്ടുതീയില്‍പ്പെട്ട വൃക്ഷമെന്നപോല്‍ കാമത്തീയില്‍ എരിയുന്നു. എത്ര പരിശ്രമിച്ചാലും യുവാക്കളുടെ ഹൃദയം കറപുരണ്ടതും അശുദ്ധവുമത്രേ. തന്റെ പ്രിയപ്പെട്ടവള്‍ അരികില്‍ ഇല്ലാത്തപ്പോഴും അവന്‍ അവളുടെ സൌന്ദര്യത്തെപ്പറ്റി ചിന്തിച്ചുകൊണ്ടേയിരിക്കുന്നു. അങ്ങിനെ ആശാപാശത്തിലുഴറുന്നവനെപ്പറ്റി സദ്ജനങ്ങള്‍ക്ക്‌ മതിപ്പുണ്ടാവുകയില്ല. യൌവ്വനം ശാരീരികവും മാനസീകവുമായ രോഗങ്ങളാല്‍ ബാധിക്കപ്പെട്ടിരിക്കുന്നു. നന്മയും തിന്മയും ചിറകുകളാക്കിയ ഒരു പക്ഷിയുമായി യൌവ്വനകാലത്തെ താരതമ്യപ്പെടുത്താം. ഒരുവന്റെ സദ്ഗുണങ്ങളെ കാറ്റില്‍പ്പറത്തുന്ന മണല്‍ക്കാറ്റാണത്‌..

"യൌവ്വനം മനുഷ്യഹൃദയത്തില്‍ നൈസര്‍ഗ്ഗീകമായുള്ള നന്മകളെ അടിച്ചമര്‍ത്തി തിന്മകളെ ഉത്തേജിപ്പിക്കുന്നു. ദുഷ്ടതയ്ക്കു വളംവയ്ക്കുന്ന ഒരു കാലഘട്ടമാണ്‌ യൌവ്വനം."

യൌവ്വനത്തില്‍ മോഹവിഭ്രാന്തിയും ആസക്തിയും സഹജം. യൌവ്വനം ശരീരത്തിന്‌ അഭികാമ്യമാണെങ്കിലും മനസ്സിന്‌ അപചയമാണതിന്റെ ഫലം. മനുഷ്യന്‍ സുഖം എന്ന മരീചികയ്ക്കുവേണ്ടി പരിശ്രമിച്ചലഞ്ഞ്‌ ഒടുവില്‍ ദു:ഖത്തിന്റെ കൂപത്തില്‍ പതിക്കുന്നു. എനിക്കതുകൊണ്ട്‌ യൌവ്വനത്തോട്‌ പ്രതിപത്തിയില്ല. 

യൌവ്വനകാലം ശരീരത്തെവിട്ടു പോവുമ്പോഴും ആസക്തികള്‍ക്ക്‌ കുറവൊന്നുമുണ്ടാകുന്നില്ല എന്നു മാത്രമല്ല അവ കൂടുതല്‍ ശക്തിയോടെ മനുഷ്യന്റെ നാശത്തിനു വഴിതെളിക്കുന്നു. യൌവ്വനത്തില്‍ അതിയായി അഭിരമിക്കുന്നവന്‍ മനുഷ്യരൂപത്തിലുള്ള ഒരു മൃഗമത്രേ. പ്രലോഭനങ്ങളില്‍ വീണുപോവാതെ യൌവ്വനത്തിന്റെ ദുഷ്‌വികാരങ്ങളെ തരണം ചെയ്തു ജീവിതം നയിച്ചവര്‍ ആരാധ്യരും മഹാത്മാക്കളുമാണ്‌.. കാരണം വലിയൊരു സമുദ്രത്തിന്റെ മറുകരയെത്താന്‍ ഒരുവനു സാധിച്ചേക്കാം. എന്നാല്‍ ഇഷ്ടാനിഷ്ടങ്ങളുടെ പിടിയില്‍പ്പെടാതെ യൌവ്വനത്തിന്റെ മറുകരയെത്തുക എന്നത്‌ അതീവദുഷ്കരം തന്നെ.

