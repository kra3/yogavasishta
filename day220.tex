\section{ദിവസം 220}

\slokam{
അദൃശ്യൈവാത്തി മാംസാസ്ഥിരുധിരാദി ശരീരകാത് \\
മനോബിലവിലീനൈഷാ തൃഷ്ണാവനശുനീ നൃണാം (5/15/8) \\
}


വസിഷ്ഠൻ തുടർന്നു: അത്മാവ് സ്വയംമറന്ന് വസ്തുക്കളുമായും അവനൽകുന്ന അനുഭവങ്ങളുമായും താദാത്മ്യഭാവം കൈക്കൊള്ളുമ്പോൾ താനേ അശുദ്ധമാകുന്നു. അവിടെ ആസക്തി ഉദയം ചെയ്യുന്നു. കൂടുതൽ അനുഭവിക്കുവാനുള്ള ത്വര, ഭ്രമാത്മകതയെ പരിപോഷിപ്പിച്ചു തീവ്രമാക്കുന്നു. വിശ്വപ്രളയത്തെ നേരിടാൻ ശിവാദിദേവകൾക്കു കഴിയുമായിരിക്കും എന്നാൽ ആർത്തിയുടെ അഗ്നിയെ ചെറുക്കാൻ ആർക്കുമാവില്ല. രാമാ, എന്തെല്ലാം ദുരിതങ്ങളും കലാപങ്ങളും ഈ ലോകത്തുണ്ടായിട്ടുണ്ടോ അവയെല്ലാം ആർത്തിയുടെ ഫലമാണ്‌.. “കാണാമറയത്തിരുന്നുകൊണ്ട് ആർത്തിയെന്ന ഈ സത്വം ദേഹത്തിലെ മാംസ-രക്താസ്ഥികളെയെല്ലാം നിശ്ശേഷം ആഹരിക്കുന്നു.” 

ചില നിമിഷങ്ങളിൽ അതൊന്നടങ്ങിയപോലെ തോന്നുന്നുവെന്നാലും അടുത്ത നിമിഷം അതു വളർന്നു വലുതാവുന്നു. ആർത്തിയുടെ പിടിയിൽപ്പെട്ട് മനുഷ്യൻ ക്ഷീണിതനായി, ശോഭയറ്റ്, പരിതാപകരമായ അവസ്ഥയിൽ ദുഷ് പ്രവണതകളോടെ വിഭ്രാന്തിയിൽ കഴിഞ്ഞുകൂടുന്നു. എന്നാൽ ഈ ആർത്തിത്വര ഇല്ലാതായാലോ പ്രാണൻ ശുദ്ധദിവ്യമായിത്തീർന്ന് പവിത്രഗുണങ്ങൾ അവന്റെ ഹൃദയത്തെ അലങ്കരിക്കുന്നു. അജ്ഞാനിയുടെ ഹൃദയത്തെ മാത്രമേ ആർത്തിയുടെ ഒഴുക്ക് ബാധിക്കുകയുള്ളു. മൃഗങ്ങൾ തീറ്റയ്ക്കു കൊതിച്ച് കെണിയിൽപ്പെടുന്നതുപോലെ മനുഷ്യൻ ആർത്തിയുടെ പിന്നാലെപോയി നരകങ്ങളില്‍പ്പതിക്കുന്നു. 

ജരാജന്യമായ ആന്ധ്യം പോലും ആർത്തിത്വര നിമിഷനേരംകൊണ്ട് ഒരുവനിലുണ്ടാക്കുന്ന വിഭ്രാന്തിയോളം വരില്ല. ആർത്തി അവനെ തീരെ ചെറുതാക്കി നാണംകെടുത്തുന്നു. ഭഗവാൻ വിഷ്ണുപോലും ഭിക്ഷ യാചിക്കാൻ പോയപ്പോൾ കുള്ളനായ വാമനവേഷമെടുത്തു. അതുകൊണ്ട് ആർത്തിയെ വളരെ ദൂരെനിന്നേ ഉപേക്ഷിക്കണം.കാരണം എല്ലാ ദു:ഖങ്ങൾക്കും കാരണമായി മനുഷ്യനെ അധ:പ്പതിപ്പിക്കുന്നത് ആർത്തിയാണല്ലോ. 

എന്നാൽ സൂര്യനുദിക്കുന്നതും പ്രഭാസിക്കുന്നതും ആര്‍ത്തിയാല്‍ത്തന്നെയാണ്. കാറ്റടിക്കുന്നു; പർവ്വതങ്ങൾ നിലകൊള്ളുന്നു; ഭൂമി ജീവജാലങ്ങൾക്കു താങ്ങാകുന്നു; എന്നുവേണ്ട, മൂന്നു ലോകങ്ങളും നിലനില്‍ക്കുന്നത് പോലും ഈ ത്വരകൊണ്ടാണ്‌.. ത്രിലോകങ്ങളിലെ ജീവജാലങ്ങളെ ബന്ധിപ്പിക്കുന്നത് ആർത്തിയെന്ന ഈ കയറാണ്‌. ഈ കയറൊഴിച്ച് എത്ര ബലമുള്ളതാണെങ്കിലും  മറ്റേതുകയറും വേണമെങ്കിൽ നമുക്കു പൊട്ടിക്കാം. അതിനാൽ രാമാ, ആശയ-സങ്കൽ പ്പ ധാരണകൾ എല്ലാം ഉപേക്ഷിച്ച് ആസക്തികളിൽ നിന്നു വിടുതൽ നേടൂ. ചിന്തകളോ ധാരണകളോ ഇല്ലെങ്കിൽ മനസ്സില്ല എന്നറിയൂ. 

ആദ്യമായി ‘ഞാൻ’, ‘നീ’, ‘അത്’, തുടങ്ങിയ ചിന്തകളെ മനസ്സിൽ നിന്നു കളയുക. കാരണം ഈ ധാരണകളുണ്ടാക്കുന്ന മനോദൃശ്യങ്ങളാണ്‌ ആഗ്രഹാസക്തികളേയും പ്രത്യാശകളേയും ഉണ്ടാക്കുന്നത്. അങ്ങിനെയുള്ള മനോദൃശ്യങ്ങൾ നിന്റെ ഉള്ളിലുയരാൻ ഇടയാവാതെയിരുന്നാൽ നീയും ജ്ഞാനിയാകും. ആർത്തിയും അഹംകാരവും രണ്ടല്ല. എല്ലാ പാപങ്ങൾക്കും ഉറവിടമതാണ്‌.. രാമാ, ജ്ഞാനത്തിന്റെ, വിവേകത്തിന്റെ വാളുകൊണ്ട് അഹംകാരത്തിന്റെ വേരറുത്ത് ഭയമകറ്റൂ. 

