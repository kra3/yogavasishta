\newpage
\section{ദിവസം 034}

\slokam{
യുക്തിയുക്തമുപാദേയം വചനം ബലകാദപി\\
അന്യതൃണമിവ ത്യാജ്യമപ്യുക്തം പദ്മജന്മനാ (2/18/3)\\
}

വസിഷ്ഠന്‍ തുടര്‍ന്നു: ഈ ശാസ്ത്രത്തിലെ വിജ്ഞാനത്തിന്റെ വിത്ത്‌ പാകുന്നവന്‌  താമസംവിനാ സത്യസാക്ഷാത്കാരമെന്ന ഫലം കിട്ടുന്നു. മാനുഷീകമാണെങ്കിലും സത്യം വെളിപ്പെടുമ്പോള്‍ സര്‍വ്വത്മനാ സ്വീകരിക്കുകയും ദിവ്യമെന്നുപറഞ്ഞുതന്നാലും അസത്യമാണെങ്കില്‍ അതിനെ ഉപേക്ഷിക്കുകയും വേണം. "ചെറിയ കുട്ടി പറഞ്ഞാലും വിജ്ഞാനപ്രദമായ വാക്കുകള്‍ നാം സ്വീകരിക്കണം;  മറിച്ചാണെങ്കില്‍ ബ്രഹ്മാവു പറഞ്ഞതാണെങ്കില്‍ ക്കൂടി വെറും പുല്ലുപോലെ തള്ളിക്കളയണം."

ഇതിലെ ശാസ്ത്രോക്തികള്‍ കേട്ട്‌, ചിന്തിച്ച്‌, സാധകന്‌ അളവില്ലാത്ത അറിവിന്റെ വെളിപാടും ദൃഢനിശ്ചയവും ഇളക്കമേതുമില്ലാത്ത ആത്മശാന്തിയുടെ ശീതളിമയും സ്വായത്തമാക്കാം. അവര്‍ണ്ണനീയമായ മഹത്വമാര്‍ന്ന് അയാള്‍ താമസംവിനാ സ്വയം ഒരു ഋഷിയാവുന്നു. മായ, അല്ലെങ്കില്‍ മോഹവിഭ്രാന്തിയെന്ന മായാജാലം എന്തെന്നു തിരിച്ചറിഞ്ഞ ആ മുനി എണ്ണമില്ലാതുള്ള ബ്രഹ്മാണ്ഡങ്ങളുടെ അടിസ്ഥാനമായി വിരാജിക്കുന്ന ആ അപരിച്ഛിന്നമായ ബോധസത്തയെ അനന്തമായ ദിവ്യദൃഷ്ടിയാല്‍ സാക്ഷാത്കരിക്കുന്നു. അയാള്‍ ഓരോ അണുവിലും അനന്തതയെ ദര്‍ശിക്കുന്നതുകൊണ്ട്‌ സര്‍ഗ്ഗപ്രക്രിയകളുടെ ഉയര്‍ച്ച താഴ്ച്ചകളില്‍ ചഞ്ചലചിത്തനാവുന്നില്ല. അതുകൊണ്ട്‌ അയാള്‍ സ്വമേധയാ വന്നുചേരുന്നതിനെ സര്‍വ്വാത്മനാ സ്വീകരിക്കുന്നു. അയാള്‍ പൊയ്പോയതിനുപുറകേ പരിശ്രമിച്ചോടുന്നില്ല; തനിക്കു നഷ്ടപ്പെട്ടതിനേക്കുറിച്ചു ദു:ഖിക്കുന്നുമില്ല. 

രസകരമായ വളരെയേറെ കഥകളാല്‍ അലംകൃതമാകയാല്‍ ഈ ഗ്രന്ഥത്തിലെ അറിവുകള്‍ മനസ്സിലാക്കാന്‍ എളുപ്പമുള്ളതാണ്‌. ഈ ഗ്രന്ഥം പഠിച്ച്‌ അതിലെ അര്‍ത്ഥതലങ്ങള്‍ ധ്യാനിച്ചുറച്ചവന്‌ മറ്റു തപസ്സുകളോ ആചാരങ്ങളോ മന്ത്രജപമോ ചെയ്യേണ്ട ആവശ്യമില്ല. അറിവ്‌ നല്‍കുന്ന മുക്തിലാഭത്തേക്കാള്‍ മഹത്തരമായി  മറ്റ്  എന്തുണ്ട് ? കാണപ്പെടുന്ന ലോകത്തിന്റെ മായാവലയത്തില്‍ ജ്ഞാനി കുടുങ്ങുന്നില്ല. ചിത്രപടത്തില്‍ വിദഗ്ധമായി വരച്ചുവച്ചിട്ടുള്ള പാമ്പിനെ ആര്‍ക്കാണു ഭയം? ഇഹലോകം വെറുമൊരു പ്രകടിത ദൃശ്യം മാത്രമാണെന്നറിയുമ്പോള്‍ അതില്‍ അമിതാഹ്ലാദമോ ദു:ഖമോ ഉണ്ടാവുകയില്ല. ഇത്തരം വേദഗ്രന്ഥങ്ങള്‍ സുലഭമാണെങ്കിലും ദു:ഖമൂലമായ ഇന്ദ്രിയവിഷയസുഖങ്ങള്‍ക്കുപിറകേ മനുഷ്യന്‍ പോകുന്നത്‌ എത്ര കഷ്ടം.

രാമ: സത്യം സ്വയം അനുഭവരൂപേണ ഉള്ളില്‍ തെളിഞ്ഞുകിട്ടിയില്ലെങ്കില്‍ ഉദാഹരണങ്ങള്‍കൊണ്ട്‌ മാത്രമേ സത്യജ്ഞാനം മനസ്സിലുറയ്ക്കൂ. അത്തരം സചിത്രോദാഹരണങ്ങള്‍ ഈ കൃതിയില്‍ കൊടുത്തിരിക്കുന്നത്‌ കൃത്യമായ ലക്ഷ്യങ്ങളോടെയും നിയതമായ ഉദ്ദേശ്യങ്ങളോടെയുമാണ്‌. ഉദ്ദേശ്യങ്ങള്‍ക്കപ്പുറം ഈ കഥകളുടെ സാരത്തെ വിപുലമാക്കുകയോ അവയെ അക്ഷരാര്‍ത്ഥത്തില്‍ എടുക്കുകയോ അരുത്‌. ഈ വേദഗ്രന്ഥം പഠിക്കുമ്പോള്‍ ലോകം സ്വപ്നസദൃശമായി അനുഭവപ്പെടുന്നതിനുവേണ്ടിയാണ്‌ ഉദാഹരണങ്ങള്‍ (കഥകള്‍) നിരത്തിയിരിക്കുന്നത്‌. വക്രബുദ്ധികള്‍ ഈ കഥകളെ ദുര്‍വ്യാഖ്യാനം ചെയ്യില്ല എന്നു പ്രത്യാശിക്കട്ടെ.
