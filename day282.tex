\section{ദിവസം 282}

\slokam{
ശേഷസ്തു ചേതനോ ജീവ: സ ചേച്ചേത്യേന ചേതതി\\
അന്യേന ബോദ്ധ്യമാനോഽസൌ നാത്മതത്വവപുര്‍ഭവേത് (5/59/16)\\
}

മാണ്ഡവ്യമുനി കൊട്ടാരത്തില്‍ നിന്നും പോയിക്കഴിഞ്ഞപ്പോള്‍ സുരാഗു ഇങ്ങിനെ മനനം ചെയ്തു: ’ഞാന്‍’ എന്ന് പറയുന്നതെന്താണ്? ഞാന്‍ മേരുപര്‍വ്വതമല്ല. അത് എന്റേതല്ല. ഞാനീ പര്‍വ്വതഗോത്രവര്‍ഗ്ഗമല്ല, അതെന്റെതുമല്ല. ഇതൊരു രാജ്യം! ആ ധാരണയും ഞാനിതാ ഉപേക്ഷിക്കുന്നു. ഇനി ഈ നഗരം. അതും ഞാനല്ല. അതെന്റെതുമല്ല. അതും ഞാനിതാ ഉപേക്ഷിക്കുന്നു. അതുപോലെ ഞാന്‍ കുടുംബത്തിന്റെ എല്ലാ ബന്ധുതകളും ഭാര്യാപുത്രമിത്രാദികളെയും ഉപേക്ഷിച്ച് ഇനി എന്റെ ശരീരത്തെപ്പറ്റി അന്വേഷിക്കട്ടെ. വെറും ജഡമായ, മാംസാസ്ഥികളോ രക്തമോ ആന്തരാവയവങ്ങളോ, അല്ല ഞാന്‍.  അവ ജഡവും ഞാന്‍ ചൈതന്യവുമാണ്.      

ഞാന്‍ ആസ്വാദന അനുഭവങ്ങളല്ല. അവ എന്റെതുമല്ല. ഈ ബുദ്ധിയും ഇന്ദ്രിയങ്ങളും ഞാനല്ല, അവ എന്റേതല്ല. എന്തെന്നാല്‍ അവ വെറും ജഡം. ഞാന്‍ ചൈതന്യം. ഞാന്‍ ജനനമരണങ്ങള്‍ക്ക് മൂലകാരണമായ മനസ്സല്ല. ഞാന്‍ വിവേകബോധമോ, അഹംകാരമോ അല്ല. കാരണം ഈ ധാരണകള്‍ ഉണ്ടാവുന്നത് തന്നെ ജഡമായ മനസ്സിലല്ലേ? അപ്പോള്‍പ്പിന്നെ എന്താണവശേഷിക്കുന്നത്? 

“ബാക്കി അവശേഷിക്കുന്നത് ചൈതന്യമായ ജീവനാണ്. എന്നാല്‍ അത് വിഷയ-വിഷയീ ബന്ധത്തില്‍ ആമഗ്നമാണ് എന്നതാണനുഭവം. അറിവിനും ധാരണകള്‍ക്കും നിദാനമായ വസ്തു ആത്മാവല്ല.” അതിനാല്‍ അറിയപ്പെടാവുന്നതിനെ, ആ വസ്തുവിനെ  ഞാന്‍ ഇതാ ഉപേക്ഷിക്കുന്നു. ഇനിയും ബാക്കി നിലനില്‍ക്കുന്നത് ശുദ്ധമായ ബോധം മാത്രം. അതില്‍ സംശയത്തിന്റെ കളങ്കലേശമില്ല. ഞാന്‍ അനന്തമായ ആത്മാവാണ്. ആത്മാവില്‍ പരിമിതികളില്ല.

സൃഷ്ടാവായ ബ്രഹ്മാവ്, ദേവരാജാവായ ഇന്ദ്രന്‍, മൃത്യുദേവനായ യമന്‍, വായുഭഗവാന്‍, തുടങ്ങിയ എണ്ണമറ്റ ദേവതകള്‍പോലും ഈ അനന്താവബോധത്തില്‍ കോര്‍ത്തുവച്ച മണികളത്രേ. സര്‍വ്വശക്തമായ ബോധം, അതായത് ചിത്ശക്തി, വസ്തുബോധത്തിന്റെ പരിമിതികളില്ലാത്തതാണ്. എല്ലാത്തിന്റെയും സത്തയാണെങ്കിലും അത് ഭാവാഭാവങ്ങള്‍ക്കതീതമാണ്. എല്ലാ ജീവജാലങ്ങളെയും അത് വ്യാപരിച്ചിരിക്കുന്നു. എല്ലാത്തിന്റെയും സൌന്ദര്യം അതാണ്‌.. പ്രഭയുമതാണ്. എല്ലാ രൂപങ്ങളുടെയും അവയുടെ പരിണാമങ്ങളുടേയും സത്ത അതാണ്‌.. എന്നാല്‍ അത് ഇവയ്ക്കെല്ലാം അതീതമത്രേ. 
    
എല്ലായ്പ്പോഴും അനന്താവബോധമെന്നത് എല്ലാത്തിന്റെയും എല്ലാമെല്ലാമാണ്. അതാണ്‌ അസ്തിത്വത്തിന്റെ പതിന്നാല് ലോകതലങ്ങളായി നിലകൊള്ളുന്നത്. ഈ പ്രപഞ്ചത്തിന്റെ നിലനില്‍പ്പെന്ന ധാരണയും ആ ബോധമാണ്. സുഖദു:ഖാദി ധാരണകള്‍ വെറും മിഥ്യ. കാരണം ഈ അനന്താവബോധം സര്‍വ്വവ്യാപിയും സര്‍വ്വശക്തവുമത്രേ.
 
ഞാന്‍ പ്രബുദ്ധനായിരിക്കുമ്പോള്‍ അത് ആത്മാവ്. അതിന്‍റെ ഭ്രമകല്‍പ്പനയില്‍ ഞാന്‍ രാജാവ്‌!. . അതിന്‍റെ കൃപയാലാണ് ശരീര മനസ്സുകള്‍ പ്രവര്‍ത്തിക്കുന്നത്. പ്രപഞ്ചത്തിലെ എല്ലാം തുള്ളുന്നത് അതിന്‍റെ താളത്തിനാണ്. പ്രജകള്‍ക്കു നല്‍കുന്ന അനുഗ്രഹശിക്ഷകളെപ്പറ്റി വ്യസനിച്ച ഞാനെത്ര മഠയന്‍! ഞാനുണര്‍ന്നു കഴിഞ്ഞു. കാണേണ്ടതെല്ലാം കണ്ടും കഴിഞ്ഞു. പ്രാപിക്കാനുള്ളതെല്ലാം എനിക്ക് സ്വായത്തവുമായി.  
  
സുഖം, ദു:ഖം, ശിക്ഷ, രക്ഷ, ആഹ്ലാദം, അനുഗ്രഹം, വേദന, എന്നിവയെല്ലാം എന്താണ്? എല്ലാം ബ്രഹ്മം മാത്രമല്ലേ? ദു:ഖത്തിനും ഭ്രമചിന്തകള്‍ക്കും എന്താണൊരു ന്യായീകരണം? അനന്താവബോധമല്ലാതെ യാതൊന്നും സത്യത്തില്‍ ഇല്ലല്ലോ? അല്ലയോ അഭൌമസൌന്ദര്യധാമമായ ദൈവമേ, നിനക്ക് നമസ്കാരം! അനന്താവബോധമേ നിനക്ക് നമോവാകം.