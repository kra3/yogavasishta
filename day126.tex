\newpage
\section{ദിവസം 126}

\slokam{
അനഭ്യസ്തവിവേകം ഹി ദേശകാലവശാനുഗം\\
മന്ത്രൌഷധിവശം യാതി മനോ നോദാരവൃത്തിമത്‌ (3/105/15)\\
}

വസിഷ്ഠന്‍ തുടര്‍ന്നു: രാമ: കുറച്ചുനേരം കഴിഞ്ഞ്‌ രാജാവ്‌ കണ്ണുതുറന്നു. ഭയംകൊണ്ടദ്ദേഹം വിറയ്ക്കാന്‍ തുടങ്ങി. താഴെവീഴാന്‍ തുടങ്ങിയപ്പോള്‍ മന്ത്രിമാര്‍ അദ്ദേഹത്തെ താങ്ങിയെടുത്തു. അവരെയെല്ലാം കണ്ട്‌ വിസ്മയത്തോടെ രാജാവു ചോദിച്ചു: നിങ്ങളൊക്കെ ആരാണ്‌? എന്നെ എന്താണു ചെയ്യുന്നത്‌? വിഷമത്തിലായ മന്ത്രിമാര്‍ പറഞ്ഞു: പ്രഭോ, അങ്ങു ഞങ്ങളുടെ വീര രാജാവല്ലേ? വിജ്ഞാനിയെങ്കിലും അങ്ങയെ ഒരു വിഭ്രമം പിടികൂടി കിഴടക്കിയിരിയിരിക്കുന്നു. അങ്ങയുടെ മനസ്സിനെന്തുപറ്റി? ഭാര്യ, മക്കള്‍ തുടങ്ങി തുലോം ചെറിയ വിഷയങ്ങളില്‍ ആസക്തിയോടെ ആമഗ്നരായവര്‍ മാത്രമേ മോഹവിഭ്രാന്തിക്കടിപ്പെടൂ. അങ്ങയേപ്പൊലെയുള്ള ജ്ഞാനികള്‍ക്ക്‌ അതുണ്ടാവുകവയ്യ. മാത്രമല്ല അങ്ങ്‌ പരമ്പൊരുളില്‍ അതീവഭക്തിയുള്ളവനുമാണ്‌.. "ജ്ഞാനമാര്‍ജ്ജിക്കാത്തവരെ മാത്രമേ മായക്കാഴ്ച്ചകളും ലഹരിമരുന്നുകളും ബാധിക്കുകയുള്ളു. മാനസ്സ്‌ പൂര്‍ണ്ണവികാസം പ്രാപിച്ചവനെ അത്തരം ബാധകള്‍ തീണ്ടുകയില്ല"

ഇതുകേട്ട്‌ രാജാവിനു കുറച്ചു സ്വബോധം തിരികെ കിട്ടി. എന്നാല്‍ ജാലവിദ്യക്കാരനെ നോക്കുമ്പോള്‍ അദ്ദേഹം പിന്നെയും പേടിച്ചു വിറച്ചു. എന്നിട്ടയാളോടു ചോദിച്ചു: മായാജാലക്കാരാ, നീ എന്നോടെന്താണീ ചെയ്തത്‌? നീ എനിക്കുമേല്‍ മോഹവലയമിട്ടിരിക്കുന്നു. വിജ്ഞാനിയെപ്പോലും മായ കീഴടക്കുന്നു. ഞാന്‍ ഈ ശരീരത്തിലിരുന്നുകൊണ്ട് ചെറിയൊരു സമയത്തിനുള്ളില്‍  അത്ഭുതകരമായ  പലതരം മായക്കാഴ്ച്ചകളാണ്‌ അവിടെക്കണ്ടത്‌....

സഭാവാസികള്‍ക്കുനേരേ തിരിഞ്ഞ്‌ അദേഹം തന്റെ അനുഭവങ്ങള്‍ ഇങ്ങിനെ വിവരിച്ചു: ഈ ജാലവിദ്യക്കാരന്‍ തന്റെ മയില്‍പ്പീലി വീശിയനിമിഷം ഞാന്‍ എന്റെ മുന്നില്‍ നിന്നിരുന്ന ആ കുതിരപ്പുറത്തു ചാടിക്കയറി. അപ്പോള്‍ ഞാന്‍ ചെറിയൊരു മനോവിഭ്രാന്തി അനുഭവിച്ചു. പിന്നെ ഞാന്‍ ഒരു നായാട്ടിനുപോയി. ആ കുതിര എന്നെയൊരു വരണ്ട മരുപ്രദേശത്തേക്കാണ്  കൊണ്ടുപോയത് . അവിടെ ജീവജാലങ്ങള്‍ യാതൊന്നുമില്ല. ഒന്നും കിളിര്‍ത്തു വളരുന്നുമില്ല. അവിടെ ജലമില്ല, പക്ഷേ വല്ലാത്ത തണുപ്പായിരുന്നു. ഞാന്‍ അതീവ ദു:ഖിതനായി, ദിവസം മുഴുവനും അവിടെക്കഴിഞ്ഞു. വീണ്ടും അതേ കുതിരപ്പുറമേറി അത്രതന്നെ ഭീകരമല്ലാത്ത മറ്റൊരു മരുഭൂവിലെത്തി. ഞാനൊരു മരക്കീഴില്‍ വിശ്രമിക്കേ കുതിര ഓടിപ്പോയി. കുറച്ച്നേരം കൂടി വിശ്രമിച്ചപ്പോഴേക്കും സൂര്യാസ്തമയമായി. പേടിച്ച്‌ വിറച്ച് ഞാന്‍ ഒരു പൊന്തക്കാട്ടില്‍ ഒളിച്ചു. ആ രാത്രിക്ക്‌ ഒരു യുഗത്തേക്കാള്‍ ഏറെ ദൈര്‍ഘ്യം തോന്നി. നേരം പുലര്‍ന്നു. സൂര്യനുദിച്ചുയര്‍ന്നു. കുറച്ചുകഴിഞ്ഞ്‌ ഇരുണ്ട നിറമുള്ളൊരു പെണ്‍കൊടി കറുത്ത വസ്ത്രവും ധരിച്ച്‌ ഒരു തളികയില്‍ ഭക്ഷണവുമായി വരുന്നതു കണ്ടു. അവളെ സമീപിച്ച്‌ അല്‍പ്പം ഭക്ഷണത്തിനായി ഞാന്‍ യാചിച്ചു. എനിക്കു വിശക്കുന്നുണ്ടായിരുന്നു. അവള്‍ എന്നെ അവഗണിച്ചെങ്കിലും ഞാന്‍ അവളെ പിന്തുടര്‍ന്നു. അവസാനം അവള്‍ പറഞ്ഞു: ഞാന്‍ ഭക്ഷണം തരാം, പക്ഷേ എന്നെ അങ്ങു വിവാഹം ചെയ്യാം എന്നു സമ്മതിച്ചാല്‍മാത്രം. ഞാന്‍ സമ്മതിച്ചു. അവളെനിക്കു ഭക്ഷണം തന്നു. എന്നിട്ടവളുടെ പിതാവിനെ പരിചയപ്പെടുത്തി. അയാള്‍ അവളേക്കാള്‍ ഭീകരരൂപിയായിരുന്നു. ഉടനെ ഞങ്ങള്‍ മൂവരുംകൂടി അവരുടെ ഗ്രാമത്തിലെത്തി. അവിടെ വലിയൊരു സദ്യവട്ടം നടക്കുന്നുണ്ടായിരുന്നു. അതില്‍ ചോരയും മാംസവും നിറഞ്ഞു കവിഞ്ഞു. അവര്‍ എന്നെ അവളുടെ വരനെന്നു സദസ്സിനു  പരിചയപ്പെടുത്തി വലിയ ബഹുമാനത്തോടെ സ്വീകരിച്ചു. എന്നെ സന്തോഷിപ്പിക്കാന്‍ അവര്‍ പറഞ്ഞ ഭീതിജനകമായ കഥകള്‍ എന്നില്‍  വേദനയാണ് ഉണ്ടാക്കിയത്. രാക്ഷസീയമായ ഒരാഘോഷത്തിമിര്‍പ്പോടെ ഞാനാ പെണ്‍ കുട്ടിയെ വിവാഹം ചെയ്തു. 

