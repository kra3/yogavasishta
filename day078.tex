\newpage
\section{ദിവസം 078}

\slokam{
ജീവാ ഇത്യുച്യതേ തസ്യ നാമാണോർവാസനാവത:\\
തത്രൈവസ്തേ സ ച ശവാഗാരേ ഗഗനകേ തഥാ (3/55/6)\\
}

പ്രബുദ്ധയായ ലീലപറഞ്ഞു: ജനനമരണങ്ങളേക്കുറിച്ച്‌ അവിടുന്നു പറഞ്ഞുവന്ന കാര്യങ്ങള്‍ എനിക്ക്‌ അറിവിന്റെ വെളിച്ചമാണ്‌. ദയവായി പ്രഭാഷണം തുടര്‍ന്നാലും. 

സരസ്വതി പറഞ്ഞു: പ്രാണവായുവിന്റെ ഒഴുക്ക്‌ നില്‍ക്കുമ്പോള്‍ വ്യക്തിയുടെ ബോധം തികച്ചും നിഷ്ക്രിയമാവുന്നു. ലീലേ, ആ ബോധം നിര്‍മ്മലവും അനന്തവും ശാശ്വതവുമാണെന്നകാര്യം ഓര്‍മ്മിക്കുക. അത്‌ ഉണ്ടായിമറയുന്ന ഒന്നല്ല. അത്‌ ചരാചരങ്ങളില്‍ , അകാശത്ത്‌, മലകളില്‍ , അഗ്നിയില്‍ , വായുവില്‍ എല്ലാം എപ്പോഴുമുള്ളതത്രേ. പ്രാണന്‍ നിലയ്ക്കുമ്പോള്‍ ആ ശരീരം മരിച്ചു, അല്ലെങ്കില്‍ അതു ജഢമാണ്‌ എന്നു പറയുന്നു. പ്രാണന്‍ അതിന്റെ സ്രോതസ്സിലേയ്ക്ക്‌ - വായുവിലേയ്ക്ക്‌ - മടങ്ങുന്നു. ഓര്‍മ്മകളില്‍നിന്നും വാസനകളില്‍നിന്നും വിടുതല്‍ നേടിയ ബോധം ആത്മാവായി അവശേഷിക്കുന്നു.

"ഓര്‍മ്മകളും വാസനകളും കുടികൊണ്ടിരിക്കുന്ന ആ സൂക്ഷ്മശരീരത്തിന്‌ ജീവന്‍ എന്നു പേര്‌. അത്‌ ശവശരീരത്തിനടുത്തു തന്നെ തങ്ങിനില്‍ക്കുന്നു." അതിനെ 'പ്രേതം' - പിരിഞ്ഞുപോയ ജീവന്‍' എന്നും പറയുന്നു. ജീവന്‍ ഇതുവരെയുണ്ടായിരുന്ന ആശയങ്ങളും കാഴ്ച്ചകളുമുപേക്ഷിച്ച്‌ സ്വപ്നത്തിലെയും ദിവാസ്വപ്നത്തിലേയും ദൃശ്യങ്ങള്‍ എന്നപോലെ പുതിയ ധാരണകളില്‍ എത്തുന്നു. ചെറിയൊരു ബോധവിസ്മൃതിക്കുശേഷം ജീവന്‍ മറ്റൊരു ശരീരത്തെ, മറ്റൊരു ലോകത്ത്‌, മറ്റൊരായുസ്സില്‍ സങ്കല്‍പ്പിച്ചു തുടങ്ങുന്നു. ലീലേ, ആറുതരത്തിലാണ്‌ പിരിഞ്ഞുപോയ ഈ ജീവനുകള്‍ വിഭജനം ചെയ്തിട്ടുള്ളത്‌. നിന്ദ്യം, അതിനിന്ദ്യം, അതീവ നിന്ദ്യം (കൊടും പാപികള്‍ ); ഉത്തമം, അത്യുത്തമം, മഹത്തുക്കള്‍ എന്നിങ്ങനെയാണീ ആറു തരക്കാര്‍ . ഇവയിലും ഉപ വിഭാഗങ്ങളുണ്ട്‌. ചില മഹാ പാപികളുടെ കാര്യത്തില്‍ മുന്‍പു പറഞ്ഞ ബോധവിസ്മൃതി ഏറെക്കാലത്തേയ്ക്ക്‌ നീണ്ടുനിന്നേക്കാം. അവര്‍ പുഴുക്കളായും മൃഗങ്ങളായും പുനര്‍ജനിക്കുന്നു. ചെറുപാപികള്‍ വീണ്ടും മനുഷ്യജന്മമെടുക്കുന്നു. ധര്‍മ്മിഷ്ഠരില്‍ ഉത്തമരായവര്‍ സ്വര്‍ഗ്ഗത്തിലേയ്ക്കുയര്‍ന്ന് അവിടെ ജീവിതമാസ്വദിക്കുന്നു. പിന്നീടവര്‍ ഭൂമിയിലേയ്ക്കു തിരികെവന്ന് ധനികരും സദ്ഗുണസമ്പന്നരുമായവരുടെയിടയില്‍ ജനിക്കുന്നു. മദ്ധ്യവര്‍ത്തികളായ ധര്‍മ്മിഷ്ഠര്‍ ഗന്ധര്‍വ്വലോകത്തുപോയി തിരികെ ഭൂമിയിലെത്തി ബ്രാഹ്മണകുടുംബങ്ങളില്‍ ജനിക്കുന്നു.മരിച്ചവരില്‍ ധര്‍മ്മിഷ്ഠര്‍ക്കുപോലും തങ്ങളുടെ ജീവിതകാലത്ത്‌ ചെയ്തുപോയ അനീതിയുടെ പരിണിതഫലം അനുഭവിക്കേണ്ടിവരുന്നു. അതിനായി അവര്‍ ഉപദേവതമാരുടെ സവിധങ്ങളില്‍ക്കൂടി കടന്നു പോകുന്നു.

