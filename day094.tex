\newpage
\section{ദിവസം 094}

\slokam{
വിനാ പരാപകാരേണ തീക്ഷ്ണാ മരണമീഹതേ\\
വേദനാദ്രോധിതാ സൂചീ കര്‍മപാശോ പ്രലംബതേ (3/70/66)\\
}

വസിഷ്ഠന്‍ തുടര്‍ന്നു: നിലത്തുള്ള ചെളിയിലും പൊടിയിലും, വൃത്തിഹീനമായ വിരലുകളില്‍ , തുണികളിലെ നൂലിഴകളില്‍ , ദേഹത്തിലെ പേശികള്‍ക്കുള്ളില്‍ , പൊടികൊണ്ടുമൂടിയ വൃത്തിഹീനമായ തൊലിപ്പുറത്ത്‌, കൈകള്‍ വിണ്ടുകീറിയ ഇടങ്ങളില്‍ , ജരബാധിച്ച ശരീരഭാഗങ്ങളില്‍ , ഈച്ചയാര്‍ക്കുമിടങ്ങളില്‍ , ശവശരീരങ്ങളില്‍ , ജീര്‍ണ്ണിച്ച ഇലകള്‍ കൂട്ടിയിട്ടയിടങ്ങളില്‍ , നല്ല മരങ്ങളില്ലാത്തയിടങ്ങളില്‍ , വൃത്തിയില്ലത്ത വസ്ത്രം ധരിക്കുന്നവരില്‍ , ആരോഗ്യശീലം ഇല്ലാത്തവരില്‍ , കാടുവെട്ടിത്തെളിച്ച്‌ ബാക്കിയായ മരകുറ്റികളില്‍ കെട്ടിനില്‍ക്കുന്ന വെള്ളത്തില്‍ കൂത്താടികള്‍പെരുകി കൃമികീടങ്ങളുണ്ടായതില്‍ , കെട്ടിക്കിടക്കുന്ന വെള്ളത്തില്‍ , മലിനജലത്തില്‍ , നഗരവീഥിയ്ക്കു നടുവിലെ തുറന്ന ഓടകളില്‍ , വഴിയാത്രക്കാര്‍ തങ്ങുന്ന സത്രങ്ങളില്‍ , ആനയും കുതിരയും ഏറെയുള്ള നഗരങ്ങളില്‍ എല്ലാം സൂചിക ഒളിച്ചുതാമസിക്കുന്നു. 


സ്വയം അവളൊരു സൂചി -തുന്നാനുള്ളത്‌))-- ---ആയതിനാല്‍ പാതയോരത്തുകിടക്കുന്ന അഴുക്കുതുണികള്‍ കൂട്ടിത്തുന്നി അവളണിയുന്നു. രോഗികളുടെ ദേഹത്തിലവളോടിക്കളിക്കുകയാണ്‌..  തുന്നല്‍ക്കാരന്റെ സൂചിയും കഠിനമായ , തുടര്‍ച്ചയായ അദ്ധ്വാനംകാരണം തളര്‍ച്ച അനുഭവപ്പെട്ട്‌ നിലത്തുവിഴാം.  അതുപോലെ സൂചികയ്ക്കും ഈ നശീകരണം മടുത്തു. സൂചിയുടെ സ്വഭാവം കുത്തിത്തുളയ്ക്കലാണെന്നതുപോലെ സൂചികയുടെ സ്വഭാവം ക്രൂരതയാണ്‌..  തന്നിലൂടെ കടന്നുപോവുന്ന നൂലിനെ സൂചി വിഴുങ്ങുന്നതുപോലെ സൂചിക അവളുടെ ഇരകളെ നശിപ്പിച്ചുകൊണ്ടേയിരുന്നു. അതിക്രൂരരായവര്‍ പോലും ചിലപ്പോള്‍ മറ്റുള്ളവരുടെ പട്ടിണിയും ദയനീയാവസ്ഥയും ഏറെനാള്‍ കണ്ടുകണ്ട്‌ ദയയുള്ളവരാകുന്നതായി കാണാറുണ്ട്‌. സൂചികയും എണ്ണമറ്റ നൂലുകള്‍ തന്റെ വസ്ത്രത്തില്‍ തുന്നിച്ചേര്‍ത്തിട്ടുണ്ട്‌ (അവളുടെ കര്‍മ്മ സഞ്ചയം). അതവളെ അലോസരപ്പെടുത്തി. അവളുടെ മുഖം അവള്‍തന്നെ തുന്നിയ ഒരു കറുത്തതുണികൊണ്ട്‌ മൂടപ്പെട്ടതായി അവള്‍ക്കു തോന്നി. അവളുടെ കണ്ണൂകള്‍ കെട്ടിയിരുന്നു. 'ഈ കണ്‍കെട്ട്‌ ഞാനെങ്ങിനെ കീറിക്കളയും?' എന്നവള്‍ ആലോചിച്ചു. അവള്‍ സൂചിയെന്നനിലയില്‍  ശുഭ്ര വസ്ത്രത്തിലും (സദ്ജനങ്ങള്‍ ) ജീര്‍ണ്ണവസ്ത്രത്തിലും (ദുര്‍ജ്ജനങ്ങള്‍ ) യഥേഷ്ടം കയറിയിറങ്ങി. വിഡ്ഢികളോ ദുഷ്ടന്മാരോ തങ്ങള്‍ ചെയ്യുന്ന കാര്യങ്ങളെ വിവേചനബുദ്ധിയോടെ സമീപിക്കാറില്ലല്ലോ. 

"ആരുടേയും ശല്യമോ പ്രകോപനമോ കൂടാതെ സൂചിക മറ്റുള്ളവരുടെ നാശത്തിനും മരണത്തിനുമായി പ്രവര്‍ത്തിക്കുന്നു. ആപത്കാരിയായ ഈ നൂലിന്റെ ബന്ധനംകൊണ്ട്‌ അതും തൂക്കിയിട്ട്‌ അവള്‍ അലയുന്നു"

ജീവസൂചിക എന്നും അവള്‍ അറിയപ്പെടുന്നു. എല്ലാ ജീവജാലങ്ങളിലും പ്രാണന്റേയും അപാനന്റേയും സഹായത്താല്‍ ജീവശക്തിയായി അവള്‍ ജീവനെ ദുരിതപ്പെടുത്തുന്നു. തീവ്രവും സൂചി കുത്തുന്നതുപോലെയുമുള്ള വേദനകൊണ്ട്‌ (വാതം, രക്തവാതം മുതലായവ) ഒരുവന്റെ മനസ്സു കെടുത്തുന്നു. കാലുകളില്‍ (സൂചിപോലെ) തുളഞ്ഞുകയറി രക്തം കുടിക്കുന്നു. എല്ലാ ദുഷ്ടരേയും പോലെ അന്യരുടെ ദുരിതത്തില്‍ അവള്‍ സന്തോഷിക്കുന്നു. 

വസിഷ്ഠന്‍ ഇത്രയും പറഞ്ഞപ്പോഴേയ്ക്ക്‌ മറ്റൊരു ദിനം കൂടി കഴിഞ്ഞു. സന്ധ്യാ വന്ദനത്തിനുള്ള സമയമായി. സഭ അടുത്തദിവസം കൂടുവാനായി പിരിഞ്ഞു. 

