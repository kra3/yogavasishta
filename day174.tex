\section{ദിവസം 174}

\slokam{
സമ്യഗാലോകനാത്സത്യാദ്വാസനാ പ്രവിലീയതേ\\
വാസനാവിലയേ ചേത: ശമമായാതി ദീപവത് (4/34/28)\\
}

വസിഷ്ഠൻ തുടർന്നു: അല്ലയോ രാമ: ശംഭരനെ ദാമൻ, വ്യാളൻ, കടൻ എന്നീ രാക്ഷസന്മാർ ഉപേക്ഷിച്ചുപൊയ്ക്കഴിഞ്ഞപ്പോൾ അവർ അഹംഭാവത്തിനടിമപ്പെട്ടുപോയ കാര്യം അയാൾ മനസ്സിലാക്കി. അതുകൊണ്ടയാൾ കൂടുതൽ രാക്ഷസന്മാരെ ഉണ്ടാക്കാൻ തീരുമാനിച്ചു. പക്ഷേ ഇത്തവണ അവർക്ക് ആത്മജ്ഞാനവും വിവേകവും നൽകി. കാരണം ദാമാദികളെപ്പോലെ ‘അഹം ഭാവമെന്ന’ കെണിയിൽ അവർ വീണുപോകരുതല്ലോ. ശംഭരൻ തന്റെ മായാശക്തികൊണ്ട് ഭീമൻ, ഭാസൻ, ദൃഢൻ എന്നിങ്ങനെ മൂന്നുപേരെ സൃഷ്ടിച്ചു. അവർ സർവ്വശക്തരും ആത്മജ്ഞാനനിരതരും, നിർമമതരും പാപരഹിതനും ആയിരുന്നു. അവർ വിശ്വത്തെമുഴുവൻ തൃണസമാനമായി കരുതി. അവർ ദേവസേനകളുമായി യുദ്ധം തുടങ്ങി. ഏറെക്കാലം പൊരുതിയിട്ടും അവരെ അഹംഭാവം തീണ്ടിയില്ല. ഏപ്പോഴെല്ലാം അഹംഭാവം തലയുയർത്താൻ തുടങ്ങിയോ അപ്പോഴെല്ലാം അവർ ‘ഞാനാരാണ്‌’ എന്ന അന്വേഷണത്തിലൂടെ അതിനെ മറികടന്നു. അവർക്കതിനാൽ മരണഭയമുണ്ടായിരുന്നില്ല. അപ്പപ്പോള്‍ തങ്ങളുടെ മുന്നിലുള്ള കർമ്മമെന്തോ അത് യാതൊരുവിധ ആസക്തികളും കൂടാതെ അവർ ചെയ്തുവന്നു. ‘ഞാനാണ്‌ ഇതു ചെയ്തത്’ എന്നവർക്കു തോന്നിയതേയില്ല. തങ്ങളുടെ നാഥനായ ശംഭരൻ ഇച്ഛിച്ച പണികളെല്ലാം യാതൊരാഗ്രഹങ്ങളും ഇല്ലാതെ, സമബുദ്ധിയോടെ അവർ ചെയ്തുവന്നു.

ദേവസൈന്യം പെട്ടെന്നുതന്നെ പരജയപ്പെട്ടു. അവർ ഭഗവാൻ വിഷ്ണുവിനെ ശരണം പ്രാപിച്ചു. അദ്ദേഹത്തിന്റെ ആജ്ഞപ്രകാരം അവർ പുതിയൊരിടംകണ്ടെത്തി അവിടെ വാസമുറപ്പിച്ചു. അതുകഴിഞ്ഞ് ഭഗവാൻ സ്വയം ശംഭരനുമായി യുദ്ധംചെയ്ത് അവനെ വധിച്ച് വിഷ്ണുപദം പൂകിച്ചു. ഭഗവാൻ സ്വയം ഭീമനേയും ഭാസനേയും ദൃഢനേയും വധിച്ചു. ദേഹം വീണമാത്രയിൽ അവർ പ്രബുദ്ധരായി. കാരണം അവരിൽ അഹംഭാവം ഉണ്ടായിരുന്നില്ലല്ലോ.

രാമ, ഉപാധികളിൽപ്പെട്ട മനസ്സുതന്നെയാണ്‌ ബന്ധനം. “മനസ്സ് ഉപാധികളിൽ നിന്നു മോചനമാവുമ്പോൾ മുക്തിയായി. സത്യം സാക്ഷാത്കരിക്കുമ്പോൾ, അല്ലെങ്കിൽ വ്യക്തമായി അറിയുമ്പോൾ മനോപാധികൾ കൊഴിഞ്ഞു പോവുന്നു. ഉപാധികളില്ലാത്ത മനസ്സുമൂലം ബോധമണ്ഡലം അതീവനിർമ്മലമാവുന്നു. ജ്വാലയണച്ച വിളക്കുപോലെ പ്രശാന്തമാവുന്നു. ”

ആത്മാവാണ്‌ സകലതും - ഒരുവന്‌ ആലോചിക്കാവുന്നതെല്ലാം അത്മാവുതന്നെയെന്ന് അറിയുന്നതാണ്‌ പ്രബുദ്ധത. ഉപാധി, മനസ്സ്, എന്നൊക്കെയുള്ള വാക്കുകൾക്ക് സത്യത്തിൽ അർത്ഥമൊന്നുമില്ല. കാരണം സത്യാന്വേഷണവേളയിൽ ഈ വാക്കുകളുടെ വ്യർത്ഥത വെളിവാകുന്നു. അതാണ്‌ പ്രബുദ്ധത. ഈ പ്രബുദ്ധതയുടെ ഉദയം തന്നെയാണ്‌ മുക്തി.

ദാമൻ, വ്യാളൻ, കടൻ എന്നിവർ അഹംഭാവമെന്ന ഉപാധിയാൽ പരിമിതപ്പെട്ട മനസ്സിനെ പ്രതിനിധാനം ചെയ്യുന്നു. ഭീമൻ, ഭാസൻ, ദൃഢൻ എന്നിവർ ഉപാധികളില്ലാത്ത മനസ്സിനെ പ്രതിനിധാനം ചെയ്യുന്നു. രാമാ, നാമെല്ലാം ആദ്യത്തെക്കൂട്ടരെപ്പോലെയാകാതെ ഭീമാദികളെപ്പോലെയാവണം. അതിനാലാണ്‌ ഇക്കഥ അതിബുധിമാനും എനിക്കുപ്രിയപ്പെട്ടവനുമായ നിനക്കു ഞാൻ പറഞ്ഞു തന്നത്. 
