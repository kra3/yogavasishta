\section{ദിവസം 263}

\slokam{
മനോരാജ്യമപി പ്രാജ്ഞാ ലഭന്തേ വ്യവസായിന:\\
ഗാധിനാ സ്വപ്നസംദൃഷ്ടം ഗത്വാ ലബ്ധമഖണ്ഡിതം  (5/47/37)\\
}

വസിഷ്ഠന്‍ തുടര്‍ന്നു: അതിനുശേഷം ഗാധി തന്റെ ഭ്രമാത്മകദര്‍ശനത്തില്‍ നിന്നും സ്വതന്ത്രനായി ഉണര്‍ന്നു. ബോധമുണര്‍ന്ന അദ്ദേഹം ‘ഞാന്‍ ഗാധിയാണ്’ എന്ന് തിരിച്ചറിഞ്ഞു. തന്റെ അനുഷ്ഠാനങ്ങള്‍ ചെയ്തുതീര്‍ത്ത് ഗാധി നദിയില്‍ നിന്നും കരയ്ക്ക്‌ കയറി. അദ്ദേഹം ‘ഞാന്‍ ആരാണ്?, എന്താണ് ഞാന്‍ കണ്ടത്? എങ്ങിനെയാണാ ദര്‍ശനം എന്നിലുണ്ടായത്?’ എന്നെല്ലാം ചിന്തിക്കാന്‍ തുടങ്ങി. ക്ഷീണിച്ച് അവശമായിരുന്നതു കൊണ്ടാണ് തന്റെ മനസ്സ് ഇങ്ങിനെയൊരു വിസ്മയവിദ്യ പ്രകടിപ്പിച്ചതെന്ന് അദ്ദേഹം വിചാരിച്ചുറച്ചു.

അവിടെനിന്നും തിരികെ നടക്കുമ്പോഴും താന്‍ കണ്ട ദര്‍ശനത്തെപ്പറ്റിയും അതില്‍ തന്റെ ബന്ധു മിത്രാദികളായിരുന്നവരെപ്പറ്റിയും അതിലെ മാതാപിതാക്കളെപ്പറ്റിയും അദ്ദേഹം ആലോചിച്ചുകൊണ്ടിരുന്നു.

അദ്ദേഹം ആലോചിച്ചു: തീര്‍ച്ചയായും ഞാന്‍ കണ്ടതെല്ലാം വെറും ഭ്രമദൃശ്യങ്ങള്‍ മാത്രം. കാരണം ഇപ്പോള്‍ എനിക്കൊന്നിനെപ്പറ്റിയും ആശയസങ്കല്‍പ്പങ്ങളും ധാരണകളും ഇല്ലല്ലോ. കുറച്ചു ദിവസങ്ങള്‍ക്ക് ശേഷം മറ്റൊരു ബ്രാഹ്മണന്‍ അദ്ദേഹത്തെ സന്ദര്‍ശിച്ചു. ഗാധി അതിഥിയെ സല്‍ക്കരിച്ച് ഉപചാരങ്ങള്‍ നല്‍കി. അദ്ദേഹം അതിഥിയോട് ചോദിച്ചു: അങ്ങെന്താണ് പരിക്ഷീണനായിരിക്കുന്നത്? എന്തുപറ്റി? അതിഥി പറഞ്ഞു: ഞാന്‍ ഉള്ളത് പറയാം. അങ്ങ് വടക്ക് കീര എന്നൊരു രാജ്യമുണ്ട്. അവിടെ ഞാനൊരുമാസം സുഖഭോഗങ്ങളോടെ നഗരവാസികളുടെ അതിഥിയായി കഴിഞ്ഞു. അപ്പോള്‍ ഞാനൊരു കഥ കേട്ടു. അതി വിചിത്രമായ ഒരു കഥ.

ഒരു ചണ്ഡാളനായ കാട്ടുജാതിക്കാരന്‍ എട്ടുകൊല്ലം ആ രാജ്യം ഭരിച്ചു എന്നും എന്നാല്‍ അവസാനം അയാളുടെ തനിനിറം അവര്‍ക്ക് ബോധ്യപ്പെട്ടു എന്നും അവര്‍ പറഞ്ഞു. അയാളുടെ ഭരണമേല്‍പ്പിച്ച അപമാനത്താല്‍ അനേകം ബ്രാഹ്മണര്‍ ആത്മാഹുതി ചെയ്തുവത്രേ.

അതുകേട്ടിട്ട് എനിക്കും സ്വയം ആകെ അശുദ്ധി വന്നതുപോലെയൊരു തോന്നല്‍ . അതുകൊണ്ട് ഞാനുടനെ പ്രയാഗ എന്ന പുണ്യസ്ഥലത്ത് പോയി കഠിനമായ തപസ്സിലേര്‍പ്പെട്ടു. കൂടുതല്‍ കാലം ഉപവസിച്ചതിനാലാണു ഞാനിങ്ങിനെ ക്ഷീണിതനായിരിക്കുന്നത്. ഞാനിന്നാണാവ്രതം അവസാനിപ്പിക്കുന്നത്. അന്ന് രാത്രി അവിടെ ഗാധിയുടെ അതിഥിയായിക്കഴിഞ്ഞശേഷം ബ്രാഹ്മണന്‍ അവിടം വിട്ടുപോയി.

ഗാധി വീണ്ടുമിങ്ങിനെ ആലോചിച്ചു. ഞാനെന്റെ ഭ്രമകല്‍പ്പനയില്‍ കണ്ട ദൃശ്യം എന്റെ അതിഥിയ്ക്ക് യാഥാര്‍ത്ഥ്യമായിരുന്നു! എന്നാലീ കഥയുടെ പുറകിലെ കാര്യമെന്തെന്ന് സ്വയം കണ്ടുപിടിച്ചിട്ട് തന്നെ കാര്യം!. അങ്ങിനെ നിശ്ചയിച്ചുറച്ച് ഗാധി ഭൂതമണ്ഡലമെന്ന പ്രദേശത്തേയ്ക്ക് പോയി. “ഗാധി തന്റെ ഭ്രമകല്‍പ്പനയില്‍ കണ്ടതെല്ലാം അവിടെയും കണ്ടു. ഉയര്‍ന്നു വികസിച്ച അവബോധമുള്ളവര്‍ക്ക് സ്വപ്രയത്നഫലമായി മനസാ സങ്കല്‍പ്പിച്ചതെല്ലാം നടപ്പിലാക്കാന്‍ കഴിയും.” അവിടെ അദ്ദേഹം താമസിച്ചിരുന്ന ഗ്രാമം കണ്ടു. അതദ്ദേഹത്തിന്റെ ബോധതലത്തില്‍ ആഴത്തില്‍ പതിഞ്ഞുകിടന്നിരുന്നുവല്ലോ. അവിടെ തന്റെ കുടിലില്‍ (നായാട്ടുകാരന്റെ കുടിലില്‍ താനുപയോഗിച്ചിരുന്ന സാമഗ്രികളെല്ലാം അങ്ങിനെതന്നെയുണ്ടായിരുന്നു. എന്നാല്‍ വീട് വളരെ മോശം അവസ്ഥയിലായിരുന്നു. 

കുടുംബാംഗങ്ങള്‍ മാംസം തിന്നിട്ടവശേഷിപ്പിച്ച നിരവധി മൃഗങ്ങളുടെ അസ്ഥികൂടങ്ങള്‍ അവിടെക്കണ്ടു. ഒരു ശ്മശാനംപോലെ തോന്നിച്ച ആ സ്ഥലം അദ്ദേഹം വിസ്മയത്തോടെ നോക്കി നിന്നു. അവിടെനിന്നും മറ്റൊരു ഗ്രാമത്തില്‍പ്പോയി അദ്ദേഹം ആ നാട്ടുകാരോടിങ്ങിനെ ചോദിച്ചു: 'അവിടെ ഭൂതമണ്ഡല ഗ്രാമത്തിലെ ഇന്ന വീട്ടില്‍ താമസിച്ചിരുന്ന ആ നായാട്ടുകാരനെപ്പറ്റി നിങ്ങള്‍ക്കെന്തെങ്കിലും അറിയാമോ?' അവര്‍ പറഞ്ഞു: മഹാത്മന്‍, തീര്‍ച്ചയായും ഞങ്ങള്‍ക്കറിയാം. അവിടെ ഭീകരനായ ഒരു ചണ്ഡാളനാണ് വസിച്ചിരുന്നത്. അയാള്‍ ഏറെ വയസ്സാവുന്നതുവരെ ജീവിച്ചിരുന്നു. അയാളുടെ ബന്ധുക്കളെല്ലാം മരിച്ചു കഴിഞ്ഞപ്പോള്‍ അദ്ദേഹം കീര എന്നൊരു രാജ്യത്ത് ചെന്ന് അവിടുത്തെ രാജാവായി എട്ടുകൊല്ലം ഭരിച്ചു. അദ്ദേഹം കാരണം അനേകം പേര്‍ മരണപ്പെട്ടതറിഞ്ഞു അദ്ദേഹവും സ്വയം മരണത്തിനു വിട്ടുകൊടുത്തു. അങ്ങെന്താണ് അയാളെപ്പറ്റി ചോദിച്ചത്? അങ്ങയുടെ ബന്ധുവായിരുന്നോ അയാള്‍ ? അല്ലെങ്കില്‍ അയാളുമായി ബന്ധമുള്ള ആരെയെങ്കിലും അങ്ങേയ്ക്കറിയാമോ?

ഇതുകേട്ട് ഗാധി വീണ്ടും വിസ്മയചകിതനായി.