\section{ദിവസം 215}

\slokam{
യഥാ ശിലമയീ കന്യാ ചോദിതാപി ന നൃത്യതി\\
തഥേയം കലനാ ദേഹേ ന കിഞ്ചിദവബുദ്ധ്യതേ (5/13/65)\\
}

വസിഷ്ഠൻ തുടർന്നു: ഞാനിപ്പറഞ്ഞ ആന്തരീകമായ മേധാശക്തി ഉണർന്നില്ലെങ്കിൽ ജീവനില്‍ യാതൊരറിവും ഉണ്ടാകയില്ല. വെറും  ചിന്തകളിലൂടെയുണ്ടാവുന്ന അറിവ് സത്തായ അറിവല്ലല്ലോ. ചിന്തകൾക്ക് മൂല്യമുണ്ടാവുന്നത് ബോധത്തിന്റെ വെളിച്ചത്തിലാണ്‌.. സുഗന്ധദ്രവ്യം വെച്ച പാത്രത്തിനുണ്ടാവുന്ന സൗരഭ്യത്തിനു കാരണം അതിൽ വെച്ചിട്ടുള്ള വസ്തുവിന്റെ പ്രാഭവമാണല്ലോ. ഇങ്ങിനെ കടം വാങ്ങിയ ബുദ്ധികൊണ്ടാണ്‌ ചിന്തകൾക്ക് അനന്താവബോധത്തിന്റെ ഒരു ചെറുകണികയോളമെങ്കിലും  അറിയാനിടവരുന്നത്. അനന്തമായ പരമപ്രകാശത്തിന്റെ പ്രഭയിൽ മാത്രമേ മനസ്സ് പൂർണ്ണമായും വികാസം പ്രാപിച്ച് പൂത്തുവിടരുകയുള്ളു.

“കല്ലിൽ കൊത്തിവച്ച നർത്തകീ ശിൽപ്പത്തിനോട് നൃത്തമാടാൻ ആവശ്യപ്പെട്ടാൽ അതു സാധിക്കാത്തതുപോലെ ബുദ്ധിശക്തിയുണ്ടെന്നു തോന്നിക്കുമെങ്കിലും ചിന്തകൾക്ക് സ്വയം ഒന്നും അറിയാൻ കഴിയുകയില്ല.” അതിന്‌ ബോധത്തിന്റെ വെളിച്ചം കൂടിയേ തീരൂ. ഛായാപടത്തിലെ പോർക്കളത്തിന്‌ ശരിയായ പോരാട്ടത്തിലുണ്ടാകുന്നപോലെ ഹുങ്കാരമുണ്ടാക്കാനാവുമോ? ഒരു ശവശരീരത്തിന്‌ എഴുന്നേറ്റോടാനാവുന്നതെങ്ങിനെ? ശ്രദ്ധാപൂർവ്വം കല്ലിൽ കൊത്തിവച്ച സൂര്യന്റെ ചിത്രത്തിന്‌ ഇരുട്ടിനെ അകറ്റുവാനാകുമോ? അതുപോലെ ജഢവസ്തുവായ മനസ്സിന്‌ സ്വന്തമായി എന്തുചെയ്യാൻ കഴിയും?

മരുഭൂമിയിലെ മരുപ്പച്ചയ്ക്കുപോലും ജലസദൃശതയുണ്ടാവുന്നത് സൂര്യരശ്മിയുടെ സാന്നിദ്ധ്യംകൊണ്ടു മാത്രമാണ്‌.. അതുപോലെ മനസ്സ് ബുദ്ധിമത്താണെന്നും കർമ്മനിരതമാണെന്നും തോന്നുന്നത് ആന്തരീകമായി ബോധത്തിന്റെ വെളിച്ചം  പരിലസിക്കുന്നതുകൊണ്ടാണ്‌.. അജ്ഞാനികൾ ജീവന്റെ സഞ്ചാരം മനസ്സാണെന്നു തെറ്റിദ്ധരിക്കുന്നു. പക്ഷെ അത് ജീവൻ- പ്രാണശക്തി മാത്രമാണ്‌.. എന്നാൽ ഏതൊരുവന്റെ ബുദ്ധിശക്തി ഛിന്നഭിന്നമാക്കപ്പെടാതെയും ചിന്തകളാൽ പരിമിതപ്പെടാതെയുമിരിക്കുന്നുവോ അവനിൽ പ്രസ്ഫുരിക്കുന്നത് പരമാത്മാവു തന്നെയാണ്‌..

വ്യക്തികൾ വെച്ചുപുലർത്തുന്ന ധാരണകളായ ‘ഇതു ഞാൻ’, ‘ഇതെന്റേത്’ എന്നിവ ജീവശക്തിയുടെ ചിന്താ സഞ്ചാരങ്ങളാണ്‌. ബുദ്ധി അവയുമായി താദാത്മ്യം പ്രാപിക്കുമ്പോൾ അത് ജീവൻ, ജീവാത്മാവ് എന്നെല്ലാം അറിയപ്പെടുന്നു. ബുദ്ധി, മനസ്സ്, ജീവൻ തുടങ്ങിയവ വെറും നാമങ്ങൾ മാത്രമാണ്‌.. ജ്ഞാനികൾപോലും അവ ഉപയോഗിക്കുന്നുണ്ടെങ്കിലും വാക്കുകള്‍ ആത്യന്തികമായ സത്യമല്ലെന്നറിയുക. വെറും ആപേക്ഷികതലത്തിലേ അവയ്ക്ക് സാംഗത്യമുള്ളു. സത്യത്തിൽ മനസ്സില്ല, ബുദ്ധിയില്ല, ശരീരമെടുത്ത ജീവികളും ഇല്ല. ആത്മാവു മാത്രമേ എപ്പോഴും എല്ലായിടത്തും ഉണ്മയായുള്ളു.

ആത്മാവുതന്നെയാണ്‌ ലോകം, ആത്മാവാണു കാലം. ആത്മാവാണ്‌ പരിണാമപ്രക്രിയയും. അതീവസൂക്ഷ്മമായി നിലകൊള്ളുന്നതിനാൽ അതിന്റെ അസ്തിത്വം തന്നെ അത്ര സുവിദിതമല്ല എന്ന് മാത്രം. പ്രതിഫലനമോ പ്രകടനമോ മാത്രമായി പ്രത്യക്ഷമാകുമ്പോഴും അതു സത്യമായി സാക്ഷാത്കരിക്കപ്പെടുന്നുണ്ട്. സത്യം എന്നത് എല്ലാ വിശദീകരണങ്ങൾക്കുമതീതമാണ്‌.. ആത്മവിദ്യയുടെ നേരറിവിലൂടെ മാത്രമേ അതനുഭവേദ്യമാവൂ. ഉൾവെളിച്ചം പ്രകാശിക്കുമ്പോൾ മനസ്സില്ല. വെളിച്ചമുള്ളപ്പോൾ ഇരുട്ടിനെവിടെയാണിടം?

ഇന്ദ്രിയവസ്തുക്കളെ അനുഭവിക്കുന്നതിനായിട്ട് ബോധത്തെ വിഷയമായിക്കാണുമ്പോൾ ആത്മാവിനെ പാടേ മറന്നുപോയിട്ടെന്നപോലെ അവിടെ മനോസൃഷ്ടികളെക്കുറിച്ചുള്ള ചിന്തകളുയരുകയായി. 
