 
\section{ദിവസം 003}

വാല്‍മീകി തുടര്‍ന്നു: ``ആകാശത്തിന്റെ നീലിമ ഒരു ദൃശ്യ സംഭ്രമം
മാത്രമാണെന്നതുപോലെ ഈ കാണപ്പെടുന്ന ലോകം ആകെ ചിന്താക്കുഴപ്പം പിടിച്ചതാണ്‌.
ആയതിനാല്‍ എനിക്കു തോന്നുന്നത്‌ ഈ ലോകത്തിനെപ്പറ്റി അധികം വിചിന്തനം
ചെയ്ത്‌ മനസ്സു ഭ്രമിപ്പിക്കുന്നതിനു പകരം അതിനെ നമ്മുടെ
ചിന്തയില്‍പ്പെടുത്താതിരിക്കുകയാണു നല്ലത്‌ എന്നാണ്‌.''~

ഇക്കാണപ്പെടുന്ന പ്രപഞ്ചം യാഥാര്‍ത്ഥ്യമല്ലെന്ന ഉറച്ച ബോധം ഉള്ളില്‍
അങ്കുരിച്ചു വളര്‍ന്നുനിറഞ്ഞാലല്ലാതെ നമുക്ക്‌ ദു:ഖനിവൃത്തിയോ ഉണ്മയുടെ
സാക്ഷാത്കാരമോ സാദ്ധ്യമല്ല. യോഗവാസിഷ്ഠം ~അതീവ ശ്രദ്ധയോടെ പഠിച്ചാല്‍ ഈ
ബോധം നമ്മില്‍ വളര്‍ത്തിയുറപ്പിക്കാന്‍ സാധിക്കും. അങ്ങിനെ നാം
അനുഭവിക്കുന്ന ഈ ദൃശ്യ-വസ്തു പ്രപഞ്ചം എന്നത്‌ ഉണ്മയില്‍ അസ്ഥിരമായ
വസ്തുവിനെ കണ്ടുവെന്നു തോന്നുന്ന വെറും വിഭ്രാന്തി മാത്രമാണെന്നുള്ള
ദൃഢബോധം നമ്മിലുണ്ടാവും. ഈ ഗ്രന്ഥം പഠിക്കാത്ത ഒരുവന്റെയുള്ളില്‍ അനേകലക്ഷം
വര്‍ഷങ്ങള്‍ കഴിഞ്ഞാലും ഈ ജ്ഞാനം ഉണരുകയില്ല.

മോക്ഷം എന്നത്‌ നമ്മിലുള്ള എല്ലാ 'വാസന'കളേയും സമൂലം നീക്കം ചെയ്യുക
എന്നതാണ്‌. ഈ വാസനകളാകട്ടെ രണ്ടു തരത്തിലാണ്‌. ഒന്ന് പരിശുദ്ധവും
പുണ്യപ്രദവും ആകുമ്പോള്‍ മറ്റേത്‌ അശുദ്ധവും പാപജന്യവും ആണ്‌. പാപ വാസനകള്‍
ജനനഹേതുവാകുമ്പോള്‍ പുണ്യവാസനകള്‍ മോക്ഷഹേതുവാണ്‌. അശുദ്ധവാസനകള്‍
അജ്ഞാനത്തേയും ~അഹങ്കാരത്തേയും വളര്‍ത്തി പുനര്‍ജന്മങ്ങള്‍ക്കു
ബീജമായിത്തീരുന്നു. ഈ വാസനാ ബീജങ്ങളെ തീര്‍ത്തും ഉപേക്ഷിച്ചാല്‍ ശരീരത്തെ
നിലനിര്‍ത്തുവാന്‍ വേണ്ട പരിശുദ്ധ മനോപാധികള്‍ മാത്രം ബാക്കിയാവുന്നു.
ജീവന്‍ മുക്തരായവരില്‍ പോലും ഇത്തരം വാസനകള്‍ നിലനില്‍ക്കുന്നുണ്ട്‌. ഇവ
പുനര്‍ജന്മത്തിനു കാരണമാകുന്നില്ല. എന്തെന്നാല്‍, അവ നിലനില്‍ ക്കുന്നത്‌
ഭൂതകാലകര്‍മ്മങ്ങളുടെ സംവേഗശക്തിയാലാണ്‌- ഇപ്പോഴത്തെ കര്‍മ്മങ്ങളുടെ
പ്രേരണയാല്‍ അല്ല എന്നര്‍ത്ഥം. ശ്രീരാമന്‍ ഒരു മുക്തതാപസനേപ്പോലെ
പ്രബുദ്ധമായ ജീവിതം നയിച്ച കഥ ഞാനിനി പറഞ്ഞുതരാം. ജനനമരണസംബന്ധിയായ എല്ലാ
തെറ്റിദ്ധാരണകളും നീക്കാന്‍ നിനിക്കീ അറിവുതകും.~

ഗുരുകുലത്തിലെ പഠനശേഷം ശ്രീരാമന്‍ അച്ഛന്റെ കൊട്ടാരത്തില്‍ പലവിധ
വിനോദങ്ങളില്‍ ഏര്‍പ്പെട്ട്‌ കഴിഞ്ഞുവന്നു. ഒരു ദിവസം തന്റെ ദേശം മുഴുവന്‍
കാണാനുള്ള ആഗ്രഹത്തോടെ ശ്രീരാമന്‍ അച്ഛനോട്‌ തീര്‍ഥാടനത്തിനുള്ള അനുവാദം
ചോദിച്ചു. മഹാരാജാവ്‌ ശുഭമായ ഒരുദിനം തിരഞ്ഞെടുത്തു. ആ ദിവസം കുടുംബത്തിലെ
മുതിര്‍ന്നവരുടെ സ്നേഹപൂര്‍ണ്ണമായ അനുവാദാശീര്‍വാദങ്ങളോടെ രാമന്‍ യാത്ര
പുറപ്പെട്ടു. രാമനും അനുജന്മാരും ഹിമാലയം മുതല്‍ താഴോട്ട്‌ രാജ്യമെല്ലാം
ചുറ്റിക്കണ്ട്‌ തലസ്ഥാനത്ത്‌ മടങ്ങിയെത്തി. അതുകണ്ട്‌ രാജ്യവാസികള്‍
അതീവസന്തുഷ്ടരായി.
