
\section{ദിവസം 234}

\slokam{
ധ്യാതൃധ്യേയധ്യാനഹീനോ  നിര്‍മല: ശാന്തവാസന: \\
ബഭൂവാവാതദീപാഭോ ബലി: പ്രാപ്തമഹാപദ: (5/27/33) \\
}

ശുകമുനി പോയിക്കഴിഞ്ഞപ്പോള്‍ ബലി ഇങ്ങിനെ ചിന്തിച്ചു: എന്റെ ഗുരുനാഥന്‍ പറഞ്ഞത് സത്യവും ഉചിതവുമാണ്. തീര്‍ച്ചയായും എല്ലാം അനന്താവബോധമല്ലാതെ മറ്റൊന്നുമല്ല. ആ ബോധസത്ത ‘ഇത് സൂര്യനാണ്’ എന്ന് ചിന്തിക്കുമ്പോള്‍ ഇരുട്ടില്‍ നിന്നും വ്യതിരിക്തമായി സുര്യന്‍ നിലകൊള്ളുന്നു. വെളിച്ചത്തെ ഇരുട്ടില്‍ നിന്നും തിരിച്ചറിയാന്‍ പര്യാപ്തമാക്കുന്നത് ഈ ബോധമാണ്. ഭൂമിയെ ഭൂമിയായും ആകാശദിക്കുകളെ അപ്രകാരവും ലോകത്തെ ലോകമായും നിലനിര്‍ത്തുന്നത് ബോധമാണ്.  

ബോധം ഒരു പര്‍വ്വതത്തെ തിരിച്ചറിഞ്ഞില്ലെങ്കില്‍ അതിനു നിലനില്‍പ്പുണ്ടോ? ഇക്കാണുന്നതെല്ലാം ബോധം തന്നെ. ഇന്ദ്രിയങ്ങളും ദേഹവും മനസ്സിലുദിക്കുന്ന ആശകളും അകത്തുള്ളതും പുറത്തുള്ളതും ആകാശവും മാറ്റമെന്ന പ്രഹേളികയുമെല്ലാം ബോധം മാത്രം. ബാഹ്യവസ്തുക്കളുമായി എനിക്കുണ്ടാകുന്ന സമ്പര്‍ക്കവും അനുഭവങ്ങളുമെല്ലാം ബോധമാണുണ്ടാക്കുന്നത്. ശരീരമല്ല അതിനു ഹേതു. ശരീരമില്ലാതെതന്നെ ഞാന്‍ ബോധസ്വരൂപമാണ്. അത് തന്നെയാണീ വിശ്വത്തിന്റെ ആത്മാവ്.

രണ്ടാമതൊന്നില്ലാതെ (അദ്വൈതം) ബോധം എങ്ങും നിറഞ്ഞു വിളങ്ങുമ്പോള്‍ എനിക്കാരാണ് സുഹൃത്ത്? ആരാണെനിയ്ക്ക് ശത്രു? ഈ ബലിയെന്നു വിളിക്കുന്നയാളിന്റെ തല അരിഞ്ഞു കളഞ്ഞാലും അനന്താവബോധത്തിന്റെ തലയ്ക്ക് എന്ത് പറ്റാനാണ്? വെറുപ്പ് തുടങ്ങിയ ഇതര ഗുണങ്ങളും ഈ ബോധത്തിലെ വ്യതിയാനങ്ങള്‍ മാത്രമാണ്. എന്നാല്‍ മനസ്സും അതിന്റെ മാറ്റങ്ങളും, വെറുപ്പ്, ആസക്തി എന്നീ ഗുണങ്ങള്‍ പോലും യാഥാര്‍ത്ഥ്യമാവുക അസാദ്ധ്യം. കാരണം അനന്തതയില്‍ വികലതകള്‍ ഉണ്ടാവുന്നതെങ്ങിനെ? അത് നിത്യശുദ്ധമാണ്. അനന്തമാണ്‌. ബോധം അതിന്റെ പേരല്ല. അതൊരു സാമ്രാജ്യമത്രേ. അതിനു നാമമില്ല. രൂപമില്ല.

ഞാന്‍ ശാശ്വതമായ പരംപൊരുളാണ്. വിഷയങ്ങളും ആഖ്യാനങ്ങളും എനിക്കാവശ്യമില്ല. വശീകരണഗുണമുളള വസ്തു-വിഷയ സംബന്ധിയായ യാതൊരു ധാരണകള്‍ക്കും വശംവദമാവാത്തതും നിതാന്തസ്വതന്ത്രവുമായ ആ സര്‍വ്വവ്യാപിയെ- ബോധത്തെ, ഞാന്‍ നമിക്കട്ടെ. വിഷയ-വിഷയീ വ്യത്യാസമേതുമില്ലാത്ത എന്റെ ആത്മസത്തയേയും ഞാന്‍ പ്രണമിക്കുന്നു. എല്ലാ പ്രകടിതഭാവങ്ങളിലും ഉള്ള പ്രഭാസ്ഫുരണവും അതുതന്നെയാണെന്ന് നിശ്ചയം. അനുഭവങ്ങള്‍ക്കായുള്ള ത്വരയടങ്ങിയ ബോധസ്വരൂപമാണ് ഞാന്‍.. ആകാശം പോലെ അനന്തമാണു ഞാന്‍.. സുഖദു:ഖങ്ങളെന്നെ ചഞ്ചലപ്പെടുത്തുന്നില്ല. അവയെന്നെ എങ്ങിനെവേണമെങ്കിലും ബാധിക്കട്ടെ. എനിക്കവയില്‍ യാതൊരു പ്രതിപത്തിയുമില്ല. സുഖത്തോടോ ദു:ഖത്തോടോ - രണ്ടിനോടുമെനിക്ക് പക്ഷപാതവുമില്ല.

ഒരു വസ്തുവില്‍ നിന്നും മറ്റൊരു വസ്തുവിലേയ്ക്കുള്ള ഊര്‍ജ്ജപ്രവാഹം – ചൈതന്യത്തിന്റെ ചലനം, നഷ്ടലാഭങ്ങളെ ഉണ്ടാക്കുന്നില്ല. ബോധം മാത്രം ഉള്ളപ്പോള്‍ ചിന്തകളോ അവയുടെ പരിണാമവികാസങ്ങളോ ചുരുങ്ങലോ ഒന്നും  ബോധത്തില്‍ മാറ്റമുണ്ടാക്കുന്നില്ല. അതുകൊണ്ട് ആത്മാവില്‍ പരമശാന്തി കൈവരുന്നതുവരെ ഞാന്‍ കര്‍മനിരതനായി തുടരും. 

വസിഷ്ഠന്‍ തുടര്‍ന്നു: പിന്നീട്  ഓംകാരം ജപിച്ച് ബലി മൌനമായിരുന്നു. “എല്ലാ സംശയങ്ങളും ഒഴിഞ്ഞ്, വസ്തുധാരണകള്‍ നീക്കി വിഷയ-വിഷയീ ഭേദഭാവമില്ലാതെ, ചിന്ത-ചിന്തകന്‍--ചിന്ത്യവിഷയം (ധ്യാനം, ധ്യാനി, ധ്യാനവിഷയം) എന്ന തോന്നലുകളില്ലാതെ ബലി അവിടെയിരുന്നു. ഉദ്ദേശ ലക്ഷ്യങ്ങളൊന്നുമില്ലാതെ പ്രശാന്തനായി, മനസ്സടങ്ങി, കാറ്റില്ലാത്തയിടത്തു കത്തിച്ച  ദീപമെന്നപൊലെ അദ്ദേഹം പരമപദത്തില്‍ അഭിരമിച്ചു.” അദ്ദേഹമങ്ങിനെ ഏറെ നാള്‍ കഴിഞ്ഞു. 

