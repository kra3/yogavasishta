\newpage
\section{ദിവസം 006}

\begin{center}
നിരസ്താസ്ഥോ നിരാശേസൗ നിരീഹോസൗ നിരാസ്പദ:\\
ന മൂഠോ ന ച മുക്തോസൗ തേന തപ്യാമഹേ ഭൃശം (1/10/45)\\
\end{center}

വാല്‍മീകി തുടര്‍ ന്നു: തന്റെ ഗുരുവായ വസിഷ്ഠന്റെ ആഗ്രഹപ്രകാരം ദശരഥന്‍ ഒരു സേവകനോട്‌ ശ്രീരാമനെ സഭയിലേയ്ക്കു കൂട്ടിക്കൊണ്ടു വരാന്‍ ആവശ്യപ്പെട്ടു. സേവകന്‍ മടങ്ങിവന്ന് രാമന്‍ ഉടനെ വന്നുചേരുന്നതാണെന്ന് അറിയിച്ചു. എന്നിട്ടു പറഞ്ഞു: "രാജകുമാരന്‍ ഇപ്പോള്‍ വളരെ വിഷാദവാനായി കാണപ്പെടുന്നു. മാത്രമല്ല ആരുടേയും സാമീപ്യം ഇഷ്ടപ്പെടുന്നുമില്ല" ഇതുകേട്ട്‌ സംഭ്രമത്തോടെ ദശരഥന്‍ രാമന്റെ പള്ളിയറയിലെ സേവകനോട്‌ രാമന്റെ ശാരീരികവും മാനസീകവുമായ പ്രശ്നങ്ങള്‍ എന്താണെന്ന് ചോദിച്ചു. 


സേവകനും ആശങ്കാകുലനായിരുന്നു. അയാള്‍ പറഞ്ഞു: "തീര്‍ത്ഥാടനം കഴിഞ്ഞു വന്നതില്‍പ്പിന്നെ രാജകുമാരനില്‍ എന്തോ വലിയൊരു മാറ്റം കാണുന്നുണ്ട്‌. സ്നാനത്തിലോ പൂജയിലോ താല്‍പ്പര്യം കാണിക്കുന്നില്ല. അന്ത:പുരത്തിലെ ആരുമായും അടുപ്പമില്ല. ആഭരണങ്ങളിലോ വിലപിടിച്ച കല്ലുകളിലോ താല്‍പ്പര്യമില്ല. മനം കവരുന്നതും സന്തോഷമുണ്ടാക്കുന്നതുമായ എന്തു കൊടുത്താലും വിഷാദമാര്‍ന്ന ഒരു നോട്ടമാണു കുമാരന്‌. രാജനര്‍ത്തകിമാരു പോലും അദ്ദേഹത്തിനൊരു ശല്യമാണ്‌. ആഹാരം, നിദ്ര, വിശ്രമം, കുളി, ഇരിപ്പ്‌, കിടപ്പ്‌ എല്ലാം യാന്ത്രീകമായി ഒരു മൂകനെയും ബധിരനേയും പോലെയാണിപ്പോള്‍ ചെയ്യുന്നത്‌. പലപ്പോഴും തന്നോടു തന്നെ പുലമ്പുന്നതു കാണാം. "ഈ സമ്പത്തുകൊണ്ടും ഐശ്വര്യം കൊണ്ടും എന്തുകാര്യം? ശത്രുതയും മിത്രഭാവവും കൊണ്ടെന്തു കാര്യം? ഒന്നും യാദാര്‍ത്ഥ്യമല്ല" എന്നെല്ലാം. മിക്കവാറും സമയം അദ്ദേഹം മൌനത്തിലാണ്‌. വിനോദപരിപാടികളില്‍ തീരെ താല്‍പ്പര്യം കാണിക്കുന്നില്ല. ഏകാന്തതയില്‍ കുമാരന്‍ അഭിരമിക്കുന്നു. സ്വയം ഏതോ ചിന്തയില്‍ മുഴുകിയിരിക്കുന്നു. കുമാരന്‌ എന്താണ്‌ സംഭവിച്ചതെന്നോ എന്താണാ മനസ്സിലെ ചിന്തകളെന്നോ എന്താണദ്ദേഹം ആഗ്രഹിക്കുന്നതെന്നോ നമുക്കറിയാന്‍ കഴിയുന്നില്ല. ദിനം പ്രതി ശരീരം ക്ഷീണിച്ചു വരുന്നു. 

വീണ്ടും വീണ്ടും കുമാരന്‍ പാടുന്നത്‌ ഇതാണ്‌: "കഷ്ടം! ആ പരമപദം പ്രാപിക്കാന്‍ ശ്രമിക്കുന്നതിനു പകരം നാം നമ്മുടെ ഊര്‍ജ്ജം പലവിധത്തില്‍ പാഴാക്കിക്കളയുന്നു. ആളുകള്‍ തങ്ങളുടെ ദു:ഖത്തെപ്രതി ഉറക്കെ കരയുന്നു. തങ്ങള്‍ അഗതികളാണെന്നു വിലപിക്കുന്നു. എന്നാല്‍ ആരും തന്നെ ആ ദു:ഖങ്ങളുടെ മൂലകാരണങ്ങളില്‍ നിന്നും വഴിമാറിപ്പോകാന്‍ കൂട്ടാക്കുന്നില്ല." ഇതുകണ്ട്‌ കുമാരന്റെ എളിയ സേവകരായ ഞങ്ങള്‍ എന്തുചെയ്യണം എന്നറിയാതെ വല്ലാതെ വിഷമിക്കുകയാണ്‌. 'അദ്ദേഹത്തിന്‌ പ്രത്യാശകളില്ല. ആഗ്രഹങ്ങളും. ഒന്നിനോടും മമതയില്ല. ഒന്നിനേയും ആശ്രയിക്കുന്നുമില്ല. അദ്ദേഹത്തിന്‌ മതിഭ്രമമോ ബുധിമാന്ദ്യമോ ഇല്ല. പക്ഷേ അദ്ദേഹം പ്രബോധവാനാണെന്നു പറയാനും വയ്യ.' ചിലസമയം മനസ്സുതളര്‍ന്ന് ആത്മഹത്യാപരമായ ചിന്തകള്‍ ക്കു വശംവദനാവുന്നു: "ഈ സമ്പത്തുകൊണ്ടും അമ്മമാരേക്കൊണ്ടും മറ്റു ബന്ധുജനങ്ങളേക്കൊണ്ടും എന്തു പ്രയോജനം? ഈ രാജ്യം കൊണ്ട്‌ എന്തു പ്രയോജനം? ഈ ലോകത്ത്‌ ഉല്‍ കര്‍ഷേച്ഛ കൊണ്ടും എന്താണു കാര്യം?" 


പ്രഭോ അങ്ങേയ്ക്കു മാത്രമേ കുമാരന്റെ ഈ അവസ്ഥയ്ക്കു പരിഹാരം കണ്ടുപിടിക്കുവാനാവൂ.
