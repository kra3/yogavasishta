 
\section{ദിവസം 121}

\slokam{
സ്വയം പ്രഹരതി സ്വാന്തം സ്വയമേവ സ്വേച്ഛയാ\\
പലായതേ സ്വയം ചൈവ  പശ്യാജ്ഞാനവിജൃംഭിതം (3/99/36)\\
}

വസിഷ്ഠന്‍ തുടര്‍ന്നു: അല്ലയോ രാമ: ഈ വന്‍ കാട്‌ ദൂരെയൊന്നുമല്ല. ഈ അജ്ഞാതമനുഷ്യന്‍ ജീവിക്കുന്നത്‌ അജ്ഞാതമായ ഒരു നാട്ടിലുമല്ല. ഈ ലോകം തന്നെയാണാ കാട്‌. . വലിയൊരു ശൂന്യതയാണതെങ്കിലും അന്വേഷണത്തിന്റെ വെളിച്ചത്തിലൂടെമാത്രമേ ആ സത്യമറിയാനാവൂ. ഈ വെളിച്ചമാണ്‌ 'ഞാന്‍' എന്ന പ്രഹേളിക. ഈ ജ്ഞാനം, എല്ലാവരും അംഗീകരിക്കുന്ന ഒന്നല്ല. അതിനെ നിരാകരിക്കുന്നവര്‍ ദുരിതാനുഭവങ്ങളിലൂടെ അനവരതം കടന്നുപൊയ്ക്കൊണ്ടേയിരിക്കുന്നു. ഈ അറിവിനെ ഉള്‍ക്കൊണ്ടവരാണ്‌ പ്രബുദ്ധരായ മഹത്തുക്കള്‍.   എണ്ണമറ്റ പ്രകടിതഭാവങ്ങളോടുകൂടിയ മനസ്സാണ്‌ ആയിരം കയ്യുകളുള്ള മനുഷ്യന്‍.. മനസ്സ്‌ സ്വയം വാസനകളാല്‍ ശിക്ഷിക്കപ്പെടുന്നു. ശാന്തിയേതുമില്ലാതെ അലയുകയും ചെയ്യുന്നു. കഥയിലെ പൊട്ടക്കിണര്‍ നരകവും വാഴത്തോപ്പ്‌ സ്വര്‍ഗ്ഗവുമാണ്‌.. മുള്‍ച്ചെടികള്‍ നിറഞ്ഞ നിബിഢവനം ലൌകീകമനുഷ്യന്റെ ജീവിതമാണ്‌.. ഭാര്യ, കുട്ടികള്‍ , സമ്പത്ത്‌ എല്ലാം അവനെ ശല്യപ്പെടുത്തിക്കൊണ്ടിരിക്കുന്നു.

അപ്പോള്‍ മനസ്സ്‌ ചുറ്റിത്തിരിഞ്ഞ്‌ നരകത്തിലെത്തുന്നു. പിന്നെ സ്വര്‍ഗ്ഗത്തില്‍ ; ഒടുവില്‍ തിരികെ മനുഷ്യലോകത്തുമെത്തുന്നു. വിവേകത്തിന്റെ വെളിച്ചം ഭ്രമാത്മകമനസ്സിനെ ഉണര്‍ത്താന്‍ ശ്രമിച്ചാലും മനസ്സതിനെ നിരാകരിക്കുന്നു. ഈ ജ്ഞാനത്തെ തന്റെ ശത്രുവെന്നു കണക്കാക്കുന്നു. പിന്നെ അവന്‍ ദു:ഖത്തിലാണ്ടു വിലപിക്കുകയായി. ചിലപ്പോള്‍ അവന്‍ പക്വതയില്ലാത്ത അറിവിന്റെ പ്രകാശധവളിമയില്‍ , ശരിയായ അറിവുറയ്ക്കാതെ തന്നെ ലോകസുഖങ്ങളെല്ലാം ഉപേക്ഷിക്കുന്നു. അത്തരം സന്യാസം പിന്നീട്‌ കൂടുതല്‍ ദു:ഖത്തിനിടയാക്കുന്നു. എന്നാല്‍ ഈ സന്യാസം പക്വമായ അറിവിന്റെ നിറവില്‍ , മനോവ്യാപാരങ്ങളേപ്പറ്റി ചെയ്ത അത്മാന്വേഷണത്തിന്റെ ഫലമായി ഉണ്ടായതാണെങ്കില്‍ അതു പരമാനന്ദത്തിലേയ്ക്കു നയിക്കുന്നു. ഒരുപക്ഷെ ആ നിലയെയിലെത്തിയ മനസ്സ്‌ പൂര്‍വ്വകാലത്തെ സന്തോഷം, ഉല്ലാസം എന്നീ ധാരണകളെ അമ്പരപ്പോടെയാവും കാണുക. കഥയിലെ മനുഷ്യന്റെ അവയവങ്ങള്‍ ഓരോന്നായി മുറിഞ്ഞ്‌ കൊഴിഞ്ഞുപോയപോലെ സംന്യസ്ഥനായ ജ്ഞാനിയുടെ മനസ്സില്‍നിന്നും ലീനവാസനകള്‍ അപ്രത്യക്ഷമാവുന്നു.

"അവിദ്യയുടെ ലീലയെന്തെന്നു നോക്കൂ. അതൊരുവനെ സ്വേച്ഛയാ സ്വയം ഉപദ്രവിപ്പിച്ച്‌ അവനെ യാതൊരു കാര്യവുമില്ലാതെ അവിടെയുമിവിടെയും ഓടിച്ച്‌ അര്‍ത്ഥശൂന്യമായ പരിഭ്രാന്തിയില്‍ ആഴ്ത്തുന്നു." ആത്മാജ്ഞാനത്തിന്റെ വെളിച്ചം എല്ലാവരിലും വീഴുന്നുണ്ടെങ്കിലും മനുഷ്യന്‍ സ്വന്തം ലീനവാസനകളുടെ പ്രേരണയാല്‍ ലോകത്തിലലയുകയാണ്‌.. മനസ്സാണെങ്കില്‍ ഇതിനു വളംവെച്ച്‌ ദുരിതങ്ങളുടെ തീവ്രത വര്‍ദ്ധിപ്പിച്ച്‌ അവനെ വട്ടംചുറ്റിക്കുന്നു. സ്വയം മോഹത്താലും ചാപല്യത്താലും ആശയാലും പ്രത്യാശയാലും അവന്‍ ബദ്ധനാകുന്നു. ദു:ഖാനുഭവങ്ങള്‍ അവനെ അശാന്തനും നിരാശനുമാക്കുന്നു. എന്നാല്‍ ജ്ഞാനി അതിനെ കൈവിടാതെ ആത്മാന്വോഷണസാധന തുടരുന്നപക്ഷം ദു:ഖം അവനെ തീണ്ടുകയില്ല. നിയന്ത്രണമേതുമില്ലാത്ത മനസ്സാണ്‌ ദു:ഖകാരണം. എന്നാല്‍ അതിനെ ആത്മാന്വേഷണത്തിലൂടെ അറിഞ്ഞാല്‍ സൂര്യോദയത്തില്‍ മൂടല്‍മഞ്ഞെന്നപോലെ ദു:ഖം അലിഞ്ഞില്ലാതാവുന്നു.
