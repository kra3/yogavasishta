\section{ദിവസം 159}

\slokam{
ദൃശ്യം പശ്യൻ സ്വമാത്മാനം ന ദൃഷ്ടാ സമ്പ്രപശ്യതി\\
പ്രപഞ്ചാക്രാന്ത സംവിത്തേ: കസ്യോദേതി നിജാ സ്ഥിതി: (4/18/27)\\
}

വസിഷ്ഠൻ തുടർന്നു: രാമ: ഒരുവിത്തിനുള്ളിൽനിന്നും മരം ഉണ്ടാവുന്നത് ആ വിത്തിനെ നശിപ്പിച്ചിട്ടാണ്‌.. എന്നാൽ പരബ്രഹ്മം ലോകസൃഷ്ടി ചെയ്യുന്നത് സ്വയം നശിച്ചിട്ടല്ല. മരം (ലോകം) പ്രത്യക്ഷമാവുമ്പോൾത്തന്നെ വിത്ത് (ബ്രഹ്മം) മാറ്റമൊന്നുമില്ലാതെ നിലകൊള്ളുന്നു. അതുകൊണ്ട് യാതൊരുവിധത്തിലും ബ്രഹ്മത്തിനെ മറ്റൊരു വസ്തുവുമായും താരതമ്യം ചെയ്യുക വയ്യ. മരം, വിത്ത്, തുടങ്ങിയവയ്ക്ക് നിയതമായ നിർവ്വചനങ്ങളുണ്ടല്ലോ, എന്നാൽ ബ്രഹ്മം എന്നത് നാമ-രൂപ-രഹിതവും നിർവ്വചനാതീതവുമത്രേ. ഈ ബ്രഹ്മം തന്നെയാണ്‌ വൈവിദ്ധ്യമാർന്നവകളെ പ്രത്യക്ഷപ്പെടുത്തുന്നത്. എന്നാൽ മറ്റൊരുവിധത്തിൽ നോക്കിയാൽ അതിനു മാറ്റമൊന്നുമില്ല, അതു മറ്റൊന്നായി പരിണമിക്കുന്നില്ല, അത് ശാശ്വതമാണ്‌. അചലമാണ്‌..

അതിനാൽ ബ്രഹ്മത്തെക്കുറിച്ച് ആർക്കുമൊരു തത്വം പറയാനാവില്ല. ബ്രഹ്മമാണ്‌ ഇക്കാണായതെല്ലാമായത് എന്നോ ബ്രഹ്മമല്ല പ്രത്യക്ഷലോകമായത് എന്നോ പറയുക വയ്യ. “ആത്മാവിനെ ഒരു വസ്തുവായി ദർശിക്കുമ്പോൾ അവിടെ ദൃഷ്ടാവ് സാക്ഷാത്കരിക്കപ്പെടുന്നില്ല. വസ്തുപ്രപഞ്ചമെന്ന അനുഭവം ഉള്ളിടത്തോളം ഒരുവനിൽ ആത്മസാക്ഷാത്കാരം സാദ്ധ്യമല്ല.” മരുഭൂമിയിൽ കാനൽ ജലം കാണുമ്പോൾ അവിടെ ഉയരുന്ന ചൂടുകാറ്റിനെ നാം കാണുന്നില്ല. എന്നാൽ ചൂടുകാറ്റിനെപ്പറ്റി അറിയുന്നയാൾ കാനൽ ജലം കാണുന്നുമില്ല. ഒന്ന് സത്യമായിരിക്കുമ്പോള്‍ മറ്റേത് അങ്ങിനെയല്ല. ലോകത്തിലെ വസ്തുക്കളെ കാണുന്ന കണ്ണുകൾക്ക് സ്വയം കാണാൻ കഴിവില്ല. വസ്തുനിഷ്ടത എന്ന ധാരണ നിലനിൽക്കുമ്പോൾ ആത്മസാക്ഷാത്കാരം ഉണ്ടാവുകയില്ല.

ബ്രഹ്മം ആകാശം പോലെ സൂക്ഷ്മമാണ്‌.. യാതൊരു പരിശ്രമങ്ങൾ കൊണ്ടും അതിനെ സാക്ഷാത്കരിക്കുവാനാവില്ല. കാണപ്പെടുന്ന വസ്തുക്കൾ തന്നിൽ നിന്നും വിഭിന്നമാണെന്ന തോന്നൽ ഉള്ളിലുള്ളിടത്തോളം (ദൃഷ്ടിയും ദൃഷ്ടാവും വെവ്വേറെയെന്ന തോന്നൽ) ബ്രഹ്മസാക്ഷാത്കാരം അകലെത്തന്നെയാണ്‌..  ദൃഷ്ടിയും ദൃഷ്ടാവും തമ്മിലുള്ള ഈ അന്തരം ഇല്ലാതായി രണ്ടും ഒന്നായി ‘കാണുമ്പോൾ’ മാത്രമേ സത്യസാക്ഷാത്കാരം സാദ്ധ്യമാവൂ.

യാതൊരു വിഷയവും വിഷയത്തിൽ നിന്നും പരിപൂർണ്ണമായും വ്യത്യസ്ഥ സ്വഭാവമുള്ളതല്ല. വിഷയിയെ വിഷയമായി ‘കാണുക’ എന്നതും സാദ്ധ്യമല്ല. വാസ്തവത്തിൽ വിഷയി (ആത്മാവ്) വിഷയമായി (വസ്തുക്കൾ) കാണപ്പെടുകയാണ്‌. ഇതല്ലാതെ മറ്റുവിഷയങ്ങൾ ഇവിടെയില്ല. എന്നാൽ വിഷയി (ആത്മാവ്) മാത്രമേ ഉള്ളു എന്നു പറയുമ്പോൾ അത് വിഷയം ആവുന്നതെങ്ങിനെ? (വിഷയമില്ലാതെ വിഷയി ഇല്ലല്ലോ). ഇവിടെ വിഷയി-വിഷയ വിവേചനം ഇല്ല തന്നെ.

പഞ്ചസാര വിവിധ മധുരപലഹാരങ്ങളാവുന്നത് അതിന്റെ സഹജഭാവമായ മാധുര്യത്തിനു കോട്ടമൊന്നും ഉണ്ടാവാത്തവിധത്തിലാണ്‌..  അതുപോലെ ബ്രഹ്മം, അനന്താവബോധം, സ്വയം സങ്കല്‍പ്പിച്ചുണ്ടാക്കുന്ന അനന്തമായ വൈവിധ്യങ്ങൾ അതിന്റെ സഹജഭാവത്തെ മാറ്റുന്നില്ല. ഈ എണ്ണമില്ലാത്ത വൈജാത്യങ്ങൾ അനന്തമായിത്തന്നെ പ്രകടമായിക്കൊണ്ടിരിക്കുന്നു. 
