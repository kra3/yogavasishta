\newpage
\section{ദിവസം 106}

\slokam{
മഹാതാമേവ സംപര്‍ക്കാത്പുനര്‍ദു:ഖം ന ബാദ്ധ്യതേ\\
കോ ഹി  ദീപശിഖാഹസ്ഥസ്തമസാ പരിഭൂയതേ (3/82/8)\\
}

വസിഷ്ഠന്‍ തുടര്‍ന്നു: രാജാവിന്റെ ഈ വാക്കുകള്‍കേട്ട്‌ കാര്‍ക്കടി പരമശാന്തിയെ പ്രാപിച്ചു. അവളുടെ രാക്ഷസീയരൂപം അവളെവിട്ടുപോയി.

അവള്‍ അവരോടു പറഞ്ഞു: ജ്ഞാനികളേ, നിങ്ങള്‍ രണ്ടും ആരാധ്യരാണ്‌.. നിങ്ങളുടെ സദ്സംഗം കൊണ്ട്‌ ഞാന്‍ ഉണര്‍ന്നിരിക്കുന്നു. "മഹത്തുക്കളുടെ സദ്സംഗം ആസ്വദിക്കുന്നവര്‍ക്ക്‌ ഇഹലോക ദുരിതങ്ങള്‍ അനുഭവിക്കേണ്ടി വരികയില്ല. കയ്യില്‍ ദീപമുള്ളവന്‌ ഒരിടത്തും ഇരുട്ട്‌ കാണാനാവാത്തതുപോലെയാണത്‌.". പറയൂ, നിങ്ങള്‍ ക്കുവേണ്ടി എന്താണു ഞാന്‍ ചെയ്യേണ്ടത്‌?

രാജാവു പറഞ്ഞു: മഹിളാമണീ, എന്റെ രാജ്യത്ത്‌ അനേകജനങ്ങള്‍ വാതം കൊണ്ടും ഹൃദയരോഗങ്ങള്‍കൊണ്ടും കഷ്ടപ്പെടുന്നുണ്ട്‌.. രാജ്യത്ത്‌ അതിസാരവും പടര്‍ന്നിരിക്കുന്നു. ഇതിനെക്കുറിച്ച്‌ അന്വേഷിച്ചറിഞ്ഞ്‌ പരിഹാരം തേടാനാണ്‌ ഞാനും മന്ത്രിയും ഈ രാത്രിയില്‍ കൊട്ടാരംവിട്ട്‌ പുറത്തുവന്നത്‌.. എന്റെ വിനീതമായ അഭ്യര്‍ത്ഥന ഇത്രമാത്രം: എന്റെ പ്രജകളുടെ ജീവനെടുക്കരുത്‌.. (കാര്‍ക്കടി രാജാവിന്റെ അഭ്യര്‍ത്ഥന സ്വീകരിച്ചു.) ഇനി പറഞ്ഞാലും- നിന്റെ ദയവിന്‌ ഞാനെന്തു പകരം തരും? നിന്റെ വിശപ്പെങ്ങിനെയാണകറ്റുക?

കാര്‍ക്കടി പറഞ്ഞു: ഒിക്കല്‍ ഹിമാലയത്തില്‍പ്പോയി തപസ്സുചെയ്ത്‌ ഈ ദേഹമുപേക്ഷിക്കണമെന്ന് എനിക്കാഗ്രഹമുണ്ടായിരുന്നു. എന്നാല്‍ ഇപ്പോള്‍ ഞാനാ ആഗ്രഹം ഉപേക്ഷിച്ചിരിക്കുന്നു. ഞാന്‍ എന്റെ ജീവിതകഥ നിങ്ങള്‍ക്കായി പറയാം. പണ്ട്‌ ഞാന്‍ ഭീമാകാരയായ ഒരു രാക്ഷസിയായിരുന്നു. എനിക്ക്‌ മനുഷ്യരെ തിന്നാന്‍ ആര്‍ത്തിയുണ്ടായിട്ട്‌ അതു തീര്‍ക്കാന്‍ ഞാന്‍ തപസ്സിലേര്‍പ്പെടുകയും ചെയ്തു. ബ്രഹ്മാവില്‍ നിന്നുകിട്ടിയ വരത്താല്‍ ഞാന്‍ ഒരു വിഷൂചികയായി. അതിസാരം (വൈറസ്‌)) ജനങ്ങള്‍ക്ക്‌ പറയാനരുതാത്ത ദുരിതങ്ങള്‍ വിതച്ചു. മാതാപിതാക്കളില്‍ നിന്നും മക്കളിലേയ്ക്ക്‌ ഞാന്‍ ലുക്കീമിയ അണുക്കളെ പകര്‍ച്ച വ്യാധിയായി കടത്തിവന്നിരുന്നു. ബ്രഹ്മാവിനാല്‍ രചിതമായ ഒരു മന്ത്രത്താലാണ്‌ എന്നെ ഒരു പരിധിവരെ നിയന്ത്രിക്കാന്‍ സാധിച്ചത്‌.. നിങ്ങളും ഈ മന്ത്രം പഠിച്ചാലും- ഇതുകൊണ്ട്‌ വാതം, ഹൃദയരോഗങ്ങള്‍ , ലുക്കീമിയ, തുടങ്ങിയ രക്തസംബന്ധിയായ ദോഷങ്ങള്‍ ഇല്ലാതാക്കാം.

മൂവരും നദീതീരത്തു പോയി. അവിടെവച്ച്‌ കാര്‍ക്കടിയില്‍ നിന്നും രാജാവിന്‌ മന്ത്രോപദേശം കിട്ടി. ഈ മന്ത്രം ജപത്തിലൂടെയാണ്‌ പ്രാബല്യത്തിലാവുക. നന്ദിപൂര്‍വ്വം രാജാവു പറഞ്ഞു: അല്ലയോ ദയാശീലേ, നീയെന്റെ സുഹൃത്തും ഗുരുവുമാണ്‌.. സദ്ജനങ്ങള്‍ സൌഹൃദത്തിനെ വിലമതിക്കുന്നു. ദയവായി സൌമ്യസുന്ദരമായ ഒരു രൂപം സ്വീകരിച്ചാലും. എന്നിട്ട്‌ എന്റെ കൊട്ടാരത്തില്‍ വന്ന് ആതിഥ്യം സീകരിച്ചു വസിച്ചാലും. സജ്ജനങ്ങളെ തീരെ ബാധിക്കാതെ കഴിയൂ, പാപികളേയും കള്ളന്മാരേയും നിനക്കു ഭക്ഷിക്കാന്‍ തരാം. കാര്‍ക്കടി സമ്മതിച്ചു. അവള്‍ ഒരു സുന്ദരയുവതിയായി രാജാവിന്റെ കൂടെ അതിഥിയായി താമസം തുടങ്ങി. രാജാവ്‌ കള്ളന്മാരേയും കൊള്ളക്കാരേയും അവള്‍ക്ക്‌ വിട്ടുകൊടുത്തു. എന്നും രാത്രിയില്‍ അവള്‍ തന്റെ ഭീമാകാരം പൂണ്ട്‌ അവരെ ആഹരിച്ചുപോന്നു. പകല്‍ സമയം അവള്‍ തന്റെ സുന്ദരരൂപത്തില്‍ രാജാവിന്റെ സുഹൃത്തായി വിലസി. ഭക്ഷണം കഴിഞ്ഞാല്‍ അവള്‍ പലപ്പോഴും വര്‍ഷങ്ങളോളം സമാധിയില്‍ ആയിരിക്കും. പിന്നീടവള്‍ സാധാരണ ബോധത്തിലേയ്ക്കും ജീവിതത്തിലേയ്ക്കും തിരിച്ചുവരുന്നു. 

