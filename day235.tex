\section{ദിവസം 235}

\slokam{
ന ചിഞ്ചിദപി കര്‍ത്തവ്യം യദി നാമ മയാധുനാ \\
തത്കസ്മാന്ന കരോമീദം കിംചിത്പ്രകൃതകര്‍മ്മ വൈ (5/29/19) \\
}

വസിഷ്ഠന്‍ തുടര്‍ന്നു: ബലി മഹാരാജാവിന്റെ പ്രജകളായ അസുരന്മാര്‍ കൊട്ടാരത്തിലേയ്ക്ക് ഓടിയെത്തി ധ്യാനനിരതനായിരിക്കുന്ന രാജാവിനെ വളഞ്ഞു. എന്താണ് സംഭവിക്കുന്നതെന്നറിയാഞ്ഞ് അവര്‍ ഗുരുവായ ശുക്രനെ മനസാ സ്മരിച്ചു വരുത്തി. ബലി ധ്യാനത്തിന്റെ അതീന്ദ്രിയമായ  അബോധാവസ്ഥയിലാണെന്ന് ശുക്രന്‍ മനസ്സിലാക്കി. അദ്ദേഹം സന്തോഷസൂചകമായ ഒരു ചെറുപുഞ്ചിരിയോടെ അസുരന്മാരോടിങ്ങിനെ പറഞ്ഞു: ഇതെത്ര മഹാഭാഗ്യം! അല്ലയോ അസുരന്മാരേ, ബാലിരാജന്‍ സ്വപ്രയത്നത്താല്‍ പരമപദപ്രാപ്തി കൈവരിച്ചിരിക്കുന്നു. അദ്ദേഹമങ്ങിനെ ആത്മസ്വരൂപത്തില്‍ വിരാജിച്ചിരുന്നുകൊള്ളട്ടെ. ഇഹലോകത്തെപ്പറ്റി തിരിച്ചറിയുന്ന മാനസികവ്യാപാരം അദ്ദേഹത്തില്‍ ഇപ്പോഴില്ല. അതുകൊണ്ട് രാജാവിനോടിപ്പോഴൊന്നും ചോദിക്കണ്ട. അജ്ഞാനത്തിന്റെ ഇരുട്ടകന്നാല്‍പ്പിന്നെ ആത്മജ്ഞാനത്തിന്റെ സൂര്യോദയമായി. ബാലിരാജന്റെ അവസ്ഥ അതാണ്‌.. കുറച്ചു കഴിഞ്ഞ് അദ്ദേഹം താനേ ഈയവസ്ഥയില്‍നിന്നും എഴുന്നേല്‍ക്കും. അദ്ദേഹത്തിന്‍റെ ബോധമണ്ഡലത്തില്‍ ഈ ലോകം വീണ്ടും മുളപൊട്ടിവിടരും. അതുകൊണ്ട് നിങ്ങളേവരും നിങ്ങളുടെ ജോലികളിലേയ്ക്ക് തിരിച്ചു പോയാലും.  അദ്ദേഹം ഒരായിരം കൊല്ലം കഴിഞ്ഞു മാത്രമേ  ഈ ലോകബോധത്തിലേയ്ക്ക് തിരിച്ചു വരൂ.
 
ഇതു കേട്ട അസുരന്മാര്‍ താന്താങ്ങളുടെ കര്‍മമങ്ങളിലേയ്ക്ക് മടങ്ങിപ്പോയി. ആയിരം ദേവ വര്‍ഷങ്ങള്‍ ധ്യാനത്തില്‍ കഴിഞ്ഞ ബലി ദേവഗന്ധര്‍വ്വന്മാരുടെ സംഗീതം കേട്ടുണര്‍ന്നു. അദ്ദേഹത്തില്‍ നിന്നും ഉദ്ഗമിച്ച ഒരലൌകിക പ്രഭ നഗരത്തെ ശോഭായമാനമാക്കി. അസുരന്മാര്‍ വീണ്ടും കൊട്ടാരത്തില്‍ എത്തും മുമ്പ് അദ്ദേഹമിങ്ങിനെ ആലോചിച്ചു. കുറച്ചുനേരത്തേയ്ക്ക് ഞാനിരുന്ന ഈ അവസ്ഥ എത്ര അത്ഭുതകരമായിരുന്നു ! എനിയ്ക്ക് ആ സ്ഥിതിയില്‍ തുടരണമെന്നുണ്ട്. പക്ഷെ ബാഹ്യലോകത്തിലെ കാര്യങ്ങള്‍ക്കായി  ഞാന്‍ എന്ത് ചെയ്യണം?  ഇപ്പോള്‍ എന്റെ ഹൃദയം പരമാനന്ദത്താല്‍ പൂരിതമാണ്.

അപ്പോഴെയ്ക്ക് അസുരന്മാര്‍ ബാലിക്ക് ചുറ്റും വന്നു കൂടിയിരുന്നു. അപ്പോഴും ബലി ഇങ്ങിനെ ചിന്തിച്ചു: ഞാന്‍ ബോധമാണ്. എന്നില്‍ വക്രതയും  വൈചിത്ര്യങ്ങളൊന്നുമില്ല. എന്നില്‍ ഉപേക്ഷിക്കാനോ എനിക്ക് നേടാനോ എന്തുണ്ട്? എന്ത് തമാശയാണിത്! ഞാന്‍ മുക്തി ആശിക്കുന്നു. പക്ഷെ ഞാനെപ്പോഴാണ് ബന്ധിതനായത്?ആരാണെന്നെ ബന്ധിച്ചത്? എങ്ങിനെയാണതുണ്ടായത്? ഇപ്പോഴും ഞാനെന്തു കൊണ്ടാണ് മുക്തി ആശിക്കുന്നത്? ബന്ധവും മുക്തിയും ഒന്നും ഇല്ല. ധ്യാനസപര്യ കൊണ്ടും ധ്യാനിക്കാത്തതു കൊണ്ടും എന്താണ് ഞാന്‍ നേടാന്‍ പോവുന്നത്? ധ്യാനമെന്ന ഭ്രമത്തില്‍നിന്നും മുക്തനായി എന്തുസ്ഥിതിയാണോ എപ്പോഴുമുള്ളത് അതില്‍ത്തന്നെ നിലകൊള്ളുന്നതാണ് നല്ലത്. എനിക്ക് നേടാനും നഷ്ടപ്പെടാനും ഒന്നുമില്ല.

ഞാന്‍ ധ്യാനാവസ്ഥയും ധ്യാനമില്ലാത്ത അവസ്ഥയും ഒന്നും ആഗ്രഹിക്കുന്നില്ല. ആഹ്ലാദമോ അനാഹ്ലാദമോ എനിക്ക് വേണ്ട. പരമപദവും ലോകവും ഒന്നും എനിക്ക് വേണ്ട. ഞാന്‍ ജീവിക്കുന്നില്ല, മരിച്ചിട്ടുമില്ല. ഞാന്‍ സത്തും അസത്തുമല്ല. അനന്തസത്യവസ്തുവായ എനിക്ക് നമസ്കാരം! ഈ ലോകമെന്റെ സാമ്രാജ്യമായിക്കൊള്ളട്ടെ. ഞാനങ്ങിനെതന്നെ നിലകൊള്ളട്ടെ. എനിക്കീ സാമ്രാജ്യം ഇല്ലാതാവട്ടെ. എനിക്ക് മാറ്റമൊന്നുമുണ്ടാവുകയില്ല.

ധ്യാനംകൊണ്ടും സാമ്രാജ്യം കൊണ്ടും ഞാനെന്തു ചെയ്യാനാണ്? കാര്യങ്ങള്‍ അങ്ങിനെത്തന്നെ ആവട്ടെ. ഞാന്‍ ആര്‍ക്കും അധീനമല്ല. എനിക്കാരും അധീനരായില്ല. എനിക്കാരുമില്ല. ആരുമെന്റേതല്ല.! “ഞാന്‍ എന്നറിയപ്പെടുന്ന എനിക്ക് ചെയ്യേണ്ടതായി ഒന്നുമില്ല. അപ്പോള്‍പ്പിന്നെ സഹജമായി, സ്വാഭാവികമായി വന്നുചേരുന്ന കര്‍മ്മങ്ങളെ എനിക്കെന്തുകൊണ്ട് ചെയ്തുകൂടാ?” ഇത്രയും പറഞ്ഞ് ബലി രാജാവ് തന്റെ പ്രജകളെ തന്റെ പ്രഭയേറിയ കണ്ണുകള്‍ കൊണ്ട് നോക്കി. ആ നോട്ടം, ഉദയസൂര്യന്‍ താമരയെ നോക്കുമ്പോലെ ചാരുതയാര്‍ന്നതായിരുന്നു.

