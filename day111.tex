 
\section{ദിവസം 111}

\slokam{
മനോ ഹി ജഗതാം കര്‍തൃ മനോ ഹി പുരുഷ: പര:\\
മന: കൃതം കൃതം ലോകേ ന ശരീരകൃതം  കൃതം (3/89/1)\\
}

സൂര്യന്‍ തുടര്‍ന്നു: "മനസ്സുതന്നെയാണ്‌ ലോക സൃഷ്ടാവ്‌.. മനസ്സു തന്നെയാണ്‌ പരമപുരുഷന്‍.. മനസ്സിനാല്‍ ചെയ്യപ്പെടുന്നതാണു കര്‍മ്മം. ശരീരംകൊണ്ടു ചെയ്യുന്നത്‌ കര്‍മ്മമല്ല." മനസ്സിന്റെ ശക്തി നോക്കൂ! ദൃഢമായ ചിന്തകൊണ്ട്‌ മഹാത്മാവിന്റെ പുത്രരായ ആ പത്തുപേര്‍ സൃഷ്ടാക്കളായി. എന്നാല്‍ ഏതൊരുവന്‍ 'ഞാനീ ചെറിയ ശരീരമാണ്‌' എന്നു ചിന്തിക്കുന്നുവോ അവനു മൃത്യു സുനിശ്ചയമാണ്‌.. ഒരുവന്റെ ബോധം ബാഹ്യലോകത്തേക്ക്‌ ഉന്മുഖമാകുമ്പോള്‍ സുഖദു:ഖങ്ങളെന്ന ദ്വന്ദങ്ങളുണ്ട്‌.. എന്നാല്‍ യോഗിയുടെ ദൃഷ്ടി ഉള്ളിലേയ്ക്കാണ്‌.. അവിടെ സുഖദു:ഖങ്ങള്‍ എന്ന ധാരണകള്‍ ഇല്ല. ഇതിനെക്കുറിച്ചുള്ള ഒരു കഥ ഞാന്‍ പറയാം.

മഗധ രാജ്യത്ത്‌ ഇന്ദ്രദ്യുമ്നന്‍ എന്നുപേരായ ഒരു രാജാവു വാണിരുന്നു. അദ്ദേഹത്തിന്റെഭാര്യ അഹല്യ. ആ സ്ഥലത്ത്‌ ഇന്ദ്രന്‍ എന്നു പേരായി ദുര്‍മ്മാര്‍ഗ്ഗിയെങ്കിലും സുന്ദരനായ ഒരു ചെറുപ്പക്കാരന്‍ ഉണ്ടായിരുന്നു. ദേവേന്ദ്രന്‍ മുനിപത്നിയായ അഹല്യയെ വശീകരിച്ച കഥ അഹല്യാ റാണി ഒരുദിവസം പ്രഭാഷണമദ്ധ്യേ കേട്ടു. അതുകേട്ട്‌ രാജ്ഞിക്ക്‌ ഇന്ദ്രന്‍ എന്ന ചെറുപ്പക്കാരനോട്‌ പ്രേമം തോന്നി. പ്രേമം മൂത്ത്‌ തന്റെ തോഴിമാരുടെ സഹായത്തോടെ അവള്‍ ഇന്ദ്രനെ തന്റെ അരമനയിലേയ്ക്ക്‌ കൊണ്ടുവന്നു. തുടര്‍ന്ന് അവരിരുവരും രഹസ്യമായി സന്ധിച്ചു സുഖിച്ചു വന്നു. അഹല്യയ്ക്ക്‌ ഇന്ദ്രനെപ്പറ്റിയല്ലാതെ മറ്റൊരു ചിന്തയുമില്ലായിരുന്നു. അതുകൊണ്ടവള്‍ നോക്കുന്നിടത്തൊക്കെ ഇന്ദ്രനെക്കണ്ടു. അവനെക്കുറിച്ചുള്ള ചിന്തകള്‍ അവളുടെ മുഖത്തെ പ്രഫുല്ലമാക്കി. അവരുടെ പ്രേമം മൂത്തപ്പോള്‍ ജനമറിഞ്ഞു; രാജാവിന്റെ ചെവിയിലും കാര്യമെത്തി. ക്രുദ്ധനായ രാജാവ്‌ അവരെ ശിക്ഷിക്കാനായി പലതുംചെയ്തു. തണുത്ത വെള്ളത്തില്‍ അവരെ മുക്കി; തിളച്ച എണ്ണയിലവരെ വറുത്തു; ആനയുടെ കാലുകളില്‍ ബന്ധിച്ചു; ചാട്ടവാറുകൊണ്ടടിച്ചു. ഇന്ദ്രന്‍ പൊട്ടിച്ചിരിച്ചുകൊണ്ട്‌ രാജാവിനോടു പറഞ്ഞു: എനിക്കീ ലോകം മുഴുവനും എന്റെ പ്രിയപ്പെട്ടവള്‍ - അഹല്യയല്ലാതെ മറ്റൊന്നുമല്ല. ഈ ശിക്ഷകളൊന്നും ഞങ്ങളെ എശുകയില്ല. ഞാന്‍ മനസ്സുമാത്രമാണ്‌.. മനസ്സാണ്‌ വ്യക്തി. നിങ്ങള്‍ക്ക്‌ ശരീരത്തെ ശിക്ഷിക്കാം; എന്നാല്‍ നിങ്ങള്‍ക്ക്‌ മനസ്സിനെ ശിക്ഷിക്കാനോ ചെറുതായിപ്പോലും മാറ്റാനോ കഴിയില്ല. മനസ്സ്‌ എന്തിലെങ്കിലും ആമഗ്നമായിരിക്കുമ്പോള്‍ ശരീരത്തിനെന്തു സംഭവിച്ചാലും മനസ്സിനെയത്‌ ബാധിക്കുന്നില്ല.

മനസ്സിന്‌ ശാപത്താലോ അനുഗ്രഹത്താലോ ചഞ്ചല്യമുണ്ടാവുന്നില്ല. വലിയൊരു മാമല കേവലം ചെറിയ വന്യജീവികളുടെ കൊമ്പുകൊണ്ട്‌ കുത്തിയിളക്കാനാവുകയില്ലല്ലോ. ശരീരമല്ല മനസ്സിനെയുണ്ടാക്കുന്നത്‌, മറിച്ച്‌ മനസ്സാണ്‌ ശരീരത്തെ സൃഷ്ടിക്കുന്നത്‌.. മനസ്സുമാത്രമാണ്‌ ശരീരത്തിന്റെ വിത്ത്‌.. മരം മരിക്കുമ്പോഴും വിത്ത്‌ നശിക്കുന്നില്ല. എന്നാല്‍ വിത്ത്‌ നശിക്കുമ്പോള്‍ അതിലെ വൃക്ഷവും നശിക്കുന്നു. ശരീരം നശിച്ചാല്‍ മനസ്സിന്‌ സ്വയം മറ്റൊരു ശരീരം സൃഷ്ടിക്കുവാനാവും. 

