 
\section{ദിവസം 112}

\slokam{
പ്രതിഭാസമുപായാതി യഥ്യദസ്യ  ഹി ചേതസ:\\
തത്തത്‌പ്രകടതാമേതി സ്ഥൈര്യം സഫലതാമപി  (3/91/17)\\
}

സൂര്യന്‍ തുടര്‍ന്നു: ഭഗവാനേ, അതുകഴിഞ്ഞ്‌ രാജാവ്‌ ഭരതമുനിയെ സമീപിച്ച്‌ ശാരീരികശിക്ഷകള്‍ക്കൊന്നും വഴങ്ങാത്ത ഈ കമിതാക്കള്‍ക്കുചിതമായ ശിക്ഷയായി ഒരു ശാപം നല്‍കണമെന്ന് അഭ്യര്‍ത്ഥിച്ചു. മുനി അവരെ ശപിച്ചു. പക്ഷേ അവര്‍ മുനിയോടു പറഞ്ഞു: കഷ്ടം! നിങ്ങള്‍ക്കു രണ്ടാള്‍ക്കും ബുദ്ധി അല്‍പ്പം കുറവുണ്ട് എന്ന് തോന്നുന്നു. ഞങ്ങളെ ശപിച്ചതുകൊണ്ട്‌ നിങ്ങളാര്‍ജ്ജിച്ച തപോബലത്തിനു കുറവുണ്ടായിരിക്കുന്നു എന്നതാണ് സത്യം. ആ ശാപം കൊണ്ട്‌ ഞങ്ങളുടെ ശരീരത്തെ നശിപ്പിക്കാം; എന്നാല്‍ ഞങ്ങള്‍ക്കതുകൊണ്ട്‌ ഒന്നും നഷ്ടമാവുന്നില്ല. ഞങ്ങളുടെ മനസ്സിനെ ആര്‍ക്കും നശിപ്പിക്കാനാവില്ല.

മുനിശാപം മൂലം അവരുടെ ദേഹം നശിച്ചു. ഈ ശരീരത്തില്‍നിന്നും പിന്നിടവര്‍ മൃഗങ്ങളായും പക്ഷികളായും ജന്മമെടുത്തു. അവസാനം ഉത്തമമായ ഒരു ഗൃഹത്തില്‍ മനുഷ്യ ദമ്പതികളായി. ഇപ്പോഴും പരസ്പരമുള്ള സമ്പൂര്‍ണ്ണാനുരാഗത്തിന്റെ ശക്തികൊണ്ട്‌ അവര്‍ ഭാര്യാ ഭര്‍ത്താക്കന്മാരായി ജന്മമെടുത്തിരിക്കുന്നു. കാട്ടിലെ മരങ്ങള്‍പോലും ഇവരുടെ പ്രേമവായ്പ്പുകണ്ട്‌ പ്രചോദിതരായി. മുനിശാപംകൊണ്ടും ഇവരുടെ മനസ്സു നശിപ്പിക്കാന്‍ കഴിഞ്ഞില്ല. അതുപോലെ, ഭഗവാനേ ഈ പത്തുപേരുടെ സൃഷ്ടികളെ യാതൊരുവിധത്തിലും സ്വാധീനിക്കാന്‍ അങ്ങേയ്ക്കാവില്ല. അവര്‍ സ്വയം സൃഷ്ടിസര്‍ഗ്ഗത്തില്‍ തുടരുന്നതുകൊണ്ട്‌ അങ്ങേയ്ക്കെന്താണു നഷ്ടം? അവര്‍ അവരുടെ മനസ്സിന്റെ സൃഷ്ടികളെ തുടരട്ടെ. സ്ഫടികത്തില്‍ നിന്നും  അതിനുള്ളിലെ പ്രതിഫലനം മാത്രമായി നീക്കി മാറ്റാനരുതാത്തതുപോലെയാണത്‌. ഭഗവന്‍, അങ്ങയുടെ ബോധമണ്ഡലത്തില്‍ ,  സ്വാഭീഷ്ടമനുസരിച്ച്  ഒരു ലോകം സ്വയം സൃഷ്ടിച്ചാലും.

സത്യത്തില്‍ അനന്താവബോധവും മനസ്സും (ഒരുവന്റെ ബോധം) അനന്തമായ ആകാശവുമെല്ലാം ഒരേയൊരു വസ്തുവാണ്‌.. അനന്താവബോധം സര്‍വ്വവ്യാപിയാണ്‌. . അതുകൊണ്ടീചെറുപ്പക്കാര്‍ സൃഷ്ടിച്ച ലോകങ്ങളെന്തായാലും അങ്ങേക്കിഷ്ടം പോലെയുള്ള  മറൊരു ലോകനിര്‍മ്മിതി ചെയ്താലും.

ബ്രഹ്മാവ്‌ വസിഷ്ഠനോടു പറഞ്ഞു: അപ്രകാരമുള്ള സൂര്യവചനം കേട്ട്‌ എന്റെ സ്വാഭാവപ്രകടനം എന്ന നിലയില്‍ ഞാന്‍ ലോകങ്ങളെ ഉണ്ടാക്കാന്‍ തുടങ്ങി. ഈ ഉദ്യമത്തില്‍ ആദ്യത്തെ പങ്കാളിയാകാന്‍ ഞാന്‍ സൂര്യനെ ക്ഷണിച്ചു. ഈ സൂര്യന്‍ ചെറുപ്പക്കാരുടെ സൃഷ്ടിയില്‍ പങ്കെടുത്തതുകൂടാതെ എന്റെ സൃഷ്ടിയില്‍ മനുഷ്യകുലത്തിന്റെ പൂര്‍വ്വികനായ പ്രജാപതിയായും വര്‍ത്തിച്ചു. സൂര്യന്‍ ഈ രണ്ടു ജോലികളും ഭംഗിയായി ചെയ്തു. എന്റെ ഉദ്ദേശങ്ങള്‍ക്കനുസൃതമായി അദ്ദേഹം ഈ ലോകനിര്‍മ്മിതി നടത്തി. "എന്തൊക്കെ ഒരാളുടെ ബോധമണ്ഡലത്തില്‍ പ്രത്യക്ഷമാവുന്നുവോ അതുണ്ടാവുന്നു, നിലനില്‍ക്കുന്നു, ഫലപ്രാപ്തിയുമുണ്ടാവുന്നു. അതാണ്‌ മനസ്സിന്റെ ശക്തി!"

ആ മഹാത്മാവിന്റെ പുത്രന്മാര്‍ സ്വമന:ശ്ശക്തികൊണ്ട്‌ ലോകസൃഷ്ടാക്കളുടെ സ്ഥാനത്തെത്തിയപോലെ ഞാനും ലോകസൃഷ്ടാവായി. മനസ്സാണ്‌ ഇവിടെ വസ്തുക്കളെ പ്രകടമാക്കുന്നത്‌...  മനസ്സില്‍ത്തന്നെയാണ് ശരീരബോധമുണ്ടാവുന്നതും ശരീരമെന്ന 'കാഴ്ച്ച' പ്രകടമാവുന്നതും. 
