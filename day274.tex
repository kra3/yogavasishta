\section{ദിവസം 274}

\slokam{
ക്ഷീയതേ മനസി ക്ഷീണേ ദേഹ: പ്രക്ഷീണവാസന:\\
മനോ ന ക്ഷീയതേ ക്ഷീണേ ദേഹേ തത്ക്ഷപയേന്മന: (5/53/66)\\
}

ഉദ്ദാലകന്‍ തന്റെ മനനം തുടര്‍ന്നു: മനസ്സ് സ്വയം ശരീരത്തില്‍ നിന്നും വിഭിന്നമായി കാണുന്നതോടെ അത് തന്റെ ഉപാധികളും ധാരണകളും ഉപേക്ഷിക്കുന്നു. തന്റെ ക്ഷണികാസ്ഥിത്വത്തെപ്പറ്റി ബോധമുണ്ടായി വിജയിക്കുന്നു. മനസ്സും ശരീരവും പരസ്പരം ശത്രുതയിലാണ്. അവയുടെ അന്ത്യത്തോടെ പരമാനന്ദം സാദ്ധ്യമാവുന്നു. അവര്‍ രണ്ടുമുള്ളപ്പോള്‍ പരസ്പരമുള്ള കലഹം മൂലം, എണ്ണമറ്റ ദുരിതാനുഭവങ്ങളുണ്ടാവുന്നു. മനസ്സാണ് സ്വന്തം ചിന്താശക്തിയിലൂടെ ശരീരത്തിന് ജന്മം നല്‍കുന്നത്. ഈ ശരീരത്തിന് തുടര്‍ച്ചയായി ദു:ഖങ്ങള്‍ നല്‍കിക്കൊണ്ടിരിക്കുന്നതും മനസ്സ് തന്നെയാണ്. ഇങ്ങിനെ കഷ്ടപ്പെടുത്തുന്നതുകൊണ്ട് ശരീരം മനസ്സിനെ (അതായത് സ്വന്തം സൃഷ്ടാവിനെ) ഇല്ലായ്മചെയ്യാന്‍ ഇച്ഛിക്കുന്നു.     

ഈ ലോകത്തില്‍ ആരും സുഹൃത്തായോ ശത്രുവായോ ഇല്ല. കാരണം താല്‍ക്കാലികമായി ആരാണോ സുഖാനുഭവങ്ങള്‍ നല്‍കുന്നത്, അവനാണ് സുഹൃത്ത്. ദു:ഖാനുഭവദാദാക്കള്‍ ശത്രുക്കളും.മനസ്സും ശരീരവും നിരന്തരമിങ്ങിനെ പരസ്പരം പോരടിച്ചു കൊണ്ടിരിക്കുമ്പോള്‍ എങ്ങിനെയാണ് സുഖമുണ്ടാവുക? മനസ്സിനെ ഇല്ലാതാക്കിയാലാണ് സുഖമുണ്ടാവുക എന്നതുകൊണ്ട് ശരീരം എല്ലാ ദിവസവും ദീര്‍ഘനിദ്രയില്‍ അതിനായി പരിശ്രമിക്കുന്നു. എന്നാല്‍ ആത്മജ്ഞാനമുണ്ടാവുന്നത് വരെ ശരീരവും മനസ്സും പരസ്പരം പ്രബലപ്പെടുത്തി ഒരേ ലക്ഷ്യത്തോടെ നീങ്ങുന്നതായാണു കാണുന്നത്. ജലവും അഗ്നിയും പരസ്പര വിരുദ്ധമാണെങ്കിലും അവ ഒരുമിച്ചു പ്രവര്‍ത്തിക്കുന്നത് പാചകത്തിന്റെ കാര്യത്തില്‍ നമുക്കനുഭവമാണല്ലോ.

“മനസ്സൊടുങ്ങുമ്പോള്‍ ശരീരവുംഇല്ലാതാവുന്നു. അത് സംഭവിക്കുന്നത് ചിന്താശക്തിയും മനോപാധികളും നിലയ്ക്കുന്നതുകൊണ്ടാണ്. എന്നാല്‍ ശരീരമില്ലാതാവുമ്പോള്‍ മനസ്സ് നിലയ്ക്കുന്നില്ല. അതിനാല്‍ മനസ്സിനെ ഇല്ലാതാക്കാനാണ് സാധകന്‍ ശ്രമിക്കേണ്ടത്.”

 ചിന്താശതങ്ങളാകുന്ന മരങ്ങള്‍ നിറഞ്ഞ കാടാണ് മനസ്സ്. അതിലെ വള്ളിച്ചെടികളാണാസക്തികള്‍ .  അതിനെ ഇല്ലാതാക്കിയാലെനിക്ക് നിത്യാനന്ദം പ്രാപിക്കാം. മനസ്സ് മരിച്ചുകഴിഞ്ഞ് മാംസാസ്ഥിരക്തസഞ്ചയമായ ഈ ശരീരം നിലനിന്നാലുമില്ലെങ്കിലും എനിക്കൊന്നുമില്ല. കാരണം ‘ഞാന്‍’ ഈ ശരീരമല്ല എന്നുറപ്പ്. ശവശരീരം പ്രവര്‍ത്തനക്ഷമമല്ലല്ലൊ. ആത്മജ്ഞാനമുള്ളിടത്ത് മനസ്സും ഇന്ദ്രിയങ്ങളുമില്ല. ധാരണകളും ആശാസങ്കല്‍പ്പങ്ങളും ഇല്ല. ഞാനാ പരമപദത്തിലെത്തിയിരിക്കുന്നു. ഞാന്‍ വിജയിയായിരിക്കുന്നു. എനിക്ക് മോക്ഷമായിരിക്കുന്നു. ഞാന്‍ നിര്‍വാണപദം പ്രാപിച്ചിരിക്കുന്നു.
  
എണ്ണക്കുരുവില്‍നിന്നും ആട്ടിയെടുത്ത എണ്ണയ്ക്ക് ആ കുരുവുമായി പിന്നെ യാതൊരു ബന്ധവുമില്ലാത്തതുപോലെ മനോബുദ്ധിന്ദ്രിയങ്ങളുമായുള്ള എന്റെ ബന്ധങ്ങളെല്ലാം അവസാനിച്ചിരിക്കുന്നു. ഇപ്പോളെനിക്ക് മനസ്സും ശരീരവും ഇന്ദ്രിയങ്ങളും വെറും കളിപ്പാട്ടങ്ങളാണ്.
നിര്‍മലത, ആശകളുടെ പൂര്‍ണ്ണസഫലത (അതിനാല്‍ത്തന്നെ ആശകളില്ലാത്ത അവസ്ഥ), സൌഹൃദം, സത്യം, വിവേകം, പ്രശാന്തത, മധുരഭാഷണം, മഹത്വം, പ്രഭ, എകാത്മകത, അവികലവും അവിഛിന്നവുമായ അനന്താവബോധം എന്നിവയാണ് എന്റെ സന്തതസഹചാരികള്‍ . എല്ലായ്പ്പോഴും എല്ലാം എവിടെയും ഏതുവിധേനെയും നടക്കുന്നത് എന്നിലായതിനാല്‍ പ്രീതികരമോ അപ്രീതികരമോ ആയ ഒന്നിനോടും എനിക്കാമുഖ്യമോ വിരോധമോ ഇല്ല. മനസ്സൊടുങ്ങി, ഭ്രമങ്ങളടങ്ങി, ദുഷ്ചിന്തകളില്ലാതെ ഞാനെന്നില്‍ത്തന്നെ പരമപ്രശാന്തിനാണിപ്പോള്‍.