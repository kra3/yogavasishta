\section{ദിവസം 247}

\slokam{
സംബന്ധ: കോഽസ്തു ന: കാമൈര്‍ഭാവാഭാവൈരഥേന്ദ്രിയൈ:\\
കേന സംബദ്ധ്യതേ വ്യോമ കേന സംബാദ്ധ്യതേ മന:    (5/35/32)\\
}

പ്രഹ്ലാദന്‍ ധ്യാനം തുടര്‍ന്നു: ആത്മാവ് ആകാശത്തിലെ നിശ്ശൂന്യതയാണ്. ചരവസ്തുക്കളുടെ ചലനമാണത്. ഭാസുരവസ്തുക്കളിലെ പ്രഭയാണത്. ദ്രവവസ്തുക്കളിലെ സ്വാദാണത്. ഭൂമിയുടെ ദൃഢതയും അഗ്നിയിലെ ചൂടും ചാന്ദ്രശീതളിമയും എന്നുവേണ്ട ഈ ലോകത്തിന്റെ അസ്തിത്വം തന്നെ ആത്മാവിലാണ്. 

പദാര്‍ത്ഥങ്ങളിലെ സഹജഗുണങ്ങള്‍പോലെ ദേഹത്തിന്റെ നാഥനാണാത്മാവ്. എല്ലായിടത്തുമുള്ള ‘അസ്തിത്വ’ത്തിന്റെ നിലനില്‍പ്പുപോലെ, എല്ലാക്കാലവും സമയത്തിനു നിലനില്‍പ്പുള്ളതുപോലെ, ആത്മാവ് എല്ലാ ദേഹങ്ങളിലും ദേഹിയായി ശാരീരികവും മാനസീകവുമായ എല്ലാ പ്രഭാവങ്ങളോടും കൂടി നിലകൊള്ളുന്നു. ആത്മാവ്, ശാശ്വതമായ അസ്തിത്വമാണ്. ദേവതകളെപ്പോലും പ്രബുദ്ധരാകുന്ന ചൈതന്യവിശേഷം.

ഞാന്‍ ആത്മാവ് മാത്രമാണ്. എന്നില്‍ മറ്റുധാരണകളോ വിഷയങ്ങളോ ഇല്ല. പാറിനടക്കുന്ന പൊടിപടലങ്ങള്‍ ആകാശത്തെ ബാധിക്കാത്തതുപോലെ ജലത്തില്‍ വളരുന്ന ചെടിയാണെങ്കിലും താമരയിതളുകളെ ജലം നനയ്ക്കാത്തതുപോലെ എന്നെ ഒന്നും ബാധിക്കുന്നില്ല. ശരീരം സന്തോഷസന്താപങ്ങളെ അനുഭവിച്ചോട്ടെ. അതുകൊണ്ട് ആത്മാവിനെന്തു ചേതം? ചരടുകൊണ്ടുണ്ടാക്കിയ തിരിയാണ് വിളക്കിലെരിയുന്നതെങ്കിലും തീനാളത്തെ തളച്ചിടാന്‍ ആ ചരടിനാവുമോ? യാതൊരു പദാര്‍ത്ഥങ്ങള്‍ക്കും ആത്മാവിനെ ബാധിക്കുവാനാവില്ല. ആത്മാവ് വസ്തു പരിമിതികള്‍ക്കെല്ലാം  അതീതമത്രേ.
   
“അസ്തിത്വ-അനസ്തിത്വ ധാരണകളില്‍ നിന്നും ഇന്ദ്രിയങ്ങളില്‍ നിന്നും സംജാതമാവുന്ന ആസക്തികളും ആത്മാവും തമ്മിലെങ്ങിനെ, എന്ത്  ബന്ധമുണ്ടാവാനാണ്? ആര്‍ക്കാണ് ആകാശത്തെ ബന്ധിക്കാനാവുക? ആര്‍ക്കാണ് മനസ്സിനെ തളയ്ക്കാനാവുക?” ശരീരത്തെ നൂറായി നുറുക്കിയാലും ആത്മാവിനെ അത് ബാധിക്കില്ല. മണ്‍കുടത്തെ ഉടച്ചുപൊടിച്ചാലും കുടത്തിനുള്ളിലെ  ആകാശത്തിനെന്തു നാശമാണുണ്ടാവുക? ഈ മനസ്സെന്ന പിശാച് വെറും വാക്കുകളാല്‍ മാത്രം നിലകൊള്ളുന്ന വെറും അസത്താണ്. അത് നശിച്ചാല്‍ എന്ത് നഷ്ടപ്പെടാനാണ്?

എന്നില്‍ സന്തോഷസന്താപധാരണകളാല്‍ ചാഞ്ചാടിക്കൊണ്ടിരുന്ന ഒരു മനസ്സുണ്ടായിരുന്നു. പക്ഷെ ഇപ്പോളാ ധാരണകള്‍ നശിച്ചിരിക്കുന്നുവെങ്കില്‍ എന്റെ മനസ്സെവിടെപ്പോയി? ‘ഒന്ന് മറ്റൊന്നിനെ ആസ്വദിക്കുന്നു’ എന്നും ‘ഒരാള്‍ മറ്റൊരാളെ കാണുന്നു’ എന്നും ‘ഒരാള്‍ ദുരിതമനുഭവിക്കുന്നു’ എന്നും മറ്റുമുള്ള ധാരണകള്‍ ഏതു മൂഢനാണ് വെച്ചുപുലര്‍ത്തുക? ആസ്വാദനം പ്രകൃതിക്കാണ്. മനസ്സാണ് 'ഗ്രഹിക്കുന്നതും' ധാരണകളെയുണ്ടാക്കുന്നതും. ദുരിതാനുഭവങ്ങള്‍ ദേഹത്തിനാണ്. ദുഷ്ടനായവന്‍ വാസ്തവത്തില്‍ വിഡ്ഢിതന്നെയാണ്. എന്നാല്‍ മുക്തനോ ഈ പ്രശ്നങ്ങളൊന്നുമില്ല.

എനിക്ക് സുഖത്തിനായുള്ള ആസക്തിയോ സുഖാനുഭാവങ്ങളോടു വേറുപ്പോ ഇല്ല. വരുന്നത് വരട്ടെ. പോകുന്നത് പോയ്ക്കൊള്ളട്ടെ*. വൈവിദ്ധ്യമാര്‍ന്ന അനുഭവധാരണകള്‍ എന്നിലുണരാതെയിരിക്കട്ടെ. എന്റെ ദേഹത്തിലവ അടിഞ്ഞുകൂടാതെയിരിക്കട്ടെ. ഞാനവയിലോ അവയില്‍ ഞാനോ ഇല്ല. അജ്ഞാനമെന്ന ശത്രുവിന്റെ വരുതിയില്‍പ്പെട്ട് അതിന്റെ അടിമയായി എന്റെ വിജ്ഞാനവിവേകങ്ങള്‍ നഷ്ടപ്പെട്ടിരുന്നു. എന്നാലിപ്പോള്‍ തീവ്രമായ സ്വപ്രയത്നത്താലും വിഷ്ണുകൃപയാലും എന്നില്‍ വിവേകമുദിച്ചിരിക്കുന്നു. അത്മജ്ഞാനമെന്ന മാസ്മരീക ശക്തിയാല്‍ അഹംകാരമെന്ന പിശാചിനെ ഞാന്‍ ഓടിച്ചിരിക്കുന്നു. ഭ്രമകല്‍പ്പനകള്‍ ഇല്ലാത്ത ഞാന്‍ പരമദൈവതം തന്നെയാണ്. ഞാന്‍ അറിയേണ്ടതെല്ലാം അറിഞ്ഞുകഴിഞ്ഞു. കാണാന്‍ യോഗ്യമായതെല്ലാം കണ്ടും കഴിഞ്ഞു. എല്ലാത്തിലും അതീതമായി, എത്താനൊരിടമില്ലാത്ത, കാലദേശാനുസാരിയല്ലാത്ത  ആ പരമോന്നതതലത്തില്‍ ഞാനെത്തിക്കഴിഞ്ഞിരിക്കുന്നു.

*ആയാതമായതമലംഘനീയം ഗതം ഗതം സര്‍വ്വ മുപേക്ഷണീയം.