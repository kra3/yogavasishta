\section{ദിവസം 221}

\slokam{
സർവത്ര വാസനാത്യാഗോ രാമ രാജീവലോചന\\
ദ്വിവിധ: കഥ്യതേ തജ്ഞൈർജ്ഞേയോ ധ്യേയശ്ച മാനദ (5/16/6)\\
}

രാമൻ പറഞ്ഞു: ഭഗവൻ, അങ്ങെന്നോട് അഹംകാരവും  അതുണ്ടാക്കുന്ന എല്ലാ ത്വരകളേയും ഉപേക്ഷിക്കാൻ പറഞ്ഞുവല്ലോ. അപ്പോൾപ്പിന്നെ ഞാൻ അഹംകാരം ഉപേക്ഷിക്കുന്നതിനോടൊപ്പം എന്റെ ശരീരവും അഹംകാരജന്യങ്ങളായ മറ്റെല്ലാ കാര്യങ്ങളും ഉപേക്ഷിക്കണമല്ലോ? ഈ ശരീരവും പ്രാണനുമെല്ലാം അഹംകാരത്തിനെ ആധാരമായാണല്ലോ സ്ഥിതിചെയ്യുന്നത്? വേര്‌ (അഹംകാരം) അറുത്താൽപ്പിന്നെ മരം (ദേഹാദികൾ) വീഴുമെന്നുറപ്പ്. അഹംകാരത്തെ ഉപേക്ഷിച്ച് എനിക്കെങ്ങിനെ ജീവിതം നയിക്കാൻ കഴിയും ?

വസിഷ്ഠൻ പറഞ്ഞു: “രാമാ, ധാരണകളേയും ഉപാധികളേയും ആശയസങ്കൽപ്പങ്ങളേയും ഉപേക്ഷിക്കുന്നത് രണ്ടു വിധത്തിലാണ്‌.. ഒന്ന് നേരറിവിന്റെ, അതായത് സാക്ഷാത്കാരത്തിന്റെ നിറവിലും മറ്റേത് ധ്യാനത്തിന്റെ മാർഗ്ഗത്തിലുമാണ്‌ സാധിക്കുക.” അവയെ ഞാൻ വിശദമായി പറഞ്ഞു തരാം. ‘ഞാനീ വിഷയവസ്തുക്കളുടെ അധീനതയിലാണ്‌, ഞാൻ അവയെ ആശ്രയിച്ചാണു നിലകൊള്ളുന്നത്' എന്നും മറ്റുമുള്ള ചിന്ത  വെറും ഭ്രമകൽപ്പനയാണെന്ന് ഒരുവൻ തികച്ചും ബോധവാനായിരിക്കണം. എന്നാല്‍ അവയെക്കൂടാതെ 'എനിയ്ക്കു' ജീവിക്കാൻ സാദ്ധ്യമല്ല എന്നും ഈ വസ്തുക്കൾക്ക് എന്നെക്കൂടാതെ ഒരസ്തിത്വവുമില്ലെന്നും അയാളറിയണം. എന്നിട്ട് തീക്ഷ്ണമായ ധ്യാനസപര്യയിലൂടെ ’ഞാനീ വസ്തുക്കളുടെ ഉടമസ്ഥനല്ല; അവ എന്റെ ഉടമസ്ഥരുമല്ല' എന്ന തിരിച്ചറിവിലേക്ക് എത്തിച്ചേരണം.

അങ്ങിനെ അഹംകാരത്തെ തീവ്രസാധനയിലൂടെ ഉപേക്ഷിച്ചശേഷം ഒരു ലീലയെന്നപോലെ ഒരുവൻ തനിക്കു സഹജമായി വന്നുചേരുന്ന കർമ്മങ്ങൾ ഭംഗിയായി ചെയ്യണം. ഈ കർമ്മങ്ങൾ സ്വാഭാവികമായി അനുഷ്ഠിക്കുമ്പോൾ ഹൃദയവും മനസ്സും പ്രശാന്തശീതളമായിരിക്കും. അങ്ങിനെ ധ്യാനസാധനകൊണ്ട് അഹംകാരം ഇല്ലാതാകുന്ന ഒരവസ്ഥ സംജാതമാകും. ഇനി അദ്വൈത സത്യത്തിന്റെ നേരറിവിൽ ഒരുവൻ അഹംകാരത്തെ ഉപേക്ഷിക്കുന്ന രീതി - ദേഹാഭിമാനമോ, ’ഇതെന്റേത്‘ എന്ന തോന്നലോ ഇല്ലാത്ത ഒരവസ്ഥയാണിത്.

ജീവിച്ചിരിക്കുമ്പോൾത്തന്നെ ധ്യാനസപര്യകൊണ്ട് വെറുമൊരു ലീലപോലെ അഹംകാരത്തെ ഇല്ലാതാക്കി വർത്തിക്കുന്നവൻ ജീവന്മുക്തനത്രേ. അതുപോലെ നേരറിവിന്റെ നിറവിൽ അഹംകാരത്തെ വേരോടെ പിഴുതുകളഞ്ഞവനും മുക്തനാണ്‌.. ജനകനെപ്പോലുള്ളവർ ധ്യാനമാർഗ്ഗികളാണ്‌.. നേരറിവായി ബ്രഹ്മസാക്ഷാത്കാരം നേടിയവർ ദേഹബോധത്തിനതീതരാണ്‌.. ഇവിടെപ്പറഞ്ഞ രണ്ടു മാർഗ്ഗത്തിൽ സഞ്ചരിച്ചു ലക്ഷ്യം നേടിയവരും ബ്രഹ്മലീനരത്രേ. ഇഷ്ടാനിഷ്ടങ്ങളാൽ ചഞ്ചലപ്പെടാത്തെ, ഈ ലോകത്തിൽ എല്ലാ കർമ്മധർമ്മങ്ങളോടുംകൂടി ജീവിക്കുമ്പോഴും അകമേ യാതൊരു വിഷയങ്ങളും ബാധിക്കാതെ, ദീർഘനിദ്രയിലെന്നപോലെ വർത്തിക്കുന്നവനാണ്‌ മുക്തൻ.

വസിഷ്ഠമുനി ഇത്രയും പറഞ്ഞു നിർത്തിയപ്പോഴേയ്ക്കും മറ്റൊരു ദിനം കൂടി അവസാനിച്ചു. സഭ പിരിഞ്ഞു. 
