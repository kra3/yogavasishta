\newpage
\section{ദിവസം 101}

\slokam{
സര്‍വ്വാത്മകത്വാന്നൈവാസൌ ശൂന്യോ ഭവതി കര്‍ഹിചിത്\\
യാദസ്തി ന തദസ്തീതി വക്താ മന്താ ഇതി സ്മൃത: (3/80/10)\\
}

മന്ത്രി മറുപടിയായി പറഞ്ഞു: തീര്‍ച്ചയായും ഞാന്‍ ഉത്തരം പറയാം. താങ്കളുടെ ചോദ്യങ്ങള്‍ എല്ലാം വിരല്‍ ചൂണ്ടുന്നത്‌ പരമാത്മാവിലേയ്ക്കാണ്‌...  ആ അത്മാവ്‌ ആകാശത്തേക്കാള്‍ സൂക്ഷ്മമാണ്‌. . അതിന്‌ നാമമില്ല; അതു വിവരണാതീതവുമാണ്‌.. മനസ്സേന്ദ്രിയങ്ങള്‍ക്ക്‌ അതിനെ സമീപിക്കാനോ അറിയാനോ കഴിയില്ല. അതു ശുദ്ധ ബോധമാണ്‌.. ഒരു വിത്തിനുള്ളില്‍ എങ്ങിനെയാണോ വലിയൊരു മരം അതിസൂക്ഷ്മായി കുടികൊള്ളുന്നത്‌, അപ്രകാരം ഈ വിശ്വം മുഴുവന്‍, ബോധമെന്ന അണുവിലാണുള്ളത്‌.. വിശ്വത്തിന്‌ ബോധമായിമാത്രമേ നിലനില്‍പ്പുള്ളു. ആ ബോധം മാത്രമേ ഉണ്മയായുള്ളു. എല്ലാവരുടേയും അനുഭവം വച്ചു നോക്കിയാലും ആ ബോധം തന്നെയാണ്‌ എല്ലാവരിലും കുടികൊള്ളുന്ന ആത്മാവ്‌.. അതുള്ളതുകൊണ്ട്‌ മറ്റ്‌ എല്ലാമുണ്ട്‌. .

അത്മാവ്‌ അകാശം പോലെ ശൂന്യമാണ്‌.. എന്നാല്‍ അത്‌ 'ഒന്നുമില്ലായ്മ' അല്ല. കാരണം അത്‌ ബോധമാണല്ലോ. മനസ്സേന്ദ്രിയങ്ങളാല്‍ അറിയാന്‍ കഴിയാത്ത ഒന്നായതിനാല്‍ അതിനെ ഉണ്ടെന്നും ഇല്ലെന്നും പറയാം. എല്ലാറ്റിന്റേയും ആത്മസത്തയാകയാല്‍ അത്‌ ആര്‍ക്കും 'അനുഭവിക്കാവുന്ന' ഒരു വസ്തുവല്ല. ഏകമാണെങ്കിലും അനേകം അണുക്കളില്‍ പ്രതിഫലിക്കുന്നതുകൊണ്ട്‌ അവയ്ക്കെല്ലാം ഉണ്മയുള്ളതുപോലെ തോന്നുന്നു. ഒന്ന് പലതായി കാണപ്പെടുന്നു. എന്നാല്‍ കാണപ്പെടുന്നവ സത്തായ വസ്തുവല്ല - സ്വര്‍ണ്ണാഭരണത്തില്‍ സ്വര്‍ണ്ണം മാത്രമാണു മൂല്യവസ്തു എന്നതുപോലെയാണത്‌. "അത്‌ നിശ്ശൂന്യ്മായ ഒന്നുമില്ലായ്മയല്ല, കാരണം അതെല്ലാവരുടേയും ആത്മാവാണ്‌.. അതങ്ങിനെയാണെന്നു പറയുന്നവന്റേയും അതംഗീകരിക്കാത്തവന്റേയും സത്ത ആത്മാവു തന്നെയാണ്‌.". അന്തരീക്ഷത്തില്‍ തങ്ങിനില്‍ക്കുന്ന കര്‍പ്പൂരഗന്ധത്തില്‍നിന്നും അതിന്റെ അസ്തിത്വം അനുഭവിക്കുന്നതുപോലെ അത്മാവിന്റെ അസ്തിത്വം അനുഭവിക്കാനാകും. അതുതന്നെയാണ്‌ എല്ലാവരുടേയും അത്മാവ്‌-. ബോധം. അതാണ്‌ പദാര്‍ത്ഥങ്ങളുടെ സഞ്ചയമായ ഈ ദൃശ്യപ്രപഞ്ചത്തിനു കാരണമാകുന്നതും. ഒഴുക്കുവെള്ളത്തിന്റെ സ്വഭാവമായ ചുഴികളും വെള്ളം തന്നെ; അതുപോലെ സഹജമായി, അനന്തമായ വിശ്വാവബോധസമുദ്രത്തിലെ ചുഴികളായി ത്രിലോകങ്ങള്‍ ഉണ്ടായിമാറി മറയുന്നു. 

ഈ ബോധം, മനസ്സേന്ദ്രിയങ്ങള്‍ക്ക്‌ അറിയാനാവാത്തതുകൊണ്ട്‌ നിശ്ശൂന്യമാണെന്നു തോന്നാമെങ്കിലും അതു ശരിയല്ല. ആത്മാന്വേഷണത്തിലൂടെ അതിനെ അറിയാമല്ലോ. ബോധമെന്നത്‌ ഭാഗിക്കാനരുതാത്തതായതുകൊണ്ട്‌ ഞാന്‍ നീയും; നീ, ഞാനുമാണ്‌. എന്നാല്‍ ആ അവിച്ഛിന്നബോധം ഞാനോ, നീയോ ഒന്നും ആയി മാറിയിട്ടില്ല. 'നീ', 'ഞാന്‍' എന്നിങ്ങനെയുള്ള തെറ്റിദ്ധാരണകള്‍ ഉപേക്ഷിച്ചാല്‍ ഉണരുന്ന അവബോധം, നീയോ ഞാനോ, മറ്റ്‌ എല്ലാമോ അല്ല. അവബോധം മാത്രമേ സത്തായിട്ടുള്ളു. അനന്തമായതുകൊണ്ട്‌ ആത്മാവ്‌ ചലിക്കുന്നുവെങ്കിലും ചലിക്കുന്നില്ല. എന്നാല്‍ ഓരോ അണുക്കളിലും അതു ദൃഢീകരിച്ചിരിക്കുന്നു. ആത്മാവ്‌ വരുന്നും പോവുന്നുമില്ല; കാരണം കാലവും ദൂരവും പോലുള്ള ആശയങ്ങള്‍ക്കുറവായതും ഈബോധം മാത്രമാണ്‌.. എല്ലാം ഉള്‍ക്കൊണ്ടിരിക്കേ ആത്മാവിന്‌ എങ്ങോട്ടുപോകാന്‍ കഴിയും? ഒരിടത്തുനിന്നും മറ്റൊരിടത്തേയ്ക്കു മാറ്റിയതുകൊണ്ട്‌ ഒരു കുടത്തിനകത്തെ ആകാശത്തിന്‌ സ്ഥലമാറ്റം ഉണ്ടാകുന്നില്ല. എല്ലാം സ്ഥിതിചെയ്യുന്നത്‌ ആകാശത്തിലാണല്ലോ. 

