\section{ദിവസം 200}

ഭാഗം അഞ്ച് - ഉപശമ പ്രകരണം ആരംഭം.

\slokam{
ഭോഗാസ്ത്യക്തും ന ശക്യന്തേ തത്യാഗേന വിനാ വയം\\
പ്രഭവാമോ ന വിപദാമഹോ സങ്കടമാഗതം (5/2/21)\\
}

വാല്‍മീകി തുടർന്നു: ദേവന്മാരും ഉപദേവന്മാരും മഹർഷിമാരും മറ്റു സഭാവാസികളും വസിഷ്ഠമുനിയുടെ വിവേകവിജ്ഞാനഭരിതമായ വാക്കുകൾ പരിപൂർണ്ണശ്രദ്ധയോടെ സാകൂതം കേട്ടിരുന്നു. ദശരഥരാജാവും മന്ത്രിമാരുമെല്ലാം തൽക്കാലത്തേയ്ക്ക് അവരുടെ ഔദ്യോഗീക ജോലികളും ഉല്ലാസപ്രവൃത്തികളും നിർത്തിവച്ച് മാമുനിയുടെ പ്രഭാഷണത്തിൽ നിന്നുള്ള അറിവുനേടാൻ ജാഗരൂകരായിരുന്നു. മദ്ധ്യാഹ്നസമയമറിയിക്കുന്ന ശംഖുനാദം മുഴങ്ങിയപ്പോൾ സഭാവാസികൾ ചെറിയൊരിടവേള എടുത്തു. വൈകുന്നേരങ്ങളിൽ പ്രഭാഷണം കഴിയുമ്പോൾ അവർ വിശ്രമത്തിനായി പിരിയുകയും ചെയ്തു. രാജാക്കന്മാരും മറ്റും സഭയിൽ നിന്നു മടങ്ങുമ്പോൾ അവരുടെ വസ്ത്രാഭരണങ്ങൾ മിന്നിത്തിളങ്ങി. പ്രപഞ്ചത്തിന്റെ ഒരു ചെറുപകർപ്പായി രാജസഭ കാണപ്പെട്ടു. സഭ പിരിഞ്ഞു കഴിഞ്ഞപ്പോൾ ദശരഥരാജാവ് മുനിമാർക്ക് അർഘ്യം നല്കി പൂജിച്ച് അവരുടെ അനുഗ്രഹാശിസ്സുകൾ നേടി. രാമനും അനുജന്മാരും മുനിമാരെ കൽക്കൽവീണു നമസ്കരിച്ചു. വസിഷ്ഠമുനിയോട് വിട വാങ്ങി അവർ വിശ്രമത്തിനായി അന്തപ്പുരത്തിലേക്കു പോയി. അന്നത്തെ പഠനം സമാപിച്ചു. രാത്രിയിൽ രാമനൊഴികെ എല്ലാവരും നിദ്രയിലാണ്ടു.

രാമൻ വസിഷ്ഠമുനിയുടെ വാക്കുകളോർമ്മിച്ചു ധ്യാനിച്ചു: ഈ ലോകമെന്ന പ്രകടനം എന്താണ്‌? ആരാണീ നാനാവിധത്തിൽ കാണപ്പെടുന്ന മനുഷ്യരും ജീവജാലങ്ങളും? എങ്ങിനെയാണവരിവിടെ കാണപ്പെടുന്നത്? എവിടെ നിന്നാണവർ വന്നെത്തിയത്? എങ്ങോട്ടാണവരുടെ യാത്ര? മനസ്സിന്റെ സ്വഭാവങ്ങളെന്തൊക്കെയാണ്‌? എങ്ങിനെയതിനെ പ്രശാന്തമാക്കാം? പ്രപഞ്ചത്തിലെ ചിന്താക്കുഴപ്പത്തിനു കാരണമായ മായ എങ്ങു നിന്നാണുത്ഭവിച്ചത്? എപ്പോഴാണതിനൊരറുതി വരിക? ഈ മായയുടെ അവസാനമെന്നത് അഭികാമ്യമോ അല്ലയൊ? അനന്താവബോധത്തിൽ എങ്ങിനെയാണ്‌ പരിമിതിയെന്ന ധാരണ ഉദിച്ചത്? ഇന്ദ്രിയ സംയമനത്തിനായും മനോ നിയന്ത്രണത്തിനായും വസിഷ്ഠമുനി ഉപദേശിച്ച മാർഗ്ഗനിർദ്ദേശങ്ങൾ എങ്ങിനെയാണു കൃത്യമായി നടപ്പാക്കേണ്ടത്?

”സുഖാസ്വാദനം ഉപേക്ഷിക്കുക അസാദ്ധ്യം. എന്നാൽ ദു:ഖനിവാരണത്തിന്‌ അത്തരം ആസ്വാദനം അവസാനിപ്പിക്കാതെ സാദ്ധ്യവുമല്ല. ഇതൊരു പ്രശ്നം തന്നെ!.“ എന്നാൽ മനസ്സ് ഇതിലെ വളരെ പ്രധാനപ്പെട്ട ഒരു ഘടകമാണ്‌.. ആയതിനാൽ മനസ്സ് ഒരിക്കൽ ലോകത്തിന്റെ മായികതയിലൊന്നും പെടാതെ പരമശാന്തി അനുഭവിച്ചു കഴിഞ്ഞാൽപ്പിന്നെ അതുപേക്ഷിച്ച് ഇന്ദ്രിയസുഖത്തിനു പിറകേ പായുകയില്ല. എന്റെ മനസ്സ് എന്നാണു നിർമ്മലമാവുക? എന്നാണതിനു പരമ്പൊരുളിൽ അഭിരമിക്കാനാവുക? സമുദ്രത്തിൽ അലകളടങ്ങുന്നതുപോലെ എന്റെ മനസ്സെന്നാണ്‌ അനന്തതയിൽ അഭിരമിച്ചു ശാന്തമാവുക? എന്നിൽ ആസക്തികളെന്നാണവസാനിക്കുക? സമതാദർശനമെന്ന അനുഗ്രഹത്തിനു ഞാൻ പാത്രമാവുന്നതെപ്പോൾ? ലൗകീകതയെന്ന ഈ തീവ്രജ്വരത്തിൽ നിന്നെനിക്കു മുക്തി എപ്പോഴാണുണ്ടാവുക? മനസ്സേ മാമുനിമാർ അരുളിത്തന്ന ജ്ഞാനത്തിന്റെ തെളിമയിൽ നീ ഉറച്ചുനിൽക്കുമോ? മേധാശക്തീ, നീ എന്റെ സുഹൃത്താണല്ലോ. വസിഷ്ഠമുനി തന്ന പാഠങ്ങൾ നമുക്ക് ധ്യാനിച്ചുറപ്പിക്കാം.  അങ്ങിനെ നമുക്ക് രണ്ടാൾക്കും ലൗകീകാസ്തിത്വത്തിൽ നിന്നും തത്ജന്യമായ ദു:ഖദുരിതങ്ങളിൽ നിന്നും രക്ഷനേടാം. 
