\newpage
\section{ദിവസം 017}

\slokam{
തരന്തി മാതംഗഘടാതരംഗം രണാംബുധിം യേ മയി തേ ന ശൂരാ:\\
സുരസ്ത ഏവേഹ മനസ്തരംഗം ദേഹേന്ദ്രിയാംബോധിമിമം തരന്തി (1/27/9)\\
}


രാമന്‍ തുടര്‍ന്നു: അങ്ങിനെ ബാല്യത്തിലും യൌവ്വനത്തിലും വാര്‍ദ്ധക്യത്തിലും മനുഷ്യന്‍ സുഖം അനുഭവിക്കുന്നില്ല. ഇഹലോകവസ്തുക്കള്‍ ഒന്നും ആര്‍ക്കും സൌഖ്യമേകാന്‍ ഉദ്ദേശിച്ചിട്ടുള്ളവയുമല്ല. മനസ്സ്‌ വൃഥാ സുഖം തേടി ഈ വസ്തുക്കളില്‍ അലയുന്നു എന്നു മാത്രം. ഇന്ദ്രി യസുഖങ്ങളുടെ പിടിയില്‍പ്പെടാത്ത അഹംകാരമുക്തന്‍ മാത്രമേ 'സുഖ'മായിരിക്കുന്നുള്ളു. അങ്ങിനെയുള്ളവര്‍ വളരെ വിരളമാണു താനും. 

"ശക്തിമത്തായ ഒരു സൈന്യത്തെ വിജയിച്ചവന്‍ എന്റെ കണക്കില്‍ വീരനായകനൊന്നുമല്ല. എന്നാല്‍ മനസ്സേന്ദ്രിയങ്ങളാകുന്ന സംസാരസാഗരത്തെ വെന്നവനാണ്‌ യഥാര്‍ത്ഥത്തില്‍ വീരന്‍".." 

ക്ഷണനേരത്തില്‍ നഷ്ടപ്പെടുന്ന നേട്ടം ഒരു നേട്ടമല്ല; സ്ഥിരമായി നിലനില്‍ക്കുന്നതിനെ സാക്ഷാത്കരിക്കുന്നതാണ്‌ യഥാര്‍ത്ഥ നേട്ടം. എത്ര കഠിനമായി അദ്ധ്വാനിച്ചാലും മനുഷ്യന്‌ അത്തരം ഒരു നേട്ടമുണ്ടാവുന്നില്ല. എന്നാല്‍ ക്ഷണികമായ നേട്ടങ്ങളും താല്‍ക്കാലീകമായ ബുദ്ധിമുട്ടുകളും ആവശ്യപ്പെടാതെതന്നെ അവനെ തേടിയെത്തുന്നു. പകല്‍ മുഴുവനും അവിടേയുമിവിടേയും കറങ്ങി നടന്ന് സ്വാര്‍ത്ഥകാര്യങ്ങളില്‍ തിരക്കുപിടിച്ചുഴറി ഒരിക്കല്‍പ്പോലും നന്മയുടെ വഴിക്കു തിരിയാത്ത ഒരുവന്‌ രാത്രിയില്‍ എങ്ങിനെ ഉറങ്ങാന്‍ കഴിയുന്നു എന്നു ഞാന്‍ അത്ഭുതപ്പെടുന്നു മഹര്‍ഷേ!.

തന്റെ ശത്രുക്കളെയെല്ലാം വെന്ന് സമ്പത്തിന്റേയും ആര്‍ഭാടങ്ങളുടേയും നടുവില്‍ സുഖിമാന്‍ എന്നു സ്വയം വിശേഷിപ്പിച്ച്‌ കഴിയുന്നവനേയും മരണം പിടികൂടുന്നതെങ്ങിനെയെന്ന് ഈശ്വരനേ അറിയൂ. ഈ ലോകമെന്നത്‌ പരിചയമില്ലാത്ത അനേകംപേര്‍ താല്‍ക്കാലികമായി ലക്ഷ്യമേതുമില്ലാതെ ഒരുമിക്കുന്ന തീര്‍ത്ഥാടനകേന്ദ്രം മാത്രം. അക്കൂട്ടത്തില്‍ ഭാര്യ, പുത്രന്‍, സുഹൃത്തുക്കള്‍ എല്ലാം ഉണ്ട്‌. മണ്‍പാത്രമുണ്ടാക്കുന്നവന്റെ ചക്രം പോലെയാണ്‌ ലോകം - ചക്രം അനങ്ങുന്നില്ല എന്നു കാഴ്ച്ചയില്‍ തോന്നുമെങ്കിലും അതിവേഗതയില്‍ അതു ചുറ്റിക്കൊണ്ടേയിരിക്കുന്നു എന്നതാണ്‌ സത്യം. അതുപോലെ ലോകവും അനുനിമിഷം മാറ്റത്തിനുവിധേയമായിക്കൊണ്ടിരിക്കുന്നുവെങ്കിലും കാഴ്ച്ചയില്‍ അത്‌ സുസ്ഥിരമെന്നു തോന്നുകയാണ്‌.

ലോകം ബന്ധപ്പെടുന്നവരെയെല്ലാം മന്ദബുദ്ധികളാകുന്ന ഒരു വിഷച്ചെടിയാണ്‌. ഇഹലോകത്തിലെ എല്ലാ വീക്ഷണങ്ങളും കറപുരണ്ടതാണ്‌; എല്ലാ രാജ്യങ്ങളും ദുഷ്ടതനിറഞ്ഞതാണ്‌; മനുഷ്യരെല്ലാം മരണവിധേയരാണ്‌; എല്ലാ കര്‍മ്മങ്ങളും വ്യാജവുമാണ്‌. പലയുഗങ്ങള്‍ വന്നുപോയിക്കഴിഞ്ഞിരിക്കുന്നു. അവ 'കാല'ത്തിന്റെ വെറും മാത്രകള്‍ മാത്രം. ഒരു യുഗത്തിന്റെ കാലയളവും നിമിഷനേരവും തമ്മില്‍ യഥാര്‍ത്ഥത്തില്‍ വ്യത്യാസമില്ല. രണ്ടും സമയത്തിന്റെ അളവുകള്‍. ദേവദൃഷ്ടിയില്‍ യുഗമെന്നത്‌ വെറുമൊരു നിമിഷം മാത്രം. പഞ്ചഭൂതങ്ങളില്‍ ഒന്നായ ഭൂമിയുടെ പരിഷ്കൃത ഭാവമായ ഈ ലോകത്തിനുമേല്‍ നമ്മുടെ ശ്രദ്ധയും പ്രതീക്ഷയും അര്‍പ്പിക്കുന്നത്‌ എത്ര വ്യര്‍ത്ഥം! 

