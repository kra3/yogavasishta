\newpage
\section{ദിവസം 129}

\slokam{
മനോമാത്രം ജഗത്‌ കൃത്സ്നം മന: പര്യന്തമണ്ഡലം\\
മനോ വ്യോമ മനോ ഭൂമിര്‍മനോ വായുര്‍മനോ മഹാന്‍ (3/110/15)\\
}

വ്സിഷ്ഠന്‍ തുടര്‍ന്നു: ആദിയില്‍ പരമപുരുഷനില്‍ (അനന്താവബോധത്തില്‍ ഒരു വിഭജനം ഉണ്ടായി. എങ്ങിനെയെന്നാല്‍ , അനന്തത, കാഴ്ച്ചവസ്തുവും കാഴ്ച്ചക്കാരനും ആയി. ഈ ദൃഷ്ടാവ്‌ കാഴ്ച്ചയെ സമ്പൂര്‍ണ്ണമായി ഉള്‍ക്കൊള്ളാന്‍, അറിയാന്‍ ശ്രമിക്കവേ ആകെ ചിന്താക്കുഴപ്പമായി. യാഥാര്‍ഥ്യമെന്ത്‌, കാഴ്ച്ചയെന്ത്‌ എന്നു തിരിച്ചറിയാന്‍ വയ്യാത്ത ഒരവസ്ഥ. ഈ ചിന്താക്കുഴപ്പത്തിന്റെ ഫലമായി അനന്താവബോധത്തില്‍ പരിമിതമായ അനേകം ധാരണകള്‍ ഉണ്ടായി. ഈ പരിമിതമനസ്സ്‌ എണ്ണമില്ലാത്ത സങ്കല്‍പ്പങ്ങളും ആശയങ്ങളുമുണ്ടാക്കി. അവ ആത്മാവിനെ ക്ഷീണിതമാക്കി ഉണ്മയെ മറച്ച്‌ ദു:ഖാനുഭവം നല്‍കുന്നു. മനസ്സിവയെ പര്‍വ്വതാകാരമാക്കി അവതരിപ്പിക്കുന്നു. ഈ ആശയങ്ങളും അനുഭവങ്ങളും മനസ്സില്‍ മുദ്രിതമായി വാസനകളാവുന്നു. ഇവയില്‍ മിക്കതും പ്രകടമാവാതെ ലീനമായാണ്‌ നിലകൊള്ളുന്നത്‌.. മനസ്സവയെ ഉപേക്ഷിക്കുമ്പോള്‍ സൂര്യോദയത്തില്‍ മൂടല്‍ മഞ്ഞെന്നപോലെ മൂടുപടം നീങ്ങി സത്യം തെളിയുന്നു. അതോടെ ദു:ഖത്തിനവസാനവുമായി. അതുവരെ കുട്ടികള്‍ തുമ്പികളെ കളിപ്പിക്കുമ്പോലെ, മനസ്സ്‌ കളിയായി ലീലയാടുകയാണ്‌..

മലിനമായ മനസ്സ്‌ വെറുമൊരു വിളക്കുകാലില്‍ ഭൂതത്തെ കാണുന്നു. അതു ബന്ധങ്ങളെ നശിപ്പിക്കുന്നു. സുഹൃത്തുക്കള്‍ക്കിടയില്‍പ്പോലും സംശയത്തിന്റെ കരിനിഴല്‍ വീഴ്ത്തി അവരെ ശത്രുക്കളാക്കുന്നു. മദ്യപിച്ചു ലക്കുകെട്ടവന്‌ ലോകം മുഴുവന്‍ അവനുചുറ്റും കറങ്ങുന്നതായി കാണുന്നതുപോലെയത്രേ ഇത്‌.. തീവ്രവ്യഥക്കടിമപ്പെട്ട മനസ്സ്‌ ഭക്ഷണത്തെ വിഷമാക്കുന്നു. രോഗത്തിനും മരണത്തിനും അതു കാരണമാകുന്നു. വാസനാമാലിന്യം നിറഞ്ഞ മനസ്സാണ്‌ ഭ്രമകല്‍പ്പനകള്‍ക്കും ഭയത്തിനും മോഹവിഭ്രാന്തികള്‍ക്കും കാരണം. ഇതിനെയെല്ലാം വേരോടെ കളയാന്‍ ഒരുവന്‍ പരിശ്രമിക്കേണ്ടതുണ്ട്‌..

മനസ്സല്ലാതെ മനുഷ്യന്‍ പിന്നെ മറ്റ്‌ എന്താണ്‌? ശരീരം ജഢമാണ്‌. അതിനു സ്വന്തമായി ചൈതന്യമില്ല. മനസ്സ്‌ ജഢമാണെന്നു പറയുകവയ്യ. അത്‌ ചൈതന്യവത്താണെന്നും ഉറപ്പില്ല. മനസ്സിന്റെ ചെയ്തികളാണ്‌ കര്‍മ്മങ്ങളായി പ്രകടമാവുന്നത്‌.. മനസ്സുകൊണ്ട്‌ ഉപേക്ഷിക്കുന്നതാണ്‌ സംന്യാസം.

"മനസ്സാണീ ലോകമഖിലം. മനസ്സാണ്‌ അന്തരീക്ഷം; മനസ്സാണ്‌ ആകാശം; മനസ്സാണ്‌ ഭൂമി; മനസ്സാണ്‌ വായു; മനസ്സ്‌ അതിമഹത്താണ്‌.". മനസ്സിനു ജളത്വം ബാധിച്ചവനത്രേ മൂഢന്‍.. എന്നാല്‍ മരണത്തില്‍ ശരീരത്തിലെ പ്രജ്ഞ നഷ്ടപ്പെടുമ്പോള്‍ ശവത്തെ ആരും 'വിഡ്ഢി' എന്നു വിളിക്കുകയില്ല. മനസ്സു കാണുന്നു; കണ്ണുകള്‍ അങ്ങിനെയുണ്ടായി. മനസ്സു കേള്‍ ക്കുന്നു; കാതുകള്‍ അങ്ങിനെയുണ്ടായി. മറ്റ്‌ ഇന്ദ്രിയങ്ങളും ഉണ്ടാക്കിയത്‌ മനസ്സാണ്‌..

ഏതു മധുരമെന്നും ഏതു കയ്പ്പെന്നും നിശ്ചയിക്കുന്നത്‌ മനസ്സാണ്‌.. ആരാണ്‌ ശത്രുവെന്നും ആരാണ്‌ മിത്രമെന്നും മനസ്സാണു തീരുമാനിക്കുന്നത്‌.. മനസ്സിലാണ്‌ കാലത്തെപ്പറ്റിയുള്ള ധാരണകളുള്ളത്‌.. ലവണരാജാവിന്റെ അനുഭവത്തില്‍ ഒരുമണിക്കൂറിന്‌ ഒരായുസ്സിന്റെ നീളമുണ്ടായിരുന്നല്ലോ. മനസ്സാണ്‌ സ്വര്‍ഗ്ഗനരകങ്ങളെ നിശ്ചയിക്കുന്നത്‌. അതുകൊണ്ട്‌ മനസ്സിനെ ജയിച്ചാല്‍ ഇന്ദ്രിയങ്ങളടക്കം എല്ലാത്തിനേയും ജയിച്ചു. 

