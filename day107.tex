 
\section{ദിവസം 107}

\slokam{
അവിബോധാദയം വാദോ ജ്ഞാതേ ദ്വൈതം ന വിധ്യതേ\\
ജ്ഞാതേ സംശാന്തകലനം മൌനമേവാവശിഷ്യതേ (3/84/25)\\
}

വസിഷ്ഠന്‍ തുടര്‍ന്നു: കാര്‍ക്കടി ഇന്നും രാജാവിന്റെ പിന്മുറക്കാരെ സംരക്ഷിച്ചുപോരുന്നു. അവള്‍ ഞണ്ടിന്റെ രൂപസാദൃശ്യമുള്ള ഒരു രാക്ഷസന്റെ പുത്രിയായിരുന്നു. രാക്ഷസര്‍ വെള്ള, കറുപ്പ്‌, പച്ച, ചുവപ്പ്‌ എന്നീ വിവിധ നിറങ്ങളുള്ളവരും വിവിധ തരക്കാരുമാണ്‌... കാര്‍ക്കടി കറുത്തവളായിരുന്നു. അവളുടെ ചോദ്യങ്ങളും രാജാവിന്റെ ഉത്തരങ്ങളും എന്റെ ഓര്‍മ്മയിലെത്തിയതുകൊണ്ട്‌ ഞാന്‍ അവളുടെ കഥ പറഞ്ഞുവെന്നേയുള്ളു. ചെറിയൊരു വിത്തില്‍ വന്മരത്തിന്റെ എല്ലാ പ്രഭാവങ്ങളും - ഇലകള്‍ , പൂക്കള്‍ , കായ്കള്‍ , എല്ലാം..- അടങ്ങിയിരിക്കുമ്പോള്‍ അവിടെ വ്യതിരിക്തത ഇല്ല. അതുപോലെ ഈ വൈവിദ്ധ്യമാര്‍ന്ന പ്രപഞ്ചം അനന്താവബോധത്തില്‍ നിന്നും വ്യാപൃതമത്രേ. രാമ: എന്റെയീ വാക്കുകള്‍ കേട്ടാല്‍ത്തന്നെ നീ പ്രബുദ്ധനാവുമെന്നെനിക്കുറപ്പുണ്ട്‌... വിശ്വം ബ്രഹ്മത്തില്‍നിന്നുദ്ഭൂതമാണെന്നും അത്‌ ബ്രഹ്മത്തില്‍നിന്നും വിഭിന്നമല്ലെന്നും അറിയുക.

രാമന്‍ ചോദിച്ചു: ഈ ഏകാത്മകത എന്നതാണുണ്മ എങ്കില്‍ എന്തിനാണു നാം 'ഇന്നതിലൂടെ അതിനെ പ്രാപിക്കാം' എന്നും മറ്റും പ്രസ്താവിക്കുന്നത്‌?

വസിഷ്ഠന്‍ പറഞ്ഞു: വേദങ്ങളില്‍ വാക്കുകള്‍ ഉപയോഗിച്ചിരിക്കുന്നത്‌ കാര്യങ്ങള്‍ പഠിപ്പിക്കാനാണ്‌.. കാര്യവും കാരണവും, ആത്മാവും ഈശ്വരനും, വ്യത്യാസവും അതിന്റെ അഭാവവും, വിദ്യയും അവിദ്യയും, വേദനയും സുഖവും, ഇങ്ങനെയുള്ള ദ്വന്ദസംജ്ഞകളെ വികസിപ്പിച്ചെടുത്തിട്ടുള്ളത്‌ പഠനത്തിനായാണ്‌.. അവ സ്വയം യാഥാര്‍ഥ്യമല്ല. "ഈ ചര്‍ച്ചയും തര്‍ക്കവും നടക്കുന്നത്‌ അവിദ്യകൊണ്ടു മാത്രമാണ്‌. ജ്ഞാനമുണരുമ്പോള്‍ ദ്വന്ദതയില്ലാതാവുന്നു. സത്യസാക്ഷാത്കാരമുണ്ടാവുമ്പോള്‍ എല്ലാ വിവരണങ്ങളും നിലയ്ക്കുന്നു. മൌനം മാത്രം ബാക്കിയാവുന്നു." അപ്പോള്‍ നീയറിയും, 'ഒന്നു' മാത്രമേയുള്ളു എന്ന്. അതിന്‌ ആദിയും അന്തവുമില്ല. പക്ഷേ സത്യത്തെ നിര്‍വ്വചിക്കാന്‍ പദങ്ങള്‍ ഉപയോഗിക്കുന്നിടത്തോളം ദ്വന്ദത അനിവാര്യമാണ്‌.. എന്നാല്‍ ഈ ദ്വന്ദത സത്യമല്ല. എല്ലാ വിഭാഗീയതകളും ഭ്രമാത്മകമാണ്‌...

ഞാന്‍ വേറൊരുദാഹരണം പറയാം. ശ്രദ്ധിച്ചു കേട്ടാലും. എന്റെ വിശദീകരണങ്ങളാകുന്ന ശക്തിയേറിയ മരുന്നുകൊണ്ട്‌ നിന്റെ രോഗവും മനപ്രയാസവുമെല്ലാം മാറും. ഇക്കാണുന്ന ലോകം ഇഷ്ടാനിഷ്ടങ്ങള്‍ നിറഞ്ഞ മനസ്സാണ്‌. മനസ്സ്‌ അവയില്‍നിന്നും സ്വതന്ത്രമായാല്‍ ഈ പ്രത്യക്ഷലോകത്തിന്‌ അവസാനമായി. മനസ്സിലെ ബോധമാണ്‌ എല്ലാ പദാര്‍ത്ഥങ്ങള്‍ക്കും ബീജമാവുന്നത്.  മനസ്സിന്റെ ജഢമായ വശമാണ്‌ ഈ ദൃശ്യപ്രപഞ്ചമെന്ന മായക്കാഴ്ച്ചകള്‍ക്ക്‌ കാരണം. ബോധത്തിന്റെ സര്‍വ്വവ്യാപിത്വം കാരണം മനസ്സ്‌ 'അറിയപ്പെടുന്നതായി' ഭവിക്കുന്നു. അത്‌ വിശ്വനിര്‍മ്മിതിക്കു കാരണവുമാവുന്നു. മനസ്സ്‌, ഒരു കുഞ്ഞിന്റെ ഭാവന പോലെ ലോകത്തെ ഉണ്ടാക്കുന്നു. മനസ്സിനു തെളിച്ചമുണ്ടാവുമ്പോള്‍ അത്‌ തനിക്കുള്ളില്‍ അനന്താവബോധത്തെ അനുഭവിക്കുന്നു. വിഷയം-വിഷയി വിഭജനം എങ്ങിനെയെന്ന് ഇനി ഞാന്‍ പറഞ്ഞുതരാം

