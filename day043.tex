 
\section{ദിവസം 043}

\slokam{
ദൃഷ്ടൃദർശനദൃശ്യാനാം മധ്യേ യദ് ദർശനം സ്ഥിതം\\
സാധോ തദവധാനേന സ്വാത്മാനമവബുദ്ധ്യസേ (3/9/75)\\
}

രാമന്‍ പിന്നേയും ചോദിച്ചു: ഭഗവന്‍ , എന്റെ ധാരണ പിഴച്ചിരിക്കുന്നു. അങ്ങു പറഞ്ഞ ആ അവസ്ഥ എങ്ങിനെയാണെനിക്കു സ്വാംശീകരിക്കാനാവുക?

വസിഷ്ഠന്‍ പറഞ്ഞു: രാമ: മുക്തി എന്നുപറഞ്ഞാല്‍ അത്‌ നിരുപാധികമായ ബ്രഹ്മമാണ്‌. അതുമാത്രമാണ്‌.  'ഞാന്‍ ', 'നീ' എന്നീ ധാരണകള്‍ , വെറും ധാരണകള്‍ മാത്രമാണ്‌. അവ ഒരിക്കലും ഉണ്ടായിട്ടേയില്ല. ബ്രഹ്മമാണ്‌ ഇക്കാണപ്പെടുന്ന എല്ലാമായത്‌ എന്നു നമുക്കെങ്ങിനെ പറയാനാവും? രാമ: ആഭരണങ്ങളില്‍ ഞാന്‍ സ്വര്‍ണ്ണം മാത്രം കാണുന്നു. അലകളില്‍ ഞാന്‍ ജലം കാണുന്നു. വായുവില്‍ ഞാന്‍ ചലനം മാത്രം കാണുന്നു. ആകാശത്തില്‍ ഞാന്‍ ശൂന്യത ദര്‍ശിക്കുന്നു. മരീചികയില്‍ ഞാന്‍ താപത്തെ തിരിച്ചറിയുന്നു. അതുപോലെ ഞാന്‍ ബ്രഹ്മത്തെയാണറിയുന്നത്‌. വിശ്വത്തെയല്ല.

പ്രപഞ്ചമെന്ന ധാരണ അനാദിയായ അവിദ്യ തന്നെയാണ്‌. എങ്കിലും സത്യാന്വേഷണം കൊണ്ട്‌ അപ്രത്യക്ഷമാവുന്നതാണീ അവിദ്യ. തുടങ്ങിയതിനു മാത്രമേ അവസാനമുള്ളു. ഈ ലോകം ഒരിക്കലും ഉണ്ടായിട്ടില്ല; പക്ഷേ അങ്ങിനെ കാണപ്പെടുകയാണ്‌. ഈ സത്യത്തെയാണ്‌ ഈ അദ്ധ്യായത്തില്‍ വിശദീകരിച്ചു പറയുന്നത്‌. 

ഇക്കഴിഞ്ഞ വിശ്വപ്രളയത്തിനു മുന്‍പ്‌ കാണപ്പെട്ടവയെല്ലാം പ്രളയത്തില്‍ അപ്രത്യക്ഷമായി. അതോടെ അനന്തത മാത്രം അവശേഷിച്ചു. അത്‌ ശൂന്യമോ രൂപമുള്ളതോ അല്ല. ദൃഷ്ടിയോ ദൃഷ്ടാവോ അല്ല. അതുണ്ടെന്നോ ഇല്ലെന്നോ പറയാനും വയ്യ. കണ്ണൂം നാക്കും ചെവിയും അതിനില്ലെങ്കിലും അതു കാണുന്നു, ഭക്ഷിക്കുന്നു, കേള്‍ക്കുന്നു. അതിനു കാരണമൊന്നുമില്ല. അത്‌ ജനിക്കാത്തതാന്‌. എന്നാല്‍ അതാണ്‌ എല്ലാറ്റിന്റേയും കാരണമായി, ജലം തിരകള്‍ക്കെന്നപോലെ, വര്‍ത്തിക്കുന്നത്‌.

എല്ലാത്തിന്റേയും സത്തയായിരിക്കുന്നത്‌ ഈ അനന്തവും ശാശ്വതവുമായ പ്രഭാവിശേഷമത്രേ. ഈ പ്രഭയിലാണ്‌ മൂന്നു ലോകങ്ങളും ഒരു മരീചികയിലെന്നപോലെ തിളങ്ങിവിളങ്ങുന്നത്‌. ഈ അനന്തത സ്പന്ദിതമാവുമ്പോള്‍ ലോകങ്ങള്‍ പ്രകടിതമായി കാണപ്പെടുന്നു. ആ സ്പന്ദനം  നിലയ്ക്കുമ്പോള്‍ ലോകങ്ങള്‍ അപ്രത്യക്ഷവുമാവുന്നു. ഒരു തീക്കൊള്ളി ചുഴറ്റുമ്പോള്‍ ഉണ്ടാവുന്ന അഗ്നിചക്രം, ചുഴറ്റല്‍ നിര്‍ത്തുമ്പോള്‍ അപ്രത്യക്ഷമായി തീക്കൊള്ളി അചലമായി നിലകൊള്ളുന്നതുപോലെയാണിത്‌.  പ്രകടനാത്മകമാണെങ്കിലും  അല്ലെങ്കിലും അത്‌ എല്ലായിടത്തും എപ്പോഴും ഒരുപോലെ നിലകൊള്ളുന്നു. ഈ സത്യമറിയാത്തപ്പോള്‍ മോഹവിഭ്രാന്തിയും അറിയുമ്പോള്‍ എല്ലാ ആസക്തിയില്‍ നിന്നും ആശങ്കകളില്‍നിന്നും വിടുതലും ലഭിക്കുന്നു.

അതാണ്‌ കാലവും, പ്രകടിതവസ്തുക്കളെക്കുറിച്ചുള്ള ധാരണയും, വസ്തുക്കളും, കര്‍മ്മവും, രൂപവും. സ്വാദും, മണങ്ങളും, ശബ്ദങ്ങളും, സ്പര്‍ശവും, ചിന്തകളുമെല്ലാം. നാമറിയുന്നതെല്ലാം അതാണ്‌. അതിനാലാണ്‌ നാമറിയുന്നതും. "അതാണ്‌ ദൃഷ്ടാവും ദൃഷ്ടിയും ദൃശ്യവും (കാണലും, കാണുന്നതും കാണുന്നവനും). നീ ഇതറിയുമ്പോള്‍ ആത്മസാക്ഷാത്കാരമായി."

