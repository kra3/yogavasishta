\section{ദിവസം 167}

\slokam{
യസ്യാന്തർവാസനാരജ്ജ്വാ ഗ്രന്ഥിബന്ധ: ശരീരിണ:\\
മഹാനപി ബഹൂശോപി സ ബാലേനാപി ജീയതേ (4/27/20)\\
}

വസിഷ്ഠൻ തുടർന്നു: താനുണ്ടാക്കിയ മൂന്നു പുതിയ രാക്ഷസപുത്രന്മാരുടെ നേതൃത്വത്തിൽ ദേവന്മാരോട് യുദ്ധംചെയ്യാൻ ശംഭരൻ തന്റെ ശേഷിക്കുന്ന സൈന്യത്തെ അയച്ചു. ദേവസൈന്യവും യുദ്ധത്തിനു തയ്യാറായി അണിനിരന്നു. ആയുധമൊന്നുമില്ലാതെ കൈക്കരുത്തുകൊണ്ടുള്ള യുദ്ധമാണവിടെ ആദ്യം നടന്നത്. അതിഭീകരമായി പോരാട്ടം നടന്നു. പിന്നീടവർ പലേവിധത്തിലുള്ള ആയുധങ്ങളും അസ്ത്രങ്ങളും ഉപയോഗിക്കാൻ തുടങ്ങി. നഗരങ്ങളും ഗ്രാമങ്ങളും ഗുഹകളും മൃഗങ്ങളുമെല്ലാം യുദ്ധത്തിൽ നശിച്ചു. ഇടവിട്ടിടവിട്ട് രണ്ടു കൂട്ടർക്കും ജയ പരാജയങ്ങൾ ഉണ്ടായി. പ്രധാന അസുരനേതാക്കളായ ഈ മൂന്നുപേർ ദേവന്മാരുടെ പ്രധാന സേനാപതികളെ കണ്ടെത്താൻ ശ്രമിച്ചെങ്കിലും അവർക്കതിനു കഴിഞ്ഞില്ല. രാക്ഷസർ ശംഭരനെക്കണ്ടു കാര്യം പറഞ്ഞു. ദേവന്മാർ സൃഷ്ടാവായ ബ്രഹ്മാവിനോട് പരാതി പറഞ്ഞു. അദ്ദേഹം ഉടനേ അവർക്കു മുന്നിൽ പ്രത്യക്ഷനായി. ഈ മൂന്നു രാക്ഷസന്മാരെ വകവരുത്താനുള്ള മാർഗ്ഗമാണ്‌ ദേവന്മാർ ബ്രഹ്മാവിനോട് യാചിച്ചത്.

ബ്രഹ്മാവു പറഞ്ഞു: ശംഭരനെ ഇപ്പോൾ കൊല്ലാൻ കഴിയില്ല.ഒരു നൂറു കൊല്ലം കഴിഞ്ഞ് ഭഗവാൻ വിഷ്ണുവിന്റെ കൈകൊണ്ടാണവനു മരണം. അതിനാൽ ഇപ്പോൾ ഈ മൂന്നു രാക്ഷസന്മാരോട് തോറ്റിട്ടെന്നവണ്ണം പിന്മാറുന്നതാണ്‌ ബുദ്ധി. കാലക്രമത്തിൽ, ഈ യുദ്ധത്തിലേർപ്പെട്ട് ജയിച്ചതിന്റെ അഹംഭാവം അവരിൽ അങ്കുരിക്കുമ്പോൾ മനസ്സ് ഉപാധികൾക്കു വശംവദമായി അവരിൽ വാസനാമാലിന്യം അടിഞ്ഞുകൂടും. ഇപ്പോള്‍ ഈ മൂവർക്കുള്ളിൽ അഹംഭാവം ലേശം പോലുമില്ല; അഹത്തിന്റെ സന്താനങ്ങളാണല്ലോ വാസനാമാലിന്യങ്ങൾ. അവരിലിപ്പോൾ ആഗ്രഹങ്ങളോ ക്രോധമോ ഇല്ല. അവരിപ്പോൾ അജയ്യരത്രേ. “ആരിലാണോ അഹംഭാവവും തൽ ജന്യങ്ങളായ മനോപാധികളും ഉള്ളത്, അയാൾ എത്ര മഹാനായിരുന്നാലും, എത്ര വിദ്വാനായിരുന്നാലും ഒരു ചെറിയ കുട്ടിക്കുപോലും അയാളെ തോല്‍പ്പിക്കാൻ കഴിയും”. വാസ്തവത്തിൽ ‘ഞാൻ’, ‘എന്റെ’ തുടങ്ങിയ ധാരണകളാണ്‌ ദു:ഖങ്ങളേയും ദുരിതങ്ങളേയും ആകർഷിച്ചു വരുത്തുന്നത്. സ്വന്തം ശരീരവുമായി താതാത്മ്യം പ്രാപിക്കുന്നവൻ ദുരിതത്തിലാണ്ടു പോവുന്നു. എന്നാൽ ആത്മാവിനെ സർവ്വവ്യാപിയായിക്കാണുന്നവൻ ദു:ഖനിവൃത്തി നേടുന്നു. ഇവരെ സംബന്ധിച്ചിടത്തോളം ത്രിലോകങ്ങളിലും ആത്മാവല്ലാതെ മറ്റൊന്നുമില്ല. തന്നിൽ നിന്നു വിഭിന്നമായി ഒന്നുമില്ലാത്തതുകൊണ്ട് അവർ യാതൊന്നിനായും ആഗ്രഹിക്കുന്നുമില്ല. മനസ്സ് ഉപാധികളാൽ ബന്ധിതമായവനെ എളുപ്പത്തിൽ പരാജയപ്പെടുത്താം എന്നാൽ മനോപാധികൾ ഇല്ലാത്തപക്ഷം ഒരു കൊതുകിന്റെ ജീവിതം പോലും അനശ്വരമായി തീരാം. ഉപാധികളുള്ള മനസ്സ് ദുരിതങ്ങളും ഉപാധിരഹിതമായ മനസ്സ് ആനന്ദവും അനുഭവിക്കുന്നു. ഉപാധികൾ അല്ലെങ്കിൽ ആസക്തികൾ ഒരുവനെ പരിക്ഷീണനാക്കുന്നു. അതിനാൽ ഈ മൂവരെ എതിർക്കാൻ ധൃതി പിടിക്കേണ്ടതില്ല.

‘ഞാൻ’, ‘എന്റെ’ എന്നീ ധാരണകൾ അവരിൽ ഉണ്ടാക്കാൻ എന്തെല്ലാം നിങ്ങൾക്ക് ചെയ്യാൻ കഴിയുമോ അതെല്ലാം ചെയ്യുക. ശംഭരന്റെ സൃഷ്ടികളായ അവർക്ക് പ്രത്യേക ജ്ഞാനമൊന്നുമില്ലാത്തതുകൊണ്ട് നിങ്ങളെറിയുന്ന ‘ഇര’യിൽ അവർ കൊത്താതിരിക്കില്ല. അപ്പോൾ നിങ്ങൾക്കവരെ നിഷ് പ്രയാസം ജയിക്കാം. 

