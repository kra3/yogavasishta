 
\section{ദിവസം 051}

\slokam{
സ്വയമസ്തം ഗതേ ബാഹ്യേ സ്വജ്ഞാനാദുദിതാ ചിതി:\\
സ്വയം ജഡേശു ജാഡ്യേന പദം സൗഷുപ്തമാഗതാ (3/14/67)\\
}

വസിഷ്ഠന്‍ തുടര്‍ന്നു: "എന്നെ മുറിക്കാന്‍ കഴിയില്ല; എന്നെ ജ്വലിപ്പിക്കാന്‍ ആവില്ല; എന്നെ നനയ്ക്കാനാവില്ല; എന്നെ ഉണക്കാന്‍ ആവില്ല; ഞാന്‍ ശാശ്വതമായി നിലകൊള്ളുന്നു. സര്‍വ്വവ്യാപിയും ഒരിക്കലുമിളകാത്തതും മാറ്റങ്ങള്‍ക്കുവിധേയമാവത്തതുമാണ്‌ ഞാന്‍". ഇതാണു സത്യം. ആളുകള്‍ പൊതുവേ തര്‍ക്കിക്കാനും ചിന്താക്കുഴപ്പമുണ്ടാക്കനും താല്‍പ്പര്യപ്പെടുന്നവരാണ്‌. പക്ഷേ രാമ: നീ അതിനെല്ലാം അതീതനാണ്‌. മാറ്റമില്ലാത്തവയില്‍ മാറ്റങ്ങളെ അരോപിക്കുന്നത്‌ അജ്ഞാനികളാണ്‌. ആത്മജ്ഞാനികളായ ഋഷിവര്യന്മാരുടെ ദൃഷ്ടിയില്‍, ബോധത്തില്‍ മാറ്റങ്ങളൊന്നും ഉണ്ടായിട്ടില്ല. രാമ: ഈ ബോധം തന്നെയാണ്‌ ആകാശമായി പരന്നു വികാസം പ്രാപിച്ചത്‌; പക്ഷേ അതിനു മാറ്റങ്ങള്‍ ഉണ്ടായില്ല. അതിനുശേഷം ബോധം, ചലനം സഹജമായ വായുവായി പ്രത്യക്ഷമായി. അതേ ബോധം തന്നെയാണ്‌ ജലം, അഗ്നി, ധാതുക്കളോടുകൂടിയ മണ്ണ്‌ , ഒക്കെയായത്‌. ജീവികളുടെ ശരീരവും അതത്രേ. "ബാഹ്യമായി, 'അറിയപ്പെടാനുള്ളവ'യേപ്പറ്റിയുള്ള ധാരണകള്‍ നീങ്ങുമ്പോള്‍ ആത്മജ്ഞാനം ഉദിക്കുന്നു. അതുപോലെ ജഢത്വവും അലംഭാവവും, അജ്ഞാനവും അടിഞ്ഞുകൂടുമ്പോള്‍ ഗാഢനിദ്രയുമായി"

അതിനാല്‍ ബോധം മാത്രമേ എക്കാലവും നിലനില്‍ക്കുന്നതായുള്ളു എന്നറിയുക. അകാശം ഉണ്ടെന്നും ഇല്ലെന്നും പറയാം; അതുപോലെ ലോകം നിലനില്‍ക്കുന്നു എന്നും ഇല്ലെന്നും പറയാം. തീയിന്‌ ചൂടെങ്ങിനെയോ, ശംഖിനു വെണ്മയെങ്ങിനെയോ, പര്‍വ്വതത്തിന്റെ അചലതയെങ്ങിനെയോ, ജലത്തിന്റെ ദ്രവഗുണമെങ്ങിനെയോ, കരിമ്പുതണ്ടിന്റെ മാധുര്യമെങ്ങിനെയോ, പാലിലെ വെണ്ണയെങ്ങിനെയോ, മഞ്ഞുകട്ടയുടെ ശീതളിമയെങ്ങിനെയോ, ദീപാലങ്കാരത്തിന്റെ പ്രഭയെങ്ങിനെയോ, കടുകുമണിയിലെ എണ്ണയെങ്ങിനെയോ, നദിയുടെ ഒഴുക്കെങ്ങിനെയോ, തേനിന്റെ മാധുര്യമെങ്ങിനെയോ, ആഭരണങ്ങളില്‍ സ്വര്‍ണ്ണമെങ്ങിനെയോ, പൂക്കളില്‍ സൌരഭ്യമെങ്ങിനെയോ അതുപോലെയാണ്‌ പ്രപഞ്ചത്തില്‍ ബോധം. ലോകം നിലനില്‍ ക്കുന്നത്‌ ബോധത്തിലാണ്‌. ബോധത്തിന്റെ ശരീരമാണ്‌ വിശ്വം. അതില്‍ ഭിന്നതകളോ വിവേചനമോ വിഭാഗീയതയോ ഇല്ല.

അതുകൊണ്ട്‌ ഈ പ്രപഞ്ചത്തെ യാഥാര്‍ത്ഥ്യമെന്നും അയാഥാര്‍ത്ഥ്യമെന്നും പറയാം. യാഥാര്‍ത്ഥ്യം, എന്തുകൊണ്ടെന്നാല്‍ വിശ്വാവബോധത്തിനു സ്വയം നിലനില്‍പ്പുണ്ട്‌. അയാഥാര്‍ത്ഥ്യം, എന്തുകൊണ്ടെന്നാല്‍ വിശ്വം വിശ്വമായി സ്വയം നിലനില്‍ക്കുന്നില്ല. ബോധത്തില്‍ അധിഷ്ഠിതമാണതിന്റെ നിലനില്‍ പ്പ്‌.   

ബോധമെന്നത്‌ ഭാഗങ്ങളായി വിഭജിക്കാവുന്ന ഒന്നല്ല. അതിന്‌ അവയവങ്ങള്‍ ഇല്ല. അതില്‍ പര്‍വ്വതങ്ങളും, കടലും, ഭൂമിയും, നദികളും ഒന്നും ബോധത്തില്‍ നിന്നു വേറിട്ട്‌ വ്യതിരിക്തമായി നിലകൊള്ളുന്നില്ല. അതിനാല്‍ ബോധത്തില്‍ ഭാഗങ്ങളോ അവയവങ്ങളോ ഇല്ല. പ്രപഞ്ചത്തിന്റെ അയാഥാര്‍ത്ഥ്യ വസ്തുതയെ കാരണമാക്കി പ്രപഞ്ചത്തിന്റെ നിലനില്‍പ്പിനടിസ്ഥാനമായ ബോധവും യാഥാര്‍ത്ഥ്യമല്ല എന്ന നിഗമനത്തില്‍ എത്തുന്നത്‌ അര്‍ത്ഥവത്തല്ല. കാരണം അതു നമ്മുടെ അനുഭവങ്ങള്‍ക്കു നിരക്കുന്നതല്ല. ബോധത്തിന്റെ സാന്നിദ്ധ്യം ആര്‍ക്കും നിഷേധിക്കാനാവില്ല.

(മൂന്നാം ദിവസം സൂര്യാസ്തമയമായി. സഭ പിരിഞ്ഞു)

