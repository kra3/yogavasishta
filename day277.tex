\section{ദിവസം 277}

\slokam{
ഉപശശാമ ശനൈര്‍ദിവസൈരസൌ\\
കതിപയൈ സ്വപദേ വിമലാത്മനി\\
തരുരസ: ശരദന്ത ഇവാമലേ\\
രവികരൌജസി ജന്മദശാതിഗ:   (5/55/23)\\
}

ശുദ്ധ സത്വസ്വരൂപത്തെപ്പറ്റി രാമന്‍ ചോദിച്ചതിനുത്തരമായി വസിഷ്ഠന്‍ ഇങ്ങിനെ പറഞ്ഞു: വിഷയബോധം, അതായത് വസ്തു- പദാര്‍ത്ഥ- അസ്തിത്വത്തെപ്പറ്റിയുള്ള ചിന്തകളില്ലാതെ മനസ്സൊടുങ്ങിക്കഴിയുമ്പോള്‍ ബോധം അതിന്റെ സ്വരൂപമായിത്തന്നെ യാതൊരുപാധികളുമില്ലാതെ നിലകൊള്ളുന്നു. അതാണ്‌ ശുദ്ധസ്വരൂപം. അതായത് വസ്തുധാരണകളോ ആശയങ്ങളോ ഇല്ലാത്ത ബോധത്തിന് വ്യക്തിഗതസ്വഭാവങ്ങളൊന്നുമില്ല. അതാണ്‌ ശുദ്ധസത്വം.  ബാഹ്യവസ്തുക്കളും അന്തര്‍ധാരണകളും ബോധത്തില്‍ വിലയിക്കുമ്പോള്‍ ശുദ്ധസത്വമായി. ദേഹത്തോട് കൂടിയോ അല്ലാതെയോ ഉള്ള മുക്തന്മാരുടെയെല്ലാം ദിവ്യദര്‍ശനം ഇത്തരത്തിലുള്ളതാണ്.   

ഈ ദര്‍ശനം പൂര്‍ണ്ണമായും അറിവുറച്ചവനും, ആഴമേറിയ ധ്യാനത്തില്‍ ആമഗ്നനായവനും, ആത്മജ്ഞാനിക്കും ലഭ്യമാകുമ്പോള്‍ അജ്ഞാനിക്ക് ഇതപ്രാപ്യമത്രേ. മാമുനിമാരും ത്രിമൂര്‍ത്തികളും ഈയൊരവബോധതലത്തിലാണ് വിരാജിക്കുന്നത്. ഈ ഉന്നതതലത്തിലെത്തിയ ഉദ്ദാലകനും കുറേക്കാലമങ്ങിനെ കഴിഞ്ഞു. കാലക്രമത്തില്‍ ‘ഈ മൂര്‍ത്തരൂപം ഉപേക്ഷിക്കുകതന്നെ’ എന്നൊരു ചിന്ത അദ്ദേഹത്തില്‍ അങ്കുരിച്ചു. ഉടനെതന്നെ മലമുകളിലെ ഒരു ഗുഹയില്‍പ്പോയി പത്മാസനത്തില്‍ അര്‍ദ്ധനിമീലിത നേത്രങ്ങളുമായി അദ്ദേഹമിരുന്നു. എന്നിട്ട് നവദ്വാരങ്ങളെയും അടച്ചുപിടിച്ചു. ഗുദദ്വാരം ഉപ്പൂറ്റി കൊണ്ടമര്‍ത്തിപ്പിടിച്ച് ഇന്ദ്രിയങ്ങളെ ഓരോന്നായി ഹൃദയത്തിലേയ്ക്ക് പിന്‍വലിച്ചു. പ്രാണനെ അടക്കിപ്പിടിച്ച് ദേഹത്തെ സമതുലിതാവസ്ഥയില്‍ നിലനിര്‍ത്തി. താടിയെല്ലുകള്‍ അല്‍പ്പം അകത്തിവെച്ച്, നാവിന്‍തുമ്പ് വായുടെ ഉള്ളിലെ മുകളറ്റമായ താലുവില്‍ മുട്ടിച്ചുവെച്ചു.  

അദ്ദേഹത്തിന്‍റെ ഉല്‍ക്കാഴ്ച അകത്തേയ്ക്കോ പുറത്തേയ്ക്കോ, മുകളിലേയ്ക്കോ താഴേയ്ക്കോ ലക്ഷ്യമാക്കിയിരുന്നില്ല. അതിനു വിഷയങ്ങളുമായി ബന്ധമുണ്ടായിരുന്നില്ല. എന്നാലത് നിശ്ശൂന്യമായിരുന്നുമില്ല. അദ്ദേഹം പരമമായ അനന്താവബോധത്തില്‍ ആനന്ദതുന്ദിലനായി നിലകൊണ്ടു. ആനന്ദാനുഭവത്തിനുമപ്പുറം ശുദ്ധസ്വരൂപമായ അവബോധം മാത്രമായിരുന്നു ആ തലം. ഒരു ചിത്രപടംപോലെ ഉദ്ദാലകന്‍ ആ അഭൌമതലത്തില്‍ ഏറെനേരം നിലകൊണ്ടു.

 “ക്രമേണ അദ്ദേഹം പരിപൂര്‍ണ്ണ പ്രശാന്തതയിലെത്തി. സ്വയം സ്വരൂപത്തില്‍ വിലയിച്ചു. ജനനമരണചക്രത്തിന്റെ വരുതിയില്‍ നിന്നും എന്നെന്നേയ്ക്കുമായി മുക്തനായി.” അദ്ദേഹത്തിന്‍റെ സംശയങ്ങളെല്ലാം തീര്‍ന്നു. എല്ലാ ചിന്തകളും അവസാനിച്ചു. എല്ലാ മാലിന്യങ്ങളും ഒഴിഞ്ഞു. സ്വര്‍ഗ്ഗരാജ്യത്തെ ചക്രവര്‍ത്തിയുടെ സുഖത്തെപ്പോലും ത്രുണവല്‍ഗണിക്കാവുന്ന, വിവരണാതീതമായ ആനന്ദത്തിന്റെ ഉത്തുംഗത്തില്‍ അദ്ദേഹമെത്തി. അങ്ങിനെയൊരാറുമാസം കഴിഞ്ഞു.

അങ്ങിനെയിരിക്കെ ഒരുദിവസം ഒരു ഭക്തന്റെ പ്രാര്‍ത്ഥനയുടെ ഫലമായി പാര്‍വ്വതീദേവിയുടെ നേതൃത്വത്തില്‍ കുറേ ദേവതമാര്‍ ഉദ്ദാലകന്‍ ധ്യാനത്തിലിരുന്നയിടത്ത് എത്തി. ദേവന്മാര്‍പോലും പൂജിക്കുന്ന ആ ദേവി സൂര്യതാപത്താല്‍ വരണ്ടുണങ്ങിയ ഉദ്ദാലകന്റെ ദേഹമെടുത്ത് തന്റെ  ശിരസ്സിലണിഞ്ഞു. ഇതാണ്‌ ഏതോരന്വേഷകനേയും ഉത്തേജിപ്പിച്ച് അവനില്‍ ഉന്നതമായ ആത്മജ്ഞാനത്തിന്റെ ഉണര്‍വ്വുണര്‍ത്തുന്ന ഉദ്ദാലകചരിതം.

