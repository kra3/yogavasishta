\section{ദിവസം 172}

\slokam{
സർവാതിശയ സാഫല്യാത് സർവം സർവത്ര സർവദാ\\
സംഭവത്യേവ തസ്മാത്വം ശുഭോദ്യോഗം ന സംത്യജ (4/33/1)\\
}

വസിഷ്ഠൻ തുടർന്നു: അല്ലയോ രാമാ, ഉൽസാഹത്തോടെ, ശുഷ്കാന്തിയോടെ ചെയ്യുന്ന ഏതൊരു പ്രവൃത്തിയും ഫലപ്രദമാവുകതന്നെ ചെയ്യും. അതിനാൽ ശരിയായ പ്രവർത്തനങ്ങളെ ഉപേക്ഷിക്കാതിരിക്കുക. തീർച്ചയായും അത്യുൽസാഹത്തോടെ ഒരു പ്രവർത്തനത്തിലേർപ്പെടും മുൻപ് അതിന്റെ വരും വരായ്കളെക്കുറിച്ച് നന്നായി ചിന്തിക്കണം. പ്രയത്നത്തിനനുസരിച്ച് പ്രയോജനമുണ്ടാവുമോ എന്നാലോചിക്കുകയും വേണം. ഇങ്ങിനെ പരിചിന്തനംചെയ്ത് നോക്കുമ്പോൾ മനസ്സിലാവും അത്യുൽസാഹത്തോടെയുള്ള പ്രവർത്തനത്തിനു യോഗ്യമായുള്ളത് ആത്മാന്വേഷണം മാത്രമാണെന്ന്. അത് സുഖദു:ഖങ്ങളെ വേരോടെ നശിപ്പിക്കുന്ന ഒന്നാണല്ലോ. നിന്നിലെ സുഖാന്വേഷണത്വര ഉണ്ടാക്കിയ എല്ലാ വസ്തുബോധവും മനസ്സിൽ നിന്നു ദൂരെക്കളഞ്ഞാലും. സന്താപത്തിന്റെ കറപുരളാത്ത സന്തോഷം എന്തെങ്കിലുമുണ്ടോ? പരബ്രഹ്മത്തിന്റെ തലത്തിൽ ആത്മനിയന്ത്രണം ഉള്ളതും ഇല്ലാത്തതും തമ്മിൽ അന്തരമേതുമില്ല. അത് ഫലത്തിൽ ഒന്നുതന്നെ. എന്നാൽ ആത്മനിയന്ത്രണസാധന നിന്നിൽ ആനന്ദവും പവിത്രതയും നിറ്യ്ക്കും. അതിനാൽ അഹംഭാവത്തെ വെടിഞ്ഞ് ആത്മനിയന്ത്രണത്തോടെ ജീവിച്ചാലും.

സത്യാന്വേഷണം സാധനയാക്കുക. സദ്ജനങ്ങളുടെ സംസർഗ്ഗം തേടുക. ശസ്ത്രാധിഷ്ഠിതമായ പാതയിൽ ജീവിച്ച് കാമക്രോധലോഭങ്ങളെ അതിജീവിച്ചവരാണ്‌ വിവേകശാലികളായ സദ്ജനങ്ങൾ. സദ്ജനങ്ങളുടെ സാമീപ്യമാത്രയിൽ ആത്മജ്ഞാനം ഉണരുന്നു. അതേസമയം അവരുടെ സാന്നിദ്ധ്യംകൊണ്ട് വസ്തുതാബോധം -വസ്തുക്കൾ സത്യമാണെന്ന ധാരണ- കുറഞ്ഞു കുറഞ്ഞു വന്ന് ഒടുവിൽ ഇല്ലാതെയാവുകയും ചെയ്യുന്നു. അങ്ങിനെ വിഷയപ്രപഞ്ചമില്ലാതെയാകുമ്പോൾ പരമസത്യം മാത്രം ശേഷിക്കുന്നു. കടിച്ചു തൂങ്ങിനിൽക്കാൻ വിഷയങ്ങളൊന്നും തന്നെ അവശേഷിക്കുന്നില്ലാത്തതുകൊണ്ട് വ്യക്തിഗതജീവൻ ആ സത്യത്തിൽ അഭിരമിക്കുന്നു. ഈ ലോകമെന്ന ‘വസ്തു’ ഒരിക്കലും സൃഷ്ടിക്കപ്പെട്ടിട്ടില്ല. സ്ഥിതിചെയ്തിട്ടില്ല. ഇനിയും ഉണ്ടാകാൻ പോകുന്നുമില്ല. പരംപൊരുള്‍ മാത്രമേ എക്കാലവും ഉണ്മയായി ഉള്ളു. അതു മാത്രമാണ്‌ സത്യം. ഇങ്ങിനെ ആയിരം തരത്തിൽ ഞാനീ വസ്തുപ്രപഞ്ചത്തിന്റെ അയാഥാർത്ഥ്യത്തെപ്പറ്റി നിനക്കു വിവരിച്ചു തന്നു.

അത് ശുദ്ധമായ അനന്താവബോധമല്ലാതെ മറ്റൊന്നുമല്ല. അത് അവിച്ഛിന്നമാണ്‌.. ‘ഇതാണുണ്മ’ എന്നോ ‘ഇതുണ്മയല്ല’ എന്നോ അതിനെ നിര്‍വചിക്കാന്‍ ആവില്ല. അനന്താവബോധമെന്ന ഈ അത്യത്ഭുതപ്രകടനം തന്നെയാണ്‌ ലോകമായി നാം കാണുന്നതും. അതു മറ്റൊന്നാവുക വയ്യ. സൂര്യപ്രകാശവും സൂര്യരശ്മിയും എന്ന മട്ടിലുള്ള തരംതിരിവ് എത്ര അപഹാസ്യമാണോ അതുപോലെയാണ്‌ വിഷയവും വിഷയിയും വേറിട്ടിരിക്കുന്നത്. വസ്തുവും അതിന്റെ നിഴലും തമ്മിലുമുള്ള ബന്ധവും അങ്ങിനെയാണല്ലോ. മാറ്റങ്ങൾക്കോ വിഭജനങ്ങൾക്കോ വിധേയമല്ലാത്ത ഒരേയൊരു ബോധം മാത്രമേയുള്ളു. ഈ അവബോധം തന്റെ കണ്ണു തുറക്കുമ്പോൾ വിശ്വസൃഷ്ടിയും അടയ്ക്കുമ്പോൾ വിശ്വവിലയനവും നടക്കുന്നു എന്നു വേണമെങ്കിൽ പറയാം 

