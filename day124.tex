 
\section{ദിവസം 124}

\slokam{
അബദ്ധോ ബദ്ധ ഇത്യുക്ത്വാ കിം ശോചസി  മുധൈവ ഹി\\
അനന്തസ്യാത്മ തത്വസ്യ കിം കഥം കേന ബധ്യതേ (9)\\
}

വസിഷ്ഠന്‍ തുടര്‍ന്നു: ഒരു മൂഢന്‍ മാത്രമേ സ്വന്തം ആശയങ്ങളില്‍ മോഹിതനാവൂ. ജ്ഞാനിക്കാ മോഹമുണ്ടാവുന്നില്ല. ഒരു വിഡ്ഢിയേ ശാശ്വതമായതിനെ അശാശ്വതമെന്ന് കരുതി ഭ്രമചിത്തനാവൂ. അഹംകാരം എന്നത്‌ ആത്മാവിന്റെ ഭൌതീകവസ്തുക്കളുമായുള്ള ചാര്‍ച്ച മൂലമുണ്ടാവുന്ന ഒരാശയം മാത്രമാണ്‌.. എല്ലാം അനന്താവബോധമാണെന്ന ആ ഏകസത്യം നിറഞ്ഞുനില്‍ക്കുമ്പോള്‍ എങ്ങിനെയാണ്‌ ഈ അഹംകാരം ഉദയം ചെയ്തത്‌? സത്യത്തില്‍ മരുഭൂമിയിലെ മരുപ്പച്ചയുടെ യാഥാര്‍ഥ്യമേ അഹംകാരസങ്കല്‍പ്പത്തിനുള്ളു. അതുകൊണ്ട്‌ രാമാ, നിന്റെ അപക്വവും വസ്തുനിഷ്ഠമല്ലാത്തതുമായ കാഴ്ച്ചപ്പാട്‌ മാറ്റിയാലും. സത്യത്തിലധിഷ്ഠിതമായ, ആനന്ദഘനമായ സമ്യക്‌ ദര്‍ശനത്തില്‍ അഭിരമിച്ചാലും. സത്യത്തെ അന്വേഷിച്ചറിയൂ. അസത്തിനെ ഉപേക്ഷിക്കൂ.

"നീ എപ്പോഴും സ്വതന്ത്രനാണ്‌.. എന്തിനാണ്‌ നീ സ്വയം ബന്ധിതനെന്നു വിലപിക്കുന്നത്‌? ആത്മാവ്‌ അനന്തമാണ്‌.. അപ്പോള്‍ പിന്നെ ആര്‌, ആരെ, എങ്ങിനെയാണ്‌ ബന്ധിക്കുക?" ആത്മാവില്‍ വിഭിന്നതകളില്ല. കാരണം എല്ലാം പരബ്രഹ്മം മാത്രം. അപ്പോള്‍ ബന്ധനം എന്നതും മോചനം എന്നതും എന്താണ്‌? നിനക്ക്‌ വേദനാനുഭവം ഉണ്ട്‌ എന്നത്‌ അജ്ഞാനാവസ്ഥയിലെ ഒരു ചിന്തമാത്രം. വാസ്തവത്തില്‍ നീ വേദനകള്‍ക്ക്‌ അതീതനാണ്‌.. അവ ആത്മാവില്‍ ഇല്ല. ശരീരം വീഴുകയോ എഴുന്നേല്‍ക്കുകയോ മറ്റു പ്രപഞ്ചങ്ങളിലേയ്ക്കു പോവുകയോ ചെയ്താലും അതൊന്നും എന്നെ ബാധിക്കയില്ല. ഞാന്‍ ശരീരത്തില്‍ മാത്രം ഒതുങ്ങി പരിമിതപ്പെട്ടവനല്ല. ശരീരവും ആത്മാവും തമ്മിലുള്ള ബന്ധം മേഘവും കാറ്റും പോലെയാണ്‌.. താമരയും വണ്ടും പോലെയാണ്‌.. മേഘം നീങ്ങുമ്പോള്‍ വായു ആകാശത്തില്‍ വിലയിക്കുന്നു. താമര വിടരുമ്പോള്‍ മൊട്ടിനുള്ളിലെ വണ്ട്‌ ആകാശത്തേയ്ക്ക്‌ പറന്നുപോകുന്നു.

ശരീരം വീഴുമ്പോള്‍ ആത്മാവിന്‌ നാശമുണ്ടാവുന്നില്ല. ആത്മജ്ഞാനത്തിന്റെ തീയിലെരിഞ്ഞാലല്ലാതെ മനസ്സടങ്ങുകയില്ല. അത്മാവും അങ്ങിനെതന്നെ. മരണം എന്നത്‌ ചിരപ്രതിഷ്ഠമായ ആത്മാവിനെ മറയ്ക്കുന്ന ഒരു മറയാണ്‌. ആ മറയുണ്ടാക്കിയിരിക്കുന്നത്‌ കാലം, ദൂരം എന്നീ ഊടുപാവുകളെക്കൊണ്ടാണ്‌.. വിഡ്ഢികളേ മരണത്തെ ഭയക്കൂ. നിന്റെ ലീനവാസനകളെ ഉപേക്ഷിച്ചാലും. പക്ഷികള്‍ പറക്കമുറ്റുമ്പോള്‍ മുട്ടയുടെ തോടുപൊട്ടിച്ചാണല്ലോ വിഹായസ്സിലേയ്ക്കുയരുന്നത്‌.. അവിദ്യാജന്യമായതിനാല്‍ ഈ വാസനകള്‍ നശിപ്പിക്കാന്‍ അത്ര ഏളുപ്പമല്ല. അവ അന്തമില്ലാത്ത ദു:ഖങ്ങള്‍ക്ക്‌ കാരണമാവുന്നു. സ്വയം പരിമിതപ്പെടുത്തുന്ന മനസ്സിന്റെ ഈ പ്രവണതയാണ്‌ അനന്തത്തിന്‌ അതിരുകള്‍ കല്‍പ്പിക്കുന്നത്‌.. എന്നാല്‍ സൂര്യന്‍ മൂടല്‍മഞ്ഞിനെ ഇല്ലാതാക്കുമ്പോലെ സ്വയംകൃതമായ പരിമിതികളെ ആത്മാന്വേഷണം കൊണ്ട്‌ മറികടക്കാം.

വാസ്തവത്തില്‍ ആത്മാന്വേഷണത്വരതന്നെ നമ്മില്‍ മാറ്റങ്ങള്‍ വരുത്തുന്നു. തപസ്സോ മറ്റു ചര്യകളോ കൊണ്ട്‌ ഇതില്‍ പ്രയോജനമൊന്നുമില്ല. ഉള്ളിലുയരുന്ന ജ്ഞാനത്തിന്റെ വെളിച്ചത്തില്‍ മനസ്സ്‌ ഭൂതകാലമുപേക്ഷിച്ച്‌ നിര്‍മ്മലമാവുമ്പോള്‍ അത്‌ ലീനവാസനകളെ ഉപേക്ഷിക്കുന്നു. മനസ്സ്‌ അത്മാന്വേഷണം നടത്തുന്നത്‌ ആത്മാവിനെ ആത്മാവില്‍ത്തന്നെ വിലയിക്കാനാണ്‌.. ഇതാണ്‌ മനസ്സിന്റെ സ്വഭാവം. രാമ: ഇതാണ്‌ പരമലക്ഷ്യം. അതിനായി പരിശ്രമിച്ചാലും.

മറ്റൊരു ദിവസം കൂടി അവസാനിക്കവേ സഭ പിരിഞ്ഞു. 
