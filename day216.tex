\section{ദിവസം 216}

\slokam{
പ്രാണശക്തൗ നിരുദ്ധായാം മനോ രാമ വിലീയതേ\\
ദ്രവ്യച്ഛായാനു തദ്ദ്രവ്യം പ്രാണരൂപം ഹി മാനസം (5/13/83)\\
}

വസിഷ്ഠൻ തുടർന്നു: പരമ്പൊരുളിൽ ഉയർന്നുവരുന്ന ഒരു ചിന്താശകലമാണ്‌ വ്യക്തിഗതബോധം. ഈ ബോധശകലം സ്വദേഹബോധത്തിൽ നിന്നും ചിന്തയിൽനിന്നും സ്വതന്ത്രമാവുമ്പോൾ മുക്തിയായി. ലോകമെന്ന ഈ കാഴ്ച്ചയ്ക്കു കാരണം അനന്താവബോധത്തിലുണ്ടാകുന്ന ചിന്താസ്ഫുരണങ്ങളത്രേ. അവ പരിമിതപ്പെട്ട വ്യക്തിഗതബോധങ്ങൾക്കു ബീജമാവുന്നു. ഇങ്ങിനെ  തന്റെ സഹജസ്ഥിതിയായ പ്രശാന്തതയിൽ നിന്നും സ്വയം അകന്നെന്നപോലെ, ചിന്താശകലങ്ങളാൽ കളങ്കപ്പെട്ടപോലെ, വ്യക്തിബോധത്തിൽ ചിന്താശക്തി ഉദയംചെയ്തു. അതിലൂടെ മനസ്സ് ഒരു പ്രപഞ്ചത്തെ സങ്കല്‍പ്പിച്ചു ചിന്തിച്ചുണ്ടാക്കി. “രാമാ, പ്രാണശക്തിയെ നിയന്ത്രിക്കുന്നതിലൂടെ മനസ്സും നിയന്ത്രിക്കപ്പെടുന്നു. വസ്തുവിനെ മാറ്റിയാൽപ്പിന്നെ അതിന്റെ നിഴലുമാത്രമായുണ്ടാവുന്നതെങ്ങിനെ? പ്രാണശക്തിയുടെ നിയന്ത്രണം കൊണ്ട് മനസ്സില്ലാതാകുന്നു.”

പ്രാണശക്തിയുടെ സഞ്ചാരംകൊണ്ടാണ്‌ ഒരുവൻ തനിക്കെവിടെയെങ്കിലും വെച്ചുണ്ടായ അനുഭവങ്ങളെ ഓർമ്മിക്കുന്നത്. അതിനു മനസ്സെന്നു പറയുന്നു. പ്രാണന്റെ ചലനം അനുഭവിക്കുന്നതു മനസ്സാണ്‌.. പ്രാണശക്തിയെ ഇനിപ്പറയുന്ന മാർഗ്ഗങ്ങളിലൂടെ നിയന്ത്രണവിധേയമാക്കാം. നിർമമത, പ്രാണായാമം, പ്രാണസഞ്ചാരങ്ങളെക്കുറിച്ചുള്ള അന്വേഷണം, മേധാശക്തികൊണ്ട് ദു:ഖങ്ങളെ ഇല്ലാതാക്കൽ, പരമ്പൊരുളിനെക്കുറിച്ചുള്ള നേരറിവ്, അല്ലെങ്കിൽ അനുഭവം എന്നിവയിലൂടെ പ്രാണനെ നിയന്ത്രിക്കാനാവും.

ഒരു കല്ലിൽ ബുദ്ധിയുണ്ടെന്നനുമാനിക്കാൻ മനസ്സിനാവും, എന്നാൽ മനസ്സിന്‌ സ്വയം ബുദ്ധിയുടെ കണികപോലുമില്ല. കേവലം ജഢമായ പ്രാണശക്തിയാണ്‌ ചലനത്തെ നയിക്കുന്നത്. ബുദ്ധി, ബോധം, വിവേകം എന്നിവ ശുദ്ധവും ശാശ്വതവും സർവ്വവ്യാപിയുമായ ആത്മാവിന്റെ ഭാവങ്ങളാണ്‌.. മനസ്സാണീ രണ്ടും തമ്മിലൊരു ബന്ധമുണ്ടെന്നനുമാനിക്കുന്നത്. എന്നാൽ ഒന്ന് സത്തും മറ്റേത് അസത്തുമാവുമ്പോൾ അവ തമ്മിലുള്ള ബന്ധം സാദ്ധ്യമല്ല. ആ മിഥ്യാബന്ധുതയിൽ നിന്നുണ്ടാവുന്ന എല്ലാ ധാരണകളും അപ്പോള്‍ അസംബന്ധങ്ങൾ തന്നെ. ഇതിനാണ്‌ മായ, അവിദ്യ, അജ്ഞാനം എന്നെല്ലാം പറയുന്നത്. പ്രപഞ്ചമെന്ന ഭ്രമക്കാഴ്ച്ചയാണിത്. പ്രത്യക്ഷലോകമെന്ന പ്രകടനം. പ്രാണശക്തിയും ബോധവും തമ്മിലുള്ള ബന്ധം വെറും സങ്കൽപ്പം മാത്രം. ഈ സങ്കൽപ്പമില്ലെങ്കിൽ പ്രത്യക്ഷപ്പെടാൻ ലോകവുമില്ല.

പ്രാണന്‌ ബോധവുമായുള്ള ഈ ചാർച്ചകാരണം ലോകത്തെ അറിയാനും അനുഭവിക്കാനും സാധിക്കുന്നു. എന്നാൽ ഇതെല്ലാം വ്യർത്ഥമത്രേ. ഒരു കുട്ടി ഇല്ലാത്ത ഭൂതത്തെ ഇരുട്ടിൽ കണ്ടു പേടിക്കുന്നതുപോലെ അയാഥാർത്ഥ്യമത്രേ ഇത്. സത്യത്തിൽ എല്ലാം അനന്താവബോധത്തിലെ സഞ്ചാരങ്ങൾ മാത്രം. പരിമിതമായ ഏതെങ്കിലും ഘടകങ്ങൾക്ക് അനന്തതയെ ബാധിക്കാനാവുമോ? താഴെക്കിടയിലുള്ള ഒരു വസ്തുവിന്‌ ഉയർന്ന തലത്തിലുള്ള ഒന്നിനെ സ്വാധീനിക്കാൻ എങ്ങിനെ സാധിക്കും?

അതുകൊണ്ട് രാമാ, മനസ്സും പരിമിതമായ ബോധവും ഒന്നും ഉണ്മയല്ല. ഈ സത്യം നേരറിവായി മാറുമ്പോൾപ്പിന്നെ മനസ്സില്ല. കാരണം മനസ്സെന്നത് ആദ്യമേ തന്നെ വെറും തെറ്റിദ്ധാരണ മാത്രമായിരുന്നല്ലോ. അപൂർണ്ണമായ അറിവാണിതിനു കാരണം. തെറ്റിദ്ധാരണ നീങ്ങുമ്പോൾ മനസ്സില്ലാതാവുന്നു. 

