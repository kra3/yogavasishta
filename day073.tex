\newpage
\section{ദിവസം 073}

\slokam{
യോ യഥാ പ്രേരയതി മാം തസ്യ തിഷ്ഠാമി തത്ഫലാ\\
ന സ്വഭാവോന്യതാം ധത്തേ വഹ്നേരൗഷണ്യമിവൈഷ മേ (3/47/5)\\
}

രണ്ടാമത്തെ ലീല സരസ്വതീ ദേവിയോടു ചോദിച്ചു: ദേവീ, അവിടുന്ന് ഞങ്ങളെ അനുഗ്രഹിച്ചുവെങ്കിലും എന്തുകൊണ്ടാണ്‌ എന്റെ പ്രിയതമന്‌ യുദ്ധത്തില്‍ ജയമില്ലാതെ പോകുന്നത്‌? 

സരസ്വതീദേവി പറഞ്ഞു: രാജാവ്‌ വിഥുരഥന്‍ തീര്‍ച്ചയായും ദീര്‍ഘകാലം എന്നെ പൂജിച്ചിരുന്ന ആളാണ്‌. എന്നാല്‍ യുദ്ധത്തിലെ വിജയത്തിനുവേണ്ടി അദ്ദേഹം എന്നോട്‌ പ്രാര്‍ത്ഥിക്കുകയുണ്ടായില്ല. എല്ലാവരുടെയും അറിവില്‍ നിറഞ്ഞ ബോധമായതുകൊണ്ട്‌ ഓരോരുത്തരും ആവശ്യപ്പെടുന്ന കാര്യങ്ങളാണ്‌ ഞാനവര്‍ക്കു സാധിപ്പിച്ചു കൊടുക്കുന്നത്‌. "ഒരാള്‍ എന്നില്‍നിന്നെന്തു ഫലമാവശ്യപ്പെട്ടാലും ഞാനാ ഫലമവര്‍ക്കു നല്‍കും; എന്നാല്‍ അഗ്നിയ്ക്ക്‌ തന്റെ സ്വാഭാവമായ ചൂട്‌ പ്രസരിപ്പിക്കാതിരിക്കാനാവില്ല എന്നറിയുക". അദ്ദേഹം മുക്തിയാണാഗ്രഹിച്ചത്‌. അതിനാല്‍ അദ്ദേഹത്തിനു മുക്തിലഭിക്കും. എന്നാല്‍ സിന്ധുരാജാവ്‌ പൂജചെയ്ത്‌ എന്നോടാവശ്യപ്പെട്ടത്‌ യുദ്ധത്തിലെ വിജയമാണ്‌. അതുകൊണ്ട്‌ വിഥുരഥന്‍ യുദ്ധത്തില്‍ ചരമമടഞ്ഞ്‌ കാലക്രമത്തില്‍ നിങ്ങളോട്‌ ചേര്‍ന്ന്, ഒടുവില്‍ മുക്തിപദം പ്രാപിക്കുന്നതാണ്‌. ഈ സിന്ധു രാജാവ്‌ ചക്രവര്‍ത്തിയായി സാമ്രജ്യം വാഴും.

വസിഷ്ഠന്‍ തുടര്‍ന്നു: ഈ വനിതകള്‍ യുദ്ധം വീക്ഷിച്ചുകൊണ്ടിരിക്കേ കിഴക്ക്‌ ഈ യുദ്ധത്തിന്റെ പര്യവസാനം എന്തെന്നറിയാനുള്ള ആകാംക്ഷയുള്ളതുപോലെ സൂര്യന്‍ ആ യുദ്ധദൃശ്യം ഒന്നൊളിഞ്ഞുനോക്കി. ആയിരം സേനകളാല്‍ ചുറ്റപ്പെട്ട്‌ ഇരു രാജാക്കന്മാരും പൊരിഞ്ഞ യുദ്ധം ചെയ്തു. പല വലിപ്പത്തിലും ആകൃതിയിലും ആയിരുന്നു അവരുടെ പരിഘങ്ങള്‍ . ഭൂമിയില്‍ ഒരു മുനമാത്രം ഉണ്ടായിരുന്ന അമ്പ്‌ ആയിരം മുനകളുമായി ആകാശത്തെത്തി താഴോട്ടുപതിക്കുമ്പോള്‍ പതിനായിരക്കണക്കിനു മുനകളുമായി നാശം വിതച്ചു. രണ്ടു രാജാക്കന്മാരും ഒന്നിനൊന്നു കിടപിടിക്കുന്ന യുദ്ധനിപുണരായിരുന്നു. വിഥുരഥന്‍ ജന്മനാ രണവീരനായിരുന്നു. ശത്രുരാജാവിന്റെ വീര്യം ഭഗവാന്‍ നാരായണന്റെ വരപ്രസാദലബ്ധിയില്‍ നിന്നുണ്ടായതാണ്‌. ഒരുനിമിഷം വിഥുരഥന്‌ ജയമുറപ്പാവുമെന്ന സ്ഥിതിവന്നപ്പോള്‍ രണ്ടാമത്തെ ലീല അതിസന്തോഷത്തോടെ സരസ്വതീ ദേവിക്ക്‌ അദ്ദേഹത്തെ ചൂണ്ടിക്കാട്ടി. എന്നാല്‍ അടുത്ത നിമിഷം ശത്രു രക്ഷപ്പെട്ടു. ഇരുവരും പരസ്പരം നിരോധിക്കുന്ന മട്ടിലുള്ള അസ്ത്രശസ്ത്രങ്ങള്‍ വര്‍ഷിച്ചു പൊരുതി. വിഷാദാസ്ത്രത്തെ നേരിട്ടത്‌ പടയാളികള്‍ ക്കുനേരേ ആവേശാസ്ത്രംതൊടുത്താണ്‌. സര്‍പ്പാസ്ത്രത്തെ ചെറുത്തത്‌ ഗരുഡാസ്ത്രം കൊണ്ട്‌. വരുണാസ്ത്രത്തെ ആഗ്നേയാസ്ത്രം കൊണ്ട്‌. രണ്ടുപേരും വൈഷ്ണവാസ്ത്രം ഉപയോഗിച്ചു. രണ്ടുപേരുടേയും രഥം യുദ്ധത്തില്‍ തകര്‍ന്നുവെങ്കിലും അവര്‍ ഭൂമിയില്‍ നിന്നു യുദ്ധം തുടര്‍ന്നു. വിഥുരഥന്‍ മറ്റൊരു രഥത്തിലേറാന്‍ തുടങ്ങവേ സിന്ധു രാജാവ്‌ അദ്ദേഹത്തെ വെട്ടി വീഴ്ത്തി. താമസിയാതെ വിഥുരഥന്റെ ദേഹം കൊട്ടാരത്തിലെത്തിച്ചു. സരസ്വതീ ദേവിയുടെ സാന്നിദ്ധ്യം ഉള്ളതുകൊണ്ട്‌ ശത്രുസൈന്യത്തിന്‌ കൊട്ടാരത്തില്‍ പ്രവേശിക്കാനായില്ല.
