 
\section{ദിവസം 012}

\slokam{
അശക്ത്തിരാപദസ്തൃഷ്ണാ മൂകതാ മൂഢബുദ്ധിതാ\\
ഗൃധ്നുതാ ലോലതാ ദൈന്യം സർവം ബാല്യേ പ്രവർത്തതേ (1/19/2)\\
}

രാമന്‍ പറഞ്ഞു: അറിവില്ലാത്തതുകൊണ്ട്‌ എല്ലാവരാലും ആഹ്ലാദകരമെന്നു പറയപ്പെടുന്ന ബാല്യം പോലും ക്ലേശം നിറഞ്ഞതാണ്‌ മഹാമുനേ. "നിസ്സഹായത, അപകടങ്ങള്‍ , കൊതികള്‍ , തന്റെകാര്യങ്ങള്‍ പറയാനുള്ള കഴിവില്ലായ്മ, തികഞ്ഞ വിഡ്ഢിത്തങ്ങള്‍ ചെയ്തുകൂട്ടുക, കാര്യങ്ങളില്‍ കളി, ഉറപ്പില്ലായ്മ, ബലഹീനത എല്ലാം ബാല്യത്തിന്റെ പ്രത്യേകതകളാണല്ലോ". 

കുട്ടികള്‍ പെട്ടെന്നു തന്നെ വികാരത്തിനടിപ്പെടുന്നു. അവനില്‍ പെട്ടെന്ന് ദേഷ്യം വരുന്നു. പെട്ടെന്നു കണ്ണുനീരും പൊടിയുന്നു. ഒരുപക്ഷേ മരിക്കാന്‍ പോവുന്നവന്റെയോ വൃദ്ധന്റേയോ, തീരാരോഗത്തിനടിപ്പെട്ടവന്റേയോ മുതിര്‍ന്നവരുടേയോ വേദനയേക്കാള്‍ തീവ്രമാണ്‌ ഒരു കുട്ടിയുടെ വേദനാനുഭവം എന്നു പറയാം. കാരണം അവന്റെ അവസ്ഥ ഒരു മൃഗത്തിന്റേതുമായി സാമ്യമുള്ളതാണ്‌.. മറ്റുള്ളവരുടെ ദയവിലാണ്‌ അവന്‍ ജീവിക്കുന്നത്‌.. കുട്ടിയുടെ ചുറ്റും നടക്കുന്ന എണ്ണിയാലൊടുങ്ങാത്ത സംഭവങ്ങള്‍ അവനെ സംഭ്രമിപ്പിക്കുന്നു. അവന്റെയുള്ളില്‍ അവ ആശങ്കയും ചിന്താക്കുഴപ്പവും, മതിഭ്രമവും ഭയവും ഉണ്ടാക്കുന്നു. കുട്ടികള്‍ പെട്ടെന്നു തന്നെ മറ്റുള്ളവരെ അനുകരിക്കാന്‍ തുടങ്ങും- പ്രത്യേകിച്ച്‌ ദുഷ്ടബുദ്ധികളിലേയ്ക്ക്‌ അവര്‍ അറിയാതെ ആകര്‍ഷിക്കപ്പെടുന്നു. അച്ഛനമ്മമാരുടെ ശിക്ഷയും അച്ചടക്കനടപടികളും ആണ്‌ അതിന്റെ ഫലം. ബാല്യം പരാധീനതയുടെ ഒരു കാലം തന്നെയാണ്‌. .

കുട്ടി കാഴ്ച്ചയില്‍ നിഷ്കളങ്കനായിരിക്കാം. എന്നാല്‍ സത്യത്തില്‍ ദുഷ്ടവാസനകളും, ദോഷങ്ങളും ഞരമ്പുരോഗം ബാധിച്ചപോലുള്ള പെരുമാറ്റവും എല്ലാം അവനിലുമുണ്ട്‌.. പകല്‍സമയം ഒരിരുണ്ട പൊത്തില്‍ ഒളിച്ചിരിക്കുന്ന ഊമനേപ്പോലെയാണവ. മഹാമുനേ ബാല്യകാലം ആഹ്ലാദകരമാണെന്നു പറയുന്ന വിഡ്ഢികളെപ്പറ്റി എനിയ്ക്കു കഷ്ടം തോന്നുന്നു. കരയലും കണ്ണീരൊലിപ്പിക്കലുമാണ്‌ കുട്ടികളുടെ പ്രധാന ജോലിയെന്നു തോന്നുന്നു. അതിന്‌ വേണ്ടതു കിട്ടാത്തപ്പോള്‍ ഹൃദയം പൊട്ടുമാറു കരയുന്നു. സ്കൂളില്‍ പോയാലോ അദ്ധ്യാപകരുടെ ശിക്ഷകളും അനുഭവിക്കണം. അതും ദു:ഖകാരണം തന്നെ.

ഒരു കുട്ടി കരയുമ്പോള്‍ അച്ഛനമ്മമാര്‍ അവനെ സമാധാനിപ്പിക്കാനായി ഈ ലോകം തന്നെ അവനു വാഗ്ദ്ദാനം ചെയ്യുന്നു. അതുമുതല്‍ അവന്‍ ഈ ലോകത്തെ 'മൂല്യ'വത്തായി കാണുന്നു. വസ്തുക്കളില്‍ ആശയുണ്ടാവുകയും ചെയ്യുന്നു. അവര്‍ പറയും "ഞാന്‍ നിനക്ക്‌ അമ്പിളിമാമനെ കളിക്കോപ്പായി പിടിച്ചു തരാം". അതു വിശ്വസിച്ച്‌ കുട്ടി അവന്റെ ഉള്ളംകയ്യില്‍ ചന്ദ്രനെ പിടിക്കാം എന്നു കരുതുന്നു. അങ്ങിനെ മോഹത്തിന്റെ വിത്ത്‌ ആ കൊച്ചു ഹൃദയത്തില്‍ പാകിക്കഴിഞ്ഞു. ചൂടും തണുപ്പും അനുഭവിക്കുമ്പോള്‍ അവന്‌ ഒന്നും ചെയ്യാനാവുന്നില്ല. അപ്പോള്‍ കുട്ടിയുടെ ജീവിതം ഒരു മരത്തേക്കാള്‍ ഭേദമുണ്ടോ? മൃഗങ്ങളേയും പക്ഷികളേയും പോലെ കുട്ടികള്‍ അവനു വേണ്ടതുകിട്ടാന്‍ വെറുതേ കൈനീട്ടുന്നു. അവന്‍ വീട്ടിലെ എല്ലാ മുതിര്‍ന്നവരേയും ഭയപ്പെട്ടു കഴിയുന്നു.
