 
\section{ദിവസം 125}

\slokam{
തീവ്രമന്ദത്വ സംവേഗാദ്‌ ബഹൂത്വാല്പത്വ ഭേദത:\\
വിളംബനേന ച ചിരം ന തു ശക്തിമശക്തിത: (3/103/15)\\
}

വസിഷ്ഠന്‍ തുടര്‍ന്നു: അനന്താവബോധത്തില്‍ പ്രത്യക്ഷമായപ്പോള്‍ മനസ്സ്‌ അതിന്റെ തല്‍സ്വഭാവം പ്രകടമാക്കി. മനസ്സിന് നീണ്ടതിനെ കുറിയതാക്കാനും, കുറിയതിനെ നീണ്ടതാക്കാനും, സ്വയം വിഭിന്നമാക്കാനും, തിരിച്ചും, കഴിയും. ഒരു പദാര്‍ത്ഥം ഏറെ ചെറുതാണെങ്കില്‍ക്കൂടി, മനസ്സ്‌ ഒന്നുതൊട്ടാല്‍ അതിനെ വലുതാക്കി സ്വായത്തമാക്കുന്നു. ഇമവെട്ടുന്ന നേരംകൊണ്ട്‌ എണ്ണമറ്റ ലോകങ്ങളുണ്ടാക്കുന്ന മനസ്സ്‌, അതേ സമയം കൊണ്ടവയെ നശിപ്പിക്കുകയും ചെയ്യുന്നു. ഓരോരോ കഥാപത്രങ്ങളെ അവതരിപ്പിക്കുന്ന കൃതഹസ്തനായ ഒരു നടനെന്നപോലെ മനസ്സ്‌ ഓരോരൊ ഭാവങ്ങള്‍ കൈക്കൊള്ളുകയാണ്‌. അത്‌ അസത്തിനെ സത്തായും തിരിച്ചും തോന്നിപ്പിക്കുന്നു. അതുകൊണ്ടുതന്നെ സുഖദു:ഖങ്ങള്‍ അനുഭവവേദ്യമാവുന്നു. ഉടമസ്ഥതയെപ്പറ്റിയുള്ള ആശങ്കകൊണ്ട്‌ സ്വയമേവ വന്നുചേരുന്നതിനെപ്പോലും കൈക്കലാക്കാനും വശത്താക്കാനും പരിശ്രമിച്ച്‌ മനസ്സതിന്റെ പരിണിതഫലങ്ങള്‍ അനുഭവിക്കുന്നു.

മരങ്ങളിലും ചെടികളിലും മാറ്റങ്ങള്‍ വരുത്താന്‍ ഋതുഭേദങ്ങള്‍ക്ക്‌ സാധിക്കുന്നതുപോലെ മനസ്സ്‌ ചിന്തകളെക്കൊണ്ടും ആശയരൂപീകരണത്തെക്കൊണ്ടും ഒരുവസ്തുവിനെ മറ്റൊന്നാക്കി കാണിക്കുന്നു. സമയവും ദൂരവും എല്ലാം മനസ്സിന്റെ വരുതിയിലാണെന്നര്‍ത്ഥം. "മനസ്സ്‌ അതിന്റെ രൂക്ഷതയോ മന്ദതയോ അനുസരിച്ച്‌; അതുണ്ടാക്കിയതോ മാറ്റം വരുത്തിയതോ ആയ വസ്തുക്കളുടെ വലുപ്പമനുസരിച്ച്‌; കാലക്രമത്തിലോ വല്ലാതെ വൈകിയോ; ചെയ്യേണ്ടതെന്തോ അതുചെയ്യുന്നു. എന്നാല്‍ സ്വയം മനസ്സിന്‌ ഒന്നും ചെയ്യാനാവില്ല എന്നതാണ്‌ സത്യം."

രാമാ, ഇതു മനസ്സിലാക്കാനായി മറ്റൊരു കഥകേട്ടാലും. ഉത്തരപാണ്ഡവം എന്ന ഒരുരാജ്യത്തുള്ള വനപ്രദേശങ്ങളില്‍ മാമുനിമാര്‍ താമസിച്ചിരുന്നു. അവിടത്തെ ഗ്രാമങ്ങള്‍ ഐശ്വര്യസമ്പൂര്‍ണ്ണമായിരുന്നു. സുപ്രസിദ്ധനായ ഹരിശ്ചന്ദ്രന്റെ പിന്മുറക്കാരനായ ലവണനായിരുന്നു ആ രാജ്യം ഭരിച്ചിരുന്നത്‌. അയാള്‍ ഉന്നതകുലജാതനും വീരനും, ദാനനിഷ്ഠനും, ധര്‍മ്മിഷ്ഠനും, രാജപദവിക്ക്‌ സര്‍വ്വദാ യോഗ്യനുമായിരുന്നു. അദ്ദേഹം തന്റെ ശത്രുക്കളെയെല്ലാം കീഴടക്കിയിരുന്നു. ശത്രുക്കളുടെ പിന്‍ഗാമികള്‍പോലും ലവണന്റെ പേരുകേട്ടാല്‍ത്തന്നെ ഞെട്ടിവിറച്ചു.

ഒരുദിവസം പതിവുപോലെ അദ്ദേഹം രാജസഭയില്‍ ആസനസ്ഥനായി. മന്ത്രിമാരും മറ്റുള്ളവരും രാജാവിന്‌ ഉപചാരങ്ങള്‍ അര്‍പ്പിച്ചു  കഴിഞ്ഞപ്പോള്‍ ഒരു ജാലവിദ്യക്കാരന്‍ വന്ന് രാജാവിനെ വണങ്ങി. അയാള്‍ പറഞ്ഞു: ഞാനിപ്പോള്‍ അങ്ങയെ ഒരു വിസ്മയം കാണിക്കാം. അദ്ദേഹം കുറച്ച്‌ മയില്‍പ്പീലികള്‍ വീശിയപ്പോള്‍ അതിസുന്ദരനായ കുതിരയെ നയിച്ചുകൊണ്ടൊരു പടയാളി അവിടെ പ്രത്യക്ഷനായി. അദ്ദേഹം രാജാവിനോട്‌ കുതിരയെ സമ്മാനമായി സ്വീകരിക്കണമെന്ന് അഭ്യര്‍ത്ഥിച്ചു. ജാലവിദ്യക്കാരന്‍ രാജാവിനോട്‌ ആ കുതിരമേല്‍ക്കയറി ലോകം ചുറ്റിവരാനും അഭ്യര്‍ത്ഥിച്ചു. രാജാവും കുതിരയെക്കണ്ടു. അദ്ദേഹം കണ്ണടച്ചനങ്ങാതെയവിടെയിരുന്നു. ഇതുകണ്ട്‌ സഭാവാസികളും നിശ്ശബ്ദരായി. സഭയില്‍ പരമശാന്തി കളിയാടി. രാജാവിന്റെ പ്രശാന്തതയെ ഭഞ്ജിക്കാന്‍ ആര്‍ക്കും ധൈര്യമുണ്ടായില്ല.

