\newpage
\section{ദിവസം 140}

\slokam{
ത്വത്താഹന്താത്മതാ തത്താ സത്താസത്താ ന കാചന\\
ന ക്വചിദ്ഭേദകലനാ ന ഭവോ ന ച രഞ്ജനാ (3/119/21)\\
}

വസിഷ്ഠൻ തുടർന്നു: അത്മാവ് അജ്ഞതകൊണ്ട് അഹംകാരത്തിൽ ആമഗ്നമായി സ്വയം വേറിട്ടു നില്ക്കുകയാണ്‌..  സ്വർണ്ണമോതിരം, സ്വയം അടിസ്ഥാനപരമായി മൂല്യവസ്തുവായ സ്വർണ്ണമാണെന്നതു മറന്ന് മോതിരമായതോടെ  ‘അയ്യോ! എന്റെ സ്വർണ്ണത്വം’ നഷ്ടമായി എന്നു വിലപിക്കുന്നതുപോലെയാണിത്.

രാമൻ ചോദിച്ചു: ഭഗവൻ, ഈ അജ്ഞാനവും അഹംകാരവും ആത്മാവിലുദിക്കുന്നതെങ്ങിനെ?

വസിഷ്ഠൻ പറഞ്ഞു: രാമാ, നമ്മുടെ ചോദ്യങ്ങൾ യാഥാർത്ഥ്യമായതിനെ, ഉണ്മയെക്കുറിച്ചുള്ളതാകണം. അയാഥാർത്ഥ്യമായതിനെക്കുറിച്ചാകരുത്. സത്യത്തിൽ ‘മോതിരം, സ്വർണ്ണം, പരിമിതപ്പെട്ട അഹംകാരം’ എന്നിവയൊന്നും നിലനിൽക്കുന്ന വസ്തുക്കളല്ല. സ്വർണ്ണവ്യാപാരി മോതിരത്തെ സ്വർണ്ണത്തിന്റെ തൂക്കം നോക്കി വിൽക്കുന്നു. കാരണം അയാൾക്ക് അത് സ്വർണ്ണം മാത്രമാണ്‌..  മോതിരത്തിലെ ‘മോതിരത്വം’ ചർച്ചയ്ക്കെടുക്കുകയാണെങ്കിൽ, അത് അനന്തതയിലെ പരിമിതപ്പെട്ട നാമരൂപത്തിനെപ്പറ്റിയുള്ള ചർച്ചപോലെയാണ്‌..  എന്നാലത് `വന്ധ്യയുടെ`  പുത്രൻ എന്നു പറയും പോലെ അസംബന്ധവുമാണ്‌.. 

അയാഥാർത്ഥ്യമായതിന്റെ നിലനിൽപ്പെന്നതും അയാഥാർത്ഥ്യമാണ്‌..  അത് അവിദ്യയിൽ ഉദിച്ച് അന്വേഷണത്താൽ മറയുന്നു. അജ്ഞാനംകൊണ്ട് മുത്തുച്ചിപ്പിയിൽ ഒരുവൻ വെള്ളി കാണുന്നു. എന്നാൽ അതിന്‌ വെള്ളി എന്ന ലോഹമായി ഒരു നിമിഷം പോലും അസ്തിത്വമുണ്ടാവുകയില്ല. അത് മുത്തുച്ചിപ്പിയാണെന്ന സത്യം മനസ്സിലുറയ്ക്കുംവരെ അജ്ഞത വിട്ടുപോവുകയില്ല എന്നു മാത്രം. മണലിൽ നിന്നും എണ്ണയൂറ്റാനാവാത്തതുപോലെ, സ്വർണ്ണമോതിരത്തിൽ നിന്നും സ്വർണ്ണം മാത്രമേ മൂല്യവത്തായി എടുക്കാനാവൂ എന്നപോലെ, വസ്തുത ഒന്നേയുള്ളു. അനന്താവബോധം മാത്രമാണ്‌ എല്ലാ നാമരൂപങ്ങളിലും പ്രഭയോടെ ജ്വലിച്ചു നിൽക്കുന്നത്. അങ്ങിനെയൊക്കെയാണ്‌ അവിദ്യയുടെ കാര്യം. മോഹവിഭ്രാന്തിയും ലോകമെന്ന പ്രക്രിയയും അപ്രകാരം തന്നെ. അസ്തിത്വമില്ലെങ്കിലും അഹംകാരമെന്ന മിഥ്യാധാരണ ഉണ്ടാവുന്നുണ്ട് . സത്യത്തിൽ അനന്തമായ ആത്മാവിൽ അഹംകാരം നിലനിൽക്കുന്നില്ല.

അനന്താത്മാവിൽ സൃഷ്ടാവില്ല, സൃഷ്ടികളില്ല, ലോകങ്ങളില്ല, സ്വർഗ്ഗമില്ല, രാക്ഷസരില്ല, ശരീരങ്ങളില്ല, ധാതുക്കളില്ല, കാലമില്ല, അസ്തിത്വമില്ല, നാശവുമില്ല. ‘നീ’ യും ‘ഞാനും’ ഇല്ല. ആത്മാവില്ല, ‘അത്’ ഇല്ല, ‘സത്യം’ ഇല്ല, ‘അസത്യം’ ഇല്ല, ഇതൊന്നുമില്ല, നാനാത്വമെന്ന ധാരണയില്ല, ധ്യാനമില്ല, സുഖാസ്വാദനവുമില്ല. ഉള്ളത് പരമശാന്തി മാത്രം. അതാണ്‌ വിശ്വമെന്നറിയപ്പെടുന്നത്. അതിന്‌ ആദിമദ്ധ്യാന്തങ്ങൾ ഇല്ല. അതെല്ലായ്പ്പോഴും ഉള്ളതാണ്‌..  മനസാ വാചാ ഉള്ള ധാരണകൾക്കെല്ലാമതീതമാണത്. സത്യത്തിൽ സൃഷ്ടിയെന്നത് ഇല്ല. അനന്തത തന്റെ അനന്താവസ്ഥയെ ഒരിക്കലും ഉപേക്ഷിച്ചിട്ടില്ല. ‘അത്’ ഒരിക്കലും ‘ഇത്’ ആയിട്ടില്ല. അത് ചലനമില്ലാത്ത അനന്തസമുദ്രം പോലെയത്രേ. അത് സ്വയം പ്രഭമായ സൂര്യനെപ്പോലെയാണെങ്കിലും അതിനു കർമ്മങ്ങളില്ല. 

അജ്ഞതയിൽ പരമപുരുഷനെ വസ്തുപ്രപഞ്ചമായി കാണുന്നു. ആകാശം സ്ഥിതിചെയ്യുന്നത് ആകാശത്തിൽത്തന്നെ! ‘സൃഷ്ടിക്കപ്പെട്ടവ’ എല്ലാം, ബ്രഹ്മത്തിൽ സ്തിതിചെയ്യുന്ന ബ്രഹ്മം തന്നെ. കണ്ണാടിയിൽ പ്രതിഫലിക്കുന്ന നഗരദൃശ്യത്തിലെ രണ്ടു വസ്തുക്കൾ തമ്മിലുള്ള അകലം പോലെയാണ്‌, അകലെ, അരികെ, വൈവിദ്ധ്യം, അവിടെ, ഇവിടെ എന്നെല്ലാം നാം വിവക്ഷിക്കുന്ന മിഥ്യാ ധാരണകൾ.

