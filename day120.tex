 
\section{ദിവസം 120}

\slokam{
യത: കുതശ്ചിദുത്പന്നം ചിത്തം യത്‌കിംചിദേവ ഹി\\
നിത്യമാത്മ വിമോക്ഷായ യോജയേദ്യത്നതോനഘ: (3/98/1)\\
}

വസിഷ്ഠന്‍ തുടര്‍ന്നു: "അല്ലയോ രാമ: മനസ്സിന്റെ ഉദ്ഭവം എന്തുതന്നെയായിരുന്നാലും എന്തുതന്നെയാവുമായിരുന്നുവെന്നാലും അതിനെ സ്വപ്രയത്നത്താല്‍ മുക്തിമാര്‍ഗ്ഗത്തിലേയ്ക്ക്‌ നയിക്കണം". ശുദ്ധമനസ്സ്‌ വസനകള്‍ ഇല്ലാത്തതാണ്‌.. അതുകൊണ്ട്‌ ആ മനസ്സിന്‌ ആത്മജ്ഞാനം പ്രാപ്തമാണ്‌.. വിശ്വം മുഴുവന്‍ മനസ്സില്‍ ഉള്‍ക്കൊണ്ടിരിക്കുന്നതിനാല്‍ ബന്ധനവും മോക്ഷവുമെല്ലാം മനസ്സിനുള്ളില്‍ത്തന്നെയാണ്‌..

ഇതിനെപ്പറ്റിയൊരു കഥ ബ്രഹ്മാവു പറഞ്ഞ്‌ ഞാന്‍ കേട്ടിരിക്കുന്നു. ശ്രദ്ധിച്ചാലും: ഒരിടത്ത്‌ വലിയൊരുകാടുണ്ടായിരുന്നു. അനേകകോടി ചതുരശ്രമൈയില്‍ വിസ്തീര്‍ണ്ണം ഈ കാടിനെ സംബന്ധിച്ച്‌ ആകാശത്ത്‌ ഒരണു എന്നപോലെയായിരുന്നു. ആ കാട്ടില്‍ ആയിരം കയ്യുകളുംകാലുകളുമുള്ള ഒരേയൊരു മനുഷ്യന്‍ മാത്രം ജീവിച്ചു വന്നു. അദ്ദേഹം തികച്ചും അസ്വസ്ഥനായിരുന്നു. അയാളുടെ കയ്യിലിരുന്ന ഗദകൊണ്ടയാള്‍ സ്വയം പീഢിപ്പിച്ചു. എന്നിട്ട്‌ ആ താഡനത്തെ ഭയന്ന് പരവശനായി അയാള്‍ ഓട്ടം തുടങ്ങി. അങ്ങിനെ അയാള്‍ ഒരുപൊട്ടക്കിണറ്റില്‍ വീണു. പിന്നെ അയാള്‍ അതില്‍നിന്നു പുറത്തുവന്നു; വീണ്ടും ഗദകൊണ്ടുള്ള താഡനമേറ്റ്‌ ഭയന്ന് ഓടി മറ്റൊരു കാട്ടില്‍കയറി. അവിടെനിന്നും രക്ഷപ്പെട്ട്‌ വീണ്ടും സ്വന്തം ഗദാപ്രഹരമേറ്റ്‌ വലഞ്ഞോടി ഒരു വാഴത്തോപ്പിലെത്തി. അവിടെയും  ഭയപ്പെടാന്‍ മറ്റാരുമുണ്ടായിരുന്നില്ലെങ്കിലും അയാള്‍ കണ്ണീരൊഴുക്കി ഉറക്കെ വിലപിച്ചു. പണ്ടേപ്പോലെ അയാള്‍ സ്വയം പ്രഹരിച്ച്‌ ഭയന്ന് ഓടിക്കൊണ്ടിരുന്നു. ഞാന്‍ ഇതെല്ലാം എന്റെ ദിവ്യദൃഷ്ടിയാല്‍ ദര്‍ശിച്ചിട്ട്‌ അവനെ ഒരു നിമിഷം പിടിച്ചു നിര്‍ത്തി ചോദിച്ചു: 'നീ ആരാണ്‌?' അയാള്‍ തീവ്രദു:ഖിതനാകയാല്‍ എന്നെ ശത്രുവെന്നു വിളിച്ചു ആദ്യം ഉറക്കെ കരഞ്ഞു; പിന്നീട്‌ ഉറക്കെ ചിരിച്ചു. അതുകഴിഞ്ഞ്‌ അയാള്‍ തന്റെ ശരീരം, അവയവങ്ങള്‍ ഓരോന്നായി ഉപേക്ഷിക്കാന്‍ തുടങ്ങി. ഉടനേ തന്നെ അവിടെ മറ്റൊരുവന്‍ ആദ്യത്തെയാളെപ്പോലെതന്നെ സ്വയം പ്രഹരിച്ചും ഉറക്കെ നിലവിളിച്ചും ഓടിവരുന്നതു കണ്ടു. ഞാന്‍ അവനേയും പിടിച്ചു നിര്‍ത്തി ചോദ്യം ചെയ്യവേ അവന്‍ എന്നെ പുലഭ്യം പറഞ്ഞുകൊണ്ട്‌ അവന്റെ വഴിക്ക്‌ ഓടിപ്പോയി. ഇങ്ങിനെ ഞാന്‍ പലരേയും കണ്ടു. ചിലര്‍ എന്റെ വാക്കുകള്‍ കേട്ട്‌ അവരുടെ പൂര്‍വ്വകര്‍മ്മങ്ങള്‍ ഉപേക്ഷിച്ച്‌ പ്രബുദ്ധരായിത്തീര്‍ന്നു. മറ്റുചിലര്‍ എന്നെ അവഗണിച്ചു; ചിലര്‍ വെറുക്കുകപോലുംചെയ്തു. ചിലര്‍ അന്ധകൂപത്തില്‍നിന്നും കാട്ടില്‍നിന്നും പുറത്തുവന്നതേയില്ല. രാമാ, അങ്ങിനെയാണീ ലോകമെന്ന വന്‍വിപിനം. ആരും പ്രശാന്തമായൊരിടം അവിടെ കണ്ടെത്തുന്നില്ല, അവര്‍ ഏതു ജീവിതരീതി അവലംബിച്ചാലും ഇതിനു വ്യത്യാസമില്ല. ഇപ്പോഴും ഈലോകത്തില്‍ നിനക്ക്‌ അത്തരം ആള്‍ക്കാരെ കാണാം. നീ തന്നെ അത്തരം ഭ്രമദൃശ്യങ്ങളും അജ്ഞതനിറഞ്ഞ ജീവിതവും കണ്ടിട്ടുണ്ടല്ലോ. നിന്റെ ചെറുപ്പവും അജ്ഞതയുംകാരണം നിനക്കത്‌ അപ്പോള്‍ മനസ്സിലായില്ല എന്നേയുള്ളു.  

