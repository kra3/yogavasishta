 
\section{ദിവസം 048}

\slokam{
അസത്യം സത്യസംകാശം ബ്രഹ്മാസ്തേ ജീവശബ്ദവത്\\
ഇത്ഥം സ ജീവശബ്ദാർത്ഥ: കലനാകുലതാം ഗത: (3/13/33)\\
}

വസിഷ്ഠന്‍ തുടര്‍ന്നു: ജീവന്‍ (ജീവാത്മാവ്‌) എങ്ങിനെ ഈ ശരീരത്തില്‍ നിവസിക്കാനിടയായി എന്നു ഞാന്‍ ഇനി പറഞ്ഞു തരാം. ജീവന്‍ ചിന്തിച്ചു: 'എനിക്ക്‌ അണു സ്വഭാവവും സ്ഥാനവും ആണുള്ളത്‌'. അങ്ങിനെ ജീവന്‍ അണുസ്വഭാവം കൈക്കൊണ്ടു. തെറ്റായി സങ്കല്‍പ്പിച്ചതിനാലാണ്‌ ജീവന്‍ അണുഭാവത്തില്‍ പ്രകടമായത്‌. ഒരുവന്‍ സ്വപ്നത്തില്‍ താന്‍ മരിച്ചതായും തനിക്ക്‌ മറ്റൊരു ശരീരമുള്ളതായും കാണുമ്പോലെ അതീവ സൂക്ഷ്മമായ ശുദ്ധ അവബോധഭാവത്തിലുള്ള ജീവാത്മാവ്‌ ഘനഭാവത്തോട്‌ താദാത്മ്യം പ്രാപിക്കുകമൂലം ഘനസാന്ദ്രമാര്‍ന്ന് നിലകൊണ്ടു.

കണ്ണാടിയില്‍ നിഴലിച്ചു കണുന്ന മല കണ്ണാടിയില്‍ നിലകൊള്ളുന്നതായി തോന്നും പോലെ ജീവന്‍ പുറത്തുള്ള വസ്തുക്കളേയും പ്രവര്‍ത്തനങ്ങളേയും പ്രതിഫലിപ്പിക്കുന്നു. അചിരേണ, ആ വസ്തുക്കള്‍ തനിക്കുള്ളിലാണെന്നു ചിന്തിക്കാന്‍ തുടങ്ങുന്നു. സ്വയം താനാണ്‌ 'കര്‍ത്താവ്‌' എന്നും അനുഭവങ്ങളുടെ 'ഭോക്താവ്‌' എന്നും കരുതുന്നു. ജീവനു കാണാന്‍ തോന്നിയപ്പോള്‍ ഘനശരീരത്തില്‍ കണ്ണുകള്‍ ഉണ്ടായി. അതുപോലെ ത്വക്ക്‌, ചെവി, നാക്ക്‌, മൂക്ക്‌, കര്‍മ്മേന്ദ്രിയങ്ങള്‍ എന്നിവ അതാത്‌ ആഗ്രഹങ്ങള്‍ ജീവനില്‍ ഉയരുകമൂലം സംജാതമായി. അവബോധമെന്ന അതിസൂക്ഷ്മശരീരമായ ജീവന്‍ , ബാഹ്യമായി ശാരീരികാനുഭവങ്ങളും, അന്തരീകമായി മാനസീകാനുഭവങ്ങളും സങ്കല്‍പ്പിച്ചാണ്‌ ഈ ശരീരത്തില്‍ നിവസിക്കുന്നത്‌. "അങ്ങിനെ അയദാര്‍ത്ഥമായതിനെ യാദാര്‍ത്ഥ്യമായിക്കണ്ട്‌ പരബ്രഹ്മം, സംഭ്രാന്തിയോടെ ജീവഭാവത്തില്‍ പ്രത്യക്ഷമായിരിക്കുന്നു."

പരിമിതപ്രഭാവവും ശരീരവുമുള്ള ജീവനെന്നു സ്വയം കരുതി പരബ്രഹ്മം പുറം ലോകത്തെ അജ്ഞാനത്തിന്റെ മറയിലൂടെ വിഷയവസ്തുവായി മനസ്സിലാക്കുന്നു. ചിലര്‍ അവനെ ബ്രഹ്മാവെന്നും മറ്റുചിലര്‍ വേറെപേരുകളിലും വിളിക്കുന്നു. ഇങ്ങിനെ ജീവന്‍ സ്വയം 'അതുമിതും' ആയി സങ്കല്‍പ്പിച്ച്‌ പ്രത്യക്ഷലോകമെന്ന മായാജാലവുമായി ബന്ധിക്കപ്പെടുന്നു. എന്ന്നാല്‍ ഇതെല്ലാം സങ്കല്‍പ്പങ്ങളും ചിന്തകളും മാത്രമാണ്‌. ഇപ്പോഴും ഒന്നും സൃഷിക്കപ്പെട്ടിട്ടില്ല; ശുദ്ധവും അനന്തവുമായ ആകാശം മാത്രമേ ഉണ്മയായുള്ളു. സൃഷ്ടാവായ ബ്രഹ്മാവിന്‌ മഹാപ്രളയത്തിനു മുന്‍പത്തേപ്പോലെ ലോകസൃഷ്ടിചെയ്യാന്‍ കഴിയില്ല, കാരണം ബ്രഹ്മാവ്‌ അപ്പോഴേ മുക്തിപദം പ്രാപിച്ചിരുന്നു. വിശ്വാവബോധം മാത്രമേ അന്നുമിന്നും ഉള്ളു. അതില്‍ ലോകങ്ങളില്ല, സൃഷ്ടികളുമില്ല. ബോധ സത്തിൽ സ്വയം പ്രതിഫലിക്കുന്നതാണ്‌ സൃഷ്ടികളായി കാണപ്പെടുന്നത്‌. തികച്ചും അയദാര്‍ത്ഥമായ ദു:സ്വപ്നം ഭവിഷത്തുകള്‍ ഉണ്ടാക്കുന്നതുപോലെ അവിദ്യയില്‍ ഈലോകം ഉണ്മയാണെന്നുള്ള തോന്നലുണ്ടാവുന്നു. ശരിയായ അറിവുണരുമ്പോള്‍ ഈ സാങ്കല്‍പ്പികലോകം മറഞ്ഞുപോകുന്നു. 
