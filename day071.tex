 
\section{ദിവസം 071}

\slokam{
മൃതിർജന്മന്യസദ്രൂപ മൃത്യാം ജന്മാപ്യസന്മയം\\
വിശരേദ്വിശരാരുത്വാദനുഭൂതേശ്ച രാഘവ (3/44/26)\\
}

വസിഷ്ഠന്‍ തുടര്‍ന്നു: ആ സമയത്ത്‌ രാജ്ഞി അവിടെ വന്നുചേര്‍ന്നു. രാജ്ഞിയുടെ മുഖ്യതോഴി രാജാവിനോട്‌ ഇങ്ങിനെ പറഞ്ഞു: മഹാ രാജാവേ, അന്ത:പുരത്തിലെ സ്തീകളെയെല്ലാം ശത്രുക്കള്‍ വലിച്ചിഴച്ചുകൊണ്ടുപോവുകയാണ്‌. വിവരിക്കാനരുതാത്ത ഈ അതിദുരിതത്തില്‍നിന്നും ഞങ്ങളെ രക്ഷിക്കാന്‍ അങ്ങല്ലാതെ ആരുമില്ല.

രജാവ്‌ സരസ്വതീ ദേവിയോട്‌ വിടവാങ്ങി: ഈ ശത്രുവിനെ കീഴ്പ്പെടുത്താന്‍ ഞാന്‍ തന്നെ യുദ്ധക്കളത്തിലിറങ്ങാന്‍ പോവുന്നു. എന്റെ രാജ്ഞി അവിടുത്തെ സേവിക്കാന്‍ ഇവിടെയുണ്ടല്ലോ. അദ്ഭുതപരവശയായി ലീല രാജ്ഞിയെ നോക്കി. രാജ്ഞി, തന്റെ കൃത്യമായ പ്രതിരൂപം തന്നെ!

ലീല, സരസ്വതീ ദേവിയോടു ചോദിച്ചു: ദേവീ, ഇതെങ്ങിനെ സംഭവിച്ചു? രാജ്ഞി എന്നേപ്പോലെ തന്നെയിരിക്കുന്നു! എന്റെ യൌവ്വനത്തില്‍ ഞാന്‍ എങ്ങിനെയിരുന്നുവോ അങ്ങിനെയാണവരിപ്പോള്‍ . ഇതിന്റെ രഹസ്യം എന്താണ്‌? കൂടാതെ ഈ മന്ത്രിമാര്‍ എല്ലാം ഞങ്ങളുടെ കൊട്ടാരത്തിലുണ്ടായിരുന്നവര്‍ തന്നെയെന്നു തോന്നുന്നു. അവര്‍ നമ്മുടെ ചിന്തകളുടെ പ്രതിഫലനം മാത്രമാണെങ്കില്‍ അവര്‍ സചേതനരാണോ? അവരും അവബോധമുള്ളവരാണോ?

സരസ്വതി പറഞ്ഞു: ലീലേ, ഒരാളിന്റെയുള്ളില്‍ എന്തെന്തുകാഴ്ച്ചകള്‍ ഉയര്‍ന്നുണരുന്നുവോ അവയെല്ലാം ക്ഷണനേരത്തില്‍ അനുഭവങ്ങളാകുന്നു. ബോധം, സ്വയം വിഷയമായി (അറിയപ്പെടുന്ന വസ്തുവായി) മാറിയപോലെയാണത്‌. ബോധമണ്ഡലത്തില്‍ ലോകമെന്ന കാഴ്ച്ച ഉയരുമ്പോള്‍ ആ ക്ഷണത്തില്‍ അങ്ങിനെയായിത്തീരുകയാണ്‌. കാലം, ദൂരം, സമയദൈര്‍ഘ്യം, വസ്തുനിഷ്ഠത എന്നിവയെല്ലാം ഉണ്ടാവുന്നത്‌ പദാര്‍ത്ഥങ്ങളില്‍ നിന്നാവാന്‍ വയ്യ. കാരണം അങ്ങിനെയെങ്കില്‍ അവയും പദാര്‍ത്ഥങ്ങളാവണമല്ലോ. ഒരുവന്റെ ബോധതലത്തിലെ പ്രതിഫലനം പുറത്തും അതേപോലെ ദീപ്തമായി കാണപ്പെടുന്നു. ജാഗ്രദ വസ്ഥയില്‍ വസ്തുനിഷ്ഠമായി, സത്തായി, കാണപ്പെടുകയും അനുഭവിക്കുകയും ചെയ്യുന്ന പുറംലോകം സ്വപ്നത്തിലെ അനുഭവങ്ങളേക്കാള്‍ കൂടിയ യാഥാര്‍ത്ഥ്യമൊന്നുമല്ല. ഉറങ്ങുമ്പോള്‍ ലോകമില്ല. ഉണരുമ്പോള്‍ സ്വപ്നവുമില്ല! "മരണം ജീവനു വിപരീതമാണ്‌. ജീവിച്ചിരിക്കുമ്പോള്‍ മരണമില്ല, അതിനു നിലനില്‍പ്പില്ല. മരണത്തിനപ്പോള്‍ അസ്തിത്വമില്ല. മരണത്തിലോ, ജീവനു നിലനില്‍പ്പില്ല. കാരണം, അതത്‌ അനുഭവങ്ങളെ ചേര്‍ത്തുവയ്ക്കുന്നതെന്തോ അതിന്റെ അഭാവമാണ്‌ മറ്റേതില്‍ . അത്‌ സത്തോ അസത്തോ എന്നു പറയാന്‍ വയ്യ. ഒരുകാര്യം മാത്രം ഉറപ്പിക്കാം- എല്ലാറ്റിന്റേയും അടിസ്ഥാനം മാത്രമാണ്‌ സത്തായിട്ടുള്ളതെന്ന്. ബ്രഹ്മത്തില്‍ ഒരു വാക്കായി, ആശയമായി ഈ വിശ്വം മുഴുവന്‍ സ്ഥിതിചെയ്യുന്നു. അതു സത്യമോ അസത്യമോ അല്ല. കയറിലെ പാമ്പ്‌ സത്യമോ അസത്യമോ അല്ലാത്തതുപോലെയാണത്‌. ജീവന്റെ അനുഭവങ്ങള്‍ അപ്രകാരമത്രേ. സ്വന്തം ആശയാഭിലാഷങ്ങളെയാണ്‌ ജീവന്‍ അനുഭവിക്കുന്നത്‌. താന്‍ നേരത്തേ അനുഭവിച്ച ചില അനുഭവങ്ങള്‍ ഭാവനയില്‍ കൊണ്ടുവരുന്നു; ചില പുതിയ അനുഭവങ്ങളും സങ്കല്‍പ്പിക്കുന്നു. ചിലത്‌ സമാന സ്വഭാവമുള്ളവയും മറ്റുചിലത്‌ വിചിത്രവുമായിരിക്കും. ഇവയെല്ലാം യഥാര്‍ത്ഥ വിശകലനത്തില്‍ അസത്യമാണെങ്കിലും അവ സത്യമായി കാണപ്പെടുന്നു (അനുഭവപ്പെടുന്നു). ഈ മന്ത്രിമാരുടേയും മറ്റുള്ളവരുടേയും കാര്യം അങ്ങിനെയാണ്‌. ഇവിടെക്കാണുന്ന ലീലയും ബോധതലത്തിലെ വെറും പ്രതിഫലനം കൊണ്ടാണുണ്ടായിരിക്കുന്നത്‌. അപ്രകാരം തന്നെയാണ്‌ നീയും ഞാനുമൊക്കെ. ഈ അറിവിന്റെ നിറവില്‍ നീ പ്രശാന്തയായാലും.
