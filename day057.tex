\newpage
\section{ദിവസം 057}

\slokam{
യഥൈതത് പ്രതിഭാമാത്രം ജഗത് സർഗ്ഗാവഭാസനം\\
തഥൈതത് പ്രതിഭാമാത്രം ക്ഷണകൽപ്പാവഭാസനം (3/20/29)\\
}

സരസ്വതി തുടര്‍ന്നു: ലീലേ, നിന്റെ ഭവനവും നീയും ഞാനും ഇക്കാണായതെല്ലാം ശുദ്ധബോധമല്ലാതെ മറ്റൊന്നുമല്ല.  നിന്റെ വീട്‌ ആ വസിഷ്ഠന്റെ വീട്ടിനകത്തായിരുന്നു. അദ്ദേഹത്തിന്റെ ജീവാകാശത്തിലാണ്‌ നദികളും പര്‍വ്വതങ്ങളും മറ്റും നിലകൊണ്ടത്‌. ഇതെല്ലാം സൃഷ്ടിച്ചിട്ടും ആ മഹാത്മാവിന്റെ ഭവനം പഴയപോലെതന്നെ മാറ്റമേതുമില്ലാതെ ശേഷിച്ചു. തീര്‍ച്ചയായും ഓരോ അണുവിലും ലോകങ്ങളും അവയ്ക്കുള്ളില്‍ ലോകങ്ങളുമുണ്ട്‌.

ലീല ചോദിച്ചു: ദേവീ, അവിടുന്നുപറഞ്ഞു, വെറും എട്ടുനാള്‍ മുന്‍പു മാത്രമേ ആ ദിവ്യപുരുഷന്‍ ദേഹംവെടിഞ്ഞുള്ളു എന്ന്. എന്നാല്‍ ഞാനും എന്റെ പ്രിയനും ഏറെ നാള്‍ ജീവിച്ചുവല്ലോ. ഇതിലെ വിരുദ്ധതയെങ്ങിനെയാണു നാം മനസ്സിലാക്കേണ്ടത്‌?

സരസ്വതി പറഞ്ഞു: ലീലേ, ആകാശത്തിന്‌ പരിമിതമായ അളവുകള്‍ ഇല്ലാത്തതുപോലെ കാലത്തിനും പരിമിതമായ ദൈര്‍ഘ്യങ്ങള്‍ ഇല്ല. "പ്രപഞ്ചവും അതിലെ സൃഷ്ടിജാലവും എപ്രകാരം വെറും ദൃശ്യം മാത്രമാണോ അപ്രകാരം ഒരു ഞൊടിയിടയും ഒരു യുഗവുമെല്ലാം സങ്കല്‍പ്പം മാത്രം. അതൊന്നും സത്യമല്ല." കണ്ണിമ ചിമ്മുന്ന നേരത്തിനിടയ്ക്ക്‌ ജീവന്‍ മരണാനുഭവത്തിലൂടെ കടന്നുപോവുന്നു. അതുവരെയുണ്ടായ കാര്യങ്ങള്‍ എല്ലാം മറന്നുപോകുന്നു. അനന്തമായ അവബോധത്തില്‍ ജീവന്‍ , താന്‍ അതാണ്‌ , ഇതാണ്‌ , ഇതെന്റെ പുത്രനാണ്‌ , എനിക്കിത്ര വയസ്സായി, എന്നെല്ലാം ചിന്തിക്കുന്നു. ഈ ജീവിതത്തിലെ അനുഭവങ്ങളും മറ്റേ ജീവിതത്തിലെ അനുഭവങ്ങളും തമ്മില്‍ വ്യത്യാസം ഒന്നുമില്ല. എല്ലാം അനന്താവബോധത്തിലെ ചിന്താരൂപങ്ങള്‍ മാത്രം. ഒരേ സമുദ്രത്തിലെ രണ്ടു തിരകള്‍ . ഈ ലോകങ്ങള്‍ ഒരിക്കലും സൃഷ്ടിക്കപ്പെടാത്തതാകയാല്‍ അവയ്ക്കു നാശവുമില്ല. അങ്ങിനെയാണ്‌ നിയമം. അവയുടെയെല്ലാം ശരിയായ സ്വഭവം ബോധമാണ്‌.

സ്വപ്നത്തില്‍ ,  ജനനമരണങ്ങളും ബന്ധങ്ങളും ചെറിയൊരു കാലയളവില്‍ നാം അനുഭവിച്ചു തീര്‍ക്കുന്നു. വെറുമൊരു രാത്രിയിലെ വിരഹം കമിതാക്കള്‍ക്ക്‌ ഒരു യുഗം പോലെ നീണ്ടതാണല്ലോ. ജീവന്‍ കണ്ണുചിമ്മുന്ന നേരത്തിനിടയ്ക്ക്‌ അനുഭവവേദ്യമായതും അല്ലാത്തതുമായ വിഷയങ്ങളെക്കുറിച്ച്‌ ചിന്തിച്ചു കൂട്ടുന്നു. അവയെല്ലാം (ലോകം) യാഥാര്‍ത്ഥ്യമാണെന്നു സങ്കല്‍പ്പിക്കുകയും ചെയ്യുന്നു. താന്‍ അനുഭവിച്ചിട്ടില്ലാത്ത വിഷയങ്ങളും അവന്റെമുന്നില്‍ സ്വപ്നദൃശ്യങ്ങളേപ്പോലെ പ്രകടമാവുന്നു. ഈ ലോകവും സൃഷ്ടികളുമെല്ലാം സ്മൃതിയാണ്‌, സ്വപ്നമാണ്‌. ദൂരം, കാലം, നിമിഷം, യുഗം, എല്ലാം മോഹവിഭ്രമങ്ങള്‍ . സ്മൃതി-ഓര്‍മ്മ എന്നത്‌ ഒരു തരം അറിവാണ്‌. മറ്റൊരു തരം അറിവുള്ളത്‌ ഭൂതകാലാനുഭവങ്ങളുടെ ഓര്‍മ്മയെ ആശ്രയിച്ചുള്ളതല്ല. അത്‌ യാദൃശ്ചികമായി ഒരണുവും ബോധവും സന്ധിക്കുമ്പോള്‍ ഉണ്ടാവുന്ന ഫലപ്രാപ്തിയത്രേ. പ്രപഞ്ചത്തിന്റെ അയാഥാര്‍ത്ഥ്യത സമ്യക്കായി അറിയുക എന്നതാണ്‌ മുക്തി. അത്‌ അഹംകാരത്തേയും വിശ്വത്തേയും വെറുതേ നിരാകരിക്കുന്നതില്‍ നിന്നും ഭിന്നമാണ്‌ കാരണം ആ അറിവ്‌ അപൂര്‍ണ്ണമാണ്‌. എല്ലാം ശുദ്ധാവബോധം മാത്രമാണെന്നു സാക്ഷാത്കരിക്കുകയാണ്‌ മുക്തി.
