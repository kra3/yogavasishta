\section{ദിവസം 168}

\slokam{
നൈകട്യാതിശയാദ്യദ്വദർപണം ബിംബവദ്ഭവേത്\\
അഭ്യാസാതിശയാത്തദ്വത്തേ സാഹംകാരതാം ഗതാ: (4/29/6)\\
}

വസിഷ്ഠൻ തുടർന്നു: അപ്രകാരം അരുളിച്ചെയ്ത് ബ്രഹ്മദേവൻ അപ്രത്യക്ഷനായി. ദേവന്മാർ അവരുടെ ഗൃഹങ്ങളില്‍പ്പോയി വിശ്രമിച്ച് അസുരന്മാരെ നേരിടാനുള്ള ഒരുക്കങ്ങൾ ചെയ്തു. പുതുതായി ദേവാസുരന്മാർ തമ്മിലുണ്ടായ യുദ്ധം മുന്‍പത്തേതിലും അതിഘോരമായിരുന്നു. എല്ലായിടത്തും കൊടിയ നാശമുണ്ടായി. യുദ്ധത്തിൽ തുടർച്ചയായി ആമഗ്നമാവുകമൂലം ഈ മൂന്നു രാക്ഷസവീരന്മാരിൽ ‘ഞാൻ’ എന്നൊരു ഭാവം ഉണ്ടായി. “ഒരു കണ്ണാടി പ്രതിഫലിപ്പിക്കുന്നത് അതിനോട് ചേർത്തുവച്ച വസ്തുക്കളെയാണല്ലോ. അതുപോലെ ഒരുവന്റെ പെരുമാറ്റരീതി ബോധമണ്ഡലത്തിൽ അഹംഭാവമായി പ്രതിഫലിക്കുന്നു.” എന്നാൽ ഈ പെരുമാറ്റരീതിയെ ദൂരെ മാറ്റി ബോധത്തിൽ നിന്നും അകലത്തിൽ നിർത്തിയാൽ അവയുമായി താദാത്മ്യം പ്രാപിക്കുകയോ അഹംഭാവം ഉദിക്കുകയോ ചെയ്യുന്നില്ല. ഒരിക്കൽ ഈ അഹംഭാവം ഉദിച്ചുപോയാൽ പിന്നെ ദേഹത്തിന്റെ ആയുസ്സ് നീട്ടുക, ധനം സമ്പാദിക്കുക, ആരോഗ്യം പരിപോഷിപ്പിക്കുക, സുഖം തേടുക എന്നി പ്രവർത്തനങ്ങൾ തുടങ്ങുകയായി. രാക്ഷസരെ ഈ ആഗ്രഹങ്ങൾ ക്ഷീണിതരാക്കി.

അവരുടെ മനസ്സിലുണ്ടായ ചിന്താക്കുഴപ്പം 'ഇതെന്റേതാണ്‌', 'ഇതെന്റെ ശരീരമാണ്‌', തുടങ്ങിയ തോന്നലുകളുണ്ടാക്കി. ഈ ധാരണകൾ അവരുടെ സ്വധര്‍മ്മങ്ങള്‍ ചെയ്യുന്നതിലെ കാര്യക്ഷമതയെ സാരമായി ബാധിച്ചു. അവർക്ക് തിന്നാനും കുടിക്കാനും വലിയ ആസക്തിയുണ്ടായി. വസ്തുക്കൾ സുഖാനുഭവം നല്‍കുമെന്ന തോന്നല്‍ അവരിലുണ്ടായപ്പോള്‍  അവരുടെ സ്വാതന്ത്ര്യം പൊയ്പ്പോയി. സ്വതന്ത്രതാബോധം പോയതോടെ അവരിൽ ഭയം അങ്കുരിച്ചു. 'ഞങ്ങൾ ഈ യുദ്ധത്തിൽ മരിച്ചുപോവും', എന്നൊരു ഭീതിചിന്ത അവരിലുണ്ടായി. ദേവന്മാർ ഈ സമയം മുതലെടുത്ത് അസുരന്മാരെ ആക്രമിക്കാൻ ഒരുമ്പെട്ടു. മരണഭീതിയിൽ ഈ മൂന്നസുരന്മാർ പാലായനം ചെയ്തു. തങ്ങളുടെ രക്ഷകരെന്നു കരുതിയിരുന്ന രാക്ഷസനേതാക്കൾ യുദ്ധത്തിൽ നിന്നു തിരിഞ്ഞോടുന്നതുകണ്ട് അസുരസൈന്യത്തിന്റെ ആത്മവീര്യം നഷ്ടപ്പെട്ടു. അവർ ആയിരക്കണക്കിന്‌ മരിച്ചു വീണു. ദേവസൈന്യത്തിന്റെ ആക്രമണവൃത്താന്തം കേട്ട് ശംഭരൻ കുപിതനായി. ദാമൻ, വ്യാളൻ, കടൻ എന്നിവർ എവിടെപ്പോയി എന്നയാൾ ആക്രോശിച്ചു. ശംഭരന്റെ കോപത്തെ ഭയന്ന് മൂവരും പാതാളത്തിന്റെ അങ്ങേയറ്റത്ത് പോയൊളിച്ചു. യമദേവന്റെ സേവകർ അവർക്കവിടെ അഭയം നൽ കി. മൂന്നു കന്യകമാരെ വിവാഹം ചെയ്തു കൊടുക്കുകയും ചെയ്തു. പാതാളത്തിൽ അവർ ഏറെക്കാലം കഴിഞ്ഞുകൂടി. ഒരുദിവസം തന്റെ സേവകസന്നാഹങ്ങളൊന്നുമില്ലാതെ യമദേവൻ സ്വയം അവരെ സന്ദർശിച്ചു. അവരദ്ദേഹത്തെ തിരിച്ചറിഞ്ഞില്ല; ബഹുമാനിച്ചുമില്ല. ക്രുദ്ധനായ യമൻ അവരെ ഏറ്റവും കൊടിയ നരകങ്ങളിലേയ്ക്ക് കൊണ്ടുപോയി. അവിടെ ദുരിതമനുഭവിച്ച്, പിന്നീട് അനേകം ഹീനയോനികളിൽ ജനിച്ചു മരിച്ച് അവസാനം കാഷ്മീരിലെ ഒരു തടാകത്തിൽ മൽസ്യങ്ങളായി അവരിപ്പോഴും ജീവിക്കുന്നു. 

