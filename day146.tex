\section{ദിവസം 146}

\slokam{
ഇത്യസ്തന്തോ ന സദ്ദൃഷ്ടേർ അസദ്ദൃഷ്ടേശ്ച വാ ക്വചിത്\\
അസ്യാസ്ത്വാഭ്യുദിതം ബുദ്ധം നാബുദ്ധം പ്രതി വാനഘ (2/3/15)\\
}

വസിഷ്ഠൻ തുടർന്നു: രാമ: പ്രളയസമയത്ത് പരബ്രഹ്മത്തിൽ വിശ്വം ഒരു വിത്തായി നിലകൊണ്ടിരുന്നുവെങ്കിൽ പ്രളയാവസാനം വീണ്ടും വിശ്വമായി പ്രത്യക്ഷപ്പെടാൻ മറ്റൊരു സഹ-കാരണം കൂടി ഉണ്ടാവേണ്ടിയിരിക്കുന്നു. വിശ്വമുണ്ടാവാൻ അപ്രകാരമുള്ളൊരു സഹ-കാരണം ഉണ്ടായിരുന്നില്ല എന്നു പറയുന്നത് വന്ധ്യയുടെ പുത്രി എന്നു പറയും പോലെയുള്ള അസംബന്ധമത്രേ. അതുകൊണ്ട് അടിസ്ഥാനകാരണം എന്നത് പരബ്രഹ്മത്തിന്റെ സഹജ സ്വഭാവമാണെന്നറിയുക. പ്രളയശേഷവും ഈ വിശ്വസൃഷ്ടിയില്‍ അതപ്രകാരം തുടരുന്നു. പരബ്രഹ്മവും വിശ്വവും തമ്മിൽ കാര്യ-കാരണ ബന്ധമില്ല. ചിദാകാശത്തിൽ, അനന്താവബോധത്തിൽ, അനേക കോടി വിശ്വങ്ങൾ ഉള്ളത് തിളക്കമുള്ള പൊടിപടലങ്ങൾ പോലെയാണ്‌.. ഒരിരുട്ടുമുറിയില്‍  മേൽക്കൂരയിലെ ദ്വാരത്തിലൂടെ വരുന്ന സൂര്യരശ്മിയിൽ പൊടിപടലങ്ങൾ തിളക്കമാർന്നു കാണുന്നു. എന്നാൽ പുറത്ത് സൂര്യപ്രകാശത്തിൽ അവ ദൃശ്യമല്ല എന്നതുപോലെ പരമമായ അദ്വൈതബോധത്തിൽ വിശ്വം കാണപ്പെടുന്നില്ല. ഒരാളുടെ സ്വഭാവം അയാളിൽ നിന്നു വിഭിന്നമല്ലാത്തതുപോലെ വിശ്വം അനന്താവബോധത്തിൽ നിന്നു വിഭിന്നമല്ല.

പ്രളയാനന്തരം വിശ്വസൃഷ്ടിക്കായി ഒരു സൃഷ്ടാവുണ്ടായി. അത് സ്മരണ - ഓർമ്മയാണ്‌. ആ സ്മരണയിലുണ്ടായ ചിന്തകളാണ്‌ കാണപ്പെടുന്ന ഈ ലോകത്തിനു കാരണം. അതോ, 'ആകാശത്തിലെ അപ്പം' എന്ന പോലെയുള്ള ഒരു അയാഥാർത്ഥ്യം മാത്രം.  ഈ ചിന്തകളുടലെടുത്ത സ്മരണയ്ക്ക് ശരിയായ അടിത്തറയൊന്നുമില്ല. കാരണം കഴിഞ്ഞ ലോകചക്രത്തിലെ ബ്രഹ്മാദിദേവതകൾ എല്ലാം മുക്തിപദം പൂകിയതാണല്ലോ. സ്മരണ ഉൾക്കൊണ്ടു നിൽക്കാൻ ആരുമില്ലാത്തപ്പോൾ അതിനെങ്ങിനെ അസ്തിത്വം സാദ്ധ്യമാവും? ബോധത്തിലുയരുന്ന, പൂർവ്വാനുഭവസംബന്ധിയായോ അല്ലാതെയോ ഉള്ള സ്മരണ വിശ്വമായി കാണപ്പെടുന്നു. അങ്ങിനെ അനന്താവബോധത്തിൽ പൊടുന്നനേ കാണപ്പെടുന്ന ലോകം യാദൃശ്ചികമായ സൃഷ്ടിയാണ്‌.

ഇങ്ങിനെയുള്ള പ്രത്യക്ഷലോകം വിശ്വപുരുഷൻ എന്നപേരിൽ ഒരു ദിവ്യരൂപം ധരിച്ചു. ചെറിയൊരണുവിൽ മൂന്നുലോകങ്ങള്‍, അവയുടെ ഘടകങ്ങളായ കാലം, ദൂരം, കർമ്മം, പദാർത്ഥങ്ങൾ, പകൽ, രാത്രി എന്നിവകളോടെ കാണപ്പെടുന്നു. ആ അണുവിനുള്ളിൽത്തന്നെ  അനേകം അണുക്കളുണ്ട്; അനേകം ലോകദൃശ്യങ്ങളും. ഒരു വെണ്ണക്കല്ലിൽ തീർത്ത ശിൽപ്പത്തിന്റെ കയ്യിലൊരു ശിൽപ്പം; ആ ശിൽപ്പത്തിന്റെ കയ്യിലും ഒരു ശിൽപ്പമുണ്ട്. അങ്ങിനെ എണ്ണിയാലൊടുങ്ങാത്ത വെണ്ണക്കൽ ശിൽപ്പങ്ങളുടെ അനന്തമായ കാഴ്ച്ചപോലെയാണീ ജഗത്ത്.

അതിനാൽ രാമ: പ്രബുദ്ധനായവന്റെയും അജ്ഞാനിയുടേയും കണ്ണിൽ നിന്നും ദൃശ്യങ്ങൾ മായുകയില്ല. ജ്ഞാനിക്ക് ഇതെല്ലാം എപ്പോഴും ബ്രഹ്മം മാത്രം. അജ്ഞാനിക്ക് ഇതെപ്പോഴും ലോകം മാത്രം. തികഞ്ഞ ശൂന്യതയിൽ കാണുന്നതിനെ ‘ദൂരം’ എന്നും അനന്താവബോധത്തിൽ കാണുന്നതിനെ ‘സൃഷ്ടി’ യെന്നും വിവക്ഷിക്കപ്പെടുന്നു. സൃഷ്ടി എന്നത് വെറുമൊരു വാക്കു മാത്രം - അതിനനുയോജ്യമായ സത്യസ്ഥിതി ഇല്ലതന്നെ. 

