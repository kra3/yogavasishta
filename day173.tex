\section{ദിവസം 173}

\slokam{
അഹമർത്ഥോപരിജ്ഞാത: പരമാർത്ഥാംബരേ മലം\\
പരിജ്ഞാതോഹമർത്ഥസ്തു പരമാത്മാംബരം ഭവേത് (4/33/24)\\
}

വസിഷ്ഠൻ തുടർന്നു: “ശരിയായ അറിവിന്റെ വെളിച്ചത്തിലല്ലെങ്കിൽ, ‘ഞാൻ’ എന്നത് അനന്താവബോധത്തിലെ ഒരു തെറ്റിദ്ധാരണയാണ്‌.. എന്നാൽ ശരിയായ അറിവിന്റെ നിറവിലോ അത് അനന്തതാവബോധം തന്നെയാണ്‌.” സ്വന്തം ഉണ്മയെ തിരിച്ചറിഞ്ഞു കഴിഞ്ഞാൽ ‘ഞാൻ’ എന്നത് അഹംഭാവമാണെന്ന ധാരണ പാടേ നീങ്ങി ഒരേയൊരു അനന്താവബോധം മാത്രമാണു ഞാൻ എന്ന സാക്ഷാത്കാരം ഉണ്ടാവുന്നു. വാസ്തവത്തിൽ ‘ഞാൻ’ എന്ന ഒരു പ്രത്യേക അസ്തിത്വം ഇല്ല തന്നെ. ഈ സത്യം നിർമ്മലമനസ്സുള്ള ഒരുവനു വെളിപ്പെടുമ്പോൾ അവന്റെ അജ്ഞാനം ക്ഷണനേരംകൊണ്ട് ഇല്ലാതെയാവുന്നു. മറ്റുള്ളവർ അവരുടെ തെറ്റിദ്ധാരണയ്ക്കു മാറ്റമൊന്നുമില്ലാതെ ചെറിയകുട്ടികൾ ഭൂതപിശാചുക്കളുടെ അസ്തിത്വത്തിൽ വിശ്വസിക്കുന്നതുപോലെ കഴിഞ്ഞു കൂടുന്നു. ‘ഞാൻ’ എന്ന ഒന്നിന്‌ അസ്തിത്വമില്ല എന്ന വസ്തുത അറിയുന്നവൻ എങ്ങിനെയാണ്‌ ‘ഞാനു’ മായി ബന്ധമുള്ള സ്വർഗ്ഗം, നരകം തുടങ്ങിയ മറ്റു ധാരണകൾക്ക് വില കൽപ്പിക്കുക? ‘ഞാൻ’ എന്ന ധാരണ സത്യമാണെന്നു കരുതുമ്പോൾ മാത്രമേ സ്വർഗ്ഗലാഭത്തിനും മുക്തിപദപ്രാപ്തിക്കും പോലുമുള്ള അഭിവാഞ്ഛ ഒരുവനിൽ ഉയരുകയുള്ളു.

‘ഞാൻ’ ഉള്ളിടത്തോളം ഒരുവന്റെ ജീവിതത്തിൽ അസന്തുഷ്ടി മാത്രമേയുള്ളു. ‘അഹം’ ഇല്ലാതാകാൻ ആത്മജ്ഞാനമല്ലാതെ മറ്റൊരുമാർഗ്ഗമില്ല. ഈ ‘അഹം’ എന്ന ഭൂതം ആവേശിച്ചുകഴിഞ്ഞാൽ ഒരു ശാസ്ത്രഗ്രന്ഥത്തിനും, മന്ത്രങ്ങൾക്കും ഒന്നും അതിനെ ഒഴിപ്പിക്കാൻ കഴിയില്ല. ആത്മാവ്, അനന്താവബോധത്തിന്റെ ശുദ്ധമായ പ്രതിഫലനം മാത്രമാണെന്ന സത്യം നിരന്തരമായി സ്മരിക്കുന്നതുകൊണ്ടു മാത്രമേ അഹത്തിന്റെ വളർച്ചയെ തടയുവാനാവൂ.

ലോകമെന്ന കാഴ്ച്ച (പ്രകടനം) ഒരു മികച്ച ജാലവിദ്യക്കാരന്റെ കൺകെട്ടാണ്‌.. വിഷയ-വിഷയീ ബന്ധങ്ങളും ‘എനിയ്ക്ക്’ അവയുമായുള്ള ബന്ധവും ശുദ്ധ അസംബന്ധവും. ഈ അറിവുറയ്ക്കുമ്പോള്‍ അഹംഭാവത്തിന്റെ വേരറ്റുപോവുന്നു. ഈ അഹമാണ്‌ ‘ലോകം’ എന്ന ധാരണയ്ക്ക് കാരണമാവുന്നതെന്ന് അറിയുമ്പോൾ രണ്ടും പ്രശാന്തിയടഞ്ഞില്ലാതാവുന്നു. എന്നാൽ  അഹംഭാവത്തിന്റെ ഉന്നതമായ ഒരു തലത്തിലാണ്‌ പ്രബുദ്ധനായവൻ കഴിയുന്നത്. അതിൽ 'ഞാനാ'ണീ വിശ്വം മുഴുവനും നിറഞ്ഞുവിളങ്ങുന്നത്; 'എന്നി'ൽ നിന്നുവേറിട്ട് യാതൊന്നുമില്ല എന്ന ദൃഢമായ ധാരണയാണുള്ളത്. മറ്റൊരു തലത്തിലുള്ള അഹംഭാവത്തിൽ, 'ഞാന്‍' അതീവസൂക്ഷ്മമായ അണുമാത്രമാണെന്നും അതിനാൽ ഈ വിശ്വത്തിലെ എല്ലാറ്റിൽനിന്നും സർവ്വതന്ത്രസ്വതന്ത്രമാണെന്നും ഉള്ള അറിവാണുള്ളത്. ഇതും മുക്തിദായകമത്രേ. എന്നാൽ ദേഹാഭിമാനിയായ അഹം, ആത്മാവിനെ ശരീരത്തോട് തദാത്മ്യഭാവത്തിൽ കാണുന്നതാണ്‌ തീർച്ചയായും വർജ്ജിക്കേണ്ടത്.

തുടർച്ചയായി ഉയർന്ന തലത്തിലുള്ള ഉത്തമ ‘അഹം’ ഭാവം ഉള്ളിൽ വളർത്തിയുറപ്പിക്കുകവഴി അധമമായ അഹംഭാവത്തെ ഇല്ലാതെയാക്കാം. അധമമായ അഹംഭാവത്തെ നിരാകരിച്ച് ഉന്നതമായ ‘അഖിലവും ഞാൻ’ എന്ന ഭാവമോ ‘ഏറ്റവും സൂക്ഷ്മമവും സ്വതന്ത്രവുമാണു ഞാൻ’ എന്ന ഭാവമോ നിരന്തരം ഉള്ളിലുറപ്പിക്കുക. കാലക്രമത്തില്‍ ഈ രണ്ട് ‘ഉന്നതാഹംഭാവങ്ങളും’ വർജ്ജിക്കണം. അതിനുശേഷം എല്ലാവിധ കർമ്മങ്ങളിലും സർവ്വാത്മനാ മുഴുകിക്കഴിയുകയോ ഒന്നിലും ഏർപ്പെടാതെ ഏകാന്തതയിൽ കഴിയുകയോ ചെയ്യാം. കാരണം അങ്ങിനെയുള്ള ജ്ഞാനിക്ക് അപചയം ഉണ്ടാകുന്നതല്ല. 
