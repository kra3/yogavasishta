\section{ദിവസം 148}

\slokam{
വിവിധ ജന്മദശാം വിവിധാശയ:  സമനുഭൂയ ശരീരപരമ്പരാ:\\
സുഖമതിഷ്ഠദസൗ ഭൃഗുനന്ദനോ വരനദീസുതടേ  ദൃഢവൃക്ഷവത് (4/8/29)\\
}

വസിഷ്ഠൻ തുടർന്നു: ശുക്രൻ തന്റെ പഴയ വ്യക്തിത്വം പാടേ മറന്നുപോയിരുന്നു. കുറച്ചുനാൾ ഇന്ദ്രന്റെ സഭയിൽ ചുറ്റിനടന്നശേഷം ശുക്രൻ താൻ നേരത്തേ കണ്ടുമോഹിച്ച അപ്സരസ്സിന്റെ ചുറ്റുപാടുകളെക്കുറിച്ച് മനസ്സിലാക്കി. അവർ തമ്മിൽ കണ്ടപാടെ പരസ്പരമുള്ള ആകർഷണത്താലവർ മതിമറന്നുപോയി. ആശകളെ സഫലീകരിക്കുക എന്നത് സ്വർഗ്ഗത്തിന്റെ പ്രത്യേകതയാണല്ലോ. താൻ അപ്സരസ്സുമായി സന്ധിച്ച നന്ദനോദ്യാനത്തില്‍  രാത്രിയുടെ ഇരുട്ടു മൂടട്ടെ  എന്ന് ശുക്രൻ ചിന്തിക്കവേ അവിടം ഇരുട്ടായി. ശുക്രൻ അവിടെയുള്ള ഒരു മന്ദിരത്തിൽ പ്രവേശിച്ചു. അപ്സരസ്സ് ശുക്രനുപിറകേ അവിടെയെത്തി. അവൾ പറഞ്ഞു: പ്രിയതമാ, അങ്ങയോടുള്ള ആശയാൽ എന്റെ ഉള്ളം തപിക്കുന്നു. മൂഢന്മാർ മാത്രമേ പ്രേമത്തെ തള്ളിപ്പറയൂ. ജ്ഞാനികൾ അങ്ങിനെയല്ല. മൂന്നുലോകങ്ങളുടെ അധിപസ്ഥാനം പോലും പ്രണയബദ്ധരുടെ സംഗത്തിനോളം വിലയുള്ളതല്ല. അങ്ങയുടെ ഹൃദയത്തിൽ എനിക്കിടം തന്നാലും.” എന്നുപറഞ്ഞ് അവൾ അദ്ദേഹത്തിന്റേ നെഞ്ചിലേയ്ക്ക് ചാഞ്ഞു. ശുക്രൻ അപ്സരസ്സുമൊത്ത് ഏറെക്കാലം സ്വർഗ്ഗത്തിൽ സുഖിച്ചു കറങ്ങി നടന്നു. എട്ടു ലോകചക്രങ്ങൾ അങ്ങിനെ കഴിഞ്ഞു. ഇക്കാലംകൊണ്ട് അദ്ദേഹത്തിന്റെ ആർജ്ജിതപുണ്യം മുഴുവനും നഷ്ടമായി. അപ്സരസ്സുമായി അദ്ദേഹം താഴോട്ടു പതിച്ച് ഭൂമിയിലെത്തി.

അവരുടെ സൂക്ഷ്മശരീരം ഭൂയിൽ പതിച്ചപ്പോൾ മഞ്ഞുതുള്ളികളായി കുറേ ധാന്യമണികളിൽ കയറി. അതാഹരിച്ച മഹാബ്രാഹ്മണനിൽ നിന്നും അദ്ദേഹത്തിന്റെ ഭാര്യയിൽ അതു ബീജമായി പ്രവേശിച്ചു. ശുക്രൻ അവരുടെ പുത്രനായി അവിടെ വളർന്നു. അപ്സരസ്സ് ഒരു മാനായി. ശുക്രൻ ആ മാൻപേടയിൽ ഒരു മനുഷ്യക്കുട്ടിയെ ജനിപ്പിച്ചു. അദ്ദേഹം സ്വപുത്രനോട് ഏറെ മമതയുള്ളവനായി. ഈ മകനെക്കുറിച്ചുള്ള ആധിയിലും ആകാംക്ഷയിലും ശുക്രൻ വേഗം വൃദ്ധനായി. അയാൾ സുഖാനുഭവാസക്തനായി മരണംവരിക്കുകയും ചെയ്തു. ആഗ്രഹനിവൃത്തിക്കായി അടുത്ത ജന്മം ശുക്രൻ ഒരു രാജാവായി. തപശ്ചര്യയിലും ആത്മീയതയിലും താല്‍പ്പര്യത്തോടെ മരിച്ചതിനാൽ അടുത്തജന്മം അദ്ദേഹം ഒരു മഹാത്മാവായി.

“അങ്ങിനെ ഒരു ജന്മത്തിൽനിന്നും മറ്റൊന്നിലേയ്ക്ക് - വിവിധശരീരങ്ങളിൽ- നീങ്ങി അവയുടെ എല്ലാ ഭാഗധേയങ്ങൾക്കും വിധേയനായി, ഒരു നദിക്കരയിൽ അദ്ദേഹം തീവ്രതപസ്സിൽ ആമഗ്നനായി." അങ്ങിനെ തന്റെ അച്ഛന്റെ മുന്നില്‍  തപസ്സിരുന്ന്‍ ശുക്രൻ കുറെയേറെക്കാലം ചിലവഴിച്ചു. ശരീരം ശോഷിച്ചുണങ്ങി. അതിനിടെ ശുക്രന്റെ അശാന്തമായ മനസ്സ് ദൃശ്യങ്ങളനവധി ഒന്നിനുപുറകേ ഒന്നായി മിനഞ്ഞെടുത്തു. അവയ്ക്കനുസരിച്ച് തുടർച്ചയായി ജന്മങ്ങളെടുത്ത് ജനന മരണ ചക്രങ്ങൾ താണ്ടി, സ്വർഗ്ഗവാസവും ഭൂമിയിലേയ്ക്കുള്ള മടക്കവും, താപസന്റെ പ്രശാന്തജീവിതവും ആവർത്തിച്ചുകൊണ്ടിരുന്നു. അവയിൽ പൂർണ്ണമായി മുഴുകിക്കഴിഞ്ഞിരുന്നതിനാൽ അവ സത്യമാണെന്ന് അദ്ദേഹത്തിനുറപ്പായിരുന്നു. എല്ലാത്തരം കാലാവസ്ഥകൾക്കും വശംവദമായി അദ്ദേഹത്തിന്റെ ശരീരം എല്ലും തോലുമായി, കണ്ടാൽ ഭയമുളവാക്കുന്ന രൂപത്തിലായി. അദ്ദേഹത്തെ നരഭോജികൾ ആക്രമിച്ചില്ല. തീവ്രതപസ്സിലായിരുന്ന ഭൃഗുമഹർഷിയുടെ മുന്നിലായിരുന്നല്ലോ ശുക്രനും ധ്യാനത്തിലാണ്ടിരുന്നത്. യോഗസാധനകൊണ്ട് അദ്ദേഹവും പല സിദ്ധികളും നേടിയിരുന്നു. 

