 
\section{ദിവസം 083}

\slokam{
സദ്‌ വാസനസ്യ രൂഠായാമാതിവാഹിക സംവിദി ദേഹോവിസ്മൃതിമായാതി ഗര്‍ഭസംസ്ഥേവ യൌവനേ (3/58/16)
}

വസിഷ്ഠന്‍ തുടര്‍ന്നു: അപ്പോഴേയ്ക്കും വിഥുരഥന്റെ ജീവന്‍ പദ്മ രജാവിന്റെ ശരീരത്തിലേയ്ക്കു പ്രവേശിക്കുന്നതില്‍ നിന്നും സരസ്വതീ ദേവി തടഞ്ഞു. പ്രബുദ്ധയായ ലീല സരസ്വതിയോടു ചോദിച്ചു: ദേവീ ഞാനിവിടെ ധ്യാനത്തിലിരുന്നിട്ട്‌ എത്ര കാലം കടന്നുപോയി?.

'വത്സേ, നിന്റെ ധ്യാനം തുടങ്ങിയിട്ട്‌.. ഇപ്പോള്‍ മാസം ഒന്നു കഴിഞ്ഞിരിക്കുന്നു  പ്രാണായാമം ചെയ്തുണ്ടായ താപം കാരണം നിന്റെ ശരീരം ആദ്യത്തെ പതിനഞ്ചുദിവസം കൊണ്ട്‌ ബാഷ്പീകരിച്ചു. പിന്നെയത്‌ ഉണങ്ങിയ കരിയിലപോലെ കൊഴിഞ്ഞുവീണു. പിന്നെ കട്ടിപിടിച്ച്‌ തണുത്തുറഞ്ഞു. നീ സ്വന്തം ഇഷ്ടപ്രകാരം മരണം വരിച്ചതാണെന്ന് മന്ത്രിമാര്‍ കരുതി. അവരാകട്ടെ നിന്നെ ദഹിപ്പിക്കുകയും ചെയ്തു. ഇപ്പോള്‍ നീ നിന്റെ ആഗ്രഹപ്രകാരം സൂക്ഷ്മശരീരിയായി വന്നിരിക്കുന്നു. നിന്നില്‍ , പോയ ജന്മത്തില്‍നിന്നും കൊണ്ടുവന്ന ഓര്‍മ്മകളോ വാസനകളോ ലീനമായി അവശേഷിക്കുന്നില്ല.

"യൌവ്വനദശയില്‍ സ്വന്തം ഭ്രൂണാവസ്ഥയെപ്പറ്റി ഓര്‍മ്മകളൊന്നുമില്ലാത്തതു പോലെ സൂക്ഷ്മശരീരത്തെപ്പറ്റി ദൃഢബുദ്ധിയുറച്ചുകഴിഞ്ഞാല്‍ ഭൌതീകശരീരം വിസ്മൃതിയായി." ഇന്ന് മുപ്പത്തിയൊന്നാം ദിനമാണ്‌ നീയിവിടെയിരിക്കുന്നത്‌.....  ... വരൂ നമുക്ക്‌ മറ്റേ ലീലയുടെ അടുക്കല്‍പോയി നമ്മളാരെന്നു വെളിപ്പെടുത്താം. 

രണ്ടാമത്തെ ലീല അവരെക്കണ്ടപ്പോള്‍ അവരുടെ കാല്‍ക്കല്‍ വീണ്‌ നമസ്കരിച്ചു. സരസ്വതി അവളോടു ചോദിച്ചു: പറയൂ എങ്ങിനെയാണ്‌ നീയിവിടെ വന്നത്‌? ലീല പറഞ്ഞു: വിഥുരഥന്റെ കൊട്ടാരത്തില്‍ ഞാന്‍ മോഹാലസ്യപ്പെട്ടു വീണപ്പോള്‍ എനിക്കൊന്നിനെക്കുറിച്ചും അറിവുണ്ടായിരുന്നില്ല. പിന്നീട്‌ എന്റെ സൂക്ഷ്മശരീരം ആകാശത്തേക്കുയര്‍ന്നു. ഒരു വിമാനമാണ്‌ എന്നെ ഇവിടെയെത്തിച്ചത്‌. ഇവിടെ വിഥുരഥന്‍ ഒരു പൂമെത്തമേല്‍ കിടന്നുറങ്ങുന്നതാണ്‌ ഞാന്‍ കണ്ടത്‌. അദ്ദേഹം യുദ്ധത്തിന്റെ ക്ഷീണത്തില്‍ തളര്‍ന്നുറങ്ങുകയായതുകൊണ്ട്‌ ഞാന്‍ വീശുകയാണ്‌. . സരസ്വതി പെട്ടെന്ന് വിഥുരഥന്റെ ശരീരത്തിലേയ്ക്ക്‌ ജീവനെ കടത്തിവിട്ടു. അപ്പോള്‍ത്തന്നെ രാജാവ്‌ ഉറക്കത്തില്‍നിന്നെന്നവണ്ണം എഴുന്നേറ്റു. രണ്ടു ലീലമാരും അദ്ദേഹത്തെ നമിച്ചു.

രാജാവ്‌ പ്രബുദ്ധയായ ലീലയോടു ചോദിച്ചു: നീയാരാണ്‌? ആരാണു മറ്റേ വനിത?എവിടെനിന്നാണവര്‍ വന്നത്‌? പ്രബുദ്ധലീല പറഞ്ഞു: പ്രഭോ ഞാന്‍ കഴിഞ്ഞജന്മത്തിലെ അവിടുത്തെ ഭാര്യയാണ്‌. വാക്കും അര്‍ത്ഥവും പോലെ അവിടുത്തെ സന്തതസഹചാരിയും പ്രിയതമയും ഞാനായിരുന്നു. ഈ ലീല, അവിടുത്തെ മറ്റേ ഭാര്യയാണ്‌. എന്റെതന്നെ പ്രതിഫലനമാണവള്‍ .  അങ്ങയുടെ സന്തോഷത്തിനുവേണ്ടി ഞാനുണ്ടാക്കിയതാണവളെ. അവിടെ സുവര്‍ണ്ണസിംഹാസനത്തില്‍ ഇരിക്കുന്നത്‌ സാക്ഷാല്‍ സരസ്വതീ ദേവിയാണ്‌. നമ്മുടെയെല്ലാം സൌഭാഗ്യംകൊണ്ടാണ്‌ ദേവി ഇവിടെ സന്നിഹിതയായിരിക്കുന്നത്‌. ഇത്രയും കേട്ടപ്പോള്‍ എണീറ്റിരുന്ന് രാജാവ്‌ സരസ്വതീദേവിയെ അഭിവാദ്യം ചെയ്തു. ദേവി അദ്ദേഹത്തിന്‌ ഐശ്വര്യ സമ്പല്‍സമൃദ്ധികളും, ദീര്‍ഘായുസ്സും, സ്വരൂപസാക്ഷാത്കാരലബ്ധിയും നല്‍കി അനുഗ്രഹിച്ചു.

