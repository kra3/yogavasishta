\section{ദിവസം 162}

\slokam{
ശൂന്യ ഏവ ശരീരേന്തർബദ്ധോസ്മിതി ഭയം തഥാ\\
ശൂന്യ ഏവ കുസൂലെ തു പ്രേക്ഷ്യ സിംഹോ ന ലഭ്യതേ (4/21/50)\\
}

വസിഷ്ഠൻ തുടർന്നു: മനസ്സിന്റെ വിവിധതലങ്ങളെക്കുറിച്ച് ഞാൻ വിശദീകരിച്ചത് മനസ്സിന്റെ സ്വഭാവങ്ങളെക്കുറിച്ച് മനസ്സിലാക്കുവാൻ മാത്രമാണ്‌..  മറ്റൊരുപയോഗവും ഈ ചർച്ചകൊണ്ടില്ല. കാരണം, മനസ്സ്, എന്തിനെ തീവ്രമായി ധ്യാനിക്കുന്നുവോ അതിന്റെ രൂപഭാവങ്ങൾ ആർജ്ജിക്കുകയാണ്‌..  സ്ഥിതി, സ്ഥിതിനിരാസം, നേടൽ, ത്യജിക്കൽ, എല്ലാം മനസ്സിന്റെ ഭാവങ്ങൾ മാത്രം.

രാമൻ ചോദിച്ചു: മനസ്സ് ഈ പറഞ്ഞതുമാത്രമാണെങ്കിൽ അതെങ്ങിനെ കളങ്കിതമായി?

വസിഷ്ഠൻ പറഞ്ഞു: അതൊരു നല്ല ചോദ്യമാണു രാമാ. എന്നാൽ അതു ചോദിക്കാൻ ഉചിതമായ സമയമല്ല ഇത്. കാരണം എനിയ്ക്കു പറയാനുള്ളതു കേട്ടു കഴിഞ്ഞാല്‍പ്പിന്നെ നിനക്ക് ഈ സംശയം ഉണ്ടാവുകയില്ല. എല്ലാം നിന്റെയുള്ളിൽ സംശയലേശമില്ലാതെ തെളിയും. മുക്തിയാഗ്രഹിക്കുന്ന, അതിനായി പരിശ്രമിക്കുന്ന, എല്ലാവരുടേയും അനുഭവമാണ്‌ തന്റെ മനസ്സ് അശുദ്ധമാണ്‌ എന്ന തോന്നൽ. ഓരോരുത്തരും അവരവരുടെ വീക്ഷണത്തിനനുസരിച്ച് അതിനെ വിവിധതരത്തിൽ വിവരിക്കുന്നു. വൈവിദ്ധ്യമാർന്ന സുഗന്ധപുഷ്പങ്ങളുമായി സമ്പർക്കത്തിലാവുന്ന കാറ്റ് ആതാത് പുഷ്പഗന്ധങ്ങൾ വഹിക്കുന്നതുപോലെ മനസ്സ് വിവിധ ഭാവങ്ങളെ ആവഹിച്ച് അവയെ സാക്ഷാത്ക്കരിക്കാനുചിതമായ ശരീരങ്ങളെ സൃഷ്ടിക്കുന്നു. അവയ്ക്കനുയോജ്യമായ കർമ്മോർജ്ജവും ചൈതന്യവും ഇന്ദ്രിയങ്ങളിൽ നിറയ്ക്കുന്നു. തത്കർമ്മ- ധാരണാ- ഫലങ്ങൾ അനുഭവിക്കുകയും ചെയ്യുന്നു. ഈ മനസ്സാണ്‌ കർമ്മേന്ദ്രിയങ്ങൾക്ക് ചലിക്കാൻ കരുത്തേകുന്നത്. മനസ്സാണ്‌ കർമ്മം. കർമ്മമാണ്‌ മനസ്സ്. പൂവും അതിന്റെ സുഗന്ധവുമെന്നതുപോലെയാണത്, അവ രണ്ടല്ല.

മനസ്സിലെ തീരുമാനങ്ങളുടെ ദൃഢതയാണ്‌ കർമ്മങ്ങൾക്ക്‌ ശക്തിനല്കുന്നത്. കർമ്മങ്ങൾ ഈ തീരുമാനങ്ങൾക്ക് സാധുതയും നല്കുന്നു. എല്ലാവരുടേയും മനസ്സ് ധർമ്മം, അർത്ഥം, കാമം, മോക്ഷം എന്നിവയിലേയ്ക്ക് ഉന്മുഖമായാണിരിക്കുന്നത്. എന്നാൽ ഓരോരുത്തർക്കും ഈ പുരുഷാർത്ഥങ്ങളെപ്പറ്റി വൈവിദ്ധ്യമായ കാഴ്ച്ചപ്പാടും നിർവ്വചനങ്ങളുമാണുള്ളത്. ഒരോരുത്തർക്കും അവരുടേതാണു ശരി എന്നുറപ്പുമാണ്‌.. എങ്കിലും കപില മുനിയുടെ സിദ്ധാന്തം പിന്തുടരുന്നവരും, വേദാന്തികളും, വിജ്ഞാനവാദികളും, ജൈനന്മാരും മറ്റുള്ളവരും അവരുടെ പാത മാത്രമാണ്‌ മുക്തിയിലേയ്ക്കുള്ള ഏകമാർഗ്ഗമെന്നു ശഠിക്കുന്നു. അവരുടെ തത്വചിന്തകളും അനുഭവകഥനങ്ങളും അവരുടെ സ്വന്തം അനുഷ്ഠാനങ്ങളുടെ അടിസ്ഥാനത്തിൽ, മനോദാർഢ്യത്തോടെ ഉരുത്തിരിഞ്ഞു വന്നിട്ടുള്ളതാണ്‌.. 

രാമാ, ബന്ധനം എന്നത് വിഷയത്തെപ്പറ്റിയുള്ള ധാരണ തന്നെയാണ്‌.  ഈ ധാരണയാണ്‌ മായ, അല്ലെങ്കിൽ അജ്ഞാനം. ഇത് സത്യത്തെ മറയ്ക്കുന്ന തിമിരമാണ്‌.  അജ്ഞാനം സംശയത്തെ ഉണ്ടാക്കുന്നു. അതുണ്ടാക്കുന്ന ധാരണ വൈകൃതമായ തെറ്റിദ്ധാരണയാകുന്നു. “അജ്ഞാനിക്ക് ഇരുട്ടത്ത് ഒരൊഴിഞ്ഞ സിംഹക്കൂടിനടുത്തുകൂടിപ്പോകാൻ പേടിയാണ്‌.. അതുപോലെ അജ്ഞാനി, താൻ ഈ ശൂന്യ ശരീരത്തിൽ ബന്ധിതനാണെന്ന തെറ്റിദ്ധാരണയിൽ കഴിയുകയാണ്‌.”

‘ഞാൻ’, ‘ഈ ലോകം’, എന്നീ ധാരണകൾ വെറും നിഴലുകളാണ്‌..  സത്യമല്ല. ഇത്തരം ധാരണകളാണ്‌ വസ്തുക്കളെ (വിഷയങ്ങളെ) സൃഷ്ടിക്കുന്നത്. ഈ വസ്തുക്കളാകട്ടെ സത്യമെന്നോ അല്ലെന്നോ പറയാൻ വയ്യ. ഒരമ്മ, വീട്ടിലെ ജോലിക്കാരിയായി സ്വയം വിചാരിച്ചാൽ അങ്ങിനെയായിത്തീരുന്നു. എന്നാൽ അവർ തന്റെ ഭർത്താവിന്റെ അമ്മയായി സ്വയം വിചാരിച്ചാൽ തല്‍ക്കാലം അങ്ങിനെ പെരുമാറുന്നു. അതുകൊണ്ട് രാമാ, ‘ഞാൻ’, ‘ഇത്’, എന്നൊക്കെയുള്ള ധാരണകൾ ഉപേക്ഷിച്ച് സത്യത്തിൽ മനസ്സുറപ്പിച്ചാലും.

