 
\section{ദിവസം 075}

\slokam{
മഹാചിദ്പ്രതിഭാസത്വാന്മഹാനിയതിനിശ്ചയാത്\\
അന്യോന്യമേവ പശ്യന്തി മിഥ: സമ്പ്രതിബിംബിതാത് (3/53/25)\\
}

വസിഷ്ഠന്‍ തുടര്‍ന്നു: രാമ: സരസ്വതീദേവിയുടെ വരം ലഭിച്ച രണ്ടാമത്തെ ലീല, ആകാശത്തുയര്‍ന്ന് അവിടെ തന്റെ മകളെ സന്ധിച്ചു. കുമാരി, സ്വയം പരിചയപ്പെടുത്തി. ലീല അവളോട്‌ തന്റെ ഭര്‍ത്താവ്‌-രാജാവ്‌ കിടക്കുന്നയിടത്തേയ്ക്കു വഴി കാട്ടാന്‍ അവളോട്‌ അഭ്യര്‍ത്ഥിച്ചു. അവള്‍ അമ്മയുമൊത്ത്‌ പറന്നകന്നു. ആദ്യമവര്‍ മേഘമാര്‍ഗ്ഗവും പിന്നെ വായുമാര്‍ഗ്ഗവും കടന്നു. അതിനപ്പുറം അവര്‍ സൂര്യപഥം കടന്ന് നക്ഷത്രം നിറഞ്ഞ ആകാശത്തെത്തി. അവര്‍ അതിനുമപ്പുറം ബ്രഹ്മാ വിഷ്ണു മഹേശ്വരന്മാരുടെ സവിധങ്ങള്‍ കടന്ന് വിശ്വത്തിന്റെ നെറുകയിലെത്തി. മഞ്ഞുകട്ടയിട്ടുവച്ചപാത്രത്തില്‍നിന്നും തണുപ്പിനു പ്രസരിക്കാന്‍ പാത്രം പൊട്ടിക്കേണ്ട ആവശ്യമില്ലാത്തതുപോലെ സ്വാഭാവികമായി അവര്‍ക്കിതെല്ലാം ചെയ്യാന്‍ കഴിഞ്ഞു. സൂക്ഷ്മശരീരിയായ ലീലയില്‍ മൂര്‍ത്തീകരിച്ച ചിന്തകളായി ഈ അനുഭവങ്ങളെല്ലാം വേദ്യമായി. ഈ വിശ്വതലങ്ങള്‍ എല്ലാം കടന്ന് ലീല സമുദ്രങ്ങളും വിശ്വത്തെ പൊതിഞ്ഞിരിക്കുന്ന മറ്റു മൂലകങ്ങളും തരണം ചെയ്ത്‌ അനന്തബോധത്തിലെത്തി. അവിടെ പരസ്പരം അറിയാത്ത എണ്ണമറ്റ പ്രപഞ്ചങ്ങളുണ്ട്‌. പദ്മരാജാവിന്റെ ശരീരം കിടക്കുന്ന വിശ്വപ്രപഞ്ചത്തില്‍ ലീല പ്രവേശിച്ചു. വീണ്ടും ബ്രഹ്മാദിദേവന്മാരുടെ ലോകം കടന്ന് അവിടെ പദ്മരാജാവിന്റെ ദേഹം പൂക്കള്‍കൊണ്ടു മൂടിവച്ച കൊട്ടാരതളത്തില്‍ ലീല പ്രവേശിച്ചു. എന്നാല്‍ പെട്ടെന്നു ചുറ്റും നോക്കുമ്പോള്‍ തന്റെ മകളെ കാണാനില്ല. അവള്‍ അദ്ഭുതകരമായി അപ്രത്യക്ഷയായിരിക്കുന്നു!. 

അവള്‍ താഴെ കിടത്തിയിരിക്കുന്ന ഭര്‍ത്തൃദേഹം തിരിച്ചറിഞ്ഞു. യുദ്ധത്തില്‍ വീരമൃത്യുവടഞ്ഞതിനാല്‍ അദ്ദേഹത്തിനു വീരസ്വര്‍ഗ്ഗം തന്നെ കിട്ടിയെന്നു ചിന്തിച്ചുറച്ചു. അവള്‍ ആലോചിച്ചു: സരസ്വതീ ദേവിയുടെ കൃപയാല്‍ ഞാന്‍ ഇവിടെ ഉടലോടെ എത്തിച്ചേര്‍ന്നിരിക്കുന്നു. ഞാന്‍ ഏറ്റവും അനുഗൃഹീതയാണ്‌. അവള്‍ രാജാവിന്റെ ദേഹത്തെ വീശാന്‍ തുടങ്ങി. ആദ്യത്തെ ലീല ദേവിയോടു ചോദിച്ചു: അവളെക്കണ്ടിട്ട്‌ രാജസേവകര്‍ എന്തുചെയ്തു?

സരസ്വതി പറഞ്ഞു: രാജാവും, സേവകരും രാജകൊട്ടാരവുമെല്ലാം അനന്താവബോധം മാത്രം. "എന്നാല്‍ എല്ലാറ്റിന്റെയും അടിസ്ഥാനം സത്യവസ്തുവായ അനന്തബോധമായതിനാലും സങ്കല്‍പ്പ രചിതമായ സൃഷ്ടികളുടെ ക്രമാനുഗതസ്വഭാവത്തെപറ്റി നിശ്ചയമുള്ളതിനാലും അവര്‍ പരസ്പരം തിരിച്ചറിയുന്നു". ഭര്‍ത്താവ്‌ പറയുന്നു: 'അവള്‍ എന്റെ ഭാര്യയാണ്‌'. ഭാര്യ പറയുന്നു: 'അദ്ദേഹം എന്റെ ഭര്‍ത്താവാണ്‌'. അവള്‍ക്ക്‌ അവളുടെ പൂര്‍വ്വശരീരത്തിലേയ്ക്ക്‌ പോവുകയെന്നത്‌ അസാദ്ധ്യമാണ്‌. കാരണം പ്രകാശത്തിന്‌ ഇരുട്ടിനൊപ്പം കഴിയുക എന്നതസാദ്ധ്യം. അതുപോലെ ഒരുവനെ അവിദ്യയെന്ന അന്ധത ബാധിച്ചിട്ടുണ്ടെങ്കില്‍ അവനിലെങ്ങിനെ വിജ്ഞാനത്തിന്റെ വിളക്കു തെളിയും? ഒരുവനില്‍ സൂക്ഷ്മശരീരസംബന്ധിയായ ജ്ഞാനമുണരുമ്പൊള്‍ ഭൌതികശരീരംസത്താണെന്ന ധാരണ ഇല്ലാതാവുന്നു. ഞാന്‍ അവള്‍ക്കു നല്‍കിയ വരത്തിന്റെ ഫലമായാണിത്‌. വരലബ്ധിയില്‍ അവളിങ്ങിനെ ചിന്തിക്കുന്നു: 'അവിടുന്നെന്നെ വരബലം മൂലം ഇങ്ങിനെ ചിന്തിക്കുമാറാക്കി; ഞാന്‍ അപ്രകാരമായി' അതുകൊണ്ട്‌ അവള്‍ വിചാരിക്കുന്നു താന്‍ തന്റെ ഭര്‍ത്താവിന്റെയടുത്ത്‌ ഉടലൊടെ എത്തിയെന്ന്. ഒരാള്‍ അജ്ഞതമൂലം കയറില്‍ പാമ്പിനെ ദര്‍ശിച്ചേക്കാം എന്നാല്‍ കയറിന്‌ പാമ്പിനേപ്പോലെപെരുമാറുവാന്‍ കഴിയില്ല. 

