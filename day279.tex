\section{ദിവസം 279}

\slokam{
ധ്യൌ ക്ഷമാ വായുരാകാശം പര്‍വതാ: സരിതോ ദിശ:\\
അന്ത:കരണതത്വസ്യ ഭാഗാ ബഹിരിവ സ്ഥിതാ:    (5/56/35)\\
}

വസിഷ്ഠന്‍ തുടര്‍ന്നു:ആത്മാവിനെ ആരൊരുവന്‍ അതീന്ദ്രിയമായ ഒരു തത്വമായോ സര്‍വ്വഭൂതങ്ങളുടെയും ആത്മസ്വരൂപമായോ കാണുന്നുവോ അയാള്‍ സമതാഭാവം ആര്‍ജ്ജിച്ചവനാണ്. ആരൊരുവനില്‍ ഇഷ്ടാനിഷ്ടങ്ങളുടെ ദ്വന്ദത അവസാനിച്ചുവോ അവന് എല്ലാവരും ഒരുപോലെയാണ്. അയാള്‍ ജാഗ്രദവസ്ഥയിലും സ്വപ്നത്തിലെന്നപോലെ ബാഹ്യദൃശ്യങ്ങളെ വീക്ഷിക്കുന്നു. അയാള്‍ തിരക്കേറിയ ഗ്രാമത്തില്‍ ജീവിക്കുമ്പോഴും കാനനവാസിയായ മുനിയാണ്. അവനവന്‍റെയുള്ളില്‍ ആത്മാവിനെ ദര്‍ശിക്കുന്നവനു ഗ്രാമനഗരങ്ങള്‍ എല്ലാം കാടുപോലെ തപസ്സിനനുയോജ്യമാണ്. ഉള്ളില്‍ പ്രശാന്തിയാര്‍ജ്ജിച്ചവന് ലോകത്ത്‌ എല്ലായിടത്തും സമാധാനം കണ്ടെത്തുവാന്‍ കഴിയും. എന്നാല്‍ മനസ്സില്‍ കാലുഷ്യവും അശാന്തിയും കുടികൊള്ളുമ്പോള്‍ ലോകവും അങ്ങിനെതന്നെയാവുന്നു. ഒരുവന്‍ തന്റെ അകമേ അനുഭവിക്കുന്നത് എന്താണോ അതുതന്നെയാണ് ബാഹ്യലോകത്ത് അവന് അനുഭവവേദ്യമാവുന്നത്.
      
“വാസ്തവത്തില്‍ ആകാശം, ഭൂമി, വായു, തുടങ്ങിയ എല്ലാം അന്തരേന്ദ്രിയങ്ങളുടെ ഭാഗമാണ്. അവ പുറമേ കാണപ്പെടുന്നു എന്ന് മാത്രം.” ഇവയെല്ലാം വിത്തിലിരിക്കുന്ന മരം പോലെയാണ് നിലകൊള്ളുന്നത്. അത് ബാഹ്യമായി പ്രകടമാകുന്നത് പൂക്കളിലെ സുഗന്ധം പോലെയുമാണ്. സത്യത്തില്‍ ആന്തരം, ബാഹ്യം, എന്നിങ്ങിനെ യാതൊന്നുമില്ല. ഉള്ളത് ബോധം മാത്രം. ബോധത്തിലെന്തു സങ്കല്‍പ്പിക്കുന്നുവോ അതങ്ങിനെ പ്രകടിതമായിത്തീരുന്നു.

ആത്മാവാണ് അകത്തും പുറത്തും എല്ലാടവും. ആരിലാണോ ആത്മപ്രഹര്‍ഷം പ്രഭചൊരിഞ്ഞു നിലനില്‍ക്കുന്നത്‌, ആരാണോ സുഖദു:ഖാദികളാല്‍ ചാഞ്ചാട്ടപ്പെടാത്തത്, ആരാണോ കര്‍മ്മങ്ങളെ ശരീരത്തിന്റെ മാത്രം പ്രവര്‍ത്തനങ്ങളായി ചെയ്യുന്നത്, അയാള്‍ സമതയില്‍ അഭിരമിക്കുന്നവനാണ്. അയാള്‍ ആകാശംപോലെ നിര്‍മ്മലന്‍. ആശയറ്റവന്‍. ഉചിതവും ക്ഷിപ്രവുമായി കാര്യങ്ങള്‍ ചെയ്യുന്നവന്‍. സുഖദു:ഖാദി വിഷയങ്ങളില്‍ തടിയോ കളിമണ്ണോ പോലെ നിര്‍മ്മമന്‍..

അയാളെപ്പോഴും പ്രശാന്തനാണ്. അയാള്‍ എല്ലാരെയും എല്ലാറ്റിനെയും കാണുന്നത് ആത്മാവായാണ്. മറ്റുള്ളവരുടെ വസ്തുക്കള്‍ അയാള്‍ക്ക് തൃണസമാനം. ഭയമില്ലായ്മ അയാള്‍ക്ക്‌ നൈസര്‍ഗ്ഗികം. അയാളാണ് സത്യദര്‍ശി. എന്നാല്‍ അജ്ഞാനി ചെറുതും വലുതുമായ വസ്തുക്കളുടെ അയാഥാര്‍ത്ഥ്യം – അവ മിഥ്യയാണെന്ന്– തിരിച്ചറിയുന്നില്ല. പരമമായ നൈര്‍മ്മല്യത്തെ പ്രാപിച്ചവന്, ജീവിച്ചാലും, മരിച്ചാലും, വീട്ടിലായാലും, കാട്ടിലായാലും, സമ്പത്തിലും, ദാരിദ്ര്യത്തിലും, ആഹ്ലാദ-ദു:ഖപ്രകടനങ്ങളിലും, സംന്യാസത്തിലും, സുഗന്ധലേപനം അണിഞ്ഞാലുമില്ലെങ്കിലും, ജഡപിടിച്ച മുടിയുണ്ടായാലുമില്ലെങ്കിലും, തീയില്‍ വീണാലും, പാപകര്‍മ്മങ്ങള്‍ ചെയ്താലും, പുണ്യകര്‍മ്മങ്ങളില്‍ ഏര്‍പ്പെട്ടാലും, ഈ ലോകചക്രം തീരുംവരെ ജീവിച്ചാലും അതിനിടയില്‍ മരിച്ചാലും, ഒരിക്കലും ഒരു ച്യുതിയുമുണ്ടാവുകയില്ല.  കാരണം അയാള്‍ ഒന്നും ‘ചെയ്യുന്നില്ല’.

ഉപാധിസ്ഥമായ മനസ്സാണ് അഹംകാരവും ധാരണകളും കൊണ്ട് കറപിടിക്കുന്നത്. എന്നാല്‍ ധാരണകള്‍ ഇല്ലാതെയായാല്‍ വിജ്ഞാനം ഉദിക്കുന്നു. അപ്പോള്‍ സഹജമായും മനോമാലിന്യങ്ങള്‍ നശിക്കുന്നു. പ്രബുദ്ധനായ മുനിയ്ക്ക്‌ എന്തെങ്കിലും കര്‍മങ്ങള്‍ ചെയ്യാനോ ചെയ്യാതിരിക്കാനോ ഇല്ല. അവകൊണ്ടദ്ദേഹത്തിനു നേടാന്‍ യാതൊന്നുമില്ല. പാറക്കല്ലില്‍ നിന്നും ചെടിപൊട്ടിമുളച്ചു മരമുണ്ടായി പടര്‍ന്നു പന്തലിക്കാത്തതുപോലെ  പ്രബുദ്ധന്റെയുള്ളില്‍ ആശകള്‍ മുളപൊട്ടുന്നില്ല. അഥവാ ചില ആശകള്‍ ക്ഷണനേരത്തേയ്ക്ക് ഉയര്‍ന്നു പൊങ്ങിയാല്‍ത്തന്നെ അവ വെള്ളത്തില്‍ എഴുതിയ വരപോലെ പൊടുന്നനേ ഇല്ലാതാവുന്നു. അങ്ങിനെ പ്രബുദ്ധതയിലഭിരമിക്കുന്ന മുനിയും പ്രപഞ്ചവും തമ്മില്‍ യാതൊരന്തരവുമില്ല.