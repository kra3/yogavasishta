\section{ദിവസം 210}

\slokam{
അനാമൃഷ്ടവികല്പാംശുശ്ചിദാത്മാ വിഗതാമയ:\\
ഉദിയായ ഹൃദാകാശോ തസ്യ വ്യോമ്നീവ ഭാസ്കര: (5/12/6)\\
}

വസിഷ്ഠൻ തുടർന്നു: ജ്ഞാനസാക്ഷാത്കാരത്തിന്റെ നിറവിൽ ജനകരാജൻ തന്റെ  നിയതകര്‍മ്മമായ  രാജഭരണം വളരെ ഭംഗിയായി നിർവ്വഹിച്ചു വന്നു. അദ്ദേഹത്തിൽ ആ കർമ്മങ്ങൾ ഒരു വിധത്തിലുമുള്ള  ചിന്താക്കുഴപ്പങ്ങള്‍ ഉണ്ടാക്കിയില്ലെന്നു മാത്രമല്ല അദ്ദേഹത്തിന്റെ പ്രവർത്തനങ്ങൾക്ക് മാനസീകവും ആത്മീയവുമായ അടിത്തറയും ആർജ്ജവവും ഉണ്ടായിരുന്നു താനും. അദ്ദേഹത്തിന്റെ മനസ്സ് രാജകീയഭോഗങ്ങളിലേയ്ക്ക് ആകർഷിക്കപ്പെട്ടില്ല. നിരന്തരം സുഷുപ്തിയിലെന്നപോലെയായിരുന്നു അദ്ദേഹത്തിന്റെ സ്ഥിതി. അദ്ദേഹത്തിന്‌ ഒന്നും നേടാനോ ഉപേക്ഷിക്കുവാനോ ഇല്ലായിരുന്നു. യാതൊരു മന:ശ്ചാഞ്ചല്യവുമില്ലാതെ അദ്ദേഹം ‘ഇപ്പോൾ, ഇവിടെ’ (വർത്തമാനത്തിൽ) എന്ന ഉണർവ്വോടെ ഉത്തമജീവിതം നയിച്ചു. അദ്ദേഹത്തിന്റെ വിവേകവിജ്ഞാനാദികളിൽ വീഴ്ച്ചയോ മേധാശക്തിയിൽ  കളങ്കമോ ഒരിക്കല്‍പ്പോലും  ഉണ്ടായില്ല.

“അദ്ദേഹത്തിന്റെ ഹൃദയത്തിൽ ആത്മജ്ഞാനത്തിന്റെ വെളിച്ചം - ചിദാത്മാവുദിച്ചിരുന്നു. ചക്രവാളത്തിലുദിച്ചുയരുന്ന സൂര്യനെപ്പോലെ, മാലിന്യലേശമേശാതെ, ദു:ഖത്തിന്റെ കണികപോലുമില്ലാതെയാണാ ചിദാത്മസ്വരൂപോദയം ഉണ്ടായത്.” ഈ വിശ്വത്തിലെ എല്ലാമെല്ലാം സ്ഥിതിചെയ്യുന്നത് ചിദ്-ശക്തിയിലാണെന്നദ്ദേഹം കണ്ടു. ആത്മജ്ഞാനനിരതനാകയാൽ അനന്തമായ ആത്മാവിൽ അദ്ദേഹം എല്ലാം ദർശിച്ചു. എന്തു സംഭവിക്കുന്നതും അതതിന്റെ സഹജസ്വഭാവം മൂലമാണെന്നറിഞ്ഞതുകൊണ്ട് അദ്ദേഹത്തെ അതീവമായ ആഹ്ലാദമോ ദു:ഖമോ തീണ്ടിയില്ല. നിരന്തരമായ സമതാഭാവം അദ്ദേഹത്തിൽ നിലനിന്നിരുന്നു. ജീവന്മുക്തനായിരുന്നു അദ്ദേഹം - ജീവിച്ചിരിക്കേ മുക്തിപദം പ്രാപിച്ചയാൾ.

നിസ്സംഗനായി, എന്നാൽ ഭംഗിയായി രാജഭരണം ചെയ്തുവരുമ്പോൾ തനിക്കു ചുറ്റുമുള്ള നന്മ-തിന്മകൾ ജനകനിലുണ്ടായിരുന്ന ആത്മജ്ഞാനത്തെ വളർത്തുകയോ തളർത്തുകയോ ചെയ്തില്ല. അനന്താവബോധത്തിൽ സ്ഥിരപ്രതിഷ്ഠനായിരുന്നതുകൊണ്ട് വൈവിദ്ധ്യമാർന്ന രാജകീയ കർമ്മധർമ്മങ്ങൾ ഉചിതമായി നടത്തുമ്പോഴും അദ്ദേഹം കർമ്മരഹിതന്റെ അവസ്ഥയിലായിരുന്നു. കാരണം, അദ്ദേഹത്തിൽ എല്ലാവിധ വാസനകളും ധാരണകളും അസ്തമിച്ചിരുന്നു. അതുകൊണ്ട് പ്രത്യക്ഷത്തിൽ കർമ്മനിരതനായിരിക്കുമ്പോഴും അദ്ദേഹം ഗാഢസുഷുപ്തിയിലെന്നപോലെ  പ്രശാന്തനായിരുന്നു. ഭൂത-ഭാവി കാല കാര്യങ്ങൾ അദ്ദേഹത്തെ അലട്ടിയില്ല. ഒരു പുഞ്ചിരിയോടെ, പാരിപൂർണ്ണമായും വർത്തമാനത്തിൽ മാത്രമായി അദ്ദേഹം ജീവിച്ചു.

ജനകന്റെ നേട്ടങ്ങളെന്തായിരുന്നാലും അതുണ്ടായത് ആത്മജ്ഞാനാന്വേഷണത്തിന്റെ ശക്തിമൂലമാണ്‌... ജനകനെ മാതൃകയാക്കി സത്യാന്വേഷണം ചെയ്ത് ജ്ഞാനത്തിന്റെ പരമസീമയിലെത്താൻ നാം പരിശ്രമിക്കണം. ഒരു ഗുരുവിനെ കിട്ടിയതുകൊണ്ടോ ശാസ്ത്രഗ്രന്ഥപഠനംകൊണ്ടോ സദ്കർമ്മങ്ങൾ ചെയ്തതുകൊണ്ടോ ആത്മജ്ഞാനം ഉണ്ടാവുകയില്ല. അതിന്‌ സദ് വൃത്തരുമായുള്ള സദ്സംഗം നല്‍കുന്ന പ്രചോദനത്താല്‍  സ്വയം ചെയ്യുന്ന ആത്മാന്വേഷണം മാത്രമേ വഴിയുള്ളു. ഒരുവനില്‍ അകമേ ഉദിക്കുന്ന ജ്ഞാനപ്രകാശം മാത്രമേ അതിനു വഴികാട്ടിയായുള്ളു. ഈ വെളിച്ചം കെടാതെ സൂക്ഷിച്ചാൽ യാതൊരന്ധകാരത്തിനും അവിടെ സ്ഥാനമുണ്ടാവുകയില്ല. 
