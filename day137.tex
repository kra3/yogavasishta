\newpage
\section{ദിവസം 137}

\slokam{
സർവ്വേഷു സുഖദു:ഖേഷു സർവാസു കലനാസു ച\\
മന: കർതൃ മനോ ഭോക്തൃ, മാനസം വിദ്ധി മാനവം (3/115/24)\\
}

കുറച്ചുനേരം ധ്യാനനിരതനായിരുന്നശേഷം രാമൻ ചോദിച്ചു: മഹർഷേ, യഥാർത്ഥ്യത്തില്‍ നിലനിൽപ്പില്ലാത്ത അജ്ഞാനത്തിന്‌ ഇത്രയധികം ചിന്താക്കുഴപ്പമുണ്ടാക്കാൻ കഴിയുന്നു എന്നത് എത്ര അത്ഭുതകരം! സത്യത്തിൽ ‘ഇല്ലാത്ത’ ഈലോകം എത്ര യഥാർത്ഥ്യമായാണ്‌ നമുക്കനുഭവപ്പെടുന്നത്‌!. ഇതെങ്ങിനെ സാധിക്കുന്നുവെന്ന് ഒന്നുകൂടി വിശദമാക്കിയാലും. ലവണ മഹാരാജാവിന്‌ പലവിധ ദുരിതാനുഭവങ്ങൾ ഉണ്ടാവാൻ എന്താണു കാരണം? മാത്രമല്ല, അരാണ്‌, അല്ലെങ്കിൽ എന്താണ്‌ ഈ അനുഭവങ്ങളുടെ ഭോക്താവ്?

വസിഷ്ഠൻ പറഞ്ഞു: രാമ: ബോധം ഈ ശരീരവുമായി ഏതെങ്കിലും തരത്തിൽ ബന്ധപ്പെട്ടിരിക്കുന്നു എന്നത് സത്യമല്ല. ശരീരം ബോധമണ്ഡലത്തിലെ ഒരു ഭ്രമകൽപ്പന മാത്രമാണ്‌.. ഒരു സ്വപ്നദൃശ്യം പോലെ. ബോധം, ചൈതന്യമെന്ന (ഊര്‍ജ്ജം) വസ്ത്രമുടുത്ത് സ്വയം ഒരു ജീവനായി പരിമിതഭാവം ധരിക്കുകയാണ്‌. ജീവനോ, വിക്ഷുപ്തമായ  ഈ ഊർജ്ജത്തിന്റെ ശക്തികൊണ്ട് പ്രത്യക്ഷലോകത്തിൽ ആമഗ്നമാവുകയാണ്‌.. പൂർവ്വാർജ്ജിതകർമ്മഫലങ്ങൾ അനുഭവിക്കുന്ന, സുഖവും ദു;ഖവും അറിയുന്ന, ശരീരമെടുത്ത ഈ സത്ത, അഹംകാരം, മനസ്സ്, ജീവൻ എന്നൊക്കെ അറിയപ്പെടുന്നു. ശരീരത്തിനോ, പ്രബുദ്ധതയാർജ്ജിച്ച ജീവനോ അനുഭവങ്ങൾ ഇല്ല. അജ്ഞാനാവൃതമായ മനസ്സാണ്‌ സുഖദുഖാനുഭവങ്ങളുടെ ഭോക്താവ്. സ്വപ്നദൃശ്യങ്ങളുണ്ടാവുന്നത് ഉറക്കത്തിൽ മാത്രമാണല്ലോ. അതുപോലെ അവിദ്യയിലാണ്‌ മനസ്സ് ബാഹ്യപ്രകടനമായ പ്രത്യക്ഷലോകത്തെ ഭാവനചെയ്തുണ്ടാക്കുന്നത്. ഉറക്കമുണരുമ്പോൾ സ്വപ്നമില്ല; ജ്ഞാനത്തിന്റെ പ്രബുദ്ധതയിൽ ലോകം ഇല്ല. മനസ്സ്, അവിദ്യ, മാനസീകോപാധികൾ, വ്യക്തി ബോധം എന്ന പേരുകളിലെല്ലാം അറിയപ്പെടുന്ന ജീവനാണ്‌ അനുഭവഭോക്താവ്‌.. ശരീരം ചൈതന്യരഹിതമാകയാൽ സുഖിക്കാനോ ദുരിതം സഹിക്കാനോ അതിനാവില്ല. അവിദ്യയാണ്‌ അശ്രദ്ധയ്ക്കും അജ്ഞാനത്തിനും ഹേതു. അതിനാൽ അവിദ്യയാണ്‌ സുഖദു:ഖങ്ങളുടെ ഭോക്താവ്.

മനസ്സുതന്നെയാണ്‌ ജനിക്കുന്നതും, വിലപിക്കുന്നതും, കൊല്ലുന്നതും, പോവുന്നതും, മറ്റുള്ളവരെ ഉപദ്രവിക്കുന്നതും. ഈ ശരീരമല്ല. “സുഖ ദു:ഖാനുഭവങ്ങളിലും എല്ലാ മോഹവിഭ്രാന്തികളിലും ഭാവനകളിലും മനസ്സാണെല്ലാം ചെയ്യുന്നത്; മനസ്സാണെല്ലാം അനുഭവിക്കുന്നത്; മനസ്സ് മനുഷ്യനാണ്‌.” ലവണ രാജാവിന്‌ ദുരിതങ്ങൾ സഹിക്കേണ്ടതായി വന്നതിനെക്കുറിച്ച് ഞാൻ പറഞ്ഞുതരാം. ഹരിശ്ചന്ദ്രന്റെ കുലത്തിലാണ്‌ ലവണൻ ജനിച്ചത്.

ലവണൻ ആലോചിച്ചു: 'എന്റെ മുത്തശ്ശനായ ഹരിശ്ചന്ദ്രന്‍ വലിയ യാഗകർമ്മങ്ങള്‍  ചെയ്ത്   മഹത്വമാർജ്ജിച്ച വ്യക്തിയാണ്‌.. എനിയ്ക്കും അതേപോലെ  യാഗം ചെയ്ത് മഹത്വം പ്രാപിക്കണം.' അദ്ദേഹം മനസാ  യാഗസാമഗ്രികളെല്ലാമൊരുക്കി; മനസ്സില്‍ത്തന്നെ പരികർമ്മികളെയെല്ലാം സംഘടിപ്പിച്ചു. സ്വന്തം പൂന്തോട്ടത്തിലിരുന്ന് ഒരുവർഷക്കാലം ഭാവനയിൽ ഈ യാഗം ഭംഗിയായി ചെയ്തു. മാനസീകമായി യാഗം പൂർത്തീകരിച്ചതിനാൽ അതിന്റെഫലവും അദ്ദേഹത്തിനു ലഭിച്ചു. അതുകൊണ്ട് രാമാ, മനസ്സാണ്‌ കർത്താവും ഭോക്താവും. അതുകൊണ്ട് മനസ്സിനെ മുക്തിപദത്തിലേയ്ക്ക് നയിച്ചാലും. 

