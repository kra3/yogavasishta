\section{ദിവസം 230}

\slokam{
വിഷയാൻപ്രതി ഭോ: പുത്ര സർവാനേവ ഹി സർവഥാ\\
അനാസ്ഥാ പരമാ ഹ്യേഷാ സാ യുക്തിർമനസോ ജയേ (5/24/17)\\
}

ബലി ചോദിച്ചു: അച്ഛാ, ഈ അതിശക്തനായ മന്ത്രിയെ ഏതു മാർഗ്ഗത്തിലൂടെയാണു കീഴ്പ്പെടുത്താനാവുക?

വിരോചനൻ പറഞ്ഞു: ഈ മന്ത്രിപുംഗവൻ അജയ്യനാണെങ്കിലും അവനെ കീഴടക്കാനുള്ള വിദ്യ ഞാൻ പറഞ്ഞു തരാം. ബുദ്ധികൂർമ്മതയോടെയുള്ള കർമ്മംകൊണ്ടയാളെ ഒരൊറ്റ നിമിഷത്തിൽ പിടികൂടാം. എന്നാൽ അങ്ങിനെയല്ലാത്ത പക്ഷം അയാൾ ഒരു വിഷം മുറ്റിയ സർപ്പം പോലെ എല്ലാത്തിനേയും കരിച്ചുകളയും. അയാളെ ബുദ്ധിപൂർവ്വം സമീപിക്കുന്നയാൾ കുട്ടികളുമായെന്നപോലെ അയാളുമായി കളികളിലേർപ്പെടുന്നു. അങ്ങിനെ അയാളെ കളിയിലൂടെ കീഴ്പ്പെടുത്തുന്നു. അപ്പോള്‍ അയാൾക്ക് മന്ത്രിയെ മറികടന്ന് രാജാവിനെ ദർശിച്ച് പരമപദം പൂകാം. രാജാവിനെ കണ്ടുകഴിഞ്ഞാൽപ്പിന്നെ മന്ത്രിയെ ചൊൽപ്പടിക്കു നിർത്താം. മന്ത്രി തന്റെ വരുതിയിലായാലോ, രാജാവിനെ കൂടുതൽ അടുത്തു ചെന്ന് തെളിമയോടെ ദർശിക്കം.

രാജാവിനെ കാണുംവരെ മന്ത്രി പിടിതരില്ല. മന്ത്രി പിടിയിലായാലല്ലാതെ രാജാവിനെ ദർശിക്കാനും കഴിയില്ല. രാജാവിന്റെ  ദർശനാം ലഭിക്കാത്തിടത്തോളം കാലം ഈ മന്ത്രി ശല്യക്കാരനാണ്‌, ദു:ഖം പരത്തുന്നവനാണ്‌.. മന്ത്രിയെ പിടിയിലാക്കുംവരെ രാജാവിനെ കാണാൻ സാധിക്കില്ല. അതിനാൽ ഒരുവന്റെ ബുദ്ധിപൂർവ്വമുള്ള പരിശ്രമം ഒരേസമയത്ത് രണ്ടുതരത്തിലാവണം. ഒന്ന് രാജാവിനെ കാണുവാനുള്ള പ്രയത്നം; രണ്ട് മന്ത്രിയെ കീഴ്പ്പെടുത്തുവാനുള്ള പരിശ്രമം. നിരന്തരമായുള്ള തീവ്ര സാധന കൊണ്ട് നിനക്ക് ഇതു രണ്ടും സാദ്ധ്യമാണ്‌.. അങ്ങിനെ നിനക്ക് ഇനിയൊരിക്കലും ദു:ഖാനുഭവങ്ങളുണ്ടാവുകയില്ലാത്ത ഒരിടത്തെത്താം. നിത്യപ്രശാന്തനിരതരായ മാമുനിമാർ അവിടെയാണു വാഴുന്നത്.

മകനേ ഇനി ഇക്കാര്യങ്ങൾ തെളിച്ചുതന്നെ പറഞ്ഞു തരാം. ഞാൻ പറഞ്ഞ രാജ്യം മുക്തിപദമാണ്‌.. അവിടെ ദു:ഖങ്ങളില്ല. അവിടുത്തെ രാജാവാണ്‌ ആത്മാവ്. ആത്മാവ് എല്ലാ മണ്ഡലങ്ങൾക്കും ബോധാവസ്ഥകൾക്കും അതീതനത്രേ. മനസ്സാണ്‌ മന്ത്രി. ഈ മനസ്സാണ്‌ കളിമണ്ണിൽ നിന്നു കുടമെന്നപോലെ ഈ ലോകമുണ്ടാക്കിയത്. മനസ്സു കീഴടക്കിയാൽ എല്ലാം കീഴടക്കി. ബുദ്ധികൂർമ്മതയോടെയുള്ള പ്രവൃത്തിയെക്കൂടാതെ മനസ്സിനെ വെല്ലുക അസാദ്ധ്യം.

ബലി വീണ്ടും ചോദിച്ചു: അച്ഛാ ഏതു തരം ബുദ്ധിസാധനയാണ്‌ മനസ്സു കീഴടക്കാൻ എനിക്കുതകുക?

വിരോചനൻ പറഞ്ഞു:“ എല്ലാ സമയത്തും വിഷയവസ്തുസംബന്ധിയായ പ്രതീക്ഷകളിൽ നിന്നും, ആശകളിൽ നിന്നും, പ്രത്യാശകളിൽ നിന്നുമുള്ള പരിപൂർണ്ണ മോചനം നേടുകയാണ്‌ മനസ്സു കീഴടക്കാനുള്ള ഏറ്റവും ബുദ്ധിപരമായ മാർഗ്ഗം.” ഇങ്ങിനെ ശക്തിമാനായ മനസ്സെന്ന ഈ മദയാനയെ മെരുക്കാൻ നമുക്കു കഴിയും.

ഇത് ക്ഷിപ്രസാദ്ധ്യവും എന്നാല്‍ അതേസമയം അതികഠിനവുമണെന്നുപറയാം. തീവ്രമായ സാധനയിലേർപ്പെടാത്തവർക്കിതു കഠിനം. എന്നാൽ സ്വപരിശ്രമത്തിൽ അലംഭാവമില്ലാത്ത സാധകനിതെളുപ്പമാണ്‌.. വിത്തിടാതെ, നനച്ചു വളര്‍ത്താതെ, വിളവെടുക്കാനാവില്ല. നിരന്തരമായ സാധനകൂടാതെ മനസ്സടങ്ങുകയില്ല. അതുകൊണ്ട് ത്യാഗത്തിന്റെ, സംന്യാസത്തിന്റേതായ ഈ മാർഗ്ഗം സ്വീകരിച്ചാലും. ഇന്ദ്രിയസുഖാനുഭവങ്ങളിൽ നിന്നും പിന്തിരിഞ്ഞാലല്ലാതെ ദു:ഖസാന്ദ്രമായ ഈ ലോകത്തിലെ നട്ടംതിരിയൽ അവസാനിക്കുകയില്ല. അതിശക്തനാണെങ്കിലും ഒരുവന്‌ ലക്ഷ്യത്തിലെത്താൻ അവിടേയ്ക്ക് യാത്ര പുറപ്പെടുകതന്നെ വേണം. പരിപൂർണ്ണമായ നിർമമതയുടെ, അനാസക്തിയുടെ തലത്തിലേയ്ക്കെത്താൻ നിസ്തന്ദ്രമായ സാധന കൂടിയേ തീരൂ. 
