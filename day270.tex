\section{ദിവസം 270}

\slokam{
കദോപശാന്തമനനോ ധരണീധരകന്ദരേ\\
സമേഷ്യാമി ശിലാസ്മായം നിര്‍വികല്‍പസമാധിനാ (5/51/33)\\
}

വസിഷ്ഠന്‍ തുടര്‍ന്നു: അതീവ സൂക്ഷ്മമായി മനസ്സില്‍ നിനക്കുണ്ടാവുന്ന ധാരണകളിലും സങ്കല്‍പ്പങ്ങളിലും നീയായിട്ട് തീരുമാനങ്ങള്‍ ഒന്നും എടുക്കേണ്ടതില്ല. കാരണം മനസ്സുണ്ടാക്കിയത് കാലമാണ്. കാലം തന്നെയാണതിനെ ബാലവത്താക്കുന്നതും. കാലം ഈ ശരീരമാകുന്ന പാഴ്വള്ളിച്ചെടിയെ ഇല്ലാതാക്കും മുമ്പ് വിവേകജ്ഞാനം കൊണ്ട് മനസ്സിനെ നിന്റെ വരുതിയിലാക്കുക. എന്റെയീ വാക്കുകള്‍ ഭക്തിയോടെ മനനം ചെയ്താലും. അങ്ങിനെ നിനക്ക് പരമാനന്ദത്തെ പ്രാപിക്കാം.

രാമാ, ഇനി മഹര്‍ഷി ഉദ്ദാലകന്‍ എന്ന മാമുനി പരം പൊരുളിനെ സാക്ഷാത്കരിച്ചതിന്റെ കഥ ഞാന്‍ പറയാം. ലോകത്തിന്റെ ഒരു കോണില്‍ ഗന്ധമാദനം എന്ന് പേരായ ഒരു പര്‍വ്വതം. അതിന്റെയൊരു കൊടുമുടിയിലായി വലിയൊരു മരം. ആ പ്രദേശത്തായിരുന്നു ഉത്താലകന്‍ ജീവിച്ചിരുന്നത്. ചെറു പ്രായത്തിലേതന്നെ അദ്ദേഹത്തിനുള്ളില്‍  ആത്മജ്ഞാനത്തിനായി സ്വയം പരിശ്രമിക്കണമെന്ന ബോധം അങ്കുരിച്ചിരുന്നു. നിര്‍മല ഹൃദയനായിരുന്നുവെങ്കിലും ചെറുപ്പത്തിന്റെ ചഞ്ചലതയും അറിവില്ലായ്മയും അദ്ദേഹത്തിലുണ്ടായിരുന്നു താനും.  

തീവ്രതപസ്സിനാലും ശാസ്ത്ര പഠനത്താലും അദ്ദേഹത്തില്‍ വിജ്ഞാനം വര്‍ദ്ധിച്ചുവന്നു. ഒരിക്കല്‍ ഏകനായി ഇരിക്കുമ്പോള്‍ അദ്ദേഹമിങ്ങിനെ ആലോചിച്ചു. എന്താണീ മുക്തി? അതാണ്‌ നെടാവുന്നതില്‍ ഏറ്റവും ഉത്തമായത് എന്ന് പറയപ്പെടുന്നു. അത് കിട്ടിയാല്‍പ്പിന്നെ ഒരുവന് ദു:ഖനിവൃത്തിയായി. പിന്നീടവനു പുനര്‍ജന്മങ്ങളില്ലത്രെ! എനിക്കെപ്പോഴാണാ സ്ഥിതിയില്‍ ശാശ്വതമായി ആമഗ്നനാവാന്‍ സാധിക്കുക? ആഗ്രഹങ്ങളുടേയും ആസക്തികളുടേയും ഫലമായുണ്ടാവുന്ന മാനസീക വിക്ഷോഭങ്ങള്‍ എന്നില്‍ എപ്പോഴാണവസാനിക്കുക? ‘ഞാനിത് ചെയ്തു’, ‘എനിക്കിത് ചെയ്യാനുണ്ട്’ തുടങ്ങിയ ചിന്തകളില്‍ നിന്നുമെന്നാണ് ഞാന്‍ വിടുതല്‍ നേടുക?സദാ ജലത്തില്‍ക്കഴിയുന്ന താമരയിലയില്‍ വെള്ളമൊട്ടിപ്പിടിക്കാത്തതുപോലെ ഇഹലോകബന്ധുക്കളും അവരുടെ വിചിത്രരീതികളും എന്റെ മനസ്സിനെ ചഞ്ചലപ്പെടുത്തുന്നതെപ്പോഴാണവസാനിക്കുക?

എപ്പോഴാണ് ഞാന്‍ പരമവിജ്ഞാനത്തിന്റെ തോണിയിലേറി മുക്തിപദത്തിന്റെ തീരമണയുക? മനുഷ്യരുടെ വൈവിദ്ധ്യ പ്രവര്‍ത്തനങ്ങളെ ചെറിയ കുട്ടികളുടെ കണ്ണിലെ വിസ്മയത്തോടെ എന്നാണെനിക്കു ദര്‍ശിക്കാനാവുക? എന്റെ മനസ്സെപ്പോഴാണു പ്രശാന്തമാവുക? അനുഭവങ്ങളിലെ വിഷയ-വിഷയീ ഭിന്നത അനന്താവബോധത്തിന്റെ നിറവില്‍ എന്നിലെപ്പോഴാണവസാനിക്കുക? ഈ കാലമെന്ന പ്രഹേളികയെ അതിന്റെ സ്വാധീനത്തില്‍പ്പെടാതെ എന്നാണു ഞാന്‍ അറിയുക?

“ഒരു ഗുഹാവാസിയായി, പാറപോലെ ഉറച്ച മനസ്സില്‍ പ്രശാന്തിയോടെ ഞാനിനിയെന്നാണെന്നിലെ  ചിന്തകളുടെ സഞ്ചാരത്തെ തീരെ ഇല്ലാതാക്കുക?” ഇങ്ങിനെ ചിന്തിച്ച്  ഉദ്ദാലകന്‍ തന്റെ ധ്യാനസപര്യ തുടര്‍ന്നു. എന്നാലദ്ദേഹത്തിന്റെ മനസ്സിന്റെ ഇളക്കം നിന്നില്ല. ചില ദിനങ്ങളില്‍ ബാഹ്യവസ്തുക്കളെ മനസ്സുപേക്ഷിച്ചു നിര്‍മ്മലമായിത്തീര്‍ന്നിരുന്നു. എന്നാല്‍ മറ്റുള്ള ദിനങ്ങളില്‍ മനസ്സ് അതിദ്രുതം ചാഞ്ചാടിക്കൊണ്ടിരുന്നു. മനസ്സിന്റെ ഈ ക്ഷിപ്രമാറ്റങ്ങളില്‍ പരിതപിച്ച് അദ്ദേഹം വനത്തില്‍ അലഞ്ഞു നടന്നു. ഒരുദിനം അദ്ദേഹം വനത്തില്‍ മറ്റാരും എത്തിപ്പെടാത്ത ഏകാന്തമായ ഇരിടം കണ്ടെത്തി. അവിടെ പരമശാന്തി ലഭിക്കാന്‍ ഉചിതമായ ഒരു ഗുഹയുണ്ടായിരുന്നു. സുന്ദരമായ സ്ഥലം. ചുറ്റുമുള്ള വള്ളിച്ചെടികളില്‍ പൂക്കളുലഞ്ഞു നിന്നിരുന്നു. സുഖകരമായ താപനിലയുള്ള ആ ഗുഹ മരതകല്ലില്‍ കൊത്തിവച്ചതുപോലെ പ്രശോഭിച്ചു.