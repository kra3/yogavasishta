\section{ദിവസം 262}

\slokam{
കിം മേ ജീവിതദു:ഖേന മരണം മേ മഹോത്സവ:\\
ലോക നിന്ധ്യസ്യ ദുര്‍ജന്തോര്‍ജീവിതാന്‍മരണം വരം(5/46/43)\\
}

വസിഷ്ഠന്‍ തുടര്‍ന്നു: മന്ത്രിമാരാലും സുന്ദരികളായ ദാസിമാരാലും ഭക്ത്യാദരവോടെ പരിസേവിതനായ ഗാവലരാജാവ് തന്റെ എളിമയാര്‍ന്ന പൂര്‍വ്വകാലം അപ്പാടെ മറന്നുപോയി. എട്ടുകൊല്ലങ്ങള്‍ കടന്നുപോയി. അദ്ദേഹം കുറ്റമറ്റരീതിയില്‍ നീതിപൂര്‍വ്വം, ദയാദാക്ഷിണ്യപൂര്‍വ്വം രാജ്യം ഭരിച്ചു. ഒരുദിവസം അദ്ദേഹം ആഭരണങ്ങളോ രാജകീയ വസ്ത്രങ്ങളോ ധരിക്കാതെ സാധാരണ വേഷത്തില്‍ അന്തപ്പുരത്തില്‍ ഉലാത്തുകയായിരുന്നു. സ്വന്തം പ്രഭവത്തില്‍ ആത്മവിശ്വാസമുള്ളവര്‍ക്ക് ബാഹ്യവേഷഭൂഷാദികളില്‍ വലിയ താല്‍പ്പര്യമുണ്ടാവുകയില്ലല്ലോ. 
കൊട്ടാരത്തിനു വെളിയില്‍ നോക്കിയപ്പോള്‍ ചില ചണ്ഡാളന്മാര്‍ കൂട്ടംകൂടി നിന്ന്  തനിക്കു പരിചിതമായ ചില പാട്ടുകള്‍ പാടുന്നു. അദ്ദേഹവും കൊട്ടാരത്തിനു വെളിയില്‍ ഇറങ്ങി  ആ കൂട്ടത്തില്‍ ചേര്‍ന്നു പാട്ടുകള്‍ പാടാന്‍ തുടങ്ങി. കൂട്ടത്തില്‍ പ്രായമായ ഒരാള്‍ അദ്ദേഹത്തെ തിരിച്ചറിഞ്ഞു. ‘എടാ കടഞ്ചാ നീയെന്താണിവിടെ? നിന്റെ പാട്ടിനു സമ്മാനമായി ഇവിടുത്തെ രാജാവ് നിനക്ക് എന്തൊക്കെ തന്നു? തന്നെക്കണ്ടിട്ടെനിക്ക് സന്തോഷമായി. പഴയ കൂട്ടുകാരെ കാണുന്നതില്‍ ആരാണ് സന്തോഷിക്കാത്തത്?’

ഗാവലന്‍, കിഴവന്റെ ഈ സന്തോഷപ്രകടനം കണ്ടില്ലെന്നു നടിച്ച് കൊട്ടാരത്തിനുള്ളിലേയ്ക്ക് വേഗം തിരിച്ചുപോയി. എന്നാല്‍ രാജകൊട്ടാരത്തിലെ വനിതകളും പ്രമാണിമാരും ദൂരെനിന്നിതു കാണുന്നുണ്ടായിരുന്നു. അവര്‍ അത്ഭുതപ്പെട്ടു.  

അവരുടെ അമ്പരപ്പ് മാറിയില്ല. തങ്ങളുടെ രാജാവ് അസ്പര്‍ശ്യനായ വെറുമൊരു ചണ്ഡാളനാണെന്ന്  അവര്‍ക്ക് വിശ്വസിക്കാനായില്ല. അവര്‍ അദ്ദേഹത്തെ അവഗണിക്കാന്‍ തുടങ്ങി. നാറുന്നൊരു ശവത്തെയെന്നപോലെ രാജാവിനെ അവര്‍ വെറുത്തു തുടങ്ങി. മന്ത്രിമാരും രാജ്ഞിമാരും ദാസികളുമെല്ലാം അദ്ദേഹത്തെ അവഗണിച്ചു. ഗാവലന്‍ തന്റെ പൂര്‍വ്വരൂപത്തിലായി. കറുത്തിരുണ്ട്, ഗര്‍ഹണീയനായ ഒരു കാട്ടുജാതിക്കാരന്‍. തന്നെ.

നഗരവാസികളാരും അദ്ദേഹത്തെ വകവെച്ചില്ല. അദ്ദേഹത്തെ കാണുമ്പോഴേ അവര്‍ വഴിമാറി നടന്നു. കൊട്ടാരത്തില്‍ ആളുകള്‍ക്കിടയില്‍ വസിക്കുമ്പോഴും അദ്ദേഹം തികച്ചും ഒറ്റപ്പെട്ടിരുന്നു. രാജാവാണെങ്കിലും താന്‍ ഒരഗതിയായി എന്ന് അദ്ദേഹത്തിനു തോന്നി. ആരും രാജാവിനോട് മിണ്ടാതെയായി. അദ്ദേഹത്തിന്‍റെ ഉത്തരവുകള്‍ നടപ്പിലാക്കാതെയായി.

നഗരത്തിലെ നേതാക്കന്മാര്‍ യോഗംകൂടി ഇങ്ങിനെ പറഞ്ഞു.’കഷ്ടം! നായകളെ വേവിച്ചു തിന്നുന്ന ഈ കാട്ടുജാതിക്കാരനെ സ്പര്‍ശിക്കുകയാല്‍ നാമെല്ലാം അശുദ്ധരായിരിക്കുന്നു. മരണത്താലല്ലാതെ ഈ അശുദ്ധി നമ്മില്‍ നിന്നകലുകയില്ല. നമുക്കൊരു വന്‍ചിതയൊരുക്കി അതില്‍ചാടി ഈ നശിച്ച ജീവിതമാവസാനിപ്പിക്കാം. അങ്ങിനെ നമ്മുടെ ആത്മാവിനെ പരിശുദ്ധമാക്കാം.’ ഇങ്ങിനെ നിശ്ചയിച്ച് അവര്‍ വലിയൊരു ചിത കൂട്ടി. ഓരോരുത്തരായി അവര്‍ ചിതയില്‍ച്ചാടി. അങ്ങിനെ നാട്ടിലെ മുതിര്‍ന്നവരൊക്കെ ജീവനൊടുക്കിക്കഴിഞ്ഞപ്പോള്‍ നാട്ടില്‍ അരാജകത്വം നടമാടി.

ഗാവല രാജാവാലോചിച്ചു. “കഷ്ടം! ഞാന്‍ കാരണമാണിതൊക്കെ വന്നു കൂടിയത്! “ഞാനിനിയെന്തിനു ജീവിക്കാണം? മരണമാണെനിക്കഭികാമ്യം. മാറ്റുള്ളവരാല്‍ അപമാനിതനായി ജീവിക്കുന്നതിലും ഭേദം മരണം വരിക്കുകയാണ്.” ഇങ്ങിനെ പറഞ്ഞു ഗാവലനും നിശ്ശബ്ദനായി ചിതയില്‍ച്ചാടി മരണം വരിച്ചു. അഗ്നി ഗാവലന്റെ ശരീരത്തെ ആഹരിച്ചു തുടങ്ങിയപ്പോള്‍ ജലത്തില്‍ മുങ്ങി ധ്യാനമഗ്നനായി നാമം ജപിച്ചിരുന്ന ഗാധിയ്ക്ക് തന്റെ ബോധം വീണ്ടുകിട്ടി.

ഒരു ദിനംകൂടി അവസാനിച്ചു. കഥ പറച്ചില്‍ നിര്‍ത്തി സഭ പിരിഞ്ഞു.