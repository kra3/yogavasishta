\newpage
\section{ദിവസം 076}

\slokam{
:തസ്മിൻപ്രഥമത: സർഗേ യ യഥാ യത്ര സംവിദ:\\
(കചിതാസ്താസ്തഥാ തത്ര സ്ഥിതാ അദ്ധ്യാപി നിശ്ചലാ: (3/54/1)\\
}

സരസ്വതി പറഞ്ഞു: ജ്ഞാനസ്വരൂപതലത്തില്‍ എത്തിയവര്‍ക്കുമാത്രമേ സൂക്ഷ്മാവസ്ഥയെ
 പ്രാപിക്കാനാവൂ. മറ്റുള്ളവര്‍ക്കതു ലഭ്യമല്ല. ഈ ലീല ആ തലത്തിലെത്തിയിട്ടില്ല. അവളുടെ ഭര്‍ത്താവു ജീവിചിരുന്ന നഗരത്തിലെത്തിയതായി അവള്‍ സങ്കല്‍പ്പിച്ചിരുന്നുവെന്നു മാത്രം.

പ്രബുദ്ധയായ ആദ്യത്തെ ലീല പറഞ്ഞു: ദേവി, അതെല്ലാം അവിടുന്നു പറഞ്ഞതുപോലെ തന്നെയാകട്ടെ. എങ്കിലും ഒന്നു പറഞ്ഞാലും: എങ്ങിനെയാണു പദാര്‍ത്ഥങ്ങള്‍ക്ക്‌ ഗുണങ്ങളുണ്ടാവുന്നത്‌? അഗ്നിയ്ക്ക്‌ ചൂട്‌, മഞ്ഞിനു തണുപ്പ്‌, ഭൂമിക്ക്‌ ദൃഢത എന്നിവ എങ്ങി്‌നെ ഉണ്ടാവുന്നു? എങ്ങി്‌നെയാണ്‌ നിയതി-ലോക ക്രമം ഉണ്ടായത്‌?

സരസ്വതി പറഞ്ഞു: വത്സേ, വിശ്വപ്രളയസമയത്ത്‌ വിശ്വം സമ്പൂര്‍ണ്ണമായും അപ്രത്യക്ഷമായി. അനന്തമായ ബ്രഹ്മം മാത്രമേ പ്രശാന്താവസ്ഥയില്‍ നിലനിന്നിരുന്നുള്ളു. ഈ അനന്തതയില്‍ , അതു ബോധസ്വരൂപമാകയാല്‍ 'ഞാന്‍' എന്നും അതിനുശേഷം 'ഞാന്‍ പ്രകാശരേണുവാണ്‌' എന്നുമുള്ള തോന്നലുകളുളവായി. അവ സ്വാനുഭവവുമായി. അതിനുള്ളില്‍ വൈവിധ്യമാര്‍ന്ന ജീവജാലങ്ങളെ സങ്കല്‍പ്പിക്കുകമൂലം അവ യഥാര്‍ത്ഥ ഭാവം കൈക്കൊണ്ടു. അതിന്റെ സ്വഭാവം ശുദ്ധബോധമാകയാല്‍ അതിലെ സങ്കല്‍പ്പങ്ങള്‍ കൃത്യമായി അതേപടി നാനാവിധത്തിലുള്ള പദാര്‍ത്ഥങ്ങളായി പരിണമിച്ചു.

"എന്തൊക്കെ എവിടെയൊക്കെ എങ്ങിനെയൊക്കെ അനന്താവബോധത്തില്‍ ആദ്യസൃഷ്ടിയില്‍ സങ്കല്‍പ്പിച്ചുവോ അവയെല്ലാം അങ്ങിനെത്തന്നെ അവിടെ നിലകൊണ്ടു. അവയ്ക്ക്‌ മാറ്റമൊന്നുമില്ലാതെ ഇപ്പോഴും നിലനില്‍ക്കുന്നു." അങ്ങിനെയാണ്‌ കൃത്യമായ ഒരു ക്രമം (പ്രകൃതിനിയമം) ഇവയ്ക്കുണ്ടായത്‌. വാസ്തവത്തില്‍ ഈ ക്രമസ്വഭാവം അനന്താവബോധത്തിന്റെ സഹജഭാവമത്രേ. ഈ പദാര്‍ത്ഥങ്ങള്‍ എല്ലാം അവയുടെ സ്വഭാവസവിശേഷതകളടക്കം വിശ്വപ്രളയസമയത്ത്‌ ഒരു സാദ്ധ്യതാസാന്നിദ്ധ്യമായി നിലനിന്നിരുന്നു. മറ്റ്‌ എന്തിലേയ്ക്കാണിതിനുപോവാന്‍ കഴിയുക? എങ്ങിനെയാണ്‌ ഉള്ള ഒരു വസ്തു ഇല്ലായ്മയാവുക? കൈവളയായി കാണപ്പെടുന്ന സ്വര്‍ണത്തിന്‌ രൂപമൊന്നുമില്ലാതാവുക അസാദ്ധ്യം. ഈ സൃഷ്ടിയുടെ ഘടകങ്ങളെല്ലാം തികഞ്ഞ ശൂന്യതമാത്രമാണെങ്കിലും ഏതൊക്കെ ഘടകങ്ങള്‍ ആദിയില്‍ ചിന്താമാത്രമായി ഉണ്ടായിരുന്നുവോ, അവയുടെ സ്വഭാവസവിശേഷതകളടക്കം കൃത്യമായ ഒരു പ്രകൃതിക്രമം ഇന്നുവരെ നിലനിന്നുപോന്നിട്ടുണ്ട്‌.ഇതെല്ലാം ആപേക്ഷികതലത്തിലേ ഉള്ളു- കാരണം വിശ്വം സൃഷ്ടിക്കപ്പെട്ടിട്ടേ ഇല്ല. എല്ലാം അനന്ത അവബോധമല്ലാതെ മറ്റൊന്നുമല്ല.

വസ്തുപ്രകടനം എന്നതിന്റെ ധര്‍മ്മം തന്നെ വസ്തുവിനെ യാഥാര്‍ഥ്യം എന്നു തോന്നിപ്പിക്കുക എന്നതാണ്‌. പ്രകൃതിനിയമം- നിയതി എന്നതിന്റെ സ്വഭാവമെന്തെന്നാല്‍ അതിനെ ആര്‍ക്കും മാറ്റാനിതുവരെ കഴിഞ്ഞിട്ടില്ല എന്നതാണ്‌. അനന്താവബോധം തന്നെ ഈ ഘടകപദാര്‍ത്ഥങ്ങളെ സ്വപ്രജ്ഞയില്‍ ആലോചിച്ചുണ്ടാക്കി അവയെ അനുഭവിച്ചു. ആ അനുഭവങ്ങള്‍ മൂര്‍ത്തീകരിച്ചതായി .കാണപ്പെട്ടു

