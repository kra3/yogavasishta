 
\section{ദിവസം 088}

\slokam{
അസ്തീഹ നിയതി ബ്രാഹ്മീ ചിച്ഛക്തി: സ്പന്ദ രൂപിണീ\\
അവശ്യഭവിതവൈൃകസത്താ സകലകല്‍പഗാ (3/62/8)\\
}

രാമന്‍ ചോദിച്ചു: ബ്രഹ്മം മാത്രമേ സത്യമായുള്ളു എന്നത്‌ നിശ്ചയം! മഹര്‍ഷേ അങ്ങിനെയെങ്കില്‍ എന്തിനാണീ ലോകത്ത്‌ സത്യത്തിന്റെ പൊരുളറിഞ്ഞ മഹാത്മാക്കളും ഋഷിവര്യന്മാരും ദൈവനിയോഗത്താലെന്നവണ്ണം വര്‍ത്തിക്കുന്നത്‌? എന്താണ്‌ ഈ ദൈവം?

വസിഷ്ഠന്‍ പറഞ്ഞു: "രാമാ, അനന്താവബോധത്തിന്റെ ചൈതന്യം, ഊര്‍ജ്ജം, എല്ലായിടത്തും എല്ലായ്പ്പോഴും ചലിച്ചുകൊണ്ടേയിരിക്കുന്നു. എല്ലാ അനിവാര്യമായ ഭാവിസംഭവങ്ങള്‍ക്കും പിന്നിലെ സത്യം അതാണ്‌.. കാരണം ആ ഊര്‍ജ്ജം കാലാതീതമായി യുഗങ്ങളേയും ഭേദിച്ച്‌ നിലകൊള്ളുന്നു." ആ ശക്തിയാലാണ്‌ പ്രപഞ്ചത്തിലെ ഓരോ വസ്തുവിന്റേയും സ്വാഭാവികത ക്രമീകൃതമായിരിക്കുന്നത്‌... 'ചിത്ശക്തി' എന്നറിയപ്പെടുന്ന ഇതിന്‌ മഹാസത്ത എന്നും പേരുണ്ട്‌.. മഹാചിതി (മഹാബുദ്ധി); മഹാശക്തി, മഹാദൃഷ്ടി, മഹാക്രിയ, മഹോദ്ഭാവം, മഹാസ്പന്ദം എന്നെല്ലാം ഇതറിയപ്പെടുന്നു. ഈ ശക്തിയാണ്‌ എല്ലാറ്റിനും ഗുണഗണങ്ങള്‍ നല്‍കുന്നത്‌. എന്നാല്‍ ഇത്‌ പരബ്രഹ്മത്തില്‍ നിന്നും സ്വതന്ത്രമല്ല. 'ആകാശത്തിലെ അപ്പം' പോലെ 'യാഥാര്‍ഥ്യം'ആണത്‌ . (വ്യാവഹാരികമായി, തോന്നുന്നു എന്നര്‍ത്ഥം). മഹര്‍ഷിമാര്‍ ബ്രഹ്മത്തിനും ഈ ശക്തിക്കും തമ്മില്‍ വാക്കുകള്‍ കൊണ്ടു മാത്രം ഒരു വ്യതിരിക്തത കല്‍പ്പിച്ചിരിക്കുന്നു. എന്നിട്ടു പറയുന്നു, ഈ ശക്തിയാണ്‌ സൃഷ്ടിക്കു കാരണമെന്ന്. ശരീരവും അതിന്റെ ഭാഗങ്ങളും എന്നു പറയുമ്പോള്‍ അവ തമ്മിലെ വിവേചനം വാക്കുകളില്‍മാത്രമാണല്ലോ. 

അനന്താവബോധം സഹജമായ ശക്തിയെ തിരിച്ചറിയുന്നത്‌ ഒരുവന്‍ സ്വശരീരത്തിലെ അവയവങ്ങളെ തിരിച്ചറിയുംപോലെയാണ്‌. ഈ 'അറിവിന്‌', 'നിയതി' എന്നു പേര്‌. ഇതാണ്‌ പരമ്പൊരുളിന്റെ ശക്തി. ഇതിനാലാണ്‌ എല്ലാറ്റിന്റേയും സ്വഭാവം നിര്‍ണ്ണയിക്കപ്പെടുന്നത്‌. ഇത്‌ ദൈവം, ദിവ്യനിയോഗം എന്നെല്ലാം അറിയപ്പെടുന്നു. നീ ഇപ്പോള്‍ എന്നോടീ ചോദ്യങ്ങള്‍ ചോദിച്ചത്‌ ഇപ്പറഞ്ഞ നിയതിയുടെ നിയോഗം മൂലമാണ്‌.. ഞാന്‍ പറഞ്ഞ കാര്യങ്ങള്‍ക്കനുസരിച്ച്‌ നീ കര്‍മ്മോന്മുഖനാവുന്നതും നിയതിയുടെ നിയോഗമാണ്‌.. ഒരുവന്‍ 'എനിക്കു ദൈവം ഭക്ഷണം കൊണ്ടുതരും അതിനാല്‍ ഞാന്‍ ജോലിയൊന്നും ചെയ്യുന്നില്ല' എന്നു പറഞ്ഞിരിക്കുന്നതും നിയതിയാലാണ്‌.. രുദ്രനുപോലും നിയതിയെ മാറ്റി വെയ്ക്കാനാവില്ല. എന്നാല്‍ വിവേകശാലിയായ മനുഷ്യനെ ഇതൊന്നും സ്വപ്രയത്നത്തില്‍ നിന്നും പിന്തിരിപ്പിക്കുന്നില്ല. കാരണം നിയതി പ്രാവര്‍ത്തികമാവണമെങ്കില്‍ സ്വപ്രയത്നത്തിലൂടെ മാത്രമേ സാധിക്കൂ. 

നിയതിക്ക്‌ രണ്ടുതലങ്ങളുണ്ട്‌. മാനുഷീകവും അതിമാനുഷികവും. സ്വപ്രയത്നം ഫലപ്രദമാവുന്നത്‌ മാനുഷീക തലത്തിലും ഫലപ്രദമാവാത്തത്‌ അതിമാനുഷീക തലത്തിലുമാണ്. 'എല്ലാം ദൈവഗത്യം' എന്നു പറഞ്ഞ്‌ കര്‍മ്മരഹിതനായിരിക്കുന്നവന്റെ ജീവന്‍ താമസംവിനാ ശരീരം വിട്ടകലുന്നു. കാരണം ജീവന്‍ എന്നുപറഞ്ഞാല്‍ അതു കര്‍മ്മമാണ്‌.. അവന്‌ ശ്വാസമടക്കിപ്പിടിച്ച്‌ അത്യുന്നതമായ ബോധതലത്തിലെത്തി മുക്തിപദത്തെ പ്രാപിക്കാം. പക്ഷേ അതിന്‌ വളരെ കഠിനമായ പ്രയത്നം കൂടിയേ തീരൂ. അനന്താവബോധം തന്നെയാണ്‌ ഒരിടത്തൊന്നായും മറ്റൊരിടത്ത്‌ വേറൊന്നായും കാണപ്പെടുന്നത്‌.. ആ ബോധവും അതിന്റെ ശക്തിയും തമ്മില്‍ ഭിന്നതകളില്ല. ജലവും ഓളങ്ങളും തമ്മില്‍ വ്യത്യാസമില്ല. ശരീരവും അവയവങ്ങള്‍ തമ്മിലും വ്യതാസമില്ല. അജ്ഞാനികള്‍ മാത്രമേ അവയില്‍ ഭിന്നത ദര്‍ശിക്കൂ. 

