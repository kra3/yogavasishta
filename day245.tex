\section{ദിവസം 245}

\slokam{
ഭാവനാഭാവമാശ്രിത്യ ഭാവസ്ത്യജതി ദു:ഖതാം\\
പ്രേക്ഷ്യ ഭാവമഭാവേന ഭാവസ്ത്യജതി ദുഷ്ടതാം (5/34/99)\\
}

പ്രഹ്ലാദന്‍ തന്റെ മനനം തുടര്‍ന്നു: അനന്താവബോധം എണ്ണമറ്റ  ലോകങ്ങളെ കാലത്തിന്റെ മൂന്നു ഘട്ടങ്ങളിലും അനുഭവിക്കുന്നു. അതെല്ലാത്തിനെയും വലയം ചെയ്യുന്നു, എല്ലാറ്റിനെയും കാണുന്നു. സ്വയം മാറ്റങ്ങള്‍ക്കോ അസന്തുലിതാവസ്ഥകള്‍ക്കോ വശംവദമല്ലാത്തതുകൊണ്ട് എല്ലാക്കാലവും നിലനില്‍ക്കുന്നത് അതൊന്നേയുള്ളു. ഈ ബോധത്തിന് ഒരേസമയം മാധുര്യമേറിയതും കയ്പ്പുനിറഞ്ഞതുമായ അനുഭവങ്ങളെ സംവദിക്കുവാനാകും. അത് സദാ പ്രശാന്തതയില്‍ അഭിരമിച്ചു നിശ്ചലമായി നിലകൊള്ളുന്നു. വിവിധങ്ങളായ അവസ്ഥാവിശേഷങ്ങളെ നേരിടുമ്പോഴും അനന്തതയ്ക്ക് മാറ്റമൊന്നുമില്ല. കാരണം, അതെല്ലാ ഉപാധികള്‍ക്കും മാറ്റത്തിനും അതീതമാണ്. ധാരണകളും വിവക്ഷകളും ഇല്ലാത്ത ഒരു തലമാണത്. മാത്രമല്ല, ഒരേസമയം അതു സൂക്ഷ്മവും നാനാവിധ അനുഭവങ്ങളെ ഒരേസമയം വേദിക്കാന്‍ കഴിവുള്ളതുമാണ്. എന്നുമെല്ലായ്പ്പോഴും പ്രശാന്തവും ഘനസാന്ദ്രവുമായ സത്താണത്.
  
“ഒരിക്കലും മാറ്റങ്ങള്‍ക്കു വിധേയമായിട്ടില്ലാത്ത പൊരുളിനെ പരിണാമപ്രക്രിയയിലൂടെ കടന്നുവന്ന ഒരു സത്ത സമാശ്രയിക്കുമ്പോള്‍ അതു ദു:ഖനിവൃത്തി നേടുന്നു. അചഞ്ചലമായ ആ പൊരുള്‍ (സാക്ഷീഭാവത്തില്‍, (മനസ്സ്) അതിനെ ‘ദര്‍ശിക്കു’മ്പോഴാകട്ടെ അതിലുണ്ടായിരുന്ന ദുഷ്ടതയെല്ലാം ഇല്ലാതെയാവുന്നു.” 

അനന്താവബോധം ത്രികാല സംബന്ധിയായ ധാരണകളെ ഉപേക്ഷിക്കുമ്പോള്‍ അത് വിഷയസംബന്ധിയായ എല്ലാ കെട്ടുപാടുകളില്‍ നിന്നും മുക്തി നേടുന്നു. അപ്പോള്‍ ധാരണാവികല്‍പ്പങ്ങളൊഴിഞ്ഞ് പ്രശാന്തത കളിയാടുന്നു. അതിന് വ്യതിരിക്തമായ 'ഉണ്മ' ഇല്ലെന്നു തന്നെപറയാം, കാരണം അത് വിവരണാതീതമാണല്ലോ. ചിലരതുകൊണ്ട് ആത്മാവ് എന്നൊരു സത്ത ഇല്ല എന്നൊരു സിദ്ധാന്തം പ്രചരിപ്പിക്കുന്നു. ബ്രഹ്മം അല്ലെങ്കില്‍ ആത്മാവ് ഉണ്ടോ ഇല്ലയോ എന്ന വാദമെങ്ങിനെയായാലും  മാറ്റമില്ലാത്ത, നാശമില്ലാത്ത അവസ്ഥ തന്നെയാണ് പരമമുക്തി.

ചിന്തകളാകുന്ന ചഞ്ചലത കൊണ്ട് ബോധം മറഞ്ഞിരിക്കുന്നതായി തോന്നുന്നു. അതിനെ നമുക്ക്‌ സാക്ഷാത്കരിക്കാന്‍ കഴിയാതെ വരുന്നു. ആസക്തികളിലും വിരക്തികളിലും ആമഗ്നരായവര്‍ക്ക് അതപ്രാപ്യം തന്നെ. അവര്‍ ചിന്താവലയത്തില്‍ പെട്ടുഴറുന്നു. എന്റെ പൂര്‍വ്വികര്‍ അങ്ങിനെയായിരുന്നു. രാഗദ്വേഷങ്ങളുടെ പിടിയില്‍പ്പെട്ട്, ദ്വന്ദഭാവങ്ങളില്‍ അഭിരമിച്ച് അവര്‍ പുഴുക്കളെപ്പോലെ ജീവിതം നയിച്ചു. ആശാപിശാചുക്കളും ദുഷ്ടതയും ഒടുങ്ങി, അജ്ഞാനചിന്തകളുടെ മായാമറ നീങ്ങി, മനോവൈകല്യങ്ങള്‍ അവസാനിച്ച്, ലഭിക്കുന്ന ശരിയായ ഉള്ളുണര്‍വ്വ് ആര്‍ക്കുണ്ടോ അവന്‍ മാത്രമേ ജീവിക്കുന്നുള്ളു. അനന്താവബോധമെന്ന ഒരേയൊരു സത്തമാത്രം ഉള്ളപ്പോള്‍ അവിടെയെങ്ങിനെ മറ്റൊരു ധാരണ ഉടലെടുക്കും?

ഞാന്‍ ആത്മാവിനെ നമസ്കരിക്കുന്നു. ഞാന്‍ എന്നെ നമസ്കരിക്കുന്നു. അവിച്ഛിഹ്നമായ അനന്താവബോധം, കാണപ്പെടുന്നതും അല്ലാത്തതുമായ ലോകങ്ങളുടെ മകുടമണി തന്നെയാണ്. നിന്നെ പ്രാപിക്കുക ക്ഷിപ്രസാദ്ധ്യം. നിന്നെ ലഭിച്ചു കഴിഞ്ഞിരിക്കുന്നു. നിന്നെ തൊട്ടറിഞ്ഞിരിക്കുന്നു. നിന്നെ ഞാന്‍ സാക്ഷാത്കരിച്ചിരിക്കുന്നു. എല്ലാ വികലതകള്‍ക്കും അതീതമായി നിന്നെ ഞാന്‍ ഉയര്‍ത്തിയിരിക്കുന്നു. നീ നീയാകുന്നു. നീ ഞാനാകുന്നു. ഞാന്‍ നീയും. നിനക്കെന്റെ നമോവാകം.

നിനക്കും എനിക്കും ശിവനും ദേവദേവനും പരംപൊരുളിനും നമസ്കാരം. സ്വരൂപത്തില്‍ അഭിരമിക്കുന്ന ആത്മാവിനെന്റെ നമോവാകം. സ്വയം ആത്മാവില്‍ ആത്മാവ് സ്ഥാപിതമാകയാല്‍  അജ്ഞാനത്തിന്‍റെ മൂടുപടമോ ചിന്താ ധാരണകളാകുന്ന അവിദ്യയോ ആത്മാവിനു ബന്ധനമാകുന്നില്ല.
