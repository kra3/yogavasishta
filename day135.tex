 
\section{ദിവസം 135}

\slokam{
നാ  ഹം ബ്രഹ്മേതി സങ്കല്‍പാത്സുദൃഢാത്ബദ്ധ്യതേ മന:\\
സര്‍വ്വം ബ്രഹ്മേതി സങ്കല്‍പാത്‌ സുദൃഢാന്മുച്യതേ മന: (3/114/23)\\
}

രാമന്‍ ചോദിച്ചു: മഹാത്മന്‍, ഈ ഭയാനകമായ അജ്ഞാനത്തിന്റെ ഇരുട്ട്‌ എങ്ങിനെയാണ്‌ ഇല്ലാതെയാവുക?

വസിഷ്ഠന്‍ പറഞ്ഞു: രാമ: നമ്മുടെ നോട്ടം വിളക്കിലേയ്ക്കു തിരിഞ്ഞാല്‍പ്പിന്നെ നമ്മെസംബന്ധിച്ചിടത്തോളം ഇരുട്ട്‌ ഇല്ലാതെയായി. അതുപോലെ ആത്മാവിലേക്ക്‌ ശ്രദ്ധതിരിച്ചാല്‍ അജ്ഞാനത്തിനും അവസാനമായി. അത്മജ്ഞാനമാര്‍ജ്ജിക്കാന്‍ സ്വാഭാവികമായ ഒരുള്‍വിളി ഉണ്ടാവുന്നതുവരെ ഈ അവിദ്യ, അല്ലെങ്കില്‍ മാനസീകോപാധികള്‍ എണ്ണമറ്റ ലോകങ്ങളെ പ്രകടമാക്കിക്കൊണ്ടേയിരിക്കും. പ്രകാശമെന്തെന്നറിയാന്‍ ഒരു നിഴല്‍ ആഗ്രഹിച്ചാല്‍ വെളിച്ചത്തിന്റെ അറിവില്‍ നിഴല്‍ ഇല്ലാതെയാവുന്നു. അതുപോലെയാണ്‌ ആത്മാന്വേഷണത്തിലൂടെ അജ്ഞാനത്തിന്റെ നാശമുണ്ടാവുന്നത്‌.. രാമ: ആശയാണ്‌ അജ്ഞാനം. ആശകളൊടുങ്ങലാണ്‌ മുക്തി. മനസ്സില്‍ ചിന്തകളില്ലാത്തപ്പോഴാണിതു സാദ്ധ്യമാവുക.

രാമന്‍ ചോദിച്ചു: മഹര്‍ഷേ, അങ്ങുപറഞ്ഞല്ലോ ആത്മജ്ഞാനം ഉണ്ടാവുമ്പോള്‍ അവിദ്യ ഇല്ലാതാവുന്നു, എന്ന്. എന്താണീ ആത്മാവ്‌?

വസിഷ്ഠന്‍ പറഞ്ഞു: രാമ: ബ്രഹ്മാവു മുതല്‍ പുല്‍ക്കൊടിവരെ എല്ലാം ആത്മാവുതന്നെ. അവിദ്യയെന്നത്‌ നിലനില്‍പ്പില്ലാത്ത ഒരു അയഥാര്‍ഥ്യമാണ്‌ (അസത്ത്‌). ))  മനസ്സ്‌ എന്നുപറയുന്നത്‌ രണ്ടാമതൊരു വസ്തുവല്ല. അത്മാവിലാണ്‌ ഈ മൂടുപടത്തിന്റെ (അവിദ്യയുടെ) പ്രഭാവത്താല്‍ വിഷയം-വിഷയി എന്നീ ഭാവങ്ങളുണ്ടാവുന്നതും അവയെ അനന്താവബോധത്തില്‍നിന്നു വേറിട്ടു കാണാനിടയാവുന്നതും. അതാണ്‌ മനസ്സെന്നറിയപെടുന്നത്‌.. ഈ മൂടുപടം പോലും അത്മാവുതന്നെ.

ഈ മൂടുപടമെന്നത്‌ അനന്താവബോധത്തില്‍ ഒരു ചിന്ത, അല്ലെങ്കില്‍ ആശയം, ഉദ്ദേശം, അങ്കുരിക്കുന്നതാണ്‌.. മനസ്സ്‌ ഈ ചിന്തയുടെ സന്തതിയാണ്‌.. ഈ മനസ്സിനെ ഇല്ലാതാക്കാനും ചിന്തകളുടെ സഹായം വേണം. അതായത്‌ ചിന്തകളുടെ, ആശയങ്ങളുടെ, അവസാനമെത്താനും ചിന്തകള്‍ വേണം.  'ഞാന്‍ പരബ്രഹ്മമല്ല' എന്നൊരു സുദൃഢമായ വിശ്വാസം (ചിന്ത) മനസ്സിനെ ബന്ധിക്കുന്നു. മനസ്സിനെ സ്വതന്ത്രമാക്കാന്‍ 'എല്ലാം പരബ്രഹ്മം മാത്രം' എന്നൊരു സുദൃഢവിശ്വാസത്തിനു സാധിക്കും." ആശയങ്ങളും ചിന്തകളും ബന്ധനമാണ്‌.. അവയുടെ അവസാനം മോചനവും. അതുകൊണ്ട്‌ അവയില്‍ നിന്നു സ്വതന്ത്രനാവുക. എന്നിട്ട്‌ സ്വേഛയാ സഹജമായ കര്‍മ്മങ്ങള്‍ ചെയ്യുക.

ചിന്ത, അല്ലെങ്കില്‍ ഒരാശയമാണ്‌ ആകാശത്തിന്റെ നീലനിറത്തിനു നിദാനം. അതുപോലെ മനസ്സ്‌ ലോകത്തെ സത്തായി കാണുന്നു. ആകാശത്തിന്‌ സത്യത്തില്‍ നീലനിറമില്ല. പ്രകാശതരംഗംങ്ങളെ ഒരു പരിധിക്കപ്പുറം സ്പഷ്ടമായി കാണാന്‍ കണ്ണുകള്‍ അപര്യാപ്തമായതിനാലാണ്‌ നീലനിറം കാണപ്പെടുന്നത്‌. അതുപോലെ നമ്മുടെ ചിന്തകളുടെ പരിമിതിയാണ്‌ ലോകമെന്ന ഈ 'കാഴ്ച്ച'. രാമ: ഈ ലോകമെന്നത്‌ വെറും ഭ്രമമാണ്‌.. അതിനെക്കുറിച്ചുള്ള ചിന്തപോലും മനസ്സില്‍ ഉയരാതിരിക്കട്ടെ. 'ഞാന്‍ മതിമയങ്ങിയിരിക്കുന്നു' എന്ന ചിന്ത ദു:ഖത്തിനു കാരണമാവുന്നു. 'ഞാന്‍ ഉണര്‍ന്നിരിക്കുന്നു' എന്ന ചിന്ത ആനന്ദത്തിനും കാരണമാവുന്നു.

