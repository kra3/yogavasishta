\section{ദിവസം 255}

\slokam{
ദേഹസംസ്ഥോഽപ്യദേഹത്ത്വാദദേഹോഽസി വിദേഹദൃക്  \\
വ്യോമസംസ്ഥോഽപ്യ സക്തത്വാദവ്യോമേവ ഹി മാരുത: (5/40/4)\\
}

ഭഗവാന്‍ തുടര്‍ന്നു: ദേഹം നിലനിന്ന്‍  അതിന്റെ പ്രവര്‍ത്തനം നടക്കുമ്പോള്‍ അതിനു ജീവിതം എന്ന് പറയുന്നു. ഈ ദേഹത്തെ മറ്റൊരു ദേഹം ലഭിക്കുവാനായി ഉപേക്ഷിക്കുന്നതിനു മരണം എന്നും പറയുന്നു. പ്രഹ്ലാദാ നീയീ രണ്ടവസ്ഥകളില്‍നിന്നും സ്വതന്ത്രനാണെന്നറിയുക. നിനക്ക് ജീവിതവും മരണവും തമ്മില്‍ അന്തരമേതുമില്ല. പൊതുവേയുള്ള ചില ധാരണകള്‍ ഞാന്‍ പറഞ്ഞുവെന്നേയുള്ളു. വാസ്തവത്തില്‍ ആരും ജീവിക്കുന്നും മരിക്കുന്നുമില്ല. “നീയീ ശരീരത്തില്‍ നിലകൊള്ളുമ്പോഴും നിനക്കൊരു ദേഹബോധമില്ലാത്തതിനാല്‍ നീ വിദേഹനാണ്. സാക്ഷിയായ നീ വസ്തുനിഷ്ഠമല്ലാത്ത ബോധമാണ്. വായു ആകാശത്തു നിലകൊള്ളുമ്പോഴും അത് ആകാശത്തോടൊട്ടി നില്‍ക്കാത്തതുപോലെയും, ആകാശത്താല്‍ പരിമിതപ്പെടാത്തതു പോലെയുമായാണാ ബോധം ശരീരത്തില്‍ നിലകൊള്ളുന്നത്.”      

എന്നാല്‍ പറഞ്ഞുവരുമ്പോള്‍ നീ ദേഹവുമാണ്. കാരണം ഇന്ദ്രിയങ്ങള്‍ സംവദിക്കുന്നത് നീയറിയുന്നുണ്ടല്ലോ. ആകാശം ചെടികള്‍ക്ക് വൃക്ഷമായി വളരാന്‍ ഇടം നല്‍കുന്നു. എന്നാല്‍ ആകാശമതിന്റെ വളര്‍ച്ചയെ നിജപ്പെടുത്തുന്നില്ലല്ലോ. നീ പ്രബുദ്ധനത്രേ. നിന്നില്‍ ദേഹം അല്ലെങ്കില്‍ മൂര്‍ത്തീഭാവമെന്നത് എന്താണ്? നിനക്കൊരു രൂപമുണ്ടെന്നു തോന്നുന്നത് അജ്ഞാനികളുടെ കണ്ണില്‍ മാത്രമാണ്. നീയെന്നും എപ്പോഴും എല്ലാമെല്ലാമായ ആ  പരമബോധപ്രകാശമാണ്. നിന്നെ സംബന്ധിച്ചിടത്തോളം ദേഹമോ ദേഹമില്ലായ്മയോ എന്താണ്? നിനക്ക് തള്ളാനും കൊള്ളാനും എന്താണുള്ളത്? വസന്തകാലമാണങ്കിലും  പ്രളയദിനമാണെങ്കിലും ജീവസംബന്ധിയായ എല്ലാ ആശയധാരണകള്‍ക്കും അതീതമായി വര്‍ത്തിക്കുന്നവനെന്ത് വ്യത്യാസം? എല്ലാ അവസ്ഥകളിലും അവന്‍ ആത്മജ്ഞാനത്തില്‍ ഉറച്ചിരിക്കുന്നു. ഈ പ്രപഞ്ചത്തിലെ എല്ലാ ജീവജാലങ്ങള്‍ക്കും നാശമോ ഉയര്‍ച്ചയോ ഉണ്ടായിക്കൊള്ളട്ടെ, അവന്‍ ആത്മനിഷ്ഠനാണ്.

മാറ്റങ്ങള്‍ക്കു വിധേയമാകാതെയും മരണത്തിനു തൊടാന്‍ കഴിയാതെയും ഈ  ദേഹത്ത് പരമപുരുഷന്‍ കുടികൊള്ളുന്നു. ‘ഞാനീ ശരീരത്തിന്റെതാണെന്നോ, ഈ ശരീരം എന്റെതാണ്’ എന്നോ ഉള്ള ധാരണകളില്ലാതെയായാല്‍പ്പിന്നെ, ‘ഞാനിതുപേക്ഷിക്കും’, ‘ഞാനിതുപേക്ഷിക്കില്ല’; ‘ഞാനിത് ചെയ്തു’, ‘ഇനി ഞാനിത് ചെയ്യും’ തുടങ്ങിയ പ്രസ്താവനകള്‍ക്ക് യാതൊരര്‍ത്ഥവുമില്ല. പ്രബുദ്ധരായവര്‍ സദാ കര്‍മ്മനിരതരായിരിക്കുമ്പോഴും ഒന്നും ‘ചെയ്യുന്നില്ല’. കര്‍മ്മം ചെയ്യാതെയല്ല അവരീ കര്‍മ്മരഹിതമായ അവസ്ഥയെ പ്രാപിക്കുന്നത്. ഈ ‘കര്‍മ്മരാഹിത്യം നിന്നെ അനുഭവങ്ങളില്‍ നിന്നും സ്വതന്ത്രനാക്കും. വിത്തിടാതെയിരുന്നാല്‍  വിളവെടുക്കേണ്ടതായി വരികയില്ലല്ലോ. 
     
അങ്ങിനെ ‘ഞാന്‍ ചെയ്യുന്നു’, ‘ഞാന്‍ അനുഭവിക്കുന്നു’ തുടങ്ങിയ ധാരണകള്‍ അവസാനിച്ചുകഴിഞ്ഞാല്‍ പ്രശാന്തിയായി, മുക്തിയായി. അങ്ങിനെയുള്ളയാള്‍ക്ക് എന്ത് സംന്യസിക്കാന്‍? എന്ത് നേടാന്‍? വിഷയ-വിഷയീ ഭാവങ്ങളില്ലാതെയായാല്‍ മാത്രമേ മുക്തിയുള്ളു. അങ്ങിനെ നിത്യമുക്തരായ നിന്നെപ്പോലുള്ളവര്‍ സദാ സുഷുപ്തിയിലെന്നവണ്ണം ഇഹലോകത്ത് വാഴുന്നു. പ്രഹ്ലാദാ, പകുതി ഉറക്കത്തിലെന്നപോലെ ഈ ലോകത്തെ കാണൂ. പ്രബുദ്ധര്‍ സന്തോഷത്തില്‍ അമിതമായി ആഹ്ലാദിക്കയോ ദു:ഖത്തില്‍ അമിതമായി വിലപിക്കുകയോ ചെയ്യുകയില്ല. അടുത്തുള്ള വസ്തുക്കളെ പ്രതിഫലിപ്പിക്കുന്ന സ്ഫടികം പ്രത്യേക ശ്രമമൊന്നും കൂടാതെ അനിച്ഛാപൂര്‍വ്വമാണല്ലോ പ്രവര്‍ത്തിക്കുന്നത്.

അവര്‍ ജ്ഞാനത്തില്‍ പൂര്‍ണ്ണമായും ഉണര്‍ന്നവരാണ്. എന്നാലവര്‍ ലോകദൃഷ്ടിയില്‍ ഉറക്കത്തിലെന്നവണ്ണം കഴിയുന്നു. അഹംകാരത്തെയും അതിന്റെ പരിവാരങ്ങളെയുംകൊണ്ട് നടക്കാത്ത അവര്‍ ശിശുക്കളെപ്പോലെ സ്വതന്ത്രരായി വര്‍ത്തിക്കുന്നു. പ്രഹ്ലാദാ നീ വിഷ്ണുപദം പ്രാപിച്ചിരിക്കുന്നു. ബ്രഹ്മാവിന്റെ ഒരുദിവസം മാത്രമായ ഒരുലോകചക്രത്തോളം നീ പാതാളലോകം വാണാലും. 