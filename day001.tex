\section{ദിവസം 001}

ഋഷിവര്യനായ അഗസ്ത്യനോട്‌ സുതീക്ഷ്ണമുനി ചോദിച്ചു: ``മാമുനേ
മുക്തിലാഭത്തിനായി ഏറ്റവും ശ്രേഷ്ഠമായത്‌ കര്‍മ്മമാര്‍ഗ്ഗമാണോ അതോ
ജ്ഞാനമാര്‍ഗ്ഗമോ? ദയവായി പറഞ്ഞു തന്നാലും''.

അഗസ്ത്യമുനി മറുപടി അരുളി: ``പക്ഷികള്‍ക്ക്‌ പറക്കാന്‍ രണ്ടു ചിറകുകള്‍
ആവശ്യമുള്ളതുപോലെ കര്‍മ്മവും ജ്ഞാനവും സമ്യക്കായി വര്‍ത്തിക്കുമ്പോഴാണ്‌
പരമ ലക്ഷ്യമായ മോക്ഷം സാദ്ധ്യമാവുക. കര്‍മ്മം മാത്രമായോ ജ്ഞാനം മാത്രമായോ
മോക്ഷം സാദ്ധ്യമല്ല തന്നെ. രണ്ടും വേണ്ട പോലെചേര്‍ന്നാല്‍ മാത്രമേ നാം
മോക്ഷത്തിലേയ്ക്കു നയിക്കപ്പെടൂ''.~

അങ്ങയുടെ ചോദ്യത്തിനുത്തരമായി ഞാന്‍ ഒരു കഥ പറയാം. ഒരിടത്ത്‌ അഗ്നിവേശന്റെ
പുത്രനായി കാരുണ്യ എന്നൊരുവന്‍ ഉണ്ടായിരുന്നു. ഈ ചെറുപ്പക്കാരന്‌
ശാസ്ത്രങ്ങള്‍ എല്ലാം പഠിച്ചു സാരം ഹൃദിസ്ഥമാക്കിയതിന്റെ പരിണിതഫലമായി
ജീവിതത്തോട്‌ ഒരു തരം ഉദാസീനത ഉണ്ടാവാനിടയായി. കാരുണ്യയോട്‌ എന്തുകൊണ്ടാണ്‌
തന്റെ നിയതകര്‍മ്മങ്ങളില്‍ വീഴ്ച വരുത്തുന്നതെന്ന് അഗ്നിവേശന്‍ ചോദിക്കവേ
അയാള്‍ പറഞ്ഞു: ``ചില വേദശാസ്ത്രങ്ങള്‍ പ്രകാരം ഒരുവന്‍ തന്റെ
കര്‍മ്മങ്ങള്‍ ശാസ്ത്രോക്തമായി കൃത്യമായിത്തന്നെ മരണം വരെയും
ചെയ്തുകൊണ്ടേയിക്കണമെന്ന് പറയുന്നുണ്ടല്ലോ. മറ്റു ചില ശാസ്ത്രങ്ങളില്‍
കര്‍മ്മത്യാഗംകൊണ്ടു മാത്രമേ അമര്‍ത്ത്യത കൈവരുകയുള്ളു എന്നാണു പറയുന്നത്‌.
ഈ രണ്ടു മതങ്ങളില്‍ ഏതാണ്‌ എനിക്ക്‌ ശുഭകരം? ഞാന്‍ എന്തുചെയ്യണമെന്ന്
പറഞ്ഞു തന്നാലും. അങ്ങ്‌ എന്റെ പിതാവും ഗുരുവുമാണല്ലോ''.

അഗ്നിവേശന്‍ പറഞ്ഞു: ഞാന്‍ നിനക്കൊരു കഥ പറഞ്ഞുതരാം. അതു കേട്ടിട്ട്‌ അതിലെ
ഗുണാഗുണങ്ങള്‍ സ്വയം വിശകലനം ചെയ്ത്‌ നിനക്കിഷ്ടം പോലെ ഒരു തീരുമാനത്തില്‍
എത്താം. ഒരിക്കല്‍ ദേവസുന്ദരിയായ സുരുചി ഹിമാലയ പര്‍വ്വതശിഖരങ്ങളിലൊന്നില്‍
ഇരിക്കുമ്പോള്‍ ഇന്ദ്രന്റെ സന്ദേശവാഹകരിലൊരാള്‍ ആകാശമാര്‍ഗ്ഗേ പോകുന്നതു
കണ്ടു. എന്താണ്‌ അയാളുടെ യാത്രോദ്ദേശം എന്ന് സുരുചി ചോദിച്ചതിനു മറുപടിയായി
അയാള്‍ പറഞ്ഞു: ഗന്ധമാദന പര്‍വ്വതത്തില്‍ ഒരിടത്ത്‌ അരിഷ്ടനേമി എന്നൊരാള്‍
തന്റെ രാജ്യമെല്ലാം പുതനു നല്‍കിയിട്ടു വന്ന് തീവ്രമായ
തപസ്സിലേര്‍പ്പെട്ടിരിക്കുന്നു. ഇന്ദ്രന്റെ നിര്‍ദ്ദേശപ്രകാരം ഞാന്‍
അപ്സരസ്ത്രീകളേയും കൂട്ടി ആ രാജര്‍ഷിയെ സ്വര്‍ഗ്ഗത്തിലേയ്ക്ക്‌
കൊണ്ടുപോവാനായി അവിടെ ചെന്നു. സ്വര്‍ഗ്ഗത്തിന്റെ ഗുണദോഷങ്ങളെപ്പറ്റി
രാജര്‍ഷി എന്നോട്‌ ചോദിച്ചറിഞ്ഞു. ``സ്വര്‍ഗ്ഗത്തില്‍ ഉത്തമജീവിതം
നയിച്ചവരേയും മധ്യമാര്‍ഗ്ഗികളേയും അധമജീവിതം നയിച്ചവരേയും അവരവരുടെ
കര്‍മ്മഗുണമനുസരിച്ചുള്ള അനുഭവങ്ങള്‍ക്കു വിധേയരാക്കിയ ശേഷം അതത്‌
കര്‍മ്മനുസാരിയായ ഫലങ്ങള്‍ അനുഭവിച്ചു തീരുമ്പോള്‍ അവരെ
മര്‍ത്ത്യലോകത്തിലേയ്ക്കു തിരികെ അയക്കുകയാണു ചെയ്യുന്നത്‌''. ഇതുകേട്ട്‌
സ്വര്‍ഗ്ഗവാസത്തിനായുള്ള ഇന്ദ്രന്റെ ക്ഷണം രാജര്‍ഷിയായ അരിഷ്ടനേമി
സ്വീകരിച്ചില്ല. ഇന്ദ്രന്‍ ഒരിക്കല്‍ കൂടി ഋഷിക്കൊരു സന്ദേശവുമായി എന്നെ
അയച്ചു. ക്ഷണം പൂര്‍ണ്ണമായി നിരസിക്കും മുന്‍പ്‌ മഹര്‍ഷി വാല്‍മീകിയോട്‌
പര്യാലോചിക്കുന്നത്‌ ഉചിതമായിരിക്കും എന്ന് അദ്ദേഹത്തെ അറിയിക്കാനാണ്‌
ഞാന്‍ പോയത്‌. രാജര്‍ഷി വാല്‍മീകിയോട്‌ ചോദിച്ചു: ``ജനന മരണങ്ങളില്‍ നിന്നു
മുക്തി നേടാന്‍ ഏറ്റവും നല്ല മാര്‍ഗ്ഗമെന്താണ്‌?'' ഇതിനുത്തരമായി
ശ്രീരാമനും ~വ്സിഷ്ഠമുനിയുമായി ~ ഉണ്ടായ സംഭാഷണം~വാല്‍മീകി~വിവരിച്ചു
പറഞ്ഞു കൊടുത്തു.

