\section{ദിവസം 153}

\slokam{
സ്വയ വാസനയാ ലോകോ യദ്യത്കർമ്മ കരോതി യ:\\
സ തഥൈവ തദാപ്നോതി നേതരസ്യേഹ കര്‍തൃതാ (4/13/11)\\
}

കാലം (യമരാജന്‍) തുടർന്നു: അല്ലയോ മഹർഷേ, ദേവന്മാരും അസുരന്മാരും മനുഷ്യരും എല്ലാം ബ്രഹ്മം എന്നറിയപ്പെടുന്ന അനന്താവബോധത്തിൽ നിന്നും വേറിട്ട ഒന്നല്ല. ഇതു മാത്രമാണു സത്യം. മറ്റെല്ലാ അവകാശവാദങ്ങളും തെറ്റാണ്‌...  ദേവതകളും മറ്റും ‘ഞാൻ പരബ്രഹ്മമല്ല’ എന്ന  തെറ്റിദ്ധാരണയ്ക്കു വശംവദരായി സ്വയം അശുദ്ധിയാരോപിച്ച് താഴോട്ടുപതിക്കുന്നതായി സങ്കൽപ്പിക്കുന്നു. അവരും അനന്താവബോധത്തിൽത്തന്നെയാണ്‌.... എങ്കിലും സ്വയം അതില്‍ നിന്നും വിഭിന്നരാണെന്ന മതിഭ്രമത്തിലാണവർ എന്നുമാത്രം.

നിത്യശുദ്ധരായവർ സ്വയമേവ തങ്ങളിൽത്തന്നെ അശുദ്ധി ആരോപിക്കുന്നതാണ്‌ അവരുടെ കർമ്മങ്ങൾക്കും അവയുടെ പരിണിതഫലങ്ങളായ സന്തോഷം, സന്താപം, അജ്ഞാനം, പ്രബുദ്ധത എന്നിവയ്ക്കെല്ലാം ബീജമാവുന്നത്. ഈ ജീവികളിൽ ചിലർ ശിവനെപ്പോലെയും വിഷ്ണുവിനെപ്പോലെയും ശുദ്ധരത്രേ. മനുഷ്യരേയും ദേവന്മാരെപ്പോലെയും അല്പം കളങ്കപ്പെട്ടവരാണ്  മറ്റൊരുകൂട്ടർ. മരങ്ങളും ചെടികളും കട്ടിപിടിച്ച മോഹത്താൽ ബന്ധിതരാണ്‌.. മറ്റുചിലർ പുഴുക്കളെപ്പോലെ അജ്ഞാനബന്ധനത്തിലാണ്‌...  ചിലർ ജ്ഞാനത്തിൽ നിന്നും വളരെ അകലെയാണ്‌...   കുറച്ചുപേർ ബ്രഹ്മാവിഷ്ണു മഹേശ്വരന്മാരെപ്പോലെ പ്രബുദ്ധരും മുക്തിയാർജ്ജിച്ചവരുമാണ്‌..

ഇങ്ങിനെ അജ്ഞതയുടേയും മോഹവിഭ്രമങ്ങളുടേയും ചക്രത്തിൽ ചുറ്റുന്നതിനിടയ്ക്ക് പരമസത്യത്തിന്റെ, ജ്ഞാനത്തിന്റെ പടിയിൽ ഒന്നു ചവിട്ടാനിടയായാൽ അവൻ ക്ഷണത്തിൽ മുക്തനായി. ഇവരിൽ മരങ്ങളെപ്പോലെ വേരുറച്ച അജ്ഞതയ്ക്കടിമയാവാത്തവർ, എന്നാൽ മോഹവലയത്തിൽനിന്നുമിനിയും പുറത്തു വരാത്തവർ, അന്വേഷണപാതയിൽ വേദഗ്രന്ഥങ്ങളെയും സദ്ഗുരുക്കളേയും ആശ്രയിക്കേണ്ടതാണ്‌..  ഈ വേദഗ്രന്ഥങ്ങൾ രചിച്ചിട്ടുള്ളത് പ്രബുദ്ധരാണ്‌..  നികൃഷ്ട ജീവിതമുപേക്ഷിച്ച് മലിനചോദനകൾ ഇല്ലാതെയായവരും അവിദ്യയുടെ നിദ്രയിൽനിന്നുമുണർന്നവരുമായ സാധകർക്കു മാർഗ്ഗനിർദ്ദേശം നൽകുവാനാണ്‌ ഈ മഹദ് ഗ്രന്ഥങ്ങൾ. ഇത്തരം സാധകർക്ക് അവയിൽ സഹജമായ താൽപ്പര്യം ഉണ്ടാവും.

മഹർഷേ, മനസ്സാണ്‌ സുഖദു:ഖങ്ങൾ അനുഭവിക്കുന്നത്. ശരീരമല്ല. ശരീരമെന്നത് മനസ്സിന്റെ സങ്കൽപ്പങ്ങളുടെ ഫലമാണ്‌..  മനസ്സുമായി വേറിട്ട് ശരീരത്തിന്‌ അസ്തിത്വമില്ല. അങ്ങയുടെ പുത്രൻ മനസ്സിൽ സങ്കൽപ്പിച്ചത് അനുഭവിച്ചു. ഞങ്ങൾക്കതിൽ യാതൊരുത്തരവാദിത്തവുമില്ല. എല്ലാ ജീവികൾക്കും അവരവരുടെ ഉള്ളിലങ്കുരിക്കുന്ന സാദ്ധ്യതകൾക്കും കഴിവുകൾക്കുമനുസരിച്ചുള്ള കർമ്മങ്ങളാണ്‌ സ്വായത്തമാക്കാൻ കഴിയുക. മറ്റ് ആർക്കുമതിൽ പങ്കില്ല. അതിമാനുഷർക്കും, ദൈവങ്ങൾക്കും അതിന്റെ ഗതിമാറ്റാൻ ആവില്ല. വരൂ, നമുക്ക് അങ്ങയുടെ മകൻ തപസ്സുചെയ്യുന്നിടത്തേയ്ക്ക് പോകാം. സ്വർഗ്ഗത്തിലെ നൈമിഷികസുഖാനുഭവത്തിനുശേഷമാണദ്ദേഹം അവിടെയെത്തിയിരിക്കുന്നത്.

യമൻ ഭൃഗുമഹർഷിയെ കൂട്ടിക്കൊണ്ട് യാത്രയായി. വസിഷ്ഠമുനി എട്ടാം ദിവസത്തെ കഥാകഥനം അവസാനിപ്പിച്ചു. സഭ പിരിഞ്ഞു. 

