\newpage
\section{ദിവസം 045}

\slokam{
അദാവേവ ഹി യന്നാസ്തി കാരണാസംഭവാത്സ്വയം\\
വർത്തമാനേപി തന്നാസ്തി നാശ: സ്യാത്തത്ര കീദൃശ: (3/11/13)\\
}

രാമന്‍ പറഞ്ഞു: ഭഗവന്‍ , വിശ്വപ്രളയസമയത്ത്‌ നാമിപ്പോള്‍ കണ്ടുകൊണ്ടിരിക്കുന്ന ലോകം എവിടെപ്പോയി മറയുന്നു?

വസിഷ്ഠന്‍ പറഞ്ഞു: വന്ധ്യയുവതിയുടെ പുത്രന്‍ എവിടെയാണ്‌? അവന്‍ എവിടെപ്പോയി മറഞ്ഞു?. വന്ധ്യയുടെ പുത്രന്‌ അസ്തിത്വമില്ല. ഒരിക്കലും ഉണ്ടായിരുന്നിട്ടില്ല. ഈ സാദൃശ്യം നിന്നെ അമ്പരപ്പിക്കുന്നുവെങ്കില്‍ അത്‌ നീയീ ലോകത്തെ വാസ്തവമായി കണ്ടതുകൊണ്ടാണ്‌. ഉദാഹരണത്തിന്‌ ഒരു സ്വര്‍ണ്ണമാലയില്‍ 'മാലത്വം '  അല്ലെങ്കിൽ 'ആഭരണത്വം' എന്നൊരു സത്യവസ്തു ഉണ്ടോ അതോ അതിലുള്ളത്‌ 'സ്വര്‍ണ്ണം' എന്ന സത്യവസ്തുവാണോ? ശൂന്യതയില്‍ നിന്നു സ്വതന്ത്രമായി ആകാശമുണ്ടോ? അതുപോലെ പരബ്രഹ്മത്തില്‍ നിന്നും വിഭിന്നമായി ലോകം എന്നൊരു 'വസ്തു' ഇല്ല. മഞ്ഞുകട്ടയില്‍നിന്നും തണുപ്പ്‌ വേര്‍തിരിക്കാവുന്നതല്ല, എന്നതുപോലെ ബ്രഹ്മത്തില്‍ നിന്നും വേറെയല്ല ലോകം. മരീചികയിലെ ജലം  ' ഉണ്ടായി '  ഇല്ലാതാവുന്നില്ല. അതുപോലെ ലോകം പരം പൊരുളില്‍ നിന്നുദ്ഭവിച്ച്‌ എങ്ങുംപോയി മറയുന്നില്ല. 

"ഈ ലോകത്തിന്റെ സൃഷ്ടിക്ക്‌ കാരണമൊന്നുമില്ല. അതുകൊണ്ടുതന്നെ ഇതിന്‌ ആദിയില്ല. ഇപ്പോഴും ഇതു നിലനില്‍ക്കുന്നില്ല. പിന്നെങ്ങിനെ ഇതിനു നാശമുണ്ടാവും?". ബ്രഹ്മത്തില്‍നിന്നും 'ഉണ്ടായതല്ല' ലോകം എന്നു നീ സമ്മതിക്കുകയും, ലോകത്തിന്റെ പ്രത്യക്ഷഭാവത്തിനു നിലനില്‍ക്കാനാവുന്നത്‌ ആ ബ്രഹ്മത്തിനെ ഉപാധിയാക്കിയാണെന്നുറപ്പിക്കുകയും ചെയ്യുമ്പോള്‍ ലോകമെന്ന ഒന്നിനു നിലനില്‍പ്പില്ല എന്നും ബ്രഹ്മം മാത്രമേ ഉണ്മയായി ഉള്ളൂ എന്നുമറിയാം.

അതൊരു സ്വപ്നം പോലെയാണ്‌. അജ്ഞാനാവസ്ഥയില്‍ ഒരുവന്റെ  ധിഷണ, സ്വയം പല സ്വപ്നസാമഗ്രികളായി അവന്റെയുള്ളില്‍ പ്രത്യക്ഷപ്പെടുന്നു. അവയെല്ലാം ആ ധിഷണ (ബുദ്ധി) മാത്രമാണല്ലോ. അതുപോലെ സൃഷ്ടിയാരംഭത്തില്‍ ഇത്തരമൊരു 'പ്രത്യക്ഷപ്പെടല്‍ ' സംഭവിച്ചു. എങ്കിലും ഇതു ബ്രഹ്മത്തില്‍ നിന്നും സ്വതന്ത്രമല്ല. ബ്രഹ്മത്തില്‍ നിന്നു വേറിട്ട്‌ അതിനൊരസ്തിത്വമില്ല. അതിനാലത്‌ നിലനില്‍ക്കുന്നില്ല.

രാമന്‍ പറഞ്ഞു: മഹാത്മാവേ അങ്ങിനെയാണെങ്കില്‍ , ഈ ലോകത്തിന്‌ ഇത്രയേറെ തന്മയീഭാവം വന്നതെങ്ങിനെ? 'കാണുന്നവന്‍ ' ഉള്ളിടത്തോളം 'കാണപ്പെടുന്നതുണ്ട്‌'. 'കാണപ്പെടുന്നത്‌' ഉള്ളിടത്തോളം 'കാണുന്നവനുമുണ്ട്‌'. ഇതു രണ്ടും ഇല്ലാതാവുമ്പോള്‍ മാത്രമാണ്‌ മുക്തി. ചെളിപുരളാത്ത ഒരു കണ്ണാടി എപ്പോഴും ഒന്നിനെയല്ലെങ്കില്‍ മറ്റൊന്നിനെ പ്രതിഫലിപ്പിച്ചുകൊണ്ടിരിക്കും. അതുപോലെ സാധകനില്‍ ഈ സൃഷ്ടി പ്രക്രിയ തുടര്‍ ന്നുകൊണ്ടേയിരിക്കും. എന്നാല്‍ 'സൃഷ്ടി' അവാസ്തവീകമാണെന്നു സാക്ഷാത്കരിച്ചു കഴിഞ്ഞാല്‍ പിന്നെ സാധകന്‍ ഇല്ല. പക്ഷേ ഈയൊരു സാക്ഷാത്കാരനിറവിലെത്തുക എളുപ്പമേയല്ല.

വസിഷ്ഠന്‍ പറഞ്ഞു: നിന്റെ സംശയങ്ങള്‍ കഥകളിലൂടെ ഞാന്‍ മാറ്റിത്തരാം. അതുമൂലം നിനക്ക്‌ സൃഷ്ടിയുടെ യാഥാര്‍ത്ഥ്യം സാക്ഷാത്കരിച്ച്‌ പ്രബുദ്ധമായ ഒരു ജീവിതം നയിക്കാം. 

