\section{ദിവസം 181}

\slokam{
ബ്രഹ്മ ചിദ്ബ്രഹ്മ ച മനോ ബ്രഹ്മ വിജ്ഞാനവസ്തു ച\\
ബ്രഹ്മാർഥോ ബ്രഹ്മ ശബ്ദശ്ച ബ്രഹ്മ ചിദ്ബ്രഹ്മ ധാതവ: (4/40/29)\\
}

വസിഷ്ഠൻ തുടർന്നു: രാമ, ഇക്കാണായ ലോകമെന്ന സൃഷ്ടി മുഴുവൻ അനന്താവബോധത്തിലെ (പരബ്രഹ്മത്തിലെ) ചിത്ശക്തിയിലുളവായ ഇച്ഛ ആകസ്മീകമായി പ്രകടിതമായതുമൂലം ഉണ്ടായതാണ്‌..  ഈ ഇച്ഛതന്നെയാണ്‌ സാന്ദ്രതയാർന്ന്, മനസ്സായി ഉദിച്ച്, ഇച്ഛാവസ്തുവിന്‌ അസ്തിത്വം നല്കുന്നത്.  വസ്തുപ്രപഞ്ചത്തിലെന്നപോലെ, മനസ്സതിനെ ക്ഷണത്തിൽ പുനർനിർമ്മിക്കുന്നു. ഈ അവസരത്തിൽ അനന്താവബോധം തന്റെ സഹജഭാവം ഉപേക്ഷിച്ചപോലെ ഒരു പ്രതീതിയുണ്ടാവുന്നുണ്ട്. ഈ അനന്താവബോധം സ്വയം ഒരു ശുദ്ധശൂന്യമായ ഇടത്തെ കണ്ടുപിടിച്ചെന്നപോലെ ചിത്-ശക്തി അവിടെ ആകാശത്തിന്‌ ഒരസ്തിത്വം നല്കുന്നു. അതേ ചിത്-ശക്തിയിൽ സ്വയം പലതാവാനുള്ള ഇച്ഛ ഉണ്ടാവുന്നു. ഈ ഇച്ഛതന്നെയാണ്‌ പിന്നീട് ബ്രഹ്മാവായും പരിവാരസൃഷ്ടികളായുള്ള അനേകം ജീവികളായും കണക്കാക്കപ്പെടുന്നത്. അങ്ങിനെയാണ്‌ അനന്താവബോധത്തിന്റെ ആകാശത്ത് പതിന്നാലു ലോകങ്ങൾ പ്രത്യക്ഷമായത്. അവയിൽ ചിലത് സാന്ദ്രമായ അന്ധകാരത്തിൽ മുങ്ങിയവയാണ്‌..  ചിലത് പ്രബുദ്ധതയിലെത്താൻ വെമ്പിനിൽക്കുന്നു. ഇനിയും ചിലത് പൂർണ്ണമായ പ്രബുദ്ധതയുടെ നിറവായി നിലകൊള്ളുന്നു.

രാമാ, ഈ ലോകത്ത് പല ജീവജാലങ്ങളും ഉണ്ടെങ്കിലും മനുഷ്യനു മാത്രമേ സത്യത്തെ സാക്ഷാത്കരിക്കാനുള്ള യോഗ്യതയുള്ളു. ഈ മനുഷ്യവർഗ്ഗത്തിൽത്തന്നെ പലരും ശോകത്തിനും ഭ്രമത്തിനും, വെറുപ്പിനും ഭയത്തിനും വശംവദരാണ്‌..  ഇവയെപ്പറ്റി ഞാനിനി വിശദമായി പറഞ്ഞുതരാം.

ആരാണീ ലോകം സൃഷ്ടിച്ചതെന്നും എങ്ങിനെയിതു സംഭവിച്ചുവെന്നും എല്ലാമുള്ള വിവരണങ്ങൾ ഗ്രന്ഥരചനയ്ക്കും വ്യാഖ്യാനാഖ്യാനങ്ങൾക്കും വേണ്ടി മാത്രമാണ്‌. ഇതൊന്നും സത്യസംബന്ധിയല്ല. അനന്താവബോധത്തിൽ മാറ്റങ്ങളുണ്ടാവുക, വിക്ഷേപങ്ങൾ സംഭവിക്കുക എന്നതൊന്നും സത്യമല്ല, സാംഭവ്യവുമല്ല. എന്നാൽ അവ സത്യമെന്നപോലെ തോന്നുന്നു. സങ്കൽപ്പത്തിൽപ്പോലും അനന്താവബോധമല്ലാതെ മറ്റൊന്നും എങ്ങുമില്ല. അതിനെ വിശ്വ സൃഷ്ടാവെന്നു വിളിക്കുന്നതും  വിശ്വം സൃഷ്ടിക്കപ്പെട്ടു എന്നു ധരിക്കുന്നതും ശുദ്ധ  അസംബന്ധമാണ്‌..  ഒരു ദീപത്തിൽ നിന്നും മറ്റൊരു ദീപം കൊളുത്തുമ്പോൾ അവ തമ്മിൽ സൃഷ്ടി-സൃഷ്ടാവ് ബന്ധം ഉണ്ടാവുന്നതെങ്ങിനെ? അഗ്നി ഒന്നല്ലേയുള്ളു ?. സൃഷ്ടിയെന്നത് കേവലം ഒരു വാക്കുമാത്രം. അതിനനുസൃതമായി, ഉണ്മയായി യാതൊന്നുമില്ല.

“ബോധം ബ്രഹ്മമാണ്‌; മനസ്സ് ബ്രഹ്മമാണ്‌; ബുദ്ധി ബ്രഹ്മമാണ്‌.; ബ്രഹ്മം മാത്രമേ സത്യമായുള്ളു. ശബ്ദവും വാക്കും ബ്രഹ്മമാണ്‌... സകലതിന്റെയും ഘടകങ്ങളും ബ്രഹ്മമല്ലാതെ മറ്റൊന്നല്ല.” എല്ലാം ബ്രഹ്മം മാത്രം. ലോകമെന്നത് വെറും മിത്ഥ്യ. മാലിന്യം നീങ്ങുമ്പോൾ വസ്തു വെളിപ്പെടുന്നതുപോലെ, ഇരുട്ടിലാണ്ടുകിടന്ന പദാർത്ഥങ്ങൾ ഇരുളകലുന്നതോടെ തെളിഞ്ഞു കാണാനാകുന്നതുപോലെ, അജ്ഞതനീങ്ങിയാൽ സത്യത്തെ സാക്ഷാത്കരിക്കാം. 

