\section{ദിവസം 287}

\slokam{
അന്ത:സക്തം മനോബദ്ധം മുക്തം സക്തിവിവര്‍ജിതം\\
അന്ത:സംസക്തിരേവൈകം കാരണം ബന്ധമോക്ഷയോ:  (5/67/34)\\
}

വസിഷ്ഠന്‍ തുടര്‍ന്നു: അങ്ങിനെ പരസ്പരം സംസാരിച്ചും ലോകകാര്യങ്ങളെപ്പറ്റി വിചാരംചെയ്തും അവര്‍ താമസംവിനാ പരമപദം പൂകി. അതുകൊണ്ട് രാമാ, ആത്മജ്ഞാനമല്ലാതെ ഈ ലൌകീക ബന്ധനത്തിന്റെ കേട്ടറുക്കാനും സംസാരമെന്ന ഈ സാഗരം തരണം ചെയ്യാനും  ഉതകുന്ന മറ്റൊരു മാര്‍ഗ്ഗവും ഇല്ലെന്നറിയുക. പ്രബുദ്ധന്റെ ദൃഷ്ടിയില്‍ അലകടല്‍ ഒരുചെറുവെള്ളക്കുഴി മാത്രമാണ്. ദൂരെയുള്ളൊരാള്‍ക്കൂട്ടത്തെ നോക്കിക്കാണുന്നതുപോലെയാണയാള്‍ സ്വന്തം ദേഹത്തെ നോക്കുന്നത്. അതിനാല്‍ ദേഹത്തിനേല്‍ക്കുന്ന ദു:ഖാനുഭവങ്ങള്‍ അയാളെ ബാധിക്കുന്നില്ല. 

ശരീരം നിലനില്‍ക്കുന്നതുകൊണ്ട് ആത്മാവിന്റെ സര്‍വ്വവ്യാപിത്വത്തിനു കോട്ടമൊന്നുമില്ല. അലകള്‍ ഉള്ളത് കൊണ്ട് സമുദ്രത്തിന്റെ മഹത്വത്തിനെന്തെങ്കിലും കോട്ടമുണ്ടോ? വെള്ളത്തില്‍ ആന്തോളനം ചെയ്യുന്ന അരയന്നത്തിനും, വെള്ളത്തില്‍ കിടക്കുന്ന പാറക്കല്ലിനും അതില്‍ പൊന്തിക്കിടക്കുന്ന തടിക്കഷണത്തിനും വെള്ളവുമായി എന്ത് ബന്ധമാണുള്ളത്?    

അതുപോലെ പരമാത്മാവിന് ഈ പ്രത്യക്ഷപ്രപഞ്ചവുമായി യാതൊരു ബന്ധവുമില്ല. വെള്ളത്തിലേയ്ക്ക് കടപുഴകി വീഴുന്ന മരം വലിയ അലകളുണ്ടാക്കുന്നതുപോലെയാണ് ദേഹത്തിനുണ്ടാകുന്ന സുഖദു:ഖങ്ങള്‍ ആത്മാവിന് വേദ്യമാവുന്നത്. ജലത്തിന്റെ സാമീപ്യം കാരണം തടിക്കഷണം അതില്‍ പ്രതിബിംബിക്കുന്നതുപോലെ ദേഹം ആത്മാവില്‍ പ്രതിഫലിക്കുന്നു. ജലത്തില്‍ നിപതിക്കുന്ന കല്ലുകള്‍ വെള്ളത്തെയോ സ്വയമോ നോവിക്കുന്നില്ല. അതുപോലെ ദേഹം മറ്റു പദാര്‍ത്ഥസഞ്ചയങ്ങളുമായി (ഭാര്യാപുത്രാദികള്‍ , വിഷയ വസ്തുക്കള്‍) സമ്പര്‍ക്കത്തിലേര്‍പ്പെടുന്നത് കൊണ്ട് ആര്‍ക്കും (ഒന്നിനും) ഒരു ദുരിതവും അനുഭവിക്കേണ്ടിവരുന്നില്ല.  

ഒരു കണ്ണാടിയിലെ പ്രതിഫലനത്തെ സത്തെന്നും അസത്തെന്നും പറയാന്‍ വയ്യ. അത് വിവരണാതീതമാണ്. അതുപോലെ ആത്മാവില്‍ പ്രതിഫലിക്കുന്ന ദേഹം വിവരണങ്ങള്‍ക്കു വഴങ്ങാത്ത, സത്യമെന്നും അസത്യമെന്നും പറയാനരുതാത്തതുമായ  ഒരു പ്രതിഭാസമത്രേ. അജ്ഞാനി തന്റെ മുന്നില്‍ കാണുന്ന എല്ലാത്തിനേയും സത്യമെന്ന് കരുതുന്നു, ജ്ഞാനി അങ്ങിനെയല്ല. ജലത്തില്‍ പ്രതിഫലിച്ച മരക്കഷണവും വെള്ളവും തമ്മില്‍ യാതൊരു ബന്ധവുമില്ലാത്തതുപോലെ  ദേഹവും ആത്മാവും തമ്മില്‍ ശരിക്കും ബന്ധുതയൊന്നുമില്ല. വാസ്തവത്തില്‍ അങ്ങിനെയൊരു ബന്ധം നിലനില്‍ക്കണമെങ്കില്‍ ദ്വന്ദത ഉണ്ടാകണമല്ലോ. ഒരേയൊരു അനന്താവബോധം മാത്രമുള്ളപ്പോള്‍ വിഷയ-വിഷയീ ഭിന്നതകള്‍ക്കും  ബന്ധങ്ങള്‍ക്കുമെന്താണ് സാംഗത്യം? 

ഇവിടെ ദ്വന്ദത ആരോപിക്കപ്പെടുകയാണ്. ഒരിക്കലും ദു:ഖങ്ങള്‍ തൊട്ട്തീണ്ടാനിടയില്ലാത്ത ശരീരം സ്വയം വൃഥാ പരിതപിക്കുകയാണ്. പിശാചിനെ കണ്ടെന്നു കരുതുന്നവന്‍ പിശാചിനെ കാണുന്നു. ചിന്തയുടെ ശക്തികൊണ്ട് ഭാവനാസൃഷ്ടി മാത്രമായ ബന്ധുത്വം യാഥാര്‍ത്ഥ്യമാണെന്നപോലെ ആയിത്തീരുന്നു. ആത്മാവിനു സുഖദു:ഖങ്ങള്‍ ഇല്ല. എന്നാല്‍ അത് സ്വയം ശരീരമാണെന്നു കരുതി അനുഭവങ്ങളെ സ്വാംശീകരിക്കുന്നു.

ഈ ഭാവനാ ബന്ധുത്വം ഉപേക്ഷിക്കുന്നതാണ് മുക്തി. വിഷയങ്ങളില്‍ ആസക്തരല്ലാതെയും അവയുമായി താതാത്മ്യം പ്രാപിക്കാതെയും ഇരിക്കുന്നവര്‍ക്ക് ദു:ഖത്തില്‍ നിന്നും എന്നെന്നേയ്ക്കുമായി രക്ഷയായി. ഈ മനോപാധികളാണ് വാര്‍ദ്ധക്യം, മരണം, വിഭ്രാന്തി, എന്നിവയ്ക്കെല്ലാം കാരണം. അതവസാനിക്കുമ്പോള്‍ മോഹസമുദ്രത്തിനപ്പുറം താണ്ടാം. ഉപാധികളോടു കൂടിയ മനസ്സ് മഹാമുനിമാരെപ്പോലും മോഹവലയത്തിലാക്കുന്നു. എന്നാല്‍ ഉപാധികളൊഴിഞ്ഞ മനസ്സുള്ള ഗൃഹസ്ഥന്‍ മോഹത്തിനടിമയാകുന്നില്ല.

“മനസ്സ് ഉപാധികള്‍ക്ക് (ആസക്തി, താതാത്മ്യഭാവം, ഉള്‍പ്രതിപത്തികള്‍) വശംവദമാവുമ്പോള്‍ അത് ബന്ധനം. മുക്തി എന്നത് ഈ ബന്ധനത്തില്‍ നിന്നുമുള്ള മോചനമാണ്. ആന്തരീകമായ പ്രതിപത്തികള്‍ തന്നെയാണ് അവിച്ഛിന്നമായതിനു ഭിന്നത സങ്കല്‍പ്പിച്ചുണ്ടാക്കുന്നത്. അതാണ്‌ ബന്ധനങ്ങളെ ഉണ്ടാക്കുന്നതും മോക്ഷത്തിനു ഹേതുവാകുന്നതും. " ഉപാധികളൊഴിഞ്ഞ മനസ്സോടെ ചെയ്യുന്ന കര്‍മ്മങ്ങള്‍ ‘അകര്‍മ്മങ്ങള’ത്രേ. എന്നാല്‍ ഉപാധികളുള്ള മനസ്സ് കര്‍മ്മരഹിതമായിരിക്കുമ്പോഴും അത് കര്‍മ്മനിരതം തന്നെയാണ്. കര്‍മ്മവും അകര്‍മ്മവും മനസിലാണ്. ശരീരം ഒന്നും ചെയ്യുന്നില്ല. അതിനാല്‍ ദൃഢചിത്തത്തോടെ ആന്തരീകമായി ഉണ്ടാവുന്ന ഈ മിഥ്യാഭിന്നത പൂര്‍ണ്ണമായും ഒഴിവാക്കേണ്ടതാണ്.

