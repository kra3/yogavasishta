\newpage
\section{ദിവസം 050}

\slokam{
ചേത്യസംവേദനാജ്ജീവോ ഭവത്യായാതി സംസൃതിം\\
തദസംവേദനാദ്രൂപം സമായാതി സമം പുന: (3/14/36)\\
}

വസിഷ്ഠന്‍ തുടര്‍ന്നു: "അറിയപ്പെടുന്നവയെക്കുറിച്ചുള്ള തെറ്റിദ്ധാരണമൂലം ബോധം ജീവഭാവത്തില്‍ സംസാരചക്രത്തിലെ ആവര്‍ത്തനം തുടര്‍ന്നുകൊണ്ടിരിക്കുന്നു. അറിവും അറിവാളിയും രണ്ടാണെന്ന തെറ്റിദ്ധാരണ നിങ്ങുമ്പോള്‍ സമതുലിതാവസ്ഥയിലേയ്ക്കു തിരിച്ചുവരുകയും ചെയ്യുന്നു."

ചിലപ്പോള്‍ ക്രമീകമായും മറ്റുചിലപ്പോള്‍ അല്ലാതേയും ഈ മഹാജീവന്‍, പോയ ജന്മത്തില്‍നിന്നും ആര്‍ജ്ജിച്ച ദ്വന്ദതയും വ്യക്തിത്വവും മൂലം വ്യക്തിഗതജീവനാവുന്നു. ബോധത്തിന്റെ വിവരണാതീതവും മാസ്മരീകവുമായ പ്രഭാവത്താല്‍ അഹംകാരമെന്നറിയപ്പെടുന്ന നാമരൂപങ്ങള്‍ ഉണ്ടാവുന്നു. അതേ ബോധം സ്വരൂപത്തെ അനുഭവിച്ചറിയാന്‍ ഇച്ഛിക്കുമ്പോള്‍ വിശ്വാവബോധമാവുന്നു. പക്വതയില്ലാത്തവര്‍ മാത്രമേ ഇതൊരു മായക്കാഴ്ച്ചയോ 'മാറ്റമോ' ആണെന്നു കരുതുകയുള്ളു, കാരണം ബോധമല്ലാതെ മറ്റൊരുണ്മയുമില്ല. സമുദ്രം ജലമാണ്‌. അലകള്‍ ജലമാണ്‌. ഈ അലകള്‍ സമുദ്രോപരി വര്‍ത്തിക്കുമ്പോള്‍ ഓളങ്ങളുണ്ടാവുന്നു. അവയും ജലം തന്നെ. അതുപോലെയാണീ വിശ്വം. അലകളെ വ്യക്തിത്വമുള്ളവയായി സമുദ്രം 'കണ്ടാല്‍' എന്നതുപോലെ വിശ്വബോധം ജീവന്‌ വ്യക്തിത്വം കല്‍പ്പിച്ചിരിക്കുന്നു. അങ്ങിനെ അഹംകാരം (ഞാന്‍) ഉണ്ടായി. ഇതാണ്‌ ബോധത്തിന്റെ മാസ്മരീക പ്രഭാവം. അതുതന്നെയാണ്‌ വിശ്വമായതും.

സ്വയം, ബോധത്തില്‍ നിന്നു വിഭിന്നമല്ലെങ്കിലും അഹംകാരം ഒരു ജീവനില്‍ അങ്കുരിക്കുമ്പോള്‍ അത്‌ പ്രപഞ്ചത്തിലെ നാനാ നാമരൂപങ്ങളില്‍ താദാത്മ്യം പ്രാപിക്കുകയും അവ ഉണ്ടാവുകയും ചെയ്യുന്നു. ഏകത്വത്തില്‍ നാനാത്വമുദിക്കുന്നു. രാമ: 'ഞാന്‍, നീ' തുടങ്ങിയ ചിന്തകളും ജീവന്‍, അവയുടെ ഉല്‍ പ്പത്തി ഇവയെക്കുറിച്ചുള്ള ധാരണകളും ഉപേക്ഷിക്കൂ. ഇവയെല്ലാം പൊയ്ക്കഴിഞ്ഞാല്‍ നീ യാദാര്‍ത്ഥ്യത്തിനും അയാദാര്‍ത്ഥ്യത്തിനും മദ്ധ്യേ സത്യത്തെ സാക്ഷാത്കരിക്കും. ഈ മേഘപടലങ്ങള്‍ മാറിയാല്‍ അവിച്ഛിന്നമായ ആ പ്രബോധപ്രഭയില്‍ നിനക്കഭിരമിക്കാം. ആ പ്രഭ ഒരിക്കലും ഭാസുരമല്ലാതിരുന്നിട്ടില്ല. യാദാര്‍ത്ഥ്യവും അല്ലാത്തതും തമ്മിലുള്ള വ്യത്യാസം നമുക്കറിയില്ല. ഈ ബോധമെന്നത്‌ 'അറിയപ്പെടാവുന്ന' ഒന്നല്ല. അതിനു സ്വയം അറിയാന്‍ ഇച്ഛതോന്നുമ്പോള്‍ അത്‌ വിശ്വമെന്നറിയപ്പെടുന്നു. മനസ്സ്‌, ബുദ്ധി, അഹംകാരം, പഞ്ചഭൂതങ്ങള്‍, ലോകം, എന്നുവേണ്ട ലോകത്തിലെ എണ്ണമറ്റ നാമരൂപങ്ങള്‍ എല്ലാം ബോധം തന്നെ. ഒരു മനുഷ്യനും അവന്റെ പ്രവൃത്തികളും വേര്‍തിരിക്കാനരുതാത്തത്‌ അവ ഒരേവസ്തുവിന്റെ നിശ്ചലാവസ്ഥയും ചലനാത്മകതയും ആയതിനാലാണ്‌. ജീവന്‍, മനസ്സ്‌, തുടങ്ങിയ എല്ലാ കമ്പനങ്ങളും ബോധം മാത്രമാണ്‌.
