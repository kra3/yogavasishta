\section{ദിവസം 284}

\slokam{
തത്വാവബോധോ ഭഗവന്‍സര്‍വാശാതൃണപാവക:\\
പ്രോക്ത: സമാധി ശബ്ദേന നതു തൂഷ്ണീമവസ്ഥിതി:  (5/62/8)\\
}

പരിഘന്‍ പറഞ്ഞു: അല്ലയോ രാജാവേ, അടിയുറച്ച സമതാദര്‍ശനത്തോടെ ചെയ്യുന്ന കര്‍മ്മങ്ങള്‍ മാത്രമേ ആനന്ദമുളവാക്കുകയുള്ളൂ. അങ്ങ് ചിന്തകളും ധാരണകളും ശല്യപ്പെടുത്താത്ത മനസ്സോടെ സമാധിസ്ഥിതിയില്‍ ദൃഢചിത്തനാണോ?

സുരാഗു പറഞ്ഞു: മഹാത്മന്‍, എന്തുകൊണ്ടാണ് ചിന്തകളും ധാരണകളും ഇല്ലാത്ത മാനസീകാവസ്ഥയെ മാത്രം സമാധിയെന്നു പറയുന്നത്? ഒരാള്‍ സത്യസാക്ഷാത്കാരം നേടിയ ആളാണെങ്കില്‍ അയാള്‍ നിരന്തരം കര്‍മ്മങ്ങളില്‍ ഏര്‍പ്പെട്ടാലും അല്ലെങ്കില്‍ എപ്പോഴും ധ്യാനനിരതനായിരുന്നാലും അയാളിലെ സമാധി അവസ്ഥയ്ക്ക് മാറ്റമുണ്ടാവുമൊ?

ഇല്ല. പ്രബുദ്ധതയില്‍ എത്തിയവന്‍, ലോകകാര്യങ്ങളില്‍ ആമഗ്നനായിരിക്കുമ്പോഴും എപ്പോഴും സമാധിസ്ഥനത്രേ. എന്നാല്‍ മനസ്സ് ശാന്തമല്ലാത്തവന്‍ പത്മാസനത്തില്‍ ഇരുന്നാലും  സമാധിയവസ്ഥ അനുഭവിക്കുന്നില്ല. “സത്യത്തെപ്പറ്റിയുള്ള നേരറിവ്, അതായത് ബ്രഹ്മം, എല്ലാ ആശകളേയും പ്രത്യാശകളെയും ഉണങ്ങിയ പുല്‍ത്തുമ്പുകളെപ്പോലെ എരിച്ചുകളയുന്ന അഗ്നിയാണ്. അതാണ്‌ സമാധി. വെറുതെ നിശ്ശബ്ദമായി ഇരിക്കുന്നതല്ല.”

ശാശ്വതമായ സംതൃപ്തിയുടെ തലമാണ് സമാധി. കാര്യങ്ങളെപ്പറ്റിയുള്ള തെളിഞ്ഞ അറിവ്, അഹംകാരത്തിന്റെ അഭാവം, ദ്വന്ദഭാവങ്ങളുടെ സ്വാധീനമില്ലായ്മ, ആശങ്കകളില്ലാത്ത മനസ്സ്, യാതൊരു വസ്തുക്കളേയും ഗ്രഹിക്കാനോ ത്യജിക്കാനോ ഉള്ള ആഗ്രഹമില്ലായ്മ എന്നിവയാണ് സമാധി ലക്ഷണങ്ങള്‍ .  ആത്മജ്ഞാനോദയം ഉണ്ടാകുമ്പോള്‍ത്തന്നെ സാധകനില്‍ സമാധിസ്ഥിതി രൂഢമൂലമാവുന്നു. അവനതൊരിക്കലും നഷ്ടപ്പെടുകയില്ല; ഒരു നിമിഷം പോലും അത് തടസ്സപ്പെടുകയുമില്ല. കാലം മുന്നോട്ടുള്ള ഗമനം മറന്നാലും ആത്മജ്ഞാനി ആത്മാവിനെ മറക്കുകയില്ല. പദാര്‍ത്ഥങ്ങള്‍ അതായിത്തന്നെ നിലകൊള്ളുന്നത് പോലെ ആത്മജ്ഞാനിയായ മഹര്‍ഷി ആത്മജ്ഞാനിയായിത്തന്നെ നിലനില്‍ക്കും.

അതിനാല്‍ ഞാന്‍ എപ്പോഴും ഉണര്‍ന്നിരിക്കുന്നു. നിത്യശുദ്ധനാണ് ഞാന്‍. . സ്വയം ശാന്തനും സമാധിസ്ഥനുമാണ് ഞാന്‍.. അതങ്ങിനെയല്ലേ വരൂ? ആത്മാവല്ലാതെ എന്തെങ്കിലും ഉണ്ടോ? എല്ലായ്പ്പോഴും എല്ലാവിധത്തിലും അവകളുടെ സത്തായി ആത്മാവ് മാത്രം ഉള്ളപ്പോള്‍ സമാധിസ്ഥിതിയല്ലാത്ത ഒരവസ്ഥ വാസ്തവത്തില്‍ ഉണ്ടോ? സമാധിയെ എങ്ങിനെയാണ് നിര്‍വചിക്കുക? 

പരിഘന്‍ പറഞ്ഞു: ശരിയാണ് രാജാവേ. അങ്ങ് പൂര്‍ണ്ണപ്രബുദ്ധന്‍ തന്നെയാണ്. അങ്ങ് ആനന്ദത്തിന്റെ സ്വയംപ്രഭയാല്‍ പ്രശോഭിതനാണ്. പ്രശാന്തനും നിര്‍മ്മലനും മധുരപ്രകൃതിയുമാണ്. അങ്ങില്‍ അഹംകാരലേശമില്ല. അങ്ങില്‍ ഒന്നിനോടും ആശയോ വെറുപ്പോ എന്തിനെയെങ്കിലും ത്യജിക്കാനുള്ള വ്യഗ്രതയോ ഇല്ല എന്ന് ഞാന്‍ അറിയുന്നു.

സുരാഗു തുടര്‍ന്നു: മഹര്‍ഷേ, ആഗ്രഹിക്കാനോ ത്യജിക്കാനോ വാസ്തവത്തില്‍ യാതൊന്നുമില്ല. ബാഹ്യമായ ആ വസ്തുക്കളെല്ലാം അങ്ങിനെ നോക്കിയാല്‍ വെറും ധാരണകളും സങ്കല്‍പ്പങ്ങളും മാത്രമാണെന്ന് കാണാം. ബാഹ്യമായ യാതൊന്നും സ്വന്തമാക്കാന്‍ യോഗ്യതയുള്ളതല്ല എന്ന് തിരിച്ചറിയുമ്പോള്‍ യാതൊന്നും ഉപേക്ഷിക്കാനും അനുയോജ്യമല്ലാ എന്ന് തിരിച്ചറിയും. നന്മ-തിന്മകള്‍, വലുപ്പച്ചെറുപ്പങ്ങള്‍, തള്ളേണ്ടത്- കൊള്ളേണ്ടത്, എന്ന തരത്തിലുള്ള തരംതിരിവുകള്‍ , എല്ലാം അഭികാമ്യതയുമായി ബന്ധപ്പെട്ടിരിക്കുന്നു. എന്നാല്‍ ഈ അഭികാമ്യത എന്ന ധാരണയ്ക്ക് അര്‍ത്ഥമൊന്നുമില്ലാത്തപ്പോള്‍പ്പിന്നെ മറ്റുള്ളവയെപ്പറ്റി എന്തുപറയാന്‍? ലോകത്ത് കാണപ്പെടുന്ന യാതൊന്നിനും സത്തയില്ല. പര്‍വ്വതങ്ങള്‍ക്കും, സമുദ്രങ്ങള്‍ക്കും, കാടുകള്‍ക്കും, സ്ത്രീപുരുഷന്മാര്‍ക്കും, മറ്റ് വസ്തുക്കള്‍ക്കും ഒന്നും വാസ്തവത്തില്‍ അസ്തിത്വം ഇല്ല. അതിനാല്‍ അവയില്‍ നമുക്ക് ആശയെന്തിനാണ്? ആശയില്ലാത്തപ്പോഴാണ് മനസ്സില്‍ പരമപ്രശാന്തി കളിയാടുന്നത്.
