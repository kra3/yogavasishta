\newpage
\section{ദിവസം 128}

\slokam{
മനോവിലാസ: സംസാര ഇതി യസ്യാം പ്രതിയതേ\\
സര്‍വ്വശക്തേരനന്തസ്യ വിലാസോ ഹി മനോജഗത്‌ (3/109/25)\\
}

രാജാവ്‌ തുടര്‍ന്നു: ക്ഷാമത്താല്‍ വലഞ്ഞ്‌ പലയാളുകളും നാടുവിട്ട്‌ മറ്റിടങ്ങളിലേയ്ക്ക്‌ കുടിയേറിപ്പാര്‍ത്തു. ഭാര്യയോടും മക്കളോടുമുള്ള ആസക്തിമൂലം ചിലര്‍ ആ മണ്ണില്‍ത്തന്നെ ജീവിക്കാന്‍ ശ്രമിച്ച് ചത്തൊടുങ്ങി. കുറേപ്പേരെ വന്യമൃഗങ്ങള്‍ കൊന്നുതിന്നു. ഞാനും എന്റെ ഭാര്യയോടും മക്കളോടും കൂടി അവിടം വിട്ടു. രാജ്യത്തിന്റെ അതിര്‍ത്തിയില്‍ക്കണ്ട ഒരു തണല്‍മരത്തിനുകീഴെ ചെറിയകുട്ടികളെ തോളില്‍ നിന്നിറക്കിവച്ച്‌ കുറച്ചുനേരം ഞാന്‍ വിശ്രമിച്ചു. ഏറ്റവും ഇളയകുട്ടിയുടെ നിഷ്കളങ്കതയും ഓമനത്വവുംകാരണം എനിക്കവന്‍ ഏറെ പ്രിയപ്പെട്ടവനായിരുന്നു. അവന്‍ വിശന്നു കരയുന്നു. നമ്മുടെ കയ്യില്‍ മാംസമൊന്നും ഇല്ലെന്നു പറഞ്ഞിട്ടും അവനതുകൂട്ടാക്കാതെ നിര്‍ബ്ബന്ധം പിടിച്ചു കരഞ്ഞു. അവസാനം സഹികെട്ട്‌ ഞാന്‍ പറഞ്ഞു: 'തിന്ന്, എന്റെ മാംസം തന്നെയാവട്ടെ'. കളങ്കമില്ലാത്ത അവന്‍ പറഞ്ഞു: 'തരൂ'. എന്റെ ഈ ദുര്‍വിധിയിലും ആസക്തിയിലും  എനിക്കു  സ്വയം വല്ലാത്ത അവമതിപ്പു തോന്നി. എന്തൊരു കഷ്ടം! ഇനിയും വിശപ്പു താങ്ങാന്‍ കുട്ടിക്കാവുമായിരുന്നില്ല. ഈ ദുരിതങ്ങള്‍ അവസാനിപ്പിക്കാന്‍ ഏറ്റവും നല്ല മാര്‍ഗ്ഗം ആത്മഹത്യയാണെന്നു തീരുമാനിച്ച്‌ അടുത്തുകണ്ട്‌ തടിക്കഷണങ്ങളെടുത്ത്‌ ഞാന്‍ ഒരു ചിതയൊരുക്കി. ചിതയിലേയ്ക്കു കയറുമ്പോഴാണ്‌ ഞാന്‍ പേടിച്ചുവിറച്ച്‌ എഴുന്നേറ്റത്‌.. അപ്പോള്‍ ഞാനിവിടെ ഈ രാജസഭയില്‍ ഇരിക്കുന്നു! നിങ്ങള്‍ എന്നെ ഉപചാരങ്ങള്‍കൊണ്ടു പൊതിയുന്നു! എന്തൊരു മറിമായം!

രാജാവ്‌ ഇത്രയും പറഞ്ഞപ്പോള്‍ ജാലവിദ്യക്കാരന്‍ അപ്രത്യക്ഷനായി. മന്ത്രിമാര്‍ പറഞ്ഞു: അയാളൊരു വെറും ജാലവിദ്യക്കാരനല്ല. അയാള്‍ പണത്തിനോ സമ്മാനത്തിനോവേണ്ടി കാത്തുനിന്നില്ലല്ലോ. തീര്‍ച്ചയായും ഏതോ ഒരു ദേവത അങ്ങേയ്ക്കും ഞങ്ങള്‍ ക്കും മായാപ്രപഞ്ചത്തെ ഒന്നു കാണിച്ചുതരാന്‍ ചെയ്തതാണിത്‌.. "ഇതില്‍നിന്നും ഒരുകാര്യം സുവ്യക്തമാണ്‌.. കാണപ്പെടുന്ന ഈ ലോകം മനസ്സിന്റെ സൃഷ്ടിയാണ്‌.. മനസ്സ്‌ സര്‍വ്വശക്തനായ അനന്തപുരുഷന്റെ ലീലയുമാണ്‌.". ഈ മനസ്സിന്‌ ജ്ഞാനമുറച്ച മഹാന്മാരെപ്പോലും വിഡ്ഢികളാക്കാന്‍ കഴിയുന്നു. അല്ലെങ്കില്‍പ്പിന്നെ സര്‍വ്വകലാവല്ലഭനും ജ്ഞാനിയുമായ രാജാവിന്‌ ഈ വിഭ്രാന്തി എങ്ങിനെയുണ്ടായി? ഇതു ജാലവിദ്യയല്ല. സാധാരണ ജാലവിദ്യക്കാരന്‍ സമ്മനങ്ങള്‍ക്കുവേണ്ടിയാണ്‌ ജോലിചെയ്യുന്നത്‌.. ഇത്‌ മായാശക്തിതന്നെ. അതാണ്‌ അയാള്‍ പ്രതിഫലത്തിനു കാത്തു നില്‍ക്കാതിരുന്നത്‌..

വസിഷ്ഠന്‍ പറഞ്ഞു: രാമ: ഞാന്‍ അന്നാ സഭയില്‍ ഉണ്ടായിരുന്നതുകൊണ്ടാണ്‌ ഇത്ര കൃത്യമായി ഇതെല്ലാമറിഞ്ഞത്‌.. മനസ്സ്‌ ഇങ്ങനെയൊക്കെ സത്തിനെ മറച്ച്‌ ശിഖരങ്ങളും കായും പൂവും ഉള്ള മായക്കാഴ്ച്ചകള്‍ കാണിക്കുകയാണ്‌.. ഈ ഭ്രമത്തെ ഇല്ലാതാക്കി ശാന്തിയടഞ്ഞാലും. 

