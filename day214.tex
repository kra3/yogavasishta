\section{ദിവസം 214}

\slokam{
ചേത്യേന രഹിതാ യൈഷാ ചിത്തദ്ബ്രഹ്മ സനാതനം\\
ചേത്യേന സഹിതാ യൈഷാ ചിത്സേയം കലനോച്യതേ (5/13/53)\\
}

വസിഷ്ഠൻ തുടർന്നു: നീ എല്ലാറ്റിനേയും അറിയുന്ന ആത്മാവാണ്‌..  നീ അജനാണ്‌.. പരം പൊരുളാണ്‌..  സർവ്വവ്യാപിയായ ആത്മാവിൽ നിന്നും വിഭിന്നനല്ല നീ. ആത്മാവല്ലാതെ വേറൊരു വിഷയമില്ല എന്ന അറിവുറച്ചവന്‌ സുഖദു:ഖജന്യങ്ങളായ പ്രശ്നങ്ങൾ ബാധിക്കുകയില്ല. കളിമണ്ണിനും സ്വർണ്ണക്കട്ടയ്ക്കും ഒരേ മൂല്യം കൽപ്പിക്കുന്നവന്‌ ഒന്നിനോടും ഇഷ്ടാനിഷ്ടങ്ങളില്ല. അങ്ങിനെ, പ്രത്യക്ഷലോകമെന്ന ദൃശ്യത്തെ ന്യായീകരിക്കുന്ന വാസനകൾ യാതൊന്നുമില്ലാത്തവൻ യോഗിയാണ്‌..  അയാൾ എന്തൊക്കെ ചെയ്താലും, എന്തൊക്കെ ആസ്വദിച്ചാലും, എന്തൊക്കെ നശിപ്പിച്ചാലും അയാളുടെ ബോധം സർവ്വസ്വതന്ത്രവും സുഖദു:ഖങ്ങളെ ഒരേപോലെ കാണാൻ പര്യാപ്തവുമത്രേ. അഭിമതം, അനഭിമതം എന്നിങ്ങനെ കർമ്മങ്ങളെ വേർതിരിക്കാതെ ചെയ്യുന്നവരെ ആ കർമ്മങ്ങൾ പിടികൂടുന്നില്ല. അനന്താവബോധമല്ലാതെ മറ്റൊന്നുമില്ല എന്നു ദൃഢമായുറച്ചവനെ സുഖാനുഭവചിന്തകൾ അലട്ടുന്നില്ല. അവൻ ആത്മനിയന്ത്രണമുള്ളവനും പ്രശാന്തനുമാണ്‌.. 

മനസ്സിന്റെ സഹജഭാവം ജഢമാണ്‌..  അത് അനന്താവബോധത്തിൽ നിന്നും ബുദ്ധിയെ കൈക്കൊണ്ട് അനുഭവലാഭത്തിയായി പരിശ്രമിക്കുന്നു. ചിത്ശക്തിയുടെ പ്രഭാവത്താൽ വന്നുചേരുന്ന വിഷയങ്ങളുമായി മനസ്സിനു സമ്പർക്കമുണ്ടാവുന്നു. മനസ്സ് അവബോധത്തിന്റെ ദയവിലാണ്‌ പ്രവർത്തനോന്മുഖമാവുന്നത്. അതീ പ്രപഞ്ചത്തെപ്പറ്റിയുള്ള വൈവിദ്ധ്യമാർന്ന ധാരണകളെ ചിന്തകളായി നിലനിർത്തുന്നു. മനസ്സിന്റെ വെളിച്ചം ഈ അവബോധമൊന്നു മാത്രമാണ്‌..  അല്ലെങ്കിൽ ഈ ജഢമനസ്സിന്‌ ബുദ്ധിയോടെ വർത്തിക്കാനാവുന്നതെങ്ങിനെ? ബോധമണ്ഡലത്തിലെ ചൈതന്യത്തിലുണ്ടാകുന്ന മിഥ്യയായ ചലനമാണ്‌ മനസെന്ന് വേദശാസ്ത്ര വിജ്ഞാനമുള്ളവർ പറയുന്നു. സർപ്പശീൽക്കാരം പോലെ മനസ്സിൽ നിന്നു വിക്ഷേപിക്കുന്നവയാണ്‌ ചിന്തകളും ആശയങ്ങളും.

“ബോധത്തിൽനിന്നും സങ്കൽപ്പധാരണകൾ മാറ്റിയാല്‍ അവശേഷിക്കുന്നതു പരബ്രഹ്മം. ബോധത്തിനോട് സങ്കൽപ്പധാരണകൾ ചേർന്നാൽ ചിന്തകൾ.” അവയിലൊരംശമെന്നപോലെ ഹൃദയത്തിൽ ഉണ്മയായി അതു നിലകൊള്ളൂന്നു. ഇത്‌ പരിമിതബുദ്ധി അല്ലെങ്കിൽ വ്യക്തിഗത ബോധം എന്നറിയപ്പെടുന്നു. എന്നാൽ ഈ പരിമിതബോധം തന്റെ മൂലസ്വഭാവമായ അനന്താത്മകതയെ ‘മറന്ന്’ ജഢമായി തുടരുകയാണ്‌..  അതുപിന്നെ ചിന്താശക്തിയായി ഇഷ്ടാനിഷ്ടങ്ങളുടെ പ്രകടനങ്ങളായ ‘തള്ളലും കൊള്ളലും’ സഹജഭാവങ്ങളാക്കുന്നു.

വാസ്തവത്തിൽ അനന്താവബോധമാണിതൊക്കെ 'ആയിത്തീർ'ന്നത്. എന്നാൽ ആ വസ്തുതയിലേയ്ക്ക്, അനന്തതയിലേയ്ക്ക്, ഉണരുംവരെ അതിന്‌ സ്വയം ആത്മാവാണു താനെന്ന അറിവില്ല. അതുകൊണ്ട് മനസ്സിനെ ശാസ്ത്രോക്തങ്ങളായ നിർമമത, ഇന്ദ്രിയനിയന്ത്രണം മുതലായവയുടെ സഹായത്തോടെ ഉണർത്താനുള്ള പരിശ്രമം ചെയ്യണം. മേധാശക്തി അങ്ങിനെ ഉണർത്തിയാൽ അതു പരബ്രഹ്മമായി തെളിയുകയായി. അല്ലെങ്കിൽ അതു ലോകാനുഭവങ്ങൾ ആർജ്ജിച്ചുകൊണ്ടേയിരിക്കും 
