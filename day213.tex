\section{ദിവസം 213}

\slokam{
ധാവമാനമധോഭാഗേ ചിത്തം പ്രത്യാഹരേദ്ബലാത്\\
പ്രത്യാഹാരേണ പതിതമധോ വാരീവ സേതുനാ (5/13/30)\\
}

വസിഷ്ഠൻ തുടർന്നു: ആരിലാണോ വിഷയാദികൾക്കായുള്ള അത്യാഗ്രഹവും അവയോടുള്ള നിരാസവും എന്നീ ദ്വന്ദഭാവങ്ങൾ ഒടുങ്ങിയത്, അവർ ഒന്നിനുവേണ്ടിയും ആശിക്കുന്നില്ല. ഒന്നിനേയും അവർക്ക് തള്ളാനുമില്ല. ഈ ദ്വന്ദചോദനകൾ (ആഗ്രഹ-നിരാസങ്ങൾ) ഒടുങ്ങിയാലല്ലാതെ മനസ്സിൽ പ്രശാന്തിയും സമതാഭാവവും ഉണ്ടാവുകയില്ല. ‘ഇതു സത്ത’, ‘ഇതസത്ത’ എന്നിങ്ങനെയുള്ള തരംതിരിവ് ഉള്ളിലുള്ളവർക്കും സമാധാനമുണ്ടാവുകയില്ല. ‘ഇതു ലാഭം’, ‘ഇതു നഷ്ടം’ എന്നും ‘ഇതു ശരി’, ‘ഇതു തെറ്റ്’ എന്നുമെല്ലാം വേർതിരിച്ച് കാണുന്നവരിൽ സമത, നൈർമ്മല്യം, നിർമമത എന്നീ സദ്ഗുണങ്ങൾ എങ്ങിനെയുണ്ടാവാനാണ്‌? ബ്രഹ്മം എന്ന ഒന്നേയൊന്നുമാത്രം എന്നെന്നും ഒന്നായും പലതായും നിലകൊള്ളുമ്പോൾ ശരി തെറ്റുകൾക്കെന്തു സ്ഥാനം? മനസ്സ് ഇഷ്ടാനിഷ്ടങ്ങളിൽ അഭിരമിക്കുമ്പോൾ സമതാഭാവം ഇല്ലല്ലോ?.

സഹജമായും പ്രയത്നരഹിതമായും ആഗ്രഹ-നിരാസ ദ്വന്ദചോദനകളില്ലാത്തവരിൽ പ്രത്യാശയില്ലായ്മ, ഭയമില്ലായ്മ, നൈഷ്കർമ്മ്യം, ദൃഢത, സമത, ധൈര്യം, സഹിഷ്ണുത, മേധാശക്തി, അനാസക്തി, നന്മ, കുടിലതയില്ലായ്മ, സദ്ഭാഷണം, സൗഹൃദം, ജ്ഞാനം, സംതൃപ്തി എന്നീ ഗുണങ്ങൾ വിരാജിക്കുന്നു.

“മനസ്സിന്റെ ഒഴുക്കിനെ അധ:പ്പതനത്തിൽ നിന്നും തടയുക എന്നത് സാധകന്റെ കടമയാണ്‌.. പുഴയുടെ ഒഴുക്കിനെ തടയാനായി അണകെട്ടാറുണ്ടല്ലോ.” ബാഹ്യവസ്തുക്കളുമായുള്ള ബന്ധമെല്ലാമവസാനിപ്പിച്ചശേഷം വൈവിദ്ധ്യമാർന്ന കർമ്മങ്ങളിലേർപ്പെട്ടിരിക്കുകയാണെങ്കിലും മനസ്സിനെ ഉള്ളിലേക്കുന്മുഖമാക്കി അവിടെയുള്ള എല്ലാറ്റിനേയും പറ്റി ചിന്തിക്കുക. വിജ്ഞാനത്തിന്റെ മൂർച്ചയേറിയ വാളുകൊണ്ട് മനോപാധികളാകുന്ന വലയുടെ കണ്ണികളെ  അറുത്തെറിയുക. ആസക്തികൾ, ധാരണകൾ, ചോദനകൾ, സ്വീകാര-നിരാസങ്ങൾ എല്ലാം ഈ വലയുടെ കണ്ണികളത്രേ. അവയാണല്ലോ ലോകമെന്ന ഈ പ്രകടനത്തിനുത്തരവാദികൾ. മനസ്സിനെ മനസ്സുകൊണ്ടറുത്തുമാറ്റുക. ശുദ്ധനൈർമ്മല്യാവസ്ഥയെ പ്രാപിച്ചാൽ അതിൽത്തന്നെ ദൃഢമായുറച്ചുനിൽ ക്കുക.

മനസ്സുകൊണ്ട് മനസ്സിനെ നീക്കി മനസ്സിനെ നിരാകരിക്കുന്ന ആ ചിന്താശകലത്തെപ്പോലും ഉപേക്ഷിക്കുക. അങ്ങിനെ നിനക്കീ പ്രത്യക്ഷലോകത്തെ അപ്പാടെ ഇല്ലാതാക്കാം. ഇങ്ങിനെ ലോകമില്ലാതായാല്‍പ്പിന്നെ ഭ്രമകൽപ്പനകൾ ഇല്ല. മനസ്സിൽ ലോകം വിണ്ടും വീണ്ടും പ്രത്യക്ഷപ്പെടുകയില്ല. ലോകത്തിന്റെ അയാഥാർത്ഥ്യസ്ഥിതിയിൽ രൂഢമൂലമായിരിക്കുമ്പോഴും ലൗകീകകർമ്മങ്ങൾ ഭംഗിയായി അനുഷ്ഠിക്കുക. ആ കർമ്മങ്ങളിൽ ആശയോ പ്രതീക്ഷകളോ അർപ്പിച്ച് അവയോടു ബന്ധം സ്ഥാപിക്കേണ്ടതില്ല. സമതയിലഭിരമിച്ച് സ്വാഭാവികമായി, സഹജമായി വന്നുചേരുന്ന കർമ്മങ്ങളെ ഭംഗിയായി ചെയ്തുതീർക്കുക. എന്താണു വന്നുചേർന്നതെന്നതിനെപ്പറ്റി ചിന്തിക്കാതെ, ഒന്നും പ്രതീക്ഷിക്കാതെ, ചോദിക്കാതെ വന്നുചേർന്നവയെ നിർമമതയോടെ സ്വീകരിക്കുക. ഭഗവാൻ സ്വയം കർത്താവും അകർത്താവുമാണെന്നു പറയപ്പെടുന്നു. നീയും അതുപോലെയാവുക. അനിച്ഛാപൂർവ്വം കർമ്മം ചെയ്യുക. അങ്ങിനെ നിയതകർമ്മങ്ങളുടെ കർത്താവും അകർത്താവുമാവുക. 
