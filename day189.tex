\section{ദിവസം 189}

\slokam{
ക്രിയാവിശേഷബഹുലാ ഭോഗൈശ്വര്യഹതാശയാ:\\
നാപേക്ഷന്തേ യദാ സത്യം ന പശ്യന്തി ശഠാസ്തദാ (4/48/1)\\
}

വസിഷ്ഠൻ തുടർന്നു: "രാമ, സുഖത്തിനും അധികാരത്തിനും വേണ്ടി ലോകത്തിലെ വിവിധങ്ങളായ കാര്യങ്ങളിൽ ആമഗ്നരായിക്കഴിയുന്നവർക്ക് സത്യാന്വേഷണത്വരയുണ്ടാവുകയില്ല. കാരണം അവരതു കണുന്നില്ലല്ലോ." ജ്ഞാനിയെങ്കിലും ഇന്ദ്രിയസുഖസമ്പാദനാസക്തി തീരെ വിട്ടുപോവാത്തവൻ സത്യം കാണുന്നുണ്ടെങ്കിലും  അതോടൊപ്പം അയാള്‍  ഭ്രമദൃശ്യങ്ങളും കണുന്നുണ്ട് . എന്നാൽ ലോകത്തിന്റെയും ജീവന്റെയും സഹജസ്വഭാവം മുഴുവനായി മനസ്സിലാക്കിയ ഒരുവൻ പ്രത്യക്ഷലോകത്തെ ഉപേക്ഷിച്ചവനാണ്‌.. അവന്‌ ഇനി ജനന മരണങ്ങളില്ല. അവനാണ്‌ മുക്തൻ. അജ്ഞാനി ജീവിക്കുന്നത് സ്വന്തം ദേഹത്തിന്റെ സംരക്ഷയ്ക്കായാണ്‌.. അത്മാവല്ല അവന്റെ വിഷയം. രാമ, അജ്ഞാനിയാവാതെ ജ്ഞാനനിഷ്ഠനാവൂ. ഇതിനെക്കുറിച്ച് രസകരമായ ഒരു കഥ ഞാൻ പറയാം.

നന്ദനോദ്യാനങ്ങൾ കൊണ്ടു നിറഞ്ഞ മഗധരാജ്യത്ത് ദാസുരൻ എന്ന പേരുള്ള ഒരു മഹർഷി ജീവിച്ചിരുന്നു. ശ്വാസമടക്കിയും മറ്റും കഠിനമായ തപ:ശ്ചര്യയിലേർപ്പെട്ടിരുന്നു അദ്ദേഹം. വലിയ തപസ്വിയായ അദ്ദേഹത്തിന്‌ ലൗകീകസുഖങ്ങളിൽ തീരെ താൽപ്പര്യമുണ്ടായിരുന്നില്ല. അദ്ദേഹം ശാസ്ത്രജ്ഞാനിയുമായിരുന്നു. സാരലോമനെന്ന മറ്റൊരു മഹർഷിയായിരുന്നു അദ്ദേഹത്തിന്റെ പിതാവ്‌..  കഷ്ടമെന്നു പറയട്ടേ, ചെറുപ്രായത്തിലേ അദ്ദേഹത്തിന്റെ അച്ഛനമ്മമാർ മരിച്ചുപോയിരുന്നു. വനദേവതമാർക്ക് അനുകമ്പ തോന്നി അവർ അദ്ദേഹത്തോടിങ്ങിനെ പറഞ്ഞു: അല്ലയോ ജ്ഞാനിയായ ബാലാ, എന്തിനാണിങ്ങിനെ അജ്ഞാനികളെപ്പോലെ കരയുന്നത്? നീയൊരു മഹർഷിയുടെ മകനല്ലേ? ഈ പ്രത്യക്ഷലോകത്തിന്റെ ക്ഷണികതയെപ്പറ്റി നിനക്കറിയാമല്ലോ. ഈ ലോകത്തിന്റെ സ്ഥിതി അതാണ്‌.. വസ്തുക്കൾ പ്രത്യക്ഷമായി, പ്രകടമായി, കുറച്ചുകാലം നിലനിന്ന് ഇല്ലാതാകുന്നു. എന്തൊക്കെ കാണപ്പെടുന്നുവോ അവയെല്ലാം ആപേക്ഷികമാണ്‌. ബ്രഹ്മാവായാൽപ്പോലും അതങ്ങിനെതന്നെയാണ്‌. അവയുടെയെല്ലാം അന്ത്യം അനിവാര്യവുമാണ്‌.. ഇതു സംശയലേശമില്ലാത്ത സത്യമാണ്‌.. അതുകൊണ്ട് നീ അച്ഛനമ്മമാരുടെ മരണത്തിൽ വൃഥാ വിലപിക്കേണ്ടതില്ല. ഇതുകേട്ട്  ദു:ഖശാന്തി കൈവന്ന ബാലൻ അച്ഛനമ്മമാര്‍ക്കു വേണ്ട അന്ത്യകർമ്മങ്ങൾ എല്ലാം ചെയ്തു.

പിന്നീട് അയാൾ വളരെ നിഷ്കർഷതയോടെയുള്ള ആത്മീയജീവിതം നയിക്കാനാരംഭിച്ചു. അനേകം കർമ്മങ്ങളുള്ളതിനെ അദ്ദേഹം ചെയ്യേണ്ടവയെന്നും ചെയ്യരുതാത്തവയെന്നും രണ്ടായി തരം തിരിച്ചു. സത്യമിനിയും സാക്ഷാത്കരിക്കാതിരുന്നതുകൊണ്ട് അദ്ദേഹം യഗാദികളും മറ്റും അവയുടെ എല്ലാ നിയതക്രമങ്ങളുമനുസരിച്ച് ചെയ്തുവന്നു. ഈ കർമ്മങ്ങളുടെ ഫലമായി ലോകം മുഴുവൻ മാലിന്യം നിറഞ്ഞതാണെന്നൊരു ധാരണ അദ്ദേഹത്തിലുടലെടുത്തു. നിർമ്മലമായ ഒരിടം തേടി അവസാനം അദ്ദേഹം ഒരു മരമുകളിൽ വസിക്കാമെന്നു തീരുമാനിച്ചു. അതിനായി അദ്ദേഹം തീവ്രമായ ഒരു യാഗം നടത്തി. തന്റെ തലയറുത്ത് യാഗാഗ്നിയിൽ ഹോമിച്ചു. ഉടനേതന്നെ അഗ്നിഭഗവാൻ പ്രത്യക്ഷപ്പെട്ട് അഭീഷ്ടസിദ്ധി വരമായി നല്കി. 'നിന്റെ ഹൃദയാഭീഷ്ടം നടപ്പിലാവട്ടെ'. ദാസുരന്റെ അർഘ്യം സ്വീകരിച്ച് അഗ്നി അപ്രത്യക്ഷനായി. 
