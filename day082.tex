 
\section{ദിവസം 082}

\slokam{
ദേഹാദ് ദേഹാന്തര പ്രാപ്തി: പൂര്‍വ്വ ദേഹം വിനാ സദാ \\
ആതിവാഹികദേഹേഽസ്മിന്‍ സ്വപ്നേഷ്വിവ വിനശ്വരീ (3/57/22)\\
}

വസിഷ്ഠന്‍ തുടര്‍ന്നു: അവിടെ പദ്മരാജാവിന്റെ ശരീരത്തിനടുക്കല്‍ തന്റെ ഭര്‍ത്താവിനെ ഒരു വിശറികൊണ്ട്‌ വീശിക്കൊണ്ട്‌ പതിഭക്തിയോടെ മറ്റേ ലീല ഇരിക്കുന്നു! സരസ്വതീ ദേവിയും ആദ്യത്തെ ലീലയും അവളെ കണ്ടെങ്കിലും അവള്‍ക്ക്‌ അവരെ കാണാനായില്ല.

രാമന്‍ ചോദിച്ചു: ആദ്യം അങ്ങുപറഞ്ഞു ആദ്യത്തെ ലീല അവളുടെ ശരീരം തല്‍ക്കാലത്തേയ്ക്കുപേക്ഷിച്ചിട്ടാണ്‌ സരസ്വതിയുടെ കൂടെ സൂക്ഷ്മശരീരിയായി യാത്രപോയതെന്ന്. എന്നാല്‍ ഇപ്പോള്‍ അങ്ങ്‌ ആ ലീലയുടെ ശരീരത്തെപ്പറ്റി ഒന്നും പറയുകയുണ്ടായില്ലല്ലോ?

വസിഷ്ഠന്‍ പറഞ്ഞു: ആദ്യത്തെ ലീല, പ്രബുദ്ധയായിത്തീര്‍ന്നതുകൊണ്ട്‌ അവളുടെ സൂക്ഷ്മശരീരത്തിന്റെ അഹംകാരഭാവം ഇല്ലാതായി. അപ്പോള്‍ അതിനു സ്ഥൂലശരീരവുമായുള്ള ബന്ധവും മഞ്ഞുരുകും പോലെ വിട്ടകന്നുപോയി. വസ്തവത്തില്‍ സൂക്ഷ്മശരീരിയായ ലീലയുടെ അജ്ഞാനമാണ്‌ അവള്‍ക്കൊരു സ്ഥൂലശരീരമുണ്ടെന്ന തോന്നലുണ്ടാക്കിയത്‌. ഒരാള്‍ ഉറക്കത്തില്‍ 'ഞാനൊരു മാനാണ്‌' എന്നു സ്വപ്നംകാണുന്നു എന്നിരിക്കട്ടെ. ഉറക്കമുണരുമ്പോള്‍ ആ മാനിനെ കാണാനില്ലെന്നുപറഞ്ഞ്‌ അയാള്‍ തിരഞ്ഞു നടക്കുമോ? അതുപോലെയാണ്‌ ലീലയുടെ സ്ഥൂലശരീരവും.

ഭ്രമബാധിതന്റെ മനസ്സിലെ ഭാവനകള്‍ അങ്ങിനെതന്നെ മൂര്‍ത്തീകൃതമാവുന്നു. എന്നാല്‍ ഭ്രമം വിട്ടകന്നുകഴിഞ്ഞാല്‍ ഭാവനയിലുണ്ടായതെല്ലാം അതോടെ ഇല്ലാതാവുന്നു. കയറില്‍ പാമ്പിനെകണ്ടയാള്‍ കയറെന്ന സത്യാവസ്ഥ മനസ്സിലാക്കിയല്‍ അതിനുശേഷം 'പാമ്പിനു' സാംഗത്യം ഒന്നുമില്ലല്ലോ. ആവര്‍ത്തിച്ചുള്ള ഭ്രമകല്‍പ്പനമൂലം അസത്തിനെ സത്തെന്നു തെറ്റിദ്ധരിക്കുന്ന രീതി നമ്മില്‍ രൂഢമൂലമായിരിക്കുന്നു! "ഒരു സൂക്ഷ്മശരീരത്തില്‍ നിന്നും മറ്റൊന്നിലേയ്ക്കുള്ള കൂടുമാറ്റത്തിന്‌ ആദ്യത്തേതിനെ നശിപ്പിക്കേണ്ടതില്ല. സ്വപ്നത്തില്‍ നാമൊരു രൂപമെടുത്തിട്ട്‌ മറ്റൊന്നിലെയ്ക്കു മാറുമ്പോള്‍ ആദ്യത്തെ രൂപം ഉപേക്ഷിക്കേണ്ടതില്ലല്ലോ." 

യോഗിയുടെ ശരീരം സത്യത്തില്‍  സൂക്ഷ്മവും അദൃശ്യവുമത്രേ. എന്നാല്‍ അജ്ഞാനികള്‍ക്ക്‌ അതു പ്രത്യക്ഷമായി തോന്നുകയാണ്‌. അങ്ങിനെയുള്ള അജ്ഞാനികളാണു പറയുന്നത്‌ 'ആ യോഗി അന്തരിച്ചു' എന്ന്. എവിടെയാണ്‌ ശരീരം? എന്താണു നിലനില്‍ക്കുന്നത്‌? എന്താണു നശിക്കുന്നത്‌? എന്തുണ്മയാണോ അതുമാത്രം ഉണ്ട്‌. ഇല്ലാതാവുന്നതോ, ഭ്രമം മാത്രം!.

രാമന്‍ ചോദിച്ചു: മഹാത്മന്‍, യോഗിയുടെ ഭൌതീകശരീരം പിന്നെ സൂക്ഷ്മശരീരമായി മാറുമോ? 

വസിഷ്ഠന്‍ പറഞ്ഞു: എത്ര തവണ ഞാന്‍ പറഞ്ഞു രാമാ, എന്നിട്ടും നിനക്ക്‌ മനസ്സിലാകുന്നില്ലല്ലോ! സൂക്ഷ്മശരീരം മാത്രമേ ഉള്ളൂ. തുടര്‍ച്ചയായ ഭാവനകൊണ്ട്‌ അതൊരു ഭൌതീകശരീരവുമായി ബന്ധിക്കപ്പെട്ടതായി തോന്നുകയാണ്‌. അജ്ഞാനിയായ ഒരുവന്‍ മരിച്ചിട്ട്‌ ദേഹം തീയിലെരിച്ചുകളയുമ്പോള്‍ സൂക്ഷ്മശരീരം അവശേഷിക്കുന്നു. എന്നാല്‍ യോഗി, സ്വയം പ്രബുദ്ധനായതുകൊണ്ട്‌ ജീവിച്ചിരിക്കേ തന്നെ സൂക്ഷ്മശരീരിയാണ്‌. ഭൌതീക ശരീരം ഭാവനയാണ്‌. സത്തല്ല. ശരീരവും അവിദ്യയും ഒന്നുതന്നെ. അവ രണ്ടും വേറെയാണെന്നുള്ള ചിന്തയാണ്‌ സംസാരം, അഥവാ ആവര്‍ത്തന ചരിത്രം! 

