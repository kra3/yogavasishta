\section{ദിവസം 227}

\slokam{
തസ്മാദാസാമനന്താനാം തൃഷ്ണാനാം രഘുനന്ദന\\
ഉപായസ്ത്യാഗ ഏവൈകോ ന നാമ പരിപാലനം (5/21/5)\\
}

വസിഷ്ഠൻ തുടർന്നു: അങ്ങിനെ തന്റെ സഹോദരന്റെ വാക്കുകള്‍കേട്ട് പവനന്‌ ബോധോദയമുണ്ടായി. രണ്ടാളും നേരറിവിന്റെ, ജ്ഞാനത്തിന്റെ, പ്രഭാവം കൊണ്ടുണ്ടായ പ്രബുദ്ധതയിൽ അഭിരമിച്ചു. അവർ കാട്ടിൽ ഇഷ്ടം പോലെ അലഞ്ഞു നടന്നു. അവരെ യാതൊന്നും ബാധിച്ചില്ല. എണ്ണയില്ലാത്ത ദീപമണയുന്നതുപോലെ കുറച്ചുകാലം കഴിഞ്ഞ് അവർ ദേഹമുപേക്ഷിച്ചു മുക്തരായി.

“രാമാ, ആസക്തിയാണ്‌ എല്ലാ ദു:ഖങ്ങൾക്കും കാരണം. എല്ലാ ആസക്തികളേയും ഉപേക്ഷിച്ച് അവയിലൊന്നും ആമഗ്നമാവാതിരിക്കുക എന്ന ഒരൊറ്റ മാർഗ്ഗമേ ബുദ്ധിയുള്ളവർ തിരഞ്ഞെടുക്കൂ.” കൂടുതൽ ഇന്ധനമിടുമ്പോൾ തീ ആളിക്കത്തും. അതുപോലെ ചിന്തകൾ കൂടുതൽ ചിന്തകളെയുണ്ടാക്കുന്നു. അവ പെരുകി വളരുന്നതു തടയണമെങ്കിൽ ചിന്തകൾ ഇല്ലാതാക്കുക എന്നതേ വഴിയുള്ളു. അതിനാൽ രാമാ, ചിന്താരാഹിത്യത്തിന്റെ തേരിലേറി അപരിമിതവും കൃപാനിർഭരവുമായ ഒരന്തര്‍ദർശനത്താൽ ദു:ഖനിബദ്ധമായ ലോകത്തെ നോക്കിക്കാണൂ.

ഇതാണു ബ്രാഹ്മി-സ്ഥിതി. എല്ലാ ഭവരോഗങ്ങളിൽ നിന്നും ആസക്തികളിൽ നിന്നുമൊഴിഞ്ഞ ശുദ്ധമായ, സ്വതന്ത്രമായ സ്ഥിതിയാണിത്. മൂഢനായി വർത്തിച്ചിരുന്നവൻ പോലും ഈ സ്ഥിതിയെത്തിയാൽ എല്ലാ ഭ്രമകൽപ്പനകൾക്കും അതീതനാകും. ജ്ഞാനത്തെ സുഹൃത്താക്കി അവബോധത്തെ സഹധർമ്മിണിയാക്കി ഈ ലോകത്തിൽ ജീവിക്കുന്നവർ ഭ്രമങ്ങൾക്കു വശംവദരാവുകയില്ല. ആസക്തിരഹിതമായ മനസ്സിനു ലഭിക്കാത്തതായി വിലപിടിപ്പുള്ള യാതൊന്നും ത്രിലോകങ്ങളിലുമില്ല. ദേഹമെടുത്തതുകൊണ്ടുണ്ടാവുന്ന ഉയർച്ച താഴ്ച്ചകൾ ഇങ്ങിനെ ആസക്തിശമനം വന്നവരെ ബാധിക്കുകയില്ല. മനസ്സു നിറയണമെങ്കിൽ അതിനെ നിർമമമാക്കണം. ആശകൾകൊണ്ടും പ്രത്യാശകൾകൊണ്ടും നിറച്ചാൽ മനസ്സിനു നിറവുണ്ടാവുകയില്ല. ആസക്തികളും ആർത്തികളുമില്ലാത്തവരെ സംബന്ധിച്ചിടത്തോളം  ത്രിലോകങ്ങൾക്ക്  പശുക്കുളമ്പിന്റെയത്രയേ വിസ്തൃതിയുള്ളു. യുഗങ്ങളോ ചെറിയൊരുനിമിഷം മാത്രവും.

അനാസക്തന്റെ മന:ശീതളത, ഹിമാലയത്തിന്റെ ശീതളിമയെ നിഷ് പ്രഭമാക്കും. പൂർണ്ണചന്ദ്രന്റെ പൂനിലാവും, സമുദ്രത്തിന്റെ നിറവും, ഐശ്വര്യദേവതയും ഒന്നും അനാസക്തന്റെ മനസ്സിന്റെ ഭാസുരതയ്ക്കു സമമാവില്ല. ആശ, പ്രത്യാശ, ആർത്തി തുടങ്ങിയ ശിഖരങ്ങൾ മുറിച്ചാൽപ്പിന്നെ മനസ്സെന്ന വൃക്ഷത്തിന്‌ സ്വരൂപത്തിലേയ്ക്കു മടങ്ങാം. മനോദാർഢ്യത്തോടെ ആശകൾക്ക് എന്റെ മനസ്സിൽ ഇടംകൊടുക്കുകയില്ല എന്നുറച്ചാൽ നിനക്കു ഭയമുണ്ടാവുകയില്ല.

മനസ്സിൽ ചിന്തകളുടെ സഞ്ചാരം നിലയ്ക്കുമ്പോൾ, ആശകളും ആർത്തികളും വലയ്ക്കാതിരിക്കുമ്പോൾ പിന്നെ മനസ്സില്ല. ‘അമനസ്സ്’എന്ന സ്ഥിതിയാണു മുക്തി. ആഗ്രഹങ്ങളും പ്രത്യാശകളും മനസ്സിൽ കൊണ്ടുവരുന്ന ചിന്തകൾക്ക് ‘വൃത്തി’ എന്നു പറയുന്നു - അതായത് ചിന്തകളുടെ സഞ്ചാരം. ആശകളില്ലാത്തപ്പോൾ വൃത്തിയുമില്ല. കാരണത്തെ നീക്കം ചെയ്തപ്പോൾ കാര്യം അപ്രത്യക്ഷമായി! അതുകൊണ്ട് മന:സമാധാനമുണ്ടാവാൻ അതിനെ ശല്യപ്പെടുത്തുന്ന കാരണങ്ങളെ (ആശയും ആർത്തിയും) നീക്കം ചെയ്താൽ മതി.
