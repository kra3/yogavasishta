 
\section{ദിവസം 036}

ഭാഗം 3. ഉത്പത്തി പ്രകരണം ആരംഭം 

\slokam{
യഥാ രസ: പദാർത്ഥേഷു യഥാ തൈലം തിലാദിഷു\\
കുസുമേഷു യഥാ മോദസ്തഥാ ദൃഷ്ടരി ദൃശ്യധീ: (3/1/43)\\
}

വസിഷ്ഠന്‍ തുടര്‍ന്നു: ഇനി ഞാന്‍ സൃഷ്ടിയെപ്പറ്റിയും അതിന്റെ രഹസ്യത്തെപ്പറ്റിയും പറയാം. ലൌകീക വസ്തുവിനെ സത്തായി കണക്കാക്കുംവരെ മാത്രമേ ബന്ധനം നിലനില്‍ക്കൂ. ആ തെറ്റിദ്ധാരണ നീങ്ങുന്നതോടെ ബന്ധനത്തിനും അവസാനമായി. ഇവിടെ 'സൃഷ്ടിക്കപ്പെട്ടതിനു' മാത്രമേ വളര്‍ച്ച, തളര്‍ച്ച, നശിക്കല്‍ , സ്വര്‍ഗ്ഗ-നരക യാത്രകള്‍ എന്നിവയും പിന്നെ അവസാനം മുക്തിപദപ്രാപ്തി എന്നീ അവസ്ഥകള്‍ ഉള്ളു. വിശ്വപ്രളയത്തില്‍ സൃഷ്ടിക്കപ്പെട്ട എല്ലാം അനന്തതയില്‍ വിലീനമാവുന്നു. അതിനെയാണ്‌ ആശയവിനിമയത്തിനുവേണ്ടി ബ്രഹ്മം, ആത്മാവ്   , സത്യം എന്നീ പേരുകളാല്‍ മഹാത്മാക്കള്‍ വിളിക്കുന്നത്‌. ഇതേ പരമാത്മാവാണ്‌ ഒരുവന്റെയുള്ളില്‍ 'ഞാന്‍ , നീ' എന്ന ദ്വന്ദഭാവത്തേയും സ്ഫുരിപ്പിക്കുന്നത്‌. 

പ്രശാന്തമായ കടലിലെ ജലപ്പരപ്പില്‍ വായുക്ഷോഭം മൂലമോ മറ്റോ തിരകള്‍ ഉണ്ടാവുന്നതുപോലെ മനസ്സുണ്ടാവുന്നു. എന്നാല്‍ സ്വര്‍ണ്ണാഭരണങ്ങളുടെയെല്ലാം സത്ത സ്വര്‍ണ്ണമാണെന്നതുപോലെ സൃഷ്ടികളുടെ ഗുണഗണങ്ങളും സൃഷ്ട്യുന്മുഖമായ സര്‍ഗ്ഗശക്തിയും എല്ലാം സൃഷടാവില്‍ അന്തര്‍ലീനമാണെന്നറിയുക. ആഭരണങ്ങള്‍ (മാല, വള, മോതിരം..) നിലനില്‍ ക്കുന്നത്‌ സ്വര്‍ണ്ണത്തിന്റെ അസ്തിത്വത്തിലാണ്‌. എന്നാല്‍ സ്വര്‍ണ്ണത്തിന്റെ സ്വത്ത്വത്തിന്‌ ആഭരണങ്ങള്‍ അനുപേക്ഷണീയമല്ല. മനസ്സ്‌ പരമാത്മാവില്‍ നിന്നും വിഭിന്നമല്ല; കാരണം അതിന്‌ സ്വന്തമായി ഒരസ്തിത്വമില്ല. 

മരുഭൂമിയിലെ മരീചിക നല്ലൊരു നദിപോലെ കാണപ്പെടുന്നതുപോലെ സൃഷ്ടിയും തികച്ചും സത്തായി തോന്നുന്നു. 'ഞാന്‍ , നീ' എന്നതാണ്‌ സത്യം എന്ന ധാരണയുള്ളിടത്തോളം മുക്തി സാധ്യമല്ല. നാവുകൊണ്ട്‌ വെറുതേ അത്തരം ധാരണകളെ തള്ളിപ്പറഞ്ഞതുകൊണ്ട്‌ യാതൊരു പ്രയോജനവുമില്ല. കാരണം ആ ജല്‍പ്പനം പോലും ഏകാഗ്രതയ്ക്കു വിഘ്നമായിത്തിരുന്നു. രാമ: സൃഷ്ടിയെന്നത്‌ സത്യമാണെങ്കില്‍ അതിന്റെ അവസാനം എന്നത്‌ സാദ്ധ്യതയേതുമില്ലാത്തതാണ്‌. ഇല്ലാത്തതിന്‌ നിലനില്‍പ്പുണ്ടാവുകയില്ല എന്നതും ഉള്ളതിനെ ഇല്ലാതാക്കാന്‍ സാധിക്കുകയില്ല എന്നതും അലംഘനീയ നിയമമാണ്‌. തപസ്സ്‌, ധ്യാനം, മറ്റുകര്‍മ്മപരിപാടികള്‍ എന്നിവകൊണ്ട്‌ ഇതവസാനിപ്പിക്കാനോ പ്രബുദ്ധത കൈവരിക്കാനോ സാദ്ധ്യമല്ല. സൃഷ്ടികള്‍ നിലനില്‍ക്കുന്നിടത്തോളം നിര്‍വ്വികല്‍പ്പ സമാധി (ചിന്താരഹിതമായ അവസ്ഥ) പോലും അസാദ്ധ്യം. നിവ്വികല്‍പ്പ സമാധി സാദ്ധ്യമാണെന്നുതന്നെ വിചാരിക്കുക. എന്നാല്‍ അതില്‍ നിന്നും ഒരുനിമിഷം പിന്തിരിഞ്ഞാല്‍ സൃഷ്ടികളും ബന്ധപ്പെട്ട ദുരിതാദികളും മനസ്സില്‍ ഉയരുകയായി. ചിന്തകളുടെ സഞ്ചാരമാണ്‌ സൃഷ്ടിധാരണയ്ക്കു കാരണം. 

"എള്ളില്‍ എണ്ണപോലെ, പൂവില്‍ സുഗന്ധം പോലെ വസ്തു അവബോധമെന്നത്‌ ബോധമുള്ളവന്റെ സഹജാവസ്ഥയാണ്‌." സ്വപ്നം കാണുന്നവന്റെ അനുഭവത്തില്‍ മാത്രമേ സ്വപ്നവസ്തുക്കള്‍ക്ക്‌ അസ്തിത്വമുള്ളു. അതുപോലെ വസ്തു അവബോധം അനുഭവിക്കുന്നവന്റെ ബോധത്തില്‍ മാത്രമേ നിലനില്‍ക്കുന്നുള്ളു. വിത്തിലുള്ള ചെടി സമയമാവുമ്പോള്‍ മുളപൊട്ടി പുറത്തുവരുമ്പോലെ ഈ സാധ്യത സൃഷ്ടിധാരണയായി പ്രകടിതമാവുന്നു. 
