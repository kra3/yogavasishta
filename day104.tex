\newpage
\section{ദിവസം 104}

\slokam{
ചിദണോരന്തരേ സന്തി സമഗ്രാനുഭവാണവ:\\
യഥാ മധുരസസ്യാന്ത: പുഷ്പപത്രഫലാശ്രിയ: (3/81/35)\\
}

രാജാവ്‌ തുടര്‍ന്നു: ആത്മജ്ഞാനത്തിന്റെ വെളിച്ചത്തിലാണ്‌ എല്ലാ അനുഭവങ്ങളും പ്രദീപ്തമാവുന്നത്‌.. എന്നാല്‍ ആത്മജ്ഞാനം സ്വപ്രകാശിതമാണ്‌.. സൂര്യന്‍ മുതലിങ്ങോട്ട്‌ എല്ലാ വെളിച്ചവും ഇല്ലാതായാല്‍ ഏതുവെളിച്ചംകൊണ്ടാണൊരുവന്‍ കാണുക (അറിയുക)? ആ ഉള്‍വെളിച്ചം മാത്രമുണ്ടാവും. ആ വെളിച്ചം പുറത്തുള്ള വസ്തുക്കളെ പ്രകാശിപ്പിക്കുന്നു എന്നു തോന്നുകയാണ്‌. മറ്റ്‌ എല്ലാ വെളിച്ചങ്ങളും അജ്ഞാനാന്ധകാരത്തില്‍ നിന്നും വിഭിന്നമല്ല. അവയെല്ലാം തിളങ്ങുന്നു എന്നു തോന്നുന്നു എന്നു മാത്രം. മൂടല്‍മഞ്ഞും മേഘവും തമ്മില്‍ യഥാര്‍ത്ഥത്തില്‍ യാതൊരു വ്യത്യാസവുമില്ലെങ്കിലും (രണ്ടും വസ്തുക്കളെ മറയ്ക്കുന്നു) കാഴ്ച്ചയില്‍ മേഘം വെളിച്ചത്തെ മറയ്ക്കുന്നതായും മൂടല്‍മഞ്ഞ്‌ വെളിച്ചത്തെ പ്രസരിപ്പിക്കുന്നതായും കാണുന്നു. എന്നാല്‍ ബോധമെന്ന ഉള്‍വെളിച്ചം അകത്തുള്ളതിനേയും പുറത്തുള്ളതിനേയും രാത്രിയോപകലോ എന്ന ഭേദമില്ലാതെ പ്രദീപ്തമാക്കുന്നു. അത്ഭുതമെന്നുപറയട്ടെ ഈ വെളിച്ചമാണ്‌ അവിദ്യയുടെ ഇരുട്ടിനെ നീക്കാതെതന്നെ അജ്ഞാനത്തിന്റെ ഫലങ്ങളെ പ്രകാശിപ്പിക്കുന്നത്‌. (അറിയാന്‍ ഇടയാക്കുന്നത്‌). രാത്രിപകലുകള്‍ എന്ന സങ്കല്‍പ്പത്തെക്കൊണ്ട്‌ എപ്പോഴും ജാജ്വല്യമാനമായ സൂര്യന്‍ തന്റെ പ്രാഭവം വെളിവാക്കുന്നു. അതുപോലെ ആത്മാവ്‌ സ്വരൂപം വെളിപ്പെടുത്തുന്നത്‌, ബോധം, അജ്ഞാനം എന്നി സങ്കല്‍പ്പങ്ങളെക്കൊണ്ടാണ്‌. .

"അണുമാത്രമായ ബോധമണ്ഡലത്തില്‍ എല്ലാ അനുഭവങ്ങളും കുടികൊള്ളുന്നു. ഒരു തേന്‍ തുള്ളിയില്‍ പൂക്കളുടേയും, കായ്കളുടേയും, ഇലകളുടെയും സൂക്ഷ്മസത്ത ഉള്‍ക്കൊണ്ടിട്ടുണ്ടല്ലോ." ആ ബോധത്തില്‍നിനാണ്‌ എല്ലാ  അനുഭവങ്ങളും ഉദ്ഭൂതമായി വികസ്വരമാവുന്നത്‌.. അനുഭവമെന്നത്‌ ഒരേയൊരു 'അനുഭവി' (ബോധം) തന്നെയാണ്‌.. ഒരനുഭവത്തെ എങ്ങിനെയെല്ലാം വിവരിച്ചാലും അവയെല്ലാം ബോധമണ്ഡലത്തിലെ അനുഭവം മാത്രമാണ്‌.. എല്ലാം അനന്താവബോധം മാത്രം. എല്ലാ കൈകാലുകളും ആ ബോധത്തിനധീനമാണെങ്കിലും അതീവ സൂക്ഷ്മമായതിനാല്‍ അതിന്‌ അവയവങ്ങളില്ല. ഇമവെട്ടുന്ന നേരംകൊണ്ട്‌ ഈ അനന്താവബോധം യുഗങ്ങളെ അനുഭവിച്ച്‌ തീര്‍ക്കുന്നു. ചെറിയൊരു സ്വപ്നത്തില്‍ ഒരാള്‍ യുവത്വവും ജരാനരയും, മരണവും അനുഭവിക്കുന്നതു പോലെയാണത്‌.. ബോധമണ്ഡലത്തില്‍ കാണപ്പെടുന്ന എല്ലാം ബോധം തന്നെയാണ്‌.. കല്ലില്‍കൊത്തിയുണ്ടാക്കിയ ശില്‍പ്പം കല്ലു തന്നെയാണല്ലോ. വന്മരത്തിന്റെ ഭാവി സാദ്ധ്യതകള്‍ എല്ലാം ഒരു ചേറുവിത്തില്‍ അന്തര്‍ലീനമായിരിക്കുന്നതുപോലെ അണുമാത്രയായ ബോധത്തില്‍ ഭൂത-ഭാവി വര്‍ത്തമാനങ്ങളോടു കൂടിയ വിശ്വം ഉള്‍ക്കൊണ്ടിരിക്കുന്നു. അതുകൊണ്ട്‌ ആത്മാവ്‌ സ്വയം കര്‍ത്താവോ ഭോക്താവോ അല്ല എന്നിരിക്കിലും എല്ലാറ്റിന്റേയും കര്‍ത്താവും ഭോക്താവും ബോധത്തില്‍നിന്നും വിഭിന്നമല്ല. ബോധാണുവില്‍ കര്‍മ്മം ചെയ്യുന്നവനും ഫലങ്ങള്‍ അനുഭവിക്കുന്നവനും സഹജമായി കുടികൊള്ളുന്നു.

ഈ ലോകം സൃഷ്ടിക്കപ്പെട്ടിട്ടേയില്ല. അതിനാല്‍ത്തന്നെ അപ്രത്യക്ഷമാവുകയുമില്ല. അതിനെ 'അയഥാര്‍ത്ഥ്യം' എന്ന് ആപേക്ഷികമായി പറയുന്നു. എങ്കിലും നിരുപാധികമായി നോക്കിയാല്‍ ലോകവും ബോധമണ്ഡലവും വിഭിന്നങ്ങളല്ല. 
