\section{ദിവസം 166}

\slokam{
മഹാനരകസാമ്രാജ്യേ മത്തദുഷ്കൃത വാരണാ:\\
ആശാശരശലാകാഢ്യാ ദുർജയാ ഹീന്ദ്രിയാരയ:(4/24/1)\\
}

വസിഷ്ഠൻ തുടർന്നു: രാമ: അതി ഭീകരമായ നരകമെന്ന സാമ്രാജ്യത്തിൽ ദുഷ്കര്‍മ്മങ്ങള്‍ മദമിളകിയ ആനകളെപ്പോലെ അലഞ്ഞു മേയുകയാണ്‌.. ഈ ക്രിയകൾക്കുത്തരവാദികളായ ഇന്ദ്രിയങ്ങളാകട്ടെ ഭോഗതൃഷ്ണയുടെ ഒഴിയാത്ത ആവനാഴികളാൽ സുസജ്ജവുമാണ്‌.. അതുകൊണ്ട് ഇന്ദ്രിയങ്ങളെ വെല്ലുക എളുപ്പമല്ല.

“ഈ നന്ദികെട്ട ഇന്ദ്രിയങ്ങൾ തങ്ങളുടെ വാസസ്ഥലമായ ദേഹത്തിനെത്തന്നെയും നശിപ്പിക്കുന്നു. എന്നാൽ വിവേകമുള്ളവന്‌ ജീവനെ അപായപ്പെടുത്താതെതന്നെ ഈ തൃഷ്ണകളെ നിയന്ത്രിക്കാൻ കഴിയും." ആനയ്ക്കിട്ട വിലങ്ങുപോലെ നിരുപദ്രവമായി, എന്നാൽ കാര്യക്ഷമമായിത്തന്നെ ഇതു സാദ്ധ്യമാണ്‌.. കല്ലും മണ്ണും കൊണ്ടു പടുത്തുയർത്തിയ നഗരം ഭരിക്കുന്ന ഒരു രാജാവിന്റെ ആനന്ദത്തേക്കാൾ എത്രയോ ഉയർന്നതാണ്‌ ഇന്ദ്രിയനിയന്ത്രണം വന്ന ഒരുവൻ അനുഭവിക്കുന്ന ആനന്ദാവസ്ഥ.! ഇന്ദ്രിയാസക്തികൾ ക്ഷീണിതമാവുന്ന മുറയ്ക്ക് ഒരുവന്റെ ബുദ്ധിക്ക് തെളിമയേറുന്നു. എന്നാൽ പരമസത്യം സാക്ഷാത്കരിക്കുമ്പോൾ മാത്രമേ എല്ല്ലാ ആസക്തികളും പരിപൂർണ്ണമായി ഇല്ലാതാവുകയുള്ളു. വിവേകശാലിയുടെ മനസ്സ് അയാളുടെ ഉത്തമഭൃത്യനും, നല്ലൊരുപദേഷ്ടാവും, ഇന്ദ്രിയങ്ങളുടെ സേനാപതിയും, ഉത്തമയായ ഭാര്യയും, സുരക്ഷയേകുന്ന പിതാവും, ഉത്തമസുഹൃത്തുമാണ്‌.. മനസ്സയാളെ ഉത്തമകർമ്മങ്ങളിലേയ്ക്കുന്മുഖനാക്കുന്നു.

രാമാ, സത്യത്തിലുറച്ചു നിന്ന് മനസ്സില്ലാത്ത ഒരവസ്ഥയുടെ സ്വാതന്ത്ര്യത്തോടെ ജീവിച്ചാലും. ദാമന്‍, വ്യാളന്‍, കടന്‍,  മുതലായ രാക്ഷസന്മാരെപ്പോലെ പെരുമാറാതിരിക്കുക. ഈ മൂവരുടെ കഥ ഞാനിനി പറഞ്ഞുതരാം. നരകത്തിൽ ശംഭരൻ എന്ന പേരുള്ള ഒരു പ്രബലനായ രാക്ഷസൻ ഉണ്ടായിരുന്നു. അയാൾ കൺകെട്ടുവിദ്യയിൽ നിപുണനായിരുന്നു. അയാള്‍ ഒരു മായികനഗരമുണ്ടാക്കി. അതിൽ നൂറു സൂര്യന്മാരും, നടക്കുകയും സംസാരിക്കുകയും ചെയ്യുന്ന സ്വർണ്ണനിർമ്മിതങ്ങളായ സത്വങ്ങളും, വിലപിടിച്ച കല്ലുകൾ കൊണ്ടുണ്ടാക്കിയ അരയന്നങ്ങളും, മഞ്ഞുകട്ടയുടെ തണുപ്പുള്ള തീയും ഉണ്ടായിരുന്നു. കൂടാതെ അയാൾക്ക് സ്വന്തമായി ആകാശചാരികളായ യക്ഷകിന്നരന്മാരും ഉണ്ടായിരുന്നു. സ്വർഗ്ഗവാസികളായ ദേവന്മാർ അയാളെ ഭയന്നുകഴിഞ്ഞു. അയാൾ ഉറങ്ങിക്കിടന്നപ്പോഴോ വിഇടുവിട്ടു ദൂരെപ്പോയിരുന്നപ്പോഴോ ഒക്കെ സൂത്രത്തിൽ ദേവന്മാർ അയാളുടെ സൈന്യത്തെ കൊന്നൊടുക്കി. കുപിതനായ അസുരൻ സ്വർഗ്ഗത്തെ ആക്രമിച്ചു കീഴടക്കി. ശംഭരന്റെ മായാവിദ്യയിൽ വിരണ്ടുപോയ ദേവന്മാർ ഓടിയൊളിച്ചു. അയാൾക്ക് അവരെ കണ്ടെത്താനായില്ല. എന്നാൽ അസുരന്റെ സേനാംഗങ്ങളെ തരംകിട്ടുമ്പോഴൊക്കെ അവർ കൊന്നുകളഞ്ഞു. തന്റെ സേനാംഗങ്ങളെ സംരക്ഷിക്കാൻ ശംഭരൻ മൂന്നു സത്വങ്ങളെ ഉണ്ടാക്കി. ദാമന്‍, വ്യാളന്‍, കടന്‍, എന്നീ രാക്ഷസരാണവര്‍. അവർക്ക് മുൻ ജന്മങ്ങൾ ഒന്നുമുണ്ടായിരുന്നില്ല. അതുകൊണ്ട് അവരുടെ മനസ്സിൽ യാതൊരു ധാരണകളും വാസനകളായി അവശേഷിച്ചിരുന്നില്ല. അവർക്കു ഭയമോ, സംശയങ്ങളോ, മറ്റു ചിത്തവൃത്തികളോ ഉണ്ടായിരുന്നില്ല. അവർ ശത്രുക്കളിൽ നിന്നും ഓടിയകന്നില്ല, കാരണം അവർക്കു മരണഭയം ഉണ്ടായിരുന്നില്ല. യുദ്ധം, ജയം, പരാജയം എന്നീ വാക്കുകളുടെ അർത്ഥവും അവർക്കറിയില്ലായിരുന്നു. അവർ സ്വതന്ത്ര ജീവനുകൾ ആയിരുന്നില്ല. ശംഭരന്റെ ആജ്ഞാനുവർത്തികളായ വെറും യന്ത്രമനുഷ്യരായിരുന്നു അവർ. അവരുടെ പെരുമാറ്റം, എല്ലാ വാസനകളും ഉപാധികളും ഒഴിഞ്ഞവരുടേതുപോലെയായിരുന്നു. എന്നാൽ അവർക്ക് സത്യസാക്ഷാത്കാരം ഉണ്ടായിരുന്നില്ലല്ലോ. തന്റെ സൈന്യത്തെ രക്ഷിക്കാൻ അജയ്യരായ ഈ മൂന്നുപേരുള്ളതിൽ ശംഭരൻ ആഹ്ളാദിച്ചു. 

