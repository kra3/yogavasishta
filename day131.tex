\newpage
\section{ദിവസം 131}

\slokam{
സര്‍വ്വം സര്‍വ്വഗതം ശാന്തം ബ്രഹ്മ സംപദ്ധ്യതേ തദാ\\
അസങ്കല്‍പ്പനശസ്ത്രേണ ച്ഛിന്നം ചിത്തം ഗതം യദാ  (3/111/15)\\
}

വസിഷ്ഠന്‍ തുടര്‍ന്നു: ഈ മനസ്സെന്ന ഭൂതത്തെ ജയിക്കാനുള്ള മാര്‍ഗ്ഗം സ്വപ്രയത്നത്തിന്റെ, അല്ലെങ്കില്‍ ആത്മജ്ഞാനത്തിന്റെ സഹായത്താല്‍ സുഖാനുഭവങ്ങള്‍ക്കായുള്ള അതിന്റെ ആര്‍ത്തിപൂണ്ട  ആശകള്‍ ഉപേക്ഷിക്കുക എന്നതാണ്‌.. വാസ്തവത്തില്‍ ഇതിന്‌ പ്രത്യേകിച്ച്‌ കഠിനാദ്ധ്വാനം ഒന്നും ആവശ്യമില്ല. ചെറിയ കുട്ടികളുടെ ശ്രദ്ധ എളുപ്പത്തില്‍ തിരിച്ചുവിടുന്നതുപോലെ, ഉചിതമായ ഭാവമുണ്ടാക്കിയെടുക്കുന്നതിലൂടെ, നമുക്കും ഇതു സാധിക്കാം. അത്യാര്‍ത്തികളെ ഉപേക്ഷിക്കാന്‍ കഴിയാത്തവന്‍ നശിക്കുന്നു; കാരണം അതിലൂടെ മാത്രമേ ഒരുവന്‍ നന്മയിലേയ്ക്കു നയിക്കപ്പെടുകയുള്ളു. തീവ്രമായ സാധനയിലൂടെ മനസ്സിനെ ജയിക്കാം. അങ്ങിനെ ക്ലേശലേശംകൂടാതെ വ്യക്തിബോധത്തിന്‌ അനന്താവബോധത്തില്‍ ആമഗ്നമാവാം. അപ്പോള്‍ വ്യക്തിത്വത്തിന്റെ ചട്ടക്കൂടെന്ന പരിമിതി ഇല്ലാതായിക്കഴിഞ്ഞു. ഇത്‌ തീര്‍ച്ചയായും എളുപ്പത്തില്‍ സാധിക്കാവുന്നതാണ്‌.. ഇതിനു കഴിയാത്തവന്‍ മനുഷ്യരൂപത്തിലുള്ള കഴുകന്മാരത്രേ.

മുക്തിപദപ്രാപ്തിക്ക്‌ മനോനിയന്ത്രണമല്ലാതെ മറ്റ്‌ മാര്‍ഗ്ഗങ്ങള്‍ ഒന്നുമില്ല. അതായത്‌ മനസ്സിന്റെ ആര്‍ത്തികളെ ഉപേക്ഷിക്കാന്‍ തീവ്രപരിശ്രമം നടത്തുക എന്നതു തന്നെ. മനസ്സിനെ 'കൊല്ലാന്‍' സംശയലേശമന്യേ ഒരുറച്ച തീരുമാനമെടുക്കുക. ഇതു സാധിച്ചില്ലെങ്കില്‍ ഗുരുനാഥന്റെ നിദ്ദേശങ്ങള്‍ ക്കും, വേദഗ്രന്ഥപഠനത്തിനും മന്ത്രജപത്തിനുമെല്ലാം തൃണമൂല്യം മാത്രമേയുള്ളു എന്നറിയുക. "അസങ്കല്‍പ്പം എന്ന ആയുധംകൊണ്ട്‌ മനസ്സിന്റെ വേരറുത്താല്‍ മാത്രമേ ഒരുവന്‌ സര്‍വ്വവ്യാപിയും പരമശാന്തവുമായ പരബ്രഹ്മപദം പ്രാപിക്കുവാന്‍ കഴിയൂ." തെറ്റായ ധാരണകളുടെയും ദു;ഖത്തിന്റെയും സന്തതിയാണ്‌ സങ്കല്‍പ്പം, അല്ലെങ്കില്‍ ഭാവന. ഇത്‌ ആത്മജ്ഞാനംകൊണ്ട്‌ ഇല്ലാതാക്കിയാല്‍പ്പിന്നെ പരമശാന്തിയായി. എന്തുകൊണ്ടാണിത്‌ വളരെ ക്ലേശകരമായി തോന്നുന്നത്‌? മന്ദബുദ്ധികളാല്‍ ഉണ്ടാക്കപ്പെട്ട വിധി, നിയോഗം, ദൈവങ്ങള്‍ തുടങ്ങിയ പരിമിത സങ്കല്‍പ്പങ്ങളെ ഉപേക്ഷിക്കൂ എന്നിട്ട്‌ സ്വപ്രയത്നം കൊണ്ട്‌, ആത്മജ്ഞാനംകൊണ്ട്‌, മനസ്സിനെ 'അമനസ്സ്‌' ആക്കൂ. അനന്താവബോധത്തില്‍ വ്യക്തിബോധം മുങ്ങി വിലയനം  പ്രാപിച്ചപോലെ  ഇല്ലാതാവട്ടെ. എന്നിട്ട്‌ എല്ലാത്തിനും അപ്പുറം പോവുക.

പരംപൊരുളുമായി താദാത്മ്യം പ്രാപിച്ച നിന്റെ മേധാശക്തി, അനശ്വരമായ ആത്മാവില്‍ വിലീനമാവട്ടെ. അങ്ങിനെ മനസ്സടക്കി എല്ലാ വികല്‍പ്പങ്ങളുമൊഴിഞ്ഞ ഒരുവന്‌ മൂന്നുലോകങ്ങളെ കീഴടക്കുക എന്നതുപോലും പാഴ്വേലയായിത്തോന്നും. ഇതിന്‌ വേദപരിജ്ഞാനമോ, സമൂഹത്തിലെ ഉയര്‍ച്ച താഴ്ചകളോ പ്രശ്നമല്ല. ആത്മജ്ഞാനം മാത്രമാണു കാര്യം. എന്താണിതിനു പ്രയാസം? ഇതൊരാള്‍ക്ക്‌ പ്രയാസമെന്നു തോന്നിയാല്‍പ്പിന്നെ  എങ്ങിനെയാണീ ലോകത്ത്‌ അത്മജ്ഞാനമില്ലാതെ നിഷ്പ്രയാസം ജീവിക്കുക? അതും പ്രയാസമേറിയതുതന്നെയല്ലേ? ഒരിക്കലും മരണമില്ലാത്ത ആത്മാവിനെ അറിഞ്ഞവന്‌ മരണത്തെ പേടിയില്ല. ബന്ധുമിത്രാദികളുടെ വേര്‍പാട്‌ അവനെ ബാധിക്കുന്നില്ല. 'ഇതു ഞാന്‍', 'ഇതെന്റേത്‌', എന്നീ ചിന്തകളാണ്‌ മനസ്സ്‌.. ഇവയെ മാറ്റിയാല്‍ പിന്നെ മനസ്സില്ല. അത്‌ ഭയരഹിതമായ അവസ്ഥയാണ്‌.. വാള്‍ മുതലായ ആയുധങ്ങള്‍ ഭയജനകമാണ്‌.. എന്നാല്‍ അഹംകാരം നശിപ്പിക്കുന്ന വിജ്ഞാനമെന്ന ആയുധം അഭയമാണ്‌ നല്‍കുന്നത്‌.
