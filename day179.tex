\section{ദിവസം 179}

\slokam{
നാനന്ദം ന നിരാനന്ദം ന ചലം നാചലം സ്ഥിരം\\
ന സന്നാസന്ന ചൈതേഷാം മദ്ധ്യം ജ്ഞാനിമനോ വിദു: (4/38/10)\\
}

വസിഷ്ഠൻ തുടർന്നു: 'ഞാൻ ഇതു ചെയ്യുന്നു' എന്ന് തോന്നുന്ന കർത്തൃത്വഭാവം സന്തോഷത്തെയും ദു:ഖത്തെയും ഉണ്ടാക്കുന്നവയോ യോഗികൾക്കുമാത്രം കിട്ടുന്ന പ്രശാന്തതയെ നല്കുന്നതോ ആയേക്കാം. എന്നാൽ ജ്ഞാനിക്ക് അത് വെറും സാങ്കല്‍പ്പീകമായ ഒന്നത്രേ. അജ്ഞാനിയെ സംബന്ധിച്ചിടത്തോളം അത് സത്യമാണെന്നാണ്‌ അനുഭവം. എന്നാൽ എന്താണീ തോന്നലിന്റെയെല്ലാം ഉറവിടം? എന്തെങ്കിലും നേടണമെന്ന ആഗ്രഹവുമായി മനസ്സ് അതിനായി പരിശ്രമം നടത്തുമ്പോഴാണ്‌ ഈ ധാരണ ഉടലെടുക്കുന്നത്. ആ കർമ്മത്തിന്റെ ഫലഭോക്താവായി മനസ്സ് സ്വയം അവരോധിക്കുന്നു. കർമ്മഫലങ്ങൾ അനുഭവിക്കേ 'ഞാൻ ഇതാസ്വദിക്കുന്നു' എന്ന ധാരണ ഉണ്ടാവുന്നു. ഈ രണ്ടു ധാരണകളും (കർത്തൃത്വം, ഭോക്തൃത്വം) ഒരേ നാണയത്തിന്റെ രണ്ടു വശങ്ങളാണ്‌.. കർമ്മത്തിലേർപ്പെട്ടാലുമില്ലെങ്കിലും, സ്വർഗ്ഗത്തിലാണെങ്കിലും നരകത്തിലാണെങ്കിലും, മാനസീകോപാധികൾ എന്തുതന്നെയായിരുന്നാലും അതെല്ലാം അനുഭവവേദ്യമാവുന്നത് മനസ്സിലത്രേ. അതിനാൽ അജ്ഞാനിയിൽ, ഉപാധികളാൽ പരിമിതപ്പെട്ടവനിൽ, 'ഞാൻ ഇതു ചെയ്യുന്നു' എന്ന ധാരണ എന്തെങ്കിലും ചെയ്യുമ്പോഴും ചെയ്യാതിരിക്കുമ്പോഴും സഹജമാണ്‌.. എന്നാൽ ഉപാധികളുടെ പരിമിതികളേതുമില്ലാത്ത, പ്രബുദ്ധനും, ജ്ഞാനിയുമായ ഒരുവനിൽ അത്തരം ധാരണകൾ ഉണ്ടാവുന്നതേയില്ല.

ഇതിനെക്കുറിച്ചുള്ള സത്യം  അറിവാവുമ്പോൾ ഉപാധികളുടെ പ്രാഭവം കുറഞ്ഞുവരുന്നു. അതുകൊണ്ട് വിവേകശാലിയായ ഒരുവൻ സജീവമായി പ്രവർത്തനങ്ങളിൽ ഏർപ്പെട്ടിരിക്കുമ്പോഴും കർമ്മഫലങ്ങളിൽ താൽപ്പര്യമൊന്നും ഇല്ലാത്തവനായി ഇരിക്കുന്നു. അയാൾ തന്റെ ജീവിതത്തിലൂടെ കർമ്മങ്ങളെ പ്രവര്‍ത്തികളാക്കുന്നുവെങ്കിലും കർമ്മഫലങ്ങളിൽ അയാൾക്ക് താൽപ്പര്യം ഒന്നുമില്ല. വന്നുചേരുന്ന കർമ്മഫലങ്ങൾ തന്റെ സ്വരൂപത്തിൽ നിന്നും, ആത്മാവിൽ നിന്നും വിഭിന്നമായി അയാൾ കാണുന്നുമില്ല. എന്നാൽ വൈവിദ്ധ്യമാര്‍ന്ന മാനസീകാവസ്ഥകളിൽ മുഴുകി മുങ്ങിയിരിക്കുന്നവരുടെ ഭാവം ഇതല്ല. മനസ്സ് എന്തുചെയ്യുന്നുവോ അതാണ്‌ കർമ്മം. അതിനാൽ മനസ്സാണ്‌ കർമ്മങ്ങളുടെ കർത്താവ്. ശരീരമല്ല. മനസ്സാണ്‌ ഈ ലോകമെന്ന കാഴ്ച്ച. അവിടെയാണതുയർന്നത്. അവിടെയാണത് നിലകൊള്ളുന്നത്. എന്നാൽ വിഷയങ്ങളും അനുഭവവേദ്യമായ മനസ്സും പ്രശാന്തമാവുമ്പോൾ ബോധം മാത്രം അവശേഷിക്കുന്നു.

"പ്രബുദ്ധന്റെ മനസ്സ് ആനന്ദത്താൽ നിറഞ്ഞതോ ആനന്ദലേശമില്ലാത്തതോ ആയ അവസ്ഥയിൽ അല്ലെന്നാണ്‌ വിജ്ഞാനികൾ പറയുന്നത്. ചലമോ അചലമോ അല്ല അത്. സത്തോ  അസത്തോ അല്ല അത്. ഈ രണ്ടിനും ഇടയ്ക്കുള്ള അവസ്ഥയിലത്രേ ആ മനസ്സ്." അയാളുടെ അപരിമേയമായ അവബോധം ആനന്ദപൂർവ്വം ഈ പ്രത്യക്ഷലോകത്ത് തന്റെ ഭാഗം ഒരു നാടകത്തിലെന്നപോലെ ഭംഗിയായി അഭിനയിക്കുന്നു. മാനസീകോപാധികൾ അജ്ഞാനിയിലാണല്ലോ ഉള്ളത്. അവയാണ്‌ കർമ്മങ്ങളുടെയും തൽഫലങ്ങളുടെയും ഗതിവിഗതികൾ നിശ്ചയിക്കുന്നത്. ഉപാധികളേതുമില്ലാത്തതുകൊണ്ട് ജ്ഞാനി എന്നും ആനന്ദത്തിലാണ്‌... അവന്റെ കർമ്മങ്ങൾ, ‘കർമ്മങ്ങൾ’അല്ല. അവയിലൂടെ അവൻ പാപപുണ്യങ്ങൾ ആർജ്ജിക്കാനിടയാകുന്നില്ല. അവൻ ഒരു ശിശുവിനേപ്പോലെ പെരുമാറുന്നു. ദു:ഖിതനെപ്പോലെ കാണപ്പെടുമ്പോഴും അവൻ ദു:ഖത്തിനടിമയല്ല. അവൻ ഈ കാണപ്പെടുന്ന ലോകത്തോട് യാതൊരുവിധ ആസക്തിയും ഇല്ലാത്തവനാണ്‌. മനസ്സിന്റെയോ ഇന്ദ്രിയങ്ങളുടെയോ പ്രവർത്തനങ്ങളിലും അവന്‌ ഇഷ്ടാനിഷ്ടങ്ങളില്ല. മുക്തിയെന്നോ ബന്ധനമെന്നോ ഉള്ള ധാരണകൾ അവൻ വെച്ചുപുലർത്തുന്നില്ല. അവൻ ആത്മാവിനെ കാണുന്നു. ആത്മാവിനെ മാത്രം. 

