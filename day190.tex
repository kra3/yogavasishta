\section{ദിവസം 190}

\slokam{
ജ്ഞാനം ത്വമേവാസ്യ വിഭോ കൃപയോപദിശാധുനാ\\
കോ ഹി നാമ കുലേ ജാതം പുത്രം മൗർഖ്യേണ യോജയേത് (4/51/28)\\
}

വസിഷ്ഠൻ തുടർന്നു: അപ്പോൾ മുനി തന്റെ മുന്നിൽ ഒരു വലിയ കദംബവൃക്ഷം ഗാംഭീര്യത്തോടെ നില്‍ക്കുന്നതായിക്കണ്ടു. അതിന്റെ കയ്കൾ (ശാഖകൾ) തന്റെ പ്രിയപ്പെട്ട ആകാശത്തിന്റെ കണ്ണീരൊപ്പി (മഴത്തുള്ളികൾ) നിൽക്കുന്നതായിത്തോനി. സ്വർഗ്ഗത്തിനും ഭൂമിയ്ക്കുമിടക്കുള്ള ഇടം മുഴുവനും വ്യാപരിച്ച്, തന്റെ ആയിരം കരങ്ങൾ (ശാഖകൾ) വിരിച്ച് സൂര്യചന്ദ്രന്മാരാകുന്ന കണ്ണുകളുമായി ഭഗവാന്റെ വിശ്വരൂപം പോലെ  അതങ്ങിനെ വിരാജിക്കുകയാണ്‌.  ആകാശചാരികളായ ദിവ്യ മഹർഷിമാർക്കുമേൽ ഈ വൃക്ഷം പുഷ്പവൃഷ്ടി ചെയ്യുന്നു . ആ മരത്തിലധിവസിക്കുന്ന തേനീച്ചകളാണെങ്കിൽ മഹർഷിമാരെ പ്രകീർത്തിക്കുന്ന ഗാനാലാപനത്തിൽ മുഴുകിയുമിരിക്കുന്നു. (അതീവ മനോഹരമായ വർണ്ണനയാണീ മരത്തെക്കുറിച്ച് മൂലത്തിലുള്ളത്)

സ്വർഗ്ഗത്തിനും ഭൂമിക്കുമിടയ്ക്കുള്ള ഒരു തൂണുപോലെ നില്‍ക്കുന്ന ആ മരത്തിലേയ്ക്ക് മുനി വലിഞ്ഞു കയറി അതിന്റെ ഏറ്റവും മുകളിലൊരു ശാഖയിൽ ഇരിപ്പുറപ്പിച്ചു. അല്പനേരം അദ്ദേഹത്തിന്റെ ദൃഷ്ടി ചുറ്റുമുള്ള എല്ലായിടത്തും പതിച്ചു. വിശ്വപുരുഷന്റെ ദർശനം അദ്ദേഹത്തിനവിടെ ലഭിച്ചു. (ഈ ദർശനത്തെപറ്റിയും മൂലത്തിൽ മനോഹരമായ ഒരു വിവരണമുണ്ട്). കദംബ വൃക്ഷം തന്റെ ഇരിപ്പിടമാക്കിയതിനാൽ അദ്ദേഹത്തിന്‌ കദംബദാസുരൻ എന്ന പേരും വന്നു. മരക്കൊമ്പിലിരുന്ന് അദ്ദേഹം തന്റെ തപശ്ചര്യകൾ നടത്തി. വേദശാസ്ത്രോചിതങ്ങളായ കർമ്മങ്ങൾ ചെയ്താണദ്ദേഹത്തിനു പരിചയം. അങ്ങിനെയുള്ള കർമ്മങ്ങൾ തന്നെയാണദ്ദേഹം ഇവിടെയിരുന്നും അനുഷ്ടിച്ചത്. എന്നാൽ ഇപ്പോൾ മനസാ ആണെന്നുമാത്രം. മനസാ ഉള്ള തപശ്ചര്യകളും ഒരേപോലെ ഫലപ്രദമാകയാൽ മുനിയുടെ മനസ്സും ഹൃദയവും അതീവനിർമ്മലമായി. അദ്ദേഹത്തില്‍ ശുദ്ധജ്ഞാനം സാക്ഷാത്കരിച്ചു.

ഒരുദിവസം പുഷ്പങ്ങൾകൊണ്ടുള്ള വസ്ത്രവുമണിഞ്ഞ് ഒരപ്സരസ്സ് അദ്ദേഹത്തിനുമുന്നിൽ പ്രത്യക്ഷയായി. അതീവസുന്ദരി. അദ്ദേഹം ചോദിച്ചു: അല്ലയോ സുന്ദരീ, കാമദേവനെപ്പോലും മയക്കാൻപോന്ന സൗന്ദര്യമാർന്ന നീയാരാണ്‌?. അവൾ പറഞ്ഞു: ഞാൻ വനദേവതയാണ്‌.. ഈ ലോകത്തിൽ അങ്ങയെപ്പോലെയുള്ള ജ്ഞാനികളുടെ സാന്നിദ്ധ്യമാത്രേണ സാധിക്കാത്ത യാതൊരു കാര്യവുമില്ല. ഞാനിവിടെ വനത്തിലെ ഒരുൽസവത്തിനു പോയിരുന്നു. അവിടെ ഞാൻ അനേകം ദേവതമാരെ അവരുടെ കുഞ്ഞുങ്ങളുമായി കണ്ടു. എന്നാൽ ഞാൻ മാത്രം പുത്രഭാഗ്യമില്ലാത്തവളായിപ്പോയി. ഞാൻ ദു:ഖിതയാണ്‌.. എന്നാൽ അങ്ങീ വനത്തിലുള്ളപ്പോൾ ഞാനെന്തിനു ദു:ഖിക്കണം? എനിക്കൊരു പുത്രനെത്തരൂ. അതുസാധിച്ചില്ലെങ്കിൽ ഞാനെന്റെയീ ശരീരം ഭസ്മമാക്കും. മുനിയൊരു വള്ളിച്ചെടി പൊട്ടിച്ചെടുത്ത് അവൾക്കു കൊടുത്തിട്ടു പറഞ്ഞു: ഈ ചെടി ഒരുമാസത്തിനുള്ളിൽ പൂവിടും. നീയൊരു പുത്രനു ജന്മം നല്കുകയും ചെയ്യും. സന്തോഷമായി പോയാലും. വനദേവത നന്ദിയോടെ യാത്രയായി.

അവൾ പന്ത്രണ്ടുകൊല്ലം കഴിഞ്ഞ് അത്രയും പ്രായമുള്ള മകനുമായി മുനിയെക്കാണാൻ വന്നു. ഭഗവൻ, ഇതാ അങ്ങയുടെ പുത്രൻ. ഞാനവന്‌ എല്ലാത്തരം വിദ്യാഭ്യാസവും നൽകിയിട്ടുണ്ട്. “അങ്ങവന്‌ ആത്മജ്ഞാനം നൽകണമെന്ന് ഞാൻ അപേക്ഷിക്കുന്നു. തന്റെ മകനൊരു മൂഢനായിക്കാണാൻ ആരാണാഗ്രഹിക്കുക?” മാമുനി അതു സമ്മതിച്ചു. വനദേവത വന്നവഴിയേ  മടങ്ങിപ്പോയി. അന്നുമുതൽ മുനി ആത്മവിദ്യയിലെ എല്ലാ ശാഖകളും കുമാരനെ പഠിപ്പിക്കാനാരംഭിച്ചു. 
