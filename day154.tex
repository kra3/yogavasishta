\section{ദിവസം 154}

\slokam{
യോ ന ശാസ്ത്രേണ തപസാ ന ജ്ഞാനേനാപി വിദ്യയാ\\
വിനഷ്ടോ മേ മനോ മോഹ: ക്ഷീണോസൗ ദർശനേന വാം (4/14/31)\\
}

വസിഷ്ഠൻ തുടർന്നു: രാമ: ഭൃഗു മഹർഷിയും കാലദേവനായ യമനും സാമംഗാ നദിക്കരയിലേയ്ക്ക് പുറപ്പെട്ടു. അവർ ആകാശമാർഗ്ഗേ മന്ദാര പർവ്വതമുകളിലെത്തിയപ്പോൾ ഉത്തമരും പ്രബുദ്ധരുമായ മാമുനിമാർ വാഴുന്ന വനപ്രദേശങ്ങൾ കണ്ടു. ഗജവീരന്മാർ അവിടെ മേഞ്ഞു നടക്കുന്നു. അപ്സരസ്സുകൾ പ്രബുദ്ധരായ ഋഷിപുംഗവന്മാരുടെമേൽ കളിയായി പുഷ്പവൃഷ്ടി നടത്തുന്നു. വനത്തിൽ വാഴുന്ന സന്യാസികളേയും അവരവിടെ കണ്ടു. ഗ്രാമപ്രദേശങ്ങളും പട്ടണങ്ങളുമുള്ള ഒരിടത്ത് അവർ ഇറങ്ങി. താമസിയാതെ അവർ സാമംഗ നദിക്കരയിലെത്തിച്ചേർന്നു.

ഭൃഗുമുനി അവിടെ തന്റെ പുത്രനെ മറ്റൊരു ശരീരത്തില്‍ ജീവിക്കുന്നതായി കണ്ടു. തികച്ചും വേറെയൊരു രൂപഭാവവും സ്വഭാവവും. അദ്ദേഹത്തിന്റെ പ്രശാന്തമായ മനസ്സ് പ്രബുദ്ധതയിൽ നിമഗ്നമായിരിക്കുന്നു. പ്രപഞ്ചജീവികളെക്കുറിച്ചുള്ള തീവ്രധ്യാനത്തിൽ അവരുടെ നിയോഗങ്ങളെക്കുറിച്ചുള്ള ചിന്തയിൽ അദ്ദേഹം മുഴുകിയിരിക്കുന്നു. ചൈതന്യവാനായ ഈ യുവാവ് മനശ്ശാന്തിയുടെ പരമപദത്തിലെത്തി ചിന്തകളും പ്രതിചിന്തകളുമവസാനിച്ചവനായിരിക്കുന്നു. നിർമ്മലമായൊരു സ്ഫടികമെന്നപോലെ തനിക്കുചുറ്റുമുള്ള യാതൊന്നിനേയും പ്രതിഫലിപ്പിക്കാൻ താൽപ്പര്യപ്പെടാതെ അയാളിരിക്കുന്നു. ‘ഇതെനിയ്ക്കു നേടണം’ അല്ലെങ്കിൽ 'ഇതെനിയ്ക്കു വർജ്ജ്യമാണ്‌' എന്നും മറ്റുമുള്ള യാതൊരു ചിന്തകളും അവനെ അലട്ടുന്നില്ല.

യമരാജന്‍ യുവാവിനെ ചൂണ്ടിക്കാട്ടി ഭൃഗുവിനോടിങ്ങിനെ പറഞ്ഞു: ഇതാ അങ്ങയുടെ പുത്രൻ. ‘എഴുന്നേൽക്കുക’ എന്ന വാക്കുകേട്ട ശുക്രൻ കണ്ണുതുറന്നു. ഭാസുരപ്രഭയാർന്ന രണ്ടുപേരെ മുന്നിക്കണ്ട ശുക്രൻ അവരെ ഉപചാരപൂർവ്വം ഒരു പാറമേൽ സ്വീകരിച്ചിരുത്തി. സൗമ്യമധുരമായി അദ്ദേഹമവരോടു പറഞ്ഞു: അല്ലയോ ദിവ്യന്മാരേ നിങ്ങളുടെ ദർശനത്താൽ ഞാൻ അനുഗൃഹീതനായിരിക്കുന്നു. “നിങ്ങളുടെ ദർശനം കിട്ടിയപ്പോൾത്തന്നെ എന്നിലെ മോഹങ്ങളെല്ലാം നശിച്ചിരിക്കുന്നു. ഈ മോഹവിഭ്രാന്തികൾ ഇല്ലാതാക്കാൻ വേദപഠനംകൊണ്ടോ തപശ്ചര്യകൾ കൊണ്ടോ ജ്ഞാനംകൊണ്ടോ സാദ്ധ്യമല്ല തന്നെ. അമൃതിന്റെ മഴപോലും മഹാത്മാക്കളുമായുള്ള സത്സംഗത്തിനോളം അനുഗ്രഹപ്രദമല്ല. നിങ്ങളുടെ പാദസ്പർശമേറ്റ മണ്ണുപോലും ദിവ്യമത്രേ.

ഭൃഗുമഹർഷി പറഞ്ഞു: നീ പ്രബുദ്ധനാണല്ലോ. അജ്ഞാനിയല്ലാത്തതിനാൽ നിനക്ക് എല്ലാം ഓർമ്മിക്കുവാനാകും. ഉടനേ തന്നെ ശുക്രന്‌ തന്റെ പൂർവ്വസ്ഥിതിയെപ്പറ്റി സ്മരണയുണ്ടായി. കണ്ണടച്ച് ധ്യാനനിമഗ്നനായി അതദ്ദേഹം ഉറപ്പുവരുത്തുകയും ചെയ്തു.

ശുക്രൻ പറഞ്ഞു: നോക്കൂ, ഞാൻ അനവധി രൂപങ്ങളെടുത്തു . അവയിലൂടെ അനേകം സുഖദു:ഖാനുഭവങ്ങളിലൂടെ കടന്നുപോയി. ജ്ഞാനവും മോഹവും അനുഭവിച്ചു. ഞാനൊരു ദുഷ്ടരാജാവായിരുന്നു. അത്യാഗ്രഹിയായ കച്ചവടക്കാരനായിരുന്നു. അലഞ്ഞുതിരിഞ്ഞുനടക്കുന്ന ഒരവധൂതനുമായിരുന്നു ഞാന്‍. ഞാനനുഭവിക്കാത്ത സുഖങ്ങളൊന്നുമില്ല. ഞാൻ ചെയ്യാത്ത കർമ്മങ്ങളില്ല. സന്തോഷവും ദു;ഖവും ഞാനനുഭവിക്കാത്തതായി വേറെയില്ല. ഇപ്പൊൾ ഞാനൊന്നിനും വേണ്ടി ആഗ്രഹിക്കുന്നില്ല. പ്രകൃതി അതിന്റെ വഴിക്കു നീങ്ങട്ടെ. പിതാവേ വന്നാലും; എന്റെ പഴയ ശരീരം വരണ്ടുണങ്ങിക്കിടക്കുന്നയിടത്തേയ്ക്ക് നമുക്ക് പോകാം.

