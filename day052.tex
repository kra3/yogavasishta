\newpage
\section{ദിവസം 052}

\slokam{
വർജയിത്വാ ജ്ഞവിജ്ഞാനം ജഗച്ഛബ്ദാർത്ഥ ഭാജനം\\
ജഗദ്ബ്രഹ്മസ്വശബ്ദാനാമർത്ഥേ നാസ്ത്യേവ ഭിന്നതാ (3/15/10)\\
}

വസിഷ്ഠന്‍ തുടര്‍ന്നു: രാമ: സ്വപ്നസമയത്ത്‌ അപ്പോഴത്തെ കാഴ്ച്ചകളെല്ലാം യാഥാര്‍ത്ഥ്യമായിരുന്നു എങ്കിലും ഉണര്‍ന്നിരിക്കുന്ന അവസ്ഥയില്‍ , നാം കണ്ട സ്വപ്നത്തിലെ വസ്തുക്കള്‍ക്ക്‌ ഉണ്മയില്ല എന്നു നാം അറിയുന്നു. അതുപോലെ ഈ ലോകം പദാര്‍ത്ഥസഞ്ചയമെന്നു കാണപ്പെടുന്നുവെങ്കിലും സത്യത്തില്‍ ഇതു ബോധം മാത്രമാണ്‌. മരീചികയില്‍ സൂക്ഷ്മമായോ താല്‍ക്കാലികമായോ ഒരരുവി ഇല്ല തന്നെ. അതുപോലെ ഒരു യാഥാര്‍ത്ഥ്യലോകമെന്നത്‌, ഇല്ല. ഉള്ളത്‌ ശുദ്ധബോധം മാത്രം.

"അജ്ഞാനത്തിലധിഷ്ഠിതമായ അറിവാണ്‌ ലോകമെന്ന ധാരണയില്‍ കടിച്ചുതൂങ്ങി നില്‍ക്കുന്നത്‌. സത്യത്തില്‍ , ലോകം, ബ്രഹ്മം, അനന്തത, ആത്മന്‍ എന്നീ വാക്കുകള്‍ എല്ലാം സൂചിപ്പിക്കുന്നത്‌ ഒന്നിനെയാണ്‌."  ഉണര്‍ന്നിരിക്കുന്നവന്‌ സ്വപ്നത്തില്‍ക്കാണപ്പെട്ട നഗരം എത്ര സത്യമാണോ അത്ര സത്യമാണീ ലോകം. ലോകം, വിശ്വാവബോധം എന്നീ വാക്കുകള്‍ പര്യായങ്ങളത്രേ. ഈ ആശയങ്ങള്‍ വ്യക്തമാക്കാന്‍ ഞാന്‍ മണ്ഡപന്‍ എന്നൊരാളിന്റെ കഥ പറയാം ശ്രദ്ധിച്ചു കേട്ടാലും.

'രാമ: ഒരിക്കല്‍ ഒരിടത്ത്‌ പദ്മ എന്നു പേരായ ഒരു രാജാവുണ്ടായിരുന്നു. അദ്ദേഹം എല്ലാംകൊണ്ടും ഉത്തമനായിരുന്നു. സ്വഭാവമഹിമകൊണ്ട്‌ അദ്ദേഹം രാജവംശത്തിന്റെ കീര്‍ത്തി വര്‍ദ്ധിപ്പിച്ചു. സമുദ്രം അതിന്റെ അതിരുകളുടെ അധികാരത്തെ ബഹുമാനിക്കുന്നതുപോലെ അദ്ദേഹം ധാര്‍മ്മികമായ ആചാരങ്ങളെല്ലാം ആദരവോടെ അനുഷ്ഠിച്ചുപോന്നു. സൂര്യന്‍ ഇരുട്ടിനെ കീഴടക്കുന്നതുപോലെ അദ്ദേഹം തന്റെ ശത്രുക്കളെ വെന്നു. സമൂഹത്തിലെ തിന്മകളെ തീയില്‍ വീണു ചാരമാവുന്ന വൈക്കോലുപോലെ അദ്ദേഹം ഇല്ലായ്മചെയ്തു. ദേവന്മാര്‍ക്കു സ്വര്‍ഗ്ഗം എന്നതുപോലെ മഹാത്മാക്കള്‍ രാജാവിനെ ആശ്രയിച്ചു. അദ്ദേഹം നന്മയുടെ ഇരിപ്പിടമായിരുന്നു. കൊടുങ്കാറ്റിലാടുന്ന വള്ളിപോലെ യുദ്ധക്കളത്തില്‍ അദ്ദേഹം ശത്രുക്കളെ വിറപ്പിച്ചു. അദ്ദേഹം വിദ്വാനും കലാനിപുണനുമായിരുന്നു. ഭഗവാന്‍ നാരായണന്‌ അസാദ്ധ്യമായി ഒന്നുമില്ലാത്തതുപോലെ ഈ രാജാവിന്‌ എന്തു നേട്ടവും ക്ഷിപ്രസാദ്ധ്യമായിരുന്നു. രാജാവിന്റെ ഭാര്യ ലീല അതീവ സുന്ദരിയും സ്ത്രീകളില്‍ വെച്ച്‌ നൈപുണ്യമേറിയവളുമായിരുന്നു. നാരായണപ്രിയയായ ലക്ഷീദേവി ഭൂമിയില്‍ അവതരിച്ചതുപോലെ രാജ്ഞി കാണപ്പെട്ടു. സൌമ്യമായ വാക്കുകളും, മന്ദനടയും ഉള്ള അവളുടെ പ്രസാദമധുരമായ പുഞ്ചിരി നിലാവിന്റെ ശീതളിമപോലെയായിരുന്നു. പേലവമായ കയ്യുകളോടെ വെണ്‍നിറമുള്ള അവള്‍ തേനിന്റെ മധുരിമ തൂകി വിരാജിച്ചു. ഗംഗാജലം പോലെ സ്ഫടികശുദ്ധമായിരുന്നു അവളുടെ ദേഹം. സ്പര്‍ശനമാത്രയില്‍ ആനന്ദദായകമാണല്ലോ ഗംഗാജലം, അതുപോലെ ലീലയുടെ സ്പര്‍ശനം പോലും ആനന്ദാനുഭൂതിദായകമായിരുന്നു. തന്റെ ഭര്‍ത്താവില്‍ അതീവ ഭക്തയായ പദ്മയ്ക്ക്‌ അദ്ദേഹത്തിനെ എങ്ങിനെ പരിചരിക്കണം, സംപ്രീതനാക്കണം എന്നെല്ലാം അറിയാമായിരുന്നു. അവള്‍ രാജാവുമായി ഒന്നുചേര്‍ന്ന് അദ്ദേഹത്തിന്റെ സുഖദു:ഖങ്ങളെ പങ്കുവെച്ചു ജീവിച്ചുവന്നു. 

വാസ്തവത്തില്‍ അവള്‍ പദ്മ രാജാവിന്റെ സൂക്ഷ്മശരീരമായിരുന്നു. എന്നാല്‍ രാജാവിനു ക്രോധം വരുമ്പോള്‍ അവളില്‍ ഭയം പ്രതിഫലിച്ചു.
