\section{ദിവസം 191}

\slokam{
ജയതി ഗച്ഛതി വൽഗതി ജൃംഭതേ\\
സ്ഫുരതി ഭാതി ന ഭാതി ച ഭാസുര:\\
സുത മഹാമഹിമാ സ മഹീപതി:\\
പതിരപാംവി വാതരയാകുല: (4/52/29)\\
}

വസിഷ്ഠൻ തുടർന്നു: ഈ സമയത്ത് ഞാനാ മരത്തിനുമുകളിലൂടെ ആകാശ ഗമനത്തിലായിരുന്നതിനാൽ മഹർഷി തന്റെ മകനോട് പറഞ്ഞതു ഞാൻ കേട്ടു.

ദാസുരമുനി പറഞ്ഞു: ഇഹലോകത്തെപ്പറ്റി എനിക്കു പറയാനുള്ളത് എളുപ്പം നിനക്കു മനസ്സിലാക്കാൻ ഞാനൊരു കഥ പറയാം. ഒരിടത്ത് മൂന്നുലോകങ്ങളും കീഴടക്കാനുള്ളത്ര അതിശക്തിമാനായ ഒരു രാജാവുണ്ടായിരുന്നു. ഖൊത്തൻ എന്നാണദ്ദേഹത്തിനെ പേര്‌.. പ്രപഞ്ചശക്തികളെ നിയന്ത്രിക്കുന്ന ദേവതകളെല്ലാം അദ്ദേഹത്തിന്റെ ആജ്ഞയ്ക്കു കാത്തുനിന്നു. അദ്ദേഹം ചെയ്യുന്ന എണ്ണമറ്റ കര്‍മ്മങ്ങള്‍ സന്തോഷവും ദു:ഖവും ഒരുപോലെ പ്രദാനംചെയ്തു. അദ്ദേഹത്തിന്റെ ശൗര്യത്തെ വെല്ലുവിളിക്കാൻ ആർക്കുമാവില്ല. ആയുധങ്ങൾകൊണ്ടോ അഗ്നികൊണ്ടോ ഒന്നും അദ്ദേഹത്തോടെതിരിടുന്നത് ആകാശത്തെ കൈമുഷ്ടികൊണ്ട് പ്രഹരിക്കുംപോലെ  അസംബന്ധം. ഇന്ദ്രൻ, വിഷ്ണു, ശിവൻ, ഇവർക്കൊന്നും അദ്ദേഹത്തിന്റെയത്ര സാഹസീകത ഉണ്ടായിരുന്നില്ല. ഈ രാജാവിന്‌ മൂന്നു ശരീരങ്ങളായിരുന്നു - ഉത്തമം, മദ്ധ്യമം, അധമം. ഈ മൂന്നു ശരീരങ്ങൾ ലോകത്തെ മുഴുവനായി ഗ്രസിച്ചിരുന്നു. രാജാവ് ആകാശത്തിലാണധിവസിച്ചിരുന്നത്. അവിടെയദ്ദേഹം പതിന്നാലു വീഥികളും മൂന്നു വൃത്തഖണ്ഡങ്ങളുമുണ്ടാക്കി. നന്ദനോദ്യാനങ്ങൾ, കായികവിനോദങ്ങൾക്കായി പർവ്വതശൃംഘങ്ങൾ,വള്ളിച്ചെടികളും മുത്തുകളും നിറഞ്ഞ ഏഴു തടാകങ്ങൾ, എന്നിവ അവിടെയുണ്ടായിരുന്നു. അവിടെ രണ്ടു പ്രകാശഗോളങ്ങൾ. ഒരിക്കലും കെടാത്ത അവയിലൊന്നിൽ ചൂടും മറ്റേതിൽ തണുപ്പുമായിരുന്നു.ആ നഗരിയിൽ രാജാവ് അനേകം തരം ജീവികളെ സൃഷ്ടിച്ച് അവയ്ക്ക് ആവാസമേകി. അവയിൽ ചിലത് ഉയരത്തിൽ, ചിലത് താഴെ, മറ്റുള്ളവ മദ്ധ്യത്തിൽ നിവസിച്ചു, ചില ജീവികൾക്ക് ദീർഘായുസ്സായിരുന്നു. മറ്റുള്ളവയ്ക്ക് ക്ഷണികമായ ജീവിതമേ ഉണ്ടായിരുന്നുള്ളു. അവയെല്ലാം കറുത്ത തലമുടിയുള്ളവരായിരുന്നു. അവയില്‍  ഒൻപതു ദ്വാരങ്ങളിലൂടെ വായു സഞ്ചാരം നടത്തി. അവയ്ക്ക് അഞ്ചു ദീപങ്ങളും മൂന്നു സ്തംഭങ്ങളും ഉണ്ട്. വെളുത്ത മരക്കാലുകളിലാണവ നില്‍ക്കുന്നത്. മൃദുലമായ കളിമണ്ണു തേച്ചു മിനുക്കിയ ദേഹമാണവയ്ക്ക്. രാജാവിന്റെ മായികശക്തിയായ മായയാണ്‌ ഈ ജീവികളെയെല്ലാം സൃഷ്ടിച്ചത്.

രാജാവ് സ്വയം ലീലയാടുന്നത് ഭൂതപിശാചുക്കളുമായിച്ചേർന്നാണ്‌.. അവയ്ക്കാണെങ്കിൽ അന്വേഷണങ്ങളേയും പരിശോധനകളേയും ഭയമാണ്‌.. വിവിധ മന്ദിരങ്ങളെ (ദേഹങ്ങളെ) സംരക്ഷിക്കുകയെന്നതാണ്‌ അവയുടെ ജോലി. അദ്ദേഹം ഒരിടത്തുനിന്നും നീങ്ങാനാഗ്രഹിക്കുമ്പോൾ സ്വയം പുതിയൊരു നഗരത്തെ വിഭാവനം ചെയ്യുന്നു. അതിലേയ്ക്കു താമസം മാറ്റാൻ ചിന്തിക്കുന്നു. പഴയ നഗരമുപേക്ഷിച്ച്, ഭൂതങ്ങളുമായി കുതിച്ചു ചെന്ന് പുതിയ നഗരിയെ തന്റെ വാസസ്ഥലമാക്കുന്നു. ഒരു മായാജാലംപോലെയാണീ പുതുനഗരങ്ങളുണ്ടാവുന്നത്. പിന്നെ  ആ നഗരത്തെ സംഹരിക്കാനാലോചിക്കുമ്പോൾ സ്വയം അങ്ങിനെ സംഭവിക്കുകയാണ് .

ചിലപ്പോൾ “ഞാനിനി എന്തുചെയ്യും? ഞാൻ അജ്ഞാനിയാണ് , ഞാൻ നികൃഷ്ടൻ” എന്നിങ്ങനെ അയാള്‍ വിലപിക്കുന്നു. ചിലപ്പോളയാള്‍ സന്തോഷവാൻ. ചിലപ്പോൾ ശോചനീയമായ ദു:ഖത്തിനടിമ. “മകനേ, അങ്ങിനെ ജീവിച്ചും, കീഴടക്കിയും, നടന്നും, പുഷ്ടിപ്പെട്ടും, തിളങ്ങിയും, തിളങ്ങാതെയും പ്രത്യക്ഷലോകമെന്ന ഈ സാഗരത്തിൽ ഈ രാജാവങ്ങിനെ അമ്മാനമാടുന്ന പന്തുപോലെ മേലോട്ടും താഴോട്ടും സ്വയം കളിക്കുകയാണ്‌.. കളിപ്പിക്കപ്പെടുകയുമാണ്‌ ” 
