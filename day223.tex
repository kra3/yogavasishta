\section{ദിവസം 223}

\slokam{
വയം തു വക്തും മൂർഖാണാമജിതാത്മീയചേതസാം\\
ഭോഗകർദമമഗ്നാനാം ന വിദ്മോഽഭിമതം മതം (5/18/13)\\
}

വസിഷ്ഠൻ തുടർന്നു: മുക്തനായ ഋഷി സ്വയം ആകൃഷ്ടനല്ലെങ്കിലും ലോകത്തിലെ ഭൂത ഭാവി വർത്തമാനകാല സംഭവങ്ങളെ അദ്ദേഹം കൗതുകപൂർവ്വം കാണുന്നു. എപ്പോഴും സമുചിതമായ കർമ്മങ്ങളിലേർപ്പെട്ടുകൊണ്ട് സന്തോഷകരമായ ഒരു മദ്ധ്യമാർഗ്ഗം സ്വീകരിച്ച് യാതൊരുവിധത്തിലും കർമ്മബന്ധിതനാവാതെ കഴിയുന്നു. ഒരുവിധത്തിലുമുള്ള ഉപാധികളേയും ധാരണകളേയും സ്വാംശീകരിക്കാതെ അദ്ദേഹം സ്വതന്ത്രനായി വിഹരിക്കുന്നു. സമൃദ്ധിയുടെ പരമപദം ആദ്ദേഹത്തിനു സ്വന്തം. ഇഹലോകത്തിലെ സംഭവവികാസങ്ങൾ അദ്ദേഹത്തെ ബാധിക്കുന്നില്ല. അദ്ദേഹമവയിൽ പ്രത്യേകിച്ച് സന്തുഷ്ടനോ വ്യാകുലഹൃദയനോ അല്ല. കാലുഷ്യമേറിയ സന്ദർഭങ്ങളില്‍പ്പോലും അദ്ദേഹത്തിനു പക്ഷഭേദമില്ല. എന്നാൽ കൃപയും കാരുണ്യവും അദ്ദേഹത്തിനു സഹജമാണ്‌. താനും. പ്രത്യക്ഷലോകം അദ്ദേഹത്തെ ബാധിക്കുന്നതേയില്ല.

അദ്ദേഹത്തോട് എന്തെങ്കിലും ചോദിച്ചാൽ ഉചിതവും ലളിതവുമായ മറുപടി കിട്ടും. നാം ഒന്നും അദ്ദേഹത്തോടു സംസാരിച്ചില്ലെങ്കിൽ മൗനമാണദ്ദേഹത്തിനു സഹജം. അദ്ദേഹത്തിന്‌ ഒന്നിന്റേയും ആവശ്യമില്ല. ഒന്നിനോടും അദ്ദേഹത്തിനു വെറുപ്പുമില്ല. ലോകം അദ്ദേഹത്തെ വ്യാകുലപ്പെടുത്തുന്നതേയില്ല. എല്ലാവർക്കും നല്ലതു വരുത്തുന്ന കാര്യങ്ങളണദ്ദേഹം ചെയ്യുക. അദ്ദേഹത്തിന്റെ വാദമുഖങ്ങൾ തികച്ചും വിശ്വസനീയമായിരിക്കും. ഉചിതവും അനുചിതവും എന്തെന്ന് അദ്ദേഹത്തിനു നല്ലവണ്ണം അറിയാം. മറ്റുള്ളവർ എങ്ങിനെ കാര്യങ്ങൾ നോക്കിക്കാണുന്നു എന്നദ്ദേഹത്തിനറിവുണ്ട്. പരമസത്യത്തിൽ അടിയുറച്ച് പ്രശാന്തശീതള ഹൃദയനായി അദ്ദേഹം ലോകത്തെ സാകൂതം വീക്ഷിക്കുന്നു.

അങ്ങിനെയൊക്കെയാണ്‌ ജീവന്മുക്തന്റെ - ജീവിച്ചിരിക്കേ മുക്തനായ- ഒരുവന്റെ സ്ഥിതിവിശേഷങ്ങൾ. “മനോനിയന്ത്രണം വന്നിട്ടില്ലാത്ത മൂഢന്മാരുടേയും ഇന്ദ്രിയസുഖാസക്തിയുടെ ചെളിയിലാണ്ടുമുങ്ങിയവരുടേയും ചിന്താസരണികള്‍ വിവരിക്കാൻ നമുക്കാവില്ല.” അവർക്ക് ലൈംഗീകസുഖങ്ങളിലും ലൗകീകസമ്പത്തു വർദ്ധിപ്പിക്കുന്നതിലും മാത്രമാണു ശ്രദ്ധ. സുഖദു:ഖസമ്മിശ്രമായ അനുഭവങ്ങൾ പ്രദാനം ചെയ്യുന്ന യാഗകർമ്മങ്ങളുടേയും അനുഷ്ഠാനങ്ങളുടേയും പുറകിലുള്ള ചിന്തകളും വിവരിക്കാൻ നമുക്കാവില്ല.

രാമാ, അപരിമിതമായ സമ്യക്ദർശനത്തോടെ എല്ലാ പരിമിതികളേയും ഉറപ്പോടെ ഉപേക്ഷിച്ച് ജീവിതം നയിച്ചാലും. അകമേ യാതോരാഗ്രഹങ്ങളും പ്രത്യാശകളും ഇല്ലാതെ ബാഹ്യമായി ചെയ്യേണ്ട കർമ്മങ്ങൾ ഭംഗിയായി ചെയ്യുക. എല്ലാത്തിനേയും പരിശോധിച്ചറിഞ്ഞ് പരിമിതികളില്ലാത്തതിനെ മാത്രം കണ്ടെത്തുക. അനന്തതയിൽ സദാ ധ്യാനനിഷ്ഠനായി ഈ ലോകത്തിൽ ജീവിച്ചാലും.

അകമേ പ്രത്യാശകൾ വയ്ക്കാതെ എന്നാൽ പുറമേയ്ക്ക് പ്രത്യാശാനിര്‍ഭരതയോടെ ശാന്തഹൃദയത്തോടെ മറ്റുള്ളവർ കഴിയുന്നതുപോലെ തന്നെ ജീവിക്കുക. ‘ഞാൻ ഇതു ചെയ്യുന്നു’ എന്ന ധാരണകൾ ഒന്നും വെച്ചു പുലർത്താതെ വൈവിദ്ധ്യമാര്‍ന്ന പ്രവർത്തനങ്ങളില്‍ മുഴുകുക. അങ്ങിനെ അഹംകാരലേശം പോലുമില്ലാതെ ഈ ലോകത്തിൽ വാണാലും.

വാസ്തവത്തിൽ 'ബന്ധനം' എന്നതു സത്യമല്ല. അതിനാൽ 'മുക്തി' എന്നതും സത്യത്തില്‍ 'ഇല്ലാ'ത്ത ഒന്നത്രേ. ഈ പ്രത്യക്ഷലോകമോ, ജാലവിദ്യക്കാരന്റെ വെറുമൊരു മായക്കാഴ്ച്ചമാത്രം. അതിനു സത്തയില്ല. സർവ്വവ്യാപിയായ അനന്താത്മാവിനെ ബന്ധിക്കാൻ എന്തിനു കഴിയും? എങ്ങിനെയാണതിനു മുക്തിയുണ്ടാവുക? ഈ ചിന്താക്കുഴപ്പങ്ങൾക്കെല്ലാം കാരണം സത്യത്തെ അറിയാത്തതാണ്‌.. അവിദ്യയാണത്. ജ്ഞാനമുദിക്കുമ്പോൾ ഈ ചിന്താക്കുഴപ്പങ്ങൾ ഇല്ലാതാകുന്നു. കയറിൽ കാണപ്പെട്ട സങ്കൽപ്പജന്യമായ പാമ്പിനെപ്പോലെയാണത് പൊടുന്നനേ അപ്രത്യക്ഷമാവുന്നത് .  ഒരിക്കല്‍ കയറാണതെന്ന് അറിഞ്ഞാല്‍പ്പിന്നെ ഒരിക്കലും  'പാമ്പി'നെ കാണാന്‍ കഴിയുകയില്ല.
