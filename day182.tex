\section{ദിവസം 182}

\slokam{
കുതോ ജാതേയമിതി തേ രാമ മാസ്തു വിചാരണാ
ഇമാം കഥമഹം ഹന്മീത്യേഷാ തേസ്തു വിചാരണാ (4/41/32)
}

രാമൻ ചോദിച്ചു: ഭഗവൻ, വൈവിദ്ധ്യമുണ്ടാവണമെന്ന, സ്വയം പലതാവണമെന്ന, ഇച്ഛ അനന്താവബോധത്തിൽ എങ്ങിനെയാണ്‌ ഉണ്ടാകാനിടയായത്?

വസിഷ്ഠൻ പറഞ്ഞു: രാമാ, ഞാൻ പറഞ്ഞതിൽ വിരോധാഭാസമായി ഒന്നുമില്ലെന്നറിയുക.നിനക്ക് സ്വയം അനുഭവവേദ്യമാവുമ്പോള്‍ ഞാൻ പറഞ്ഞുതന്ന സത്യത്തിന്റെ സൗന്ദര്യം നിനക്കു വെളിവാകും. സൃഷ്ടിയെപ്പറ്റി ശാസ്ത്രഗ്രന്ഥങ്ങളിൽ പ്രതിപാദിക്കുന്നത് ശിഷ്യരെ പഠിപ്പിക്കാനായി ഉപയോഗിക്കുന്ന വെറും വാക്കുകളുടെ കസർത്താണ്‌. അവ നിന്റെ വീക്ഷണത്തെയും മനസ്സിനെയും നിറം പിടിപ്പിക്കാതിരിക്കട്ടെ. വാക്കുകൾകൊണ്ട് എന്താണുദ്ദേശിച്ചതെന്ന് പിടികിട്ടിക്കഴിഞ്ഞാൽ സ്വാഭാവികമായും നീ വാക്കുകളുമായുള്ള ബന്ധം ഉപേക്ഷിക്കുമെന്നു തീർച്ച.

അനന്താവബോധത്തിൽ അങ്ങിനെ മേൽപ്പറഞ്ഞപോലെ ഒരിച്ഛയോ ഭ്രമത്തിന്റെ മൂടുപടമോ ഇല്ല. എന്നാൽ അവ നിന്റെ മുന്നിൽ ലോകമായി പ്രത്യക്ഷപ്പെട്ട് നിലകൊള്ളുകയാണ്‌.. ഇതറിയണമെങ്കിൽ അജ്ഞാനം ഇല്ലാതെയാകണം. അതില്ലാതെയാകണമെങ്കിലോ നാമുപയോഗിച്ചതുപോലെയുള്ള വാക്കുകളും വിവരണങ്ങളും വേണം താനും. അജ്ഞത സ്വയം നശിക്കുന്നതിലൂടെയാണ്‌ സത്യജ്ഞാനത്തിന്റെ വെളിച്ചം പരക്കുന്നത്. ആയുധങ്ങൾ നശിക്കുന്നത് ആയുധങ്ങൾകൊണ്ടു തന്നെയാണ്‌.. ദേഹത്തു പുരണ്ട ചെളി മാറ്റിക്കളയാൻ മറ്റുതരം  ചെളികള്‍ തന്നെ വേണം. ഒരു വിഷത്തിന്റെ പ്രതിവിധി മറ്റൊരു വിഷം. ഒരു ശത്രുവിന്റെ എതിരാളി ഇനിയൊരു ശത്രു. അങ്ങിനെ സ്വയം നശിക്കുന്നതിൽ മായ അഭിരമിക്കുന്നു. ഈ മായയെപ്പറ്റി എപ്പോൾ അറിവാകുന്നുവോ അപ്പോൾത്തന്നെ അത് അപ്രത്യക്ഷമാവുന്നു. ഈ മായയാണ്‌ സത്യത്തെ മറച്ച് ഈ വൈവിദ്ധ്യങ്ങളെ ഉണ്ടാക്കുന്നത്. എന്നാൽ കഷ്ടം! അതിന്‌ സ്വന്തം അസ്തിത്വം എന്തെന്നറിയില്ല.

ഈ അജ്ഞാനത്തിന്റെ, മായയുടെ, സ്വഭാവത്തെപ്പറ്റി അന്വേഷിക്കുന്നതുവരെ അത് നമ്മെ ഭരിക്കുന്നു. എന്നാൽ അന്വേഷണം തുടങ്ങിയാലോ അത് ഇല്ലാതാവുകയും ചെയ്യുന്നു. മായ സത്യത്തിൽ 'ഉള്ള' വസ്തുവേയല്ല. എന്നാൽ ഈ സത്യം അനുഭവജ്ഞാനമാകുന്നതുവരെ നീ എന്റെ വാക്കുകൾ വിശ്വസിച്ചാലും. പരബ്രഹ്മം മാത്രമേ സത്തായുള്ളു എന്നറിഞ്ഞവൻ മുക്തനത്രേ. മറ്റെല്ലാ അറിവുകളും നമ്മെ അജ്ഞതയിൽ ബന്ധിച്ചു നിർത്തുന്നവയാണ്‌... ആത്മജ്ഞാനത്താലല്ലാതെ അജ്ഞത ഇല്ലാതാവുകയില്ല. ആത്മജ്ഞാനം ഉണ്ടാവാൻ ശാസ്ത്രഗ്രന്ഥങ്ങളുടെ ആഴത്തിലുള്ള പഠനം അനിവാര്യമാണ്‌... ഈ അജ്ഞതയുടെ ഉറവിടം എന്തായിരുന്നാലും അതും ആത്മാവിൽത്തന്നെ നിലകൊള്ളുന്നതു തന്നെയാണ്‌.. “അതുകൊണ്ട് രാമാ ‘എന്തുകൊണ്ടാണ്‌ ഈ അജ്ഞാനം ഉണ്ടായത്?’ എന്ന് അന്വേഷിക്കുന്നതിനു പകരം ‘ ഈ അജ്ഞാനത്തെ എനിക്കെങ്ങിനെ ഇല്ലായ്മ ചെയ്യാം’ എന്നന്വേഷിച്ചാലും”

ഈ മായ (അജ്ഞാനം) ഇല്ലാതായിക്കഴിയുമ്പോൾ അതെങ്ങിനെ ഉണ്ടായി എന്നു നിനക്കറിയുവാനാവും. അജ്ഞത സത്തായ ഒന്നായിരുന്നില്ല എന്നു നീ അറിയും. അജ്ഞത അവിവേകതയുടെ  ഒരു തലത്തിൽ മാത്രമേ ഉണ്ടാവുന്നുള്ളു. എത്ര പണ്ഡിതനായാലും, വീരനായാലും അജ്ഞാനത്തിന്റെ സ്വാധീനത്തിൽപ്പെടാതെ നിവൃത്തിയില്ല. എല്ലാ ദു:ഖങ്ങൾക്കും ഉറവിടമാണത്. അതിനെ വേരോടെ പിഴുതെറിയൂ, നശിപ്പിക്കൂ. 
