 
\section{ദിവസം 074}

\slokam{
തിജഗച്ചിദണാവന്തരസ്തി സ്വപ്നപുരം യഥാ\\
തസ്യാപ്യന്തഛിദണവസ്തേഷ്വപ്യേ കൈകശോ ജഗത് (3/52/20)\\
}

വസിഷ്ഠന്‍ തുടര്‍ന്നു: വിഥുരഥന്റെ മരണശേഷം നഗരത്തില്‍ യുദ്ധാനന്തര കെടുതികളും കലാപവും ഉണ്ടായി. സിന്ധുരാജാവ്‌ തന്റെ മകനായിരിക്കും ഇനി രാജാവ്‌ എന്നു പ്രസ്താവിച്ചു. അദ്ദേഹത്തിന്റെ പ്രജകള്‍ക്ക്‌ ആഹ്ലാദമായി. പെട്ടെന്നുതന്നെ മന്ത്രിമാര്‍ കിരീടധാരണത്തിനുള്ള ഒരുക്കങ്ങള്‍ തുടങ്ങി. പുതിയ ഭരണകൂടം നഗരത്തില്‍ പട്ടാളനിയമം കൊണ്ടുവന്ന് അവിടെ സമാധാനം പുന:സ്ഥാപിക്കുകയും ചെയ്തു. വിഥുരഥന്റെ വീഴ്ച്ചകണ്ട്‌ രണ്ടാമത്തെ ലീല ബോധരഹിതയായി കുഴഞ്ഞു വീണു. ആദ്യത്തെ ലീല സരസ്വതീ ദേവിയോടു പറഞ്ഞു: നോക്കൂ, എന്റെ ഭര്‍ത്താവ്‌ പ്രേതത്തെ ഉപേക്ഷിക്കാന്‍ പോവുന്നു.

സരസ്വതി പറഞ്ഞു: ഈ കൊടും യുദ്ധവും നാശവും, മരണവുമെല്ലാം ഒരു സ്വപ്നത്തിന്റെ യാഥാര്‍ഥ്യത മാത്രമുള്ളതാണെന്നറിയുക. അവിടെയൊരു സാമ്രാജ്യമില്ല, ഭൂമിയുമില്ല. ഇക്കാര്യങ്ങളെല്ലാം നടന്നത്‌ കുന്നിന്മുകളിലുള്ള വസിഷ്ഠന്‍ എന്നുപേരായ ആ മഹാത്മാവിന്റെ വീട്ടിലാണ്‌. ഈ കൊട്ടാരവും യുദ്ധക്കളവും ബാക്കിയുള്ള മറ്റുദൃശ്യങ്ങളുമെല്ലാം നിന്റെ അന്ത:പ്പുരത്തിന്റെ ഉള്ളറകളില്‍ത്തന്നെയല്ലാതെ  മറ്റ് എങ്ങുമല്ല. വാസ്തവത്തില്‍ ഈ വിശ്വം മുഴുവനും അവിടെയുണ്ട്‌. ആ മഹാത്മാവിന്റെ  ഗൃഹത്തിനുള്ളിലാണ്‌ പദ്മരാജാവിന്റെ ലോകം; ആ രാജാവിന്റെ കൊട്ടാരത്തിനുള്ളിലാണ്‌ നിങ്ങള്‍ ഇവിടെക്കണ്ട ലോകം. ഇതെല്ലാം വെറും ഭ്രമകല്‍പ്പന മാത്രം. ഉണ്മയായുള്ളത്‌ ആ സമ്പൂര്‍ണ്ണ സത്ത മാത്രം. അതിനെ ഉണ്ടാക്കാനോ നശിപ്പിക്കാനോ ആവില്ല. അജ്ഞാനികള്‍ വിശ്വപ്രപഞ്ചമായി കണക്കാക്കുന്നത്‌ ആ അനന്താവബോധത്തെയാണ്‌. "ഒരുവനില്‍ സ്വപ്നനഗരം മുഴുവനുമായി നിലനില്‍ക്കുന്നതുപോലെ ചെറിയൊരണുവില്‍ മൂന്നുലോകങ്ങളും സുസ്ഥിതമത്രേ. ആ ലോകങ്ങളിലും പരമാണുക്കളുണ്ട്‌; അവയിലെല്ലാം ത്രിലോകങ്ങളുമുണ്ട്‌."

മോഹാലസ്യപ്പെട്ടു വീണ ആ ലീല നിന്റെ ഭര്‍ത്താവ്‌ പദ്മ രാജാവിന്റെ ശരീരം കിടത്തിയിട്ടിരിക്കുന്ന ലോകത്താണിപ്പോള്‍ . ലീല ചോദിച്ചു: ദേവീ എങ്ങിനെയാണവള്‍ അവിടേയ്ക്ക്‌ നേരത്തേ എത്തിച്ചേര്‍ന്നത്‌? അവളോട്‌ ചുറ്റും കൂടിയവര്‍ എന്തൊക്കെയാണു പറയുന്നത്‌? സരസ്വതി പറഞ്ഞു: നിങ്ങള്‍ രണ്ടാളും രാജാവിന്റെ ആശാസങ്കല്‍പ്പസൃഷ്ടികളാണ്‌. അതുപോലെ രാജാവും ഈ ഞാനുമെല്ലാം സ്വപ്നവസ്തുക്കള്‍ മാത്രം. ഈ സത്യമറിയുന്നവന്‍ വിഷയവസ്തുക്കള്‍ക്കു വേണ്ടിയുള്ള അന്വേഷണം ഉപേക്ഷിക്കുന്നു. അനന്താവബോധത്തില്‍ നാം സങ്കല്‍പ്പത്തില്‍ പരസ്പരം സൃഷ്ടിക്കുകയാണ്‌. യുവതിയായ മറ്റേ ലീല നീ തന്നെയാണ്‌. അവള്‍ എന്നെ പൂജിച്ചത്‌ ഒരിക്കലും വിധവയാവാതിരിക്കാനുള്ള വരലബ്ധിക്കായാണ്‌. അതുകൊണ്ട്‌ വിഥുരഥരാജാവ്‌ അന്തരിക്കും മുന്‍പ്‌ ലീല കൊട്ടാരം വിട്ടുപോയി. പ്രിയപ്പെട്ടവളേ നിങ്ങളെല്ലാവരും വിശ്വാവബോധത്തിന്റെ വ്യക്തിഗത സത്വങ്ങളാണ്‌. ഞാനാണ്‌ വിശ്വാവബോധം. ഞാനാണിതെല്ലാം സാദ്ധ്യമാക്കുന്നത്‌.

