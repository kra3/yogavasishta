\section{ദിവസം 217}

\slokam{
ജഢത്വാന്നി:സ്വരൂപത്വാത്സർവദൈവ മൃതം മന:\\
മൃതേന മാര്യതേ ലോകശ്ചിത്രേയം മൗർഖ്യചക്രികാ (5/13/100)\\
}

വസിഷ്ഠൻ തുടർന്നു: “മനസ്സ് ജഢമാണ്‌.. അതിനുണ്മയില്ല. അതെന്നും മൃതമാണ്‌.. എന്നാലും ഈ മൃതസത്വം ലോകത്തിലെ ജീവജാലങ്ങളെ 'കൊന്നു'കൊണ്ടേയിരിക്കുന്നു. എത്ര വിചിത്രവും വിസ്മയകരവുമാണീ വിഡ്ഡിത്തം!” മനസ്സിനു സ്വന്തമായി ആത്മാവോ, ദേഹമോ, ആലംബമോ, രൂപമോ ഇല്ല. എങ്കിലും ഈ മനസ്സ് ലോകത്തിലെ എല്ലാമെല്ലാമാണ്‌.. തീർച്ചയായും ഇതൊരു സമസ്യതന്നെ. യഥാർത്ഥ്യത്തിൽ 'ഇല്ലാത്ത മനസ്സെന്നെ പീഢിപ്പിക്കുന്നു' എന്നു പറഞ്ഞാൽ അത് താമരപ്പൂവിതൾ കൊണ്ടെന്റെ തലയടിച്ചുടയ്ക്കുന്നു എന്നു പരാതിപറയും പോലെ അസംബന്ധമാണ്. ആരെങ്കിലും താൻ മനസ്സിന്റെ ഉപദ്രവത്താൽ വലയുന്നു എന്നു പറയുന്നത് ചാന്ദ്രരശ്മിയുടെ ‘ചൂടിൽ’ ചുട്ടുപൊള്ളൂന്നു എന്നു പറയുന്നതുപോലുള്ള വിഡ്ഡിത്തമാണ്‌ . മനസ്സ് ജഢവും, മൂകവും അന്ധവുമാണ്‌..

തന്റെ മുന്നിലുള്ള സൈന്യത്തെ മുഴുവൻ അടിച്ചമർത്തി ഇല്ലാതാക്കുന്ന വീരയോദ്ധാവിനെപ്പോലും ഈ ‘ഇല്ലാത്ത’ മനസ്സ് നശിപ്പിക്കുന്നു. ചിന്തകൾ കൊണ്ടു മെനഞ്ഞ ഈ മനസ്സെന്ന പ്രഹേളികയ്ക്ക് അസ്തിത്വമില്ല. അതിന്റെ അസ്തിത്വത്തെക്കുറിച്ചുള്ള അന്വേഷണം നടത്തിയാലോ, അത് അസത്തായിത്തീരുന്നു. മൂഢത്വവും അജ്ഞതയുമാണ്‌ ഈ ലോകത്തിലെ ദു:ഖങ്ങൾക്കു കാരണം. ഈ ലോകസൃഷ്ടിയുടെ കാരണവും മറ്റൊന്നല്ല. ഇതറിഞ്ഞിട്ടുപോലും ഈ അസത്തായ ‘അവസ്തു’ വിനെ ജീവജാലങ്ങൾ പരിപോഷിപ്പിക്കുന്നത് എത്ര വിചിത്രം!

ലോകമെന്ന ഈ വിസ്മയക്കാഴ്ച്ച ഒരു സേനാനായകന്റെ ദു:സ്വപ്നം പോലെയാണ്‌.. സങ്കൽപ്പത്തിലെ ശത്രുവിന്റെ കണ്ണിൽ നിന്നും കാണാച്ചങ്ങലകൾ ബഹിര്‍വമിച്ചു  വന്ന് തന്നെ കെട്ടിവരിയുന്നു എന്ന തോന്നല്‍ !. ശത്രുവിനെപ്പറ്റിയുള്ള സ്വചിന്തകൾ സൃഷ്ടിച്ച അദൃശ്യരായ ശത്രുസേനകളാൽ താൻ ഉപദ്രവിക്കപ്പെടുന്നു. ഇങ്ങിനെ അസ്തിത്വമില്ലാത്ത, മനസ്സുണ്ടാക്കിയ ലോകത്തെ മറ്റൊരു മനസ്സ് നശിപ്പിക്കുന്നു. അതും 'അവസ്തു' തന്നെയെന്നു പറയേണ്ടതില്ലല്ലോ.

ഈ മായക്കാഴ്ച്ചയാകുന്ന ലോകം മനസ്സുതന്നെയാണ്‌.. മനസ്സിന്റെ ഈ സത്യാവസ്ഥ അറിയാൻ കഴിവില്ലാത്തവർക്ക് ശാസ്ത്രം ഉദ്ഘോഷിക്കുന്ന പരമസത്യമെന്തെന്നറിയാൻ യോഗ്യതയില്ല. അങ്ങിനെയുള്ള ഒരാളില്‍ ഈ അതിസൂക്ഷ്മമായ ശാസ്ത്രവിഷയം പതിയുകയില്ല. ആ മനസ്സിന്‌ ലോകമെന്ന മായാക്കാഴ്ചയുടെ പരിമിതമായ അറിവു തന്നെ ധാരാളം. ആ മനസ്സിൽ നിറയെ ഭയമായിരിക്കും. മധുരമായ വീണാനാദവും ദീർഘനിദ്രയിലുള്ള ബന്ധുജനങ്ങൾപോലും അവരെ ഭയപ്പെടുത്തുന്നു. അന്യരുടെ ഒച്ചപ്പാടും കൃമികീടങ്ങളുടെ ദൃശ്യവും അവർക്കു ഭയമാണ്‌.. അജ്ഞാനിയായവൻ ഈ ലോകദൃശ്യത്തിൽ, സ്വന്തം മൂഢമനസ്സിൽ ആമഗ്നനാണ്‌.. അയാൾ തന്റെ തന്നെ ഹൃദയനിവാസിയായ മനസ്സിന്റെ താപത്താൽ ചുട്ടെരിക്കപ്പെടുന്നു. ചിലപ്പോൾ സന്തോഷാനുഭവങ്ങളെ പ്രദാനംചെയ്യുമെങ്കിലും ഈ മനസ്സ് കൊടിയവിഷം പോലെ വിനാശകാരിയാണ്‌.. മൂഢമനസ്സെന്ന ഭ്രമക്കാഴ്ച്ചയിലാകയാൽ അയാൾ സത്യമറിയുന്നില്ല. ഇതെത്ര വിചിതം! 

