\newpage
\section{ദിവസം 019}

\slokam{
അപഹസ്തിത സർവ്വാർത്ഥമനവസ്ഥിതിരാസ്ഥിതാ :\\
ഗൃഹീത്വോത്സൃജ്യ  ചാത്മാനം ഭവസ്ഥിതിരവിസ്ഥിതാ (1/30/8)\\
}

രാമന്‍ തുടര്‍ന്നു: ദു:ഖത്തിന്റെ കൂപത്തില്‍ പതിച്ചുപോയ ജീവജാലങ്ങളുടെ പരിതാപകരമായ അവസ്ഥ ആലോചിച്ച്‌ എന്റെയുള്ളില്‍ ആധി നിറഞ്ഞിരിക്കുന്നു. എന്റെ മനസ്സ്‌ ചിന്താക്കുഴപ്പത്താല്‍ പതറുന്നു; ഒരോ അടിയിലും ഞാന്‍ ഭയചകിതനാണ്‌.  "ഞാന്‍ എല്ലാം ഉപേക്ഷിച്ചിരിക്കുന്നു. എന്നാല്‍ ഞാന്‍ ആ പരമസത്യത്തില്‍ ഇനിയും സ്ഥിരപ്രതിഷ്ഠനായിട്ടില്ല. ഞാന്‍ കുറച്ച്‌ സ്വതന്ത്രനായും കുറച്ച്‌ ബദ്ധനായും കഴിയുന്നു."

മുറിച്ചുകളഞ്ഞുവെങ്കിലും വേരറ്റിട്ടില്ലാത്ത വൃക്ഷം പോലെയാണു ഞാന്‍. എനിക്കു മനസ്സിനെ സമ്പൂര്‍ണ്ണമായി നിയന്ത്രിക്കണമെന്നുണ്ട്‌. പക്ഷേ എനിയ്ക്കതു പ്രാവര്‍ത്തികമാക്കാനുള്ള അറിവില്ല. അതുകൊണ്ട്‌ ഏതൊരവസ്ഥയിലാണ്‌ ദു:ഖം അനുഭവവേദ്യമാവാതിരിക്കുക എന്നെനിക്കു പറഞ്ഞു തന്നാലും. എന്നേപ്പോലെ ലൌകീകപ്രവര്‍ത്തനങ്ങളില്‍ ഏര്‍പ്പെട്ടിരിക്കുന്ന ഒരുവന്‌ എങ്ങിനെ ശാന്തിയുടേയും ആനന്ദത്തിന്റേയും പരമോന്നതതലത്തില്‍ എത്താനാവും? ഏതൊരു മനോഭാവം കൈക്കൊള്ളുമ്പോഴാണ്‌ ഒരുവനെ ലൌകീകകര്‍മ്മങ്ങളും അനുഭവങ്ങളും ബാധിക്കാതിരിക്കുക? പ്രബുദ്ധനായ ഒരുവന്‍ ഈ ലോകത്തില്‍ എങ്ങിനെ  വര്‍ത്തിക്കുന്നു? എങ്ങിനെയാണ്‌ ലോകത്തിന്‌ വെറുമൊരു പുല്‍ക്കൊടിയുടെ മൂല്യമേയുള്ളു എന്നുകരുതി മനസ്സിനെ കാമത്തിന്റെ പിടിയില്‍ നിന്നും മോചിപ്പിക്കുക? എങ്ങിനെയാണ്‌ ലോകത്തെ സ്വന്തം ആത്മാവായി സാക്ഷാത്കരിക്കുക?

ഒരുവന്‍ എങ്ങിനെയാണ്‌ ഈ ലോകത്തില്‍ വര്‍ത്തിക്കേണ്ടത്‌? തുലോം ചഞ്ചലമായ എന്റെ മനസ്സിനെ മഹാമേരുവിനേപ്പോലെ ദൃഢമാക്കാനുള്ള വിവേകബോധം എനിക്കുപദേശിച്ചുനല്‍കിയാലും. അങ്ങ്‌ പ്രബുദ്ധന്നാണെന്നു ഞാനറിയുന്നു. ഈ വിവേകബോധത്തില്‍ നിന്നും ഒരിക്കലും തിരികെ ദു:ഖത്തിലേയ്ക്കു പതിക്കാതിരിക്കുവാനുള്ള ഉപദേശം നല്‍ കിയാലും. 

ഇഹലോകം വേദനയും മരണവും നിറഞ്ഞതാണല്ലോ. ഒരുവന്റെ മനസ്സിനെ കുഴപ്പിക്കാതെ എങ്ങിനെയാണ്‌ മനസ്സിനെ ആനന്ദത്തിന്റെ സ്രോതസ്സാക്കി മറ്റുക? കളങ്കം നിറഞ്ഞ മനസ്സിനെ എങ്ങിനെയാണ്‌ ശുദ്ധീകരിക്കുക? ഏതു മഹര്‍ഷിയാണ്‌ മന:ശുചീകരണവിദ്യ നിദ്ദേശിച്ചത്‌? ഇഷ്ടാനിഷ്ടദ്വന്ദങ്ങളുടെ ഒഴുക്കില്‍പ്പെടാതെ ഒരുവന്‍ എങ്ങിനെ ജീവിക്കണം?

തീര്‍ച്ചയായും ദു:ഖത്തിനും ദുരിതത്തിനും വശംവദനാവാതെ ഈ ലോകത്തില്‍ ജീവിക്കാന്‍ ഏതോ രഹസ്യം ഉണ്ടെന്നുറപ്പാണ്‌. തീയില്‍ വീണാല്‍പ്പോലും രസത്തെ (മെര്‍ക്കുറി)  ബാധിക്കുന്നില്ലല്ലോ. എന്താണാ രഹസ്യം? മനസ്സിന്റെ വ്യാപാരങ്ങള്‍ ഈ ബ്രഹ്മാണ്ഡമായി പരന്നുകിടക്കുന്നു. എന്താണ്‌ ആ മനസ്സിനെതിരേ വര്‍ത്തിക്കുന്ന പരമാര്‍ത്ഥത്തിന്റെ രഹസ്യം? മോഹവലയത്തില്‍ നിന്നും വിടുതല്‍ നേടിയ വീരപുരുഷന്മാര്‍ ആരെല്ലാം? അവര്‍ ഏതൊക്കെ മാര്‍ഗ്ഗങ്ങളാണ്‌ അവലംബിച്ചത്‌? 

"ഈ അറിവെനിയ്ക്ക്‌ അപ്രാപ്യമോ ഞാനിതിന്‌ അയോഗ്യനോ ആണെന്ന്‌ അങ്ങുകരുതുന്ന പക്ഷം ഞാന്‍ മരണംവരെ ഉപവസിക്കാന്‍ പോവുന്നു."

ഇത്രയും പറഞ്ഞ്‌ രാമന്‍ നിശ്ശബ്ദനായി.
