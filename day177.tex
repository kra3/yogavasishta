\section{ദിവസം 177}

\slokam{
കിച്ചിനോതി ചിതം ചേത്യം തേനേദം സ്ഥിതമാത്മനി\\
അജ്ഞേജ്ഞേ ത്വന്യദായാതമന്യദസ്തീതി കല്പനാ (4/36/11)\\
}

രാമൻ ചോദിച്ചു: ഭഗവൻ, അനന്താവബോധം അതീന്ദ്രിയമാണ്‌. അപ്പോൾ അതിലെങ്ങിനെ ഈ വിശ്വം നിലകൊള്ളും എന്നു പറഞ്ഞുതന്നാലും.

വസിഷ്ഠൻ പറഞ്ഞു: രാമാ, പ്രശാന്തമായ ഒരു സമുദ്രത്തിൽ ഭാവിയിൽ ഉണ്ടായേക്കാവുന്ന തിരകൾ സാദ്ധ്യതകളായി എങ്ങിനെ നിലകൊള്ളുന്നുവോ അങ്ങിനെയാണ്‌ അനന്താവബോധത്തിൽ ഈ വിശ്വത്തിന്റെ നിലനിൽപ്പ്. സത്യത്തിൽ വ്യത്യാസമൊന്നുമില്ലെങ്കിലും വിഭിന്നങ്ങളായ സാദ്ധ്യതകള്‍ അതിലടങ്ങിയിരിക്കുന്നു. ആകാശം എല്ലായിടവും നിറഞ്ഞു തിങ്ങിയിരിക്കുന്നുവെങ്കിലും അത് പ്രകടമല്ല. ദൃഷ്ടിഗോചരമല്ല. അതുപോലെയാണ്‌ അവബോധം. അതു പ്രത്യക്ഷമല്ല. ഒരു പളുങ്കുമണിയിൽ പ്രതിഫലനമായി കാണുന്ന വസ്തു സത്യമോ മിഥ്യയോ എന്നുറപ്പിച്ചു പറയാനാവില്ല. അതുപോലെയാണീ വിശ്വം. അനന്താവബോധത്തിലെ ഒരു പ്രതിഫലനമാണിത്. ആകാശത്തു മേഞ്ഞു നടക്കുന്ന മേഘക്കീറുകൾ ആകാശത്തെ ബാധിക്കാത്തതുപോലെ അനന്താവബോധത്തിനെ ലോകത്തിലുണ്ടാവുന്ന മാറ്റങ്ങൾ ബാധിക്കുന്നില്ല. ആകാശമില്ലെങ്കിൽ മേഘങ്ങളില്ല; അനന്താവബോധമില്ലെങ്കിൽ ലോകവുമില്ല. വെളിച്ചം തിരിച്ചറിയാൻ അതിനെ വികിരണംചെയ്യാനുള്ള ഒരുപാധികൂടിയേ തീരൂ. അതുപോലെ അനന്താവബോധം വെളിപ്പെടുന്നത് വിവിധ ശരീരങ്ങളിലൂടെയാണ്‌.. നാമ രൂപരഹിതമാണതെങ്കിലും അതുണ്ടാക്കുന്ന പ്രതിഫലനങ്ങൾക്ക് നാമരൂപങ്ങളുണ്ട്.

"അവബോധം അവബോധത്തിൽ പ്രതിഫലിച്ച് അവബോധമായി പ്രോജ്ജ്വലിച്ച് അവബോധമായിത്തന്നെ നിലകൊള്ളുന്നു. എന്നാൽ സ്വയം ബുദ്ധിമാനും വിവേകിയുമാണെന്നു കരുതുന്നവർപോലും അജ്ഞതയില്‍പ്പെട്ട് ലോകമെന്ന 'ധാരണ'യ്ക്ക് വശംവദരാവുന്നു. അനന്താവബോധത്തിൽ നിന്നും വ്യത്യസ്ഥമായാണ്‌ ലോകം നിലനില്‍ക്കുന്നതെന്നു ധരിച്ചുവശാവുന്നു." അജ്ഞാനിക്ക് അനന്താവബോധം പ്രത്യക്ഷമാവുന്നത് ഭീതിജന്യമായ ലോകമെന്ന നിലയിലാണ്‌.. എന്നാൽ ജ്ഞാനിക്കോ, അതേ അവബോധം ആത്മാവായി പ്രത്യക്ഷമാണ്. അനന്താവബോധമാണ്‌ ശുദ്ധമായ അനുഭവം എന്നു പറയുന്നത്. അതിനാലാണ്‌ സൂര്യൻ ജ്വലിക്കുന്നത്; ജീവജാലങ്ങൾ ജീവിതമാസ്വദിക്കുന്നത്.

അന്താവബോധം എന്നത് 'സൃഷ്ടിക്കപ്പെട്ട' ഒന്നല്ല. അതിനു നാശമില്ല. സനാതനമായ അതിനുമുകളിൽ പ്രത്യക്ഷലോകം, തിരകൾ സമുദ്രത്തിനുമേലെയെന്നപോലെ നിലകൊള്ളുന്നു. ആ ബോധം സ്വയം ആലോചിച്ചപ്പോൾ ‘ഞാൻ’ എന്നൊരു ധാരണ ഉദയം ചെയ്തു. ഈ ധാരണയാണ്‌ നാനാത്വത്തിനു കാരണം. അനന്താവബോധം, വിത്തിനു മുളപൊട്ടാനിടം കൊടുക്കുന്ന ആകാശമാണ്‌; മുളയെ വളരാൻ സഹായിക്കുന്ന വായുവാണ്‌; അതിനു പോഷകമേകുന്ന ജലമാണ്‌; സുസ്ഥിരതനല്കുന്ന ഭൂമിയാണ്‌; പുതുജീവനെ വെളിപ്പെടുത്തുന്ന വെട്ടമാണ്‌.

വിത്തിനുള്ളിലെ ബോധമാണ്‌ കാലക്രമത്തിൽ ഫലങ്ങളായി പ്രത്യക്ഷമാവുന്നത്. അതുതന്നെയാണ്‌ വ്യത്യസ്ഥ സ്വഭാവസവിശേഷതകളോടുകൂടിയ ഋതുക്കളാവുന്നതും. വിശ്വപ്രളയംവരെയും അനന്തമായ ജീവജാലങ്ങൾക്കു കാരണമാവുന്നതും അവയ്ക്കെല്ലാം അടിസ്ഥാനമാവുന്നതും ഈ അനന്താവബോധം തന്നെയാണ്‌..

