\newpage
\section{ദിവസം 130}

\slokam{
ന സ്പന്ദതേ മനോ യസ്യ ശസ്ത്രസ്തംഭ ഇവോത്തമ:\\
സദ്‌ വസ്തുതോസൌ പുരുഷ: ശിഷ്ടാ: കര്‍ദ്ദമ കീടകാ: (3/110/63)\\
}

വസിഷ്ഠന്‍ തുടര്‍ന്നു: രാമ: സര്‍വ്വവ്യാപിയും, നിര്‍മ്മലവും, ശാശ്വതവും അനന്താവബോധവുമായ ആ പരം പൊരുളിനെ മൂടുപടംകൊണ്ടു മറയ്ക്കാന്‍ മനസ്സിനാവുന്നു എന്നതിനേക്കാള്‍ വിസ്മയകരമായി മറ്റ്‌ എന്തുണ്ട്‌? എന്നിട്ടോ, അതിനെ മനസ്സ്‌ ഈ ജഢമാത്രമായ ശരീരവുമായി കൂട്ടിക്കലര്‍ത്തി സംഭ്രമിപ്പിക്കുന്നു.

ചലനവസ്തുക്കളില്‍ മനസ്സ്‌ വായുവായി പ്രകടമാവുന്നു. ഭൂമിയുടെ ദൃഢതയും ആകാശത്തിലെ ശൂന്യതയും മനസ്സു തന്നെ. മനസ്സ്‌ മറ്റൊരിടത്താണെങ്കില്‍ കഴിക്കുന്ന ആഹാരത്തിന്റെ സ്വാദ്‌ അനുഭവവേദ്യമല്ല. മനസ്സ്‌ മറ്റൊരിടത്തുള്ളപ്പോള്‍ കണ്മുന്നിലുള്ളതുകൂടി നമുക്ക്‌ കാണാനാവില്ല. ഇന്ദ്രിയങ്ങള്‍ മനസ്സില്‍നിന്നുണ്ടായതാണ്‌-. തിരിച്ചല്ല. മനസ്സും ശരീരവും രണ്ടും രണ്ടാണെന്ന് വിഡ്ഢികളേ പറയൂ. അവ ഒന്നു തന്നെ. മനസ്സു മാത്രം. ഈ സത്യം സാക്ഷാത്കരിച്ച മഹര്‍ഷിമാര്‍ക്കു നമോവാകം. ഈ സത്യമറിഞ്ഞ മാമുനിമാര്‍ക്ക്‌, തന്നെയൊരു സ്ത്രീ വന്നു പുണര്‍ന്നാലും അതൊരു ശല്യമാവുന്നില്ല. ശരീരത്തിലൊരു തടിക്കഷണം സ്പര്‍ശിക്കുന്നതുപോലെയേ അതനുഭവമാകുന്നുള്ളു. കൈകള്‍ വെട്ടിമാറ്റിയാലും അചഞ്ചലമായിരിക്കാന്‍ അവര്‍ക്കാവും. എല്ലാ ദു:ഖങ്ങളേയും ആനന്ദാനുഭവമാക്കാനും അവര്‍ക്കാവും.

മനസ്സ്‌ മറ്റൊരിടത്താണെങ്കില്‍ അതിരസകരമായ ഒരു കഥ ചെവിയിലെത്തിയാലും നാം അതു കേള്‍ ക്കുന്നില്ല. സ്വയം പല കഥാപാത്രങ്ങളായി അഭിനയിക്കാന്‍ കഴിയുന്ന ഒരു കൃതഹസ്തനടനേപ്പോലെ സ്വപ്നം, ജാഗ്രത്ത്‌ തുടങ്ങിയ പലബോധതലങ്ങളും സൃഷ്ടിക്കാന്‍ മനസ്സിനു കഴിയുന്നു. ലവണ മഹാരാജാവിനേപ്പോലെയുള്ള ഒരു മഹാനെപ്പോലും അപരിഷ്കൃതനായ ഒരു ഗോത്രവര്‍ഗ്ഗക്കാരനായി മാറ്റാന്‍ മനസ്സിനെത്രവേഗമാണു സാധിച്ചത്‌!!. മനസ്സ്‌ സ്വന്തമായി കെട്ടിയുണ്ടാക്കുന്നതിനെയാണ്‌ അനുഭവിക്കുന്നത്‌.. ചിന്തകളാല്‍ പടുത്തുകെട്ടിയതാണ്‌ മനസ്സ്‌.. ഇതറിഞ്ഞ്‌ നീ നിനക്ക്‌ ബോധിച്ചതുപോലെ പ്രവര്‍ത്തിച്ചാലും. തുടര്‍ച്ചയായ ചിന്തകള്‍കൊണ്ട്‌ ദൃഢീകൃതമായ മനസ്സ്‌ താന്‍ ജനിച്ചെന്നും മരിച്ചെന്നും ചിന്തിക്കുന്നു. സ്വയം രൂപരഹിതമെങ്കിലും മനസ്സ്‌ ശരീരവുമായി തദാത്മ്യഭാവം കൈക്കൊള്ളുന്നു. ശരീരമുള്ള ഒരു ജീവനാണു താന്‍ എന്നു ധരിക്കുന്നു. ചിന്തകള്‍ കാരണമാണ്‌ ഒരുവനില്‍ ദേശചിന്തയും സുഖ-ദു:ഖാനുഭവങ്ങളും അങ്കുരിക്കുന്നത്‌.. വിത്തിലെ എണ്ണപോലെ ലീനമായി ഇതെല്ലാം മനസ്സിലുള്ളതുതന്നെ.

ആരൊരുവന്‍ മനസ്സിനെ സുഖം നേടാനായി വിഷയവസ്തുക്കളില്‍ അലയാന്‍ വിടുന്നില്ലയോ അവന്‍ മനസ്സിനെ വെന്നവനാണ്‌.. "തൂണില്‍ ബന്ധിച്ചവന്‍ തൂണിനെപ്പോലെ തന്നെ അചലമാണ്‌.. അതുപോലെ മഹാത്മാക്കളുടെ മനസ്സ്‌ ഉണ്മയില്‍നിന്നും വ്യതിചലിക്കുന്നില്ല. അവനാണ്‌ മനുഷ്യന്‍... മറ്റെല്ലാവരും വെറും പുഴുക്കളത്രേ." നിരന്തര ധ്യാനത്തിലൂടെ, സത്യമറിഞ്ഞവന്‍ പരമപദം പ്രാപിക്കുന്നു. 

