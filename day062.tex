 
\section{ദിവസം 062}

\slokam{
പരമാണൗ പരമാണൗ സർഗ്ഗവർഗ്ഗ നിരർഗളം\\
മഹച്ചിത്തേ സ്ഫുരന്ത്യർക്കരുചീവ ത്രസരേണവ: (3/27/29)\\
}

വസിഷ്ഠന്‍ തുടര്‍ന്നു: അങ്ങിനെ ആ മഹാത്മാവിന്റെ ഗൃഹത്തിലെ എല്ലാവരേയും അനുഗ്രഹിച്ചിട്ട്‌ അവര്‍ അപ്രത്യക്ഷരായി. ദു:ഖനിവൃത്തിവന്ന കുടുംബാംഗങ്ങള്‍ അവരവരുടെ ഇടങ്ങളിലേയ്ക്ക്‌ തിരിച്ചുപോയി. ലീല, സരസ്വതിയോട്‌ ഒരു ചോദ്യം ചോദിക്കാനായി തിരിഞ്ഞു. അപ്പോള്‍ അവരുടെ ശരീരങ്ങള്‍ രണ്ടുസ്വപ്നമൂര്‍ത്തികള്‍പോലെ മാത്രമായിരുന്നു. വിഷയവസ്തുക്കളോ മന:ശരീര സംബന്ധിയായ പ്രാണന്‍ മുതലായ ലക്ഷണങ്ങളോ അവര്‍ക്കുണ്ടായിരുന്നില്ല. ഈ 'അമൂര്‍ത്തികള്‍ ' തമ്മിലായിരുന്നു സംഭാഷണം. 

ലീല ചോദിച്ചു: നാം ആ കുടുംബത്തില്‍ പോയപ്പോള്‍ എന്തുകൊണ്ട്‌ രാജാവായ എന്റെ ഭര്‍ത്താവ്‌ എന്നെ കണ്ടില്ല?. എന്നാല്‍ 'എന്റെ കുടുംബം' എന്നെ കണ്ടതെങ്ങിനെ? ദേവി പറഞ്ഞു: അപ്പോള്‍ നീ, 'ഞാന്‍ ലീല' എന്ന അഹം ധാരണയില്‍ത്തന്നെയായിരുന്നു. ഇപ്പോള്‍ നീയാ ബന്ധനത്തില്‍ നിന്നും ഒഴിഞ്ഞ്‌ ശുദ്ധബോധമായിരിക്കുന്നു. ദ്വന്ദബോധം അസ്തമിക്കാതെ ഒരുവനില്‍ അനന്താവബോധം സ്ഫുരിക്കയില്ല. അതില്‍ വര്‍ത്തിക്കാനുമാവില്ല. ആ സ്ഥിതിയെപ്പറ്റി ഒരുവന്‌ അറിയാന്‍ കൂടിയാവില്ല. വെയിലത്തുനില്‍ക്കുന്നവന്‌ മരത്തണലിന്റെ ശീതളിമ അറിയാത്തതുപോലെയാണത്‌. എന്നാല്‍ ഇപ്പോള്‍ നിനക്ക്‌ നിന്റെ ഭര്‍ത്താവിന്റെയടുക്കല്‍ പോകാം. പഴയപോലെ പെരുമാറുകയും ചെയ്യാം.

ലീല പറഞ്ഞു: ദേവീ! അത്ഭുതം! ഇവിടെത്തന്നെ, എന്റെ ഭര്‍ത്താവായിരുന്നു ആ മഹാത്മാവ്‌. ഇപ്പോള്‍ അദ്ദേഹം രാജാവും ഞാന്‍ അദ്ദേഹത്തിന്റെ ഭാര്യയുമാണ്‌ പിന്നെയും !. അദ്ദേഹം മരിച്ചിട്ടു തിരിച്ചുവന്ന് ഇതാ രാജ്യം ഭരിക്കുന്നു.! ദയവായി എന്നെ അങ്ങോട്ടു കൊണ്ടുപൊയാലും.. സരസ്വതി പറഞ്ഞു: ലീലേ നീയും ഭര്‍ത്താവും പല ജന്മങ്ങള്‍ എടുത്തിട്ടുണ്ട്‌. ഈ ജന്മത്തില്‍ രാജാവ്‌ ലൌകീകകാര്യങ്ങളില്‍ ആണ്ടുമുങ്ങിയിരിക്കുന്നു. 'ഞാന്‍ രാജാവ്‌, ഞാന്‍ പ്രബലന്‍ , എനിയ്ക്കു സുഖമാണ്‌' തുടങ്ങിയ ചിന്തകളാണദ്ദേഹത്തിനുള്ളില്‍ . ആത്മനിഷ്ഠമായി നോക്കുമ്പോള്‍ വിശ്വം മുഴുവനും ഇവിടെ അനുഭവവേദ്യമാണെങ്കിലും ഭൌതീകമായി  നോക്കുമ്പോള്‍ ഈ തലങ്ങള്‍ക്ക്‌ അനേകകോടി കാതങ്ങള്‍ അകലമുണ്ട്‌. "അനന്താവബോധത്തില്‍ എല്ലാ അണുക്കളിലും വിശ്വപ്രപഞ്ചം വന്നുപോയിക്കൊണ്ടിരിക്കുന്നു. മേല്‍ക്കൂരയിലെ ചെറിയൊരുസുഷിരത്തിലൂടെ വരുന്ന പ്രകാശരശ്മി പൊടിപടലങ്ങളെ തിളക്കമുറ്റതാക്കുന്നതുപോലെയാണിത്‌." കടലിലെ അലകള്‍ പോലെ, ഇതു വന്നും പോയുമിരിക്കുന്നു.

ലീല ഓര്‍ത്തു: ദൈവമേ! അനന്താവബോധത്തില്‍നിന്നും ഒരു പ്രതിഫലനമായി വന്നതിനുള്ളില്‍ എനിയ്ക്ക്‌ എണ്‍പതു ജന്മങ്ങള്‍ കഴിഞ്ഞിരിക്കുന്നു. ഇപ്പോൾ ഞാനതുമനസ്സിലാക്കുന്നു. ഞാന്‍ ഒരപ്സരസ്സായിരുന്നു. ദുര്‍ന്നടത്തക്കാരിയായ മനുഷ്യസ്ത്രീയായിരുന്നു, ഒരു സര്‍പ്പമായിരുന്നു, കാട്ടുജാതിക്കാരിയായ ഒരാദിവാസിയായിരുന്നു. എന്റെ ദുഷ്ടപ്രവര്‍ത്തികളുടെ പരിണിതഫലമായി ഞാന്‍ ഒരു വള്ളിച്ചെടിയായി. മഹാനായൊരു ഋഷിയുടെ സാമീപ്യം കൊണ്ട്‌ ഞാന്‍ ഒരു മുനികുമാരിയായി, പിന്നെ രാജാവായി. ദുഷ്ടപ്രവൃത്തികള്‍ ചെയ്ത്‌ ഒരു കൊതുകായി ജന്മമെടുത്തു. പിന്നെ, തേനീച്ച, മാന്‍ , പക്ഷി, മല്‍സ്യം; പിന്നെയും ഒരപ്സരസ്സ്‌, ആമ, അരയന്നം, പിന്നേയും കൊതുക്‌, എന്നിങ്ങിനെ. അപ്സരസ്സായിരുന്നപ്പോള്‍ ദേവന്മാര്‍ എന്റെ കാല്‍ക്കല്‍ വീണിരുന്നു. ത്രാസ്സിന്റെ തട്ടുപോലെ പൊങ്ങിയും താണും എനിക്ക് ജന്മങ്ങള്‍ അനേകം ഉണ്ടായിക്കഴിഞ്ഞു.

