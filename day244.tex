\section{ദിവസം 244}

\slokam{
സര്‍വ്വഭാവാന്തരസ്ഥായ ചേത്യമുക്തചിദാത്മനേ\\
പ്രത്യക്ചേതനരൂപായ മഹ്യമേവ നമോ നമ:  (5/34/69)\\
}

പ്രഹ്ലാദന്‍ മനനം തുടര്‍ന്നു: തീര്‍ച്ചയായും ആ അനന്താവബോധം മാത്രമല്ലേ എപ്പോഴും ഉണ്ടായിരുന്നുള്ളൂ? പിന്നെയെങ്ങിനെയാണ് നിയതമായ യാതൊരു കാരണവും അടിസ്ഥാനവുമില്ലാതെ  ഈ പരിമിതപ്പെട്ട അഹംഭാവം അതില്‍ ഉയര്‍ന്നു വന്നത്? ‘ഇത് ഞാന്‍’; ‘അത് നീ’, തുടങ്ങിയ ഭ്രമാത്മക ചിന്തകള്‍ക്കും ധാരണകള്‍ക്കും കാരണമായതെന്താണ്?

ഈ ശരീരമെന്നതെന്താണ്? ശരീരമില്ലാത്ത അവസ്ഥയെന്താണ്? അരാണീ ശരീരത്തില്‍ നിവസിക്കുന്നത്? ആരാണു മരണപ്പെടുന്നത്? തീര്‍ച്ചയായും എന്റെ പൂര്‍വികര്‍ അവരുടെ അജ്ഞതമൂലം അനന്താവബോധത്തെ അവഗണിച്ചുകൊണ്ട് അവരീ ചെറിയഭൂമിയില്‍ ചുറ്റിക്കറങ്ങി.  അനന്തതയുടെ അപാരതയും ഈ ചെറുഭൂമിയുടെ പൊങ്ങച്ചഭാവവും തമ്മില്‍ എങ്ങിനെ താരതമ്യപ്പെടുത്താനാവും? ചെറുതെങ്കിലും ഇവിടെ അതിഭീതിതമായ ആസക്തിയും ആശാത്വരകളും നിറഞ്ഞിരിക്കുകയല്ലേ?

എന്നാല്‍ അനന്താവബോധത്തിന്റേതായ ദര്‍ശനം അതീവ നിര്‍മലവും പരമപ്രശാന്തി ദായകവുമാണ്. ഈ പ്രപഞ്ചത്തില്‍ സാദ്ധ്യമായതില്‍ വെച്ചേറ്റവും ഉത്തമമാണ് അനന്താവബോധത്തിന്റ ഈ ദര്‍ശനം. “എന്റെയുള്ളില്‍ സ്ഥിതിചെയ്യുന്ന ആത്മാവിനെ (അന്തര്യാമിയെ) ഞാനിതാ നമസ്കരിക്കുന്നു. എല്ലാ വസ്തു–വിഷയ ബോധങ്ങള്‍ക്കും ധാരണകള്‍ക്കും അതീതമായി വര്‍ത്തിക്കുന്ന അത് ജീവജാലങ്ങളുടെ മേധാശക്തിയാണ്.” ഇനിയും ജനിച്ചിട്ടില്ലാത്ത(അജന്‍))) എന്നില്‍ ലോകം അപ്രത്യക്ഷമായിരിക്കുന്നു. ലഭിക്കുവാന്‍ യോഗ്യമായതെന്തോ അത് ഞാന്‍ നേടിയിരിക്കുന്നു. ഞാന്‍ വിജയിയായി പരിലസിക്കുന്നു. വിശ്വാവബോധം ഉപേക്ഷിച്ച് ലഭിക്കുന്ന രാജ്യഭരണത്തില്‍ ഞാന്‍ ലേശംപോലും ആഹ്ലാദം കാണുന്നില്ല.

ഈ ലൌകീകതയില്‍ മുഴുകിയിരിക്കുന്ന അസുരവര്‍ഗ്ഗത്തിനോടെനിക്ക് കഷ്ടം തോന്നുന്നു. എന്റെ അച്ഛന്‍ എത്ര അജ്ഞാനിയായിരുന്നു! ഭൌതീകമായ നിലനില്‍പ്പില്‍ മാത്രം അഭിരമിച്ചു ജീവിതം പാഴാക്കിക്കളഞ്ഞു അദ്ദേഹം. ഏറെക്കാലം ജീവിച്ചിട്ടും, ചെറിയൊരു മണ്‍കൂന മാത്രമായ ഈ ലോകം ഭരിച്ചിട്ടും അദ്ദേഹമെന്തു നേടി? അത്തരം നിസ്സാരമായ അനേകം ലോകങ്ങളുടെ അധിപസ്ഥാനവും ആത്മാനന്ദത്തിനു സമമായി വരികയില്ല. മറ്റൊന്നുമില്ലെങ്കിലും ആത്മജ്ഞാനിയായ ഒരുവന്‍ എല്ലാം തികഞ്ഞവനത്രേ. ഇതുപേക്ഷിച്ച് മറ്റെന്തെങ്കിലും തേടുന്നവന്‍ ബുദ്ധിമാനല്ല.
 
ക്ഷണഭംഗുരമായ, നശ്വരമായ, മരുഭൂമിപോലെ വരണ്ട  ഭൌതീകമായ ഈ താല്‍ക്കാലികമായ നിലനില്‍പ്പെവിടെ? പ്രസാദമാധുരിമയേകുന്ന നന്ദനോദ്യാനം പോലെയുള്ള ആത്മസാക്ഷാത്കാരത്തിന്റെ ശാശ്വതമായ നിറവെവിടെ? മൂന്നുലോകങ്ങളുടെയും അധീശത്വവും നിലനില്‍പ്പും ബോധത്തില്‍ മാത്രമാണ്.   ഈ സത്യമനുഭവിച്ചറിഞ്ഞ് ബോധത്തിനപ്പുറം ഒന്നും ഉണ്മയായില്ല എന്ന് മനുഷ്യര്‍ തിരിച്ചറിയാത്തത് എന്തുകൊണ്ടാണ്? അവിച്ഛിന്നവും സര്‍വ്വവ്യാപിയും, സര്‍വ്വശക്തവുമായ അനന്താവബോധത്തിലൂടെ എല്ലാമെല്ലാം എല്ലായിടത്തും എപ്പോഴും സുലഭ്യമാണ്. 

സൂര്യചന്ദ്രന്മാരില്‍ പ്രഭാസിക്കുന്ന വെളിച്ചം, ദേവതകളില്‍ പരിലസിക്കുന്ന ചൈതന്യവിശേഷം, മനസ്സിന്റെ സ്വഭാവസവിശേഷതകള്‍ , പ്രകൃതിയിലെ എല്ലാ വസ്തുക്കളിലും സഹജമായി കാണപ്പെടുന്ന ഗുണഗണങ്ങള്‍, ഊര്‍ജ്ജത്തിന്റെ അനന്തകോടി പ്രസ്രണങ്ങള്‍, വിശ്വാവബോധത്തില്‍ പ്രകടിതമാവുന്ന വികാസപരിണാമങ്ങള്‍, എന്നിവയ്ക്കെല്ലാം അടിസ്ഥാനമായിരിക്കുന്ന ആത്മാവിനു മാറ്റങ്ങളോ വൈരുദ്ധ്യങ്ങളോ ഇല്ല. യാതൊരുവിധപക്ഷഭേദവും കൂടാതെ താപവും പ്രകാശവും വിതരണം ചെയ്യുന്ന സൂര്യപ്രഭപോലെ ക്ഷിപ്രനിരന്തരമായ അനന്താവബോധം പ്രപഞ്ചത്തിലെ എല്ലാറ്റിനെയും പ്രകാശമാനമാക്കുന്നു.

