\section{ദിവസം 151}

\slokam{
നനു വിജ്ഞാതസംസാരഗതയോ വയമാപദാം \\
സംപദാം  ചൈവ ഗച്ഛാമോ ഹർഷമർഷവശം വിഭോ (4/11/13)\\
}

വസിഷ്ഠൻ തുടർന്നു: യമന്റെ (കാലം) പ്രചോദനത്താൽ ഭൃഗുമഹർഷി ജ്ഞാനദൃഷ്ടിയിൽക്കൂടി  തന്റെ മകന്റെ പ്രയാണം കണ്ടു. ക്ഷണനേരത്തിൽ ശുക്രന്റെ പരകായപ്രവേശങ്ങളുടെ മുഴുവൻ കഥയും മനസ്സിലാക്കി. വിസ്മയചകിതനായ മഹർഷി തന്റെ പൂര്‍വ്വസ്ഥിതിയിലേയ്ക്ക് തിരികെ വന്നു. മകനോടുള്ള എല്ലാ ആസക്തിയും ഉപേക്ഷിച്ച് ഭൃഗുമഹര്‍ഷി  പറഞ്ഞു: ഭഗവൻ, അങ്ങ് ഭൂത ഭാവി വർത്തമാനങ്ങൾ അറിയുന്നവനാണ്‌.. ഞങ്ങളോ അൽപ്പജ്ഞാനികൾ. ഇക്കാണപ്പെടുന്ന ലോകം അവാസ്തവമെങ്കിലും സത്യമാണെന്നു തോന്നുകയാണ്‌.. വീരനായകന്മാർ പോലും മോഹത്തിനടിമയാകുന്നു. അങ്ങേയ്ക്കുള്ളിൽത്തന്നെ ഇതെല്ലാം നിലകൊള്ളുന്നു. മനസ്സിന്റെ ഭാവനകൾകൊണ്ടുണ്ടാക്കിയ മായാരൂപങ്ങളുടെ യാഥർത്ഥ്യം അങ്ങേയ്ക്കറിയാം.

എന്റെ പുത്രൻ മരിച്ചിട്ടില്ല. എന്നാൽ അങ്ങിനെ വിചാരിച്ച് എന്റെ മനസ്സ് കലുഷിതമായി. കാലമാവുന്നതിനു മുൻപ് അവനെ കൊണ്ടുപോയതാണെന്നു ഞാൻ തെറ്റിദ്ധരിച്ചു. “ഭഗവൻ, ഭൂമിയിലെ സംഭവവികാസങ്ങളുടെ ഗതിവിഗതികൾ നമുക്കറിയാമെങ്കിലും ഭാഗ്യനിർഭാഗ്യങ്ങളെന്നു നാം കരുതുന്നവയുടെ വരുതിയിൽപ്പെട്ട് നാം സന്തോഷിക്കുകയും ദു:ഖിക്കുകയും ചെയ്യുന്നു.” ഈ ലോകത്ത് ക്രോധം മൂലം ആളുകൾ ചെയ്യരുതാത്ത കർമ്മങ്ങൾ ചെയ്യുന്നു. പ്രശാന്തമനസ്സ് മനുഷ്യനെ ഉചിതകർമ്മങ്ങൾ ചെയ്യാൻ പ്രാപ്തമാക്കുകയും ചെയ്യുന്നു. ലോകാനുഭവം എന്ന മായ ഉള്ളിടത്തോളം കർമ്മങ്ങളെ ഉചിതം, അനുചിതം എന്നിങ്ങനെ തിരിക്കുന്നതിന്‌ പ്രസക്തിയുണ്ട്. അങ്ങയുടെ സഹജകർമ്മം ജീവികളുടെ മരണത്തിലേയ്ക്കു നയിക്കുന്നു. അത് നമ്മുടെ മനസ്സിനെ ക്ഷോഭിപ്പിക്കുന്നത് തികച്ചും അനുചിതം തന്നെ. അങ്ങയുടെ കൃപയാൽ ഞാൻ മകനെ വീണ്ടും കണ്ടു. മനസ്സുതന്നെയാണ്‌ ശരീരമെന്നും ഈ മനസ്സാണ്‌ പ്രപഞ്ചദർശനത്തിനു ഹേതുവെന്നും ഞാനറിയുന്നു.

കാലന്‍ പറഞ്ഞു: വളരെ ശരിയാണ്‌ മുനേ. മനസ്സുതന്നെയാണ്‌ ശരീരം. കുശവൻ മൺപാത്രമുണ്ടാക്കുന്നതുപോലെ മനസ്സാണ്‌ ശരീരത്തെ കേവലം ചിന്തകൾ കൊണ്ട് മാത്രം ‘സൃഷ്ടിക്കുന്നത്’. വെറും ഇച്ഛാശക്തികൊണ്ട് പുതിയ ശരീരങ്ങളെ നിർമ്മിക്കുകയും ഉള്ളവയെ നശിപ്പിക്കുകയും ചെയ്യുന്നത് മനസ്സാണ്‌.. മനസ്സിലാണ്‌ മായക്കാഴ്ച്ചകളും, വിഭ്രാന്തികളും യുക്തിരാഹിത്യവും മൂലം ആകാശക്കോട്ടകളുയരുന്നത്. അങ്ങിനെയാണ്‌ ശരീരത്തെ ഉള്ളിൽ പ്രകടമാക്കുന്നത്. എന്നാൽ അജ്ഞാനി ഭൗതീകശരീരത്തെ മനസ്സിൽനിന്നും വിഭിന്നമായ മറ്റൊരു രൂപമായിക്കാണുന്നു. ജാഗ്രത്-സ്വപ്ന-സുഷുപ്തികൾ എന്ന ത്രിലോകങ്ങൾ മനസ്സിന്റെ പ്രഭാവങ്ങളുടെ  ഒരാവിഷ്ക്കാരം മാത്രം. ഇത് യാഥാർത്ഥ്യമെന്നോ അല്ലെന്നോ പറയാൻ വയ്യ. മനസ്സ് പല പല ഭാവങ്ങളില്‍ ആമഗ്നമായിരിക്കുമ്പോൾ കാണുന്നതും  വൈവിദ്ധ്യം തന്നെ. 

