\section{ദിവസം 239}

\slokam{
ഗുണവാന്നിര്‍ഗുണോ ജാത ഇത്യനര്‍ഥക്രമം വിദു:\\
നിര്‍ഗ്ഗുണോ ഗുണവാഞ്ജാത ഇത്യാഹു: സിദ്ധിതം ക്രമം.\\
}

വസിഷ്ഠന്‍ തുടര്‍ന്നു: അങ്ങിനെ സ്വയം വിഷ്ണുവിന്റെ സാരൂപ്യം അവലംബിച്ചശേഷം പ്രഹ്ലാദന്‍ വിഷ്ണുപൂജ ചെയ്യേണ്ടതിനെപ്പറ്റി ഇപ്രകാരം ചിന്തിച്ചു. ‘ഇതാ ഇവിടെ ഗരുഡവാഹനനായ  മറ്റൊരു വിഷ്ണു. വിഷ്ണുവിന്റെതായി പറയപ്പെടുന്ന എല്ലാ ഛിഹ്നങ്ങളും ദിവ്യഗുണങ്ങളും ശക്തിവിശേഷവും ഉള്ള മറ്റൊരു വിഷ്ണുരൂപം. ആ വിഷ്ണുവിനെ ആചാരക്രമമനുസരിച്ചു തന്നെ  ഞാനിതാ മനസാ പൂജിക്കുന്നു. അങ്ങിനെ തീരുമാനിച്ച് എല്ലാവിധ തയ്യാറെടുപ്പുകളോടും കൂടി ശാസ്ത്രവിധിപ്രകാരം  പ്രഹ്ലാദന്‍ മനസാ വിഷ്ണുപൂജ ചെയ്തു. ബാഹ്യമായ ആചാരക്രമങ്ങളും പ്രഹ്ലാദന്‍ തുടര്‍ന്നു ചെയ്യുകയുണ്ടായി. പൂജ കഴിഞ്ഞ് പ്രഹ്ലാദന്‍ സന്തോഷചിത്തനായി. അന്നുമുതല്‍ ദിവസവും പ്രഹ്ലാദന്റെ വിഷ്ണുപൂജ ഇത്തരത്തിലുള്ളതായിരുന്നു.
  
പ്രഹ്ലാദനെ പിന്തുടര്‍ന്ന്‍ അസുരന്മാരെല്ലാവരും വിഷ്ണുഭക്തന്മാരായിത്തീര്‍ന്ന്‍   ഇപ്രകാരം വിഷ്ണുപൂജ ചെയ്യാനാരംഭിച്ചു. ഈ വാര്‍ത്ത കാട്ടുതീപോലെ സ്വര്‍ഗ്ഗത്തിലും പരന്നു. ഇന്നേവരെ വിഷ്ണുവിന്റെ ബദ്ധശത്രുക്കളായിരുന്ന അസുരവര്‍ഗ്ഗം ഇപ്പോള്‍ വിഷ്ണുഭക്തരായി മാറിയിരിക്കുന്നു.! രാക്ഷസവര്‍ഗ്ഗമെങ്ങിനെ ഭക്തന്മാരാകും? ദേവന്മാര്‍ പെട്ടെന്ന് വിഷ്ണുഭഗവാനെക്കണ്ട് വിവരം അന്വേഷിച്ചു. ‘ഭഗവാനേ, എന്താണീ മറിമായത്തിനു പിന്നില്‍ ? അസുരന്മാര്‍ അങ്ങയുടെ ശത്രുക്കളാണല്ലോ? ഇപ്പോളവര്‍ അങ്ങയുടെ ഭക്തന്മാരായതില്‍ എന്തോ കള്ളക്കളിയോ ചതിയോ ഉണ്ട്. രാക്ഷസവര്‍ഗ്ഗത്തിന്റെ ആസുരീക സ്വഭാവമെവിടെപ്പോയൊളിച്ചു? ജീവികള്‍ക്ക് പൂര്‍വ്വജന്മാര്‍ജ്ജിത സുകൃതത്താല്‍ മാത്രം ലഭ്യമാവുന്ന ഭക്തിയെങ്ങിനെയാണവര്‍ക്ക് ലഭിക്കുക? നന്മയും പവിത്രതയും ഈ അസുരന്മാരുമായി ഒത്തുപോകുന്നതെങ്ങിനെ? അവിശ്വസനീയം!

ഒരു ജീവിയുടെ സ്വഭാവം അതിന്റെ പ്രകൃത്യായുള്ള സവിശേഷതകളുമായി ബന്ധപ്പെട്ടിരിക്കുന്നു. ഒരൊറ്റ രാത്രികൊണ്ട് ഈ അസുരന്മാര്‍ ഭക്തന്‍മാരായി എന്നത് ഞങ്ങള്‍ക്ക് വിഷമമുണ്ടാക്കുന്നു. കാലക്രമം കൊണ്ട് പടിപടിയായി അവര്‍ സദ്ഗുണങ്ങളാര്‍ജ്ജിച്ചുവെന്നു പറഞ്ഞാല്‍ , അങ്ങിനെയാണവര്‍ അവിടുത്തെ ഭക്തന്മാരായത് എന്നുവച്ചാല്‍ ഞങ്ങള്‍ക്കതു മനസ്സിലാക്കാം. എന്നാല്‍ ദുഷ്ടന്മാരായിരുന്നവര്‍ പൊടുന്നനെ ഭക്തശിരോമണികളായി എന്നത് വിശ്വസിക്കാന്‍ പ്രയാസം.

ഭഗവാന്‍ പറഞ്ഞു: ദേവജനങ്ങളേ, നിങ്ങള്‍ വെറുതെ വിഷമിക്കണ്ട. സംശയിക്കുകയും വേണ്ട. പ്രഹ്ലാദന്‍ എന്റെ ഉത്തമഭക്തനായി. ഇതയാളുടെ അവസാനത്തെ ജന്മമാണ്. അതിനാല്‍ മുക്തിപദം ഇപ്പോള്‍ത്തന്നെ അയാളര്‍ഹിക്കുന്നു. അയാളിലെ അജ്ഞാനത്തിന്റെ വിത്തുകള്‍ പാടേ എരിഞ്ഞു  ചാമ്പലായിപ്പോയിരിക്കുന്നു. ഇനി അയാള്‍ക്ക് ജന്മമെടുക്കേണ്ടതില്ല.

“വാസ്തവത്തില്‍ ഒരു സദ്പുരുഷന്‍ ദുഷ്ടനായി മാറി എന്നറിയുന്നതാണ് വേദനാജനകവും അര്‍ത്ഥശൂന്യവുമായ കാര്യം. എന്നാല്‍ സദ്‌ഗുണങ്ങളില്ലാത്ത ഒരസുരന്‍ ഗുണവാനായി എന്നത് ഏറ്റവും ഉചിതമായ, അഭികാമ്യമായ കാര്യമാണെന്ന് നിങ്ങളറിയുക .” പ്രഹ്ലാദനിലെ ഈ മാറ്റം നിങ്ങള്‍ക്കും നല്ലതിനാണ്.

(\textit{മറ്റൊരു രീതിയില്‍ പറഞ്ഞാല്‍ , ഉപാധികളാല്‍ പരിമിതപ്പെട്ടവന്‍ അപരിമിതനായി എന്ന് പറയുന്നത് ശരിയല്ല. എന്നാല്‍ ഉപാധികളില്ലാത്ത ബ്രഹ്മം ഉപാധികളോടു കൂടിയതായി കാണപ്പെട്ടു എന്ന് പറഞ്ഞാല്‍ അതു ശരിയാണ്.}) 
