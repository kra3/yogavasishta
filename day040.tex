 
\section{ദിവസം 040}

\slokam{
യസ്മാദ്വിഷ്ണവാദയോ ദേവാ: സൂര്യാദിവ മരീചയ:\\
യസ്മാജ്ജഗന്ത്യാനന്താനി ബുദ്ബുദാ ജലധേരിവ (3/5/9)\\
}

രാമന്‍ ചോദിച്ചു: മഹര്‍ഷേ, ഈ മനസ്സ്‌ ആവിര്‍ഭവിച്ചത്‌ എന്തില്‍ നിന്നാണ്‌? എങ്ങിനെയാണ്‌ അതുത്ഭവിച്ചത്‌? ദയവായി എന്നെ പ്രബുദ്ധനാക്കിയാലും.

വസിഷ്ഠന്‍ പറഞ്ഞു: വിശ്വമഹാപ്രളയം കഴിഞ്ഞ്‌ അടുത്തയുഗാരംഭത്തിനു മുന്‍പ്‌ വസ്തുപ്രപഞ്ചം പരിപൂര്‍ണ്ണ സംതുലിതാവസ്ത്ഥയിലായിരുന്നു. അവിടെ അനശ്വരനും ഇതുവരെ പിറവിയെടുക്കാത്തവനും സ്വയം പ്രഭയുള്ളവനും, സര്‍വ്വ ശക്തിശാലിയും എല്ലാത്തിന്റെ ഉണ്മയുമായ പരം പൊരുളായി ഈശ്വരന്‍ ഉണ്ടായിരുന്നു. ധാരണകള്‍ക്കും വിശദീകരണങ്ങള്‍ക്കും അതീതനായ അവന്‌ ആത്മാവെന്നും മറ്റും നാമമുള്ളത്‌ ഓരോരുത്തരുടെ അഭിമതം മാത്രമാണ്‌. സത്യത്തില്‍ അവന്‍ സര്‍വ്വസ്വമാണെങ്കിലും ലോകത്തിലാരും അവനെ സാക്ഷാത്കരിക്കുന്നില്ല. ശരീരത്തിനുള്ളിലും അവന്റെ വ്യാപ്തി നിലനില്‍ക്കുമ്പോഴും അവന്‍ അകലെയാണ്‌. "സൂര്യനില്‍ നിന്നും എണ്ണമറ്റ രശ്മികള്‍ ഉദിക്കുന്നതുപോലെ അവനില്‍ നിന്നു മഹാവിഷ്ണുവടക്കം എണ്ണമറ്റ ദിവ്യ സത്വങ്ങള്‍ ആവിര്‍ഭവിക്കുന്നു. സമുദ്രജലത്തില്‍ അലകളുണ്ടാവും പോലെ അവനില്‍ നിന്നും അന്തമില്ലാതെ ലോകങ്ങള്‍ ഉദ്ഭവിക്കുന്നു."

വിശ്വാവബോധമായ അവനില്‍ എണ്ണമറ്റ വിഷയവസ്തുക്കള്‍ 'പ്രവേശി'ക്കുന്നു. ആത്മാവും പ്രപഞ്ചവും അവന്റെ പ്രകാശത്തില്‍ തിളങ്ങുന്നു. അവന്‍ സൃഷ്ടിക്കപ്പെട്ട എല്ലാ ജീവജാലങ്ങളുടേയും സ്വഭാവസവിശേഷതകളെ നിയന്ത്രിക്കുന്നു. മരുഭൂമിയിലെ കാനല്‍ ജലം പ്രത്യക്ഷവും അപ്രത്യക്ഷവുമാവുന്നപോലെ ലോകങ്ങള്‍ ഉണ്ടായി മറയുന്നു. അവന്റെ രൂപം (ലോകം) ഇല്ലാതാവുമ്പോഴും ആത്മാവ്‌ മാറ്റമേതുമില്ലാതെ നിലകൊള്ളുന്നു. 

അവന്‍ എല്ലാറ്റിലും അധിവസിക്കുന്നു. അവന്‍ മറയ്ക്കപ്പെട്ടിരിക്കുന്നു എന്നാല്‍ , സമൃദ്ധമാണുതാനും. അവന്റെ സാന്നിദ്ധ്യംകൊണ്ടു തന്നെ ഈ ജഢമായ ലോകവും അതിലധിവസിക്കുന്നവരും എപ്പോഴും കര്‍മ്മനിരതരായിരിക്കുന്നു. അവന്‍ സാര്‍വ്വഭൌമനും, സര്‍വ്വശക്തനും സര്‍വജ്ഞനുമാകയാല്‍ ചിന്താമാത്രയില്‍ വസ്തുക്കളെ സൃഷ്ടിക്കാന്‍ അവനു കഴിയുന്നു. രാമ: ഈ പരം പൊരുളിനെ അറിയാന്‍ വിജ്ഞാനം കൊണ്ടല്ലാതെ മറ്റൊന്നുകൊണ്ടും കഴിയില്ല. യാഗാദികര്‍മ്മങ്ങള്‍കൊണ്ടും സാദ്ധ്യമല്ല.

ഈ  ആത്മാവ്‌  അടുത്തല്ല, ദൂരേയുമല്ല. അതു ദൂരെ എവിടേയൊ ഇരിക്കുന്ന ഒന്നല്ല; വിരല്‍ ത്തുമ്പത്തല്ലെന്നുമില്ല. ആനന്ദാനുഭവമായി കാണപ്പെടുന്ന അവനെ സ്വരൂപമായേ സാക്ഷാത്കരിക്കാനാവൂ. തപസ്സ്‌, ദാനം, യാഗകര്‍മ്മാദികള്‍ , വ്രതം എന്നിവയൊന്നും ആത്മസാക്ഷാത്കാരത്തിലേയ്ക്കു നയിക്കുകയില്ല. മഹാത്മാക്കളുടെ സത്സംഗവും വേദഗ്രന്ഥങ്ങളുടെ പഠനവും മാത്രമേ അതിനു സഹായകമായുള്ളു. കാരണം അവ മോഹത്തേയും അജ്ഞാനത്തേയും നീക്കുന്നു. മുക്തിപാതയിലെ യാത്രയില്‍ ആത്മാവുമാത്രമേ സത്യവസ്തുവായുള്ളൂ എന്ന അറിവുറച്ചവനും ദുരിതങ്ങളെ മറികടക്കേണ്ടതായുണ്ട്‌. തപസ്സും സന്യാസവും സ്വാര്‍ജ്ജിത പീഢനങ്ങളത്രേ.

മറ്റുള്ളവരെ കബളിപ്പിച്ചുണ്ടാക്കുന്ന ധനംകൊണ്ട്‌ എത്ര ദാനധര്‍മ്മങ്ങള്‍ ചെയ്താലും എന്തു ഫലം? അത്തരം ദാനകര്‍മ്മങ്ങളുടെ ഉചിത  പ്രതിഫലം മാത്രമേ ലഭിക്കൂ. മതപരമായ അനുഷ്ഠാനങ്ങള്‍ ഒരുവന്റെ പൊങ്ങച്ചം കൂട്ടാനേ ഉപകരിക്കൂ. എന്നാല്‍  ആത്മാവിനെക്കുറിച്ചുള്ള അവിദ്യയകറ്റാന്‍ ഒരേ ഒരു പോംവഴിയേയുള്ളു. നിശ്ചയദാര്‍ഢ്യത്തോടെ ഇന്ദ്രിയസുഖങ്ങള്‍ക്കായുള്ള ആസക്തി ഉപേക്ഷിക്കുക എന്ന ഒറ്റമൂലിയാണത്‌.
