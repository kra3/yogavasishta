 
\section{ദിവസം 136}

\slokam{
തസ്മാന്‍മനോനുസന്ധാനം ഭാവേഷു ന കരോതി യ:\\
അന്തശ്ചേതന യത്നേന സ ശാന്തിമധിഗച്ഛതി (3/114/48)\\
}

വസിഷ്ഠന്‍ തുടര്‍ന്നു: മനസ്സ്‌ മോഹ ജഢിലവും, മൂഢത്വം നിറഞ്ഞ ആശയങ്ങളില്‍ അഭിരമിക്കുന്നതുമായാല്‍ അതു ഭ്രമാത്മകമാവും. എന്നാല്‍ മനസ്സ്‌ പ്രബോധാത്മകവും ഉന്നതവുമായ ആശയങ്ങളിലാണഭിരമിക്കുന്നതെങ്കില്‍ അത്‌ പ്രബുദ്ധമാവും. മനസ്സില്‍ അവിദ്യാചിന്ത സന്ദേഹത്തോടെ തൂങ്ങിനില്‍ക്കുന്നുവെങ്കില്‍ ആ ചിന്തകള്‍ ദൃഢീകരിക്കപ്പെടും. എങ്കിലും അത്മസാക്ഷാത്കാരം നേടിയാല്‍പ്പിന്നെ അവിദ്യ അപ്രത്യക്ഷമാവും. മനസ്സ്‌ എന്തുനേടുവാന്‍ ആഗ്രഹിക്കുന്നുവോ അവയെ ലഭ്യമാക്കാന്‍ പഞ്ചേന്ദ്രിയങ്ങള്‍ അവയുടെ മുഴുവന്‍ ഊര്‍ജ്ജവുമെടുത്ത്‌ സദാ സന്നദ്ധമായി നില്‍ക്കുന്നു. "അതുകൊണ്ട്‌ ഏതൊരുവന്‍ മനസ്സിനെ വിഷയചിന്തകളിലും ആശയങ്ങളിലും മുഴുകാന്‍ അനുവദിക്കുന്നില്ലയോ, അവന്‍ ആത്മബോധലാഭത്തിനായി പരിശ്രമിക്കുന്നതിലൂടെ ശാന്തി കണ്ടെത്തുന്നു."

തുടക്കത്തിലേ ഇല്ലാതിരുന്ന ഒന്നും ഇപ്പോഴും സത്തായിട്ട്‌ ഇല്ല. ആദിയിലേ ഉണ്ടായിരുന്നതും ഇപ്പോഴുള്ളതുമായ ഒന്ന്, പരബ്രഹ്മം മാത്രമാണ്‌...  ഇതിനെപ്പറ്റി ധ്യാനിക്കുന്നതിലൂടെ പ്രശാന്തി കൈവരുന്നു. ബ്രഹ്മം, ശാന്തി തന്നെയാണ്‌.. ഒരുവന്‍ മറ്റൊന്നിനേയും ഒരിക്കലും ഒരിടത്തും ധ്യാനിക്കേണ്ടതില്ല. സുഖസമ്പാദനം എന്ന ചിന്തയുടെ വേരുതന്നെ എല്ലാശക്തിയുമുപയോഗിച്ച്‌, ബുദ്ധികുശലതയോടെ നാം അറുത്തുമാറ്റണം. അജ്ഞാനമാണ്‌ വാര്‍ദ്ധക്യത്തിനും മരണത്തിനും ഹേതു. ആശകളും ആസക്തികളും ശാഖോപശാഖകളായി വളര്‍ന്ന് പന്തലിക്കുന്നത്‌ മാനസീകോപാധികള്‍ എന്ന ഈ അജ്ഞാനത്താലാണ്‌...  ഈ ശാഖകളുടെ സ്വഭാവം, 'ഇതെന്റെ ധനം', 'ഇതെന്റെ മക്കള്‍ ' മുതലായവയാണ്‌...

ഈ പുറംതോടായ ശരീരത്തില്‍ 'ഞാന്‍' എന്നറിയപ്പെടുന്ന വസ്തു എവിടെയാണ്‌? രാമ: സത്യത്തില്‍ , 'ഞാന്‍', 'എന്റെ', തുടങ്ങിയ ഭാവങ്ങള്‍ക്ക്‌ അസ്തിത്വമില്ല. ഒരേയൊരാത്മാവു മാത്രമേ സത്യമായുള്ളു. അജ്ഞാനത്തിന്റെ ഫലമായാണ്‌ കയറില്‍ ഒരുവന്‍ പാമ്പിനെ കാണുന്നത്‌..  പ്രബുദ്ധര്‍ക്ക്‌ അത്തരം തെറ്റിദ്ധാരണയില്ലല്ലോ. പ്രബുദ്ധരായവരുടെ ദര്‍ശനത്തില്‍ അനന്താവബോധം മാത്രമേ ഉണ്മയായുള്ളു. രാമ: അജ്ഞതയില്‍ തുടരാതെ ഋഷിയാവൂ. ഈ ലോകമെന്ന കാഴ്ച്ചയെ നല്‍കുന്ന മാനസീകോപാധികളെ നശിപ്പിച്ചാലും. ഒരജ്ഞാനിയേപ്പോലെ, ഈ ശരീരം നിന്റെ സ്വത്വമാണെന്നു ധരിച്ച്‌ ദു:ഖിതനായിരിക്കുന്നതെന്തിന്‌? ഈ ശരീരവും ആത്മാവും ഒരുമിച്ചു സ്ഥിതിചെയ്യുന്നു എന്നു തോന്നുന്നുവെങ്കിലും അവ വേര്‍പിരിക്കാനരുതാത്തതല്ല. ശരീരം മരിക്കുമ്പോള്‍ ആത്മാവ്‌ മരിക്കുന്നില്ല.

രാമ: എന്തത്ഭുതമാണെന്നു നോക്കൂ, മനുഷ്യര്‍ പരബ്രഹ്മം മാത്രമേയുള്ളൂ എന്ന സത്യം മറന്ന് അസത്യമായ, അവിദ്യയാണ്‌ ഉണ്മയെന്ന് വിശ്വസിക്കുകയാണ്‌...  രാമ: ഈ അവിദ്യയുടെ അസ്തിത്വത്തെപ്പറ്റിയുള്ള മൂഢവിശ്വാസം നിന്നില്‍ വേരുറക്കാതിരിക്കട്ടെ. ബോധമണ്ഡലം മലീമസമായാല്‍ അത്‌ അന്തമില്ലാത്ത ദുരിതാനുഭവങ്ങളെ ക്ഷണിച്ചുവരുത്തുന്നു. അജ്ഞാനം ഉണ്മയല്ലെങ്കിലും അതിന്‌ ശരിയായ ദുരിതാനുഭവം നല്‍കാന്‍ കഴിയും. അവിദ്യകൊണ്ടാണ്‌ മരീചികപോലെയുള്ള ഭ്രമങ്ങള്‍ ഉണ്ടാവുന്നത്‌...  ആകാശഗമനം, പറക്കല്‍ , തുടങ്ങി വിവിധങ്ങളായ ദൃശ്യവിസ്മയങ്ങള്‍ , സ്വര്‍ഗ്ഗനരകങ്ങളിലെ അനുഭവങ്ങള്‍ എല്ലാം അവിദ്യയാല്‍ ഉണ്ടാവുന്നു. അതുകൊണ്ട്‌ രാമ:, ദ്വന്ദഭാവനകള്‍ക്കു കാരണമായ മാനസീകോപാധികളെ തീര്‍ത്തുമുപേക്ഷിച്ച്‌ സര്‍വ്വ സ്വതന്ത്രനായാലും. അങ്ങിനെ നിനക്ക്‌ സമാനതകളില്ലാത്ത ഉത്കൃഷ്ടത സ്വായത്തമാക്കാം. 

