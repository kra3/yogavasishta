\newpage
\section{ദിവസം 026}

\slokam{
ഏവം കർമ്മസ്ഥകർമ്മാണി കർമ്മപ്രൗഢ സ്വവാസനാ\\
വാസനാ മാനസോ നാന്യാ മനോഹി പുരുഷ: സ്മൃത: (2/9/17)\\
}

രാമന്‍ ചോദിച്ചു: ഭഗവന്‍, അങ്ങ്‌ സത്യജ്ഞാനിയാണ്‌. ആളുകള്‍ ദൈവം, വിധി, ഈശ്വരന്‍ എന്നൊക്കെപ്പറയുന്ന വസ്തു എന്താണെന്നു പറഞ്ഞു തന്നാലും.

വസിഷ്ഠന്‍ പറഞ്ഞു: ഒരാളുടെ പൂര്‍വ്വകര്‍മ്മങ്ങളുടെ ഫലം അവന്‌ നല്ലതോ ചീത്തയോ ആയ അനുഭവങ്ങളാവുന്നതാണ്‌ വിധി, നിയോഗം, ദൈവം എന്നൊക്കെ അറിയപ്പെടുന്നത്‌. ആ ഫലാനുഭവങ്ങളുടെ നന്മയും തിന്മയും ദൈവഗത്യമാണെന്നു ജനങ്ങള്‍ പറയുന്നു. ഒരു വിത്തു മുളപൊട്ടി വലുതായി വന്മരമാവുമ്പോള്‍ അത്‌ ദൈവത്തിന്റെ ചെയ്തിയാണെന്നു കരുതപ്പെടുന്നു. പക്ഷേ എനിക്കു തോന്നുന്നത്‌ നിയോഗമെന്നത്‌ ഒരുവന്റെ സ്വന്തം കര്‍മ്മങ്ങളുടെ സഫലീകരണം തന്നെയാണെന്നാണ്‌. മനുഷ്യമനസ്സില്‍ അനേകം വാസനകള്‍ ലീനമായിരിക്കുന്നു. അവയാണ്‌ മനസാ-വാചാ-കര്‍മ്മണാ ഉള്ള പ്രവര്‍ത്തനങ്ങള്‍ക്ക്‌ പ്രചോദനമാവുന്നത്‌. തീര്‍ച്ചയായും ഒരുവന്റെ പ്രവര്‍ത്തികള്‍ വാസനകളുടെ അടിസ്ഥാനത്തിലാണ്‌ വര്‍ത്തിക്കുന്നത്‌.

"അങ്ങിനെയാണ്‌ കര്‍മ്മങ്ങളുടെ നിജസ്ഥിതി: കര്‍മ്മം എന്നത്‌ വാസനകളില്‍ വെച്ചേറ്റവും പ്രബലമായ വാസനാവിശേഷത്തിന്റെ പ്രകടിതഭാവമാണ്‌. ഈ വാസനകള്‍ മനസ്സില്‍ നിന്നു ഭിന്നമല്ല. മനുഷ്യന്‍ മനസ്സുതന്നെയാണ്‌" 

മനസ്സ്‌, വാസനകള്‍, കര്‍മ്മം, വിധി (ദൈവം) എന്നീ തരംതിരിക്കലുകള്‍ സത്യത്തിലുണ്ടോ എന്നാര്‍ക്കും അറിയാന്‍ കഴിയില്ല. അതിനാല്‍ ജ്ഞാനികള്‍ അവയെ പ്രതീകാത്മകമായി സൂചിപ്പിക്കുന്നു. 

രാമന്‍ വീണ്ടും ചോദിച്ചു: മഹാത്മന്‍, പൂര്‍വ്വജന്മങ്ങളില്‍ നിന്നുമാര്‍ജ്ജിച്ച വാസനകളാണ്‌ എന്നേക്കൊണ്ട്‌ കര്‍മ്മംചെയ്യിക്കുന്നതെങ്കില്‍ കര്‍മ്മങ്ങളില്‍ എനിക്കെവിടെയാണ്‌ സ്വാതന്ത്ര്യം?

വസിഷ്ഠന്‍ പറഞ്ഞു: രാമ: പൂര്‍വ്വജന്മാര്‍ജ്ജിത വാസനകള്‍ രണ്ടുതരമുണ്ട്‌. നിര്‍മ്മലവാസനകളും മലിനവാസനകളും. നിര്‍മ്മലമായവ നമ്മെ മുക്തിയിലേയ്ക്കു നയിക്കുന്നു. മറ്റേത്‌ കൂടുതല്‍ ബന്ധനങ്ങളിലേയ്ക്കും. നാം ബോധസ്വരൂപമായ ആത്മാവാണ്‌. വെറും ഭൌതീകവസ്തുവല്ല. നിന്റെ കര്‍മ്മങ്ങള്‍ ക്കു പ്രചോദനമായിരിക്കുന്നത്‌ മറ്റാരുമല്ല. നീ തന്നെയാണ്‌. അതുകൊണ്ട്‌ നിന്റെ നിര്‍മ്മലവാസനകളെ മലിനവാസനകളെ അപേക്ഷിച്ച്‌ പ്രബലമാക്കാന്‍ നിനക്കു സ്വാതന്ത്രമുണ്ട്‌. മനസ്സിനെ ശരിയായ വഴിയ്ക്കുതിരിച്ച്‌, മലിനവാസനകളെ പതുക്കെപ്പതുക്കെ ഉപേക്ഷിക്കണം. കാരണം അവയെ പെട്ടെന്നുപേക്ഷിച്ചാലുള്ള ഫലം ചിലപ്പോള്‍ തീക്ഷ്ണമായേക്കാം. ഉപയോഗിക്കാതിരിക്കുന്നതിനാല്‍ മലിനവാസനകള്‍ ക്രമേണ ശക്തിഹീനമാവും. അതുപോലെ സദ്‌വാസനകള്‍ കര്‍മ്മമാര്‍ഗ്ഗങ്ങളിലൂടെ ആവര്‍ത്തിച്ചുപയോഗിക്കുന്നതിലൂടെ അവ പ്രബലപ്പെടുകയും ചെയ്യും. അങ്ങിനെ നീ ചെയ്യുന്ന പ്രവര്‍ത്തികള്‍ എല്ലാം നിര്‍മ്മലമായ വാസനകളുടെ പ്രകടനമായ സദ്‌കര്‍മ്മങ്ങള്‍ മാത്രമാവും. അങ്ങിനെ ആദ്യം ദുര്‍വ്വാസനകളെ നിര്‍മ്മാര്‍ജ്ജനംചെയ്ത്‌ സദ്കര്‍മ്മങ്ങളില്‍ പാകം വരുമ്പോള്‍ ആ പ്രവര്‍ത്തികളെപ്പോലും ഉപേക്ഷിക്കാം. അപ്പോള്‍ നിനക്ക്‌ സദ്‌വാസനകളില്‍ നിന്നുദിക്കുന്ന ബോധപ്രഭയില്‍ പരമസത്യത്തെ സാക്ഷാത്കരിക്കാം.
