\section{ദിവസം 240}

\slokam{
ത്രിഭുവനഭാവനാഭിരാമകോശം \\
സകലകളങ്കഹരം പരം പ്രകാശം \\
അശരണശരണം ശരണ്യമീശം \\
ഹരിമജമച്യുതമീശ്വരം പ്രപദ്ധ്യേ (5/33/19) \\
}

വസിഷ്ഠന്‍ തുടര്‍ന്നു: അങ്ങിനെ ദേവന്മാരെ സമാധാനിപ്പിച്ച് വിഷ്ണുഭഗവാന്‍ അപ്രത്യക്ഷനായി. ദേവന്മാര്‍ അവരവരുടെ സവിധങ്ങളിലേയ്ക്ക് മടങ്ങി. അപ്പോഴെയ്ക്ക് അവര്‍ പ്രഹ്ലാദനുമായി സൌഹൃദത്തിലുമായി. പ്രഹ്ലാദന്‍ ദിവസവും മനസാ വാചാ കര്‍മ്മണാ വിഷ്ണുപൂജ ചെയ്തു വന്നു. 

ഇങ്ങിനെയുള്ള പൂജയുടെ തല്‍ക്ഷണഫലമെന്ന നിലയ്ക്ക് പ്രഹ്ലാദനില്‍ പവിത്രഗുണങ്ങളായ വിവേകം, അനാസക്തി എന്നിവ സംജാതമായി. അദ്ദേഹം സുഖം തേടിയില്ല. ആ മനസ്സ് അതിനായി ആഗ്രഹിച്ചതേയില്ല. സുഖാസക്തി ഉപേക്ഷിച്ചതോടെ പ്രഹ്ലാദന്റെ മനസ്സിനുപാധിയായി മറ്റൊന്നും ഉണ്ടായിരുന്നില്ല. പ്രഹ്ലാദന്റെ ഈയവസ്ഥ അറിഞ്ഞ വിഷ്ണുഭഗവാന്‍ പാതാളലോകത്ത് അദ്ദേഹം പൂജ ചെയ്തിരുന്ന സ്ഥലത്തെത്തി. ഭഗവാനെ കണ്ട പ്രഹ്ലാദന്‍ അതീവ സന്തോഷത്തോടെ വീണ്ടും വീണ്ടും പൂജ തുടര്‍ന്നു. പ്രഹ്ലാദന്‍ പ്രാര്‍ത്ഥിച്ചു:

“മൂന്നുലോകങ്ങളും വിരാജിക്കുന്ന ആ ഭഗവാനില്‍ ഞാന്‍ അഭയം തേടുന്നു. എല്ലാവിധ അന്ധകാരങ്ങളും അജ്ഞാനങ്ങളും ഇല്ലാതാക്കുന്ന പരമപ്രകാശമാണാ ഭഗവാന്‍. അഗതികള്‍ക്കൊരാശ്രയമായുള്ളതും ഭഗവാന്‍ മാത്രം. ഇനിയും ജനിച്ചിട്ടില്ലാത്ത (അതിനാല്‍ത്തന്നെ മരണമില്ലാത്ത) ആ ഭഗവാനിലുള്ള അഭയം മാത്രമേ അഭികാമ്യമായുള്ളു. എക്കാലവും ഉറപ്പുള്ള സംരക്ഷ ആ ഭഗവല്‍സവിധത്തില്‍ മാത്രമേയുള്ളൂ.
     
നീലക്കല്ലുപോലെയോ നീലത്താമാരപോലെയോ പ്രഭാഭാസുരമാണവിടുത്തെ രൂപം. ശിശിരകാലത്തെ തെളിഞ്ഞ നീലാകാശത്തിന്റെ കിരീടം പോലെ ആ തിരുരൂപമെത്ര  പ്രോജ്ജ്വലം! കൈകളില്‍ വിഷ്ണുഛിഹ്നങ്ങളുമായി നില്‍ക്കുന്ന അവിടുത്തെ ഞാന്‍ ആശ്രയിക്കുന്നു. വേദശാസ്ത്രങ്ങളുല്‍ഘോഷിക്കുന്ന സത്യമാണവിടുത്തെ ശബ്ദം. അവിടുത്തെ നാഭീപങ്കജത്തിലത്രേ സൃഷ്ടാവായ ബ്രഹ്മദേവന്‍ നിലകൊള്ളുന്നത്! അവിടുന്നു സര്‍വ്വഭൂതങ്ങളുടേയും ഹൃദയനിവാസിയത്രേ!

ആരുടെ കാല്‍നഖങ്ങളാണോ ആകാശതാരകള്‍ പോലെ തിളക്കമാര്‍ന്നത്, ആരുടെ മുഖകമലമാണോ ചന്ദ്രബിംബംപോലെ പുഞ്ചിരിപൂണ്ടത്, ആരുടെ ഹൃദയത്തിലാണോ ഗംഗാനദിയൊഴുക്കുന്നതും പ്രഭാകിരണങ്ങള്‍ പേറുന്നതുമായ രത്നക്കല്ലുകള്‍ പ്രശോഭിക്കുന്നത്, ആരാണോ ശരല്‍ക്കാലഗഗനത്തെ വസ്ത്രമാക്കിയത്, ആ ഭഗവാനാണെനിക്കഭയം.

ആരുടെയുള്ളിലാണോ ഈ സ്ഥൂലമായ വിശ്വപ്രപഞ്ചം കുറവുകളൊന്നുമില്ലാതെ നിലകൊള്ളുന്നത്, ആരാണോ അജനും മാറ്റങ്ങള്‍ക്കു വിധേയമല്ലാത്തതായുമുള്ളത്, ആരുടെ ദേഹമാണോ ഐശ്വര്യഗുണങ്ങളാല്‍ നിര്‍മ്മിതമായുള്ളത്, ആരാണോ ആലിലയില്‍ പള്ളികൊള്ളുന്നത്, ആ പരമപദത്തെ ഞാന്‍ സമാശ്രയിക്കുന്നു. 

അസ്തമയസൂര്യന്റെ കാന്തിപോലെ പരമസൌന്ദര്യം വഴിയുന്ന ലക്ഷീദേവി ആരുടെ വാമഭാഗമാണോ അലങ്കരിക്കുന്നത്, ആ ഭഗവാനാണെനിക്കാശ്രയം. മൂലോകങ്ങളിലെയും താമരപ്പൂക്കള്‍ക്ക് സൂര്യനെന്നപോലെ വിരാജിക്കുന്ന, അജ്ഞാനാന്ധകാരം പാടേ മാറ്റുന്ന  മണിവിളക്കായി നിലകൊള്ളുന്ന, അനന്താവബോധമായ, പ്രപഞ്ചദു:ഖങ്ങളെയും ദുരിതങ്ങളേയും ഇല്ലാതാക്കുന്ന ആ ഭഗവാനില്‍ ഞാന്‍ അഭയം തേടുന്നു.
