 
\section{ദിവസം 039}

\slokam{
ന ദൃശ്യമസ്തി സദ്രൂപം ന ദൃഷ്ടാ ന ച ദർശനം\\
ന ശൂന്യം ന ജഡം നോ ചിച്ഛാന്തമേവേദമാതതം (3/4/70)\\
}

വാല്‍മീകി പറഞ്ഞു: മഹര്‍ഷിയുടെ വാക്കുകളെക്കുറിച്ചു ധ്യാനിക്കാനും ലോകത്തിന്റെ മറ്റുഭാഗങ്ങളില്‍ വെളിച്ചം വീശാനുമായി സൂര്യന്‍ പടിഞ്ഞാറുള്ള മലകള്‍ക്കുപിന്നിലേയ്ക്കു മറഞ്ഞു. സഭ വൈകുന്നേരത്തെ പ്രാത്ഥനകള്‍ക്കായി പിരിഞ്ഞു. അടുത്ത പ്രഭാതത്തില്‍ എല്ലാവരും സഭയില്‍ സന്നിഹിതരായി. രാമന്‍ പറഞ്ഞു: മനസ്സെന്നാല്‍ എന്തെന്ന് ദയവായി വിശദമാക്കിയാലും.

വസിഷ്ഠന്‍ പറഞ്ഞു: ശൂന്യമായ ഇടത്തിന്‌ ആകാശം എന്നു പേര്‌. അതുപോലെ മനസ്സെന്നത്‌  'ഒന്നുമില്ലാത്ത' നിശ്ശൂന്യതയാണ്‌. മനസ്സ്‌ സത്യമാണെങ്കിലും അല്ലെങ്കിലും വസ്തുക്കളെപ്പറ്റിയുള്ള അവബോധമാണത്‌. മനസ്സ്‌ , ചിന്തകളാണ്‌ - രണ്ടുംതമ്മില്‍ വ്യത്യാസമേതുമില്ല. ആദ്ധ്യാത്മികശരീരത്താല്‍ മൂടിയിരിക്കുന്ന ആത്മാവാണ്‌ മനസ്സ്‌. ഭൌതീകശരീരത്തെ അല്ലെങ്കില്‍ വസ്തുബോധത്തെ നിലനിര്‍ത്തുന്നത്‌ മനസ്സാണ്‌. അവിദ്യ, സംസാരം, മനോ-വസ്തു, ബന്ധനം, അന്ധകാരം, ജഢം എന്നെല്ലാം അറിയപ്പെടുന്നത്‌ ഈ ഒന്നിനെത്തന്നെയാണ്‌. അനുഭവം തന്നെയാണ്‌ മനസ്സ്‌. പഞ്ചേന്ദ്രിയങ്ങളിലൂടെ അറിയുന്നതാണ്‌ (ദര്‍ശിക്കുന്നതാണ്‌) മനസ്സ്‌.

ഈ വിശ്വം മുഴുവനും ഓരോ പരമാണുവിലും കുടികൊള്ളുന്ന ബോധത്തില്‍നിന്നും വിഭിന്നമല്ല. ഓരോ ആഭരണങ്ങളുടേയും സത്ത സ്വര്‍ണ്ണം തന്നെയാണല്ലോ. സ്വര്‍ണ്ണത്തില്‍ ആഭരണങ്ങളുടെ 'സാദ്ധ്യത' ലീനമാണ്‌. അതുപോലെ വസ്തുക്കള്‍ ദൃഷ്ടാവില്‍ ലീനമത്രേ. (വിഷയം വിഷയിയില്‍ സാധ്യതകളായി കുടികൊള്ളുന്നു.) വിഷയിയില്‍ നിന്നും വിഷയവസ്തുവിനെ തീര്‍ത്തും നീക്കംചെയ്താല്‍ ബോധം മാത്രം അവശേഷിക്കുന്നു. ഇതില്‍ സാധ്യതയായിപ്പോലും വിഷയവസ്തുവിന്റെ സാന്നിദ്ധ്യമില്ല. ഈ സാക്ഷാത്കാരം മൂലം ഇഷ്ടാനിഷ്ടങ്ങള്‍ , രാഗദ്വേഷങ്ങള്‍ തുടങ്ങിയ ദൂഷ്യങ്ങളും ഞാന്‍ , നീ, ലോകം തുടങ്ങിയ തെറ്റിദ്ധാരണകളും ഇല്ലാതാവുന്നു. വസ്തു അവബോധം ഒരു വാസനയായിപ്പോലും അവശേഷിക്കാതിരിക്കുമ്പോള്‍ പരിപൂര്‍ണ്ണ സ്വാതന്ത്ര്യമാണ്‌. 

രാമന്‍ ചോദിച്ചു: മഹാത്മന്‍ , വിഷയവസ്തു സത്യമാണെങ്കില്‍ അത്‌ നശിക്കുകയില്ല. എന്നാല്‍ അത്‌ അസത്താണെങ്കിലും നാം അതിനെ അങ്ങിനെത്തന്നെ അറിയുന്നില്ലെങ്കില്‍ എങ്ങിനെ  നമുക്കതിനെ അതിജീവിക്കാനാവും?

വസിഷ്ഠന്‍ പറഞ്ഞു: രാമ: മഹാത്മാക്കള്‍ ഈ പ്രശ്നത്തിനു പരിഹാരം കണ്ടെത്തിയതായി നമുക്കറിയാം. ആകാശം തുടങ്ങിയ ബാഹ്യവസ്തുക്കള്‍ , 'ഞാന്‍ ' മുതലായ മനോവിഷയങ്ങള്‍ എന്നിവ നിലനില്‍ക്കുന്നത്‌ നാമങ്ങളില്‍ മാത്രമാണ്‌. "വാസ്തവത്തില്‍ വസ്തുപ്രപഞ്ചമോ, വിഷയാവബോധം കൊള്ളുന്ന ഞാനോ, അവബോധമെന്ന പ്രതിഭാസമോ, ശൂന്യതയോ, ജഢത്വമോ ഒന്നും ഉണ്മയില്‍ നിലനില്‍പ്പുള്ളതല്ല. വിശ്വബോധം (ചിത്ത്‌) മാത്രമേ ഉണ്മയായിട്ടുള്ളൂ." ഇതിനുള്ളില്‍ മനസ്സാണ്‌ ഇന്ദ്രജാലത്തിലെന്നവണ്ണം നാനാവിഷയങ്ങളും, വൈവിധ്യമാര്‍ന്ന കര്‍മ്മാനുഭവങ്ങളും, ബന്ധനത്തിലാണെന്ന തോന്നലും, ബന്ധവിമോചനത്തിനുള്ള ആഗ്രഹങ്ങളും ഉണ്ടാക്കുന്നത്‌. 
