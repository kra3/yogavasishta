 
\section{ദിവസം 070}

\slokam{
ക ഇവാസ്മിൻപരിത്രാതാ സ്യാദിത്യാദീനവീക്ഷിതൈ:\\
ഉത്പലാലീവ വർഷദ്ഭി: പരിരോദിത സൈനികാ: (3/43/59)\\
}

സരസ്വതി പറഞ്ഞു: രാജാവേ, അങ്ങീ യുദ്ധത്തില്‍ ചരമമടയും. എന്നിട്ട്‌ അങ്ങയുടെ പഴയ സാമ്രാജ്യം തിരിച്ചുപിടിക്കും. ഈ ശരീരത്തിന്റെ മരണശേഷം അങ്ങ്‌ മുന്‍പുണ്ടായിരുന്ന നഗരത്തില്‍ അങ്ങയുടെ മകളുടേയും മന്ത്രിമാരുടേയും അടുത്തുവരും. ഞങ്ങള്‍ ഇപ്പോള്‍ തിരികെപ്പോകുന്നു. നിങ്ങളെല്ലാവരും സമയമാകുമ്പോള്‍ ഞങ്ങളെ അനുഗമിക്കുന്നതാണ്‌. ഒരാനയുടേയും, കുതിരയുടേയും ഒട്ടകത്തിന്റേയും ഗമനരീതികള്‍ ഒട്ടു വ്യത്യസ്തമാണല്ലോ.

വസിഷ്ഠന്‍ തുടര്‍ന്നു: സരസ്വതി ദേവി ഇതു പറഞ്ഞുകൊണ്ടിരിക്കുമ്പോള്‍ ഒരു ദൂതന്‍ പാഞ്ഞുവന്ന് തലസ്ഥാനനഗരിയെ ശത്രുസൈന്യം കയറി നശിപ്പിക്കുന്നതായി അറിയിച്ചു. കൊള്ളിവെയ്പ്പുമൂലം നഗരം എരിഞ്ഞടങ്ങുകയായിരുന്നു. നഗരത്തില്‍ കൊള്ളക്കാര്‍ ഉറക്കെ വിളിച്ചാര്‍ത്തുകൊണ്ട്‌ കവര്‍ച്ച ചെയ്തു. രണ്ടു ദിവ്യവനിതകളും, രാജാവും മന്ത്രിമാരും എല്ലാം ഒരു വാതായനത്തിനടുത്തേയ്ക്ക്‌ നീങ്ങി ഭയാനകമായ ആ നഗരക്കാഴ്ച്ചകള്‍ കണ്ടു. നഗരം ഒരു പുകയുടെ ഇരുണ്ട ഗോളത്തിനകത്തായി. ആകാശത്തുനിന്നും തീയാണു വര്‍ഷിച്ചിരുന്നത്‌. അര്‍ദ്ധചന്ദ്രാകൃതിയിലുള്ള ചാട്ടുളികള്‍ ആകാശത്തില്‍ മിന്നലുണ്ടാക്കി. വലിയ പാറകള്‍ പോലുള്ള ആയുധഗോളങ്ങള്‍ വീടുകള്‍ക്കുമുകളില്‍ വീണ്‌ അവയെ തരിപ്പണമാക്കി. വീഥികളേയും അവ നശിപ്പിച്ചു. നഗരവാസികളുടെ ദയനീയമായ കരച്ചില്‍ അവരെല്ലാം കേട്ടു. വിലാപങ്ങളും കണ്ണീരും സ്ത്രീകളുടേയും കുട്ടികളുടേയും അലമുറകളും എങ്ങും നിറഞ്ഞു. ഒരാള്‍ ഇങ്ങിനെ വിലപിച്ചു: 'കഷ്ടം, ആ സ്ത്രീയുടെ അച്ഛന്‍ മരിച്ചു; അമ്മയും, ജ്യേഷ്ഠനും കുഞ്ഞുമകനും എല്ലാവരും പോയി എന്നിട്ടും അവള്‍ ബാക്കിയായി. അവളുടെ ഹൃദയം എരിഞ്ഞുരുകുകയാണ്‌.' മറ്റൊരാള്‍ കൂകിയാര്‍ത്തു:'വേഗം ആ വീട്ടില്‍ നിന്നിറങ്ങിയോടൂ, അതിപ്പോള്‍ നിലം പൊത്തും'.'നോക്കൂ എല്ലാ വീടുകള്‍ക്കുമുകളിലും ആകാശത്തുനിന്നും ആയുധവര്‍ഷം തന്നെയാണ്‌' എന്നു മറ്റൊരാള്‍ . ഗൃഹങ്ങള്‍ക്ക്‌ ചുറ്റുമുള്ള മരങ്ങള്‍ തീയിലെരിഞ്ഞു ചാമ്പലായി. അവിടമെല്ലാം തരിശായിത്തീര്‍ന്നു

യുദ്ധസ്ഥലത്തുനിന്നും ആനകളേപ്പോലെയുള്ള ചാട്ടുകള്‍ ആകാശത്തുയര്‍ന്നു പൊങ്ങി. അവയില്‍ നിന്നും തീ പെയ്തു. എല്ലാ നഗരവീഥികളും അടഞ്ഞുകിടന്നു. സ്വന്തം കുടുംബത്തോടുള്ള ഒട്ടല്‍ നിമിത്തം ആളുകള്‍ തങ്ങളുടെ ഭാര്യമാരേയും കുട്ടികളേയും തിരഞ്ഞ്‌ തീപിടിച്ച വീടുകളില്‍ നിന്നും ഓടിപ്പോകാതെ തങ്ങിനിന്നു. രാജവനിതകളേപ്പോലും ശത്രുഭടന്മാര്‍ പിടികൂടി ഉപദ്രവം തുടങ്ങി. വിലപിച്ചു കണ്ണീരൊഴുക്കുന്ന അവര്‍ എന്തുചെയ്യണമെന്നറിയാതെ ഉഴറി. "കഷ്ടം ഞങ്ങള്‍ എന്തു ചെയ്യട്ടെ! ഈ ദുരിതത്തില്‍നിന്നും ആരാണു ഞങ്ങളെ രക്ഷിക്കുക?. അവരെ ഭടന്മാര്‍ അപ്പോഴേക്കും വളഞ്ഞു കഴിഞ്ഞിരുന്നു". അപ്രകാരമാണ്‌ പരമാധികാരത്തിന്റേയും സാമ്രാജ്യങ്ങളുടെയും രാജ്യങ്ങളുടേയും കീര്‍ത്തി പരക്കുന്നത്‌!

* അത്യാധുനീക യുദ്ധസന്നാഹങ്ങളുമായി ഈ യുദ്ധത്തിനുള്ള സാമ്യം നോക്കുക. നഗരവാസികള്‍ ക്കുമേല്‍പ്പോലും ബോംബുവര്‍ഷിക്കല്‍ എന്നത്‌ അത്ര പുതിയ ഏര്‍പ്പാടൊന്നുമല്ല.!

