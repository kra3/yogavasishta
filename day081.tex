 
\section{ദിവസം 081}

\slokam{
യഥാ വാസനയാ ജന്തോര്‍വിഷമപ്യമൃതായതേ\\
അസത്യ: സത്യതാമേതി പദാര്‍ത്ഥോ ഭാവനാത് തഥാ (3/56/31)\\
}

വസിഷ്ഠന്‍ തുടര്‍ന്നു: കടപുഴകിവീഴാന്‍ പോകുന്ന വൃക്ഷത്തില്‍ നിന്നും പക്ഷികള്‍ പറന്നകലുന്നതുപോലെ വിഥുരഥന്റെ ശരീരത്തില്‍ നിന്നും ജീവന്‍ വിട്ടുപോയി. അദ്ദേഹത്തിന്റെ പ്രജ്ഞ, സൂക്ഷ്മശരീരമായി ആകാശത്തേയ്ക്കുയര്‍ന്നു. ലീലയും സരസ്വതിയും ഇതുകണ്ട്‌ ആ ജീവനെപിന്തുടര്‍ന്നു. ഏതാനും നിമിഷങ്ങള്‍ കഴിഞ്ഞപ്പൊള്‍ ആ സൂക്ഷ്മശരീരം സചേതനമായി, മരണശേഷമുണ്ടായ താല്‍ക്കാലികമായ അബോധാവസ്ഥയില്‍ നിന്നും  പുറത്തുവന്നു. രാജാവിന്റെ ചിന്തയില്‍ തന്റെ സ്ഥൂലശരീരം ബന്ധുക്കളാല്‍ചുറ്റപ്പെട്ട്‌ ഉപചാരപൂര്‍വ്വം സംസ്കരിക്കാനായി കിടത്തിയിരിക്കുന്നതു കണ്ടു. അദ്ദേഹം (സൂക്ഷ്മദേഹം) വീണ്ടും യാത്രചെയ്ത്‌ യമരാജന്റെ സവിധത്തിലെത്തി. രാജാവ്‌ പാപകര്‍മ്മങ്ങള്‍ ഒന്നും ആര്‍ജ്ജിച്ചിട്ടില്ല,  അതിനാല്‍ അദ്ദേഹത്തെ പൂര്‍വ്വ ശരീരത്തിലേയ്ക്ക്‌ കടക്കാന്‍ അനുവദിച്ചിരിക്കുന്നുവെന്നും യമന്‍ തന്റെ കിങ്കരന്മാരെ അറിയിച്ചു. പദ്മ രാജാവിന്റെ ദേഹം എണ്ണത്തോണിയില്‍ ഭദ്രമായിസൂക്ഷിച്ചിരുന്നുവല്ലോ. ക്ഷണത്തില്‍ ഈ സൂക്ഷ്മശരീരം, പദ്മ രാജന്റെ കൊട്ടാരത്തില്‍പ്രവേശിച്ച്‌ രജാവിന്റെ ദേഹം കിടത്തിയ   അറയിലെത്തി. തീര്‍ച്ചയായും പദ്മ രാജാവിന്റെ അഹംകാരവുമായിബന്ധപ്പെട്ടാണ്‌ വിഥുരഥനുമായുള്ള ചാര്‍ച്ച ഉണ്ടായത്‌. വിദേശ  ങ്ങളില്‍ ദൂരെയാത്ര ചെയ്യുന്നയാള്‍ക്ക് നാട്ടില്‍ തന്റെ സമ്പത്ത്  കുഴിച്ചിട്ടയിടം എപ്പോഴും ഓര്‍മ്മയിലുണ്ടാകും.   എവിടെ യാത്രയില്‍  ആയിരുന്നാലും അതിനോട്‌ അയാള്‍ക്ക്  മമതയുണ്ടാകുമല്ലോ. അതുപോലെയാണ്‌ വിഥുരഥന്റെ അവസ്ഥയും. 

രാമന്‍ ചോദിച്ചു: മഹാത്മന്‍, ഒരുവന്റെ ബന്ധുക്കള്‍ അവനുവേണ്ടി മരണാനന്തര കര്‍മ്മങ്ങള്‍ ശരിയായി ചെയ്തില്ലെങ്കില്‍ അയാള്‍ക്ക്‌ എങ്ങിനെയാണ്‌ സൂക്ഷ്മശരീരം ലഭിക്കുക?

വസിഷ്ഠന്‍ പറഞ്ഞു; കര്‍മ്മങ്ങള്‍ ശരിയായി ചെയ്താലുമില്ലെങ്കിലും, ദിവംഗതനായ ആള്‍ കര്‍മ്മങ്ങള്‍ ശരിയായി ചെയ്തു എന്നു വിശ്വസിക്കുന്ന പക്ഷം അയാള്‍ക്ക്‌ സൂക്ഷ്മശരീരത്തിന്റെ പ്രയോജനം ലഭിക്കുന്നതാണ്‌. ഒരുവന്റെ ബോധം എങ്ങിനെയോ അവന്‍ അങ്ങിനെയായിത്തീരുന്നു. ഇതെല്ലാവര്‍ക്കുമറിയാവുന്ന സത്യമത്രേ. പദാര്‍ത്ഥങ്ങള്‍ (വസ്തുക്കള്‍ , വിഷയങ്ങള്‍ ) ഉണ്ടാവുന്നത്‌ ഒരാളുടെ ഭാവനയിലാണ്‌. പദാര്‍ത്ഥങ്ങളില്‍നിന്നും ഭാവനകളുണ്ടാവുകയും ചെയ്യുന്നു. "ഒരുവന്റെ ഭാവനകൊണ്ട്‌ വിഷം അമൃതായി മാറുന്നു. അയാഥാര്‍ത്ഥ്യമായ വസ്തു യാഥാര്‍ഥ്യമാണെന്നു തോന്നുന്നു. ഇതെല്ലാം രൂഢമൂലമായ വിശ്വാസം മൂലം സംജാതമാവുകയാണ്‌."

കാരണമില്ലാതെ യാതൊരു കാര്യവും ഇന്നുവരെ ഒരിടത്തുമുണ്ടായിട്ടില്ല. അതുകൊണ്ട്‌ ഈ ഭാവനകളും ചിന്തകളും 'ഉള്ളതാണെന്ന്' പറയുകവയ്യ. എന്നാല്‍ അനന്താവബോധത്തിന്റെ ആ ഒരു 'അഹൈതുകഹേതു- അകാരണകാരണം' കൊണ്ടുമാത്രമാണ്‌ എന്തെങ്കിലും സൃഷ്ടിക്കപ്പെടുകയോ ഉദിച്ചുയരുകയോ ചെയ്യുന്നുള്ളു.

ഒരുകാര്യം പക്ഷെ വ്യക്തമാണ്‌. ബന്ധുജനങ്ങള്‍ വിശ്വാസപൂര്‍വ്വം നടത്തുന്ന മരണാനന്തരകര്‍മ്മങ്ങള്‍ ജീവപ്രജ്ഞയെ മുന്നോട്ടുള്ള ഗമനത്തിനായി സഹായിക്കുന്നുണ്ട്‌. മരിച്ചവ്യക്തി അതീവ ദുഷ്ടത നിറഞ്ഞ ഒരാളായിരുന്നുവെങ്കില്‍ കര്‍മ്മങ്ങള്‍ കൊണ്ട്‌ വലിയപ്രയോജനമൊന്നുമില്ല. 

നമുക്ക്‌ പദ്മ രാജാവിന്റെ കൊട്ടാരത്തിലേയ്ക്കു പോവാം. ആദ്യം ഞാന്‍പറഞ്ഞ പോലെ ലീലയും സരസ്വതിയും മനോഹരമായ ആകൊട്ടാരത്തിലേയ്ക്ക്‌ വീണ്ടും പ്രവേശിച്ചു.അവിടെ എണ്ണത്തോണിയില്‍ രാജാവിന്റെ ദേഹം സൂക്ഷിച്ചിരുന്നു. എല്ലാ രാജസേവകരും അവിടെ ദീര്‍ഘനിദ്രയിലായിരുന്നു. 

