\section{ദിവസം 155}

\slokam{
ജ്ഞാനസ്യ ച ദേഹസ്യ യാവദ്ദേഹമയം ക്രമ:\\
ലോകവദ്വ്യവഹാരോയം സക്ത്യാസക്ത്യാധവാ സദാ (4/15/35)\\
}

വസിഷ്ഠൻ തുടർന്നു: ഭൃഗുപുത്രനായ ശുക്രന്റെ ജീർണ്ണിച്ചുവരണ്ട ദേഹമിരിക്കുന്നയിടത്ത് അവരെത്തി. അതുകണ്ട് ശുക്രൻ വിലപിച്ചു: ദേവകന്യകകളും അപ്സരസ്സുകളും പുകഴ്ത്തി ബഹുമാനിച്ചിരുന്ന ദേഹമിതാ കൃമികീടങ്ങളുടെ വാസസ്ഥലമായിരിക്കുന്നു. ചന്ദനം പൂശിയിരുന്ന ദേഹമിപ്പോൾ പൊടിമൂടിയിരിക്കുന്നു. ശരീരമേ! നീയിപ്പോൾ ശവമെന്നാണറിയപ്പെടുന്നത്. ഭയാനകമാണെനിക്ക് ഈ കാഴ്ച്ച. വന്യമൃഗങ്ങൾക്കുപോലും ഭയം ജനിപ്പിക്കുന്നു ഈ ദൃശ്യം. എല്ലാ ഇന്ദ്രിയചോദനകളും ഒഴിഞ്ഞ് ആശയങ്ങളുടേയും ചിന്തകളുടേയും കെട്ടുപാടുകൾ തീണ്ടാതെ ഈ ശരീരം സർവ്വതന്ത്രസ്വതന്ത്രമായിരിക്കുന്നു. മനസ്സെന്ന പിശാചിൽ നിന്നു മുക്തമായി പ്രകൃതിദുരിതങ്ങൾപോലും ബാധിക്കാത്ത അവസ്ഥയിലാണത്. മനസ്സെന്ന കുരങ്ങന്റെ വികൃതികളെല്ലാം ഒഴിഞ്ഞ് ശരീരമെന്ന ഈ മരം വേരോടെ കടപുഴകിയിരിക്കുന്നു. ഇവിടെയീ ഘോരവിപിനത്തിൽ ദു:ഖവിമുക്തമായ ഈ ശരീരം കാണാനിടയായത് എന്റെ സൗഭാഗ്യമത്രേ.

രാമൻ ചോദിച്ചു: മഹാത്മൻ, ശുക്രൻ എണ്ണമറ്റ ജന്മങ്ങളിലൂടെ കടന്നുപോയി എന്നു പറഞ്ഞല്ലോ. പിന്നെയെന്താണ്‌ ഭൃഗുപുത്രന്റെ രൂപത്തിലുള്ള ഈ ശരീരദർശനമാത്രയിൽ അതിന്റെ നിയോഗങ്ങളെപ്പറ്റി വിലപിച്ചത്?

വസിഷ്ഠൻ പറഞ്ഞു: അതിനു കാരണം, മറ്റു ജന്മങ്ങളും ശരീരങ്ങളുമെല്ലാം ശുക്രന്റെ മനോവിഭ്രാന്തിമാത്രമായിരുന്നു. ഭൃഗുപുത്രനായ ശുക്രനുണ്ടായ വിഭ്രാന്തികൾ. കഴിഞ്ഞയുഗാവസാനത്തിൽ അനന്താവബോധത്തിന്റെ ഇച്ഛപ്രകാരം ജീവാത്മാവ് ഭക്ഷണരൂപത്തിൽ ഭൃഗുമുനിയിൽ പ്രവേശിച്ചതാണ്‌ ശുക്രനെന്ന പുത്രനായി ജന്മമെടുത്തത്. ആ ജന്മത്തിലാണ്‌ അദ്ദേഹം ബ്രാഹ്മണകുമാരനനുയോജ്യമായ യാഗകർമ്മാദികൾ ചെയ്തത്. പിന്നെന്തുകൊണ്ടാണ്‌ ഇപ്പോൾ വാസുദേവനായിരിക്കുന്ന ശുക്രൻ ആ ദേഹം കണ്ട് ദു:ഖിച്ചത്?   "ഒരുവൻ ജ്ഞാനിയാണെങ്കിലും അജ്ഞാനിയാണെങ്കിലും ശരീരത്തിന്റെ ധർമ്മം, പ്രകൃതിനിയമം, തെറ്റാതെ മുറപോലെ നടക്കും. ശരീരമെടുത്ത വ്യക്തിത്വം ലോകോചിതമായി, സക്തിയോടെയോ അനാസക്തിയോടെയോ ലോകത്തിൽ വർത്തിക്കും." രണ്ടും തമ്മിലുള്ള വ്യത്യാസം മനോനിലയെ ആശ്രയിച്ചിരിക്കുന്നു. ജ്ഞാനിക്ക് അനുഭവങ്ങൾ മുക്തിപ്രദായകമാണ്‌.. അജ്ഞാനിക്കോ, അവ ബന്ധഹേതുവുമാണ്‌..

ശരീരമുണ്ടോ, വേദന വേദനാജനകവും, സുഖാനുഭവം സുഖദായകവുമാണ്‌..  ജ്ഞാനിക്ക് രണ്ടിലും ആസക്തിയില്ല. ദു:ഖത്തിൽ വിലപിച്ചും സുഖത്തിൽ സന്തോഷിച്ചും ജ്ഞാനി ഒരജ്ഞാനിയേപ്പോലെ പെരുമാറിയാലും അയാളുടെ പ്രബോധാവസ്ഥയിൽ മാറ്റമില്ല. ആരൊരാളുടെ ഇന്ദ്രിയങ്ങൾ സ്വതന്ത്രം, എന്നാൽ കർമ്മേന്ദ്രിയങ്ങൾ നിയന്ത്രണാധീനം ആണെങ്കിൽ അയാൾ മുക്തനത്രേ. എന്നാൽ ആരൊരാളുടെ ഇന്ദ്രിയങ്ങൾ നിയന്ത്രിതമെങ്കിലും കർമ്മേന്ദ്രിയങ്ങൾ നിയന്ത്രണമില്ലാത്തതാണെങ്കിൽ അയാൾ ബന്ധനത്തിലാണ്‌..

ജ്ഞാനിക്ക് സമൂഹത്തിൽനിന്ന് ഒന്നും നേടുവാനില്ലെങ്കിലും അയാളുടെ പെരുമാറ്റം ഉചിതമായിരിക്കും. രാമ: നീ സ്വയം നിർമ്മലമായ അനന്താവബോധമാണെന്ന അറിവിന്റെ നിറവിൽ എല്ലാ ആസക്തികളും ഉപേക്ഷിക്കൂ.  എന്നിട്ട് ചെയ്യേണ്ടതെല്ലാം ഭംഗിയായി ചെയ്തുതീർക്കൂ. 

