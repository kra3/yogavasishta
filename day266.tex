\section{ദിവസം 266}

\slokam{
വര്‍ത്തമാനമാനായാസം ഭജദ്ബാഹ്യാദിയാ ക്ഷണം\\
ഭൂതം ഭവിഷ്യദഭജധ്യാതി ചിത്തമചിത്തതാം  (5/50/16)\\
}

വസിഷ്ഠന്‍ തുടര്‍ന്നു: 'വിശ്വമായ'യാകുന്ന ഭ്രമകല്‍പ്പന വലിയ ചിന്താക്കുഴപ്പത്തിന് കാരണമാകുന്നു. മാത്രമല്ല, അത് നമ്മുടെ സംതുലിതാവസ്ഥയെ തകിടം മറിക്കുന്നു. അതിനെ മനസ്സിലാക്കുക എന്നത് അതീവ ദുഷ്കരം! കേവലം ഒരുമണിക്കൂര്‍ നേരത്തെ സ്വപ്നത്തിലെ മായികതയും ഒരു ജന്മം മുഴുവന്‍ നീണ്ടുനില്‍ക്കുന്ന ആ ഗോത്രവര്‍ഗ്ഗക്കാരന്റെ ജീവിതമെന്ന മായക്കാഴ്ചയും അതിലെ വൈവിദ്ധ്യമാര്‍ന്ന അനുഭവങ്ങളും തമ്മില്‍ യാതൊരു സാമ്യവുമില്ല. മായക്കാഴ്ചയില്‍ കണ്ട കാര്യങ്ങളും നമ്മുടെ കണ്മുന്നില്‍ കാണുന്ന കാര്യങ്ങളും തമ്മിലെങ്ങിനെയാണ് നാം ബന്ധിപ്പിക്കുക? എന്താണ് സത്യത്തില്‍ ഉള്ളത്? അല്ലെങ്കില്‍ അയാഥാര്‍ത്ഥമായത്? എന്താണ് ശരിയ്ക്കും  മാറ്റങ്ങള്‍ക്കു വിധേയമായത്? അതാണ്‌ രാമാ ഞാന്‍ പറയുന്നത്, ജാഗരൂകമല്ലാത്ത മനസ്സിനെ ഈ വിശ്വമായ അന്തമില്ലാത്ത കഷ്ടതകളിലേയ്ക്കാണു നയിക്കുന്നതെന്ന്.    

രാമന്‍ ചോദിച്ചു: മഹാത്മന്‍, പക്ഷെ ഇത്ര ശക്തിയോടെ ചുറ്റുന്ന ഈ മായാചക്രത്തെ എങ്ങിനെയാണ് നാം നിയന്ത്രിക്കുക?

വസിഷ്ഠന്‍ പറഞ്ഞു: മനസ്സിനെ അച്ചുതണ്ടാക്കിയാണീ മായ ചുറ്റിക്കറങ്ങുന്നത്. അങ്ങിനെയാണത് ഭ്രമാത്മകമായ മനസ്സിനെ മായികതയില്‍ തളയ്ക്കുന്നത്. തീവ്രമായ സ്വപരിശ്രമത്തിലൂടെയും ബുദ്ധിശക്തിയിലൂടെയുമാണീ ചുറ്റല്‍ അവസാനിപ്പിക്കുവാന്‍ കഴിയുക. അച്ചുതണ്ട് അനങ്ങാതിരുന്നാല്‍പ്പിന്നെ ചക്രമുരുളുകയില്ലല്ലോ. അതുപോലെ മനസ്സ് നിലച്ചാല്‍പ്പിന്നെ മായക്കാഴ്ചയില്ല. ഈ വിദ്യ അറിയാത്തവരും അഭ്യസിക്കാത്തവരും അന്തമറ്റ ദുരിതങ്ങളിലൂടെ കടന്നുപോകുന്നു. എന്നാല്‍ സത്യം വെളിപ്പെട്ടാലുടന്‍ ദു:ഖങ്ങള്‍ക്കറുതിയായി. മനസ്സിനെ കീഴടക്കുക എന്ന ഒരൊറ്റ മരുന്നേയുള്ളു ലോകമെന്ന ഈ ‘തെറ്റിദ്ധാരണയെ’ അല്ലെങ്കില്‍ ഭവരോഗത്തെ ഇല്ലാതാക്കാന്‍... അതുകൊണ്ട് രാമാ, തീര്‍ത്ഥാടനം, ദാനം, അനുഷ്ഠാനങ്ങള്‍ എല്ലാമുപേക്ഷിച്ച് മനസ്സിനെ നിന്റെ വരുതിയില്‍ നിര്‍ത്താന്‍ പരിശ്രമിക്കൂ. 

ഈ പ്രത്യക്ഷലോകം മനസ്സിലാണ് കുടികൊള്ളുന്നത്. ഒരു കുടം പൊട്ടുമ്പോള്‍ കുടത്തിനകത്തെ പരിമിതമായ ആകാശം ‘ഇല്ലാതാവുന്നത്’ കുടാകാശമെന്ന സങ്കല്‍പ്പസീമ ഇല്ലാതാവുന്നതിനാലാണ്. അതുപോലെ മനസ്സില്ലാതെയാവുമ്പോള്‍  മനസ്സിനകത്തെ സങ്കല്‍പ്പജന്യമായ ലോകവും ഇല്ലാതെയാകും. കുടമുടയുമ്പോള്‍ അതില്‍ക്കുടുങ്ങിയിരുന്ന എറുമ്പിനും സഞ്ചാര സ്വാതന്ത്ര്യമായി. അതുപോലെ മനസ്സില്ലാതാകുന്നതോടെ മനസ്സിലുള്ളിലെ പരിമിതമായ ലോകസങ്കല്‍പ്പവും ഇല്ലാതായി. നിനക്കു മുക്തിയായി.

“ബോധത്തെ പ്രയത്നലേശം കൂടാതെ ബാഹ്യാഭിമുഖമാക്കി ഇവിടെ, ഇപ്പോള്‍ , വര്‍ത്തമാനത്തില്‍ ജീവിക്കൂ. മനസ്സ് ഭൂത-ഭാവി കാര്യങ്ങളിലേയ്ക്ക് കോര്‍ത്തു ബന്ധിപ്പിക്കാതെയിരുന്നാല്‍ അത് മനസ്സില്ലാത്ത അവസ്ഥയാണ് (അമനസ്സ്). അനുനിമിഷം മനസ്സ് അതതുകാര്യങ്ങളിലേര്‍പ്പെട്ട് അയത്നലളിതമായി അവിടെത്തന്നെ ഉപേക്ഷിക്കുന്നതായാല്‍പ്പിന്നെ ‘അമനസ്സായി’. പരിപൂര്‍ണ്ണ നൈര്‍മ്മല്യമായി."    

മനസ്സ്ചഞ്ചലപ്പെട്ടുകൊണ്ടിരിക്കുമ്പോഴാണ് അത് സ്വന്തം വിക്ഷേപങ്ങളാകുന്ന വൈവിദ്ധ്യത അനുഭവിക്കുന്നത്. മേഘങ്ങളുണ്ടെങ്കില്‍ മാത്രമാണല്ലോ മഴപെയ്യുക. അനന്താവബോധം സ്വയം ഒരു മനസ്സിന്റെ പരിമിതിയില്‍ നില്‍ക്കുമ്പോഴാണ് വിക്ഷേപം സംഭവിക്കുക. എന്നാല്‍ അനന്താവബോധം പരിമിതമനസ്സെന്ന ഒന്നിലേയ്ക്കു ചുരുങ്ങാതിരിക്കുമ്പോള്‍ ജനനമരണചക്രമെന്ന മായയുടെ വേര് തന്നെ കരിഞ്ഞുപോയി എന്നര്‍ത്ഥം. അവിടെ പരിപൂര്‍ണ്ണതയായി.