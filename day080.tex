 
\section{ദിവസം 080}

\slokam{
ന തു ജാഡ്യം പ്രഥക്കിഞ്ചി ദസ്തി നാപി ച ചേതനം\\
നാത്ര ഭേദോഽസ്തി സര്‍ഗാദൌ സത്താസാമാന്യകേന ച (3/55/57)\\
}

സരസ്വതി തുടര്‍ന്നു: അനന്ത അവബോധത്തിന്റെ ഭാഗമായ മേധാ ശക്തി സ്വയം ഒരു മരമാണെന്നു നിനച്ചപ്പോള്‍ അതു മരമായി. കല്ലെന്നുസങ്കല്‍പ്പിച്ചപ്പോള്‍ കല്ലായി. പുല്ലെന്നു വിചാരിച്ചപ്പോള്‍ പുല്ലായി. "ജീവനുള്ള വസ്തുവും അല്ലാത്തതും തമ്മില്‍ വ്യത്യാസമൊന്നുമില്ല. ജഢവസ്തുവും ജീവനുള്ളവയും തമ്മിലും വ്യത്യാസമില്ല. കാരണം, എല്ലാറ്റിലും എല്ലായിടത്തും അനന്താവബോധം ഒരേപോലെ സുസ്ഥിതമത്രേ" വ്യത്യാസമുണ്ടാവുന്നത്‌ ഓരോന്നുമായി താദാത്മ്യം പ്രാപിക്കാനുള്ള ബുദ്ധിയുടെ ത്വരകൊണ്ടുമാത്രമാണ്‌. ഒന്നേ ഒന്നുമാത്രമായ അവബോധം പദാര്‍ത്ഥവസ്തുക്കളില്‍ പല നാമങ്ങളില്‍ അറിയപ്പെടുന്നു. അതുപോലെതന്നെയാണ്‌ പുഴുവായും എറുമ്പായും പറവയായും ഉള്ള ബുദ്ധിയുടെ താദാത്മ്യഭാവം കൊണ്ട്‌ അവകളായിത്തീരുന്നത്‌. ആ സത്തയില്‍ താരതമ്യപ്പെടുത്താന്‍ മറ്റൊന്നില്ല! വ്യതിരിക്തതയെന്ന ധാരണയില്ല. ഉത്തരധ്രുവത്തില്‍ നിവസിക്കുന്നവര്‍ക്ക്‌ ദകഷിണധ്രുവത്തിലെ ജനങ്ങളെപ്പറ്റി അറിയില്ല. അവരതുകൊണ്ട്‌ തമ്മില്‍ത്തമ്മില്‍ താരതമ്യപ്പെടുത്തുന്നുമില്ല. 

ഈ ബുദ്ധിശക്തി തിരിച്ചറിഞ്ഞു കല്‍പ്പിച്ചുവച്ചപാര്‍ത്ഥങ്ങള്‍ അങ്ങിനെത്തന്നെ നിലകൊണ്ടു. അവ മറ്റു പദാര്‍ത്ഥങ്ങളില്‍നിന്നും വിഭിന്നമല്ല. അവയ്ക്ക്‌ സചേതനമെന്നും അചേതനമെന്നും വ്യത്യാസം കല്‍പ്പിക്കുന്നത്‌ പാറപ്പുറത്തുണ്ടായ തവളയും അതിനപ്പുറത്ത്‌ ചെളിക്കുണ്ടിലുണ്ടായ തവളയും വെവ്വേറെയണെന്നു - ഒന്നു ജീവനില്ലാത്തതും മറ്റേത്‌ ജീവനുള്ളതും-പറയുമ്പോലെയാണ്‌. മേധാശക്തി എന്തു സ്വയം 'ആയിത്തീര്‍ന്നു' എന്നു വിചാരിച്ചുവോ അത്‌ അങ്ങിനെ തന്നെയായി സൃഷ്ടിയാരംഭം മുതല്‍ നിലകൊണ്ടു. അത്‌ എല്ലായിടത്തും,എന്നും നിലനില്‍ക്കുന്ന അനന്ത ബോധത്തിന്റെ ഭാഗമാണല്ലോ. അത്‌ ആകാശമായും വായുവായും സ്വയം സചേതനമായും അചേതനമായുമെല്ലാം അലോചിച്ചു. അവയെല്ലാമുണ്ടായത്‌ ഈ ബുദ്ധിശക്തിയുടെ സങ്കല്‍പ്പമായാണ്‌. ഈ പ്രത്യക്ഷമായ കാഴ്ച്ചകളൊന്നും സത്തല്ല. അവ യാഥാര്‍ഥ്യമാണെന്നു തോന്നുന്നുവെന്നേയുള്ളു. 
ലീലേ, നോക്കൂ, വിഥുരഥ രാജാവിന്റെ ജീവന്‌ പദ്മ രാജാവിന്റെ ദേഹത്തുപ്രവേശിക്കാന്‍ ആഗ്രഹമുണ്ടെന്നെനിക്കു തോന്നുന്നു.

പ്രബുദ്ധയായ ലീല പറഞ്ഞു: ദേവീ നമുക്കങ്ങോട്ടു പോവാം.

സരസ്വതി പറഞ്ഞു: പദ്മ രാജാവിന്റെ ഹൃദയത്തിലെ അഹംകാര തത്വവുമായി അനുരണനം ചെയ്ത്‌ വിഥുരഥന്‍ മറ്റൊരു ലോകത്തെയ്ക്കു പോവുകയാണെന്നു ചിന്തിക്കുന്നു. നമുക്ക്‌ നമ്മുടെ വഴിയേ പോവാം. ഒരാള്‍ക്ക്‌ മറ്റൊരാളുടെ പാതയില്‍ സഞ്ചരിക്കാന്‍പറ്റുകയില്ലല്ലോ.  
