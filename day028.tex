 
\section{ദിവസം 028}

\slokam{
മോക്ഷദ്വാരേ ദ്വാരപാലാശ്ചത്വാര: പരികീർത്തിതാ :\\
ശമോ വിചാര: സന്തോഷശ്ചതുർത്ഥ: സാധുസംഗമ: (2/11/59)\\
}

വസിഷ്ഠന്‍ തുടര്‍ന്നു: ജനങ്ങളെ ആത്മപ്രബുദ്ധരാക്കാന്‍ എല്ലാ യുഗങ്ങളിലും ബ്രഹ്മാവ്‌ അനേകം മാമുനിമാരെയും കൂടെ എന്നേയും മനസാ സൃഷ്ടിക്കുന്നു. എല്ലാവരും ധാര്‍മ്മീകമായി അവനവന്റെ കര്‍ത്തവ്യനിവ്വഹണം നടത്തുന്നുവെന്നുറപ്പാക്കാന്‍ ന്യായപൂര്‍വ്വം ഭരണം നടത്തുന്ന രാജാക്കന്മാരെയും സൃഷ്ടിക്കുന്നു. ഈ രാജാക്കന്മാര്‍ താമസംവിനാ കാമത്തിലും സുഖഭോഗങ്ങളിലും അധികാരങ്ങളിലും മോഹിതരായി വൈവിദ്ധ്യമാര്‍ന്ന താത്പ്പര്യങ്ങള്‍മൂലം യുദ്ധങ്ങളില്‍ ഏര്‍പ്പെട്ട്‌ അവസാനം പശ്ചാത്തപിക്കുന്നു. അവരുടെ അവിദ്യയെ നീക്കാന്‍ മഹര്‍ഷിമാര്‍ ആത്മോപദേശം നല്‍കാറുണ്ട്‌. രാമ: പണ്ടുകാലത്ത്‌ ചക്രവര്‍ത്തിമാര്‍ ഈ ഉപദേശങ്ങളെ ഇഷ്ടപ്പെടുകയും ബഹുമാനിക്കുകയും ചെയ്തിരുന്നു. അതിനാല്‍ ഇത്‌ രാജവിദ്യ എന്നു വിഖ്യാതമായി.

രാമ: നിന്നില്‍ ഇപ്പോള്‍ സംജാതമായ നിര്‍മ്മമത ശുദ്ധവിവേകത്തില്‍ നിന്നുദിച്ചതാണ്‌. അത്‌ വെറുപ്പുകൊണ്ടോ സാഹചര്യങ്ങളുടെ സമ്മര്‍ദ്ദം മൂലമോ ഉണ്ടാവുന്നതിനേക്കാള്‍ എത്രയോ ഉന്നതമാണ്‌!. അത്തരം അനാസക്തി ഈശ്വരകാരുണ്യം കൊണ്ടേ ലഭ്യമാവൂ. ഈ കാരുണ്യം പാകതവന്ന വിവേകവുമായി സന്ധിക്കുന്നത്‌   ഹൃദയം അനാസക്തമാവുന്ന നിമിഷത്തിലാണ്‌. ഈ പരമവിജ്ഞാനം ഹൃദയത്തില്‍ ഉണരുംവരെ ഒരുവന്‍ ജനന മരണ ചക്രത്തില്‍ ചുറ്റിക്കൊണ്ടേയിരിക്കും. ഏകാഗ്രചിത്തത്തോടെ, ഞാന്‍ വെളിപ്പെടുത്തുന്ന ഈ പരമവിദ്യയെ  ശ്രദ്ധിച്ചാലും. അവിദ്യയെന്ന കാനനത്തെ ഇത്‌ നശിപ്പിക്കുന്നു. ഈ വനത്തിലാണല്ലോ മനുഷ്യന്‍ ചിന്താക്കുഴപ്പത്തിലാണ്ട്‌ അന്തമില്ലാത്ത ദുരിതാനുഭവങ്ങള്‍ അനുഭവിക്കുന്നത്‌. 

ആരെങ്കിലും ഈ ആത്മവിദ്യ നേടാന്‍ ഉത്സുകനാണെങ്കില്‍ അയാള്‍ പ്രബുദ്ധനായ ഒരു ഗുരുവിനെ സമീപിച്ച്‌ ഉത്തമായ മനോഭാവത്തോടെ ഉചിതമായ ചോദ്യങ്ങള്‍ ചോദിക്കണം. അങ്ങിനെ അതവന്റെ ആത്മസത്തയുടെ ഭാഗമായിത്തീരും. അനാവശ്യവും അപ്രസക്തവുമായ ചോദ്യങ്ങള്‍ ചോദിക്കുന്നവന്‍ വിഡ്ഢിയാണ്‌. അതിലേറെ മൂഢനാണ്‌ മഹാത്മാക്കളുടെ ഉപദേശമനുസരിച്ച്‌ പ്രവര്‍ത്തിക്കാത്തവന്‍. വിഡ്ഢികളുടെ വിഫലമായ ചോദ്യങ്ങള്‍ക്ക്‌ മറുപടി പറയുന്നവനെ മഹര്‍ഷി എന്നു വിളിക്കാന്‍ പറ്റില്ലതന്നെ.

രാമ: നീ സാധകരില്‍ വെച്ച്‌ അഗ്രഗണ്യനാണ്‌. നീ സത്യവസ്തുവിനെപ്പറ്റി ധ്യാനിച്ചത്‌ ഉത്തമമായ അനാസക്തിയുടെ ആര്‍ജ്ജവത്താലാണല്ലോ. ഞാന്‍ പറയാന്‍ പോവുന്ന കാര്യം നിന്റെ ഹൃദയത്തില്‍ ഉറയ്ക്കുമെന്നെനിക്കറിയാം. മനസ്സൊരു ചഞ്ചലവാനരനായതുകൊണ്ട്‌ സാധകന്‍ ആത്മജ്ഞാനത്തെ ഹൃദയത്തില്‍ പ്രതിഷ്ഠിക്കാന്‍ തീര്‍ച്ചയായും കഠിനയത്നംതന്നെ ചെയ്യേണ്ടതുണ്ട്‌. മാത്രമല്ല അയാള്‍ ദുര്‍ജ്ജന സംസര്‍ഗ്ഗം ഉപേക്ഷിക്കുകയും വേണം.

"രാമ: മോക്ഷകവാടത്തിലെ നാലു ദ്വാരപാലകര്‍ ഇവരാണ്‌: ആത്മസംയമനം, അന്വോഷണത്വര; സംതൃപ്തി,  സത്സംഗം." ബുദ്ധിമാനായ സാധകന്‍ ഈ നാലുപേരുമായി സൌഹൃദം സ്ഥപിക്കാന്‍ അനവരതം ശ്രമിച്ചുകൊണ്ടേയിരിക്കണം. അവരില്‍ ഒരാളെയെങ്കിലും തന്റെ ഉത്തമസുഹൃത്താക്കുകയും വേണം. 
