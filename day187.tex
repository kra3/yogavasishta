\section{ദിവസം 187}

\slokam{
അനാഗതാനാം ഭോഗാനാമവാഞ്ഛനമകൃത്രിമം\\
ആഗതാനാം ച സംഭോഗ ഇതി പണ്ഡിത ലക്ഷണം (4/46/8)\\
}

വസിഷ്ഠൻ തുടർന്നു: ഈ വിശ്വപ്രപഞ്ചത്തിലെ എല്ലാം - ധനം, ഭാര്യ, പുത്രൻ എന്നുവേണ്ട എല്ലാമെല്ലാം മനസ്സിന്റെ മായാജാലം മാത്രമാണെന്നറിയുന്ന ഒരുവൻ അവയുടെ നഷ്ടത്തിൽ ദു:ഖിക്കുകയോ അവയുടെ ഉല്‍ക്കര്‍ഷത്തില്‍ അമിതാഹ്ളാദം കൊള്ളൂകയോ ചെയ്യുകയില്ല. വാസ്തവത്തില്‍ അവയുടെ അഭിവൃദ്ധി ഒരുവനില്‍ അസന്തുഷ്ടിയാണുണ്ടാക്കേണ്ടത്. കാരണം അജ്ഞാനത്തിന്റെ സാന്ദ്രത കൂട്ടുവാനാണല്ലോ അതുപകരിക്കുക. മന്ദബുദ്ധിയിൽ ആസക്തിയും സംഗവും ജനിപ്പിക്കുന്ന അതേ കാര്യം ജ്ഞാനിയിൽ വിരക്തിയും നിർമമതയുമാണുണ്ടാക്കുന്നത്.

“പ്രയത്നം കൂടാതെ കിട്ടാനിടയില്ലാത്ത അനുഭവങ്ങൾക്കായി ആഗ്രഹിക്കാതിരിക്കുകയും പ്രയത്നലേശമില്ലാതെ സ്വമേധയാ വന്നുചേർന്നവയെ അനുഭവിക്കുന്നതുമാണ്‌ പണ്ഡിതന്റെ ലക്ഷണം.” സംസാരസാഗരമെന്ന മായക്കാഴ്ച്ചയിൽ മുങ്ങിത്താഴാതെയിരിക്കാൻ കഴിയണമെങ്കിൽ ഏതെങ്കിലും വിധത്തിൽ ഇന്ദ്രിയ സുഖാനുഭാവാസക്തികളിൽ നിന്നും പിന്തിരിഞ്ഞാൽ മാത്രമേ സാധിക്കൂ. വിശ്വത്തിന്റെ ഏകതാത്മകത്വം അറിഞ്ഞ ഒരുവൻ ആഗ്രഹനിവൃത്തിക്കുവേണ്ടിയും ആഗ്രഹനിരാസത്തിനായും ഉള്ള എല്ലാ പ്രവർത്തനങ്ങൾക്കും അതീതനത്രേ. അതുകൊണ്ട് രാമാ, സത്തിനും അസത്തിനും അതീതമായി എല്ലാടവും നിറഞ്ഞു വിളങ്ങുന്ന സർവ്വവ്യാപിയായ ആത്മാവിനെ, അനന്താവബോധത്തെ സാക്ഷാത്കരിച്ചാലും. അകത്തുള്ളതും പുറത്തുള്ളതുമായ ഒന്നിനേയും ഉപേക്ഷിക്കുകയോ അവയിൽ കടിച്ചു തൂങ്ങുകയോ ചെയ്യേണ്ടതില്ല. ഈദൃശമായ ആത്മവിദ്യയിൽ അഭിരമിക്കുന്ന ജ്ഞാനിയുടെ മനസ്സിനു പരിമിതികളില്ല. അതിന്‌ നിറഭേദമോ ഉപാധികളോ ബാധകവുമല്ല. അയാൾ ആകാശം പോലെ, എന്തൊക്കെ സംഭവിച്ചാലും യാതൊന്നിനാലും മലിനപ്പെടാതെ ഇരിക്കുന്നു.

ഇന്ദ്രിയവിഷയങ്ങളിൽ ‘എന്റേത്’ എന്ന ധാരണ വച്ചു പുലർത്താതിരിക്കുക. അപ്പോൾ കർമ്മനിരതനാണെങ്കിലും അല്ലെങ്കിലും നീ അജ്ഞാനത്തിൽ ആണ്ടു പോവുകയില്ല. ഇന്ദ്രിയസുഖങ്ങൾ അഭിലഷണീയമാണെന്ന്‍ നിന്റെ ഹൃദയത്തിൽ തോന്നുന്നില്ലെങ്കിൽ നീ അറിയാനുള്ളതെന്തോ അതറിഞ്ഞിരിക്കുന്നു. നിനക്ക് ജനനമരണചക്രത്തിൽ ചുറ്റേണ്ടതായി വരില്ല. ശരീരബോധത്തോടെയോ അല്ലാതെയോ, ഇഹലോകത്തിലേയോ പരലോകത്തിലേയോ സുഖാനുഭവങ്ങളിൽ ആകർഷിക്കപ്പെടാത്ത ഒരുവന്‌ മുക്തിപദം അവനാഗ്രഹിച്ചാലുമില്ലെങ്കിലും സ്വയമേവ വന്നുചേരുന്നതാണ്.

രാമാ, ഈ മനോപാധികളാകുന്ന അജ്ഞാനസാഗരത്തെ താണ്ടുവാനുതകുന്ന ആത്മജ്ഞാനമെന്ന ചങ്ങാടം കണ്ടെത്തിയവൻ പിന്നെയതില്‍ മുങ്ങിപ്പോകുകയില്ല. ആത്മജ്ഞാനമാർഗ്ഗം അറിയാത്തവന്‍ അതിൽ തീർച്ചയായും മുങ്ങുകതന്നെ ചെയ്യും. അതുകൊണ്ട് രാമാ, മൂർച്ചയേറിയ വായ്ത്തല പോലെയുള്ള മേധാശക്തികൊണ്ട് അത്മാവിന്റെ സ്വരൂപത്തെ അറിയൂ, എന്നിട്ട് ആ സ്വരൂപത്തിൽ ആത്മജ്ഞാനത്തോടെ അഭിരമിക്കൂ. അങ്ങിനെ ആത്മജ്ഞാനമാർജ്ജിച്ച മഹർഷിമാരുടെ ജീവിതം നിനക്കും നയിക്കാം. അവർക്ക് അനന്താവബോധത്തെയും ലോകമെന്ന ഈ വിക്ഷേപത്തേയും നന്നായറിയാം. അവർ കർമ്മങ്ങളിൽ അഭിരമിക്കുകയോ കർമ്മങ്ങളെ ഉപേക്ഷിക്കുകയോ ചെയ്യുന്നില്ല. രാമാ, നീയും ആത്മജ്ഞാനത്തെ കൈവരിച്ചിരിക്കുന്നു. നീ പ്രശാന്തനായിരിക്കുന്നു. 

