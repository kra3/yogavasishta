\newpage
\section{ദിവസം 007}

\slokam{
കിം നാമേദം വത സുഖം യേയം സംസാരസംതതി:\\
ജായതേ മൃതയേ ലോകോ മ്രിയതേ ജനനായ ച (1/12/7)\\
}

വിശ്വാമിത്രന്‍ പറഞ്ഞു: അങ്ങിനെയെങ്കില്‍ രാമനോട്‌ ഇവിടെ വരാന്‍ പറയുക. കുമാരന്റെ അവസ്ഥ വെറും മനോവിഭ്രമത്തിനേതല്ല. മറിച്ച്‌ അത്‌ വിജ്ഞാനത്തിന്റേയും നിര്‍വ്വികാരതയുടെയുമാണ്‌. അത്‌ ചൂണ്ടിക്കാട്ടുന്നത്‌ കുമാരന്റെ പ്രബുദ്ധതയെയാണ്‌.. കുമാരനെ വിളിച്ചുകൊണ്ടുവരൂ. ഈ വിഷണ്ണാവസ്ഥയെ നമുക്ക്‌ മാറ്റിയെടുക്കാം.

വാല്‍മീകി പറഞ്ഞു: അപ്പോള്‍ മഹാരാജാവ്‌ രാമനെ വിളിച്ചുകൊണ്ടുവരാന്‍ പള്ളിയറസേവകനോട്‌ കല്‍പ്പിച്ചു. എന്നാല്‍ ഈ സമയം രാമന്‍ സ്വയം അച്ഛനെ മുഖം കാണിക്കാന്‍ എഴുന്നേറ്റുവന്ന് ദൂരെ നിന്നു തന്നെ അച്ഛനേയും മഹര്‍ഷിമാരേയും അഭിവാദ്യം ചെയ്തു. യുവാവായ രാമന്റെ തേജസ്സുറ്റ മുഖത്ത്‌ അവര്‍ വയസ്സിനപ്പുറം പ്രശാന്തമായ ഒരു പക്വത ദര്‍ശിച്ചു. രാജാവിന്റെ കാല്‍ക്കല്‍ വീണ രാമനെ പിതാവ്‌ ആലിംഗനത്തോടെ എഴുന്നേല്‍പ്പിച്ചു. "മകനേ എന്താണു നിന്നെ വ്യാകുലപ്പെടുത്തുന്നത്‌? വിഷാദം മറ്റു ആപത്തുകളെ വിളിച്ചു വരുത്തും എന്നറിയുക". വസിഷ്ഠനും വിശ്വാമിത്രനും രാജാവിന്റെ വാക്കുകള്‍ ശരിവച്ചു.

ശ്രീ രാമന്‍ പറഞ്ഞു: "പൂജ്യരേ ഈ ചോദ്യത്തിനു ഞാന്‍ മറുപടി പറയാം. അച്ഛന്റെ കൊട്ടാരത്തില്‍ അതീവ സന്തോഷവാനായി ഞാന്‍ വളര്‍ന്നു. വിദ്യാഭ്യാസത്തിന്‌ എനിക്കു കിട്ടിയത്‌ അഗ്രഗണ്യരായ ഗുരുവരന്മാരെയാണ്‌.. ഈയിടയ്ക്ക്‌ ഞാന്‍ ഒരു തീര്‍ത്ഥാടനത്തിനു പോയിരുന്നു. ആ സമയത്ത്‌ എന്റെയുള്ളില്‍ കടന്നു കൂടിയ ഒരു ചിന്ത എന്റെയുള്ളിലെ എല്ലാ പ്രത്യാശകളേയും കവര്‍ന്നു കളഞ്ഞു. എന്റെ ഹൃദയം ഇപ്രകാരം ചോദ്യങ്ങള്‍ തുടങ്ങി. "എന്തിനെയാണ്‌ മനുഷ്യര്‍ സുഖം എന്നു പറയുന്നത്‌? വസ്തുക്കള്‍ എപ്പോഴും പരിണാമത്തിനു വിധേയമായ ഈ ലോകത്ത്‌ നിത്യസുഖം സാദ്ധ്യമാണോ? ലോകത്തിലെ എല്ലാ ജീവജാലങ്ങളും ജനിയ്ക്കുന്നതു മരിക്കുവാനല്ലേ? മരിക്കുന്നതോ, വീണ്ടും ജനിക്കുവാനും! ഈ ക്ഷണികമായ പ്രതിഭാസമാണ്‌ എല്ലാ ദു:ഖങ്ങള്‍ക്കും പാപങ്ങള്‍ക്കും ഹേതു. എന്നാല്‍ ഇതിന്റെ അര്‍ത്ഥം എനിക്കു മനസ്സിലാവുന്നില്ല. ബന്ധമൊന്നുമില്ലാത്തവര്‍ കൂടിച്ചേരുമ്പോള്‍ മനസ്സ്‌ അവരുമായി ഒരു ബന്ധുത്വം ഉണ്ടാക്കിയെടുക്കുന്നു. എല്ലാം മനസ്സിനെ, മനസ്ഥിതിയെ, ആശ്രയിച്ചിരിക്കുന്നു. എന്നാല്‍ അന്വേഷി ക്കുമ്പോള്‍ നാമറിയുന്നു മനസ്സ്‌ എന്നതും സത്യമായ ഒന്നല്ല എന്ന്. എന്നാല്‍ നാം അതിനാല്‍ മയക്കപ്പെട്ടിരിക്കുകയാണ്‌. മരുഭൂമിയിലെ കാനല്‍ജലത്തിനു പുറകേ നാം ദാഹമടക്കാന്‍ ഓടി നടക്കുന്നു. നാം ആര്‍ക്കും വിറ്റുപോയ അടിമകള്‍ അല്ലെങ്കിലും നാം നയിക്കുന്നത്‌ യാതൊരു സ്വാതന്ത്ര്യവും ഇല്ലാത്ത അടിമകളുടെ ജീവിതമത്രേ. ഈ സത്യം അറിയാതെ ലോകമെന്ന ഈ കാനനത്തില്‍ നാം ലക്ഷ്യമില്ലാതെ അലയുകയാണ്‌. എന്താണീ ലോകം? എന്താണ്‌ ജന്മമെടുത്ത്‌ വളര്‍ന്ന് മരിക്കുന്നത്‌? ഈ സങ്കടാവസ്ഥ എങ്ങിനെയാണവസാനിക്കുക? എന്റെ ഹൃദയത്തില്‍ ദു:ഖത്തിന്റെ ചോര പൊടിയുമ്പോഴും എന്റെ സുഹൃത്തുക്കളുടെ വികാരത്തെ മാനിച്ച്‌ ഞാന്‍ കണ്ണീര്‍ പൊഴിക്കുന്നില്ല".
