 
\section{ദിവസം 005}

\begin{center}
കാലേ കാലേ പ്രഥഗ്ബ്രഹ്മൻ ഭൂരിവീര്യവിഭൂതയ:\\
ഭൂതേഷ്വഭ്യുദയം യാന്തി പ്രലീയന്തേ ച കാലത:  (1/8/29)\\
\end{center}

വാല്‍മീകി തുടര്‍ ന്നു: ദശരത്ഥ മഹാരാജന്റെ വാക്കുകള്‍ കേട്ട്‌ സം പ്രീതനായ വിശ്വാമിത്രന്‍ തന്റെ ആഗമനോദ്ദേശം അറിയിച്ചു. "മഹാരാജന്‍, ഞാന്‍ നടത്തുന്ന ഒരു യാഗം വിജയകരമായി പൂര്‍ത്തീകരിക്കാന്‍ എനിയ്ക്ക്‌ അങ്ങയുടെ സഹായം ആവശ്യമുണ്ട്‌. ഞാന്‍ യാഗം ചെയ്യുന്നിടത്ത്‌ രാക്ഷസരായ ഖരന്റേയും ദൂഷണന്റേയും അനുയായികള്‍ അതിക്രമിച്ചുവന്ന് യജ്ഞകുണ്ഡം മലിനമാക്കുന്നു. യാഗത്തിന്റെ വ്രതവിധിപ്രകാരം അവരെ എനിയ്ക്കു ശപിക്കാനും വയ്യ. അങ്ങേയ്ക്ക്‌ എന്നെ സഹായിക്കാന്‍ കഴിയും. അങ്ങയുടെ പുത്രന്‍ രാമന്‌ ഈ ദുഷ്ടപ്പരിശകളെ നിഷ്പ്രയാസം നേരിട്ടു നശിപ്പിക്കാന്‍ സാധിക്കും. ഇതിനു പകരമായി ഞാന്‍ രാമന്‌ പലേവിധങ്ങളായ അനുഗ്രഹങ്ങളും നല്‍കാം. അത്‌ താങ്കള്‍ക്ക്‌ ഉന്നതമായ പ്രശസ്തിയെ പ്രദാനം ചെയ്യും. മകനോടുള്ള മമതകൊണ്ട്‌ അങ്ങ്‌ സ്വന്തം കര്‍ത്തവ്യത്തെ മറക്കാനിടവരുത്തരുത്‌... ഈ ലോകത്ത്‌ മഹാന്മാര്‍ ആരും തന്റെ കഴിവിനനുസരിച്ചേ സമ്മാനങ്ങള്‍ നല്‍കൂ. അങ്ങ്‌ ശരി എന്നു പറഞ്ഞാല്‍ ആ നിമിഷം രാക്ഷസന്മാര്‍ കൊല്ലപ്പെട്ടുകഴിഞ്ഞു എന്നു ഞാന്‍ കണക്കാക്കും. കാരണം രാമന്‍ ആരെന്ന് എനിയ്ക്കറിയാം. ഈ വസിഷ്ഠമുനിയ്ക്കും സഭയിലെ മറ്റു മഹാന്മാര്‍ക്കും അതറിയാം. ഇനി ഒട്ടും താമസമരുത്‌. മഹാരാജന്‍, രാമനെ എന്റെകൂടെ അയച്ചാലും."

മഹര്‍ഷിയുടെ ഈ അനഭിമതമായ ആവശ്യംകേട്ട്‌ രാജാവ്‌ സ്തബ്ധനായിപ്പോയി. അല്‍പ്പനേരം കഴിഞ്ഞ്‌ അദ്ദേഹം പറഞ്ഞു: "ഭഗവന്‍, രാമന്‌ വെറും പതിനാറു വയസ്സേ ആയിട്ടുള്ളു. യുദ്ധത്തില്‍ പങ്കെടുക്കാനുള്ള യോഗ്യത കൈവന്നിട്ടില്ല. മാത്രമല്ല ഇതുവരെ അവന്‍ ഒരു യുദ്ധം കണ്ടിട്ടുകൂടിയില്ല. രാമനുപകരം എന്നോട്‌ ആജ്ഞാപിച്ചാലും. ഞാനും എന്റെ സൈന്യവും അങ്ങയുടെ ആജ്ഞാനുസാരം രാക്ഷസരെ സമൂലം ഇല്ലായ്മ ചെയ്തുകൊള്ളാം. എനിയ്ക്കു രാമനെ വിട്ടുതരാന്‍ വയ്യ. ഞാന്‍ രാവണന്‍ എന്ന പേരുള്ള അതീവശക്തിശാലിയായ ഒരു രാക്ഷസനെപ്പറ്റി കേട്ടിട്ടുണ്ട്‌. അയാളാണോ ഈ യാഗങ്ങളെ മുടക്കുന്നത്‌? എന്നാല്‍ ഒരു വിധത്തിലും അങ്ങയെ സഹായിക്കാന്‍ എനിക്കാവില്ല. കാരണം ദേവന്മാര്‍ പോലും അയാള്‍ക്കെതിരേ പൊരുതാന്‍ അശക്തരാണ്‌.. ചില കാലങ്ങളില്‍ അങ്ങിനെയുള്ള അതിശക്തരായ സത്വങ്ങള്‍ ഈ ഭൂമിയില്‍ ജനിക്കുന്നു. കാലക്രമേണ അവ ഇഹലോകവാസം വെടിയുകയും ചെയ്യുന്നു."

വിശ്വാമിത്രന്‍ പെട്ടെന്നു ക്രോധിഷ്ഠനായതുകണ്ട്‌ വസിഷ്ഠമുനി ഇടപെട്ട്‌ ശ്രീരാമനെ വിശ്വാമിത്രന്റെ കൂടെ അയക്കാന്‍ രാജാവിനെ ഉപദേശിച്ചു. "ഒരു രാജാവ്‌ താന്‍ കൊടുത്ത വാഗ്ദ്ദാനത്തില്‍ നിന്നും ഒരിക്കലും പിന്മാറരുത്‌.. ധര്‍മ്മനിഷ്ഠയില്‍ രാജാവ്‌ എല്ലാവര്‍ക്കും മാതൃകയായിരിക്കണം. തീര്‍ച്ചയായും വിശ്വാമിത്രന്റെ കൂടെ രാമന്‍ സുരക്ഷിതനാണെന്നറിയുക. അദ്ദേഹം പ്രബലന്‍ മാത്രമല്ല അനേകം അജയ്യങ്ങളായ ദിവ്യാസ്ത്രങ്ങള്‍ കൈവശമുള്ളയാളുമാണ്‌". 
