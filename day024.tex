\newpage
\section{ദിവസം 024}

\slokam{
ശാസ്ത്രൈ: സദാചര വിജൃംഭിത  ദേശധർമൈര്‍ - \\
യത്കല്പിതം ഫലമതീവ ചിരപ്രരൂഢമ്\\
തസ്മിൻഹൃദി സ്പുരതി ചോപനമേതി ചിത്ത-\\
മംഗാവലീ തദനു പൗരുഷമേതദാഹു: (2/6/40)\\
}

വസിഷ്ഠമുനി രണ്ടാം ദിവസത്തെ പ്രഭാഷണം സമാരംഭിച്ചു. " രാമ: പ്രയത്നത്തിന്റെ സ്വഭാവത്തിനനുസരിച്ചാണ്‌ ഫലസിദ്ധി ഉണ്ടാവുന്നത്‌. ഇതാണ്‌ സ്വപ്രയത്നം എന്നുപറയുന്നതിന്റെ സാരം. ഇതുതന്നെയാണ്‌ ദിവ്യനിയതി, ഭാഗധേയം, വിധി എന്നെല്ലാമറിയപ്പെടുന്നത്‌.

ദു:ഖാനുഭവങ്ങളുണ്ടാവുമ്പോള്‍ "കഷ്ടം എന്തു ദുരിതം!" എന്നോ "എന്റെ വിധി ഇതാണല്ലോ!" എന്നോ ആളുകള്‍ കരഞ്ഞുവിളിക്കുന്നു. രണ്ടിന്റേയും സാരം ഒന്നുതന്നെ. ഭൂതകാലത്തു നാം തന്നെ ചെയ്തുകൂട്ടിയ കര്‍മ്മങ്ങളാണ്‌ നിയോഗമായിത്തീരുന്നത്‌. വര്‍ത്തമാനകാലം ഭൂതകാലത്തേക്കാള്‍ അനന്തശക്തമാണ്‌. ഭൂതകാല കര്‍മ്മഫലങ്ങളില്‍ സംതൃപ്തരായി ഇപ്പോള്‍ പ്രയത്നമൊന്നും ചെയ്യാതെ എല്ലാം നിയോഗമെന്നുകരുതി കഴിഞ്ഞുകൂടുന്നവര്‍ വിഡ്ഢികളത്രെ. ഇപ്പോഴത്തെ കര്‍മ്മങ്ങള്‍ക്ക്‌ വിധിവശാല്‍ തടസ്സങ്ങളേര്‍പ്പെടുന്നത്‌ ആ പ്രയത്നങ്ങള്‍ വേണ്ടത്ര പ്രബലമല്ലാ എന്നതുകൊണ്ടാണ്‌. അതീവശക്തിമാനായ ഒരെതിരാളിയെ നേരിടേണ്ടിവരുമ്പോള്‍ മണ്ടനും അവശനുമായവന്‍ 'വിധിവൈപരീത്യം' എന്നു വിലപിച്ച്‌ തോറ്റുപിന്‍വാങ്ങുന്നു. ചിലപ്പോള്‍ വലിയ പ്രയത്നമൊന്നും കൂടാതെ ഒരുവന്‍ വലിയ ലാഭങ്ങള്‍ നേടുന്നു. ഉദാഹരണത്തിന്‌ പഴയ ഒരു രീതിയനുസരിച്ച്‌ അനന്തരാവകാശിയില്ലാതെ രാജാവു ദിവംഗതനാവുമ്പോള്‍ കൊട്ടാരത്തിലെ ആന ചിലപ്പോള്‍ വെറുമൊരു വഴിപോക്കനെ തിരഞ്ഞെടുത്ത്‌ മാലചാര്‍ത്തി രാജാവായി വാഴിക്കുന്നു. ഇത്‌ ഒരാകസ്മികതയോ ദിവ്യാനുഗ്രഹമോ ഒന്നുമല്ല. വഴിപോക്കന്റെ, പൊയ്പോയ ജന്മങ്ങളിലെ സ്വകര്‍മ്മഫലം ഇപ്പോള്‍ അനുഭവവേദ്യമാവുന്നു എന്നേയുള്ളു. 

ചിലപ്പോള്‍ കര്‍ഷകന്റെ പ്രയത്നം മുഴുവന്‍ ഒരു ചുഴലിക്കാറ്റുകൊണ്ട്‌ വൃഥാവിലാവുന്നു. കൊടുങ്കാറ്റ്‌, കര്‍ഷകന്റെ പ്രയത്നത്തിനേക്കാള്‍ പ്രബലമാണ്‌. അതുകൊണ്ട്‌ കൃഷിക്കാരന്‍ കൂടുതല്‍ കഠിനമായി ആര്‍ജ്ജവത്തോടെ ജോലി ചെയ്യേണ്ടിയിരിക്കുന്നു. അവന്‍ തന്റെ നഷ്ടത്തെച്ചൊല്ലി ദു:ഖിച്ചിട്ടു കാര്യമില്ല. ഈ വിഷാദം ന്യായീകരിക്കാമെങ്കില്‍ അനിവാര്യമായ മരണത്തെക്കുറിച്ച്‌ അയാള്‍ എന്തുകൊണ്ട്‌ ദിവസവും ദു:ഖിക്കുന്നില്ല?

സ്വപ്രയത്നത്താല്‍ എന്തു നേടാം, എന്തു നേടാനാവില്ല എന്ന് ജ്ഞാനിക്കറിയാം. അതുകൊണ്ട്‌ ഇതിനെയെല്ലാം നമുക്കതീതമായി പുറത്തുള്ള മറ്റേതോ ഒരു ശക്തിയില്‍ ആരോപിക്കുന്നത്‌ തികഞ്ഞ അജ്ഞതയാണ്‌. "ദൈവം എന്നെ സ്വര്‍ഗ്ഗത്തിലേയ്ക്കും നരകത്തിലേയ്ക്കും അയക്കുന്നു" എന്നും "ഏതോ അജ്ഞാതശക്തി എന്നേക്കൊണ്ടിങ്ങനെ ചെയ്യിക്കുന്നു" എന്നും പറയുന്ന അജ്ഞാനിയുമായുള്ള സംസര്‍ഗ്ഗം തന്നെ വര്‍ജിക്കണം.

ഒരുവന്‍ സ്വയം ഇഷ്ടാനിഷ്ടങ്ങളുടെ പിടിയില്‍ നിന്നു മോചിതനാവണം. എന്നിട്ട്‌ സ്വപ്രയത്നം തന്നെ ദിവ്യനിയതി എന്നറിഞ്ഞ്‌ ധാര്‍മ്മീകമായ കര്‍മ്മങ്ങളില്‍ മുഴുകി പരമസത്യത്തെ പ്രാപിക്കണം. "ഗ്രഹപ്പിഴ, ദൈവനിയോഗം എന്നു പറഞ്ഞു നടക്കുന്ന 'വിധിവിശ്വാസി' സ്വയം പരിഹാസപാത്രമാണ്‌." ശരിയായ അറിവില്‍ നിന്നും ഉണ്ടാവുന്നതേ സ്വപ്രയത്നമാവൂ. അങ്ങിനെയുള്ള അറിവ്‌ ഒരുവന്റെ ഹൃദയത്തില്‍ പ്രകടമാവുന്നത്‌ ശാസ്ത്രോചിതമായ പഠനങ്ങളിലൂടെയും മഹാത്മാക്കളുടെ മാര്‍ഗ്ഗനിര്‍ദ്ദേശങ്ങള്‍ പിന്തുടരുന്നതിലൂടെയുമാണ്‌.
