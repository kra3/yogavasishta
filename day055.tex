 
\section{ദിവസം 055}

\slokam{
ആദർശേന്തർ ബഹിശ്ചൈവ യതാ ശൈലോനുഭൂയതേ\\
ബഹിരന്തശ്ചിദാദർശേ തഥാ സർഗോനുഭൂയതേ (3/18/5)\\
}

വസിഷ്ഠന്‍ തുടര്‍ന്നു: സഭയില്‍ എല്ലാ അംഗങ്ങളേയും കണ്ട്‌ ലീലയ്ക്ക്‌ അദ്ഭുതമായി. 'ഇതതിശയം തന്നെ. ഇവര്‍ ഒരേ സമയം രണ്ടിടത്ത്‌ ജീവിച്ചിരിക്കുന്നു. ഇവരെ ഞാന്‍ ധ്യാനാവസ്ഥയില്‍ അവിടെ കണ്ടിരുന്നു. ഇപ്പോള്‍ ഇവിടെ എന്റെ മുന്നിലൂം!' "ഒരുപര്‍വ്വതം കണ്ണാടിയിലെ പ്രതിബിംബത്തിലും പുറത്തും ഒരേസമയം കാണുന്നതുപോലെ ഈ സൃഷ്ടികള്‍ ബോധതലത്തിലും അതിനുപുറത്തും കാണപ്പെടുന്നു." 'പക്ഷേ ഇതില്‍ ഏതാണ്‌ ഉണ്മ? ഏതാണ്‌ പ്രതിബിംബം?. സരസ്വതീ ദേവിയോടു ചോദിക്കുക തന്നെ'

രാജ്ഞി, ദേവിയെ പൂജിച്ചു പ്രത്യക്ഷയാക്കിയിട്ട്‌ ചോദിച്ചു: ദേവീ, ദയവായി പറഞ്ഞാലും. ഈ ലോകം പ്രതിബിംബിക്കുന്നയിടം അതീവനിര്‍മ്മലവും അവിച്ഛിന്നവുമാണല്ലോ. അത്‌ അറിവിനു വിഷയമല്ല. ലോകം പ്രതിബിംബമായും പുറത്ത്‌ ഘനരൂപത്തില്‍ വസ്തു-പദാര്‍ത്ഥ സമുച്ചയങ്ങളായും നിലകൊള്ളുന്നു. ഇവയില്‍ ഏതാണുണ്മ?, ഏതാണു നിഴല്‍ ?

ദേവി ചോദിച്ചു: 'ആദ്യം ഇതിനു നീ മറുപടി പറയുക. നീ എങ്ങിനെയാണ്‌ ഉണ്മയേയും ഉണ്മയല്ലാത്തതിനേയും തരം തിരിക്കുന്നത്‌?'

ലീല പറഞ്ഞു: 'ഞാന്‍ ഇവിടെയും അവിടുന്ന് എന്റെ മുന്നിലും ഉണ്ട്‌. അതുണ്മയായി ഞാന്‍ പരിഗണിക്കുന്നു. എന്റെ പ്രിയതമന്‍ ഇപ്പോള്‍ ഉള്ളയിടത്തെ ഞാന്‍ അയഥാര്‍ത്ഥമായി കണക്കാക്കുന്നു.'

സരസ്വതി ചോദിച്ചു: 'സത്യമല്ലാത്തത്‌ സത്യത്തിനു കാരണമാവുന്നതെങ്ങിനെ? കാര്യം എന്നത്‌ കാരണം തന്നെയാണല്ലോ. അവതമ്മില്‍ തത്ത്വത്തില്‍ യാതൊരു വ്യത്യാസവുമില്ല. മണ്ണ് കൊണ്ടുണ്ടാക്കിയ കുടത്തില്‍ ജലം നിറയ്ക്കാം എന്നാല്‍ അതിന്റെ കാരണമായ മണ്ണിനു ജലം ഉള്‍ക്കൊള്ളാന്‍ ആവില്ല. ഈ വ്യത്യാസം മറ്റു കാരണങ്ങള്‍ കൊണ്ടാണുണ്ടാവുന്നത്‌. നിന്റെ ഭര്‍ത്താവിന്റെ മരണം പ്രാപഞ്ചികമായ ഏതു കാരണത്താലാണുണ്ടായത്‌? സ്ഥൂലമായ (പ്രാപഞ്ചികമായ) ഏതൊരു ഫലസിദ്ധിക്കും സ്ഥൂലമായ കാരണങ്ങള്‍ ഉണ്ടായിരിക്കും. അതിനാല്‍ ഒരു കാര്യത്തിന്റെ കാരണം ഉടനേതന്നെ കണ്ടെത്താനായില്ലെങ്കില്‍ ഒന്നുറപ്പിക്കാം - അതിന്റെ കാരണം മുന്‍പേ തന്നെ സ്മരണയായി നിലനിന്നിരുന്നു. ഈ സ്മരണ അകാശം പോലെ ശൂന്യമാണ്‌. അതിനാല്‍ സൃഷ്ടിയും ശൂന്യമത്രേ. നിന്റെ ഭര്‍ത്താവിന്റെ ജനനം പോലും സ്മരണയിലെ ഒരു വിഭ്രാന്തി മാത്രമാണ്‌. ഇതെല്ലാം അങ്ങനെതന്നെ. ഇത്‌ അയഥാര്‍ത്ഥവും സങ്കല്‍പ്പഫലവുമാണ്‌. സൃഷ്ടിയുടെ മോഹ വിഭ്രാന്തിയെപ്പറ്റി വ്യക്തമാക്കാന്‍ ഞാന്‍ ഒരു കഥ പറയാം. ശുദ്ധബോധത്തില്‍ , സൃഷ്ടാവിന്റെ മനസ്സിന്റെ ഒരു കോണില്‍ നീലമകുടംകൊണ്ടു മൂടിയ ഒരിടിഞ്ഞുപൊളിഞ്ഞ കോവിലുണ്ടായിരുന്നു. അതില്‍ മുറികളായി പതിന്നാലു ലോകങ്ങളുണ്ടായിരുന്നു. അതിലെ ദ്വാരങ്ങള്‍ ആകാശത്തിലെ മൂന്നു ഭാഗങ്ങളായിരുന്നു. സൂര്യന്‍ വിളക്കായിരുന്നു. അതില്‍ ചിതല്‍പ്പുറ്റുകളും (നഗരങ്ങള്‍ ) ചെറു മണ്‍കൂനകളും (മലകളും) ചെറു ജലാശയങ്ങളും (കടല്‍ ) ഉണ്ടായിരുന്നു. ഇതാണ്‌ സൃഷ്ടി; പ്രപഞ്ചം. അതിന്റെയൊരു മൂലയ്ക്ക്‌ ഒരു മഹാത്മാവ്‌ ഭാര്യയോടും കുട്ടികളോടുമൊപ്പം കഴിഞ്ഞുവന്നു. അയാള്‍ നിര്‍ഭയനും ആരോഗ്യവാനുമായിരുന്നു. ധാര്‍മ്മീകവും സമൂഹികവുമായ എല്ലാക്കാര്യങ്ങളും അയാള്‍ ഭംഗിയായി നിര്‍വ്വഹിച്ചു വന്നു.
