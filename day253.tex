\section{ദിവസം 253}

\slokam{
ദൈത്യോദ്ധ്യോഗേന വിബുധാസ്തതോ യജ്ഞതപ:ക്രിയാ:\\
തേന സംസാരസംസ്ഥാനം ന സംസാരക്രമോഽന്യഥാ  (5/38/16)\\
}

വസിഷ്ഠന്‍ തുടര്‍ന്നു: ഇങ്ങിനെ മനനം ചെയ്തു ധ്യാനിച്ചിരുന്ന പ്രഹ്ലാദന്‍ പരമാനന്ദത്തിന്റെ അഭൌമതലത്തിലേയ്ക്ക് പ്രവേശിച്ചു. ചിന്താസഞ്ചാരങ്ങളുടെ ശല്യമോ മനോപാധികളോ ഇല്ലാത്തൊരു അതീതതലമത്രേ അത്. അദ്ദേഹമവിടെയൊരു പ്രതിമപോലെയിരുന്നു. കാലമേറെ കടന്നുപോയി. അസുരന്മാരദ്ദേഹത്തിനെ ശല്യപ്പെടുത്താന്‍ ശ്രമിച്ചുവെങ്കിലും അത് നടന്നില്ല. ആയിരം കൊല്ലം കഴിഞ്ഞു. അദ്ദേഹത്തിന്‍റെ അന്ത്യമായിയെന്നു അസുരന്മാര്‍ കരുതി. പാതാളലോകങ്ങളില്‍ നാഥനില്ലാത്ത അവസ്ഥ സംജാതമായി. ഹിരണ്യകശിപു അന്തരിച്ചു. അദ്ദേഹത്തിന്‍റെ പുത്രന്‍ ’മരിച്ചതിനാല്‍ ’ (ലോകത്തെ സംബന്ധിച്ചിടത്തോളം) സിംഹാസനമേല്‍ക്കാന്‍ ആരുമുണ്ടായില്ല. അസുരന്മാര്‍ അവരുടെ തന്നിഷ്ടം പോലെ ജീവിച്ചു.

എല്ലായിടത്തും ക്രമസമാധാനം തകരാറിലായി. സമുദ്രത്തിലെ ചെറുജീവികളെ വന്‍സ്രാവുകള്‍ തിന്നുന്നപോലെ അവശരായവരെ ബലവാന്മാര്‍ കീഴടക്കി. അങ്ങിനെയിരിക്കെ വിശ്വപാലകനായ മഹാവിഷ്ണു പാല്‍ക്കടലില്‍ അനന്തസര്‍പ്പം എന്ന പള്ളിമെത്തയില്‍ ചാരിക്കിടന്ന് വിശ്വാവലോകനം ചെയ്തു. സ്വര്‍ഗ്ഗത്തും നരകത്തിലും സ്ഥിതിഗതികള്‍ മനസാ നന്നെന്നു കണ്ട് ഭഗവാന്‍ സന്തോഷിച്ചു. പിന്നീടദ്ദേഹം പാതാളലോകം ദര്‍ശിച്ചു. പ്രഹ്ലാദന്‍ ദീര്‍ഘദ്ധ്യാനത്തില്‍ ആമഗ്നനായിരിക്കുന്നു. അസുരവര്‍ഗ്ഗത്തില്‍ നിന്നും യാതൊരു ശല്യവുമില്ലാത്തതിനാല്‍ ദേവന്മാര്‍ ഐശ്വര്യ സമൃദ്ധികള്‍ ആവോളം ആസ്വദിക്കുകയാണ്.

ഭഗവാന്‍ ആലോചിച്ചു.: പ്രഹ്ലാദന്‍ അതീന്ദ്രിയ ധ്യാനത്തിലാകയാല്‍ അസുരവര്‍ഗ്ഗത്തിനു നാഥനില്ലാത്ത അവസ്ഥയാണിപ്പോള്‍.  അവരുടെ ശക്തിയെല്ലാം നഷ്ടമായിരിക്കുന്നു.  

അസുരന്മാരുടെ ഭീഷണിയില്ലാത്തപ്പോള്‍ സ്വര്‍ഗ്ഗത്തിലെ ദേവന്മാര്‍ക്കൊന്നും പേടിക്കാനില്ല. അതുകൊണ്ടുതന്നെ ഒന്നും വെറുക്കാനുമില്ല. രാഗദ്വേഷങ്ങളെന്ന ദ്വന്ദശക്തികളില്ലെങ്കില്‍ അവരും അതീന്ദ്രിയതലത്തിലേയ്ക്കുയരും. അവര്‍ക്ക് മുക്തിപദവും ലഭിക്കും. അപ്പോള്‍ ദേവന്മാരുടെ അഭാവത്തില്‍ ഭൂമിയിലുള്ളവര്‍ക്ക് ധര്‍മ്മാചരണങ്ങള്‍ അര്‍ത്ഥരഹിതമായി അനുഭവപ്പെടും. കാരണം അത്തരം കര്‍മ്മങ്ങളില്‍ പ്രീതികൈക്കൊണ്ടു പങ്കെടുക്കാന്‍ ദേവതമാരുണ്ടാകയില്ലല്ലോ. സ്വാഭാവികമായ പ്രളയകാലംവരെ ഈ വിശ്വം നിലനില്‍ക്കണം. ഇപ്പോഴത്തെ സ്ഥിതി അനുവദിച്ചാല്‍ വിശ്വസ്ഥിതി പൊടുന്നനെ അവതാളത്തിലാവും. ഇതില്‍ ഞാനൊരു നന്മയും കാണുന്നില്ല. അസുരന്മാര്‍ അവരുടെ സഹജഭാവം തുടരട്ടെ.

“അസുരന്മാര്‍ ദേവവൈരികളായി തുടര്‍ന്നാല്‍ മാത്രമേ ധാര്‍മ്മികതയും അതോടനുബന്ധിച്ച കര്‍മ്മങ്ങളും ഈ സൃഷ്ടിചക്രത്തില്‍ പുഷ്ടിപ്രാപിച്ചു നിലനില്‍ക്കകയുള്ളു.” അതുകൊണ്ട് ഞാനുടനെതന്നെ പാതാളലോകത്തു ചെന്ന് കാര്യങ്ങളെ പഴയപടിയാക്കി തീര്‍ക്കുന്നതാണ്. പ്രഹ്ലാദന് ചക്രവര്‍ത്തിപദം അഭികാമ്യമല്ലെങ്കില്‍ ഞാന്‍ മറ്റൊരാളിനെ അതിനായി കണ്ടെത്തും. പ്രഹ്ലാദന്റെ അവസാന ജന്മമാണിത്. എന്നാലീ ലോകചക്രം തീരുംവരെ അദ്ദേഹം ഈ ദേഹത്തില്‍ കഴിയണമെന്നത് ലോകധര്‍മ്മം മാത്രമാണല്ലോ.

ഞാന്‍ പാതാളലോകത്തുചെന്നു പ്രഹ്ലാദനെ അലറിവിളിച്ചുണര്‍ത്താന്‍ പോവുന്നു. മുക്തിപദത്തിലിരുന്നുകൊണ്ട് തന്നെ ഭരണം കയ്യാളാന്‍ ഞാനദ്ദേഹത്തെ ആഹ്വാനം ചെയ്യും. അങ്ങിനെ സ്വാഭാവികപ്രളയം വരെ അദ്ദേഹത്തിലൂടെ ഈ ലോകചക്രത്തെ മുന്നോട്ടു നയിക്കാന്‍ കഴിയും.
