\newpage
\section{ദിവസം 108}

\slokam{
സദസദിതി കാലഭിരാതതം യത്\\
സദസദ്ബോധവിമോഹദായിനീഭി:\\
അവിരതരചനാഭിരീശ്വരാത്മൻ\\
പ്രവിലസതീഹ മനോ മഹന്മഹാത്മൻ (3/85/39)\\
}

വസിഷ്ഠന്‍ തുടര്‍ന്നു: ഒരിക്കല്‍ സൃഷ്ടികര്‍ത്താവായ ബ്രഹ്മാവിനോട്‌ ഈ വിശ്വനിര്‍മ്മിതി ആദ്യം എങ്ങിനെയാണു സംഭവിച്ചതെന്ന് ഞാന്‍ ചോദിക്കുകയുണ്ടായി. അദ്ദേഹം ഇങ്ങിനെപറഞ്ഞു: മകനേ മനസ്സാണ്‌ ഇക്കാണായതെല്ലാം ആവുന്നത്‌. ഈ യുഗാരംഭത്തില്‍ എനിക്കെന്തു സംഭവിച്ചുവെന്ന് ഞാന്‍ പറയാം. കഴിഞ്ഞ യുഗാന്ത്യത്തില്‍ വിശ്വമുറങ്ങിക്കിടന്ന രാത്രിയുടെ അന്ത്യത്തില്‍ ഞാനുണര്‍ ന്നു. പ്രഭാതത്തിലെ പ്രാര്‍ത്ഥനകള്‍ കഴിഞ്ഞ്‌ വിശ്വനിര്‍മ്മിതി എന്ന ആശയത്തോടെ ഞാന്‍ ചുറ്റും നോക്കി. ഞാനാ അനന്തശ്ശൂന്യതയിലേയ്ക്കു നോക്കിയപ്പോള്‍ അത്‌ ഇരുണ്ടതോ ദീപ്തമോ ആയിരുന്നില്ല. എന്റെ മനസ്സില്‍ സൃഷ്ടിക്കുള്ള അഭിവാഞ്ഛയുണ്ടായപ്പോള്‍ ഹൃദയത്തില്‍ സൂക്ഷ്മമായ ദൃശ്യങ്ങള്‍ പ്രത്യക്ഷമാവാന്‍ തുടങ്ങി. മനക്കണ്ണുകൊണ്ട്‌ ഞാന്‍ അനേകം വിവിധങ്ങളായ പ്രപഞ്ചങ്ങള്‍ കണ്ടു. അവകളിലെല്ലാം എന്നേപ്പോലുള്ള സൃഷ്ടികര്‍ത്താക്കളേയും ഞാന്‍ കണ്ടു. ആ ലോകങ്ങളില്‍ ഞാന്‍ എല്ലാത്തരം ജീവജാലങ്ങളേയും കണ്ടു. മലകളും നദികളും, സമുദ്രവും കാറ്റും, സൂര്യനും സ്വര്‍ഗ്ഗവാസികളും, പാതാളവും രാക്ഷസന്മാരും എല്ലാം എനിക്കു കാണായി.

ആ വിശ്വങ്ങളിലെല്ലാം വേദങ്ങളും ആചാരമര്യാദാ പ്രമാണങ്ങളും കണ്ടു. അവയാണല്ലോ നന്മ-തിന്മകളെയും സ്വര്‍ഗ്ഗ-നരകങ്ങളെയും നിശ്ചയിക്കുന്നത്‌. അവിടെ ഞാന്‍ മുക്തി മാര്‍ഗ്ഗവും സൌഖ്യമാര്‍ഗ്ഗവും വിവരിക്കുന്ന വേദങ്ങളേയും കണ്ടു. ഇത്തരം വിവിധങ്ങളായ പല ലക്ഷ്യങ്ങളുമായി പ്രവര്‍ത്തിക്കുന്ന ആളുകളേയും ഞാന്‍ കണ്ടു. ഞാന്‍ ഏഴുലോകങ്ങളും ഏഴു ഭൂഖണ്ഡങ്ങളും മലകളും സമുദ്രങ്ങളും എല്ലാം നാശത്തിലേയ്ക്കു പായുന്നതായും കണ്ടു. ദിനരാത്രങ്ങളടക്കമുള്ള കാലഗണനകള്‍ ഞാന്‍ കണ്ടു. ദിവ്യനദിയായ ഗംഗ, മൂന്നുലോകങ്ങളേയും -സ്വര്‍ഗ്ഗം, ആകാശം, ഭൂമി എന്നിവയെ- കോര്‍ത്തിണക്കുന്നതായും ഞാന്‍ കണ്ടു. ആകാശത്തുണ്ടാക്കിയ കോട്ടപോലെ സൃഷ്ടി, അതാതിന്റെ ഭൂമിയും, സമുദ്രവും ആകാശവുമൊക്കെയായി വിസ്തൃതമായി നിലകൊണ്ടു. ഇതൊക്കെക്കണ്ട്‌ ഞാന്‍ വിസ്മയചകിതനായി. "എന്താണു ഞാനെന്റെ മനസ്സില്‍ ഇവയെല്ലാം കാണുന്നത്‌? എന്റെ കണ്ണൂകള്‍ ഇവയെ ഇതുവരെ കണ്ടിട്ടുമില്ല." ഞാന്‍ ഇതിനെപ്പറ്റി കുറേയേറെ ആലോചിച്ചു. അവസാനം സൌരയൂഥങ്ങളില്‍ ഒന്നിലെ ഒരു സൂര്യനെക്കുറിച്ച്‌ ഞാന്‍ ചിന്തിച്ചിട്ട്‌ അദ്ദേഹത്തോട്‌ എനിക്കരികിലേയ്ക്ക്‌ വരാന്‍ പറഞ്ഞു. എന്നെ അലട്ടിയിരുന്ന ഈ പ്രശ്നത്തെപ്പറ്റി ഞാന്‍ അദ്ദേഹത്തോടു ചോദിച്ചു.

സൂര്യന്‍ മറുപടിയായി പറഞ്ഞു: "അല്ലയോ മഹാത്മന്‍, സര്‍വ്വശക്തനായ സൃഷ്ടാവെന്ന നിലയില്‍ അങ്ങ്‌ ഈശ്വരന്‍ തന്നെയാണ്‌.. മനസ്സുതന്നെയാണ്‌ ഈ നിലയ്ക്കാത്ത പ്രവര്‍ത്തനങ്ങളും അന്തമില്ലാത്ത സൃഷ്ടികളൂം ആയി കാണപ്പെടുന്നത്‌.. അവിദ്യയുടെ പ്രാഭവംകൊണ്ട്‌ ഇവയെല്ലാം യാഥാര്‍ത്ഥ്യമാണെന്നുള്ള ധാരണ ഒരുവനെ ഭ്രമിപ്പിക്കുകയാണ്‌.". തീര്‍ച്ചയായും അങ്ങേയ്ക്കു സത്യമറിയാമെങ്കിലും എന്നോട്‌ ഉത്തരം പറയാന്‍ ആവശ്യപ്പെട്ടതുകൊണ്ട്‌ ഞാനിതു പറഞ്ഞു എന്നേയുള്ളു.
