\section{ദിവസം 183}

\slokam{
ശുഭാശുഭ പ്രസര പരാഹതാകൃതൗ\\
ജ്വലജ്ജരാമരണവിഷാദമൂർച്ഛിതേ\\
വ്യഥേഹ യസ്യ മനസി ഭോ ന ജായതേ\\
നരാകൃതിർ ജഗതി സ രാമ രാക്ഷസ: (4/42/52)\\
}

വസിഷ്ഠൻ തുടർന്നു: അനന്താവബോധം എങ്ങിനെയാണ്‌ വൈവിദ്ധ്യമാർന്ന് ജീവനും മറ്റുമായിത്തീർന്നതെന്ന് ഞാൻ ഒന്നുകൂടി വിശദമാക്കാം. ഒരു സമുദ്രത്തിൽ നോക്കിയാൽ ചിലയിടങ്ങൾ പ്രശാന്തവും മറ്റിടങ്ങൾ പ്രക്ഷുബ്ധവുമാണെന്നു കാണാം. അതുപോലെ അനന്താവബോധം ചിലയിടത്ത് വൈവിദ്ധ്യത്തെ സ്വീകരിക്കുന്നതായും മറ്റിടങ്ങളിൽ അദ്വൈതഭാവത്തിൽ നിലകൊള്ളുന്നതായും നമുക്കറിയാനാകുന്നു. സർവ്വശക്തിമത്തായ അനന്താവബോധം അനന്തമായ വിഭൂതികളോടെ പ്രകടമാവുന്നത് സഹജമെന്നുവേണം കരുതാൻ. എന്നാൽ ഈ പ്രകടനം നടപ്പിലാക്കാൻ സമയം, ആകാശം (ദൂരം), കാര്യകാരണങ്ങൾ എന്നിവയുടെ സഹായം അനിവാര്യമാണ്‌.  അങ്ങിനെയാണ്‌ അനന്തമായ നാമരൂപങ്ങൾ ഉദ്ഭൂതമായത്. എന്നാൽ ഈ പ്രകടമായ കാഴ്ച്ചാവിശേഷങ്ങളെല്ലാം അനന്താവബോധത്തിൽ നിന്നും വിഭിന്നവുമല്ല. നാമരൂപങ്ങളായി, കാലദേശങ്ങളായി, കാര്യ-കാരണങ്ങളായി, പ്രത്യക്ഷപ്പെടുന്നവയെ അനന്തതയുമായി ബന്ധിപ്പിക്കുന്ന സത്തയ്ക്ക് സാക്ഷിബോധം (ക്ഷേത്രജ്ഞൻ) എന്നു പറയുന്നു.

ഈ ശരീരം ക്ഷേത്രമാണ്‌..  അതിന്റെ അകവും പുറവും സമ്യക്കായി അറിയുന്നവനാണ് ക്ഷേത്രജ്ഞൻ, അല്ലെങ്കിൽ സാക്ഷിബോധം. ഈ സാക്ഷിബോധമാണ്‌ ലീനവാസനകളാൽ പ്രചോദിതമായി അഹംകാരത്തെ ഉണ്ടാക്കുന്നത്. ഈ അഹംകാരം ധാരണകളും വിവക്ഷകളും ഉളവാക്കുമ്പോൾ അത് ബുദ്ധിയെന്നറിയപ്പെടുന്നു. ആലോചനയ്ക്കുള്ള ഉപാധിയാവുമ്പോൾ ഇത് മനസ്സെന്ന് അറിയപ്പെടുന്നു. ഈ പ്രജ്ഞ, മാറ്റങ്ങൾക്കു വിധേയമാവുകയോ വികൃതമാവുകയോ ചെയ്യുമ്പോൾ അത് പഞ്ചേന്ദ്രിയങ്ങളാവുന്നു. ഇവയെല്ലാം ചേർന്ന് ശരീരമാവുന്നു. പക്വതയാകുന്ന മുറ്യ്ക്ക് വലുപ്പത്തിലും നിറത്തിലും ഫലത്തിലുണ്ടാവുന്ന മാറ്റങ്ങൾ പോലെ ബോധം മാറ്റങ്ങൾക്കു വിധേയമാകുന്നതായി തോന്നുന്നു. അതോടെ അജ്ഞാനം ആഴത്തിലാഴത്തിൽ വളരുകയും ചെയ്യുന്നു. മൂഢൻ തന്റെ സത്യാന്വേഷണം അവസാനിപ്പിച്ച് സ്വയം അജ്ഞാനത്തെ ആനന്ദമെന്ന് തെറ്റിദ്ധരിച്ച് അതിനെ ആവേശത്തോടെ പുല്‍കുന്നു. കര്‍മ്മങ്ങള്‍ എന്ന, സ്വയം ഉണ്ടാക്കിയ വലയിൽ കുടുങ്ങി, താനാണ്‌ എല്ലാം ചെയ്യുന്നതെന്ന് കരുതി അവന്‍ അന്തമില്ലാത്ത ദുരിതങ്ങൾക്കു വശംവദനാകുന്നു. ഇവയെല്ലാം സ്വയംകൃതവും സ്വാഭീഷ്ടപ്രദവുമായാണ്‌ കൈവന്നതെന്ന്‍  അവനറിയുന്നതേയില്ല.

രാമാ, ഈ ലോകത്ത് ശോകത്തിന്റെ ഒരേയൊരു കാരണം മനസ്സാണ്‌.. മനസ്സ് മുഴുവൻ ദു:ഖങ്ങളും, വിപത്തുക്കളും, ആസക്തികളും, മോഹവിഭ്രമങ്ങളുമാണ്‌.. ആത്മജ്ഞാനത്തെ മറന്ന് മനസ്സ് ആഗ്രഹങ്ങളെയും ക്രോധത്തെയുമുണ്ടാക്കുന്നു. അത് ദുഷ്ച്ചിന്തകളേയും ആസക്തികളേയും വളർത്തി മനുഷ്യനെ ഇന്ദ്രിയവിഷയങ്ങളാകുന്ന എരിതീയിലേയ്ക്ക് ആട്ടിത്തെളിക്കുന്നു. രാമാ, മനസ്സിനെ ഈ അജ്ഞാനചക്രത്തിൽ നിന്നു കരകേറ്റിയാലും. “രാമാ, മനസ്സിന്റെ മലിനാവസ്ഥയിൽ സ്വയം പരിതപിക്കാത്തവൻ മനുഷ്യരൂപത്തിലുള്ള രാക്ഷസനത്രേ. നന്മയും തിന്മയും മാറിമാറിയാണ്‌ അവന്റെ മനസ്സിലെ ചിന്തകൾ. അവന്‌ ജരാനരയും, മരണവും വ്യസനവും സഹജമാണ്‌” 

