\section{ദിവസം 252}

\slokam{
ത്വദാലോകേക്ഷണോത്ഭൂതാ ത്വദാലോകേക്ഷണക്ഷയാ\\
മൃതേവ ജാതാ ജാതേവ മൃതാ കേനോപലക്ഷ്യതേ   (5/36/71)\\
}

പ്രഹ്ലാദന്‍ തന്റെ ധ്യാനം തുടര്‍ന്നു: ആത്മാവേ, നീയുള്ളതുകൊണ്ട് മാത്രമാണ് സുഖദു:ഖങ്ങള്‍ക്ക് പ്രസക്തിയുണ്ടാവുന്നത്. അവനിന്നിലാണുണ്ടാവുന്നത്. എന്നാലവയ്ക്ക് നിന്നില്‍നിന്നും  വേറിട്ടൊരസ്തിത്വമില്ല എന്നറിയുമ്പോള്‍ അവയുടെ അനന്യത്വം ഇല്ലാതെയാവുന്നു. കണ്ണടച്ചു തുറക്കുന്ന നിമിഷത്തില്‍ ദൃശ്യവിസ്മയങ്ങള്‍ ഇല്ലാതാവുംപോലെ  സുഖദു:ഖങ്ങളെന്ന ഭ്രമകല്‍പ്പനകള്‍ കണ്ണടച്ചുതുറക്കുംമുമ്പ് ഉണ്ടായി മറയുകയാണ്.     

“അവ പ്രത്യക്ഷപ്പെടുന്നത് ഉണര്‍വ്വിന്റെ ഉള്‍പ്രകാശത്താലാണ്. എന്നാല്‍ ആ പ്രകാശസ്രോതസ്സില്‍നിന്നും  വിഭിന്നമല്ല എന്നറിയുമ്പോള്‍ അവ അപ്രത്യക്ഷമാവുകയും ചെയ്യുന്നു. അവ മരിച്ചയുടനെ ജനിച്ചവയാണ്. ജനിച്ചയുടനെ മരിച്ചവയുമാണ്. ആരാണീ മാസ്മരീകതയെ എല്ലാമറിയുന്ന സത്ത?” എല്ലാമെല്ലാം എപ്പോഴും മാറിക്കൊണ്ടിരിക്കുന്നു. അപ്പോള്‍പ്പിന്നെ കേവലം ക്ഷണഭംഗുരമായ കാരണങ്ങളാല്‍ ശാശ്വതമായ കാര്യങ്ങള്‍ എങ്ങിനെ സംഭവിക്കുവാനാണ്? കടലലകള്‍ ചിലപ്പോള്‍ പൂക്കളായി തോന്നിയേക്കാം. എന്നാലവകൊണ്ടൊരു മാലകോര്‍ക്കാനാവുമോ? യാതൊരടിസ്ഥാനവുമില്ലാത്തതും നശ്വരവുമായ കാരണങ്ങളാല്‍ ദൃഢതയുള്ള കാര്യങ്ങള്‍ ഉണ്ടാവുമെന്ന് വിശ്വസിക്കുന്നതെത്ര വിഡ്ഢിത്തമാണ്? മിന്നല്‍ പിണരുകള്‍ കൊരുത്തൊരു മാലയുണ്ടാക്കി അത് കണ്ഠാഭാരണമാക്കാന്‍ കഴിയുമോ?

അല്ലയോ ആത്മാവേ, സമതയിലഭിരമിക്കുന്ന ആത്മജ്ഞാനിയുടെ ഉണര്‍വ്വാര്‍ന്ന ജാഗ്രതയില്‍ നീ സുഖദു:ഖങ്ങളെ യാഥാര്‍ത്ഥ്യമെന്നവണ്ണം സ്വീകരിച്ച് ആസ്വദിക്കുന്നു, അനുഭവിക്കുന്നു. എന്നാല്‍ അജ്ഞാനിയായ ഒരുവനില്‍ നീ എങ്ങിനെയാണ് സുഖദു:ഖങ്ങളെ അനുഭവിക്കുന്നതെന്നു പറയുക അസാദ്ധ്യം! ആത്മാവേ നീ സത്യത്തില്‍ ഒന്നിനോടും ആസക്തിയില്ലാത്ത, യാതൊരാശകളും ഇല്ലാത്ത, അഹംകാരമസ്തമിച്ച അവിച്ഛിന്നസ്വരൂപമാണ്. കേവലം സത്യമായാലുമല്ലെങ്കിലും, ഭ്രമകല്‍പ്പനകളായാലുമല്ലെങ്കിലും നീ തന്നെയാണ് കര്‍ത്താവും ഭോക്താവുമായി വൈവിദ്ധ്യതയാര്‍ന്ന് നിലകൊള്ളുന്നത്.
     
ജയ ജയ ആത്മന്‍ ജയജയ.! അനന്തവിശ്വമായി പ്രകടമായ ആത്മാവ് ജയിക്കട്ടെ. പരമപ്രശാന്തമായ ആത്മാവ് ജയിക്കട്ടെ. ശാസ്ത്രങ്ങള്‍ക്കുപോലും അതീതമായ ആത്മാവ് ജയിക്കട്ടെ. വേദശാസ്ത്രങ്ങള്‍ നിന്നെയാണാധാരമാക്കുന്നത്. നീയാണാ ശാസ്ത്രഗ്രന്ഥങ്ങളില്‍ വിരാജിക്കുന്നത്. ഇനിയും ജനിച്ചിട്ടില്ലാത്ത ആത്മാവിനു നമസ്കാരം. ഒരേസമയം മാറ്റങ്ങള്‍ക്കും നാശങ്ങള്‍ക്കും വിധേയവും അല്ലാത്തതുമായ ആത്മാവ് ജയിക്കട്ടെ. അസ്തിത്വവും അനസ്തിത്വവുമായ ആത്മാവ് ജയിക്കട്ടെ. ഒരേസമയം കീഴടക്കാനും പ്രാപിക്കാനുമാവുന്നതും എന്നാല്‍ അജയ്യനും അപ്രാപ്യനുമായ ആത്മാവേ നീ ജയിക്കുമാറാകട്ടെ.

ഞാന്‍ ആഹ്ലാദചിത്തനാണ്. ഞാന്‍ സമതയിലഭിരമിക്കുന്നു. എന്നില്‍ എല്ലാം സംതുലിതാവസ്ഥയെ പ്രാപിച്ചിരിക്കുന്നു. ഞാന്‍ അചലന്‍. ഞാന്‍ ജയന്‍. ജയിക്കാനായി ജീവിക്കുന്നവനാണ് ഞാന്‍. എനിക്ക് നമസ്കാരം. നിനക്ക് നമസ്കാരം.

നിത്യശുദ്ധസത്തയായി നീയുള്ളപ്പോള്‍ ബന്ധനമെവിടെ? സൌഭാഗ്യ-ദൌര്‍ഭാഗ്യങ്ങളെവിടെ? ജനന-മരണങ്ങളെവിടെ? ഞാന്‍ ശാശ്വതമായ പരമപ്രശാന്തതയില്‍ സദാ അഭിരമിക്കുന്നു.

