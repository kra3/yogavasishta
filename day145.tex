\section{ദിവസം 145}

ഭാഗം 4. സ്ഥിതി പ്രകരണം ആരംഭം 

\slokam{
സാകാരവടധാനാദാവങ്കുരാ: സന്തി യുക്തിമത്\\
നാകാരേ തന്മഹാകാരം ജഗദസ്തി ത്യയുക്തികം (33)\\
}

വസിഷ്ഠൻ തുടർന്നു: രാമ: ലോകസൃഷ്ടിയുടെ പുറകിലുള്ള സത്യം ഞാൻ വെളിപ്പെടുത്തിയല്ലോ. ഇനി ഞാൻ ഈ സൃഷ്ടികളുടെയും പ്രത്യക്ഷമായ ഈ ലോകത്തിന്റെയും സ്ഥിതി പരിപാലനം എങ്ങിനെയാണു നടക്കുന്നതെന്ന് നിനക്കു പറഞ്ഞു തരാം. ലോകമെന്ന ഈ മായക്കാഴ്ച്ച ഉള്ളിടത്തോളം കാലം മാത്രമേ ഈ ലോകം ‘അറിയപ്പെടുന്ന’ ഒരു വസ്തുവായി നിലനിൽക്കുന്നുള്ളു. സ്വപ്നദൃശ്യത്തിന്റെ പോലെയുള്ള സത്യാവസ്ഥയാണിതിനുള്ളത്. കാരണം ഒന്നുമില്ലായ്മയിൽനിന്നും യാതൊരുപകരണങ്ങളും കൂടാതെ, സൃഷ്ടാവായി ആരുമില്ലാതെ, ‘ഉണ്ടായതാണ്‌’ ഈ ലോകം. അതായത് ദിവാസ്വപ്നം പോലെ അയാഥാർത്ഥമാണിത്. മഴവില്ലുപോലെ, ശൂന്യതയിൽ വരച്ച ചിത്രപടമാണത്. പരന്നു വികസിച്ച മൂടൽ മഞ്ഞുപോലെ പിടിക്കാൻ ചെന്നാൽ ഒന്നുമവശേഷിക്കാത്ത ഒന്നാണത്. ചിലചിന്തകന്മാർ ലോകത്തെ ജഢമെന്നു കരുതുന്നു. ചിലർ ശൂന്യമെന്നും മറ്റുചിലർ അണുക്കളുടെ സംഘാതമാണിതെന്നും പറയുന്നു.

രാമൻ ചോദിച്ചു: വിശ്വം ബീജാവസ്ഥയിൽ പരമപുരുഷനിൽ നിലകൊള്ളുന്നു എന്നും അടുത്തയുഗാരംഭത്തിൽ പ്രകടമാവുന്നു എന്നും പറയപ്പെടുന്നു. ഇതെങ്ങിനെയാണ്‌? ഈ വിശ്വാസം വച്ചുപുലർത്തുന്നവരെ പ്രബുദ്ധരെന്നാണോ അജ്ഞാനികളെന്നാണോ കരുതേണ്ടത്?

വസിഷ്ഠൻ തുടർന്നു: ഈ കാണപ്പെട്ട വിശ്വം പ്രളയശേഷം ബീജാവസ്ഥയിൽ നിലകൊള്ളൂന്നു എന്നു പറയുന്നവർ ഈ വിശ്വം സത്യമാണെന്ന് ഉറച്ച വിശ്വാസമുള്ളവരാണ്‌.. അത് ശുദ്ധമായ അജ്ഞാനമാണ്‌ രാമാ. ഈ വികലദർശനം ഗുരുവും ശിഷ്യനും മോഹവലയത്തിലാണെനുള്ളതിന്‌ തെളിവാണ്‌.. ഒരു മരത്തിന്റെ വിത്തിൽ സൂക്ഷ്മമായുള്ളത് ഭാവിയിലുണ്ടാകാൻ പോവുന്ന മരമാണ്‌.. ആ സാദ്ധ്യതമാത്രമാണ്. വിത്ത്, മുള, ചെടി, മരം, ഒക്കെ ‘അറിവിന്റെ’ തലത്തിലുള്ള പദാർത്ഥങ്ങളാണ്‌. മനസ്സിനും ഇന്ദ്രിയങ്ങൾക്കും ‘അറിയാൻ’ കഴിയുന്ന വിഷയവസ്തുക്കളാണവ. 

എന്നാൽ മനസ്സേന്ദ്രിയങ്ങളാൽ അറിയാൻ കഴിയുന്നതിനതീതമായുള്ളവ എങ്ങിനെയാണ്‌ ലോകങ്ങൾക്ക് ബീജമാവുന്നത്? ആകാശത്തേക്കാൾ സൂക്ഷ്മമായതിന്റെ അകത്ത് വിശ്വത്തിന്റെ ബീജമെങ്ങിനെ നിലകൊള്ളൂം? കാര്യങ്ങളങ്ങിനെയിരിക്കുമ്പോൾ പരമപുരുഷനിൽ നിന്ന് വിശ്വത്തിന്റെ ബീജം എങ്ങിനെ ആവിർഭവിക്കാനാണ്‌? ശൂന്യതയിൽ എന്തിനാണ്‌ നിലനിൽക്കാൻ കഴിയുക? അഥവാ ശൂന്യതയിൽ വിശ്വമെന്നൊരു വസ്തു ഉണ്ടെങ്കിൽ അതെന്തുകൊണ്ട് കാണപ്പെടുന്നില്ല? ഒരു പാത്രത്തിനുള്ളിലെ ശൂന്യസ്ഥലത്ത് ഒരു മരമുണ്ടാവുന്നതെങ്ങിനെ? രണ്ടു വിരുദ്ധ വസ്തുക്കൾ (വിശ്വവും ബ്രഹ്മവും) ഒന്നിച്ചു നിലനിൽക്കുന്നതെങ്ങിനെ? സൂര്യനുള്ളപ്പോൾ ഇരുട്ടിനു നിലനിൽക്കാനാവുമോ? മരം വിത്തിൽ നിലകൊള്ളുന്നു എന്നുപറയുന്നതിൽ തെറ്റില്ല. കാരണം അവയ്ക്കു രണ്ടിനും നിയതമായ രൂപങ്ങളുണ്ട്. എന്നാൽ രൂപബദ്ധമല്ലാത്ത ബ്രഹ്മത്തിൽ വിശ്വബീജം നിലകൊള്ളുന്നുവെന്നു പറയുന്നത് യുക്തിയല്ല. അതിനാൽ ബ്രഹ്മവും വിശ്വവും തമ്മിൽ ‘കാര്യ-കാരണബന്ധം’ ഉണ്ടെന്നു പറയുന്നത് മൂഢത്വമാണ്‌.. സത്യത്തിൽ ബ്രഹ്മം മാത്രമേ നിലനില്‍ക്കുന്നുള്ളു. പ്രത്യക്ഷമായി, പ്രകടമായി കാണപ്പെടുന്ന ലോകവും ബ്രഹ്മം തന്നെയാണ്. 
