\section{ദിവസം 185}

\slokam{
സ്വഭാവ കല്പിതോ രാമ ജീവാനാം സർവദൈവ ഹി\\
അമോക്ഷപദസംപ്രാപ്തി സംസാരോസ്ത്യാത്മനോന്തരേ (4/44/6)\\
}

വസിഷ്ഠൻ തുടർന്നു: ഏതായാലും ഈ സൃഷ്ടികളെല്ലാം ഒരു സ്വപ്നത്തിലെന്നപോലെ മാത്രമേ സംഭവിക്കുന്നുള്ളു. കാരണം ഈ സൃഷ്ടികൾ സത്യമല്ല; സത്യമാണെന്നൊരു പ്രതീതിയുണ്ടായി എന്നുമാത്രം. അജ്ഞാനത്തെ മുഴുവനായി നീക്കിയ, മനോപാധികൾ നിലച്ചവനാണല്ലോ  ഋഷി. അയാൾക്ക് ഈ സ്വപ്ന സമാനമായ പ്രത്യക്ഷലോകത്തെപ്പറ്റി ‘അറിവ്’ ഉള്ളതായി നമുക്ക് തോന്നിയേക്കാം. എന്നാൽ അയാൾ അതിനെ ലോകമായല്ല കാണുന്നത്.

“ജീവന്‌ മുക്തിയാകും വരെ എല്ലാ ജീവജാലങ്ങളിലും എല്ലാ കാലത്തും ഈ പ്രത്യക്ഷലോകം സഹജമായി ആവിഷ്ക്കരിക്കപ്പെടുകയാണ്‌.” എല്ലാ ജീവനിലും ഒരു സാദ്ധ്യതയായി ഈ സ്തൂലശരീരമുണ്ട്. അത് ഭൗതീകമായി ഉണ്ടെന്നല്ല, ഒരു സംഭാവ്യസാദ്ധ്യതയായി, ഇച്ഛയായി, ചിന്തയായി നിലകൊള്ളുന്നു എന്നർത്ഥം.

എങ്ങിനെയാണ്‌ സൃഷ്ടാവായ ബ്രഹ്മാവ് അനന്താവബോധത്തിൽ ഉദ്ഭൂതമായതെന്ന് ഞാൻ ഒന്നു കൂടി വിശദമാക്കാം. അതുപോലെ എണ്ണമറ്റ ജീവജാലങ്ങൾ എങ്ങിനെയാണ്‌ ബോധമണ്ഡലത്തിലുദിച്ചുയർന്നതെന്നും പറയാം. കാലദേശബന്ധിതമല്ലാത്ത അനന്താവബോധം ഒരു ലീലയായി, ഈ ജീവജാലങ്ങളായി, സ്വയം അനുമാനിക്കുകയാണ്‌.. അങ്ങിനെ വിശ്വപുരുഷന്‍ സ്വയം സംജാതനായി. ഈ വിശ്വപുരുഷനാകട്ടെ വിശ്വമനസ്സും ജീവനുമാണ്‌.. ഈ പുരുഷൻ ശബ്ദത്തെ അനുഭവിക്കാനിച്ഛിക്കുമ്പോൾ ശബ്ദമുണ്ടായി. ആകാശമുണ്ടാവാൻ ഇച്ഛിക്കുമ്പോൾ ആകാശമുണ്ടായി. ശബ്ദപ്രസരണമാണ്‌ ആകാശത്തിന്റെ സഹജസ്വഭാവം. ആ പുരുഷന്‌ സ്പർശനം അനുഭവിക്കാൻ ഇച്ഛയുണ്ടായപ്പോള്‍ വായു ഉണ്ടായി. ദൃഷ്ടിഗോചരമല്ലാത്തത്ര സൂക്ഷ്മമാണത്. കാഴ്ച്ച ഇച്ഛിച്ച് വിശ്വപുരുഷൻ അഗ്നിയെ ഉണ്ടാക്കി. ഈ അഗ്നിയാണ്‌ എണ്ണമറ്റ പ്രകാശസ്രോതസ്സുകളായത്. സ്വാദനുഭവിക്കാനിച്ഛിക്കയാൽ വിശ്വപുരുഷൻ ജലമുണ്ടാക്കി. അത് അഗ്നിക്കു പ്രതിവിധിയുമായി. ഘ്രാണാനുഭവസിദ്ധിക്കായി ഇച്ഛിക്കവേ ഭൂമിയും തൽസ്വഭാവമായ ഗന്ധവും ഉണ്ടായി. ഇത്ര വൈവിദ്ധ്യമായ വിഭൂതികളുണ്ടെങ്കിലും ഈ വിശ്വപുരുഷൻ അതീവ സൂക്ഷ്മവും അവിച്ഛിന്നവുമാണ്‌..

ഈ സത്യത്തെ സ്വയം മറന്ന് വിശ്വപുരുഷൻ ആകാശത്ത് അനന്തമായ സ്ഫുലിംഗങ്ങളായി മാറി. അത് അനന്തമായ സ്ഫുലിംഗങ്ങളിൽ ഓരോന്നുമായി താദാത്മ്യം പ്രാപിക്കുമ്പോൾ അഹംകാരമായി. ഈ അഹംകാരത്തിൽ സഹജമായ ബുദ്ധിയും പ്രജ്ഞയുമുണ്ട്. ആയതുകൊണ്ട് പഞ്ചേന്ദ്രിയങ്ങളുടെ സഹായത്തോടെ സ്വയം അതൊരു ദേഹത്തെ സങ്കൽപ്പിച്ചുണ്ടാക്കുന്നു. ഇതു നാം നേരത്തെ ചർച്ച ചെയ്തതാണല്ലോ. സ്തൂലമായ ഈ ശരീരം  - ഭൌതീക വസ്തു- അങ്ങിനെ സംജാതമായി. ഈ വിശ്വപുരുഷനത്രേ ബ്രഹ്മാവ്. അദ്ദേഹം എണ്ണമറ്റ സൃഷ്ടികൾക്ക് കാരണമാവുന്നു എന്നു തോന്നുന്നു. അവരുടെ സംരക്ഷകനും ബ്രഹ്മാവത്രേ.

അദ്ദേഹം ആദ്യം അവതരിച്ചത് അനന്താവബോധത്തിലാണ്‌. എന്നാൽ അത് സ്വയം പരിമിതഭാവമുൾക്കൊണ്ട് അനന്തതയെ മറന്ന് ഭ്രൂണാവസ്ഥയിലെന്നപോലെ കഴിഞ്ഞ്, പ്രാണശക്തിയുടെ ഉത്തേജനത്താൽ, പദാർത്ഥങ്ങളുടെ ഒരു നിർമ്മിതിയായി ദേഹാഭിമാനമാർജ്ജിക്കുന്നു. എന്നാൽ അദ്ദേഹം സ്വയം തന്റെ ഉദ്ഭവത്തെപ്പറ്റി അന്വേഷണം തുടങ്ങുമ്പോൾ സ്വരൂപത്തെപ്പറ്റി ബോധമുണരുന്നു. അങ്ങിനെ സ്വയമുണ്ടാക്കിയ പരിമിതികളിൽനിന്നും മുക്തനാവുന്നു. 
