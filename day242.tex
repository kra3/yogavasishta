\section{ദിവസം 242}

\slokam{
ആ ഇദാനീം സ്മൃതം സത്യമേതത്തദഖിലം മയാ\\
നിര്‍വികല്‍പ്പചിദാഭാസ ഏഷ ആത്മാസ്മി സര്‍വഗ: (5/34/19)\\
}

പ്രഹ്ലാദന്റെ ചിന്തകള്‍ കൂടുതല്‍ ആഴത്തില്‍ ഇങ്ങിനെ തുടര്‍ന്നു: ഞാന്‍ സര്‍വ്വവ്യാപിയായ സത്യവസ്തുവാണ്. അതില്‍ വിഷയങ്ങളോ തല്‍സംബന്ധിയായ ധാരണകളോ സങ്കല്‍പ്പങ്ങളോ ഇല്ല. ഞാന്‍ ശുദ്ധബോധം മാത്രമാണ്. ഈ ബോധത്തിന്റെ പ്രാഭവത്താലാണ് ചെറിയൊരു മണ്‍കുടം മുതല്‍ സദാ പ്രോജ്ജ്വലിക്കുന്ന സൂര്യനെ വരെ നമുക്കറിയാനാവുന്നത്. “ആഹാ. ഇപ്പോള്‍ ഞാനറിയുന്നു. ഞാന്‍ സര്‍വ്വവ്യാപിയായ, ധാരണാ വികല്‍പ്പങ്ങളൊന്നുമില്ലാത്ത ആത്മാവാണ്.”
 
ആത്മാവാണ് എല്ലാ ഇന്ദ്രിയങ്ങളുടേയും ഉള്‍വെളിച്ചമായി അവയുണ്ടാക്കുന്ന അനുഭവങ്ങളെ വേദ്യമാക്കുന്നത്. വസ്തുക്കള്‍ക്ക് സ്ഥാവരസാധുതയുണ്ടെന്ന തോന്നലുണ്ടാക്കുന്നതും ആത്മാവിന്റെ ആ വെളിച്ചമാണ്. ഉപാധികളേതുമില്ലാത്ത ആ ബോധപ്രകാശത്തിന്റെ മഹിമകൊണ്ടാണ് സൂര്യന് ചൂടും ചന്ദ്രന് തണുപ്പും പര്‍വ്വതങ്ങള്‍ക്ക് ഖനവും ജലത്തിന് ദ്രവരൂപവും  ഉണ്ടായത്. മൂര്‍ത്തീകൃതമായ, പ്രത്യക്ഷമായ എല്ലാറ്റിന്റെയും കാരണം അതാണ്‌.. എന്നാല്‍ ആത്മാവ് സ്വയം കാരണരഹിതമത്രേ. ആത്മാവിന്റെ, ബോധപ്രകാശം കാരണമായാണ് വൈവിദ്ധ്യമാര്‍ന്ന എല്ലാ പ്രപഞ്ചവസ്തുക്കള്‍ക്കും അതാതിന്റെ  സ്വഭാവസവിശേഷതകള്‍ ഉണ്ടായത്. സ്വയം രൂപരഹിതമെങ്കിലും എല്ലാ രൂപങ്ങള്‍ക്കും നിദാനമായത് ആത്മാവത്രേ. എല്ലാ വൈവിദ്ധ്യങ്ങളോടും കൂടി ഈ പ്രപഞ്ചം ഉദ്ഗമിച്ചത് അതില്‍നിന്നാണ്. സൃഷ്ടി-സ്ഥിതി-സംഹാരകാരകരായ  ബ്രഹ്മാ-വിഷ്ണു-രുദ്ര ത്രിമൂര്‍ത്തികള്‍ ആത്മാവില്‍ നിന്നുണ്ടായി. എന്നാല്‍ ആത്മാവിനു കാരണമായി യാതൊന്നുമില്ല. സ്വയംപ്രഭമായ ആത്മാവിനെ ഞാന്‍ നമസ്കരിക്കുന്നു. അത് അറിവ്-അറിവാളി; വിഷയം-വിഷയി തുടങ്ങിയ എല്ലാ ദ്വന്ദങ്ങള്‍ക്കും അതീതമാണ്.

പ്രപഞ്ചത്തിലെ എല്ലാറ്റിലും ആത്മാവ് നിറഞ്ഞു വിളങ്ങുന്നു. അത് പ്രപഞ്ചവസ്തുക്കളില്‍ നിവസിക്കുന്നു. അത് എന്തെല്ലാം നിനയ്ക്കുന്നുവോ അപ്രകാരം ബാഹ്യമായി സംഭവിക്കുന്നു. അത് പ്രപഞ്ചവസ്തുക്കളെ ഇഛാമാത്രം കൊണ്ട് സൃഷ്ടിക്കുകയും ഇഛകൊണ്ട് തന്നെ സംഹരിക്കുകയും ചെയ്യുന്നു. അന്തമില്ലാത്ത വസ്തുക്കള്‍ പ്രത്യക്ഷമാവുന്നത് അനന്തമായ ബോധമണ്ഡലത്തിലാണ്. സൂര്യപ്രകാശത്തില്‍ നിഴലെന്നപോലെ അവ ഉണ്ടായി വലുതായി ചുരുങ്ങി അപ്രത്യക്ഷമാവുന്നതായി നാം കാണുന്നു. എന്നാല്‍ ഈ വെളിച്ചം, ആത്മാവ്, ഗോചരമല്ല. അതിനെ അറിയാനും സാദ്ധ്യമല്ല.  ഹൃദയസംശുദ്ധി കൈവന്നവര്‍ക്ക് മാത്രം എത്തിച്ചേരാവുന്ന ഒരു തലമത്രേ അത്. 
    
എന്നാല്‍ മഹത്പുരുഷന്മാര്‍ അതിനെ അതീവ നിര്‍മ്മലമായ അനന്തവിശ്വാവബോധമായി അറിയുന്നു. ഈ ആത്മാവ് മൂന്നു ലോകങ്ങളിലും അവിച്ഛിന്നമായി, പുല്‍ക്കൊടിമുതല്‍ ബ്രഹ്മാവ്‌ വരെ അനന്തഭാസുരപ്രഭയായി സ്വയം നിറഞ്ഞു വിളങ്ങുന്നു.  അതിന് ആദിയും അന്തവുമില്ല. അതെല്ലാത്തിലും നിറഞ്ഞിരിക്കുന്നു. ജീവ-നിര്‍ജീവ ജാലങ്ങളിലെല്ലാം ആന്തരാനുഭാവസാന്നിദ്ധ്യമായി അത് നിലകൊള്ളുന്നു. 
