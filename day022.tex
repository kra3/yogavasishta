\newpage
\section{ദിവസം 022}

\slokam{
യശ: പ്രഭൃതിനാ യസ്മൈ ഹേതുനൈവ വിനാ  പുന:\\
ഭുവി ഭോഗാ ന രോചന്തേ സ ജീവന്മുക്ത ഉച്യതേ (2/2/8)\\
}

സഭയില്‍ക്കൂടിയിരിക്കുന്ന മഹര്‍ഷിമാരോടായി വിശ്വാമിത്രന്‍ ഇങ്ങിനെ പറഞ്ഞു: ശുകനേപ്പോലെ രാമനും പരമജ്ഞാനം നേടിയിരിക്കുന്നു. ഏറ്റവും സൂക്ഷ്മങ്ങളായ വാസനകള്‍ പോലും അസ്തമിച്ചുകഴിഞ്ഞതിനാല്‍ ആത്മജ്ഞാനം വന്നവന്‌ ലൌകീകകാര്യങ്ങളില്‍ വിരക്തി സഹജമത്രേ. ഈ താത്പ്പര്യമില്ലായ്മ തന്നെ വിജ്ഞാനലക്ഷണമാണ്‌. വാസനകള്‍ ശക്തമാവുമ്പോള്‍ ബന്ധനവും, അവ ഇല്ലാതാവുമ്പോള്‍ മുക്തിയും ഉണ്ടാവുന്നു. "പ്രശസ്തിയോ മറ്റു പ്രലോഭനങ്ങളോ ആത്മ ജ്ഞാനം വന്നു മുക്തനായ ഒരു മുനിയ്ക്ക്‌ പ്രചോദനമേകുന്നില്ല. ഇന്ദ്രിയസുഖങ്ങള്‍ അയാളെ ആകര്‍ഷിക്കുന്നുമില്ല."

രാമനുവേണ്ട ഉപദേശങ്ങള്‍ നല്‍കാന്‍ ഞാന്‍ മഹര്‍ഷി വസിഷ്ടനോട്‌ അപേക്ഷിക്കുന്നു. കാരണം അതു ഞങ്ങള്‍ ക്കും പ്രചോദനപ്രദമാണ്‌. ഈ പാഠങ്ങള്‍ പരമോന്നതവിജ്ഞാനം നിറഞ്ഞതും വേദഗ്രന്ഥങ്ങളില്‍ ഏറ്റവും മികച്ചതുമാവും, നിശ്ചയം. കാരണം പ്രബുദ്ധനായ ഒരു മഹര്‍ഷിവര്യന്‍ അതീവയോഗ്യനും അനാസക്തനുമായ ഒരു ശിഷ്യനുവേണ്ടിയാണല്ലോ ഈ വിദ്യ ഉപദേശിക്കുന്നത്‌. 

വസിഷ്ടന്‍ പറഞ്ഞു: അങ്ങയുടെ ആവശ്യം ഞാന്‍ ശിരസാവഹിക്കുന്നു. രാമ: സൃഷ്ടികര്‍ത്താവായ ബ്രഹ്മാവിന്റെ മുഖദാവില്‍ നിന്നും എനിക്കരുളിക്കിട്ടിയ ആ അറിവ്‌ ഞാന്‍ അങ്ങേയ്ക്കു പകര്‍ ന്നു തരാം.

രാമന്‍ പറഞ്ഞു: ആദ്യം തന്നെ ഒരുകാര്യം പറഞ്ഞു തരിക. എന്തുകൊണ്ടാണ്‌ വേദവ്യാസനെ മുക്തനായി കണക്കാക്കാതെ മകന്‍ ശുകമുനിയെ മുക്തനായി കരുതുന്നത്‌? 

വസിഷ്ടന്‍ പറഞ്ഞു: രാമ: എണ്ണമൊടുങ്ങാത്ത ബ്രഹ്മാണ്ഡങ്ങള്‍ ഉണ്ടായി നിലനിന്നു നശിച്ചുപോയിരിക്കുന്നു. ഇപ്പോള്‍ നിലനില്‍ക്കുന്ന ബ്രഹ്മാണ്ഡങ്ങളെപ്പറ്റി മനസ്സിലാക്കുക എന്നതുപോലും അസാധ്യം. വായുവില്‍ കെട്ടിയുണ്ടാക്കുന്ന കോട്ടകള്‍പോലെ ഈ ജഗത്തും ആഗ്രഹങ്ങളുടെ, ഹൃദയത്തിലുയരുന്ന ഭാവനയുടെ, സൃഷ്ടിയാണെന്ന് പെട്ടെന്നു തന്നെ നമുക്ക്‌ ബോധ്യമാവും. ജീവജാലങ്ങള്‍ അവരുടെ ഹൃദയത്തില്‍ ഈ ലോകത്തെ ആവാഹനം ചെയ്യുന്നു. ജീവനോടെയിരിക്കുമ്പോള്‍ ഈ മിഥ്യാഭാവന ശക്തിയാര്‍ജ്ജിക്കുന്നു. മരണശേഷം അവന്‍ ഇതിനുമപ്പുറത്തുള്ള ലോകത്തെ ആവാഹനംചെയ്ത്‌ അനുഭവിക്കുന്നു. അങ്ങിനെ വാഴപ്പോളപോലെ ഒന്നിനുള്ളില്‍ ഒന്നൊന്നായി ലോകങ്ങള്‍ ഉയിര്‍ക്കൊള്ളുകയാണ്‌. വസ്തുപ്രപഞ്ചമോ സൃഷ്ടിപ്രക്രിയയോ ശരിയായ അര്‍ത്ഥത്തില്‍ സത്യമല്ല. എന്നാല്‍ ജീവിച്ചിരിക്കുന്നവര്‍ക്കും മരിച്ചവര്‍ക്കും ഇവ യാഥാര്‍ഥ്യമാണെന്നു തോന്നുകയാണ്‌. ഈ അജ്ഞതയുള്ളിടത്തോളം ദൃശ്യപ്രപഞ്ചമുണ്ടാവും.

രാമ: ഈ സംസാരസാഗരത്തില്‍ ജീവികള്‍ അവിടവിടെയായി സാമ്യരൂപത്തിലും വിഭിന്നരൂപത്തിലും പിറവിയെടുക്കുന്നു. ഈ വേദവ്യാസന്‍ സൃഷ്ടിധാരയിലെ ഇരുപത്തിമൂന്നാമത്തേതാണ്‌. അദ്ദേഹവും മറ്റ്‌ ഋഷിമാരും വീണ്ടും വീണ്ടും മൂര്‍ത്തരൂപമാര്‍ന്ന് പിന്നെ അമൂര്‍ത്തതയില്‍ ലയിച്ചുകൊണ്ടിരിക്കും. ചില ജന്മങ്ങളില്‍ അവര്‍ മറ്റുള്ളവര്‍ക്കു തുല്യരായും മറ്റുചിലതില്‍ അതുല്യരായുമിരിക്കും. ഈ ജന്മത്തില്‍ വേദവ്യാസന്‍ മുക്തനായ ഒരു ഋഷിവര്യനത്രേ. ഇങ്ങിനെയുള്ള മാമുനിമാര്‍ അനേകം ജന്മങ്ങളെടുത്ത്‌ മറ്റുള്ളവരുമായി ബന്ധങ്ങളില്‍ ഏര്‍പ്പെടുന്നു. ചിലപ്പോള്‍ അവര്‍ പഠിപ്പിലും അറിവിലും മറ്റുള്ളവര്‍ക്കൊപ്പമായിരിക്കും. മറ്റുചിലപ്പോള്‍ സ്വഭാവത്തിലും പഠിപ്പിലും അവര്‍ ഇതര മുനിമാര്‍ക്ക്‌ സമന്മാരാവണമെന്നുമില്ല.
