\newpage
\section{ദിവസം 004}

\begin{center}
കോപം വിഷാദകലനാം വിതതം ച ഹർഷം\\
നാൽ പേന കാരണവശേന വഹന്തി സന്ത:\\
സർഗേണ സംഹൃതിജവേന വിനാ ജഗത്യാം \\
ഭൂതാനി ഭൂപ ന മഹാന്തി വികാരവന്തി. (1/5/15)\\
\end{center}

വാല്‍മീകി തുടര്‍ ന്നു: കൊട്ടാരത്തില്‍ മടങ്ങിയെത്തിയ ശ്രീരാമന്‍ അച്ഛനേയും വസിഷ്ഠമുനിയേയും മറ്റു മുതിര്‍ന്നവരേയും സന്യാസികളേയും വന്ദിച്ചു. അയോദ്ധ്യാനഗരി എട്ടുദിവസത്തേയ്ക്ക്‌ ഒരുത്സവമായിതന്നെ രാമന്റെ തീര്‍ത്ഥാടനപൂര്‍ത്തി കൊണ്ടാടി. കുറച്ചുകാലം ശ്രീരാമന്‍ നിത്യകര്‍മ്മങ്ങളുമായി കൊട്ടാരത്തില്‍ കഴിഞ്ഞുകൂടി. എന്നാല്‍ അധികം താമസിയാതെ അദ്ദേഹത്തില്‍ നിഗൂഢമായ കുറേ മാറ്റങ്ങള്‍ കണ്ടു തുടങ്ങി. ശരീരം ക്ഷീണിച്ചുണങ്ങി. അവശതയും വിളര്‍ച്ചയും മുഖത്തുകാണപ്പെട്ടു. ദശരഥമഹാരാജാവ്‌ മകന്റെ അപ്രതീക്ഷിതമായ ഈ മാറ്റത്തിലും സ്വഭാവത്തിലും ആശങ്കാകുലനായി. ശാരീരികമായി എന്തുപറ്റിയെന്ന അച്ഛന്റെ ചോദ്യത്തിന്‌ 'ഒരു കുഴപ്പവുമില്ല' എന്ന മറുപടിയാണ്‌ രാമന്‍ നല്‍കിയത്‌. ദശരഥന്‍ ചോദിച്ചു: "മകനേ നിന്നെ അലട്ടുന്ന കാര്യം എന്താണ്‌?" "ഒന്നുമില്ല അച്ഛാ", രാമന്‍ പറഞ്ഞു. 


സ്വാഭാവികമായും രാജാവ്‌ വസിഷ്ഠമുനിയുടെ ഉപദേശം തേടി. വസിഷ്ഠന്‍ പറഞ്ഞു: "തീര്‍ച്ചയായും രാമന്റെ ഈ സ്വഭാവ മാറ്റത്തിനു ചില കാരണങ്ങള്‍ ഉണ്ട്‌. ഈ ലോകത്ത്‌ ഏതൊരു മഹത്‌കാര്യങ്ങളും നടക്കുന്നതിനു മുന്‍പ്‌ അതിനു ഹേതുവായ കാരണങ്ങള്‍ ഉണ്ടാവണം - പ്രപഞ്ചത്തിന്റെ ആധാരമായാലും, മഹാന്മാരുടെയുള്ളില്‍ ദു:ഖം, വിഷാദം, ആഹ്ലാദം തുടങ്ങിയ വികാരങ്ങള്‍ ഉണ്ടാവുന്നതായാലും ഉചിതമായ കാരണങ്ങള്‍ കൂടാതെ ഒന്നും സംഭവിക്കുകയില്ല."

ദശരഥന്‍ വീണ്ടും അതേപറ്റി അന്വേഷിക്കാന്‍ തുനിഞ്ഞില്ല. എന്നാല്‍ അധികം താമസിയാതെ ലോകപ്രശസ്തനായ വിശ്വാമിത്രമഹാമുനി കൊട്ടാരത്തില്‍ വന്നു ചേര്‍ന്നു. രാജാവ്‌ മഹര്‍ഷിയെ സ്വീകരിക്കാന്‍ ഓടിച്ചെന്നു."മഹര്‍ഷേ അങ്ങേയ്ക്ക്‌ സുസ്വാഗതം. ഈ ഗൃഹത്തില്‍ അങ്ങയുടെ എഴുന്നള്ളത്ത്‌ എനിയ്ക്കെത്ര സന്തോഷപ്രദമാണെന്നോ?. അന്ധനു കാഴ്ച്ച പോലെയും വരണ്ട ഭൂമിയിലെ മഴപോലെയും അനപത്യയ്ക്കു പുത്രലാഭം പോലെയും, മരിച്ചവന്റെ ഉയിര്‍ത്തെഴുന്നേല്‍പ്പു പോലെയും, നഷ്ടപ്പെട്ട നിധി തിരിച്ചു കിട്ടുമ്പോലെയുമാണ്‌ അങ്ങയുടെ ആഗമനം. മഹാമുനേ ഞാന്‍ അങ്ങേയ്ക്കുവേണ്ടി എന്താണു ചെയ്യേണ്ടത്‌? അങ്ങേയ്ക്ക്‌ എന്താവശ്യമുണ്ടെങ്കിലും ആയത്‌ നടത്തിക്കഴിഞ്ഞു എന്നു തന്നെ കരുതുക. അങ്ങെന്റെ ആരാധാനാമൂര്‍ത്തിയാണ്‌. അവിടുത്തെ ആജ്ജ്ഞയാണ്‌ എനിക്കു ഹിതം. ദയവായി പറഞ്ഞാലും."
