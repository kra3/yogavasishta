\section{ദിവസം 197}

\slokam{
മോഹ എവം മയോ മിഥ്യാ ജഗത: സ്ഥിരതാം ഗത:\\
സങ്കല്പനേന മനസാ കല്പിതോചിരത: സ്വയം (4/59/31)\\
}

വസിഷ്ഠൻ തുടർന്നു: രാമാ, ഈ ജീവിതത്തിൽ ആഹാരം നീഹാരം, മൈഥുനം ഇവയല്ലാതെ മറ്റെന്തുണ്ട്? അതുകൊണ്ടുതന്നെ ജ്ഞാനിക്ക് ഇവിടെനിന്നു ലഭിക്കാനെന്തുണ്ട്? അയാൾക്കിവിടെ ആരായാനനുയോഗ്യമായ എന്താണുള്ളത്? പഞ്ചഭൂത നിർമ്മിതമായ ഈ ലോകവും രക്ത-മാംസ-രോമാദികളാൽ നിർമ്മിക്കപ്പെട്ട ദേഹവും അജ്ഞാനിയെ സംബന്ധിച്ചിടത്തോളം അവന്റെ ഉല്ലാസത്തിനായാണ്‌ നിലനിൽക്കുന്നത്. ജ്ഞാനി അസ്ഥിരവും അയാഥാർത്ഥ്യവുമായ ഇതിലെല്ലാം ഒരു കൊടിയ വിഷമാണു ദർശിക്കുന്നത്.

രാമൻ ചോദിച്ചു: എല്ലാ ധാരണകളും ഇല്ലാതായി മനസ്സ് സൃഷ്ടാവിന്റെ തലത്തിൽ നിൽക്കുമ്പോൾപ്പിന്നെ വീണ്ടും ലോകമെന്ന ആശയം എങ്ങിനെയാണുദിച്ചു പൊങ്ങുന്നത്?

വസിഷ്ഠൻ തുടർന്നു: രാമാ, ആദ്യമുണ്ടായ സൃഷ്ടാവ് അനന്താവബോധത്തിന്റെ ഗർഭത്തിൽനിന്നും പുറത്തുവന്ന ഉടനേ ‘ബ്രഹ്മാവ്’ എന്നു ശബ്ദിച്ചു. അങ്ങിനെ അയാൾ ബ്രഹ്മാവ് എന്ന സൃഷ്ടികർത്താവായി അറിയപ്പെട്ടു. ബ്രഹ്മാവിൽ വെളിച്ചമെന്ന ധാരണ ഉദിച്ചപ്പോൾ വെളിച്ചമുണ്ടായി. ആ വെളിച്ചത്തിൽ അദ്ദേഹം തന്റെ ദേഹത്തെ മനസ്സിൽ കണ്ടതും അതപ്രകാരം ഉണ്ടായി. പ്രോജ്ജ്വലിക്കുന്ന സൂര്യൻ മുതൽ ആകാശം നിറഞ്ഞ എല്ലാ വൈവിദ്ധ്യമാർന്ന വസ്തുക്കളും ഉണ്ടായി. ആ പ്രകാശത്തെ അനന്തമായ കിരണങ്ങളുള്ളതായി സങ്കൽപ്പിച്ചു ധ്യാനിച്ചപ്പോൾ അവ വൈവിദ്ധ്യമാർന്ന ജീവജാലങ്ങളായി. വിശ്വമനസ്സാണ്‌ ബ്രഹ്മാവായും മറ്റു ജീവജാലങ്ങളായും മാറിയത്. ഈ ബ്രഹ്മാവു സൃഷ്ടിച്ച എല്ലാമിപ്പോഴുമുണ്ട്. “ഈ അയാഥാർത്ഥ്യമായ ലോകത്തിനു അസ്തിത്വഭാവം നൽകുന്നത് തുടർച്ചയായുള്ള നിലനിൽപ്പെന്ന ധാരണ ഉള്ളതുകൊണ്ടാണ്‌.. ”

എല്ലാ ജീവജാലങ്ങളും ഈ വിശ്വത്തിൽ നിലകൊള്ളുന്നത് അവയുടെ സ്വന്തം ധാരണകളാലും ആശയങ്ങളാലുമാണ്‌.. സ്വന്തം ചിന്താശക്തികൊണ്ട് വിശ്വനിർമ്മിതി ചെയ്തശേഷം ബ്രഹ്മാവ് ഇങ്ങിനെ ആലോചിച്ചു: ‘പ്രപഞ്ചമനസ്സിലുണ്ടായ ചെറിയൊരിളക്കം, കാലുഷ്യം ഹേതുവായാണ്‌ ഞാനീ സൃഷ്ടികളെയെല്ലാമുണ്ടാക്കിയത്. എനിക്കു മതിയായി. കാരണം ഇവ സ്വയം പ്രത്യുൽപ്പാദനം നടത്തി പെരുകിക്കൊള്ളൂം. ഞാനിനി വിശ്രമിക്കട്ടെ.’ ഇങ്ങിനെ ധ്യാനിച്ച് ബ്രഹ്മാവ് തീവ്രമായ ധ്യാനാവസ്ഥയിൽ വിലീനനായി വിശ്രമിച്ചു. എന്നാൽ തന്റെ സൃഷ്ടികളോട് ദയാവായ്പ്പുതോന്നി അദ്ദേഹം ആത്മവിദ്യയെ പ്രകാശിപ്പിക്കാനായി വേദശാസ്ത്രങ്ങളെ വെളിപ്പെടുത്തി. എന്നിട്ടു വീണ്ടും അദ്ദേഹം തന്റെ യോഗനിദ്രയിലേയ്ക്കു തിരിച്ചുപോയി. ബ്രഹ്മാവ്‌ നില കൊള്ളുന്ന ഈ ബ്രാഹ്മിസ്ഥിതി എന്നത്, എല്ലാ ധാരണകൾക്കും ആശയങ്ങൾക്കുമതീതമായ ഒരു തലമത്രേ. അന്നുമുതൽ സൃഷ്ടിക്കപ്പെട്ട എല്ലാത്തിനും അതുമായി ബന്ധപ്പെടുന്നവയുടെ സവിശേഷസ്വഭാവങ്ങൾ ഉണ്ടായി. നന്മയുമായി ചേർന്നപ്പോള്‍ അവ നന്മയുള്ളവയായി. ലൗകീകമായവയോടു ചേർന്നവ അങ്ങിനെയായി. അങ്ങിനെ ഇഹലോകത്തിൽ ബന്ധിക്കപ്പെടുവാനും അതിൽ നിന്നു സ്വതന്ത്രനാവാനും ഒരു ജീവനു സാധിക്കും. 
