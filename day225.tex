\section{ദിവസം 225}

\slokam{
കിം പുത്ര ഘനതാം ശോകം നയസ്യാന്ധ്യൈകകാരണം\\
ബാഷ്പധാരാധരം ഘോരം പ്രവൃട്കാല ഇവാംബുജം (5/19/26)\\
}

വസിഷ്ഠൻ തുടർന്നു: രാമാ, ഇതോടനുബന്ധിച്ച് പുരാതനമായ ഒരു കഥയുണ്ട്. ജംബുദ്വീപ് എന്ന ഭൂഖണ്ഡത്തിൽ മഹേന്ദ്രം എന്നുപേരായ ഒരു മഹാപർവ്വതം. അതിന്റെ താഴ്വാരങ്ങളിലുള്ള വനപ്രദേശങ്ങളിൽ അനേകം മാമുനിമാർ തപസ്സനുഷ്ഠിച്ചു വസിച്ചുവന്നു. അവരാ പർവ്വതപ്രദേശത്തേയ്ക്ക് അവരുടെ സ്നാനത്തിനും നിത്യോപയോഗത്തിനായും ആകാശഗംഗയെപ്പോലും കൊണ്ടുവന്നിരുന്നു. ആകാശഗംഗയുടെ തീരത്ത് ദീർഘതപൻ എന്നുപേരായ ഒരു മഹർഷി വസിച്ചിരുന്നു. അദ്ദേഹം, തന്റെ നാമം സൂചിപ്പിക്കുന്നതുപോലെ നിരന്തരം തപസ്സനുഷ്ഠിച്ചിരുന്നു. അദ്ദേഹത്തിന്‌ രണ്ടു പുത്രന്മാർ- പുണ്യനും പവനനും.

ഇവരിൽ പുണ്യൻ പൂർണ്ണപ്രബുദ്ധനായിരുന്നു. എന്നാൽ പവനന്‌ അജ്ഞാനനിവൃത്തി വന്നിരുന്നുവെങ്കിലും അയാള്‍ പ്രബുദ്ധത പ്രാപിച്ചിട്ടില്ലായിരുന്നു. അർദ്ധജ്ഞാനിയായിരുന്നു അദ്ദേഹം. അദൃശ്യവും ദുർഗ്രാഹ്യവുമായ സമയത്തിന്റെ ഒഴുക്കിൽ ദീർഘതപമുനി കാലയവനികയ്ക്കുള്ളിൽ മറഞ്ഞു. ശരീരമുപേക്ഷിക്കുമ്പോഴേയ്ക്ക് അദ്ദേഹം എല്ലാവിധ ആസക്തികളിൽ നിന്നും മുക്തനായിരുന്നു. ഒരു പക്ഷി തന്റെ കൂടുവിട്ടു പറന്നുപോകുമ്പോലെ അനായാസമായി മഹർഷി ദേഹമുപേക്ഷിച്ച് പരമപദം പൂകി. ഭർത്താവിൽ നിന്നും യോഗമുറകൾ പഠിച്ച മുനി പത്നിയും അദ്ദേഹത്തെ പിന്തുടർന്നു.

മാതാപിതാക്കളുടെ പെട്ടെന്നുള്ള വിയോഗം പവനനെ വല്ലാതെ ദു:ഖിതനാക്കി. അയാൾ ഉറക്കെ വാവിട്ടു നിലവിളിച്ചു. പുണ്യനാകട്ടെ മാതാപിതാക്കൾക്കായുള്ള മരണാനന്തര കർമ്മങ്ങൾ പ്രശാന്തതയോടെ അനുഷ്ഠിച്ചു. എങ്കിലും മരണത്തോടെയുണ്ടായ ഈ വേർപാടിനെക്കുറിച്ച് അദ്ദേഹത്തിനു വ്യാകുലതയൊന്നും ഉണ്ടായിരുന്നില്ല. അയാൾ തന്റെ സഹോദരനായ പവനന്റെ അടുക്കൽ ചെന്ന് ഇങ്ങിനെ സമാധാനിപ്പിച്ചു.

പുണ്യൻ പറഞ്ഞു: അനിയാ, നീയെന്തിനാണീ കൊടിയ ദു:ഖത്തെ ക്ഷണിച്ചുവരുത്തി മാഴ്കുന്നത്? അജ്ഞാനത്തിന്റെ അന്ധകാരം ഒന്നുമാത്രമാണ്‌ നിന്റെ കണ്ണിൽ നിന്നും ധാരധാരയായി ഒഴുകുന്ന കണ്ണീരിനു കാരണം. “നമ്മുടെ അച്ഛൻ അമ്മയോടൊപ്പം എന്നെന്നേയ്ക്കുമായി പോയത് ഏറ്റവും ഉയർന്ന ഒരു മുക്തിതലത്തിലേയ്ക്കാണ്‌.. അതെല്ലാ ജീവജാലങ്ങൾക്കും സഹജമായി എത്തിച്ചേരേണ്ട ഒരിടമാണ്‌.. ജീവിച്ചിരിക്കുമ്പോൾത്തന്നെ ആത്മജ്ഞാനികളുടെ സഹജാവസ്ഥയും അതു തന്നെ. അവർ അവരുടെ സ്വരൂപസവിധത്തിലേയ്ക്കു മടങ്ങിയതിൽ നിനക്കു ദു:ഖിക്കേണ്ട കാര്യമെന്ത്?  അറിവില്ലായ്മകൊണ്ട്, നീ അച്ഛൻ, അമ്മ, തുടങ്ങിയ ബന്ധങ്ങളുടെ ബന്ധനത്തിൽപ്പെട്ടു കേഴുന്നു. അതേ ബന്ധനങ്ങളാകുന്ന അജ്ഞാനത്തിൽ നിന്നു മുക്തിനേടിയവരെക്കുറിച്ചാണു നീ ദു:ഖിക്കുന്നത്. അവർ നിന്റെ അച്ഛനമ്മമാരോ നീ അവരുടെ മകനോ ആയിരുന്നില്ല. അനേകമനേകം ജന്മങ്ങളിലായി നിനക്ക് എണ്ണിയാലൊടുങ്ങാത്തത്ര അച്ഛനമ്മമാർ ഉണ്ടായിരുന്നു. അവർക്കും അനേകം മക്കളുമുണ്ടായിരുന്നു.  നീ നിന്റെ ഈ മാതാപിതാക്കൾക്കായി ദു:ഖിക്കുന്നുവെങ്കിൽ എണ്ണിയാലൊടുങ്ങാത്ത ആ 'സഹോദരങ്ങള്‍ക്കായും' വിലപിക്കാത്തതെന്തേ?

പാവനനായ പവനാ, നീ ലോകമായി കാണുന്നത് വെറുമൊരു മായക്കാഴ്ച്ച മാത്രമാണ്‌.. വാസ്തവത്തിൽ സുഹൃത്തുക്കളും ബന്ധുക്കളുമൊന്നും ഇല്ല. അതുകൊണ്ടു തന്നെ മരണവും വേർപെടലുമെല്ലാം വെറും മിഥ്യ. നീ കാണുന്ന മഹത്തായ ഐശ്വര്യസമ്പത്തുക്കളെല്ലാം വെറും ജാലവിദ്യകളാണ്‌.. അവയിൽ ചിലത് മൂന്നു ദിവസം, ചിലത് അഞ്ചു ദിവസം നിലനിൽക്കുന്നു!. അത്രയേയുള്ളൂ. നിന്റെ കുശാഗ്രബുദ്ധികൊണ്ട് സത്യത്തെ അന്വേഷിച്ചറിയൂ. ‘ഞാൻ’, ‘നീ’ തുടങ്ങിയ എല്ലാ ധാരണകളേയും ഉപേക്ഷിക്കൂ. ‘അദ്ദേഹം മരിച്ചു’, ‘അദ്ദേഹം എന്നെ വേർപെട്ടുപോയി’ എന്നുള്ള തോന്നലും ദൂരെക്കളയൂ. കാരണം ഇവയെല്ലാം നിന്റെ തോന്നലുകളാണ്‌.. അവ സത്യമല്ല. 
