\section{ദിവസം 218}

\slokam{
ന പശ്യത്യേവ യോഽത്യർഥം തസ്യ ക: ഖലു ദുർമതി:\\
വിചിത്രമഞ്ജരീ ചിത്രം സംദർശയതി കാനനം (5/14/3)\\
}

വസിഷ്ഠൻ തുടർന്നു: ഈ മായികലോകത്തെ സത്യമെന്നുധരിച്ച് അതിൽ വിശ്വാസമുറപ്പിച്ച് സുഖാനുഭവങ്ങൾക്കായി പരിശ്രമിക്കുന്നവർക്കായല്ല ഞാനീ സംഗതികൾ പറഞ്ഞു തരുന്നത്. “കാണാൻ വിസമ്മതിക്കുന്നവനുവേണ്ടി ഏതൊരു മന്ദബുദ്ധിയാണ്‌ വർണ്ണവൈവിദ്ധ്യമാർന്ന ഒരു വനപ്രദേശം കാട്ടിക്കൊടുക്കാൻ മെനക്കെടുക?” ഏതൊരു വിഡ്ഢിയാണ്‌ മൂക്കു മുഴുവൻ കുഷ്ടം പിടിച്ചറ്റുപോയവനെ സുഗന്ധദ്രവ്യങ്ങളെ മണത്തറിഞ്ഞു വേർതിരിക്കുന്ന സൂക്ഷ്മമായ കല അഭ്യസിപ്പിക്കാൻ ശ്രമിക്കുക? ലഹരിക്കടിമയായ ഒരുവനെ ആരാണ്‌ അതിഭൗതീകതയും തത്വശാസ്ത്രവും പഠിപ്പിക്കുക? ഗ്രാമത്തിലെ കാര്യങ്ങളെപ്പറ്റി ശ്മശാനത്തിൽക്കിടക്കുന്ന ശവത്തോട് ആരാണു തിരക്കുക? അഥവാ അങ്ങിനെ ആരെങ്കിലും ചെയ്താൽത്തന്നെ ആ വിഡ്ഢിത്തത്തെ ആരാണു തടയാൻ ശ്രമിക്കുക? അതുപോലെ അന്ധവും മൂകവുമായ മനസ്സിനെ നിയന്ത്രിക്കാൻ കഴിയാത്തവനെ ആരാണ്‌ ആത്മവിദ്യ അഭ്യസിപ്പിക്കുക?

വാസ്തവത്തിൽ മനസ്സ് എന്നതൊരു മിഥ്യയാണ്‌..  അതുകൊണ്ടുതന്നെ അതെന്നും കീഴടക്കപ്പെട്ടതുതന്നെയാണ്‌.  എന്നാൽ ഈ മനസ്സെന്ന ‘അവസ്തു’വിനെ വെല്ലാൻ കഴിയാതെ കഷ്ടപ്പെടുന്നവൻ വിഷം കഴിക്കാതെതന്നെ അതിന്റെ ദുരിതമനുഭവിക്കുന്നവനാണ്‌..  ജ്ഞാനി എല്ലായ്പ്പോഴും ആത്മാവിനെ കാണുന്നു. അയാൾ എല്ലാ ചലനങ്ങളേയും പ്രാണന്റെ ഗതിവിഗതികളായി അറിയുന്നു. ഇന്ദ്രിയങ്ങൾ അതാതിന്റെ പ്രവർത്തനങ്ങൾ നിർവ്വഹിക്കുന്നതായും അയാളറിയുന്നു.

അപ്പോൾ ഈ മനസ്സെന്നു പറയുന്നതെന്തിനെയാണ്‌? എല്ലാ ചലനങ്ങളും പ്രാണന്റേതാണ്‌.  എല്ലാ ബോധവും ആത്മാവിന്റേതുമാണ്‌.  ഇന്ദ്രിയങ്ങൾക്ക് അതതിന്റെ ശക്തിയുമുണ്ട്. അപ്പോൾപ്പിന്നെ ഇവയെ എല്ലാം ബന്ധിപ്പിച്ചു നിർത്തുന്നതെന്താണ്‌? ഇതെല്ലാം സർവ്വവ്യാപിയായ ആ അനന്താവബോധത്തിന്റെ ഭാവങ്ങളാണ്‌.  വൈവിദ്ധ്യത, അല്ലെങ്കിൽ ഭിന്നത എന്ന വാക്കിന്‌ യാതൊരു സാംഗത്യവും ഇല്ല തന്നെ. ഈ വിഭിന്നത എന്ന ധാരണ തന്നെ നിന്നിൽ എങ്ങിനെയാണുദയം ചെയ്തത്?

വ്യക്തിഗത ആത്മാവ്, അല്ലെങ്കിൽ ജീവൻ എന്നത് ബുദ്ധിമാന്മാരെപ്പോലും കുഴക്കുന്ന വെറുമൊരു വാക്കല്ലേ? വ്യക്തിബോധം, പരിമിത ബോധം എന്നെല്ലാം പറയുന്നതും അസത്തായ വെറും ഭ്രമകൽപ്പനകൾ മാത്രം. അതിനെന്തുചെയ്യാൻ കഴിയും?

ഏകാത്മകമായ സത്യത്തെ മറയ്ക്കുന്ന മനസ്സെന്ന ഒരു പ്രഹേളികയെ സങ്കൽപ്പിച്ചുണ്ടാക്കി അതിനെക്കൊണ്ടുള്ള ദു;ഖങ്ങളനുഭവിക്കുന്ന അജ്ഞരായവരുടെ നിയതിയോർത്ത് എനിക്കു കഷ്ടം തോന്നുന്നു. ഈ ലോകത്ത് മന്ദബുദ്ധികൾ ജനിക്കുന്നത് ദുരിതാനുഭവങ്ങൾ അനുഭവിച്ചു മരിക്കുവാനായാണ്‌..  ദിനംതോറും ലക്ഷക്കണക്കിനു മൃഗങ്ങൾ കൊല്ലപ്പെടുന്നു. കാറ്റിന്റെ ഊക്കിൽപ്പെട്ട് കോടിക്കണക്കിനു കൊതുകുകൾ ചാവുന്നു. സമുദ്രത്തിലെ ചെറുജീവികളെ വലിയ ജീവികൾ ഭക്ഷിക്കുന്നു. അതിലെല്ലാം ദു:ഖിക്കാനെന്തുണ്ട്?

ബലമേറിയ മൃഗങ്ങൾ ബലം കുറഞ്ഞവയെ കൊന്നു തിന്നുന്നു. ചെറിയൊരെറുമ്പു മുതൽ മഹാദിവ്യന്മാർവരെ എല്ലാവരും ജനനമരണങ്ങൾക്കടിമകളാണ്‌.  ഓരോ നിമിഷവും എണ്ണമറ്റ ജന്തുക്കൾ ജനിക്കുന്നു, മരിക്കുന്നു. ഇതൊന്നും ആളുകൾക്കിഷ്ടപ്പെടുന്നോ ഇല്ലയോ അവർ ദു:ഖിക്കുന്നോ സന്തോഷിക്കുന്നോ എന്നൊന്നും നോക്കിയല്ല സംഭവിക്കുന്നത്. അതുകൊണ്ട് അനിവാര്യമായും സംഭവിക്കുന്ന ഒന്നിനെപ്പറ്റിയും വ്യാകുലപ്പെടാതിരിക്കുകയാണ്‌ ജ്ഞാനികൾ ചെയ്യുക.
