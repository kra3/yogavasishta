\section{ദിവസം 288}

\slokam{
സംസക്തിര്‍ദ്വിവിധാ പ്രോക്താ വന്ദ്യാ വന്ധ്യാ ച രാഘവ\\
വന്ധ്യാ സര്‍വത്ര മൂഢാനാം വന്ദ്യാ തത്വവിദം നിജാ  (5/68/21)\\
}

 രാമന്‍ ചോദിച്ചു: ഭഗവന്‍, എന്താണീ ഉപാധികള്‍ ? അവയെങ്ങിനെയാണ് ബന്ധനങ്ങള്‍ക്ക് കാരണമാവുന്നത്? എന്താണ് മോക്ഷം? അതെങ്ങിനെയാണ് പ്രാപിക്കുക?

വസിഷ്ഠന്‍ തുടര്‍ന്നു: ശരീരവും ആത്മാവും തമ്മില്‍ വിവേചിച്ചറിയാനുള്ള അറിവില്ലാത്തവര്‍പോലും   ശരീരത്തിന്റെ ഉണ്മയെപ്പറ്റിമാത്രം ദൃഢമായി വിശ്വസിക്കുന്ന അവസ്ഥയാണ് സോപാധികാവസ്ഥ. ഈ ചിന്തയിലുറച്ചവന് അനന്തമായ ആത്മാവിനെ പരിമിതപ്പെട്ട ഒന്നായി മാത്രമേ കാണാനൊക്കൂ. അതിനാല്‍ അയാള്‍ സുഖാന്വേഷിയായി സ്വയം ബന്ധിക്കപ്പെടുന്നു.  

എന്നാല്‍ ആരൊരുവന്‍ ‘ഇതെല്ലാം ആത്മാവ് മാത്രം, അതിനാല്‍ എനിക്ക് ആഗ്രഹിക്കാനോ ത്യജിക്കാനോ എന്താണുള്ളത്?’ എന്ന് മനസ്സിലുറപ്പിച്ചിരിക്കുന്നുവോ അവന്‍ ഉപാധിരഹിതമായ മുക്തിയെ പ്രാപിക്കുന്നു. ‘ഞാന്‍'  ഇല്ല, 'ഞാന'ല്ലാതെ മറ്റാരുമില്ല’ എന്നു തെളിഞ്ഞ അറിവുള്ളവന്‍ സുഖം കാംക്ഷിക്കുന്നില്ല. ‘സുഖമുണ്ടാവുകയോ ഇല്ലാതാവുകയോ ചെയ്യട്ടെ’ എന്നാണയാളുടെ ഭാവം. അയാള്‍ കര്‍മ്മരഹിതനല്ല. എന്നാല്‍ കര്‍മ്മഫലങ്ങളില്‍ അയാള്‍ ആകാംഷാഭരിതനുമല്ല. അയാള്‍ക്ക്  അമിതാഹ്ലാദമോ ദു:ഖമോ ഇല്ല. മനസാ തന്റെ കര്‍മ്മഫലങ്ങളെ ഉപേക്ഷിക്കുന്നുവെങ്കിലും അയാള്‍ കര്‍മ്മങ്ങളില്‍നിന്ന് വിട്ട് നില്‍ക്കുന്നില്ല.  ഉപാധികളില്‍ നിന്ന് വിട്ടുനില്‍ക്കുന്നത് മൂലം ബന്ധനങ്ങള്‍ അഴിഞ്ഞൊഴിയുന്നു.  ഉപാധികളാണെല്ലാ എല്ലാ ദു:ഖങ്ങള്‍ക്കും ഹേതു.   

ഇനിപ്പറയുന്ന ഉദാഹരണങ്ങള്‍ കൊണ്ട് ഉപാധികളെപ്പറ്റി കൂടുതല്‍ വ്യക്തമാക്കാം:

ഒരു കഴുത തന്റെ യജമാനന്റെ കയറാല്‍ നയിക്കപ്പെട്ട് ഭയത്തോടെ അമിതഭാരം ചുമക്കുന്നു. ഭൂമിയില്‍ വേരുറപ്പിച്ച മരം ചൂടും, തണുപ്പും, മഴയും കാറ്റുമെല്ലാം സഹിക്കുന്നു. ഒരു പുഴു അതിന്റെ പരിണാമത്തിനു കാലമാവുന്നതും കാത്തു ഭൂമിയിലെ കുഴിയില്‍ കിടക്കുന്നു. വിശന്നു വലഞ്ഞ ഒരു കിളി ഹിംസ്രജന്തുക്കളെപ്പേടിച്ചു മരക്കൊമ്പിലിരിക്കുന്നു.
ഇണക്കമാര്‍ന്ന ഒരു മാന്‍പേട ശാന്തമായി പുല്ലുമേഞ്ഞുനടന്ന് ഒടുവില്‍ വേടന്റെ അമ്പിനിരയാവുന്നു. അനേകമാളുകള്‍ പുഴുക്കളായും കീടങ്ങളായും വീണ്ടും വീണ്ടും ജനിച്ചു മരിക്കുന്നു. സമുദ്രത്തിലെ അലകള്‍പോലെ എണ്ണമറ്റ ജീവജാലങ്ങള്‍ ഉണ്ടായി നശിക്കുന്നു. സ്വന്തമായി നീങ്ങാന്‍പോലും ആവാതെ അവശനായ മനുഷ്യന്‍ വീണ്ടും വീണ്ടും മരിക്കുന്നു. വള്ളിച്ചെടികളും സസ്യവര്‍ഗ്ഗങ്ങളും ഭൂമിയില്‍ നിന്നും പോഷകം വലിച്ചെടുത്ത് അവിടെത്തന്നെ തഴച്ചുവളരുന്നു.
ഈ സംസാരമെന്ന മായക്കാഴ്ച എണ്ണമറ്റ ദുരിതങ്ങളെയും ദു:ഖങ്ങളെയും തേടിപ്പിടിച്ചു കൂട്ടിക്കൊണ്ടുവരുന്ന നദിയെപ്പോലെയാണ്.

ഇതൊക്കെയാണ് ഉപാധികളെപ്പറ്റി എനിക്ക് പറയാനുള്ളത്.

"ഉപാധികള്‍ , (ആസക്തി, ഒട്ടല്‍ , അല്ലെങ്കില്‍ സ്വയംപരിമിതി എന്നിവ) രണ്ടുവിധമാണ്. വന്ദിക്കപ്പെടേണ്ടവയും വന്ധ്യമായവയും.  (അതായത് വ്യര്‍ത്ഥമായവ). മൂഢന്മാരില്‍ പൊതുവെ വന്ധ്യമായ ഉപാധികള്‍   കാണപ്പെടുന്നു. വന്ദ്യമായ ഉപാധികള്‍  സത്യദര്‍ശികളിലും. "    

അജ്ഞാനികളുടെ, അതായത് ആത്മജ്ഞാനം ഇല്ലാത്തവരുടെ മനസ്സില്‍ ശരീരാഭിമാനവും മറ്റും മൂലം ഉണ്ടാവുന്ന ഉപാധികള്‍ ജനനമരണങ്ങള്‍ക്ക് കാരണമാവുന്നു. അവ ജഢമാണ്, വ്യര്‍ത്ഥമാണ്. എന്നാല്‍ സംപൂജ്യരായവരില്‍ കാണപ്പെടുന്ന ഉപാധികള്‍ ആത്മജ്ഞാനത്തില്‍ നിന്നുണ്ടാവുന്നതാണ്. ശരിയായ അറിവിന്റെ നിറവില്‍ അവ ഒരുവന്റെ ജനനമരണചക്രത്തെ അവസാനിപ്പിക്കുന്നു.

(വന്ദ്യമായ ഉപാധികളും പ്രകൃത്യാ ഉള്ള പരിമിതികളെ അംഗീകരിക്കുന്നു. ഉദാഹരണത്തിന്, ബാഹ്യേന്ദ്രിയങ്ങളായ കണ്ണും കാതും മറ്റും - അവയുടെ പരിമിതികളോടെയാണവര്‍ കാണുന്നത്. മൂഢന്റെ ചിന്തയില്‍ ഭൌതികശരീരവും ആത്മാവും ഒന്നാണ്. സംസ്കൃതത്തില്‍ സംസക്തി എന്ന് പറയുന്നത് ഒന്നിന് മറ്റൊന്നിനോട് ചേരാനുള്ള ത്വരയാണ്. ആസക്തി എന്നുപറയുമ്പോള്‍ ദ്വന്ദതയായി. അത് ഭിന്നതയെ സൂചിപ്പിക്കുന്നു. അതായത് ആസക്തി അപരിമിതമായതിനെ ഉപാധികളാല്‍ പരിമിതപ്പെടുത്തുന്നു എന്നര്‍ത്ഥം.

