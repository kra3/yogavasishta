 
\section{ദിവസം 066}

\slokam{
യഥാ സംവിത്തഥാ ചിത്തം സാ തഥാവസ്ഥിതിം ഗതാ\\
പരമേണ പ്രയത്നേന നീയതേന്യദശാം പുന: (3/40/13)\\
}

രാമന്‍ ചോദിച്ചു: മഹാത്മന്‍, ശരീരം വലുതും ഭാരമേറിയതുമാണല്ലോ? അപ്പോള്‍പ്പിന്നെ അതെങ്ങിനെ ചെറിയൊരു താക്കോൽ സുഷിരത്തിലൂടെ കയറിയിറങ്ങും?

വസിഷ്ഠന്‍ പറഞ്ഞു: രാമ: തീര്‍ച്ചയായും 'താന്‍ ഭൌതികശരീരമാണെന്നു' വിചാരമുള്ളവന്‌ അതസാദ്ധ്യമാണ്‌. "ഞാന്‍ ശരീരമാണ്‌, എന്റെ പ്രയാണത്തില്‍ തടസ്സങ്ങള്‍ ഉണ്ട്‌ എന്നു ചിന്തിച്ചാല്‍ തടസ്സം മൂര്‍ത്തീകരിച്ചു പ്രകടമാവുന്നു. ആ ചിന്തയില്ലെങ്കില്‍ , തടസ്സവും ഇല്ല." താഴോട്ടൊഴുകുകയെന്നത്‌ ജലത്തിന്റെ സ്വഭാവം. ജ്വാലയായിമുകളിലേയ്ക്കുയരുക എന്നത്‌ അഗ്നിയുടെ സ്വഭാവം. ബോധമോ എപ്പോഴും ബോധമായി നിലനില്‍ ക്കും. എന്നാല്‍ ഈ അറിവുറച്ചിട്ടില്ലാത്തവന്‌ സ്വരൂപത്തേപ്പറ്റിയോ അതിന്റെ സൂക്ഷ്മ സ്വഭാവത്തെപ്പറ്റിയോ അനുഭവമുണ്ടാവുകയില്ല. "ഒരുവന്റെ ധാരണ എപ്രകാരമോ അപ്രകാരമാണ്‌ മനസ്സ്‌. ധാരണകളാണ്‌ മനസ്സ്‌. എന്നാല്‍ അതിന്റെ ദിശകള്‍ കഠിനപ്രയത്നത്തിലൂടെ മാറ്റാവുന്നതാണ്‌."

സ്വാഭാവികമായും ഒരാളുടെ മനസ്സിനെ (അല്ലെങ്കില്‍ ധാരണയെ) ആശ്രയിച്ചാണ്‌ അയാളുടെ പ്രവൃത്തികള്‍ . എന്നാല്‍ ഈ ശരീരം അതിസൂക്ഷ്മാണെന്നു തിരിച്ചറിവുള്ളവനെ എന്തിനാണു തടസ്സപ്പെടുത്താനാവുക? വാസ്തവത്തില്‍ എല്ലാശരീരങ്ങളും എവിടെയായാലും അവയെല്ലാം ശുദ്ധബോധം തന്നെയാണ്‌. എന്നാല്‍ ഒരുവനില്‍ ഓരോരോ ആശയങ്ങള്‍ ഉണ്ടാവുകമൂലം ഇത്തരം'പോക്കു വരവുകള്‍ ' കാണപ്പെടുന്നുവെന്നു മാത്രം. ഈ വ്യക്തിബോധമെന്നതും അനന്താവബോധമെന്നതും പദാര്‍ത്ഥബോധമെന്ന ഈ വിശ്വവും എല്ലാം ഒന്നുതന്നെ. അതിനാല്‍ സൂക്ഷ്മശരീരത്തിന്‌ എവിടെയും പ്രവേശിക്കാം ഹൃദയാഭിലാഷത്തിനൊത്ത്‌ സഞ്ചരിക്കാം. എല്ലാവരുടെ ബോധത്തിനും ഇപ്പറഞ്ഞവിധത്തിലുള്ള സാദ്ധ്യതകളും കഴിവുകളുമുണ്ട്‌. 

ഓരോരുത്തരുടെയുള്ളിലും ലോകത്തെക്കുറിച്ച്‌ വ്യത്യസ്തധാരണകളാണുള്ളത്‌. മരണവും അത്തരം അനുഭവങ്ങളും വിശ്വപ്രളയം പോലെയാണ്‌ - വിശ്വാവബോധത്തിന്റെ രാത്രിയാണത്‌. രാത്രികഴിഞ്ഞ്‌ എല്ലാവരും അവരവരുടെ മാനസീകസൃഷ്ടികളിലേയ്ക്കുണരുകയാണ്‌. അതാണ്‌ അവരുടെ ആശയങ്ങളുടെ മൂര്‍ത്തീകരണം. ഭ്രമം, ധാരണ, എല്ലാം അതാണ്‌. പ്രളയശേഷം വിശ്വപുരുഷന്‍ പ്രപഞ്ചത്തെ സൃഷ്ടിക്കുമ്പോലെ വ്യക്തികള്‍ മരണശേഷം അവരവരുടെ ലോകവും സൃഷ്ടിക്കുന്നു. എന്നാല്‍ ബ്രഹ്മാ വിഷ്ണു മഹേശ്വരന്മാര്‍ തുടങ്ങിയ ദേവതകളും മഹര്‍ഷികളും പ്രളയത്തില്‍ ശാശ്വതമുക്തിയടയുന്നു. അടുത്ത യുഗത്തിലെ അവരുടെ സൃഷ്ട്യുന്മുഖത്വം ഓര്‍മ്മകളില്‍ നിന്നല്ല. മറ്റുള്ളവരില്‍ മരണശേഷമുള്ള സൃഷ്ട്യുന്മുഖത്വം ഉണ്ടാവുന്നത്‌ പൂര്‍വ്വജന്മങ്ങളിലെ വാസനകളാല്‍ പ്രേരിതമായിട്ടാണ്‌. ഈ വാസനകള്‍ ക്കു കാരണം ആ ജന്മത്തിലെ വ്യത്യസ്ഥ അനുഭവങ്ങളാണ്‌.

