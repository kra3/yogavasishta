\section{ദിവസം 207}

\slokam{
കാകതാലീയയോഗേന സമ്പന്നായം ജഗത്സ്ഥിതൗ\\
ധൂർതേന കല്പിതാ വ്യർഥം ഹേയോപാദേയഭാവനാ (5/9/49)\\
}

ജനകമഹാരാജാവ് തുടർന്നു: അഹംഭാവമെന്ന പിശാചുബാധിച്ചൊരു മൂഢനാണു ഞാൻ. ഈ അഹംഭാവമാണ്‌ ‘ഞാൻ ഇന്നയാളാണ്‌’ എന്ന ചിന്ത എന്റെയുള്ളിലുണ്ടാക്കുന്നത്. അനേകം ദേവതകളും ത്രിമൂർത്തികൾപോലും കാലത്താല്‍  ചവിട്ടിമെതിക്കപ്പെട്ടിട്ടുണ്ടെന്നറിഞ്ഞിട്ടും ഞാനീ ജീവിതത്തെ സ്നേഹിക്കുന്നു. അനന്താവബോധത്തിന്റെ ആനന്ദമനുഭവിക്കുന്നതിനു പകരം ഞാൻ വ്യർത്ഥവ്യാപാരങ്ങളിലും ആസക്തികളിലുമാണു ദിനരാത്രങ്ങൾ ചിലവാക്കുന്നത്. ദു:ഖത്തിൽനിന്നും കൊടിയ ദു:ഖങ്ങളിലേയ്ക്കു പോയിട്ടും എന്നിൽ നിർമമത ഉയർന്നില്ല. ഈ ലോകത്ത് അഭികാമ്യമായും ഉത്തമമായും എന്തിനെയാണു ഞാൻ കണക്കാക്കേണ്ടത്? കാരണം ഏതൊന്നിനെയായാലും എങ്ങിനെ പരിപോഷിപ്പിച്ചു കാത്തുസൂക്ഷിച്ചാലും അവയെല്ലാം  നാശവിധേയമാണ്‌..  ഒടുവിൽ ഉടമസ്ഥനു ബാക്കിയാവുന്നത് ശോകം മാത്രം.

ദിനംതോറും പാപപങ്കിലവും അക്രമാസക്തവുമായ കാര്യങ്ങളാണ് ലോകത്തു വളരുന്നത്. അവ കൂടുതൽ ദുരിതങ്ങളും ദു:ഖങ്ങളും ഉണ്ടാക്കുന്നു. കുട്ടിക്കാലം അജ്ഞതയിൽ കഴിഞ്ഞുപോകുന്നു. യൗവ്വനം കാമാസക്തിയിലും സുഖാന്വേഷണത്തിലും മുഴുകിക്കഴിയുന്നു. ബാക്കികാലം കുടുംബത്തിന്റെ കാര്യങ്ങളിൽ ഉഴറി നടക്കുകയാണു മനുഷ്യൻ. മൂഢനായ അവനു ജീവിതത്തിൽ സ്വന്തമായി, ശാശ്വതമായി, എന്താണു നേടാനാവുക?

മഹത്തായ യാഗാദികർമ്മങ്ങൾ അനുഷ്ഠിച്ചാലും ഏറിയാൽ ഒരു സ്വർഗ്ഗപ്രാപ്തി, അത്രയല്ലേ ലഭിക്കൂ? എന്താണീ സ്വർഗ്ഗം? അതു ഭൂമിയിലാണോ? അതൊരധോലോകമാണോ? സന്താപലേശമേൽക്കാത്ത ഒരിടമാണോ അത്? ദു:ഖം സുഖത്തെയും സുഖം ദു:ഖത്തേയും കൂട്ടിയാണു നടപ്പ്. ഈ ഭൂമിയിലെ കുഴികളെല്ലാം ശവശരീരങ്ങൾ കൊണ്ടു നിറച്ചിരിക്കുകയാണ്‌..  അതുകൊണ്ടാവും ഭൂമിക്കിത്ര ദൃഢത തോന്നുന്നത്!.

ഈ ഭൂമിയിലെ ചില ജീവികൾക്ക് അവരുടെ ഇമചിമ്മുന്നതിന്റെ കാലയളവ് യുഗങ്ങളോളമാണ്‌..  അതുമായി താരതമ്യംചെയ്യുമ്പോൾ എന്റെ ആയുസ്സെത്ര നിസ്സാരം! തീർച്ചയായും ലോകത്ത് ഉല്ലാസഭരിതവും ദീർഘായുസ്സുള്ളതുമായ വസ്തുക്കളുണ്ട്. എന്നാൽ അവയും അന്തമില്ലാത്ത വ്യാകുലതകളും ആശങ്കകളും നമുക്ക് സമ്മാനിക്കുന്നു. ഐശ്വര്യം സത്യത്തിൽ ആപത്താണ്‌..  ഈ ആപത്ത് നമ്മുടെ മനോനിലയനുസരിച്ച് അഭികാമ്യമായേക്കാം എന്നുമാത്രം. മനസ്സു മാത്രമാണല്ലോ ഈ പ്രത്യക്ഷലോകമെന്ന മായക്കാഴ്ച്ചയ്ക്കു കാരണം. ‘ഞാൻ’, ‘എന്റെ’ എന്നീ ധാരണകൾക്കു വിത്തായത് മനസ്സു തന്നെയാണ്‌.. 

“കാകതാലീയ ന്യായേന, (കാക്കയും പനമ്പഴവും - 'കാക്ക പനങ്കയ്യില്‍ വന്നിരുന്നതുകൊണ്ടാണ്‌ പനമ്പഴം താഴെ വീണത്' എന്നു പറയുമ്പോലെ), ഈ ലോകം സൃഷ്ടിക്കപ്പെട്ടതായി തോന്നുന്നത്,  ആകസ്മികമായാണ്‌..  അതുപോലെ തന്നെയാണ്‌ അജ്ഞാനം മൂലം ‘ഇതെനിയ്ക്കു വേണം’, ‘ഇതെനിയ്ക്കു ഹിതകരമല്ല’ എന്ന തോന്നലുകളും നമ്മില്‍ ഉണ്ടാവുന്നത്.”

'ഒറ്റക്കൊരിടത്തോ നരകത്തിൽത്തന്നെയോ കഴിയുന്നതാണ്‌ ഈ ലോകത്തിലെ ജീവിതത്തേക്കാൾ ഉത്തമം. ഇച്ഛാ ധാരണകളും പ്രചോദനങ്ങളുമാണ്‌ ലോകമെന്ന ഈ പ്രകടനത്തിനു കാരണം. ഈ പ്രചോദനത്തെ ഞാനിന്നില്ലാതാക്കും. ഞാനിതുവരെ പലവിധത്തിലുള്ള അനുഭവങ്ങൾ ആസ്വദിക്കുകയും സഹിക്കുകയും ചെയ്തു. എനിക്കിനി വിശ്രമിക്കണം. ഇനി ഞാൻ ദു:ഖിക്കുകയില്ല. ഞാൻ പ്രബുദ്ധനായിരിക്കുന്നു. എന്റെ ജ്ഞാനത്തെ അപഹരിച്ച കള്ളനെ ഞാനിന്നു കൊല്ലും! എനിക്ക് മഹർഷിമാരുടെ ഉപദേശം വേണ്ടപോലെ കിട്ടിയിട്ടുണ്ട്. ഇനി ഞാൻ ആത്മജ്ഞാനം തേടാൻ പോവുന്നു.' 

