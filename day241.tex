\section{ദിവസം 241}

\slokam{
സര്‍വ്വസംഭ്രമസംശാന്ത്യൈ പരമായ ഫലായ ച\\
ബ്രഹ്മവിശ്രാന്തിപര്യന്തോ വിചാരോഽസ്തു തവാനഘ(5/34/3)\\
}

ഭഗവാന്‍ പറഞ്ഞു: പ്രഹ്ലാദാ, നിന്നില്‍ സദ്ഭാവങ്ങളുടെ ഒരു കടലുതന്നെയുണ്ട്. തീര്‍ച്ചയായും അസുരന്മാരുടെ മണിരത്നമാണ് നീ. ജന്മാദിദുഃ:ഖങ്ങള്‍ ഇല്ലാതെയാക്കാനുതകുന്ന  ഏതു വരം വേണമെങ്കിലും ചോദിച്ചു കൊള്ളൂ.
പ്രഹ്ലാദന്‍ പറഞ്ഞു: ഭാഗവാനേ അവിടുന്നു സകലജീവികളുടെയും ഹൃദയത്തില്‍ നിവസിക്കുന്നു. ഞങ്ങളുടെ അഭീഷ്ടങ്ങളെ സാധിച്ചു തരുന്നു. അനന്തവും അപരിമേയവുമായി അങ്ങ് കരുതന്നതെന്തോ അതാണെനിക്ക് വരമായി വേണ്ടത്.

“ഭഗവാന്‍ പറഞ്ഞു: പ്രഹ്ലാദാ നിനക്ക് അനന്തമായ ബ്രഹ്മത്തില്‍ വിരാജിക്കുന്നതുവരെ അചഞ്ചലമായ ആത്മാന്വേഷണത്വര ഉണ്ടാവട്ടെ. അങ്ങിനെ നിന്നിലെ ഭ്രമകല്‍പ്പനകള്‍ അവസാനിച്ച് പരമപദമെന്ന ഉന്നതഫലം അനുഗൃഹമായിത്തീരട്ടെ.”

വസിഷ്ഠന്‍ പറഞ്ഞു: ഇത്രയും പറഞ്ഞു ഭഗവാന്‍ അപ്രത്യക്ഷനായി. ഭഗവല്‍ നാമസ്തോത്ര കീര്‍ത്തനജപങ്ങളോടെ പ്രഹ്ലാദന്‍ പൂജ അവസാനിപ്പിച്ചു. എന്നിട്ടിങ്ങിനെ ആലോചിച്ചു: ‘തുടര്‍ച്ചയായി നീ അന്വേഷണം ചെയ്യുക’ എന്നാണ് ഭഗവാന്‍ അരുളിയത്. ആത്മാവിനെക്കുറിച്ച് തീവ്രമായി അന്വേഷിക്കുക തന്നെ. നടക്കുകയും, സംസാരിക്കുകയും നില്‍ക്കുകയും ഈ വിശാലലോകത്തില്‍ പ്രവര്‍ത്തിക്കുകയും ചെയ്യുന്ന ഞാന്‍ ആരാണെന്നാദ്യം കണ്ടുപിടിക്കണം. തീര്‍ച്ചയായും മലകളും മരങ്ങളും ചെടികളും നിറഞ്ഞ ജഢമായ ഈ ബാഹ്യലോകമല്ല ഞാന്‍. ചെറിയൊരു കാലയളവ്‌ മാത്രം ജീവിക്കുന്ന, പ്രാണവായുവിന്റെ നീക്കം കൊണ്ട് സംജാതമായ ഈ പരിമിതദേഹവുമല്ല ഞാന്‍.    

വെറും ജഢമായ ഇന്ദ്രിയത്താല്‍ (ചെവിയാല്‍) തിരിച്ചറിയുന്ന വാക്കോ, ശബ്ദമോ അല്ല ഞാന്‍..  കാരണം വായുവിന്റെ താല്‍ക്കാലിക സഞ്ചാരമാണല്ലോ ഈ സംവേദനത്തിന് കാരണം. അത് സ്ഥിരമല്ല. ഞാന്‍ സ്പര്‍ശവുമല്ല. അതും താല്ക്കാലീകം മാത്രം. അനന്താവബോധത്തിന്റെ പ്രഭാവത്തില്‍ മാത്രമേ അതിനും നിലനില്‍പ്പുള്ളു. ഞാന്‍ നാവിലെ രുചിയുമല്ല. കാരണം അതെപ്പോഴും മാറിക്കൊണ്ടിരിക്കുന്നു. നാവാണെങ്കില്‍ ബാഹ്യവസ്തുക്കളുമായി എപ്പോഴും ബന്ധത്തിലാണ്. ഞാന്‍ എപ്പോഴും മാറിമറഞ്ഞുകൊണ്ടിരിക്കുന്ന കാഴ്ച്ചയുമല്ല. കാഴ്ച്ചക്കാരന്റെ അറിവിന്റെ വൈചിത്ര്യമനുസരിച്ചു മാറിമറയുന്ന രൂപങ്ങളുമല്ല ഞാന്‍. ഞാന്‍ ഘ്രാണശക്തിയുമല്ല. മൂക്കിന്റെ ഭാവനാ സങ്കല്‍പ്പം മാത്രമാണ് അനിയതരൂപങ്ങളുള്ള മണം.

ഈ ഭാവനാഗുണങ്ങളൊന്നും എന്റേതല്ല. ഇന്ദ്രിയങ്ങളുടെ നടത്തിപ്പുമായി എനിക്ക് യാതൊരു ബന്ധവുമില്ല. ഞാന്‍ ശുദ്ധബോധമാണ്. ചിന്തകള്‍ക്കതീതമായ പ്രശാന്തിയാണ് ഞാന്‍.

