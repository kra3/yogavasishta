\section{ദിവസം 002}

\begin{center}
 അഹം ബദ്ധോ വിമുക്ത: സ്യാമിതി യസ്യാസ്തി നിശ്ചയ:\\
നാത്യന്തമജ്ഞോ നോ തജ്ജ്ഞ: സോസ്മിൻച്ഛാസ്ത്രേധികാരവാൻ (1.2.2)\\
\end{center}

വാല്‍മീകി പറഞ്ഞു: "ഞാന്‍ ബദ്ധനാണ്‌ എന്ന തോന്നലും, എനിയ്ക്കു മുക്തി വേണം എന്ന  ആഗ്രഹവുമുള്ളവര്‍ക്കു  പഠിക്കാനുള്ളതാണ്‌ ശ്രീരാമ-വസിഷ്ഠ  സംഭാഷണരൂപ ത്തിലുള്ള  ഈ ഗ്രന്ഥം. തികഞ്ഞ അജ്ഞാനിക്കും പൂര്‍ണ്ണവിജ്ഞാനിക്കും ഇതുകൊണ്ട്‌ പ്രയോജനമില്ല."  ഈ ഗ്രന്ഥത്തില്‍ കഥാരൂപത്തില്‍ പ്രതിപാദിച്ചിട്ടുള്ള മോക്ഷമാര്‍ഗ്ഗങ്ങളേപ്പറ്റി വിചിന്തനം ചെയ്യുന്നവര്‍ ജനന മരണചക്രം എന്ന തുടര്‍ക്കഥയില്‍ നിന്നു വിടുതല്‍ നേടുന്നു.

ഞാന്‍ രാമകഥ നേരത്തേ തന്നെ രചിച്ചതും എന്റെ ശിഷ്യനായ ഭരദ്വാജന്‌ ചൊല്ലിക്കൊടുത്തതുമാണ്‌. ഒരിക്കല്‍ മേരുപര്‍വ്വതത്തില്‍ സന്ദര്‍ശനത്തിനുപോയപ്പോള്‍ അദ്ദേഹം സൃഷ്ടികര്‍ത്താവായ ബ്രഹ്മാവിന്‌ ആ കഥ പറഞ്ഞുകൊടുത്തു. കഥയില്‍ അതീവസന്തുഷ്ടനായ ബ്രഹ്മാവ്‌ ഭരദ്വാജന്‌ ഒരു വരം നല്‍കി. ഭരദ്വാജന്‍ ആവശ്യപ്പെട്ട വരം ഇതാണ്‌. "എല്ലാ മനുഷ്യര്‍ക്കും അവരുടെ സന്താപങ്ങളില്‍ നിന്നും മുക്തിയുണ്ടാവണം. ഇതു നേടാനുള്ള ഏറ്റവും ഉചിതമായ മാര്‍ഗ്ഗം പറഞ്ഞു തരികയും വേണം"

ബ്രഹ്മാവു പറഞ്ഞു: "വാല്‍മീകി മഹര്‍ഷിയുടെ അടുക്കല്‍ പോയി അദ്ദേഹത്തോട്‌ ശ്രോതാക്കളുടെ അജ്ഞാനാന്ധകാരം നീങ്ങും വിധത്തില്‍ ശ്രീരാമ കഥാ കഥനം തുടരാന്‍ അഭ്യര്‍ത്ഥിക്കുക." അതുകൊണ്ടും പൂര്‍ണ്ണതൃപ്തനാവാതെ ബ്രഹ്മാവ്‌ ഭരദ്വാജനുമൊത്ത്‌ എന്റെ ആശ്രമത്തില്‍ വന്നു. എന്റെ ഉപചാരങ്ങള്‍ സ്വീകരിച്ചശേഷം അദ്ദേഹം പറഞ്ഞു: "മഹാമുനേ അങ്ങ്‌ രചിച്ച രാമകഥ മനുഷ്യന്‌ സംസാരസാഗരം കടക്കുവാനുള്ള തോണിയായിത്തീരും. അതിനാല്‍ വിജയകരമായി കഥാ രചന പൂര്‍ത്തിയാക്കിയാലും". ഇത്രയും പറഞ്ഞ്‌ സൃഷ്ടികര്‍ത്താവ്‌ അപ്രത്യക്ഷനായി. അപ്രതീക്ഷിതമായി ബ്രഹ്മാനുശാസനം കിട്ടിയതുകൊണ്ട്‌ ചിന്താക്കുഴപ്പത്തിലായ ഞാന്‍ ബ്രഹ്മാവ്‌ എന്താണു കല്‍പ്പിച്ചതെന്ന് ഒന്നുകൂടി പറഞ്ഞുതരാന്‍ ഭരദ്വാജനോട്‌ അഭ്യര്‍ത്ഥിച്ചു. 

ഭരദ്വാജന്‍ പറഞ്ഞു: "സകലര്‍ക്കും ദു:ഖത്തിന്റെ മറുകരയെത്താന്‍ ഉതകും വിധം ശ്രീരാമന്റെ കഥ പറയുവാനാണ്‌ ബ്രഹ്മദേവന്‍ ആവശ്യപ്പെട്ടത്‌. ഞാനും അഭ്യര്‍ത്ഥിക്കുന്നു പ്രഭോ, രാമ ലക്ഷ്മണന്മാരും മറ്റു ഭ്രാതാക്കളും എങ്ങിനെയാണ്‌ ദു:ഖനിവാരണം നടത്തിയതെന്ന് വിശദമായി പറഞ്ഞു തന്നാലും."  അങ്ങിനെ രാമലക്ഷ്മണന്മാരും, മറ്റു ഭ്രാതാക്കളും അച്ഛനമ്മമാരും രാജസഭയിലുള്ള മറ്റുള്ളവരും മുക്തിപദം പ്രാപിച്ചതിന്റെ രഹസ്യം ഞാന്‍ അനാവരണം ചെയ്തു. എന്നിട്ട്‌ ഭരദ്വാജനോട്‌ പറഞ്ഞു: "കുഞ്ഞേ നിനക്കും അവരേപ്പോലെ ജീവിച്ച്‌ ഇപ്പോള്‍ , ഇവിടെ വച്ച്‌, ദു:ഖനിവൃത്തി നേടാം." 
