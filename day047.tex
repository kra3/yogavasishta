 
\section{ദിവസം 047}

\slokam{
ജഗത: പംചകം ബീജം പംചകസ്യ ചിദവ്യയാ\\
യദ്ബീജം തത്ഫലം വിദ്ധി തസ്മാദ്ബ്രഹ്മമയം ജഗത് (3/13/9)\\
}

വസിഷ്ഠന്‍ തുടര്‍ന്നു: ആ പരംപുരുഷനില്‍ നിലകൊണ്ട  സ്പന്ദനം ഒരേസമയം ക്ഷോഭിതവും സമതുലിതവും ആയിരുന്നു. അതു കാരണം ആകാശവും വെളിച്ചവും ജഢത്വവും - അവ സൃഷ്ടിക്കപ്പെട്ടില്ല എങ്കിലും- അതില്‍ പ്രകടമായി. ബോധതലത്തില്‍ സംഭവിക്കുന്നവയായതിനാല്‍ ഇവയ്ക്ക്‌ 'അറിയപ്പെടുന്നതിന്റെ' ഗുണഗണങ്ങള്‍ സ്വായത്തമാണ്‌. അതേസമയം അവയെ 'അറിയപ്പെടുന്നവനും' സംജാതനായി. എല്ലാറ്റിനേയും പ്രകാശിപ്പിക്കുക എന്നത്‌ ബോധത്തിന്റെ സഹജശക്തിയാണ്‌. അതുകൊണ്ടതിനെ വിശ്വസാക്ഷി എന്നു വിളിക്കുന്നു. ഈ ബോധം സ്വയം 'അറിവും' 'അറിവാളി'യുമത്രേ. 

ഇത്തരം പരസ്പരബന്ധമുണ്ടാവുമ്പോള്‍ 'ഞാന്‍ ജീവന്‍ , ജീവാത്മാവ്‌' എന്ന തോന്നല്‍ ബോധത്തിലുദയം ചെയ്യുന്നു. 'അറിവിനു' വിധേയമായവയുമായി തുടര്‍ച്ചയായ താദാത്മ്യം മൂലം ശുദ്ധബോധത്തില്‍ 'അഹം'കാര ഭാവം ഉണ്ടാവുന്നു. പിന്നെ വിവേചന ബുദ്ധിയും യുക്തിവിചാരവും ഉണ്ടാവുന്നു. മനസ്സും മൂലഘടകങ്ങളും ഉണ്ടാവുന്നു. ഈ മൂലഭൂതങ്ങളുടെ ആവര്‍ത്തിച്ചുള്ള സങ്കലനം കൊണ്ട്‌ ലോകങ്ങള്‍ ഉണ്ടാവുന്നു. സ്വപ്നത്തിലെ നഗരദൃശ്യങ്ങള്‍ പോലെ ക്രമാനുഗതമായോ യാദൃശ്ചീകമായോ ഉണ്ടാവുന്ന മാറ്റങ്ങളിലൂടെ ഇക്കാണുന്ന എണ്ണമറ്റ രൂപങ്ങള്‍ ആവര്‍ത്തിച്ചു മൂര്‍ത്തീകരിക്കുകയാണ്‌. ഈ നിര്‍മ്മിതികള്‍ക്കൊന്നും ഉപകരണങ്ങളോ മണ്ണ്‍, ജലം, അഗ്നി തുടങ്ങിയ വസ്തുക്കളോ ആവശ്യമില്ല. കാരണം ബോധത്തിന്റെ സഹജഭാവമാണിതെല്ലാം. ബോധമാണിതിനെയെല്ലാം സ്വപ്നത്തിലെ നഗരങ്ങളെന്നവണ്ണം സൃഷ്ടിക്കുന്നത്‌. ഇത്‌ ശുദ്ധബോധം മാത്രമാണ്‌.

"പ്രപഞ്ചവൃക്ഷത്തിന്റെ വിത്തുകളാണ്‌ പഞ്ചഭൂതങ്ങള്‍ . അനാദ്യന്തമായ ബോധമാണ്‌ പഞ്ചഭൂതങ്ങളുടെ വിത്ത്‌. വിത്ത്‌ ഏതുപോലെയാണോ, അതുപോലെ ഫലവും (വൃക്ഷവും). അതിനാല്‍ വിശ്വം എന്നത്‌ പരബ്രഹ്മമല്ലാതെ മറ്റൊന്നുമല്ല." ഇങ്ങിനെ വിശ്വാവബോധം അതിന്റെ അനന്തപ്രഭാവത്താല്‍ വിശ്വനഭസ്സില്‍ നടത്തുന്ന ഒരിന്ദ്രജാലമാണീ പ്രപഞ്ചം. ഇതു സത്തല്ല. ഇതൊരിക്കലും സൃഷ്ടിക്കപ്പെട്ടിട്ടുമില്ല. മേല്‍പ്പറഞ്ഞ ഘടകങ്ങള്‍ പരസ്പരമുള്ള സങ്കലനം കൊണ്ട്‌ വസ്തുക്കളെ സൃഷ്ടിച്ചതായി കാണപ്പെടുന്നുവെങ്കിലും അവയെല്ലാം ആകാശക്കോട്ടകള്‍ പോലെ വെറും തോന്നല്‍ മാത്രമാണ്‌. അവയുടെ നിലനില്‍പ്പ്‌ വിശ്വാവബോധമെന്ന അടിത്തറയിലാണ്‌. അതുമാത്രമേ സത്യമായുള്ളു. 

പഞ്ചഭൂതാത്മകമായ ഈ ലോകം ആ ഘടകവസ്തുക്കളുടെ സൃഷ്ടിയാണെന്നു തെറ്റിദ്ധരിക്കരുത്‌. പരമാവബോധത്തിന്റെ സഹജപ്രഭാവം പഞ്ചഭൂതങ്ങളായി പ്രത്യക്ഷപ്പെടുന്നു എന്നു മാത്രം. ഭൂമി മുതലായ ഭൂതങ്ങള്‍ ബോധത്തില്‍ നിന്നു സ്വപനവസ്തുക്കള്‍പോലെ ഉയര്‍ന്നുവന്നു എന്നോ ഇവ അവിദ്യ കാരണം വിശ്വാവബോധത്തില്‍ ആരോപിക്കപ്പെടുന്ന വെറും കാഴ്ച്ചകള്‍ മാത്രമാണെന്നോ പറയാം. മഹാത്മാക്കള്‍ ഇങ്ങിനെയൊക്കെയാണ്‌ സത്യത്തെ ദര്‍ശിക്കുന്നത്‌ (സാക്ഷാത്കരിക്കുന്നത്‌).
