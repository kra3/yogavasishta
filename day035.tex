\newpage
\section{ദിവസം 035}

\slokam{
വിചാരയാചാര്യപരമ്പരാണാം മതേന സത്യേന സിതേന താവത്
യാവദ്വിശുദ്ധം സ്വയമേവ ബുദ്ധ്യോ ഹ്യനന്തരൂപം പരമഭ്യുപൈഷി (2/19/35)
}


വസിഷ്ഠന്‍ തുടര്‍ന്നു: കഥകള്‍ക്ക്‌ സത്യത്തെ സാധകനില്‍ എളുപ്പമെത്തിക്കുക എന്ന ഒരേയൊരു ലക്ഷ്യമേയുള്ളു. സത്യസാക്ഷാത്കാരം വളരെ പ്രധാനപ്പെട്ടതായതുകൊണ്ട്‌ കഥകള്‍ കല്‍പ്പിതങ്ങളാണെങ്കിലും അതിനു ന്യായീകരണമുണ്ട്‌. കഥകള്‍ ഭാഗികമായേ സത്യവസ്തുവിനെ മനസ്സിലാക്കാന്‍ ഉതകൂ അതിനാല്‍ ആ ഭാഗം മാത്രം സ്വീകരിച്ച്‌ ബാക്കിയുള്ള കഥാഭാഗങ്ങളെ വിട്ടുകളയുകയാണുചിതം. സത്യം അറിഞ്ഞുണരുന്നതുവരെ മാത്രമേ ഒരുവന്‌ യോഗ്യനായ ഗുരുവിന്റെ കീഴിലുള്ള വേദപഠനവും ഈ വിധ കഥകളുടേയും ആവശ്യമുള്ളു. എന്നാല്‍ പരമസാക്ഷാത്കാരം നേടി പ്രബുദ്ധത കൈവരുന്നതുവരെ അയാള്‍ സാധന ഉപേക്ഷിക്കരുത്‌. വേദങ്ങളിലെ അല്‍പമാത്രമായ അറിവ്‌ ഗുണത്തിനുപകരം ചിന്താക്കുഴപ്പത്തിനു വഴിതെളിക്കുന്നു. അപൂര്‍ണ്ണവും അപക്വവുമായ അറിവ്‌ പലേവിധ അബദ്ധധാരണകള്‍ക്കും കാരണമാവുന്നു. അല്‍പ്പജ്ഞാനികള്‍ വികടയുക്തിയുടെ പിടിയില്‍ പ്പെട്ട്‌ ഏതെങ്കിലും കല്‍പ്പിതവസ്തുവിനെ സത്യമെന്നു കരുതി, സ്വഹൃദയത്തില്‍ പരം പൊരുള്‍ കുടികൊള്ളുന്നതറിയാതെ പോകുന്നു. 

എല്ലാ തിരമാലകളുടേയും അന്ത:സത്ത സമുദ്രം തന്നെയെങ്കിലും അതിന്റെ നേരനുഭവം മാത്രമേ സത്യമാവൂ. അനുഭവിക്കുന്നവനും അനുഭവമെന്ന പ്രക്രിയയും, അനുഭവംതന്നെയും ബോധമണ്ഡലത്തിലാണുണ്ടാവുന്നത്‌. അനുഭവം മാത്രമാണുണ്മയെങ്കിലും അവിദ്യകൊണ്ട്‌ അനുഭവിക്കുന്നയാള്‍ സ്വയം അനുഭവത്തില്‍ നിന്നും ഭിന്നനായി തോന്നുന്നു. ആത്മാന്വേഷണത്തിന്റെ ഫലമായി ഉണ്ടായ അറിവിന്റെ വെളിച്ചത്തില്‍ അവിദ്യയെന്ന ഇരുട്ടു നീങ്ങി അവിച്ഛിന്നമായ ബോധം ഭാസുരമാവുമ്പോള്‍ സാധകന്റെ അന്വേഷണം പോലും അനാവശ്യമായിമാറി അത്  താനേ വിലീനമാവുന്നു. വായുവില്‍ ചലനം സഹജമെന്നതുപോലെ ഈ അനുഭവബോധത്തില്‍ വസ്തുക്കളും അവയെ സാക്ഷാത്കരിക്കുന്ന മനസ്സും സഹജമായി പ്രകടിതമത്രേ. അവിദ്യകാരണം, ഈ അവബോധക്ഷമമായ മനസ്സ്‌ 'ഞാന്‍ ഇന്നയിന്ന വസ്തുക്കളാണ്‌' എന്നു ചിന്തിച്ച്‌ ആ വസ്തുക്കളായിത്തീരുന്നു. വസ്തുക്കള്‍ നിലകൊള്ളുന്നത്‌ അവയെ അനുഭവിക്കുന്നയാളുടെ ഉള്ളില്‍ മാത്രമാണ്‌. മറ്റ്  എങ്ങും അതിന്‌ അസ്തിത്വമില്ല. 

"രാമ: ഈ അറിവു നിന്നില്‍ നേരായി ഉറയ്ക്കുന്നതുവരെ മഹത്ഗുരുക്കന്മാര്‍ പകര്‍ന്നു തരുന്ന അറിവില്‍ അഭയംതേടുക". അങ്ങിനെ നീ മഹാന്മാരില്‍ നിന്നും വിദ്യ സ്വീകരിക്കുമ്പോള്‍ നീയും അവരുടെ നിഴല്‍ പോലെ വര്‍ത്തിക്കാന്‍ തുടങ്ങും. ഒടുവില്‍ അവരുടെ സദ്ഗുണങ്ങള്‍ നിനക്കും സ്വായത്തമാവുമ്പോള്‍ നിന്നില്‍ പരം പൊരുള്‍ താനേ അറിവായുണരും. അറിവും മഹാത്മാക്കളുടെ സ്വഭാവഗുണങ്ങളുടെ അനുകരണവും പരസ്പരം പ്രചോദിപ്പിക്കുന്നു.

കുറിപ്പ്‌:
ആകാശം എന്നാല്‍ മാനം, അളവ്‌ എന്നെല്ലാം അര്‍ത്ഥം. പുസ്തകത്തില്‍ മൂന്നു പ്രധാന വാക്കുകള്‍ വരുന്നുണ്ട്‌. ചിദാകാശം, ചിത്താകാശം, പിന്നെ ഭൂതാകാശം. ചിദാകാശമെന്നാല്‍ ബോധമണ്ഡലം. ചിത്താകാശമെന്നാല്‍ മനോമണ്ഡലം. ഭൂതാകാശമെന്നാല്‍ ധാതുമണ്ഡലം. ഭഗവാന്‍ രമണമഹര്‍ഷി ഈ മൂന്നു ധാരണകളെ ഭംഗിയായി വിവരിച്ചിട്ടുണ്ട്‌. 

"ചിദാകാശം ആത്മസ്വരൂപമെന്നു പറയപ്പെടുന്നു. മനസ്സുകൊണ്ടുമാത്രമേ ഇതു കാണാന്‍ കഴിയൂ. അപ്പോള്‍പ്പിന്നെ മനസ്സൊടുങ്ങിയാല്‍ ആത്മസ്വരൂപം എങ്ങിനെ ദര്‍ശിക്കാനാവും?" ഒരാള്‍ ചോദിച്ചു. ഭഗവാന്‍ പറഞ്ഞു: "ആകാശത്തെ ഉദാഹരണമാക്കിയെടുത്താല്‍ അതിന്‌ മൂന്നു വിഭിന്ന ഭാവങ്ങളുണ്ട്‌. ചിദാകാശം, ചിത്താകാശം, ഭൂതാകാശം എന്നിവ. ചിദാകാശത്തില്‍നിന്നും 'ഞാന്‍ ' എന്നങ്കുരിക്കുന്ന 'സ്വ'ഭാവമാണ്‌ ചിത്താകാശം. ചിത്താകാശം താനേ വികസ്വരമായി പഞ്ചഭൂതങ്ങളുടെ ആകൃതിയെ പ്രാപിക്കുമ്പോള്‍ അത്‌ ഭൂതാകാശമായി. ചിത്താകാശം, ചിദാകാശത്തെ കാണാതെ ഭൂതാകാശത്തെ മാത്രം കാണുമ്പോള്‍ അതിനെ 'മനോ ആകാശം' എന്നു പറ യുന്നു. എന്നാല്‍ ഭൂതാകാശത്തെ കാണാതെ ചിദാകാശത്തെ മാത്രം ദര്‍ശിക്കുമ്പോള്‍ അതിനെ 'ചിന്മയം' - ശുദ്ധബോധം എന്നു പറയുന്നു. മനസ്സു വിലയിക്കുന്നു എന്നു പറഞ്ഞാല്‍ നാനാത്വസ്വഭാവമാര്‍ന്ന വസ്തുബോധം മറഞ്ഞ്‌ ഏകവസ്തു എന്ന ആശയം പ്രകടമാവുന്നു എന്നര്‍ത്ഥം. അപ്പോള്‍ എല്ലാം സ്വാഭാവികമായും സ്വരൂപമായി കാണപ്പെടുന്നു."

ആകാശം എന്നതിന്‌ 'മാനം' എന്ന അര്‍ത്ഥം കൂടുതല്‍ ഉചിതമാണെന്നു തോന്നുന്നു. (ഉദാഹരണത്തിന്‌ ത്രിമാനം). അനന്തബോധത്തെ ചിദാകാശമായും, ചിത്താകാശമായും ഭൂതാകാശമായും കാണുമ്പോള്‍ അത്‌ ആത്മീയം, മാനസീകം അല്ലെങ്കില്‍ ധാരണാപരം, ഭൌതീകം എന്നീ മാനങ്ങളുള്ളതായി പറയാം.

(മുമുക്ഷു പ്രകരണം എന്ന രണ്ടാം ഭാഗം അവസാനിച്ചു)
