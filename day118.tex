 
\section{ദിവസം 118}

\slokam{
കാകതാലീയയോഗേന ത്യക്ത്ത സ്ഫാരദൃഗകൃതേ:\\
ചിത്തേശ്ചേത്യാനുപാതിന്യാ: കൃതാ: പര്യായവൃത്തയ: (3/96/15)\\
}

വസിഷ്ഠന്‍ തുടര്‍ന്നു: മനസ്സെന്നത്‌ പഞ്ചേന്ദ്രിയങ്ങളിലൂടെയുള്ള അറിവാണ്‌.. . അത്‌ ബോധത്തിലെ ചലനമാണ്‌.. ഈ ചലനം കര്‍മ്മമായി പുറത്തു പ്രകടമാവുന്നു. തുടര്‍ന്ന് ആ കര്‍മ്മത്തിന്‌ ഫലപ്രാപ്തിയും ഉണ്ടാവുന്നു. സര്‍വ്വശക്തമായ അനന്താവബോധത്തില്‍ ഉദിക്കുന്ന ഉദ്ദേശ്യങ്ങളാണ്‌, ലക്ഷ്യമാണ്‌, മനസ്സ്‌.. അത്‌ സത്തിനും അസത്തിനും മദ്ധ്യേ നിലകൊള്ളുന്നപോലെ തോന്നുന്നു. എങ്കിലും അതിന്‌ ധാരണാശക്തിയുമായാണ്‌ അടുപ്പം. അനന്താവബോധത്തില്‍ നിന്നും വിഭിന്നമല്ലെങ്കിലും സ്വയം അങ്ങിനെയാണെന്ന് മനസ്സു ചിന്തിക്കുന്നു. കര്‍മ്മബന്ധിതനല്ലെങ്കിലും സ്വയം കര്‍ത്താവാണെന്ന് അഭിമാനിക്കുന്നു. മനസ്സ്‌ ഇങ്ങിനെയൊക്കെയാണ്‌.. ഈ സ്വഭാവസവിശേഷതകള്‍ മാറ്റാന്‍ സാധിക്കുകയില്ല. കാരണം, അവയില്ലെങ്കില്‍ മനസ്സില്ല. മനസ്സും ജീവനും തമ്മില്‍ വേര്‍പിരിക്കാനാവില്ല. മനസ്സില്‍ ചിന്തിക്കുന്നത്‌ നടപ്പിലാക്കാന്‍ കര്‍മ്മേന്ദ്രിയങ്ങള്‍ പരിശ്രമിക്കുന്നു. അപ്പോള്‍ മനസ്സ്‌ കര്‍മ്മവുമാണ്‌. .

മനസ്സ്‌, ബുദ്ധി, അഹംകാരം, വ്യക്തി അവബോധം, കര്‍മ്മം, മോഹം, ജനന മരണങ്ങള്‍ , വാസനകള്‍ , അറിവ്‌, പരിശ്രമം, ഓര്‍മ്മ, ഇന്ദ്രിയങ്ങള്‍ , പ്രകൃതി, മായ, ഭ്രമം, പ്രവര്‍ത്തനം, തുടങ്ങിയ വാക്കുകള്‍ ഉണ്മയുമായി ബന്ധമൊന്നുമില്ലാത്ത കേവലശബ്ദങ്ങള്‍ മാത്രമാണ്‌.. ഉണ്മയായുള്ളത്‌ അനന്താവബോധം മാത്രം .  ഈ ധാരണകളെല്ലാം ആ അവബോധത്തില്‍ നിലനില്‍ക്കുന്നതായി സങ്കല്‍പ്പിക്കപ്പെടുന്നു എന്നുമാത്രം. "കാകതാലീയ ന്യായേനയുണ്ടായ ഒരു യാദൃശ്ചികത്വമായാണ്‌ മേല്‍പ്പറഞ്ഞ ധാരണകളെല്ലാം സംജാതമായത്‌.. അനന്താവബോധം ക്ഷണനേരത്ത്‌ സ്വരൂപത്തെ മറന്നതിന്റെ ഫലമായി അതു സ്വയം 'അറിയപ്പെടുന്ന' വസ്തുവായി തെറ്റിദ്ധരിച്ചു". ഇങ്ങിനെ അവിദ്യകൊണ്ടു മൂടിയ അനന്താവബോധം, വിക്ഷോഭസ്ഥിതിയില്‍ നനാത്വങ്ങളെ കാണുന്നു. അവയുമായി താദാത്മ്യം ഭാവിക്കുന്നതാണ്‌ മനസ്സ്‌.. അതേ മനസ്സ്‌ ചിലപ്രത്യേക ധാരണകളില്‍ യുക്തിസഹമായി ദൃഢീകൃതമാവുന്നതാണ്‌ ബുദ്ധി. സ്വയം എല്ലാറ്റില്‍നിന്നും സ്വതന്ത്രനായ ഒരു വ്യക്തിയാണെന്ന് അഭിമാനിക്കുമ്പോള്‍ അത്‌ അഹംകാരമായി. തുടര്‍ച്ചയായ അന്വേഷണങ്ങളെല്ലാം അവസാനിപ്പിച്ച്‌ എണ്ണമില്ലാതെ വന്നു പോയിക്കൊണ്ടിരിക്കുന്ന ചിന്തകള്‍ക്ക്‌ വിളയാടാന്‍ ഇടയാക്കുന്നതാണ്‌ വ്യക്തിബോധം (അതാണ് മനോവസ്തു).

ബോധത്തിലെ ശുദ്ധ ചലനം കര്‍മ്മമാണെങ്കിലും അതിന്‌ ഒരു 'കര്‍ത്താവ്‌' ഇല്ല. എന്നാല്‍ ബോധം ആ കര്‍മ്മഫലം കാംക്ഷിക്കുമ്പോള്‍ അത്‌ 'കര്‍മ്മം' ആവുന്നു. 'ഞാന്‍ ഇതു മുന്‍പു കണ്ടിട്ടുണ്ടല്ലോ' എന്നൊരു ധാരണ കാണുന്നതിനെക്കുറിച്ചോ കാണാത്തതിനെക്കുറിച്ചോ ബോധത്തിലുയരുന്നതാണ്‌ ഓര്‍മ്മ (സ്മൃതി). പൂര്‍വ്വാനുഭവങ്ങളുടെ ബാക്കിഫലപ്രതീതി ബോധതലത്തില്‍ പ്രകടമായോ അല്ലാതെയോ തങ്ങി നില്‍ക്കുന്നതാണ്‌ ലീനവാസന അല്ലെങ്കില്‍ സാദ്ധ്യത. വിഭിന്നതയും നാനാത്വവും അവിദ്യ ഹേതുവായുണ്ടായ ഒരുഭ്രമം മാത്രമാണെന്ന തിരിച്ചറിവ്‌ ബോധത്തില്‍ ഉണരുമ്പോള്‍ അത്‌ ജ്ഞാനം. എന്നാല്‍ ബോധം തെറ്റായ മാര്‍ഗ്ഗത്തില്‍ ചലിച്ച്‌ കൂടുതല്‍ ആത്മവിസ്മൃതിയിലാണ്ട്‌ ഭ്രമകല്‍പ്പനയില്‍ ആസക്തമാവുമ്പോള്‍ അത്‌ മലിനമാണ്‌.. അത്‌ അന്തര്യാമിയുമായി വൈകാരികതലത്തില്‍  സംവദിക്കുമ്പോള്‍ ഇന്ദ്രിയങ്ങളാവുന്നു. അപ്രത്യക്ഷമായി വിശ്വജീവനില്‍ നിലകൊള്ളുമ്പോള്‍ അത്‌ പ്രകൃതിയാണ്‌.. ഉണ്മയും പ്രത്യക്ഷലോകവും തമ്മിലുണ്ടാവുന്ന ചിന്താക്കുഴപ്പത്തില്‍ അത്‌ മായ എന്നറിയപ്പെടുന്നു. അത്‌ അനന്തതയില്‍ വിലയിക്കുമ്പോഴാണ്‌ മുക്തി. 'ഞാന്‍ ബദ്ധന്‍' എന്നു ചിന്തിക്കുമ്പോള്‍ ബന്ധനം. 'ഞാന്‍ സ്വതന്ത്രന്‍' എന്നു ചിന്തിക്കുമ്പോള്‍ സ്വാതന്ത്ര്യം. 

