\newpage
\section{ദിവസം 122}

\slokam{
കരണം കര്‍മ്മ കര്‍ത്താ ച ജനനം മരണം സ്ഥിതി:\\
സര്‍വ്വം ബ്രഹ്മൈവ നഹ്യസ്തി തദ്വിനാ കല്‍പനേതരാ (3/100/30)\\
}

വസിഷ്ഠന്‍ തുടര്‍ന്നു: രാമ: വ്യക്തിഗതബോധം, അല്ലെങ്കില്‍ മനസ്സ്‌, പരമപുരുഷനില്‍ നിന്നുത്ഭവിച്ചതാണ്‌.. എന്നാല്‍ അത്‌ അനന്താവബോധത്തില്‍ നിന്നും വിഭിന്നമാണെന്നും അല്ലെന്നും പറയാം. സമുദ്രജലത്തില്‍നിന്നും തിരമാലകള്‍ വിഭിന്നമണെന്നും അല്ലെന്നും പറയാവുന്നതുപോലെയത്രേ ഇത്‌.. പ്രബുദ്ധപദം പ്രാപിച്ചവന്‌ മനസ്സ്‌ പരബ്രഹ്മം തന്നെ. അജ്ഞാനിയുടെ, സംസാരമെന്ന ആവര്‍ത്തനചരിതത്തിന്റെ കാരണം ഈ മനസ്സാണ്‌.. അയാഥാര്‍ത്ഥ്യമാണെങ്കിലും നാം ദ്വന്ദധാരണകളെ ഉപയോഗിക്കുന്നുവെങ്കില്‍ അതെല്ലാം പഠനാവശ്യത്തിനു മാത്രമാണെന്ന് ഓര്‍മ്മിക്കുമല്ലോ. പരബ്രഹ്മം സര്‍വ്വ ശക്തമാണ്‌.. അതിനുവെളിയിലായി ഒന്നുമില്ല. ഏകവും അദ്വിതീയവുമാണത്‌.. അതിന്റെ ചൈതന്യം, ശക്തിയാണ്‌, എല്ലാടവും, എല്ലാത്തിനേയും വ്യാപരിച്ചിരിക്കുന്നത്‌..

ശരീരമുള്ള ജീവികളില്‍ അത്‌ ചിത്ശക്തിയാണ്‌.. അതായത്‌ ധിഷണ, അല്ലെങ്കില്‍ ബോധശക്തി. അത്‌ വായുവിന്റെ ചലനമാണ്‌; ആകാശത്തിലെ ശൂന്യതയാണ്‌.. ജീവികളില്‍ 'ഞാന്‍ ഉണ്ട്‌' എന്നു സ്ഫുരിച്ചുകൊണ്ടേയിരിക്കുന്ന ആത്മബോധത്തിന്റെ പ്രഭാവമാണ്‌.. എന്നാല്‍ ഇതെല്ലാം പരബ്രഹ്മത്തിന്റെ ശക്തിവിശേഷം മാത്രമാണ്‌.. ശിഥിലീകരണശക്തിയും, ദു:ഖിതരിലെ ദു:ഖഹേതുവും, ആഹ്ലാദചിത്തരിലെ ആഹ്ലാദഹേതുവും, യോദ്ധാക്കളിലെ ശൌര്യവും, വിശ്വ സൃഷ്ടി സ്ഥിതി സംഹാര ശക്തിയും എല്ലാം അതു തന്നെ. ജീവന്‍ ബോധത്തിന്റേയും പദാര്‍ത്ഥങ്ങളുടേയും ഇടയ്ക്കാണ്‌.. അത്‌ പരബ്രഹ്മത്തിന്റെ ഒരു പ്രതിഫലനമായതിനാല്‍ ജീവന്‍ ബ്രഹ്മത്തില്‍ നിലകൊള്ളുന്നു എന്നു പറയപ്പെടുന്നു. വിശ്വത്തെ മുഴുവനും ഏകമായി കാണൂ. 'ഞാന്‍' എന്നതും പരബ്രഹ്മം തന്നെയാണെന്നറിയുക. അത്മാവ്‌ സര്‍വ്വവ്യാപിയത്രേ.

അത്മാവില്‍ ചിന്തയുണ്ടാവുമ്പോള്‍ മനസ്സുണ്ടാവുന്നു. അത്‌ പരബ്രഹ്മത്തിന്റെ പ്രാഭവമാണ്‌. . 'ഞാന്‍', 'ഇത്‌' എന്നെല്ലാമുള്ള നിരങ്കുശവിഭിന്നതകള്‍ വെറും പ്രതിബിംബങ്ങള്‍ മാത്രം. മനസ്സ്‌ വാസ്തവത്തില്‍ ബ്രഹ്മം തന്നെ. ഇടയ്ക്കിടയ്ക്ക്‌ ചിലനേരങ്ങളില്‍ ബ്രഹ്മപ്രാഭവത്തില്‍ ചിലത്‌ പ്രകടമാവാറുണ്ട്‌.. അവയും വാസ്തവത്തില്‍ ഉണ്മയായ സൃഷ്ടിയൊന്നുമല്ല; വെറും പ്രതിഫലനങ്ങള്‍ മാത്രം. സൃഷ്ടി, പരിണാമം, സ്ഥിതി, സംഹാരം എന്നിവയെല്ലാം ബ്രഹ്മത്തിലുണ്ടായി, നിലനിന്ന്, വിലയിക്കുന്നു. ബ്രഹ്മമല്ലാതെ യാതൊന്നും ഇല്ല. "കര്‍മ്മേന്ദ്രിയങ്ങള്‍ , കര്‍മ്മം, കര്‍ത്താവ്‌, മരണം, അസ്തിത്വം, എല്ലാമെല്ലാം ബ്രഹ്മം മാത്രം. ഭാവനകളടക്കം, ബ്രഹ്മമല്ലാതെ യതൊന്നുമില്ല."

ഭ്രമം, ആര്‍ത്തി, അത്യാഗ്രഹം, ആസക്തി, തുടങ്ങിയ ഒന്നും നിലനില്‍പ്പുള്ളതല്ല. ദ്വന്ദത ഇല്ലാത്തപ്പോള്‍ ഇവയ്ക്കെങ്ങിനെ നിലനില്‍പ്പുണ്ടാവും? ബന്ധനം എന്നത്‌ ഇല്ലാത്തതാണെങ്കില്‍ മോക്ഷം എന്നതും മിഥ്യയത്രേ.

രാമന്‍ ചോദിച്ചു: മഹാത്മന്‍, മനസ്സ്‌ എന്തെങ്കിലും ചിന്തിച്ചാല്‍ അതു നടപ്പാവുന്നു എന്ന് അങ്ങു പറഞ്ഞു. ഇപ്പോള്‍ അങ്ങു പറയുന്നു ബന്ധനം എന്നത്‌ മിഥ്യയാണെന്ന്. ഇതെങ്ങിനെ ശരിയാവും?

വസിഷ്ഠന്‍ പറഞ്ഞു: രാമ: മനസ്സ്‌ അജ്ഞാനത്തിന്റെ പിടിയില്‍പ്പെട്ട്‌ താന്‍ ബദ്ധനാണെന്ന് ഭാവന ചെയ്യുകയാണ്‌.. അവിദ്യമൂടിയ മനസ്സിലേ ബന്ധനമുള്ളു. ഉറക്കമുണര്‍ന്നെഴുന്നേല്‍ക്കുമ്പോള്‍ അപ്രത്യക്ഷമാവുന്ന സ്വപ്നദൃശ്യങ്ങള്‍പോലെ, ബന്ധനം, മുക്തി തുടങ്ങിയ മോഹവിഭ്രാന്തികള്‍ പ്രബുദ്ധനായ ഒരുവന്റെ ദൃഷ്ടിയില്‍ മിഥ്യയാണ്‌.

