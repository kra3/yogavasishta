\section{ദിവസം 211}

\slokam{
പ്രജ്ഞയേഹ ജഗത്സർവം സംമ്യഗേവാംഗ ദൃശ്യതേ\\
സംയഗ് ദർശനമായാന്തി നാപദോ നച സമ്പദ: (5/12/38)\\
}

വസിഷ്ഠൻ തുടർന്നു: പരിഹരിക്കാനാവാത്ത എന്തെന്തു ദു:ഖങ്ങളുണ്ടായാലും അന്ത:പ്രകാശമാവുന്ന വിജ്ഞാനത്തിന്റെ തോണിയില്‍ക്കയറി പോംവഴി കണ്ടെത്താം. എന്നാൽ ഈ വിജ്ഞാനമില്ലാത്തവൻ ചെറിയൊരു പ്രശ്നത്തിൽപ്പോലും വ്യാകുലചിത്തനാകുന്നു. ഈ ഉൾവെളിച്ചമുള്ളവന്‌, അവന്‍ പഠിക്കാത്തവനാണെങ്കിലും, ഏകാന്തനാണെങ്കിലും ദു:ഖസാഗരത്തിന്റെ മറുകരകാണുക ക്ഷിപ്രസാദ്ധ്യമത്രേ. മറ്റൊരാളിന്റെ സഹായമവനാവശ്യമില്ല. എന്നാൽ വിജ്ഞാനമില്ലാത്തവന്‌ അവന്റെ മൂലധനമടക്കം എല്ലാം നഷ്ടമാവുന്നു.

അതുകൊണ്ട് ഈ ഉൾവെളിച്ചത്തിനായി ഏവരും പരിശ്രമിക്കണം. ഫലങ്ങളാഗ്രഹിക്കുന്നയാൾ തോട്ടത്തിനെ പരിരക്ഷിക്കുമ്പോലെ ജാഗ്രതയോടെവേണം ഇതുചെയ്യാൻ. വേണ്ടപോലെ പരിപോഷിപ്പിച്ചാൽ ഈ ഉൾവെളിച്ചം ആത്മജ്ഞാനത്തെ പ്രദാനംചെയ്യുന്നു. ലൗകീകകാര്യങ്ങളിൽ ശ്രദ്ധപതിപ്പിക്കുന്നതിനുപകരം, ആതമജ്ഞാനം നേടുവാനാണ്‌ ആദ്യം ശ്രമിക്കേണ്ടത്. ആദ്യം തന്നെ ദു:ഖദുരിതങ്ങളുടെ വിളനിലമായ മനോമാന്ദ്യത്തെ നീക്കണം. കാരണം, ലോകമെന്ന  വൃക്ഷത്തിന്റെ വിത്താണത്. സ്വർഗ്ഗനരകങ്ങളിൽ നേടാവുന്നതെല്ലാം ആത്മജ്ഞാനിക്ക് ‘ഇവിടെ ഇപ്പോൾ’ സാദ്ധ്യമാണ്‌.. ജ്ഞാനത്താല്‍  മാത്രമാണ് സംസാരസാഗരം തരണംചെയ്യാൻ കഴിയുക. ദാനധർമ്മാദികൾക്കോ തീർത്ഥാടനങ്ങൾക്കോ സന്യാസത്തിനോ ഒന്നും അതിനു കഴിയില്ല. ഈ വിജ്ഞാനംകൊണ്ടാണ്‌ സിദ്ധന്മാരായ ദിവ്യപുരുഷന്മാർ അവരുടെ കഴിവുകൾ നേടിയത്. രാജാക്കന്മാരും അവരുടെ സിംഹാസനം നേടിയതും ജ്ഞാനത്താലത്രേ.

ജ്ഞാനം സ്വർഗ്ഗപാതയെ സുഗമമാക്കുന്നു. പരമമായ നന്മയെ, മുക്തിയെ പ്രദാനംചെയ്യുന്നു. ഈ വിജ്ഞാനമാണ്‌ ഒരു സാധാരണക്കാരനെ ശക്തനായൊരു എതിരാളിയെപ്പോലും ജയിക്കാൻ പര്യാപ്തനാക്കുന്നത്. കഥകളില്‍പ്പറയുന്ന, ആഗ്രഹിക്കുന്നതെന്തും നല്കുന്ന ചിന്താമണിപോലെയാണു രാമാ, ഈ ജ്ഞാനം എന്നു പറയുന്ന ഉൾവെളിച്ചം. ഈ വെളിച്ചത്തിൽ സംസാരസാഗരത്തിന്റെ മറുകരയെത്തൽ സുസാദ്ധ്യം. അതില്ലാത്തവനോ, അവനീ സാഗരത്തിൽ മുങ്ങിപ്പോകുന്നു. ജ്ഞാനപ്രകാശം ഒരുവന്റെ ബുദ്ധിശക്തിയെയും അറിവിനെയും നേരായവഴിയിൽ നയിക്കുന്നതിനാൽ ജ്ഞാനിക്ക് പാതയിൽ തടസ്സങ്ങൾ നേരിടേണ്ടി വരുന്നില്ല. വിഭ്രാന്തലേശമില്ലാത്ത മനസ്സിനെ ആശകളോ ദുഷ്ടചിന്തകളോ, പോരായ്മകളോ സമീപിക്കുകപോലുമില്ല.

“ജ്ഞാനമാകുന്ന ഉൾവെളിച്ചത്തിൽ ലോകത്തെ അതിന്റെ സഹജഭാവത്തിൽ വ്യക്തമായി കാണാം. ഇങ്ങിനെ വ്യക്തവും പൂർണ്ണവുമായ കാഴ്ച്ചയുള്ളവനെ ഭാഗ്യനിർഭാഗ്യങ്ങൾ സമീപിക്കുകകൂടിയില്ല.” സൂര്യനെ മറയ്ക്കുന്ന കാർമേഘത്തെ മാറ്റാൻ കാറ്റിനാവും. ആത്മാവിനെ വലയം ചെയ്തിരിക്കുന്ന അഹംഭാവമെന്ന ആന്ധ്യമകറ്റാൻ ജ്ഞാനമെന്ന ഉൾവെളിച്ചത്തിനാവും. ധാന്യവിള കൊയ്യാനാഗ്രഹിക്കുന്നവൻ പാടമുഴുത് മണ്ണു തയ്യാറാക്കുന്നതുപോലെ പരമാവബോധം ലക്ഷ്യമായുള്ളവൻ ആദ്യം തന്റെ മനസ്സ് ശുദ്ധീകരിക്കണം. അതിനായി ജ്ഞാനമെന്ന ഉൾവെളിച്ചംകൊണ്ട് തന്റെ ഉള്ളം നിറയ്ക്കണം. 
