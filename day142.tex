\newpage
\section{ദിവസം 142}

\slokam{
അവിദ്യയാത്മതത്ത്വസ്യ സംബന്ധോ നോപപദ്യതേ\\
സംബന്ധ: സദൃശാനാം ച യ: സ്ഫുട: സ്വാനുഭൂതിത: (3/121/33)\\
}

വസിഷ്ഠൻ തുടർന്നു: ലവണരാജൻ മോഹവിഭ്രാന്തമായ ബോധത്തിൽ ഒരു രാജകുമാരൻ ഗോത്രയുവതിയുമായി വിവാഹബന്ധത്തിലേർപ്പെടുന്ന കാഴ്ച്ച, തന്നിൽ പ്രതിഫലിച്ചു കണ്ടു. അത് അദ്ദേഹത്തിന്‌ സ്വന്തം അനുഭവമായിത്തോന്നി. ചിലപ്പോൾ ഒരുവൻ തന്റെ പൂർവ്വകാലചെയ്തികളെ പാടേ മറക്കുവാനിടയാകുന്നുണ്ടല്ലോ. അയാൾ അതതിന്റെ സമയത്ത് കർമ്മങ്ങൾക്കായി എത്ര സമയം ചിലവഴിച്ചതാണെന്നാലും ആ കാര്യങ്ങൾ അയാൾക്ക് സംഭവിച്ചതായി തോന്നുന്നേയില്ല. ഓർമ്മയിൽ ഈദൃശമായ അപഭ്രംശങ്ങൾ സ്വാഭാവീകമത്രേ. പണ്ടത്തെ കാര്യങ്ങൾ ഇപ്പോൾ സ്വപ്നം കാണുന്നപോലെ, ഗോത്രവർഗ്ഗവുമായി താൻ ബന്ധപ്പെട്ട കാര്യങ്ങൾ ലവണൻ തന്റെ മോഹദർശനത്തിൽ അനുഭവിച്ചു. വിന്ധ്യാചലത്തിന്റെ താഴ്വാരങ്ങളിലെ ഗോത്രവർഗ്ഗക്കാർക്ക് അവരുടെ മനസ്സിലും ലവണരാജാവിന്റെ ബോധമണ്ഡലത്തിലുദിച്ച ദർശനം അനുഭവവേദ്യമായതാവാം. ലവണരാജാവിന്റെയും ഗോത്രവർഗ്ഗക്കാരുടേയും മനസ്സുകൾ അന്യോന്യം അനുഭവങ്ങൾ സ്വായത്തമാക്കിയതുമാവാം. പലരും പറഞ്ഞുറപ്പിച്ചതുകൊണ്ടുമാത്രം സത്യമായി കണക്കാക്കപ്പെടുന്ന പ്രസ്താവനകളെന്നപോലെ മോഹവിഭ്രാന്തികൾ പലരും അനുഭവിക്കുന്നതുകൊണ്ട് യാഥാർത്ഥ്യമാണെന്ന പ്രതീതിയുണ്ടാവുന്നു. ഈ അനുഭവങ്ങൾ ഒരാളുടെ ജീവിതത്തിൽ സ്വാംശീകരിച്ചുകഴിഞ്ഞാൽ അവയ്ക്ക് സ്വന്തമായൊരു യാഥാർത്ഥ്യഭാവം കൽപ്പിക്കപ്പെടുന്നു. ഒരാളുടെ ബോധമണ്ഡലത്തിലെ അനുഭവം തന്നെയാണല്ലോ ഇഹലോകവസ്തുക്കളെ സംബന്ധിച്ച സത്യം.

എത്ര ആട്ടിയാലും മണലിൽ നിന്നും എണ്ണകിട്ടാനിടയില്ലാത്തതുപോലെ അവിദ്യ എന്നത് ഉണ്മയാവുക അസാദ്ധ്യം. അവിദ്യയും ഉണ്മയും തമ്മിൽ ഒരു ബന്ധവും ഉണ്ടാവുക വയ്യ. കാരണം ഒന്ന് സത്തും മറ്റേത് അസത്തുമാണ്‌.. ഒരേ തരം വസ്തുക്കൾ തമ്മിലേ ബന്ധം സാദ്ധ്യമാവൂ എന്നതാണ്‌ നമുക്കനുഭവം. ബോധം അനന്തമായതിനാലാണ്‌ ലോകത്തുള്ള എല്ലാം ബോധത്തിന്‌ അറിയാനാവുന്നത്. വിഷയി (ഭോക്താവ്) സ്വയംപ്രഭമല്ലാത്ത വിഷയത്തെ (വസ്തുവിനെ)പ്രകാശമാനമാക്കുന്നു എന്നതു ശരിയല്ല. കാരണം ബോധം സർവ്വവ്യാപിയായതിനാൽ എല്ലാം സ്വയംപ്രഭമത്രേ. അതിന്‌ പ്രത്യേകിച്ചൊരു പ്രജ്ഞാശക്തി ആവശ്യമില്ല. ബോധത്തിൽ അവബോധമുണ്ടാവുമ്പോൾ പ്രജ്ഞാശക്തി സ്വയം അങ്കുരിക്കുകയാണ്‌.. അത് ബോധം ജഢവസ്തുവിനെ 'കാണുന്നതു'കൊണ്ടുണ്ടാവുന്നതല്ല.

വിശ്വമെന്നത് ചൈതന്യവസ്തുവും ജഢവും ചേർന്ന മിശ്രിതമാണെന്നു പറഞ്ഞാൽ അതു ശരിയല്ല. കാരണം അവ തമ്മിൽ ചേരുകയില്ല. എല്ലാ വസ്തുക്കളിലും ബോധം നിറഞ്ഞുതിങ്ങി നിൽക്കുന്നതുകൊണ്ട് ഈ ബോധത്തിൽ സ്വയം അവബോധമുണ്ടാകുന്നതാണ്‌ അറിവ്. ജഢങ്ങളാണെങ്കിലും പാറക്കല്ലും മരവും തമ്മിൽ ബന്ധമുണ്ടെന്ന് ഒരാൾക്കു തോന്നിയേക്കാം. എന്നാൽ അവകളിലെ അടിസ്ഥാനഘടകങ്ങളിലാണ്‌ അത്തരം ബന്ധങ്ങളുള്ളത്. ആ ഘടകങ്ങളിൽ സംഗതമായ ചില വ്യതിയാനങ്ങളാണ്‌ അവയെ യഥാക്രമം കല്ലും മരവും ആക്കി മാറ്റിയത്. ഇത്തരം സമാനതകൾ നാവിന്റെ രുചിഭേദങ്ങളിലും മറ്റും നമുക്ക് കാണാനാവും. 

