\section{ദിവസം 249}

\slokam{
വാച്യവാചകദൃഷ്ട്യൈവ  ഭേദോ യോഽയമിഹാവയോ:\\
അസത്യാ കല്‍പ്പനൈവൈഷാ വീചിവീച്യംഭസോരിവ (5/36/8)\\
}

പ്രഹ്ലാദന്റെ ധ്യാനം തുടര്‍ന്നു: അങ്ങിനെ അവസാനം എല്ലാ അവസ്ഥകള്‍ക്കും ബോധമണ്ഡലങ്ങള്‍ക്കും അതീതമായുള്ള ആത്മാവിനെ ഞാന്‍ സാക്ഷാത്ക്കരിച്ചിരിക്കുന്നു. അല്ലയോ ആത്മാവേ എന്റെ സൌഭാഗ്യം മൂലം നിന്നെ  കണ്ടെത്തിയിരിക്കുന്നു. നിനക്ക് നമസ്കാരം. നമോവാകം. നിന്നെ ഞാനൊന്ന് മാറോടണയ്ക്കട്ടെ. നീയല്ലാതെ മൂന്നുലോകങ്ങളിലും ആരെനിക്കു സുഹൃത്തും ബന്ധുവുമായുണ്ട്? എല്ലാത്തിനെയും നശിപ്പിക്കുന്നത് നീയാണ്. എല്ലാം നല്‍കുന്നതും സംരക്ഷിക്കുന്നതും ഉയര്‍ത്തുന്നതും ചലിക്കുന്നതുമെല്ലാം നീയാണ്. നിന്നെ കണ്ടെത്തിയിരിക്കുന്നു; പ്രാപിച്ചിരിക്കുന്നു. ഇനി നീയെന്തുചെയ്യാന്‍? ഇനി നീയെന്നെവിട്ടെങ്ങു പോവാന്‍?

ലോകമായി കാണപ്പെടുന്നതെല്ലാം നീയാണ്. നിന്റെ സത്തയാണ് ലോകം മുഴുവന്‍ വ്യാപരിച്ചിരിക്കുന്നത്. അല്ലയോ ആത്മാവേ ഇനി നീയെവിടെപ്പൊയൊളിക്കാന്‍? അനാദികാലം മുതലേ എനിക്കും നിനക്കുമിടയില്‍  അജ്ഞാനത്തിന്റെ വലിയൊരു മതില്‍ക്കെട്ടുണ്ടായിരുന്നു. ഇപ്പോളിതാ ആ ചുവരിടിഞ്ഞു വീണു. നാം തമ്മില്‍ അകലമേതുമില്ല എന്നറിവായി. നേടാന്‍ യോഗ്യമായുള്ളതെല്ലാം ഞാന്‍  നേടിക്കഴിഞ്ഞു. എല്ലാ കര്‍മ്മത്തിന്റെയും ശരിയായ  കര്‍ത്താവും ഭോക്താവുമായ ആത്മാവിനു നമസ്കാരം. ആത്മാവ് തന്നെയാണ് നിത്യശുദ്ധമായ ഈശ്വരന്‍. ബ്രഹ്മാവിഷ്ണുമഹേശ്വരന്മാര്‍ക്ക് എന്റെ നമസ്കാരം.

“അല്ലയോ ആത്മാവേ, നീയും ഞാനും തമ്മിലുള്ള അന്തരം കേവലം വാക്കുകളില്‍ മാത്രമല്ലേ? വാക്കും അതുണ്ടാക്കുന്ന അര്‍ത്ഥവും തമ്മിലുള്ളതുപോലെ, തിരയും തിരയിലെ ജലവും തമ്മിലുള്ളതുപോലെ നാം തമ്മിലുള്ള അന്തരം വെറും കല്‍പ്പനയാണ്; അസത്യമാണ്.”

ദൃഷ്ടാവായും അനുഭവഭോക്താവായും നിലകൊള്ളുന്ന ആ ഏകസത്തയ്ക്ക്  നമസ്കാരം. പ്രപഞ്ചത്തിലെ എല്ലാ പദാര്‍ത്ഥങ്ങളേയും സൃഷ്ടിച്ച് അവയായിത്തന്നെ വിടര്‍ന്നു വികസിച്ചു നിലകൊള്ളുന്ന ആത്മാവിനെന്റെ നമോവാകം. സര്‍വ്വവ്യാപിയും എല്ലാത്തിന്റെയും അന്തര്യാമിയുമായ   ആത്മാവിനെന്റെ നമസ്കാരം.

ശരീരങ്ങളും പദാര്‍ത്ഥങ്ങളുമായി മൂര്‍ത്തീകരിക്കുകയും അവയുമായി താദാത്മ്യം പുലര്‍ത്തുകയും ചെയ്യുക നിമിത്തം ആത്മാവെന്ന നീ നിന്റെ സ്വരൂപത്തെയും ഭാവത്തെയും മറന്നുവെന്നപോലെയായിരുന്നു. കഷ്ടം! അതിനാല്‍ നിനക്ക് ആത്മജ്ഞാനമില്ലാതെ അന്തമില്ലാത്ത ദുരിതാനുഭവങ്ങളും ബാഹ്യധാരണകളും ജന്മജന്മാന്തരങ്ങളായി സഹിക്കേണ്ടതായി വന്നു. ബാഹ്യമായി കാണപ്പെടുന്ന ഈ ലോകമെന്നത് വെറും കല്ലും മണ്ണും മരവും മാത്രം. നീയല്ലാതെ അവയിലൊന്നും യാതൊരു സത്തയുമില്ല.

ആത്മജ്ഞാനമുണ്ടായാല്‍പ്പിന്നെ മറ്റെന്തിനെയാണ് ആഗ്രഹിക്കുക? ഭഗവാനേ അങ്ങയെ ഞാന്‍ കണ്ടെത്തിയിരിക്കുന്നു. ആ സവിധത്തില്‍ ഞാനെത്തിയിരിക്കുന്നു. ഇനി എനിക്ക് വിഭ്രാന്തിയില്ല. നിനക്ക് നമോവാകം. ഭഗവാനേ ഈ ആത്മാവ് കണ്ണിന്റെ കണ്ണാണ്. അതീ ദേഹം മുഴുവന്‍ വ്യാപരിച്ചിരിക്കുന്ന  മേധാശക്തിയായിട്ടുമെന്തേ എനിക്കു ദൃശ്യമാകുന്നില്ല? അനുഭവമാകുന്നില്ല?

സ്പര്‍ശനാനുഭവം മുതലായ ഇന്ദ്രിയാനുഭവങ്ങളെയെല്ലാം, വസ്തുക്കളെ അറിയുന്ന ആ മേധാശക്തിക്ക് എന്തുകൊണ്ടാണ് തിരിച്ചറിയാനാകാത്തത്? ഒരാളില്‍ നിന്നും അതിനു ദൂരെപ്പോകാന്‍ കഴിയുമോ? അതുതന്നെയല്ലേ ഗ്രഹണശക്തിയും ഘ്രാണശക്തിയുമെല്ലാമായി നമ്മില്‍ രോമാഞ്ചമുണ്ടാക്കുന്നത്? മറ്റുള്ള വസ്തുക്കളുടെ മധുരിമയുടെ സ്വാദനുഭവിച്ചറിയുന്ന അതിന് എന്തുകൊണ്ടാണ് സ്വന്തം മേധാശക്തിയുടെ മാധുര്യത്തെ അറിയാന്‍ കഴിയാത്തത്? അതുപോലെ അതിന്റെ മണം സ്വന്തമായി അനുഭവിക്കാന്‍ കഴിയാത്തതെന്തുകൊണ്ടാണ്?

വേദശാസ്ത്രങ്ങള്‍ പ്രകീര്‍ത്തിക്കുന്ന അതിന്റെ മഹിമ അതു സ്വയം മറന്നത് എന്തുകൊണ്ടാണ്? അല്ലയോ ആത്മാവേ നിന്നെ സാക്ഷാത്കരിച്ചിരിക്കുന്നു. ഇതുവരെ ഞാന്‍ ആഹ്ലാദിച്ചനുഭവിച്ചിരുന്ന ഇന്ദ്രിയസുഖങ്ങളില്‍ ഇപ്പോളെനിക്ക് യാതൊരു താല്‍പ്പര്യവുമില്ല.