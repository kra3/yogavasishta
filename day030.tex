\newpage
\section{ദിവസം 030}

\slokam{
സ്ഥിതോ പി സ്ഥിത ഇവ ന ഹൃഷ്യതി ന കുപ്യതി \\
യ: സുഷുപ്തസമ: സ്വസ്ഥ: സ ശാന്ത ഇതി കഥ്യതേ (2/13/76)\\
}

വസിഷ്ഠന്‍ തുടര്‍ന്നു: സംസാരസാഗരമാകുന്ന തുടര്‍ക്കഥയുടെ ഉള്ളില്‍ നിന്നും പുറത്തുകടക്കണമെങ്കില്‍ ഒരുവന്‍ മാറ്റങ്ങള്‍ ക്കു വിധേയമാക്കാത്ത സത്യവസ്തുവിനെ ആശ്രയിക്കണം. ശാശ്വതമായ ആ 'ഒന്നില്‍' മനസ്സുറച്ചവന്‍ മാത്രമാണ്‌ ഉത്തമന്‍. കാരണം അവന്‌ ആത്മനിയന്ത്രണം സ്വായത്തമായതുകൊണ്ട്‌ മനസ്സ്‌ പ്രശാന്തമാണ്‌. സുഖദു:ഖങ്ങള്‍ പരസ്പരം പോരാടി അവസാനം ഒരു സമതുലിതാവസ്ഥയെ പ്രാപിക്കുന്നു എന്ന അറിവിന്റെ നിറവിലാണ്‌ അവനില്‍ ആത്മസംയമനം സംഭവിക്കുന്നത്‌. ഈ സത്യം അറിയാത്തവന്‍ തീപിടിച്ച വീട്ടിലുറങ്ങുന്നവനേപ്പോലെ വ്യാകുലചിത്തനാണ്‌. ഈ ജീവിതവേളയില്‍ത്തന്നെ ശാശ്വതമായതെന്തെന്ന അറിവുറച്ചവന്‌ സംസാരത്തില്‍ നിന്നും മുക്തിയായി. അവന്‌ അവിദ്യയാല്‍ പുനര്‍ജന്മങ്ങളെടുക്കേണ്ടി വരികയില്ല. ചിലപ്പോള്‍ ഇപ്പറഞ്ഞ ശാശ്വതസത്യം എന്നത്‌ ഉള്ളതാണോ എന്ന സംശയം ഉണ്ടായേക്കാം. അപ്പോഴും അന്വേഷണം നിര്‍ത്തരുത്‌. കാരണം ജീവിതത്തിന്റെ സ്വഭാവത്തെപ്പറ്റിയുള്ള സമഗ്രമായ അന്വേഷണം ദു:ഖങ്ങളെ നേരിടാനും വേദനകളുടെ തീവ്രതയെ കുറയ്ക്കാനും നമ്മെ പര്യാപ്തമാക്കും. എന്നാല്‍ ശാശ്വതസത്യമെന്നത്‌ ഉണ്മയാണെങ്കിലോ, പരമശാന്തിയും ബന്ധനവിമുക്തിയുമാണ്‌ ഫലം.

ശാശ്വതമായ സത്യം കണ്ടെത്താന്‍ യാഗകര്‍മ്മാദികളോ തീര്‍ത്ഥാടനങ്ങളോ ധനമോ കൊണ്ടു സാധിക്കയില്ല.ശരിയായ അറിവിന്റെ വെളിച്ചത്തില്‍ പക്വപ്പെടുന്ന മനോനിയന്ത്രണം കൊണ്ടുമാത്രമേ അതു സാദ്ധ്യമാവൂ. മനുഷ്യരും, ദേവന്മാരും, ഉപദേവതകളും എന്നുവേണ്ട എല്ലാവരും എല്ലായ്പ്പോഴും -കിടക്കുമ്പോഴും ഇരിക്കുമ്പോഴും വീഴുമ്പോഴും- നിരന്തരം, മനസ്സിനെയടക്കാനുള്ള, ആത്മനിയന്ത്രണത്തിനുള്ള വഴികളിലൂടെ മാത്രമേ സഞ്ചരിക്കാവൂ. അറിവിന്റെ സഫലതയാണു മനോനിയന്ത്രണം. മനസ്സ്‌ പ്രശാന്തവും, നിര്‍മ്മലവും, മോഹവിഭ്രാന്തിയാല്‍ ബാധിക്കപ്പെടാതെയും, അനാസക്തവും ആകുമ്പോള്‍ അതിന്‌ മറ്റൊരു വസ്തുവും കൈവശപ്പെടുത്തുകയോ ഉപേക്ഷിക്കുകയോ ചെയ്യേണ്ടതില്ല. ഇതാണ്‌ ഞാന്‍ മുന്‍പു പറഞ്ഞ നാലു ദ്വാരപാലകരില്‍ ഒരാളായ 'ആത്മനിയന്ത്രണം'.

ശുഭോദര്‍ക്കമായതും നന്മയുള്ളതുമായ എല്ലാം ആത്മനിയന്ത്രണത്തിന്റെ പരിണിതഫലമത്രേ. തിന്മകളേയും ദുഷ്ടതയേയും അതില്ലാതാക്കുന്നു. യാതൊരുനേട്ടങ്ങളും- സ്വര്‍ഗ്ഗം പോലും ആത്മനിയന്ത്രണപ്രഹര്‍ഷത്തിനൊപ്പം വരില്ല. അതിനെ താരതമ്യപ്പെടുത്താന്‍ മറ്റൊരു ആനന്ദാവസ്ഥയുമില്ല. അപ്രകാരമുള്ള ഒരാള്‍ എല്ലാവര്‍ക്കും വിശ്വാസപാത്രമാണ്‌. അസുരവര്‍ഗ്ഗവും പിശാചുക്കള്‍ പോലും അവനെ വെറുക്കുന്നില്ല. രാമ:, ആത്മനിയന്ത്രണമാണ്‌ ശാരീരികവും മാനസീകവുമായ എല്ലാ വേദനകള്‍ക്കുമുള്ള ഒറ്റമൂലി. അത്മനിയന്ത്രണമുള്ളവന്‌ കിട്ടുന്ന ഭക്ഷണമെല്ലാം കയ്പ്പുണ്ടെങ്കില്‍ പ്പോലും സ്വാദുറ്റതാവുന്നു. ഈ കവചമണിഞ്ഞവന്‌ ദു:ഖത്തിന്റെ ആക്രമണം ഏശുകയില്ല. 

ഏതൊരുവന്‍ പ്രിയമോ അപ്രിയമോ ആയി കണക്കാക്കപ്പെടുന്ന വസ്തുക്കള്‍ അമിതാഹ്ലാദമോ ദു:ഖമോ ഇല്ലാതെ കേള്‍ ക്കുകയും, കാണുകയും, തൊടുകയും മണക്കുകയും, സ്വാദുനോക്കുകയും ചെയ്യുന്നവോ അവന്‍ ആത്മനിയന്ത്രണം വന്നവനാണ്‌. അവന്‍ എല്ലാറ്റിനേയും സമദൃഷ്ടിയോടെ കാണുന്നു. സുഖദു:ഖങ്ങളെ ഇന്ദ്രിയനിയന്ത്രണം കൊണ്ട്‌ വരുതിയിലാക്കുന്നു. "അഗാധനിദ്രയില്‍ ഒരുവന്‌ ആരോടും വെറുപ്പോ അഭിനിവേശമോ ഇല്ലാത്തതുപോലെ ഒപ്പം ഇടപഴകി ജീവിക്കുമ്പോഴും ആത്മനിയന്ത്രണമുള്ളവനെ മറ്റുള്ളവരുടെ പ്രവര്‍ത്തനങ്ങള്‍ ബാധിക്കുന്നില്ല."
