\section{ദിവസം 236}

\slokam{
യേഷു യേഷു പ്രദേശേഷു മനോ മജ്ജതി ബാലവത്\\
തേഭ്യസ്തേഭ്യ: സമാഹൃത്യ തദ്ധി തത്ത്വേ നിയോജയേത് (5/29/54)\\
}

വസിഷ്ഠന്‍ തുടര്‍ന്നു: ബലി മഹാരാജാവ് തുടര്‍ന്നും ഭംഗിയായി രാജ്യം ഭരിച്ചു. യാതൊന്നും മുന്‍കൂട്ടി തീരുമാനിച്ച് ഉറപ്പിക്കാതെ തന്നെ യദൃച്ഛയാ വരുന്ന അവസരങ്ങള്‍ക്കനുസൃതമായി അദ്ദേഹം ഭരണകാര്യങ്ങള്‍ നിര്‍വ്വഹിച്ചു. ബ്രാഹ്മണരെയും, ദേവതകളെയും മഹാത്മാക്കളേയും അദ്ദേഹം വന്ദിച്ചു പൂജിച്ചു വന്നു.  ബന്ധുജനങ്ങളോട് അദ്ദേഹം ആദരപൂര്‍വ്വം പെരുമാറി. ഭ്രുത്യജനങ്ങള്‍ക്ക് സമ്മാനങ്ങള്‍ കയ്യയച്ച് നല്‍കി. ആവശ്യക്കാര്‍ക്ക് വേണ്ടി  പ്രതീക്ഷിക്കാവുന്നതിലേറെ  ദാനധര്‍മങ്ങള്‍ ചെയ്തു. വനിതകളുമായും അദ്ദേഹം യഥേഷ്ടം കേളിയാടി. ഒന്നും അദ്ദേഹത്തിനു വര്‍ജ്ജ്യമായിരുന്നില്ല.

അങ്ങിനെയിരിക്കെ അതിമഹത്തായ ഒരു യാഗം വിപുലമായിത്തന്നെ  നടത്താനും അങ്ങിനെ പുകള്‍ നേടാനും അദ്ദേഹത്തിനു കലശലായ ആഗ്രഹം തോന്നി. അതിനുവേണ്ട സാമഗ്രികളും ആളുകളും ഉടനെ തയാറായി. യഥോചിതമായി അദ്ദേഹം ആ യാഗകര്‍മ്മം തുടങ്ങി വച്ചു. ഈ യാഗസമയത്താണ് ഭഗവാന്‍ വിഷ്ണു ബലിയുടെ പക്കല്‍ നിന്നും മൂന്നു ലോകങ്ങളുടെയും ഭരണാധികാരം മാറ്റി അത് ഇന്ദ്രന് നല്‍കാനായി വാമനരൂപം പൂണ്ട് അവതരിച്ചത്. ബലിയുടെ പക്കല്‍ നിന്നും മൂന്നു ലോകങ്ങളെയും ദാനമായി വാങ്ങിയാണ് വിഷ്ണു ഇത് സാധിച്ചത്.  രാമാ, ഈ ബലിക്കാണ്  അടുത്ത ഇന്ദ്രപദവി. വിഷ്ണുഭഗവാന്‍ ദാനം കൊടുത്ത് നിസ്വനായ ബലിയെ പാതാളലോകമായ സുതല്‌ത്തിലേയ്ക്ക് അയക്കുകയാണുണ്ടായത്. അതിനാലാണ് അദ്ദേഹം അവിടെ മുക്തനും പ്രബുദ്ധനുമായി സ്വര്‍ഗ്ഗഭരണമേറ്റെടുക്കാനുള്ള   തന്റെ ഊഴവും കാത്തിരിക്കുന്നത്. ഐശ്വര്യമോ ആപത്തോ തന്നെ സന്ദര്‍ശിക്കുന്നത് എന്ന വിവേചനം അദ്ദേഹത്തിനില്ല. സന്തോഷ-സന്താപ അനുഭവങ്ങളില്‍ അദ്ദേഹം അമിതമായി  ആനന്ദിക്കുകയോ വിലപിക്കുകയോ ചെയ്യുന്നില്ല. അനേക കോടി വര്‍ഷങ്ങള്‍ അദ്ദേഹം മൂന്നു ലോകങ്ങളും ഭരിച്ചു. എന്നാലിപ്പോള്‍ അദ്ദേഹത്തിന്‍റെ ഹൃദയം വിശ്രമിക്കുകയാണ്. ഇനിയും അദ്ദേഹം ഇന്ദ്രപദവിയില്‍ ഏറെക്കാലം മൂലോകങ്ങളും ഭരിക്കും. എന്നാലദ്ദേഹത്തിനു ഇന്ദ്രപദവിയില്‍ താല്‍പ്പര്യമൊന്നുമില്ല. രാജപദവി നഷ്ടപ്പെട്ടപ്പോഴും പാതാളത്തിലേയ്ക്ക് കൊണ്ടുപോയപ്പോഴും അദ്ദേഹം വിലപിച്ചില്ല. യാദൃഛയാ വന്നു ചേരുന്നതിനെ സര്‍വ്വാത്മനാ സ്വീകരിച്ച് സ്വയം പ്രശാന്തയില്‍ അദ്ദേഹം അഭിരമിക്കുന്നു.

ഇതാണ് ബലി മഹാരാജാവിന്റെ കഥ. രാമാ, അദ്ദേഹത്തിനുണ്ടായിരുന്ന പോലെയുള്ള ഉള്‍ക്കാഴ്ചയുമായി പരമപദത്തില്‍ നീയും അഭിരമിക്കൂ. ഉപയോഗശൂന്യവും അസത്തുമായ ലൌകീകസുഖങ്ങളില്‍ ആസക്തനാകാതിരിക്കൂ. ദൂരെക്കാണുന്ന പാറക്കല്ലുകള്‍പോലെ മാത്രമേ ഈ ലോകത്തുള്ള ആകര്‍ഷണവസ്തുക്കള്‍ക്ക് നിന്റെയുള്ളില്‍ സ്ഥാനമുണ്ടാവാന്‍ പാടുള്ളൂ. അവ നിന്റെ ശ്രദ്ധയോ ആദരവോ അര്‍ഹിക്കുന്നില്ല. ഒന്നില്‍നിന്ന് മറ്റൊന്നിലേയ്ക്ക് ചാഞ്ചാടുന്ന നിന്റെ മനസ്സ് ദൃഢമായി ഹൃദയത്തില്‍ ഉറപ്പിച്ചാലും. രാമാ, നീ അനന്താവബോധത്തിന്റെ നിതാന്തഭാസുരതയാണ്. ലോകങ്ങള്‍ നിന്നില്‍ വേരുറപ്പിച്ചിരിക്കുന്നു. നിനക്ക് സുഹൃത്തായും  ശത്രുവായും ആരുണ്ട്? നീ അനന്തതയാണ്. മാലയില്‍ കോര്‍ത്ത മണികള്‍ പോലെയാണ് നിന്നില്‍ ലോകങ്ങള്‍ നിലകൊള്ളുന്നത്. നീയാകുന്ന ആ വ്യതിരിക്ത സത്വം ജനിച്ചിട്ടില്ല, മരിക്കുകയുമില്ല. ആത്മാവാണുണ്മ. ജനനമരണങ്ങള്‍ വെറും ഭാവനകള്‍ മാത്രം. വെറും മിഥ്യ. ജീവിതത്തിലെ എല്ലാ ദുരിതങ്ങളേയും കുറിച്ച് ആരായുക. എന്നാല്‍ ഒന്നിലും ഒട്ടാതിരിക്കുക. നീയാണ് രാമാ ഭഗവാനും ഭഗവദ്പ്രഭയും. ലോകം ആ പ്രഭയിലാണ് ദര്‍ശിതമാവുന്നത്. ലോകത്തിന്, വേറിട്ട്  സ്വതന്ത്രമായി നില്‍ക്കുന്ന  ഒരസ്തിത്വമില്ല. പണ്ട് നിന്നില്‍ ഇഷ്ടാനിഷ്ടങ്ങളുടെ ദ്വന്ദശക്തികള്‍ വര്‍ത്തിക്കുന്നുണ്ടായിരുന്നുവല്ലോ. അവയെ ഉപേക്ഷിച്ചാല്‍ നിനക്ക് സമതാ ഭാവം കൈക്കൊള്ളാനാവും. അതോടെ ജനന മരണ ചക്രത്തിന്റെ അവസാനവുമാവും.

“മനസ്സ് ആഴ്ന്നിറങ്ങാന്‍ സാദ്ധ്യതയുള്ള എല്ലാത്തില്‍ നിന്നും അതിനെ പിന്‍വലിച്ച് സത്യത്തിലേയ്ക്ക് ഉന്മുഖമാക്കുക.” സത്യത്തിന്റെ നേരറിവില്ലാത്തവരും സ്വയം ഗുരുക്കളായി അവരോധം ചെയ്തവരുമായവരുടെ വാഗ്ധോരണികളില്‍ നീ വീണുപോകാതെ സൂക്ഷിക്കണം. പക്ഷേ എന്റെയീ പ്രഭാഷണം നിന്നെ നേര്‍വഴിക്കു നയിച്ച് നിന്നെ ആത്മസാക്ഷാത്കാരത്തിനു പാത്രമാക്കും എന്ന് നിനക്ക് ഞാന്‍ ഉറപ്പ് തരുന്നു.

