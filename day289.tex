\section{ദിവസം 289}

\slokam{
അന്ത: സംസംഗമംഗാനാമംഗാരം വിദ്ധി രാഘവ\\
അനന്ത: സംഗമംഗാനാം വിദ്ധി രാമാ രസായനം (5/68/50)\\
}

വസിഷ്ഠന്‍ തുടര്‍ന്നു: ശംഖചക്രഗദാധാരിയായ ഭഗവാനാണ് മുമ്പ് പറഞ്ഞ വന്ദ്യമായ ഉപാധികള്‍ മൂലം മൂന്നു ലോകങ്ങളെയും പരിപാലിക്കുന്നത്.  അതേ പവിത്ര ഉപാധികളാലാണ് സൂര്യന്‍ ജ്വലിക്കുന്നതും വിപുലമായ ഈ പ്രപഞ്ചസൃഷ്ടികളെ ബ്രഹ്മാവ്‌ സംവിധാനം ചെയ്തൊരുക്കുന്നതും. പരമശിവനും ഈ ഉപാധികളുള്ളതുകൊണ്ടാണ് തന്റെ ദൈവീകത പ്രകടമാക്കുന്നത്. സ്വയം പരിമിതപ്പെടുത്തുന്ന (അതത് വകുപ്പുകളില്‍ മാത്രം) വന്ദ്യോപാധികളാണ് എല്ലാ ദേവതകളേയും പ്രവര്‍ത്തനസജ്ജമാക്കുന്നത്. എന്നാല്‍ 'വന്ധ്യോപാധികള്‍ '  (വന്ധ്യമായ, അതായത് വ്യര്‍ത്ഥമായ ഉപാധികള്‍ ) മനസ്സിനെ സുഖാനുഭവങ്ങള്‍ക്കായി പ്രേരിപ്പിക്കുന്നു. അവ അത്തരം അനുഭവങ്ങള്‍ ആനന്ദപ്രദമാണെന്ന പ്രതീതി ജനിപ്പിച്ചു ജീവനെ മോഹിപ്പിക്കുകയും ചെയ്യുന്നു. 

അണ്ഡകടാഹത്തിലെ ഗ്രഹാദികള്‍ പോലും ഉപാധികളുടെ ഭാഗമായാണുണ്ടായി പ്രവര്‍ത്തിക്കുന്നത്. അവയാണ് സ്വര്‍ഗ്ഗവാസികളായ ദേവന്മാരും, ഭൂവാസികളായ മനുഷ്യരും, പാതാളവാസികളായ അസുരന്മാരും ആയി സമുദ്രത്തിലെ അലകള്‍ പോലെ പൊങ്ങിയും താണും ഉണ്ടായി മറയുന്നത്. കടലിലെ വലിയ ജീവികള്‍ക്ക് ചെറുജീവികളാണാഹാരം. എണ്ണമറ്റ ജീവജാലങ്ങള്‍ പരസ്പരം തിന്നു വിശപ്പടക്കി ജീവിക്കുന്നതും അങ്ങോട്ടും ഇങ്ങോട്ടും തട്ടിക്കളിക്കുന്നതും ഈ ഉപാധികള്‍ മൂലമാണ്. ഇടയ്ക്ക് തിളങ്ങിയും ഇടയ്ക്ക് മങ്ങിയും; ഇടയ്ക്ക് ജ്വലിച്ചും ഇടയ്ക്ക് ഇരുട്ടിലാണ്ടും ചന്ദ്രന്‍ ഭൂമിയെ ചുറ്റുന്നു. (ചന്ദ്രനിലെ കളങ്കം ഓര്‍മ്മിക്കുക) ഈ ജോലി മുടങ്ങാതെ നടക്കുന്നതും ഉപാധികളാലാണ്. 

രാമാ, വിസ്മയമാര്‍ന്ന ഈ കാഴ്ച കണ്ടാലും, മനസ്സിലെ ധാരണാ സങ്കല്‍പ്പങ്ങള്‍ക്കനുസൃതമായി ഈ കാഴ്ചകളിവിടെ ഉണ്ടാക്കിയതാരാണാവോ? മനസാ ഉണ്ടായ സങ്കല്‍പ്പങ്ങള്‍ക്കനുസരിച്ച്, ഉപാധികള്‍ക്കനുസരിച്ച്, ശൂന്യാകാശത്ത് പ്രപഞ്ചമുണ്ടാക്കിയാത്രെ! അതെങ്ങിനെ യഥാര്‍ത്ഥത്തില്‍ ഉള്ളതാവും? അത് മിഥ്യ മാത്രമാവാനേ തരമുള്ളൂ. ലോകവുമായി, ദേഹവുമായി സംഗത്തില്‍ നില്‍ക്കുന്ന എല്ലാ ജീവജാലങ്ങളുടെയും സുഖാന്വേഷണത്വര അവരെ കാര്‍ന്നു തിന്നുകൊണ്ടിരിക്കുന്നു. അവയുടെ എണ്ണമാണെങ്കില്‍ കടല്‍ത്തീരത്തെ മണല്‍ത്തരികളേപോലെ എണ്ണിയാല്‍ ഒടുങ്ങാത്തതാണ്.

ഈ ജീവികളിലെ മനോപാധികളുടെ പ്രതികരണമാണ് ബ്രഹ്മാവിന്റെ സൃഷ്ടിയായി നാം കാണുന്ന ഈ പ്രപഞ്ചം. ഈ ജീവജാലങ്ങള്‍ ഇവിടെ, ഇപ്പോള്‍ത്തന്നെ നരകത്തീയെരിക്കാനുതകുന്ന ഉണങ്ങിയ വിറകുകള്‍ തന്നെയാണ്. ലോകത്തുള്ള സകല ദു:ഖങ്ങളും ഈ ജീവികള്‍ക്കായി മാത്രമുള്ളതാണ്. നദികള്‍ സ്വാഭാവികമായും സമുദ്രത്തിലേയ്ക്ക് കുതിച്ചു പായുന്നതുപോലെ ദുരിതങ്ങള്‍ മനോപാധികള്‍ക്ക് വശംവദരായ ജീവികളെ തേടിച്ചെല്ലുന്നു. വെറും മിഥ്യയായ ഈ  സൃഷ്ടികള്‍ മുഴുവനും അജ്ഞാനത്താല്‍ ആവരണം ചെയ്യപ്പെട്ടിരിക്കുന്നു.

എന്നാല്‍ സുഖാന്വേഷണത്വരയെ വേരോടെ അറുക്കാന്‍ കഴിഞ്ഞാല്‍ മാനസീകോപാധികളാകുന്ന  'പരിമിതികള്‍ ' പവിത്രമായിത്തീര്‍ന്ന് ആത്മവികാസത്തിന് വഴിതെളിക്കും. “സോപാധികമായ മനസില്‍ ഉയരുന്ന പരിമിതമായതിനോടുള്ള ആസക്തി, വേദനാജനകമാണ് രാമാ. എന്നാല്‍ അനന്തതയിലേയ്ക്കുള്ള വികാസം അല്ലെങ്കില്‍ അനന്തതയോടുള്ള ഭക്തി ഈ വേദനയെ മാറ്റുന്ന ദിവ്യൌഷധമാണ്.” യാതൊന്നിനോടും ആസക്തിയില്ലാത്ത മനസ്സ് അനന്തതയുടെ പ്രശാന്തതയില്‍ അഭിരമിക്കുമ്പോള്‍ ആനന്ദം സിദ്ധമാകുന്നു.

ആത്മജ്ഞാനത്തില്‍ ദൃഢീകരിച്ചവന്‍ ഇപ്പോള്‍ , ഇവിടെവച്ച് തന്നെ മുക്തനാണ്. ജീവന്മുക്തന്‍ . 

(സംഗം, സംസംഗം, ഉപാധി, ബന്ധം, ആസക്തി, തതാത്മ്യഭാവം, എന്നിവയെല്ലാം അനന്തതയെ പരിമിതപ്പെടുത്തുന്നവയാണ്. ആത്മജ്ഞാനം പരിമിതികള്‍ക്കതീതമാണ്) 

