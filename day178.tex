\section{ദിവസം 178}

\slokam{
യേന ശബ്ദം രസം രൂപം ഗന്ധം ജാനാസി രഘവ\\
സോയമാത്മാ പരം ബ്രഹ്മ സർവമാപൂര്യ സംസ്ഥിത: (4/37/7)\\
}

വസിഷ്ഠൻ തുടർന്നു: ഈ ലോകത്തിന്റെ പ്രത്യക്ഷപ്പെടലും ഇല്ലാതാവലും അനന്താവബോധത്തിന്റെ സഹജ സ്വഭാവം തന്നെയാണ്‌.. സ്വയം അനന്താവബോധത്തിൽ നിന്നും വിഭിന്നമല്ല എന്നുള്ളതുകൊണ്ട് ലോകത്തിന്‌ അതുമായി ഒരു കാര്യ-കാരണ ബന്ധമാണുള്ളത്. ലോകമുണ്ടാവുന്നത് അതിലാണ്‌.. നിലകൊള്ളുന്നത് അതിലാണ്‌. . വിലയിക്കുന്നതും അതിലാണ്‌.. ആഴമേറിയ സമുദ്രത്തേപ്പോലെ പ്രശാന്തമാണെങ്കിലും സമുദ്രോപരിയുള്ള തിരകൾ പോലെ അതിന്‌ വിക്ഷോഭമുള്ളതായി പുറമേ കാണപ്പെടുന്നു. ലഹരിക്കടിമപ്പെട്ട് ഒരാൾ സ്വയം മറ്റൊരാളാണെന്നു കരുതുമ്പോലെ അനന്താവബോധം സ്വയം  മറ്റെന്തോ ആണെന്നു തെറ്റിദ്ധരിക്കുകയാണ്‌.. ഈ വിശ്വം സത്തോ അസത്തോ അല്ല. അത് ബോധത്തിലാണു നിലകൊള്ളുന്നത് എന്നാൽ അതു സ്വതന്ത്രമായി ബോധത്തിൽ മറ്റൊരു  വസ്തുവായി നിലകൊള്ളുന്നുമില്ല. അനന്താവബോധത്തിൽ കൂട്ടിച്ചേർക്കപ്പെട്ടതാണിതെന്ന് തോന്നുമെങ്കിലും അത് ബോധത്തെ അതിശയിക്കുന്നില്ല. സ്വർണ്ണവും ആഭരണങ്ങളും തമ്മിലുള്ള ബന്ധം പോലെയാണിത്.

"രാമാ, ഈ ആത്മാവ്, അതായത് എല്ലായിടവും നിറഞ്ഞിരിക്കുന്ന പരബ്രഹ്മം, ഒന്നു മാത്രമാണ്‌ നിനക്ക് ശബ്ദ, രസ, രൂപ, ഗന്ധങ്ങളെ അനുഭവവേദ്യമാക്കുന്നത്." അതീന്ദ്രിയവും സർവ്വവ്യാപിയുമത്രേ അത്. അദ്വൈതവും പരിശുദ്ധവുമാണത്. അതിൽ ‘മറ്റൊന്ന്’ എന്നൊരു ധാരണപോലുമില്ല. സ്ഥിതിയും അതിന്റെ നിരാസവും, നന്മയും തിന്മയും, തുടങ്ങി എല്ലാ വിഭിന്നതകളും നാനാത്വങ്ങളും അജ്ഞാനികളുടെ സങ്കൽപ്പസൃഷ്ടികളത്രേ. ഈ സങ്കൽപ്പങ്ങൾ ആത്മാവിനെ സംബന്ധിച്ചതാണോ അനാത്മാവിനെ സംബന്ധിച്ചതാണോ എന്നുള്ളത് വിഷയമേയല്ല. കാരണം അത്മാവല്ലാതെ മറ്റൊന്നുമില്ലാത്തതുകൊണ്ട് മറ്റൊന്നിനായുള്ള ആശ എങ്ങിനെയുണ്ടാകാനാണ്‌? അതുകൊണ്ട് ആത്മാവിനെ സംബന്ധിച്ചിടത്തോളം, ‘ഇതഭികാമ്യം’, ഇതഭികാമ്യമല്ല‘ എന്നുള്ള തരം ഭേദചിന്തകൾക്കു സ്ഥാനമില്ല. ആത്മാവിന് ആശകളില്ല. കർമ്മങ്ങളിലേർപ്പെടുന്ന ഉപകരണവും (അല്ലെങ്കിൽ കർമ്മി) കർമ്മം തന്നെയും ആത്മാവ് തന്നെയാണ്. അദ്വൈതമാകയാൽ (രണ്ടല്ലാത്തത്) ആത്മാവ് കർമ്മത്തിൽ ഇടപെടുന്നില്ല. സത്തായി സ്ഥിതിചെയ്യുന്ന വസ്തുവും, ആ വസ്തു സ്ഥിതിചെയ്യുന്ന ഇടവും ഒന്നാകയാൽ ഇതും ശരിയായ ഒരു ധാരണയല്ല. ആഗ്രഹങ്ങൾ ലവലേശമില്ലാത്തതു കാരണം, ആത്മാവ് കർത്താവാണെന്നോ, കർമ്മരഹിതൻ ആണെന്നോ ഉള്ള ധാരണകൾക്കും സാധുതയില്ല.

രാമാ, നീയല്ലതെ മറ്റൊന്നല്ല, ഈ പരമ്പൊരുളിന്റെ അസ്തിത്വം. അതിനാൽ എല്ലാവിധ ദ്വന്ദചിന്തകളും അകറ്റി കർമ്മനിരതമായ ഒരു ജീവിതം നയിച്ചാലും. പലേവിധങ്ങളായ കർമ്മങ്ങൾ ആവർത്തിക്കുന്നതുകൊണ്ട് നീയെന്താണു നേടുന്നത്? അതുപോലെ കർമ്മങ്ങളുപേക്ഷിച്ചാലും നീയെന്താണു നേടുന്നത്? അതുമല്ല, വേദശാസ്ത്രങ്ങളെ അക്ഷരം പ്രതി പിന്തുടർന്നതുകൊണ്ട് എന്താണു നേട്ടം? രാമാ, കാറ്റടിച്ചു വിക്ഷുബ്ധമാകാത്ത, പ്രശാന്തമായ കടൽ പോലെയാവൂ. എല്ലാടവും നിറഞ്ഞുവിളങ്ങുന്ന ആത്മാവിനെ ’കിട്ടാൻ‘ അങ്ങുമിങ്ങും അലഞ്ഞിട്ടു കാര്യമില്ല. നിന്റെ മനസ്സിനെ ലൗകീക വിഷയങ്ങളിൽ അലയാൻ അനുവദിക്കാതിരിക്കുക എന്ന ഒരു മാര്‍ഗ്ഗമേയുള്ളൂ

നീ പരമാത്മാവ് തന്നെയാണ്‌.. അനന്താവബോധമാണ്‌ . നീ മറ്റൊന്നുമല്ല! 

