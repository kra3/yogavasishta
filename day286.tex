\section{ദിവസം 286}

\slokam{
ആശാ യാവദശേഷേണ ന ലൂനാശ്ചിത്തസംഭവ:\\
വീരുദ്ധോ ദാത്രകേണേവ താവന്ന: കുശലം കുത: (5/66/11)\\
}

വസിഷ്ഠന്‍ തുടര്‍ന്നു: രാമാ, സ്വന്തം മനസ്സുകൊണ്ട് തന്നെ അവനവന്റെ മനസ്സിനെ അടക്കാന്‍ കഴിഞ്ഞില്ലെങ്കില്‍ ആത്മജ്ഞാനം സിദ്ധിക്കുകയില്ല. ചിത്രപടത്തിലെ സൂര്യന് അസ്തമയമില്ലാത്തതുപോലെ ‘ഞാന്‍, എന്റെ’ തുടങ്ങിയ ധാരണകളും ദു:ഖങ്ങളും ഒഴിയാതെ ആത്മജ്ഞാനം ഉണ്ടാവുകയില്ല. ഇതിനെക്കുറിച്ച് ഞാന്‍ ഒരു കഥ പറയാം. മൂന്നു ലോകങ്ങളും ചേര്‍ന്നത്ര വലുപ്പത്തില്‍ വലിയൊരു പര്‍വ്വതമുണ്ടായിരുന്നു. അതിന്റെ മുകളില്‍ ദേവന്മാരും മദ്ധ്യഭാഗത്ത് മനുഷ്യരും താഴെ പാതാളവാസികളും വാണിരുന്നു. സഹ്യന്‍ എന്നാണതിന് പേര്. അതില്‍ എല്ലാം ഉണ്ടായിരുന്നു. അത്രിമുനിയുടെ ആശ്രമം അവിടെയായിരുന്നു. അവിടെ രണ്ടു മാമുനിമാര്‍ - ബൃഹസ്പതിയും ശുക്രനും അവരുടെ പുത്രന്മാരായ വിലാസനും ഭാസനുമൊപ്പം ജീവിച്ചുവന്നു.  പരസ്പരം വളരെ അടുപ്പത്തിലായിരുന്നു അവര്‍. . ഇണപിരിയാത്ത കൂട്ടുകാര്‍ .  രണ്ടു ബാലന്മാര്‍ക്കും താരുണ്യമായി. 

യഥാകാലം  ബൃഹസ്പതിയും ശുക്രനും ദിവംഗതരായി. ദു:ഖാര്‍ത്തരായ ഈ ചെറുപ്പക്കാര്‍ പിതാക്കന്മാരുടെ അന്ത്യകര്‍മ്മങ്ങള്‍ യഥാവിഥി നടത്തി. പിതാക്കളുടെ നിര്യാണത്തിന്റെ ശോകത്തില്‍ അവര്‍ തങ്ങളുടെ സമ്പത്തിലും മറ്റും ശ്രദ്ധയറ്റവരായി. ഒടുവില്‍ അവര്‍ വെവ്വേറെ ദിശകളിലുള്ള കാടുകളില്‍പ്പോയി തപസ്സു ചെയ്തു ജീവിക്കാന്‍ തീരുമാനിച്ചു. വളരെക്കാലം കഴിഞ്ഞ് അവര്‍ വീണ്ടും കണ്ടുമുട്ടി.
 
വിലാസന്‍ ഭാസനോടു പറഞ്ഞു: സുഹൃത്തേ, നിന്നെ കണ്ടതെത്ര സന്തോഷപ്രദം! നമ്മള്‍ പിരിഞ്ഞിട്ട് ഇത്രകാലം നീ എന്തൊക്കെയാണ് ചെയ്തത്? നിന്റെ തപശ്ചര്യകള്‍ ഫലപ്രദമായിരുന്നോ? ലോകമെന്ന ജ്വരത്തില്‍ നിന്നും നിന്റെ മനസ്സിന് ശാന്തി ലഭിച്ചുവോ? നിനക്ക് ആത്മജ്ഞാനം ഉണ്ടായോ? നിനക്ക് സുഖമാണോ?

ഭാസന്‍ പറഞ്ഞു: നിന്നെ വീണ്ടും കണ്ടത് എന്റെ ഭാഗ്യം തന്നെ. നീയെന്റെ സുഹൃത്തും സഹോദരനുമാണ്. ഈ ലോകത്ത് അലഞ്ഞുതിരിയുന്ന നമുക്ക് പരമമായ ജ്ഞാനം ആര്‍ജ്ജിക്കാതെ, മനോവൈകല്യങ്ങള്‍ക്കറുതിവരാതെ,   ആനന്ദവും സന്തോഷവും എങ്ങിനെയുണ്ടാകാനാണ്? സംസാരസാഗരം തരണം ചെയ്താലല്ലാതെ എങ്ങിനെയാണ് നമുക്കാനന്ദിക്കാനാവുക?

“മനസ്സില്‍ നിന്നുദിക്കുന്ന ആശകളും പ്രത്യാശകളും പൂര്‍ണ്ണമായി നശിച്ചാലല്ലാതെ നമുക്ക് സൌഖ്യവും സന്തോഷവും എങ്ങിനെയുണ്ടാകും?” ആത്മജ്ഞാനം പ്രാപിക്കുന്നതുവരെ നാം ജനനമരണ ചക്രത്തില്‍പ്പെട്ടുഴന്ന് വീണ്ടും വീണ്ടും ശൈശവം, യൌവ്വനം, വാര്‍ദ്ധക്യം, മരണം, പിന്നെയും ജനനം ഇങ്ങിനെ തുടര്‍ച്ചയായി വൃഥാ കര്‍മ്മങ്ങളില്‍ മുഴുകിക്കഴിയണമല്ലോ.! കഷ്ടം!

ജ്ഞാനത്തിനെ നശിപ്പിക്കുന്നത് ആസക്തിയാണ്‌....   ഇന്ദ്രിയ സുഖാനുഭവം തേടി ജന്മങ്ങള്‍  ക്ഷണത്തില്‍ കഴിഞ്ഞ് പോവുന്നു. ഇന്ദ്രിയസുഖമെന്ന ഇരുട്ടുകിണറിലാണ് മനസ്സ് വീണുപോവുന്നത്. സംസാരത്തിന്റെ മറുകര താണ്ടാന്‍ ഉതകുന്ന, ആത്മജ്ഞാനത്തിനുതകുന്ന, ഉത്തമവാഹനമായ ഈ ശരീരം എന്തുകൊണ്ടാണ് ഈ വിഷയങ്ങളാകുന്ന ചെളിക്കുണ്ടില്‍ വീണുപോവുന്നത്? ചെറിയൊരു കമ്പനത്തില്‍ നിന്നുണ്ടായ അലയെ ഇമചിമ്മുന്ന നേരംകൊണ്ട് മനസ്സ്  വലിയൊരു തിരമാലയാക്കുന്നു. മനുഷ്യന്‍ തന്റെ ശോകങ്ങളെ നിത്യശുദ്ധമുക്തമായ ആത്മാവില്‍ ആരോപിക്കുന്നു. എന്നിട്ടതില്‍ ആമഗ്നനായി സ്വയം ദുരിതമനുഭവിക്കുന്നു.
