\newpage
\section{ദിവസം 009}

\begin{center}
ചിത്തം കാരണമർത്ഥാനാം തസ്മിൻസതി ജഗത്രയം\\
തസ്മിൻ ക്ഷീണേ ജഗത്ക്ഷീണം തച്ചികിത്സ്യം പ്രയത്നത: (1/16/25)\\
\end{center}

രാമന്‍ തുടര്‍ന്നു: ജ്ഞാനത്തിന്റെ നേര്‍ ശത്രുവായ അഹംകാരത്തെപ്പറ്റി ഓര്‍ക്കുമ്പോള്‍ എനിക്ക്‌ ഭയവും അങ്കലാപ്പും ഉണ്ട്‌.. ഈ അഹന്ത വളര്‍ന്നു പെരുകുന്നത്‌ അജ്ഞതയുടെ ഇരുട്ടിലാണ്‌. . അതാകട്ടെ എണ്ണമറ്റ പാപ വാസനകള്‍ക്കും പ്രവര്‍ത്തനങ്ങള്‍ക്കും നിദാനമാവുന്നു. എല്ലാ ദു:ഖങ്ങള്‍ക്കും മന:പ്രയാസങ്ങള്‍ക്കും കാരണം 'ഞാന്‍' ദു:ഖിക്കുന്നു എന്നും മറ്റുമുള്ള അഹം ഭാവമാണ്‌.. ഈ അഹങ്കാരമാണ്‌ എന്റെ ഏറ്റവും വലിയ രോഗമെന്നെനിയ്ക്കു തോന്നുന്നു. ലോകത്തില്‍ സുഖസമ്പാദനത്തിനുള്ള വസ്തുക്കളാകുന്ന വല വിരിച്ച്‌ ഈ അഹംകാരം ജീവജാലങ്ങളെ ബന്ധിപ്പിക്കുന്നു. ലോകത്ത്‌ എല്ലാ അത്യാപത്തുകള്‍ക്കും കാരണവും ഈ അഹംകാരം തന്നെ. അത്‌ ആത്മനിയന്ത്രണം കെടുത്തി നന്മയെ നശിപ്പിച്ച്‌ സമതാ ഭാവത്തെ ഇല്ലാതാക്കുന്നു. എനിക്ക്‌ "ഞാന്‍ രാമനാണ്‌" തുടങ്ങിയ അഹംഭാവത്തെ വിട്ടു കളഞ്ഞ്‌ ആഗ്രഹങ്ങളില്‍ നിന്നും മോചിതനായി ആത്മാവില്‍ അഭിരമിക്കണം. 
അഹംകാരത്തോടെ ഇതുവരെ ചെയ്തതെല്ലാം വൃഥാവിലായി എന്നു ഞാന്‍ അറിയുന്നു. അഹംകാരരഹിതമായ അവസ്ഥ മാത്രമാണുണ്മ. അഹംകാരത്തിന്റെ പിടിയില്‍ നില്‍ക്കുമ്പോള്‍ ഞാന്‍ ദു:ഖിതനും അതില്‍ നിന്നും മോചിതനാവുമ്പോള്‍ സന്തോഷവാനുമാണ്‌. അഹംകാരം ആര്‍ത്തിയെ പോഷിപ്പിക്കുന്നു. അഹംകാരത്തിന്റെ അഭാവത്തില്‍ ആര്‍ത്തിക്ക്‌ വളരാനാവില്ല. സാമൂഹികവും കുടുംബപരവുമായ എല്ല ബന്ധങ്ങള്‍ക്കും കാരണം അഹംകാരമത്രേ. അത്‌ ജാഗ്രതയില്ലാത്ത ആത്മാവിനെ പിടികൂടി ബന്ധിക്കുകയാണ്‌. 


എനിക്കു തോന്നുന്നത്‌ ഞാന്‍ അഹംകാരത്തില്‍നിന്നും സ്വതന്ത്രനാണെന്നാണ്‌. എങ്കിലും ഞാന്‍ കഷ്ടപ്പെടുകതന്നെയാണ്‌. എന്നെ പ്രബുദ്ധനാക്കിയാലും. മഹാത്മാക്കള്‍ക്ക്‌ സേവ ചെയ്താലല്ലാതെ ഈ മലിനമായ മനസ്സെന്ന വസ്തു കാറ്റുപോലെ അശാന്തമായി വര്‍ത്തിക്കും. ഈ മനസ്സ്‌ എന്തുകിട്ടിയാലും കൂടുതല്‍ കൂടുതല്‍ ചഞ്ചലപ്പെട്ടുകൊണ്ടേയിരിക്കും. ഒരരിപ്പയില്‍ എത്ര വെള്ളമൊഴിച്ചാലും അതു നിറയാത്തപോലെ എത്രമാത്രം ലൌകീകനേട്ടങ്ങള്‍ ഉണ്ടായാലും വസ്തുവകകള്‍ സമ്പാദിച്ചാലും മനസ്സ്‌ നിറയുകയില്ല. മനസ്സ്‌ എല്ലാ ദിശകളിലേയ്ക്കും ദ്രുതഗമനം നടത്തുന്നുവെങ്കിലും ഒരിടത്തും സുഖം കണ്ടെത്തുന്നില്ല. മരണാനന്തരം നരകത്തില്‍ കിട്ടാന്‍ പോകുന്ന കൊടും ദുരിതങ്ങളെപ്പറ്റി ആലോചിക്കാതെ മനസ്സ്‌ സുഖത്തിനു പിറകേ പായുന്നു. എന്നാല്‍ ഈ ജീവിതത്തില്‍ സുഖം ലഭിക്കുന്നുമില്ല. കൂട്ടില്‍ കിടക്കുന്ന സിംഹത്തിനേപ്പോലെ അസ്വസ്ഥമാണ്‌ മനസ്സ്‌. സ്വാതന്ത്ര്യമില്ല, സുഖവുമില്ലാത്ത അവസ്ഥ. മഹാത്മാവേ കഷ്ടം! ഞാന്‍ ആ മനസ്സു വിരിച്ച വലയില്‍ കുടുങ്ങിയിരിക്കുന്നു. കുതിച്ചു പായുന്ന നദീജലം വൃക്ഷത്തെ കടയോടെ പിഴുതെടുക്കുമ്പോലെ മനസ്സ്‌ എന്റെ സ്വത്വത്തിനെ വേരോടെ പറിച്ചിരിക്കുന്നു. കാറ്റില്‍ പറക്കുന്ന ഉണങ്ങിയ കരിയിലപോലെ മനസ്സെന്നെ ചാഞ്ചാട്ടുന്നു.

"ഈ മനസ്സാണ്‌ ലോകത്തിലെ എല്ലാ വസ്തുക്കള്‍ക്കും കാരണം. മൂന്നു ലോകവും മനസ്സെന്ന വസ്തുവിനാല്‍ നിലനില്‍ ക്കുന്നു. മനസ്സില്ലാതാകുമ്പോള്‍ ലോകവും അപ്രത്യക്ഷമാവുന്നു.
