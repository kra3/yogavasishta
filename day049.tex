 
\section{ദിവസം 049}

\slokam{
ഏവം ബ്രഹ്മ മഹാജീവോ വിദ്യതേന്താദിവർജിത:\\
ജീവകോടി മഹാകോടി ഭവത്യഥ ന കിംചന (3/14/35)\\
}

വസിഷ്ഠന്‍ തുടര്‍ന്നു: രാമ: ഞാന്‍ നേരത്തേ പറഞ്ഞപോലെ,  അഹംകാരവും അനുഭവദായികളായ അസംഖ്യം വസ്തുക്കളും നിറഞ്ഞ ഈ ലോകം സത്യത്തിൽ ഇല്ലാത്തതാണ്‌.  ഉള്ളത്‌ , പരം പൊരുളായ ബ്രഹ്മം മാത്രം. അതിനുമാത്രമേ നിലനില്‍പ്പുള്ളു. ശാന്തമായ സമുദ്രോപരി അന്തര്‍സംഘര്‍ഷത്താലും മറ്റും കാണപ്പെടുന്ന അലകള്‍പോലെ പരമ്പൊരുള്‍ താന്‍ 'ജീവന്‍ ' ആണെന്നു ചിന്തിച്ച്‌ സ്വയം ജീവഭാവം ഉയിര്‍ക്കൊള്ളുന്നു. ഉറങ്ങിക്കിടക്കുന്നയാള്‍ തനിക്കുള്ളില്‍ വൈവിദ്ധ്യമാര്‍ന്ന സൃഷ്ടികളെ (സ്വപ്നത്തില്‍ ) ഉണ്ടാക്കുന്നത്‌ ഒരിക്കലും തന്റെ സ്വത്വത്തെ നിരാകരിച്ചിട്ടല്ലാത്തതുപോലെ വെറുമൊരു ചിന്താശകലംകൊണ്ട്‌ (അല്ലെങ്കില്‍ ഇച്ഛകൊണ്ട്‌) പരബ്രഹ്മം എണ്ണമറ്റ സൃഷ്ടികള്‍ നടത്തുമ്പോഴും സ്വയം യാതൊരു മാറ്റങ്ങള്‍ക്കും വിധേയമാവുന്നില്ല. യാതൊരു കുറവും അതിനുണ്ടാവുന്നുമില്ല.

പദാര്‍ത്ഥങ്ങളുടെ രൂപവല്‍ക്കരണം തുടങ്ങിയ 'കറ'യൊന്നും പുരളാത്ത ശുദ്ധബോധസ്വരൂപമത്രെ വിശ്വാവബോധത്തിന്റെ വിരാട്‌ രൂപം. നിര്‍മ്മലബോധത്തില്‍ നിന്നുണ്ടായ ഈ വിരാട്‌ രൂപം ഉറങ്ങിക്കിടക്കുന്നവന്റെയുള്ളിലെ അന്തമില്ലാത്ത ഒരു സ്വപ്നത്തോടുപമിക്കാം. അതില്‍ രാജകൊട്ടാരങ്ങളും മറ്റു ജീവികളും ഉണ്ട്‌. ബ്രഹ്മാവു പോലും ഈ പരബ്രഹ്മത്തിന്റെ വെറുമൊരു ചിന്തയാണ്‌. വിശ്വാവബോധം സ്വന്തം അസ്തിത്വത്തെപ്പറ്റി ചിന്തിക്കുന്നതും മറ്റ്‌ പ്രത്യക്ഷമായ ദൃക്‌-ദൃശ്യ ദ്വന്ദങ്ങളും എല്ലാം വെറും സങ്കല്‍പ്പങ്ങള്‍ മാത്രം. അവയെല്ലാം നാമമാത്രമാണ്‌. അവ പെറ്റുപെരുകുന്നതും നാമങ്ങളില്‍ മാത്രം. ഈ വിരാട്‌ രൂപം വിശ്വാവബോധത്തില്‍ ചിന്താബന്ധുരമായി ഉദ്ഭവിച്ചതുപോലെ  മറ്റെല്ലാം ആ വിരാട്ടിന്റെ ചിന്തയിലാണുണ്ടായത്‌. ഒരു വിളക്കിലെ ദീപനാളത്തില്‍ നിന്നും കൊളുത്തുന്ന മറ്റുവിളക്കുകള്‍പോലെയത്രേ ഇത്‌. ആരുടെ ചിന്താ-സ്പ്ന്ദനങ്ങളാണോ ഇവയെ സൃഷ്ടിച്ചത്‌ അതുമായി ഇവയ്ക്ക്‌ വ്യത്യാസമൊന്നുമില്ല. വിരാട്‌ പുരുഷന്‍ എന്നത്‌ പരബ്രഹ്മം തന്നെ. അതുതന്നെയാണീ സൃഷ്ടികളെല്ലാം. സൃഷ്ടിയുടെ ഭാഗമാണ്‌ ജീവനും പഞ്ചഭൂതങ്ങളും അവയുടെ അവയുടെ അസംഖ്യം സമ്മിശ്രണങ്ങളും.

രാമന്‍ പറഞ്ഞു: ഭഗവന്‍ , പ്രാപഞ്ചികമായി ഒരേ ഒരു ജീവനേ ഉള്ളോ അതോ അനേകം ജീവനുകള്‍ ഉണ്ടോ?

വസിഷ്ഠന്‍ മറുപടിയായി പറഞ്ഞു: രാമ: ഒരു ജീവനോ, അനേകം ജീവനുകളോ അവയുടെ സമുച്ചയങ്ങളോ ഒന്നും യദാര്‍ത്ഥത്തില്‍ ഇല്ല. ജീവന്‍ എന്നത്‌ ഒരു നാമം മാത്രം. നിലനില്‍ ക്കുന്നത്‌ ബ്രഹ്മം മാത്രം. പരബ്രഹ്മം സര്‍വ്വശക്തമാകയാല്‍ അവന്റെ ചിന്തകള്‍ മൂര്‍ത്തീകരിക്കുന്നു. വിളക്കെടുത്ത്‌ ഇരുട്ടിനെക്കുറിച്ചുള്ള സത്യമന്വേഷിക്കാന്‍ പോവുമ്പോള്‍ അപ്രത്യക്ഷമാവുന്ന ഇരുട്ടുപോലെ അവിദ്യയാല്‍ കാണപ്പെടുന്ന നാനാത്വം അന്വേഷണമാത്രയില്‍ ഇല്ലാതാവുന്നു. "ബ്രഹ്മം മാത്രമാണ്‌ വിശ്വാത്മാവും (മഹാജീവന്‍) മറ്റ്‌ കോടിക്കണക്കായ ജീവനുകളും. ബ്രഹ്മമല്ലാതെ മറ്റൊന്നുമില്ല."

