 
\section{ദിവസം 060}

\slokam{
ഇതി ജലധി മഹാദ്രി ലോകപാല ത്രിദശ പുരാംബര ഭൂതലൈ: പരീതം\\
ജഗദുദരമവേക്ഷ്യ മാനുഷീ ദ്രാഗ്ഭുവി നിജമന്ദിരകോടരം ദദർശ (3/25/35)\\
}

വസിഷ്ഠന്‍ തുടര്‍ന്നു: കൈകള്‍ കോര്‍ത്തുപിടിച്ച്‌ സരസ്വതീ ദേവിയും ലീലയും സാവധാനം ആകാശത്തിലെ പല സ്ഥലങ്ങളിലും വിഹരിച്ചു. ഈ ആകാശമാവട്ടെ അതീവ നിര്‍മ്മലവും പരിപൂര്‍ണ്ണ ശൂന്യവുമായിരുന്നു. ഭൂമിയുടെ അച്ചുതണ്ടായ മേരുപര്‍വ്വതത്തിന്റെ മുകളില്‍ അവരല്‍പ്പം വിശ്രമിച്ചു. ചന്ദ്രമണ്ഡലത്തില്‍ നിന്നും ദൂരേയ്ക്കു പോകവേ അവര്‍ എണ്ണമറ്റ പല കൌതുകവസ്തുക്കളും കണ്ടു. അവര്‍ വലിയ വലിയ മേഘപാളികളില്‍ ചുറ്റിനടന്നു. അവര്‍ അനന്താകാശത്തില്‍ , ഹിരണ്യഗര്‍ഭത്തില്‍ - അനന്തമായ പ്രപഞ്ചങ്ങളുടേയും ജീവജാലങ്ങളുടേയും ഉറവിടത്തില്‍ - കടന്നു. അഗ്നിപ്രളയം പോലെ ഭാസുരമായ സപ്തശൈലങ്ങളെ അവര്‍ അണ്ഡകഠാഹത്തില്‍ ദര്‍ശിച്ചു. മേരുവിനുസമീപം സുവര്‍ണ്ണസമതലങ്ങളേയും ഇരുട്ടിന്റെ അഗാധഗര്‍ത്തങ്ങളേയും അവര്‍ കണ്ടു. 

അതീന്ദ്രിയശക്തികളുള്ള സിദ്ധന്മാരെയും അസുരവൃന്ദങ്ങളേയും ഭൂതപിശാചുക്കളെയും അവരവിടെ കണ്ടു. ആകാശവാഹനങ്ങള്‍ വരുന്നതും പോവുന്നതും കണ്ടു. അപ്സരസ്ത്രീകള്‍ ആടുന്നതും പാടുന്നതും കണ്ടു. പലേവിധ പക്ഷിമൃഗാദികളേയും അവരവിടെ കണ്ടു. മാലാഖമാരേയും ദേവതകളെയും കണ്ടു. ദിവ്യഗുണസമ്പന്നന്മാരും മഹാസിദ്ധന്മാരുമായ യോഗിവര്യന്മാരേയും അവര്‍ കണ്ടു. ബ്രഹ്മാവിന്റെേയും ശിവന്റേയും മറ്റു ദേവകളുടെയും ആസനസ്ഥാനങ്ങള്‍ രണ്ടു കൊതുകുകളെപ്പോലെ പറന്നു നടന്ന് അവര്‍ കണ്ടു. ചുരുക്കത്തില്‍ സരസ്വതീ ദേവിയുടെ ചിന്തയിലുണ്ടായിരുന്നതും, ലീലയെ കാണിക്കണമെന്നു കരുതിയതുമെല്ലാം അവര്‍ കണ്ടു.

ഹൃദയകമലമെന്നതുപോലെയാണത്‌ - ദിശകള്‍ അതിന്റെ ഇതളുകള്‍ ; പാതാളം അതു നിന്നു വളരുന്ന ചെളിക്കുണ്ട്‌; അതിനെ താങ്ങിനില്‍ ക്കുന്ന സര്‍പ്പം അതിന്റെ വേര്‌. അതിനുചുറ്റും ഉപ്പുവെള്ളം നിറഞ്ഞ സമുദ്രം. അതിനപ്പുറം പാല്‍ക്കടലില്‍ ശാകദ്വീപ്‌. തൈരു കടലില്‍ കുശദ്വീപാണതിനുമപ്പുറം. പിന്നെ നെയ്‌ക്കടലില്‍ ക്രൌഞ്ചദ്വീപ്‌; മുന്തിരിച്ചാറിന്റെ കടലില്‍ സാല്‍ മദ്വീപ്‌; പിന്നെയോ കരിമ്പിന്‍ നീരില്‍ ഗോമേദകദ്വീപ്‌; മധുരജലത്തില്‍ പുഷ്കരദ്വീപാണു പിന്നെ. പിന്നെയാണ്‌ വിശ്വഗഹ്വരം. ഭാസുരപ്രഭയോടെ വിളങ്ങുന്ന ലോകാലോകം എന്ന പര്‍വ്വതമുണ്ടതിനുമപ്പുറം. പിന്നെ അനന്തമായ കാടാണ്‌. അവസാനം അനന്തമായ ആകാശം, തികഞ്ഞ നിശ്ശൂന്യത. "അങ്ങിനെ കടലുകളും മലകളും പ്രപഞ്ചരക്ഷകരായ മഹത്പുരുഷന്മാരെയും ദൈവരാജ്യവും അകാശവും അണ്ഠകടാഹങ്ങളും, ലോകത്തിന്റെ ഇരിപ്പിടമായ ഉദരവും അതില്‍ തന്റെ ഭവനവും ലീല ദര്‍ശിച്ചു."

* ഭാഗവതത്തിലെ വിവരണവുമായി താരതമ്യപ്പെടുത്താവുന്നതാണീ വര്‍ണ്ണന.
