 
\section{ദിവസം 096}

\slokam{
സതി ധര്‍മിണീ ധര്‍മ്മാ ഹി സംഭവന്തീഹ നാസതി\\
ശരീരം വിദ്ധ്യതേ യസ്യ തസ്യ തത്കില തൃപ്തതി (3/73/32)\\
}

ഇന്ദ്രന്റെ അഭ്യര്‍ത്ഥനയനുസരിച്ച്‌ നാരദന്‍ കാര്‍ക്കടിയുടെ കഥ വിവരിച്ചു: ഈ നികൃഷ്ടയായ കാര്‍ക്കടി ജീവനുള്ള ഒരു സൂചിയായി ലോഹസൂചിയുടെ രൂപത്തില്‍ ഇരിക്കുന്നു. പാപപങ്കിലരായ ആളുകളുടെ പേശികളെ കുത്തിത്തുളച്ച്‌ അവരുടെ സന്ധികളേയും രക്തത്തേയും ദുഷിപ്പിക്കുന്നു. അവള്‍ കാറ്റുപോലെ ശരീരങ്ങളില്‍ക്കടന്ന് കുത്തിപ്പിളര്‍ക്കുന്ന, തുളച്ചുകയറുന്നതരം വേദനയുളവാക്കുന്നു. ശുദ്ധമല്ലാത്ത, മാംസം മുതലായ ഭക്ഷണങ്ങളാല്‍ പരിപോഷിക്കപ്പെട്ട ശരീരങ്ങളിലവള്‍ ഈ വേദനയുമായി ആഴ്ന്നിറങ്ങുന്നു. കഴുകന്‍ മുതലായ ജീവികളുടെ ശരീരത്തിലും അവള്‍ പ്രവേശിച്ച്‌ അവയെ ആര്‍ത്തിയോടെ ആഹരിക്കുന്നു. തീവ്രതപസ്സനുഷ്ഠിച്ചതിന്റെ ഫലമായി തന്റെ ഇരകളുടെ മനസ്സിലും ഹൃദയത്തിലും അവള്‍ക്കു പ്രവേശനമുണ്ട്‌. അതുകൊണ്ട്‌ ആതിഥേയന്റെ എല്ലാ പ്രവര്‍ത്തനങ്ങളിലും അവള്‍ പങ്കാളിയാവുന്നു. സ്വയം വായുപോലെ സൂക്ഷ്മരൂപിയും അഗോചരവും ആണെങ്കില്‍ ഒരാള്‍ക്ക്‌ എന്താണസാദ്ധ്യം? എങ്കിലും അവള്‍ക്ക്‌ തന്റെ വാസനാമാലിന്യം കാരണം ചിലജീവികളോട്‌ ഏറെ പ്രിയവും മറ്റുചിലതിനോട്‌ അപ്രിയവും ഉണ്ടായി. അവയോട്‌ ബന്ധപ്പെട്ട്‌, അവള്‍ അവരെച്ചുറ്റിപ്പറ്റി നിന്നു. അവള്‍ സ്വതന്ത്രയായി ചുറ്റി നടന്നുവെങ്കിലും ചിലപ്പോളൊക്കെ തന്റെ സൂചികാ രൂപത്തിലേയ്ക്കവള്‍ മടങ്ങിവന്നു. അവിവേകികളങ്ങിനെയാണല്ലോ കുഴപ്പങ്ങളില്‍പ്പെടുന്നത്‌. .

"അസ്തിത്വമുള്ളതിനു മാത്രമേ അനുഭവങ്ങളിലൂടെ കടന്നുപോവാന്‍ കഴിയൂ. ശരീരമില്ലാത്ത ഒരു സത്വത്തിന്‌ എങ്ങിനെയാണ്‌ സംതൃപ്തിയെന്ന അനുഭവം ഉണ്ടാവുക?" അങ്ങിനെ സൂചിക അതീവ ദു:ഖിതയായിരുന്നു. തന്റെ ഭീമാകാരരൂപം തിരികെകിട്ടാന്‍ അവള്‍ വീണ്ടും തപസ്സു തുടര്‍ന്നു. അവള്‍ ഒരു കഴുകന്റെ ശരീരത്തില്‍ക്കയറി. കഴുകന്‍ അവളെ ഹിമാലയത്തില്‍ ഉപേക്ഷിച്ചു പറന്നുപോയി. ഖരരൂപത്തിലുള്ള ആ സൂചിയെ അവലംബമാക്കി ഇപ്പോഴും അവള്‍  തപസ്സു തുടരുന്നു. അല്ലയോ ഇന്ദ്രാ, അവളുടെ തപസ്സിനു വിഘ്നം വരുത്തിയില്ലെങ്കില്‍ തപശ്ശക്തികൊണ്ട്‌ അവള്‍ ലോകം മുടിക്കും. 

വസിഷ്ഠന്‍ തുടര്‍ന്നു: ഇതുകേട്ട്‌ കാര്‍ക്കടി തപസ്സുചെയ്യുന്ന ഇടം കൃത്യമായി കണ്ടുപിടിക്കാന്‍ ഇന്ദ്രന്‍ വായുവിനെ നിയോഗിച്ചു. പ്രപഞ്ചമെല്ലാം, വിവിധ സൌരയൂഥങ്ങളിലുമെല്ലാം ചുറ്റിയടിച്ചുവന്ന കാറ്റ്‌ അവസാനം ഭൂമിയില്‍ ഹിമാലയ പരിസരത്ത്‌ വന്നു. അവിടം സൂര്യതാപം കൊണ്ട്‌ ചെടികളൊന്നുമില്ലാതെ ഉണങ്ങിവരണ്ട്‌ മരുഭൂമിപോലെ ആയിരുന്നു.

