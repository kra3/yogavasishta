\section{ദിവസം 269}

\slokam{
ചിത്തേന ചേതഃ: ശമമാശു നീത്വാ\\
ശുദ്ധേന ഘോരാസ്ത്രമിവാസ്ത്രയുക്ത്വാ\\
ചിരായ സാധോ ത്യജ ചഞ്ചലത്വം\\
വിമര്‍ക്കടോ വൃക്ഷ ഇവാക്ഷതശ്രീ:   (5/50/84)\\
}

വസിഷ്ഠന്‍ തുടര്‍ന്നു: ശരീരമെന്ന ഒരു കൊടുംവനത്തില്‍ വേരുറപ്പിച്ച മരം പൊലെയാണീ മനസ്സ്. ആശങ്കകളും ആകുലതകളുമാണതിന്റെ മൊട്ടുകള്‍ . വാര്‍ദ്ധക്യവും ജരാനരകളും അതിന്റെ ഫലങ്ങള്‍ .   ആഗ്രഹങ്ങള്‍ , ഇന്ദ്രിയ സുഖങ്ങള്‍ എന്നീ പൂക്കളാണതിനെ അലങ്കരിച്ചിരിക്കുന്നത്. പ്രത്യാശകളും ഉള്‍ക്കടമായ അഭിവാഞ്ഛകളും അതിന്റെ ശിഖരങ്ങള്‍ . വികടമായ താന്തോന്നിത്തമാണതിന്റെ അസംഖ്യം ഇലകള്‍ . മലപോലെ ഉറച്ചതും വെട്ടിമുറിക്കാനരുതാത്തതുമെന്നും തോന്നുന്ന ഈ വൃക്ഷത്തെ  അന്വേഷണമെന്ന മൂര്‍ച്ചയേറിയ കോടാലികൊണ്ട് വെട്ടി മുറിക്കണം.

രാമാ, ഈ മനസ്സ് ശരീരമെന്ന വനത്തില്‍ മേഞ്ഞുനടക്കുന്ന ഒരു മദയാനയാണ്. അതിന്റെ കാഴ്ച ഭ്രമങ്ങളാല്‍ മങ്ങിപ്പോയിരിക്കുന്നു. അജ്ഞാനത്തിന്റെയും ഉപാധികളുടേയും വരുതിയിലായതിനാല്‍ അതിനു സ്വയം ആത്മാനന്ദത്തില്‍ അഭിരമിക്കുക അസാദ്ധ്യം. ജ്ഞാനികളായ മഹത്തുക്കളില്‍ നിന്നും കേട്ടിട്ടുള്ള സത്യസ്വരൂപത്തെപ്പറ്റി അറിയാനാഗ്രഹമുണ്ടെങ്കിലും സ്വയമത് ഉപാധികള്‍ക്കും ഇഷ്ടാനിഷ്ടങ്ങള്‍ക്കും വൈവിദ്ധ്യങ്ങള്‍ക്കും വശംവദമാണ്. പലപ്പോഴും കാമക്രോധാദികള്‍ അതിനെ കീഴ്പ്പെടുത്തുന്നു. അതിനുള്ളത് കാമത്വരയെന്ന തുമ്പിക്കൈയാണ്. രാമാ, നീ രാജകുമാരന്മാരില്‍ വെച്ച് സിംഹമാണ്. ഈ ആനയെ നിന്റെ മൂര്‍ച്ചയേറിയ ബുദ്ധിശക്തികൊണ്ട് കീഴ്പ്പെടുത്തി തുണ്ടം തുണ്ടമാക്കിയാലും.

രാമാ മനസ്സ്, ഈ ശരീരത്തെ തന്റെ കിളിക്കൂടാക്കിയ ഒരു കാക്കയാണ്. അത് രസിക്കുന്നത് മാലിന്യത്തില്‍ക്കിടന്നാണ്. അതഭിവൃദ്ധിപ്പെടുന്നത് മാംസം കഴിച്ചാണ്. മറ്റുള്ളവരുടെ ഹൃദയമത് കുത്തിത്തുളയ്ക്കുന്നു. അതിനു താന്‍ സത്യമെന്ന് വിശ്വസിക്കുന്ന ഒരേ ഒരു മാര്‍ഗ്ഗത്തിലേ അറിവുള്ളൂ. മറ്റുള്ള ചിന്താഗതികളെ അതിനു ശ്രദ്ധിക്കാന്‍ കൂടി അറിയില്ല. കൂടുതല്‍ വഷളായിക്കൊണ്ടിരിക്കുന്ന മന്ദത ഹേതുവായി അതിന് ആന്ധ്യം പിടിപെട്ടിരിക്കുന്നു. അങ്ങിനെ ദുഷ്ടവിചാരവും പ്രവൃത്തിയും അതിന്റെ സഹജസ്വഭാവമായിരിക്കുന്നു. അത് ഭൂമിക്കൊരു ഭാരമാണ് രാമാ. അതിനെ നിന്റെ അടുക്കല്‍ നിന്നും ദൂരേയ്ക്ക് പറത്തിക്കളയൂ.

രാമാ മനസ്സ്, ഒരു ദുര്‍ഭൂതമാണ്. ആസക്തിയെന്ന പിശാചിനിയാണവനെ പരിചരിക്കുന്നത്. അജ്ഞതയുടെ കൊടുംകാട്ടിലാണവന്റെ വാസം. എണ്ണമറ്റ ശരീരങ്ങളിലൂടെ വിഭ്രാന്തനായി അവന്‍ അലഞ്ഞുതിരിയുന്നു. ഗുരുകാരുണ്യം, സ്വപ്രയത്നം, മന്ത്രോച്ചാരണം, നിര്‍മമത, അറിവ് എന്നിവയിലൂടെ ഈ ഭൂതത്തെ തുരത്തിയാലല്ലാതെ ഒരുവന് എങ്ങിനെ ആത്മജ്ഞാനം ലഭിക്കാനാണ്? 

രാമാ, മനസ്സ് അനേകം ജീവികളെ കൊന്നൊടുക്കിയ ഒരു വിഷസര്‍പ്പം തന്നെയാണ്. ഉചിതമായ ധ്യാനസപര്യയോ ശാസ്ത്രജ്ഞാനമോ ആണ് ഇവനെ കൊല്ലാന്‍ പറ്റിയ ഗരുഡന്‍.. നീയവ ശരിയായി പരിശീലിക്കൂ.

രാമാ, ഈ മനസ്സ് ഒരു മര്‍ക്കടനാണ്. അവന്‍ ഒരു സ്ഥലത്തുനിന്നും മറ്റൊരിടത്തേയ്ക്ക് സുഖങ്ങള്‍ , സ്ഥാനമാനങ്ങള്‍ എന്നീ ഫലങ്ങള്‍ക്കായി ചാടിച്ചാടി നടക്കുന്നു. ഈ ലോകചക്രത്തില്‍ ചുറ്റിക്കൊണ്ട് ആളുകളെ രസിപ്പിച്ച് അവന്‍ നടനമാടുകയാണ്. പൂര്‍ണ്ണതയാണ് നീയുദ്ദേശിക്കുന്നതെങ്കില്‍ ഇവനെ എല്ലാ രീതിയിലും തളച്ചിടണം. 
രാമാ, ഈ മനസ്സ് , അവിദ്യയുടെ, അജ്ഞാനത്തിന്റെ, കാര്‍മേഘമാണ്. ആശയ സങ്കല്‍പ്പങ്ങളെയും ധാരണകളെയും ആവര്‍ത്തിച്ച് നിരാകരിച്ച് ഈ മേഘത്തെ നീ ഉന്മൂലനം ചെയ്യണം.

“അതീവ ശക്തിയേറിയ ആയുധത്തെ നേരിടാനും മറ്റൊരായുധം കൂടിയേ തീരൂ. മനസ്സെന്ന മരുന്നുകൊണ്ട് തന്നെ മനസ്സിനെ മയക്കിക്കിടത്തൂ. അങ്ങിനെ മനസ്സിലെ എല്ലാ ചഞ്ചലത്വവും ഇല്ലാതാക്കൂ. മര്‍ക്കടന്മാര്‍ വിട്ടുപോയ മരത്തിന്റെ പ്രശാന്തതയെന്നപോലെ നീ നിലകൊണ്ടാലും.”


