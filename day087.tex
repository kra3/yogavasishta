\newpage
\section{ദിവസം 087}

\slokam{
നാസ് തമേ തി ന ചോടെതി ക്വചിത് കിഞ്ചിത് കദാചന\\
സര്‍വം ശാന്തമജം ബ്രഹ്മ ചിദ്ഘനം സുശിലാഘനം (3/61/31)\\
}

വസിഷ്ഠന്‍ തുടര്‍ന്നു: അതേസമയം, ആ അനന്തബോധത്തില്‍ കാലയളവിന്റെ ഏകകമായി (യൂണിറ്റ്‌ )  ഇമവെട്ടാനെടുക്കുന്ന സമയത്തിന്റെ പത്തുലക്ഷത്തിലൊന്ന് എന്നൊരു ധാരണയുണ്ടായി. ഇതില്‍നിന്നാണ്‌ ചതുര്യുഗം വരെയുള്ള കാലമാപിനി (ടൈം സ്കൈയില്‍ ) ഉണ്ടായത്‌. ചതുര്യുഗങ്ങളുടെ പല ആവര്‍ത്തനങ്ങള്‍ കഴിഞ്ഞാലാണ്‌ വിശ്വസൃഷ്ടിയുടെ ഒരായുസ്സുകാലം ആവുന്നത്‌. എന്നാല്‍ അനന്താവബോധം കാലത്തിന്റെ സ്വഭാവമായ വിക്ഷേപം, ആവരണം എന്നിവകളില്‍ തീരെ നിമഗ്നമായിരുന്നില്ല. കാരണം അതിന്‌ ആദിമദ്ധ്യാന്തങ്ങളില്ല. പ്രബുദ്ധവും ജാഗ്രത്തുമായ അനന്താവബോധം മാത്രമേ ഉണ്മയായുള്ളു. സൃഷ്ടികാര്യത്തിലും അതപ്രകാരമാണ്‌... അതു തന്നെയാണ്‌ അവിദ്യാകൃതമായ സൃഷ്ടിപ്രകടനവും. പ്രളയശേഷവും അതങ്ങിനെത്തന്നെ അനന്തമായി നിലകൊള്ളൂം. 

ഒരുവന്‍ തന്റെ ആത്മാവിനെ ആത്മാന്വേഷണത്തിലൂടെ സാക്ഷാത്കരിക്കുമ്പോള്‍ അവന്റെ ബോധതലം പരബ്രഹ്മം തന്നെയാണ്‌..  അപ്പോളവന്‍ എല്ലാം അനുഭവിക്കുന്നു. ശരീരത്തിന്റെ ഒരേ ഊര്‍ജ്ജം തന്നെയാണല്ലോ അവയവങ്ങളിലും പ്രസരിക്കുന്നത്‌.. ഈ പ്രത്യക്ഷലോകം നേരനുഭവത്തിന്റെ വെളിച്ചത്തില്‍ ബോധത്തിന്റെയൊരു പ്രകടനമാണെന്നതുകൊണ്ട്‌ സത്യമാണെന്ന് ഒരാള്‍ക്ക്‌ വേണമെങ്കില്‍ പറയാം. എന്നാല്‍ അതിനെ മനസ്സുകൊണ്ടും ഇന്ദ്രിയങ്ങള്‍കൊണ്ടും അറിയുമ്പോള്‍ അത്‌ യാഥാര്‍ഥ്യമല്ല. വായുവിനെ അറിയുന്നത്‌ (കാറ്റായി) അതിന്റെ ചലനം-ഗമനം കൊണ്ടാണ്‌. എന്നാല്‍ ചലനമില്ലാത്തപ്പോള്‍ വായു ഉള്ളതായി തോന്നുന്നേയില്ല. അതുപോലെ ഈ പ്രത്യക്ഷലോകം ഉള്ളതായും ഇല്ലാത്തതായും കണക്കാക്കാം. 


കാനല്‍ ജലം പോലെ കാണപ്പെടുന്ന ത്രിലോകങ്ങള്‍ നിലകൊള്ളുന്നത്‌ പരബ്രഹ്മത്തില്‍നിന്നും വിഭിന്നമായല്ല. ബ്രഹ്മത്തില്‍ സൃഷ്ടിയുള്ളത്‌ വിത്തില്‍ മുളയിരിക്കുന്നതുപോലെ, ജലത്തിനു നനവുള്ളതുപോലെ, പാലിനു മധുരമുള്ളതുപോലെ മുളകിനെരിവുള്ളതുപോലെ സഹജമായാണ്‌. എന്നാല്‍ അവിദ്യയില്‍ ഇവ ബ്രഹ്മത്തില്‍നിന്നും സ്വതന്ത്രമാണെന്നു തോന്നുന്നു. ഈ ലോകനിര്‍മ്മിതിക്ക്‌ കാരണമൊന്നും ഇല്ല. അത്‌ പരബ്രഹ്മത്തിന്റെ നിര്‍മ്മല-പ്രതിഫലനം മാത്രം.

'സൃഷ്ടി' എന്നൊരു ധാരണയുള്ളപ്പോള്‍ സൃഷ്ടി ഉള്ളതായിത്തോന്നുന്നു. എന്നാല്‍ സ്വപ്രയത്നംകൊണ്ട്‌ സൃഷ്ടിയുടെ 'അയാഥാര്‍ഥ്യ തലം' അറിഞ്ഞവന്‌ ലോകം ഇല്ല. "യാതൊന്നും ഒരിടത്തും ഒരിക്കലും സൃഷ്ടിക്കപ്പെട്ടിട്ടേയില്ല. അതിനാല്‍ത്തന്നെ യാതൊന്നിനും നാശമുണ്ടാവുകയുമില്ല. പരബ്രഹ്മമാണെല്ലാം. അതു പരമശാന്തമാണ്‌, അജമാണ്‌, നിര്‍മ്മലബോധമാണ്‌. ശാശ്വതവുമാണ്‌." ഒരോ അണുവിലും ലോകങ്ങള്‍ക്കുള്ളില്‍ ലോകങ്ങള്‍ കാണാകുന്നു. ഈ വിക്ഷേപങ്ങള്‍ക്ക്‌ കാരണമെന്താണ്‌? 'ഞാന്‍', 'ഈ ലോകം' എന്നീ ധരണകള്‍ ഇല്ലാത്തപ്പോള്‍ ഒരുവന്‍ മുക്തനാണ്‌. എന്നാല്‍ 'ഞാന്‍ ഇതാണ്‌' തുടങ്ങിയ ധാരണകളാണ്‌ നമ്മുടെ ഒരേയൊരു ബന്ധനം. ആരൊരുവന്‍ അനന്താവബോധത്തിനെ, വിശ്വത്തിന്റെ നാമരൂപങ്ങളില്ലാത്ത അടിസ്ഥാനമായി അറിയുന്നുവോ അവന്‌ സംസാരമെന്ന ആവര്‍ത്തനചക്രത്തിനെ മറികടക്കാം.
