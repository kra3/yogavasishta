\section{ദിവസം 193}

\slokam{
മാ സങ്കൽപ്പയ സങ്കൽപ്പം ഭാവം ഭാവയ മാ സ്ഥിതൗ\\
എതാവതൈവ ഭാവേന ഭവ്യോ ഭവതി ഭൂതയേ (4/54/12)\\
}

ആ ചെറുപ്പക്കാരൻ ചോദിച്ചു: അച്ഛാ എങ്ങിനെയാണീ ധാരണ, ചിന്ത, സങ്കൽപ്പം, ആശയം, എന്നിവ ഉണ്ടാവുന്നത്? വളരുന്നത്? ഒടുവിൽ ഇല്ലാതാവുന്നത്?

ദാസുരമുനി പറഞ്ഞു: മകനേ അനന്താവബോധത്തിൽ, ഈ ബോധം സ്വയം തന്നെത്തന്നെ തിരിച്ചറിയുമ്പോൾ ആശയത്തിന്റെ വിത്തു പാകിക്കഴിഞ്ഞു. അതീവ സൂക്ഷ്മമാണത്. എന്നാൽ അചിരേണ അതു വളർന്ന് ആകാശം നിറയ്ക്കുന്നതുപോലെ സ്ഥൂലമായേക്കാം. ഈ വിധം ബോധം ആശയഗ്രസ്തമായിരിക്കുമ്പോൾ വിഷയവും വിഷയിയും രണ്ടാണെന്ന ചിന്തയുണ്ടാവുന്നു. ആശയങ്ങൾ സ്വയം പെറ്റുപെരുകുന്നു. അത് ദു:ഖത്തിനു കാരണമാകുന്നു. സുഖാനുഭവമല്ല അതു നൽകുന്നത്. ഈ ആശയ ധാരണകൾ മാത്രമാണ്‌ ദു:ഖത്തിനു കാരണം. ഈ ആശയരൂപീകരണം തികച്ചും ആകസ്മികമായാണുണ്ടാകുന്നത്. കാക്കയും പനമ്പഴവും പോലെ. പനമ്പഴം വീഴുന്നതിന്‌ കാക്കയുമായി സംബന്ധമൊന്നുമില്ലെങ്കിലും അവ തമ്മില്‍  ഒരു 'കാര്യ-കാരണബന്ധം' ആരോപിക്കപ്പെടുന്നുണ്ട്. ഇപ്രകാരം അസത്താണെങ്കിലും ഈ ‘അവസ്തു’വിന്‌ വളർച്ച സാദ്ധ്യമാണ്‌... നിന്റെ ജനനവും നിലനിൽപ്പും  ഒന്നും സത്യമല്ല. നിന്നിൽ ഈ അറിവ് സാക്ഷാത്കരിക്കുമ്പോൾ അയഥാർത്ഥ്യമായ വസ്തു തന്നെ  ഇല്ലാതാവും.

“ഈ ആശയങ്ങളെ പോഷിപ്പിക്കാതിരിക്കുക. നിന്റെ അസ്തിത്വത്തിനെക്കുറിച്ചുള്ള ധാരണകളിൽ അഭിരമിക്കാതിരിക്കുക. കാരണം ഭാവിജീവിതം മൂർത്തമാവുന്നതുതന്നെ ഈദൃശ ചിന്തകളുള്ളതിനാലാണ്‌.” ആശയരൂപീകരണം അവസാനിക്കുന്നതിനെ ഭയപ്പെടേണ്ടതില്ല. ചിന്തകളില്ലാതാവുമ്പോൾ ധാരണകളും ആശയങ്ങളും അസ്തമിക്കുന്നു. മകനേ കൈവെള്ളയിലിരിക്കുന്ന ഒരു പൂവു ഞെരിച്ചു കളയുന്നതിനേക്കാൾ എളുപ്പമാണ്‌ ധാരണകൾ അവസാനിപ്പിക്കാൻ. എത്ര ക്ഷിപ്രസാദ്ധ്യമെങ്കിലും അതിനും പ്രയത്നം ആവശ്യമുണ്ടല്ലോ. എന്നാല്‍ ധാരണകളവസാനിപ്പിക്കാൻ പ്രയത്നമാവശ്യമില്ല. ഇങ്ങിനെ എല്ലാ മനോവ്യാപാരങ്ങളുമവസാനിക്കുമ്പോൾ പരമശാന്തിയായി. ദു:ഖങ്ങൾക്കറുതിയായി.

ഈ ലോകത്തിലെ എല്ലാം ഒരാശയമോ ധാരണയോ മാത്രമാണ്‌.. അതിനു പല നാമങ്ങളുണ്ട്. മനസ്സ്, ജീവൻ, ജീവാത്മാവ്, ബോധം, ഉപാധി എന്നിങ്ങനെയെല്ലാം അറിയപ്പെടുന്ന ഇതിനു സത്യത്തിൽ ഉണ്മയില്ല. അതുകൊണ്ട് എല്ലാ ചിന്തകളുമകറ്റി ശാന്തനായാലും. നിന്റെ ജീവിതവും പ്രയത്നവും വെറുതേ കളയാതിരിക്കുക. ധാരണകളുടെ തീവ്രത കുറയുന്നമുറയ്ക്ക് സുഖദു:ഖങ്ങൾ ഒരുവനെ ബാധിക്കാതെയാകുന്നു. പദാർത്ഥങ്ങളുടെ (അ)യാഥാർത്ഥ സ്വഭാവം അറിയുന്നതിനാൽ അയാൾക്ക് അവയോട് ആസക്തിയുമില്ല. പ്രതീക്ഷകൾ വച്ചുപുലർത്താത്തവന്‌ അമിതാഹ്ളാദമോ വിഷാദമോ ഉണ്ടാവുന്നതെങ്ങിനെ? ബോധതലത്തിൽ പ്രതിബിംബിക്കുന്ന മനസ്സാണ്‌ ജീവൻ. മനസ്സ് ആകാശത്ത് കോട്ടകളുണ്ടാക്കുന്നു. അത് ഭൂത-വർത്തമാന-ഭാവി കാലങ്ങളിലേയ്ക്ക് നീണ്ടു പരന്നു കിടക്കുന്നു.

ആശയധ്രുവീകരണം എന്ന സമസ്യയെ മനസ്സിലാക്കുക അസാദ്ധ്യം. എന്നാൽ ഒന്നുപറയാം; ഇന്ദ്രിയാനുഭവങ്ങൾ അവയെ വർദ്ധിപ്പിക്കുന്നു. ഇന്ദ്രിയാനുഭവങ്ങളുടെ നിരാസം ധാരണകളുടെ ഒഴുക്കവസാനിപ്പിക്കുന്നു. എന്നാൽ ഈ ധാരണകൾ യാഥാർത്ഥ്യവും, കൽക്കരിത്തുണ്ടിലെ കറുപ്പുനിറം പോലെ സുസ്ഥിരവുമാണെങ്കിൽ അവയെ നീക്കംചെയ്യുക അസാദ്ധ്യം. എന്നാൽ ധാരണകൾക്ക് ഉണ്മയില്ല. അവയെ നശിപ്പിക്കാനാവും. 

