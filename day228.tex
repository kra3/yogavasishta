\section{ദിവസം 228}

\slokam{
തമേവ ഭുക്തവിരസം വ്യാപാരൗഘം പുന: പുന:\\
ദിവസേ ദിവസേ കുർവൻപ്രാജ്ഞ: കസ്മാന്ന ലജ്ജതേ (5/22/33)\\
}

വസിഷ്ഠൻ തുടർന്നു:രാമാ, ഞാൻ മുൻപുപറഞ്ഞു തന്ന മാർഗ്ഗമല്ലെങ്കിൽപ്പിന്നെ ബലിരാജാവ് ചെയ്തതുപോലെ മനസ്സിന്റെ പരിവർത്തനത്തിലൂടെയും ഇതു സാദ്ധ്യമാണ്‌.. ഇനി ഞാൻ ബലിയുടെ കഥപറയാം. അതുകേട്ടാൽ നിനക്ക് ശാശ്വതമായ സത്യത്തെക്കുറിച്ചുള്ള അറിവുണ്ടാകും. ലോകത്തിന്റെ മറ്റൊരു ഭാഗത്ത് പാതാളം എന്നൊരു ലോകമുണ്ട്. അതിസുന്ദരികളായ രാക്ഷസസ്ത്രീകളും ഒന്നില്‍ക്കൂടുതൽ തലകളുള്ള വിചിത്ര സർപ്പങ്ങളും, വലിയ ശരീരങ്ങളുള്ള രാക്ഷസന്മാരും, വലിയ ആനകളും അവിടെയുണ്ട്. മാലിന്യം നിറഞ്ഞതും എപ്പോഴും ഭീകരമായ കടകടാരവം നിറഞ്ഞതുമാണവിടം. അവിടെയുള്ള ഗുഹകളിലും ഖനികളിലും വിലപിടിച്ച രത്നക്കല്ലുകളുണ്ട്. കപിലമുനിയുടെ പാവനമായ പാദപാംശുസ്പർശമേറ്റു പുണ്യമാർജ്ജിച്ച ഇടങ്ങളും അവിടെയുണ്ട്. സ്വർഗ്ഗവാസികളായ അപ്സരസ്സുകൾ പൂജിക്കുന്ന ഹാടകേശ്വരൻ പവിത്രമാക്കിയ ഇടവും പാതാളത്തിലാണ്. (കാലിഫോർണിയയാണ്‌ കപിലമുനി വസിച്ചിരുന്ന കപിലവനം എന്നു പറയുന്നവരുണ്ട്)

വിരോചനപുത്രനായ ബലി പാതാളത്തിന്റെ രാജാവായിരുന്നു. വിശ്വരക്ഷിതാവായ ശ്രീഹരിയാണു പാതാളത്തിന്റെയും ബലിയുടെയും രക്ഷകർത്താവ്. ദേവരാജനായ ഇന്ദ്രൻപോലും അദ്ദേഹത്തെ ബഹുമാനിച്ചിരുന്നു. ഈ ബലിരാജന്റെ പ്രഭാവത്തിന്റെ തിളക്കത്തിൽ സമുദ്രങ്ങൾപോലും വറ്റിവരണ്ടെന്നപോലെ നിഷ് പ്രഭമായി തോന്നിയിരുന്നു. അദ്ദേഹത്തിന്റെ വെറുമൊരു ദൃഷ്ടി കൊണ്ട് മലകളെപ്പോലും നീക്കാനുള്ള പ്രാഭവം അദ്ദേഹത്തിനുണ്ടായിരുന്നു. ബലി ഏറെക്കാലം പാതാളലോകം ഭരിച്ചു. കാലക്രമത്തിൽ തീവ്രമായ ഒരനാസക്തി അദ്ദേഹത്തെ ബാധിച്ചു. അദ്ദേഹമിങ്ങിനെ ആലോചിക്കാൻ തുടങ്ങി: ഞാൻ എത്രകാലം ഈ പാതാളം വാഴണം? എത്രകാലമീ ത്രിലോകങ്ങൾ ചുറ്റണം? ഈ രാജ്യം ഭരിച്ചിട്ടെനിക്കെന്തുനേടാൻ? മൂന്നു ലോകങ്ങളിലുമുള്ള എല്ലാം നാശത്തിനു വിധേയമാണെന്നറിഞ്ഞ് നാമെങ്ങിനെയാണു സന്തോഷത്തിനായി ആശിക്കുകപോലും ചെയ്യുക?

“വീണ്ടും വീണ്ടും ആവർത്തിക്കുന്ന വൃത്തികെട്ട സുഖാനുഭവങ്ങളും കർമ്മങ്ങളും ദിനംതോറും സംഭവിക്കുന്നു. എന്നിട്ട് ജ്ഞാനികൾ പോലും അതിൽ ലജ്ജിക്കാത്തതെന്തേ?” ദിനരാത്രങ്ങൾക്ക് വ്യത്യാസങ്ങൾ ഇല്ല. ചുഴിയിലേതുപോലെ ജീവിതമിങ്ങിനെ ചുറ്റുകയാണ്‌.. ഇതിങ്ങിനെ ദിവസവും ആവർത്തിച്ചുകൊണ്ടിരിക്കുമ്പോൾ ഇതിനൊരന്തം കാണാൻ എങ്ങിനെയാണ് സാധിക്കുക? ഈ ചുഴിയിലെ ചംക്രമണം നാമെത്രനാൾകൂടി തുടരണം? എന്താണതിന്റെ ഉദ്ദേശം?

അങ്ങിനെ ആലോചിച്ചിരിക്കേ അദ്ദേഹം ഓർത്തു: ആഹാ, എന്റെ ചോദ്യത്തിനുത്തരമായി അച്ഛൻ വിരോചനൻ പണ്ടിങ്ങിനെ പറഞ്ഞിരുന്നുവല്ലോ? അന്നു ഞാൻ ചോദിച്ചതിങ്ങിനെയാണ്: അച്ഛാ, ഈ പ്രത്യക്ഷ ലോകത്തിന്റെ ലക്ഷ്യമെന്താണ്‌?ഈ ചാക്രികമായ ആവർത്തനങ്ങളുടെ ഉദ്ദേശമെന്താണ്‌? എപ്പോഴാണിതവസാനിക്കുക? എപ്പോഴാണു മനസ്സിന്റെ ഭ്രമമടങ്ങുക? മറ്റൊന്നും തേടാതെ എന്തു നേടിയാലാണൊരുവൻ പരിപൂർണ്ണ തൃപ്തിയടയുക? ഈ ലോകത്തിലെ സുഖാനുഭവങ്ങളോ കർമ്മാദികളോ നമ്മെ ഇതിനു പര്യാപ്തരാക്കുകയില്ല എന്നെനിക്കറിയാം. കാരണം അവ വിഭ്രമത്തെ കൂടുതൽ വഷളാക്കുകയേയുള്ളു. എങ്ങിനെയാണെനിയ്ക്ക് പരമ പ്രശാന്തതയിൽ അഭിരമിക്കാനാവുക എന്നു ദയവായി ഉപദേശിച്ചാലും. 
