\section{ദിവസം 194}

\slokam{
കർത്താ നാസ്മി ന ചാഹമസ്മി സ ഇതി ജ്ഞാത്വൈവമന്ത: സ്ഫുടം\\
കർത്താ ചാ സ്മി സമഗ്രമസ്മി തദിതി ജ്ഞാത്വാഥവാ നിശ്ചയം\\
കോപ്യേവാസ്മി ന കിഞ്ചിദേവമിതി വാ നിർണീയ സർവോത്തമേ\\
തിഷ്ഠ ത്വം സ്വപദേ സ്ഥിതാ: പദവിദോ യത്രോത്തമം സാധവ: (4/56/49)\\
}

വസിഷ്ഠൻ തുടർന്നു: മാമുനിയുടെ വാക്കുകൾകേട്ട് ഞാൻ ആകാശത്തുനിന്നും കദംബവൃക്ഷത്തിന്റെ മുകളിലെ ഒരു ശിഖരത്തില്‍ വന്നിറങ്ങിയിരുന്നു. കുറെയേറെസമയം ഞങ്ങൾ മൂവരും ആത്മവിദ്യയെപ്പറ്റി സംസാരിച്ചിരുന്നു. ഞാനവരിൽ പരമവിജ്ഞാനത്തിന്റെ തിരി തെളിയിക്കുകയും ചെയ്തിട്ടാണ് അവിടെനിന്നും പോയത്. രാമാ ഇക്കഥ പ്രത്യക്ഷലോകത്തിന്റെ സ്വഭാവമെന്തെന്നു കാണിക്കാനായി മാത്രം പറഞ്ഞതാണ്‌.. അതിനാൽ ഈ ലോകമെത്ര സത്യമാണോ അതുപോലെതന്നെയാണ്‌. ഇക്കഥ. അത്ര ഉണ്മയേ നീ ഇക്കഥയില്‍  കാണാവൂ.

നീ ഈ ലോകത്തെയും നിന്നെയും സത്യമെന്നു (ഉണ്മയെന്ന്‍ ) ദൃഢമായി  വിശ്വസിച്ചാൽ ഒരു കുഴപ്പവുമില്ല. അങ്ങിനെയെങ്കില്‍ നീ നിന്റെ ആത്മാവിൽ സ്വയം വിശ്വാസമുറപ്പിച്ചു  വേണം ജീവിക്കാന്‍. പക്ഷേ ഈ ലോകം സത്തും അസത്തും ചേർന്ന സങ്കരമാണെന്നാണു നീ കരുതുന്നതെങ്കിൽ മാറ്റങ്ങൾക്കു വിധേയമായ ഈ ലോകത്തിൽ മാറ്റങ്ങൾക്കനുസൃതമായി മനോഭാവങ്ങളെ ക്രമീകരിച്ചു ജീവിക്കുക. എന്നാൽ ഈ ലോകം അസത്താണെന്ന്‍ നീ ഉറച്ചു വിശ്വസിക്കുന്നുവെങ്കിൽ അനന്താവബോധത്തിൽ ദൃഢമായി മനസ്സുറപ്പിക്കുക. അതുപോലെ ഈ ലോകത്തിനൊരു സൃഷ്ടാവുണ്ടെന്നു നീ വിശ്വസിച്ചാലും ഇല്ലെങ്കിലും അതു നിന്റെ നേരറിവിനെ മലീമസമാക്കാതിരിക്കട്ടെ.

ആത്മാവ് ഇന്ദ്രിയാതീതമത്രേ. ഒരാൾ കർമ്മനിരതനായാലും അതിനെ അവയൊന്നും ബാധിക്കുന്നില്ല. അരാളുടെ ജീവിതകാലം കേവലം നൂറുവർഷം മാത്രം. അപ്പോള്‍പ്പിന്നെ ഈ അനശ്വരനായ ആത്മാവ് ഈ ചെറിയൊരു കാലയളവിൽ ഇന്ദ്രിയസുഖാനുഭവങ്ങൾക്കായി അലയുന്നതെന്തിനാണ്‌? ഈ ലോകവും അതിലെ വസ്തുക്കളും സത്യമാണെങ്കില്‍പ്പോലും ബോധസ്വരൂപമായ ആത്മാവ് ജഢവിഷയങ്ങൾക്കുപുറകേ അലയുന്നു എന്നത് യുക്തിസഹമല്ല. മാത്രമല്ല, ലോകമെന്നത് ഉണ്മയല്ലെങ്കിൽ ദു:ഖമല്ലാതെ യാതൊന്നും ഈ പ്രവർത്തനങ്ങൾകൊണ്ട് ലഭിക്കുകയുമില്ല. നിന്റെ ഹൃദയത്തിലുയരുന്ന ആശകളെ ഉപേക്ഷിച്ചാലും. നീ ഈ ലോകത്തിൽ നീ മാത്രമാണ്‌... ഈ തെളിഞ്ഞ അറിവോടെ ലോകത്ത് വെറുമൊരു ലീലയായി ജീവിച്ച് വിരാജിക്കൂ.

ആത്മാവിന്റെ സാന്നിദ്ധ്യത്തിലാണ്‌ ലോകത്തിലെ എല്ലാം സംഭവിക്കുന്നത്. വിളക്കിന്റെ സാന്നിദ്ധ്യത്തിൽ വെളിച്ചം സഹജമാണല്ലോ. വിളക്കിന്‌ പ്രകാശിക്കുക എന്ന ‘ആഗ്രഹം’ ഇല്ല. എന്നാൽ സ്വയം ഒന്നും ചെയ്യാനാഗ്രഹിക്കുന്നില്ലെങ്കിലും അതിന്റെ സാന്നിദ്ധ്യത്തിലാണെല്ലാം സംഭവിക്കുന്നത്. രണ്ടു മനോഭാവങ്ങളിൽ ഏതെങ്കിലും നിനക്കു സ്വീകരിക്കാം. ഒന്ന്‌: ‘ഞാൻ സർവ്വവ്യാപിയാണ്‌; എനിക്കുചെയ്യേണ്ടതായി ഒന്നുമില്ല’. രണ്ട്: ‘ഞാനാണീ കർമ്മങ്ങളെല്ലാം ചെയ്യുന്നത്; എന്റെ ധർമ്മമാണിതെല്ലാം’. രണ്ടു രീതിയിൽ ചിന്തിച്ചാലും നിനക്ക് പരിപൂർണ്ണമായ ശാന്തിയിൽ, അനശ്വരതയിൽ അഭിരമിക്കാനാവും. രാഗദ്വേഷങ്ങൾ, ആസക്തി-അനാസക്തി തുടങ്ങിയ ദ്വന്ദങ്ങൾ നിന്നെ ബാധിക്കയില്ല. ‘അയാൾ എന്നെ സേവിച്ചു; അയാളെന്നെ ഉപദ്രവിച്ചു’ തുടങ്ങിയ മൂഢചിന്തകൾ നിനക്കുണ്ടാവുകയില്ല.

“അതുകൊണ്ട് രാമാ, ‘ഞാനല്ല ഇതൊന്നും ചെയ്യുന്നത്, ഞാൻ എന്നൊരു വ്യക്തിസത്ത ഇല്ല’ എന്നോ ‘ഞാനാണെല്ലാ കർമ്മങ്ങളും ചെയ്യുന്നത്, ഞാനാണെല്ലാം’ എന്നോ നിനക്കു കരുതാം. അല്ലെങ്കിൽ ‘ഞാൻ ആരാണ്‌?’ എന്ന ആത്മവിചാരംചെയ്ത് എന്നിൽ ആരോപിക്കപ്പെട്ട ഒന്നും 'ഞാൻഅല്ല’ എന്ന സത്യം സാക്ഷാത്കരിക്കാം. ഏറ്റവും ഉയർന്ന ബോധതലമായ ആത്മാവിൽ അഭിരമിച്ചാലും. ആ ബോധതലത്തിലാണ്‌ മഹാത്മാക്കളായ മാമുനിമാരിലെ അഗ്രഗണ്യന്മാർ നിലകൊള്ളുന്നത്.” 
