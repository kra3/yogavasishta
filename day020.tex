\newpage
\section{ദിവസം 020}

\slokam{
സകലലോകചമത്കൃതി കാരിണോപ്യഭിമതം യദി രാഘവചേതസ:\\
ഫലതി നോ തദിമേ വയമേവ ഹി സ്ഫുടതരം മുനയോ ഹതബുദ്ധയ: (1/33/46)\\
}

വാല്‍മീകി പറഞ്ഞു: മനസ്സിലെ മോഹവിഭ്രമങ്ങളെ ഇല്ലാതാക്കുന്ന, വിജ്ഞാനത്തിന്റെ അഗ്നിജ്വാലപോലെയുള്ള, രാമവചനങ്ങള്‍ കേട്ട്‌ രാജസഭയിലുള്ളവര്‍ പ്രചോദിതരായി. അവര്‍ക്ക്‌ തങ്ങളുടെ സംശയങ്ങളും തെറ്റിദ്ധാരണകളും നീങ്ങിയതായി തോന്നി. രാമന്റെ അമൃതസമാനമായ വാക്കുകള്‍ അവര്‍ പാനം ചെയ്തു. രാമന്‍ പറയുന്നതുകേട്ടിരുന്ന അവര്‍ ചിത്രപടങ്ങളിലെ ജീവനില്ലാത്ത രൂപങ്ങളേപ്പോലെ അനങ്ങാതെ അതീവശ്രദ്ധയോടെ ആ വചനങ്ങളില്‍ മുഗ്ദ്ധരായി ലയിച്ചിരുന്നു. ആരൊക്കെയാണ്‌ ശ്രീരാമന്റെ പ്രഭാഷണം ശ്രവിച്ചിരുന്നത്‌? മഹര്‍ഷികളായ വസിഷ്ഠനും വിശ്വാമിത്രനും; മന്ത്രിമാര്‍, രാജകുടുംബത്തിലെ അംഗങ്ങള്‍- മഹാരാജാവ്‌ ദശരഥന്‍, നഗരവാസികള്‍, സന്യാസിമാര്‍, സേവകര്‍, കൂട്ടിലടച്ച കിളികള്‍, അരുമമൃഗങ്ങള്‍, കൊട്ടാരലായത്തിലെ കുതിരകള്‍ എന്നിവരേക്കൂടാതെ ആകാശചാരികളായ ഗന്ധര്‍വ്വന്മാരും, മാമുനികളും, ദേവവൃന്ദവും ആ വാക്കുകള്‍ കേട്ടിരുന്നു. ദേവരാജാവായ ഇന്ദ്രനും, നരകലോകരാജാവും രാമഭാഷണം കേട്ടു. എല്ലാവരും രാമന്റെ വാക്കുകള്‍കേട്ട്‌ ആവേശഭരിതരായി സബാഷ്‌, സബാഷ്‌ എന്നു വിളിച്ചു പറഞ്ഞു. രാമനുമേല്‍ ആകാശത്തുനിന്നും പുഷ്പവൃഷ്ടിയുണ്ടായി. എല്ലാവരും രാമന്റെ വാക്കുകള്‍ ഹര്‍ഷാരവത്തോടെ സ്വീകരിച്ചു.

തീര്‍ച്ചയായും ദേവഗുരുവിനുപോലും അസാധ്യമായ ഒരു സുപ്രഭാഷണമാണ്‌ ത്കച്ചും അനാസക്തനായ രാമന്റെ നാവില്‍ നിന്നുദിച്ചത്‌. "ഈ വാക്കുകള്‍കേള്‍ക്കാനിടയായത്‌ ഞങ്ങളുടെ ഭാഗ്യാതിരേകം മൂലമാണ്‌. അതു കേള്‍ക്കേ, സ്വര്‍ഗ്ഗത്തില്‍പ്പോലും സുഖം എന്ന ഒന്നില്ല എന്നു ഞങ്ങള്‍ക്കു തോന്നി."

ഉത്തമരായ മഹര്‍ഷികള്‍ പറഞ്ഞു: " മഹാത്മാക്കള്‍ രാമന്റെ അതീവഗഹനങ്ങളായ ചോദ്യങ്ങള്‍ക്കു നല്‍കാന്‍ പോകുന്ന ഉത്തരം എല്ലാവരും കേട്ടിരിക്കേണ്ടുന്നതാണ്‌. എല്ലാവരും ദശരഥരാജന്റെ സഭയിലേയ്ക്കു വന്നാലും, വസിഷ്ഠമഹര്‍ഷിയുടെ ഉത്തരങ്ങള്‍ ശ്രദ്ധിച്ചാലും!"

വാല്‍മീകി പറഞ്ഞു: ഇതുകേട്ട്‌ മാമുനിവൃന്ദം സഭയില്‍ വന്നുചേര്‍ന്നു. രാജാവ്‌ അവരെയെല്ലാം ഉചിതമായ ആചാരോപചാരങ്ങളോടെ ആസനസ്ഥരാക്കി. "നമ്മുടെ ഹൃദയത്തില്‍ വിജ്ഞാനമകുടമായ രാമന്റെ ചോദ്യങ്ങള്‍ പരിചിന്തനത്തിനു വിധേയമായില്ലെങ്കില്‍ നമുക്കു തന്നെയാണ്‌ നഷ്ടം. നമുക്കെന്തു ബുദ്ധിവൈഭവങ്ങള്‍ ഉണ്ടെങ്കിലും ഈ ചിന്തകള്‍ നമ്മിലുണരുന്നില്ലെങ്കില്‍ നമ്മുടെ ബോധം നശിച്ചിരിക്കുന്നു എന്നുറപ്പിക്കാം."

(വൈരാഗ്യപ്രകരണം എന്ന ഒന്നാം  ഭാഗം അവസാനിച്ചു )

