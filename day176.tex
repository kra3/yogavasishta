\section{ദിവസം 176}

\slokam{
വിചാരണാസമധിഗതാത്മദീപകോ\\
മനസ്യലം പരിഗലിതേവ ധിരധീ:\\
വിലോകയൻ ക്ഷയഭവനീരസാ ഗതീർ-\\
ഗതജ്വരോ വിലസതി ദേഹപത്തനേ (4/35/69)\\
}

വസിഷ്ഠൻ തുടർന്നു: മനസ്സു തന്നെയാണ്‌ ജീവൻ. സ്വയം സങ്കൽപ്പിച്ചു വിക്ഷേപണം ചെയ്തവയെ മനസ്സിൽ അനുഭവിക്കുകയാണ്‌.. അതിലൂടെ ബന്ധിക്കപ്പെടുകയും ചെയ്യുന്നു. മനസ്സിന്റെ അവസ്ഥയാണ്‌ ജീവന്റെ പുനർജന്മങ്ങളുടെ ഗതിവിഗതികളെ നിശ്ചയിക്കുന്നത്. രാജാവാകാൻ ആഗ്രഹിക്കുന്ന ഒരാൾ താൻ രാജാവായി എന്നു സ്വപ്നത്തിൽ കാണുന്നു. ഒരാൾ തീക്ഷ്ണമായി എന്താഗ്രഹിക്കുന്നുവോ അത് ഉടനെ തന്നെയോ അല്ലെങ്കിൽ കുറേക്കഴിഞ്ഞോ അവനതു ലഭ്യമാകുന്നു. മനസ്സ് മലിനമാണെങ്കിൽ അതിന്റെ പരിണിതഫലവും അങ്ങിനെതന്നെയായിരിക്കും. അതുപോലെ നിർമ്മലമായ മനസ്സിന്റെ ഫലം നിർമ്മലമായിരിക്കും. ഉത്തമനായ ഒരുവൻ എത്ര കഷ്ടപ്പാടുണ്ടെങ്കിലും തന്റെ ആത്മാന്വേഷണത്വര ഉപേക്ഷിക്കുകയില്ല. സത്യത്തിൽ ബന്ധനവും മുക്തിയും ഒന്നും ഉണ്മയല്ല. അനന്താവബോധത്തിൽ ‘ഞാൻ ഈ ദേഹം’ എന്നൊരു ചിന്തയുണ്ടായി. അതാണ്‌ ബന്ധനമായിത്തീർന്നത്. എന്നാൽ ഈ ‘കാണുന്നതെല്ലാം’ സത്യമല്ല എന്നറിയുമ്പോൾ ഒരുവൻ അനന്താവബോധമായി പ്രോജ്വലിക്കുന്നു.

മനസ്സ് സദ്ചിന്തകളാലും തദൃശപ്രവൃത്തികളാലും നിർമ്മലമാക്കുമ്പോൾ അത് അനന്തതയുടെ സ്വഭാവം ആർജ്ജിക്കുന്നു. ശുഭ്രമായ തുണിയിൽ ചായം മുക്കാൻ എളുപ്പമാണല്ലോ. ശുദ്ധമനസ്സിൽ ദേഹം, ആത്മജ്ഞാനം, നിർമമത തുടങ്ങിയ ധാരണകൾ ഉദയം ചെയ്യുമ്പോൾ ലോകമെന്ന ‘കാഴ്ച്ച’ പ്രകടമാവുന്നു. മനസ്സ് പുറംലോകത്തിൽ ആമഗ്നമാവുമ്പോൾ വിശ്വമനസ്സ് അവിടെ നിന്നു പിൻവലിയുന്നു. എന്നാൽ മനസ്സ് എപ്പോൾ വിഷയ-വിഷയി ബന്ധത്തെ ഉപേക്ഷിക്കുന്നുവോ അതപ്പോൾ അനന്തതയിൽ വിലയിക്കുന്നു. മനസ്സിന് അനന്താവബോധത്തിൽ നിന്നും വിട്ട് ഒരു നിലനിൽപ്പില്ല. അത് ആദിയിൽ ഉണ്ടായിരുന്നില്ല. അന്ത്യത്തിലും അതില്ല. വാസ്തവത്തിൽ ഇപ്പോഴും നിലനിൽക്കുന്നില്ല. മനസ്സുണ്ട് എന്നു ചിന്തിക്കുന്നവന്‌ ദു:ഖം ഉറപ്പാണ്‌.. ഈ ലോകം സത്യത്തിൽ ആത്മാവുതന്നെ എന്നറിയുന്നവൻ ദു:ഖത്തിന്റെ മറുകര കണ്ടവനാണ്‌.. അവന്‌ ലോകം ആനന്ദവും മുക്തിയും നല്കുന്നു.

മനസ്സ് എന്നത് ആശയങ്ങളും ധാരണകളും തന്നെയാണ്‌.. അങ്ങിനെയുള്ള മനസ്സില്ലാതായാൽ ആരാണ്‌ ദു:ഖിക്കുക? ദൃഷ്ടിക്കും ദൃഷ്ടാവിനുമിടയ്ക്കുള്ള ഈ അവബോധമാണ്‌ ഉണ്മ. മനസ്സീ പൊരുളിനെ മറയ്ക്കുകയാണ്‌.. മനസ്സു നിലച്ചാൽ സത്യ സാക്ഷാത്കാരമായി. മനോപാധികൾ ഇല്ലാതായാൽപ്പിന്നെ അജ്ഞാനം, ആർത്തി, ആസക്തികൾ, ആശകൾ, വെറുപ്പ്, ഭ്രമം, മൂഢത്വം, ഭയം, സങ്കൽപ്പങ്ങൾ തുടങ്ങി എല്ലാത്തിനും അവസാനമായി. പരിശുദ്ധി, പവിത്രത, നന്മ എന്നിവ ഉദയം ചെയ്ത് അങ്ങിനെയുള്ളയാൾ ആത്മജ്ഞാനത്തിൽ അഭിരമിക്കുന്നു.

“ഏതൊരുവന്റെ ബുദ്ധിമണ്ഡലം എല്ലാ കളങ്കങ്ങളും നീങ്ങി പരിശുദ്ധമാവുകയും, ഹൃദയം ആത്മാന്വേഷണംകൊണ്ടുണ്ടായ പ്രഭയിൽ ദീപ്തമാവുകയും, ധാരണയിൽ ജനനമരണങ്ങളുടെ വ്യർത്ഥതയെപ്പറ്റി അറിവുറയ്ക്കുകയും ചെയ്യുന്നുവോ അവന്‌ പേടിയോ അകാംക്ഷയോ കൂടാതെ ദേഹമെന്ന ഈ നഗരത്തിൽ സുഖമായി വസിക്കാം.“ 

