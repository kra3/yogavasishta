\section{ദിവസം 184}

\slokam{
വിഹരന്തി ജഗത്കേചിന്നിപതന്ത്യുത് പതന്തി ച\\
കന്ദുകാ ഇവ ഹസ്തേന മൃത്യുനാവിരതം ഹതാ: (4/43/25)\\
}

വസിഷ്ഠൻ തുടർന്നു: ഇങ്ങിനെ അനന്താവബോധത്തിന്റെ ശക്തിയാൽ ആകസ്മികമായി പ്രത്യക്ഷ പ്രകടമായത് അനേകം ജീവജാലങ്ങളാണ്‌.. എണ്ണമറ്റ ഈ ജീവജാലങ്ങൾ അവരവരുടെ മനോപാധികളുടെ പരിമിതിയിൽ കുടുങ്ങി കഴിയുന്നു. അവരെ എല്ലാ രാജ്യങ്ങളിലും, അണ്ഡകടാഹത്തിലെ എല്ലായിടത്തും ആലോചിക്കാൻ കഴിയുന്ന എല്ലാ അവസ്ഥകളിലും കാണാം. അതിൽ ചിലർ ഈ യുഗത്തിലെ പുതിയ സൃഷ്ടികളാണ്‌.. മറ്റു ചിലർ പുരാതനമായുള്ളതുമാണ്‌.. ചിലർക്ക് വെറും ഒന്നോ രണ്ടോ ജന്മങ്ങളേ ഉണ്ടായിരുന്നുള്ളു. മറ്റുള്ളവർ അനേക ജന്മങ്ങൾ എടുത്തിരിക്കുന്നു. ചിലർ മുക്തിപദം പ്രാപിച്ചുകഴിഞ്ഞു. പിന്നെച്ചിലർ അദമ്യമായ ദുരിതക്കടലിലാണ്ടു കഴിയുന്നു. ചിലർ ഗഗനചാരികളായ യക്ഷകിന്നരന്മാരാണ്‌.. ചിലർ, ഉപദേവന്മാർ. ഇനിയും ചിലർ ഈ പ്രത്യക്ഷവിശ്വത്തിൽ വെവ്വേറെ തനത് വിഷയങ്ങളുടെ ആദ്ധ്യക്ഷം വഹിക്കുന്ന ദേവതകളുമാണ്‌.. ചിലർ രാക്ഷസർ, ചിലർ പിശാചുക്കൾ. ചിലർ മനുഷ്യകുലത്തിലെ നാലു വർണ്ണാശ്രമങ്ങളിൽപ്പെട്ടവർ. ചിലർ അപരിഷ്കൃതരായ ആദിവാസവർഗ്ഗത്തിൽപ്പെട്ടവർ. ചില ജീവജാലങ്ങൾ ചെടികളും പുൽവർഗ്ഗങ്ങളുമത്രേ. മറ്റുചിലവ വേരായും, കായായും ഇലയായും, വള്ളിച്ചെടികളായും പൂക്കളായും ജീവിക്കുന്നു.

ചിലർ രാജവേഷഭൂഷകൾ അണിഞ്ഞ രാജാക്കന്മാർ, മന്ത്രിമാർ എന്നിങ്ങനെ. ചിലർ കീറിപ്പറിഞ്ഞ വേഷങ്ങളില്‍, മരവുരി ചുറ്റിയ മനുഷ്യർ. അവർ ഭിക്ഷക്കാരോ ആശ്രമവാസികളോ ആവാം. ചിലവ പാമ്പുകൾ, കീടങ്ങൾ, സിംഹങ്ങൾ, പുലികൾ. മറ്റുചിലവ പക്ഷികളും ആനകളും കഴുതകളും. ചിലർ ഐശ്വര്യസമ്പൂർണ്ണമായ ജീവിതം നയിക്കുന്നു. മറ്റുള്ളവർ കഷ്ടപ്പെട്ടുഴറുന്നു. ചിലർ സ്വർഗ്ഗത്തിൽ . ചിലർ നരകത്തിൽ. ചിലർ അങ്ങുയരെ നക്ഷത്രങ്ങളുടെ ലോകത്ത്. മറ്റുചിലർ ഉണക്കമരത്തിന്റെ പൊത്തിലൊളിച്ചു കഴിയുന്നു. ചിലർ ജീവിച്ചിരിക്കുമ്പോൾത്തന്നെ ദേഹാഭിമാനബോധത്തിൽ നിന്നുയർന്ന് മുക്തിപദം നേടിയവരാണ്‌.. അവരുടെയൊപ്പം കഴിയാൻ ഭാഗ്യം സിദ്ധിച്ച ചിലരും ഇക്കൂട്ടത്തിലുണ്ട്. ചിലർ ബുദ്ധിശാലികൾ. മറ്റുചിലർ അതീവമന്ദബുദ്ധികൾ.

രാമാ, ഈ വിശ്വത്തിൽ അനന്തം ജീവജാലങ്ങളുള്ളതുപോലെ മറ്റു വിശ്വങ്ങളിലും എണ്ണമറ്റ ജീവജാലങ്ങൾ വൈവിദ്ധ്യമാർന്നതെങ്കിലും അനുയോജ്യമായ ദേഹവസ്ത്രമെന്ന പുറംതോടിനുള്ളിൽ കഴിയുന്നുണ്ട്. എന്നാൽ അവയെല്ലാം തന്താങ്ങളുടെ മനോപാധികളാൽ ബന്ധിതരാണ്‌.. “ഈ ജീവജാലങ്ങൾ ചിലപ്പോൾ ഉയർന്നുയർന്ന് അല്ലെങ്കിൽ താഴോട്ടു നിപതിച്ച് വിശ്വമാകെ അലയുന്നു. മരണം, പന്തുതട്ടിക്കളിക്കും പോലെ അവരെയിട്ടു കളിപ്പിക്കുന്നു.” സ്വന്തം മനോപാധികളാൽ പരിമിതപ്പെട്ട്, എണ്ണമറ്റ ആശകളാലും ആസക്തികളാലും സ്വാധീനിക്കപ്പെട്ട് അവ ഒരു ശരീരത്തിൽനിന്നും മറ്റൊന്നിലേയ്ക്ക് സഞ്ചരിച്ചുകൊണ്ടേയിരിക്കുന്നു. ആത്മാവിനെക്കുറിക്കുന്ന -സ്വയം അനന്താവബോധമാണെന്ന, സത്യം സാക്ഷാത്കരിക്കുന്നതുവരെ അവരീ യാത്ര തുടർന്നുകൊണ്ടേയിരിക്കും. ആത്മജ്ഞാനമാർജ്ജിച്ചുകഴിഞ്ഞാൽപ്പിന്നെ അവർക്ക് മോഹവിഭ്രാന്തികളിൽ നിന്നു മോചനമായി. പിന്നീട് അവർ ജനന-മരണ ചക്രത്തിന്റെ തലത്തിലേയ്ക്ക് തിരിച്ചു വരുന്നില്ല. 

