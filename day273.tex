\section{ദിവസം 273}

\slokam{
തേനാഹം നാമ നേഹാസ്തി ഭാവാഭാവോപപത്തിമാന്‍\\
അനഹംകാര രൂപസ്യ സംബന്ധ: കേന മേ കഥം   (5/53/15)\\
}

ഉദ്ദാലകന്റെ മനനം തുടര്‍ന്നു: വാസ്തവത്തില്‍ ബോധത്തിന് ഉപാധികളുണ്ടാവുക വയ്യ. അതിനു പരിമിതികളില്ല. അണുവിനേക്കാള്‍ സൂക്ഷ്മാതിസൂക്ഷ്മമാണത്. അതുകൊണ്ട് മനോപാധികള്‍ക്കതിനെ സ്വാധീനിക്കാനാവില്ല. മനസ്സ് അഹംകാരത്തിലും ഇന്ദ്രിയങ്ങളുടെ ഛായാ ബോധത്തിലും അഭിരമിക്കുന്നു. അതില്‍ നിന്നാണ് സ്വയാര്‍ജിതമായ  പരിമിതിയെപ്പറ്റി ബോധമുടലെടുക്കുന്നത്. ഈ അനുഭവത്തെ ആവര്‍ത്തിച്ചു വിചാരം ചെയ്ത് അഹംഭാവവും സ്വപരിമിതിയെന്ന ഭ്രമകല്‍പ്പനയും   സ്വയമൊരസ്ഥിത്വം സാധൂകരിച്ചു കല്‍പ്പിക്കുകയാണ്.

എന്നാല്‍ അവയ്ക്കൊന്നിനാലും തോടാനരുതാത്ത അവബോധമാണ് ഞാന്‍.. അതുണ്ടാകാനിടയായ അജ്ഞാനം ഹേതുവായി ശരീരം കര്‍മ്മങ്ങളില്‍ ആമഗ്നമായികൊള്ളട്ടെ, അല്ലെങ്കില്‍ കര്‍മ്മങ്ങളെ ഉപെക്ഷിച്ചുകൊള്ളട്ടെ. അവയാലൊന്നും ബാധിക്കപ്പെടാത്ത അനന്താവബോധമാണ് ഞാന്‍.. അവബോധത്തിന് ജനനമരണങ്ങളോ ഉടമസ്ഥരോ ഇല്ല. അത് സര്‍വ്വവ്യാപിയും അനന്തവുമാണ്. സ്വയം വിഭിന്നമായി ജീവിച്ചതുകൊണ്ടതിനെന്തെങ്കിലും നേടുവാനില്ല. ജനന-മരണാദികള്‍ മനോകല്‍പ്പിതങ്ങള്‍ മാത്രം . ആത്മാവുമായി അതിനു ബന്ധമില്ല. അഹംഭാവത്തിനാല്‍ പരിമിതപ്പെട്ടതിനെ മാത്രമേ ‘അറിയാന്‍’ കഴിയൂ. പരിമിതപ്പെടുത്താന്‍ കഴിയൂ. ആത്മാവ് പരിമിതികള്‍ക്കെല്ലാം അതീതമാണ്. ഭാവാഭാവങ്ങള്‍ക്കുമതീതം. അഹംഭാവമെന്നത് വൃഥാഭ്രമം. മനസ്സെന്നത് മരീചിക. പ്രപഞ്ചവസ്തുക്കള്‍ വെറും ജഢം. ആരാണവയില്‍ ‘ഇത് ഞാന്‍’ എന്നവകാശപ്പെടുന്നത്?

ദേഹം രക്താസ്ഥിസഞ്ചയം. മനസ്സ് ആത്മാന്വേഷണത്തില്‍ ഇല്ലാതാവുന്നു. ബോധത്തിലെ സ്വാര്‍ജ്ജിതപരിമിതികളായ ധാരണാ കല്‍പ്പനകള്‍ വെറും ജഢം. അപ്പോള്‍പ്പിന്നെ ഈ അഹം എന്താണ്? ഇന്ദ്രിയങ്ങള്‍ അവയുടെ സ്വാഭീഷ്ടങ്ങള്‍ സാധിതമാക്കി എപ്പോഴും വര്‍ത്തിക്കുന്നു. പ്രപഞ്ചവസ്തുക്കള്‍ ലോകവിഷയങ്ങളാണ്. പിന്നെ അഹമെവിടെ?
പ്രകൃതി അതിന്റെ നൈസര്‍ഗ്ഗിക സ്വഭാവങ്ങളനുസരിച്ച് പരസ്പര പൂരകങ്ങളായി വര്‍ത്തിക്കുന്നു. കണ്ണ് വെളിച്ചമായും കാത് ശബ്ദമായും ചേര്‍ന്നിരിക്കുന്നു. സ്വയം നിലകൊള്ളുന്നതെന്താണ്? അഹമെവിടെയാണ്?

ആത്മാവ്, എല്ലാവരുടെയും പരമാത്മാവായി എല്ലാടവും എല്ലായ്പ്പോഴും നിലകൊള്ളുന്നു. ഞാനാരാണ്? എന്താണെന്റെ രൂപം? എന്നെയുണ്ടാക്കിയതെന്തു വസ്തുകൊണ്ടാണ്? ആരാണുണ്ടാക്കിയത്? ഞാനെന്താണ് ഗ്രഹിക്കേണ്ടതും ത്യജിക്കേണ്ടതും? “ ‘ഞാന്‍’ എന്ന് വിളിക്കാന്‍ പര്യാപ്തമായ ഒന്നും ഇല്ല. ഭാവാഭാവങ്ങള്‍ക്ക് വിധേയമാവാത്ത ഒന്നുമില്ല.  അഹം എന്ന വസ്തു വാസ്തവത്തില്‍ ഇല്ലാത്തതാണെന്നിരിക്കേ ആ അഹവുമായി ആരെങ്ങിനെയാണ് ബന്ധപ്പെട്ടിരിക്കുക?”

ഇങ്ങിനെ യാതൊരുവിധ ബന്ധങ്ങളും സത്തല്ല എന്നറിയുമ്പോള്‍ ദ്വന്ദത എന്ന തെറ്റിദ്ധാരണ ഇല്ലാതെയാകുന്നു. അപ്പോള്‍പ്പിന്നെ ആകെയുള്ളത് ഒരേയൊരു വിശ്വസത്തമാത്രം. ബ്രഹ്മം, ആത്മാവ് എന്നെല്ലാം വിളിക്കപ്പെടുന്ന ആ ഒന്ന്‍.. ഞാനാണാ സത്ത. പിന്നെയെന്തിനു ഭ്രമങ്ങളില്‍പ്പെട്ടുഴറണം? സര്‍വ്വവ്യാപിയായ നിത്യശുദ്ധ ബോധം മാത്രമുള്ളപ്പോള്‍ അഹം എന്ന ധാരണതന്നെ എങ്ങിനെ സത്യമാവാനാണ്?

യാതൊരു വസ്തുക്കള്‍ക്കും സത്യത്തില്‍ സത്തയില്ല. ആത്മാവ് മാത്രമാണുണ്മ. വസ്തുക്കള്‍ക്ക് എന്തെങ്കിലും സത്ത കല്‍പ്പിച്ചു നല്‍കിയാല്‍ത്തന്നെ അവയുമായി ആത്മാവിനു ബന്ധുതയൊന്നുമില്ല. ഇന്ദ്രിയങ്ങള്‍ , മനസ്സ്, എന്നിവ അവയവയുടെ ധര്‍മ്മം ചെയ്യുന്നു. എന്നാല്‍ ബോധം അവയെ ബാധിക്കുന്നില്ല. എന്താണീ ബന്ധങ്ങള്‍ ? അവയെങ്ങിനെ സംജാതമായി? ഒരു കല്ലും ഇരുമ്പു ദണ്ഡും അടുത്തടുത്ത് കിടക്കുന്നതുപോലെ, അവ അടുത്തടുത്ത് ആപേക്ഷികമായി നില്‍ക്കുന്നു എന്നുള്ളതുകൊണ്ട് മാത്രം അവ തമ്മില്‍ ഏതെങ്കിലും ബന്ധം ആരോപിക്കുന്നത് ശരിയല്ല.