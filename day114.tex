 
\section{ദിവസം 114}

\slokam{
മനസൈവ മനസ്‌തസ്മാത്‌പൌരുഷേണ പുമാനിഹ\\
സ്വകമേവ സ്വകേനൈവ യോജയേത്‌പാവനേ പഥി  (3/92/28)\\
}

വസിഷ്ഠന്‍ ബ്രഹ്മാവിനോടു ചോദിച്ചു: ഭഗവാനേ, മാമുനിയുടെ ശാപം ഇന്ദ്രന്റെ മനസ്സിനെ ബാധിക്കാതെ ശരീരത്തെമാത്രം ബാധിക്കാനിടവന്നതെന്തുകൊണ്ടാണ്‌? ശരീരം മനസ്സില്‍നിന്നു വിഭിന്നമല്ലെങ്കില്‍ ആ ശാപം മനസ്സിനേയും ബാധിക്കേണ്ടതല്ലേ?. എങ്ങിനെയാണ്‌ മനസ്സ്‌ ബാധിക്കപ്പെടുന്നതെന്നും ബാധിക്കപ്പെടാതിരിക്കുന്നതെന്നും ദയവായി പറഞ്ഞു തരൂ.

ബ്രഹ്മാവു പറഞ്ഞു: മകനേ, ഈ വിശ്വം, ബ്രഹ്മാവു മുതല്‍ മാമലകള്‍ വരെ മൂര്‍ത്തിമത്തായുള്ള എല്ലാറ്റിലും രണ്ടുതരം ശരീരങ്ങളുണ്ട്‌. ഒന്നാമത്തേത്‌ മാനസീകശരീരം- അത്‌ അശാന്തമാണ്‌.. ക്ഷിപ്രമാണതിന്റെ പ്രവര്‍ത്തനങ്ങള്‍ . രണ്ടാമത്തേത്‌ മാംസനിബദ്ധമായ ശരീരം. അത്‌ സ്വയമായി ഒന്നും ചെയ്യുന്നില്ല. ഈ ശരീരത്തെയാണ്‌ അനുഗ്രഹങ്ങളും ശാപങ്ങളും ബാധിക്കുക. അതിനു ബുദ്ധിയോ ശക്തിയോ ഇല്ല. താമരയിലയിലെ ജലകണം പോലെ അതു ചഞ്ചാടിക്കൊണ്ടിരിക്കുന്നു. അത്‌ വിധിയെ, നിയതിയെ, മറ്റു ഘടകങ്ങളെ എല്ലാം ആശ്രയിച്ചിരിക്കുന്നു. മനസ്സാകട്ടെ ബന്ധിതമെന്നു തോന്നുമെങ്കിലും സ്വതന്ത്രമാണ്‌.. ഈ മനസ്സ്‌ ധൈര്യപൂര്‍വ്വം സ്വപ്രയത്നങ്ങളില്‍ ഏര്‍പ്പെടുമ്പോള്‍ ദു:ഖങ്ങള്‍ക്കതീതമാണ്‌.. എപ്പോഴെല്ലാം ഉദ്യമങ്ങളില്‍ ഏര്‍പ്പെടുന്നുവോ അപ്പോഴെല്ലാം ആ ഉദ്യമങ്ങള്‍ക്ക്‌ ഉചിതമായ ഫലം കാണുന്നുമുണ്ട്‌.. ഭൌതീകശരീരം ഒന്നും നേടുന്നില്ല. മാനസീകശരീരമാണ്‌ എല്ലാം നേടുന്നത്‌.. മനസ്സ്‌ എപ്പോഴും ശുദ്ധബോധത്തില്‍ ഉറച്ചിരിക്കുമ്പോള്‍ അതിനെ ശാപങ്ങള്‍ ബാധിക്കയില്ല. ശരീരം തീയിലോ ചെളിയിലോ വീണേക്കാം. എന്നാല്‍ അതിനെക്കുറിച്ച്‌ ചിന്തിച്ചാല്‍ മാത്രമേ മനസ്സിലത് അനുഭവമായി തീരുകയുള്ളു. ഇതാണ്‌ ഇന്ദ്രന്റെ അനുഭവത്തിലൂടെ നമുക്കറിയാന്‍ കഴിയുന്നത്‌..

ദീര്‍ഘതപന്‍ എന്നു പേരായ മുനിയുടെ അനുഭവവും ഇക്കാര്യം വെളിപ്പെടുത്തുന്നു. അദ്ദേഹം ഒരു യാഗം നടത്താന്‍ ആഗ്രഹിച്ചിരുന്നു. എന്നാല്‍ അതിനുവേണ്ട സാമഗ്രികള്‍ സംഘടിപ്പിക്കുന്നതിനിടയില്‍ അദ്ദേഹമൊരു കിണറ്റില്‍ വീണുപോയി. അദ്ദേഹം മനസ്സില്‍ യാഗം പൂര്‍ത്തീകരിച്ച്‌ ഭൌതീകമായി യാഗം ചെയ്താലുണ്ടാവുന്ന ഫലങ്ങളെല്ലാം നേടി. നേരത്തേ പറഞ്ഞ മഹാത്മാവിന്റെ പത്തു പുത്രന്മാര്‍ക്ക്‌ ബ്രഹ്മപദവി ലഭിച്ചതും സ്വമന:പ്രയത്നത്താല്‍ ആണല്ലോ. മാനസീകമോ ശാരീരികമോ ആയ അസുഖങ്ങള്‍ , ശാപങ്ങള്‍ , കണ്ണേറ്‌  തുടങ്ങിയവ ആത്മാഭിമുഖമായിരിക്കുന്ന മനസ്സിനെ ബാധിക്കുകയില്ല. പാറക്കല്ലില്‍ താമരപ്പൂ വീണാല്‍ അതു രണ്ടായിപൊട്ടിപ്പൊളിയുകില്ലല്ലോ. "അതുകൊണ്ട്‌ ഒരുവന്‍ മനസ്സിനെ മനസ്സുകൊണ്ടുതന്നെ ശുദ്ധപാതയിലേയ്ക്കു നയിക്കണം. ആത്മാവിനെ അത്മാവുകൊണ്ടുതന്നെ സദ്മാര്‍ഗ്ഗങ്ങളിലേയ്ക്കുന്മുഖമാക്കണം"

മനസ്സ്  വിചാരിക്കുന്നതെന്തോ അത്‌ അപ്പോള്‍ത്തന്നെ നടപ്പിലാവുന്നു. തീവ്രമായ ധ്യാനംകൊണ്ട്‌ ഒരുവനുള്ളില്‍ സമൂലമായ മാറ്റം വരുത്താനും തെറ്റായ കഴ്ച്ചപ്പാടുമൂലം ഉണ്ടായ ഭ്രമകല്‍പ്പനയില്‍ നിന്നു വിമുക്തിയാര്‍ജ്ജിക്കാനും നമുക്ക്  സാധിക്കും. മനസ്സു ചെയ്യുന്നതാണ്‌ സത്യമായി അനുഭവപ്പെടുന്നത്‌. നിലാവിന്റെ ശീതളിമയിലിരിക്കുന്നവന്‌ സൂര്യന്റെ തീവ്രതാപം മനസാ അനുഭവവേദ്യമാവാം. അതുപോലെ തിരിച്ചും. മനസ്സിന്റെ മായികശക്തി ഇപ്രകാരമെല്ലാമാണ്‌.. 

