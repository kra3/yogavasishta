 
\section{ദിവസം 085}

\slokam{
തീവ്രവേഗവതീ യാ സ്യാതത്ര സംവിദകമ്പിതാ\\
സൈവായാതി പരം സ്ഥൈര്യമാമോക്ഷം ത്വേകരൂപിണീ (3/60/53)\\
}

വസിഷ്ഠന്‍ തുടര്‍ന്നു: ഒരുവന്റെ മാനസീകഭാവങ്ങളാണ്‌ മധുരമുള്ളതിനെ കയ്പ്പു നിറഞ്ഞതാക്കുന്നതെന്നും, കയ്പ്പിനെ മധുരമുള്ളതാക്കുന്നതെന്നും യോഗിക്കറിയാം അങ്ങിനെ തന്നെയാണ്‌ സുഹൃത്തിനെ ശത്രുവാക്കുന്നതും ശത്രുവിനെ സുഹൃത്താക്കുന്നതും. അതുപോലെ സ്വയം നമ്മുടെ ദൃഷ്ടികോണ്‍ ഒന്നു മാറ്റിയും നിരന്തരമായ അഭ്യാസം കൊണ്ടും നേരത്തേ തീരെ താത്പ്പര്യമില്ലാതിരുന്ന വേദശാസ്ത്രങ്ങളിലേയ്ക്കും പൂജാദികളിലേയ്ക്കും മറ്റും നമ്മുടെ ശ്രദ്ധയെ തിരിക്കാന്‍ സാധിക്കും. കാരണം സ്വഭാവങ്ങള്‍ വിഷയങ്ങളില്‍ അല്ല. നമ്മുടെ ചിന്തകളിലാണുള്ളത്‌. കടല്‍ച്ചൊരുക്കു ബാധിച്ചവന്‌ ലോകം മുഴുവന്‍ ചുറ്റുന്നതായി തോന്നുന്നതുപോലെ അജ്ഞാനിക്ക്‌ വിഷയവസ്തുക്കളിലാണ്‌ സ്വഭാവസവിശേഷതകള്‍ കുടികൊള്ളുന്നതെന്നു തോന്നുകയാണ്‌.. മദ്യപിച്ചു മതികെട്ടവന്‌ ചുമരുള്ളിടത്ത്‌ ശൂന്യകാശം കാണാകുന്നു. മോഹവിഭ്രമം ബാധിച്ചവനെ നിലവിലില്ലാത്ത പിശാചുക്കള്‍ പിടികൂടി കൊല്ലുന്നു.

ഈ ലോകമെന്നത്‌ ആകാശത്തില്‍ ബോധത്തിന്റെ കമ്പനങ്ങള്‍ മാത്രമാണ്‌.. അജ്ഞാനിക്ക്‌ ഭൂതപിശാചുക്കളെന്നപോലെ, കാണപ്പെടുന്നുണ്ടെങ്കിലും 'ഇല്ലാത്ത'താണത്‌..  ഇതെല്ലാം മായയാണ്‌.. അതിനാല്‍ അനന്താവബോധവും വിശ്വത്തിന്റെ പ്രകടിതമായ അസ്തിത്വവും തമ്മില്‍ പൊരുത്തക്കേടൊന്നുമില്ല തന്നെ. ഉണര്‍ന്നു കിടക്കുന്നവന്റെ മഹത്തായ സ്വപ്നം പോലെയാണത്‌. രാമ:, ശിശിരത്തില്‍ മരങ്ങളിലെ ഇലകള്‍ കൊഴിയുന്നു. വസന്തത്തില്‍ അതേ മരങ്ങളില്‍ ഇലകള്‍ തളിര്‍ത്തു വളരുന്നു. അവ മരത്തിനുള്ളില്‍ ലീനമായി ഉണ്ടായിരുന്നതുതന്നെയാണ്‌. അതുപോലെ അനന്താവബോധത്തില്‍ അനുസ്യൂതമായി സൃഷ്ടി തുടര്‍ന്നുകൊണ്ടിരിക്കുന്നു. അത്‌ സ്പഷ്ടമായി കാണുന്നില്ല എന്നുമാത്രം. സ്വര്‍ണ്ണം ദ്രാവകാവസ്ഥയില്‍ കാണുന്നത്‌ അത്ര സാധാരണമല്ലല്ലോ. ഒരു യുഗത്തിലെ ബ്രഹ്മാവ്‌ മുക്തിപദം പ്രാപിച്ച്‌ അടുത്ത യുഗത്തിലെ ബ്രഹ്മാവ്‌ ഓര്‍മ്മയില്‍ നിന്നും പുതിയൊരു ലോകത്തെ വിക്ഷേപിക്കുമ്പോള്‍ ആ ഓര്‍മ്മപോലും അനന്താവബോധം മാത്രമാണ്‌.

രാമന്‍ ചോദിച്ചു: അങ്ങിനെയെങ്കില്‍ രാജാവും നഗരവാസികളും എങ്ങിനെയാണ്‌ ഒരേ വിഷയവസ്തുക്കളെ വസ്തുനിഷ്ഠമായി അനുഭവിക്കാനിടയായത്‌? 

വസിഷ്ഠന്‍ പറഞ്ഞു: എല്ലാ ജീവന്റെയും മേധാശക്തി അനന്താവബോധത്തിനെ ആശ്രയിച്ചായതുകൊണ്ടാണിങ്ങിനെ സംഭവിക്കുന്നത്‌. രാമാ, നഗരവാസികളും അദ്ദേഹം അവരുടെ രാജാവാണെന്നു തന്നെ ചിന്തിച്ചു. ചിന്തകളുടെ ആന്ദോളനം അനന്താവബോധത്തില്‍ സഹജമാണ്‌. എന്നാല്‍ അത്‌ യാതൊരു പ്രേരണകള്‍ക്കും വശംവദമായല്ല സംഭവിക്കുന്നത്‌. വജ്രത്തിനു തിളക്കം സഹജം. രാജാവിന്റെ ബുദ്ധിയില്‍ 'ഞാന്‍ വിഥുരഥ രാജാവ്‌' എന്ന ചിന്തയുണ്ടാവുന്നു. അപ്രകാരം ജീവജാലങ്ങള്‍ക്കും വിശ്വത്തിനു മുഴുവനും 'അഹം' ചിന്തയുണ്ടാവുന്നു. "ഒരുവന്റെ പ്രജ്ഞ അനന്താവബോധത്തെക്കുറിച്ചുള്ള ഈ സത്യസ്ഥിതിയില്‍ സ്ഥിരപ്രതിഷ്ഠമായാല്‍ അത്‌ പരമമായ മുക്തിപദത്തില്‍ എത്തിച്ചേരുന്നു." ഇത്‌ ഒരുവന്റെ സ്വപ്രയത്നത്തെ ആശ്രയിച്ചിരിക്കുന്നു. മനുഷ്യനെ ഒരേസമയം രണ്ടു ശക്തികള്‍ വിവിധ ദശകളിലേയ്ക്കു വലിക്കുകയാണ്‌. ഒന്ന് ബ്രഹ്മത്തെ പരമസത്യമായി സാക്ഷാത്കരിക്കുന്നതിനുതകുന്നതും മറ്റേത്‌ ലോകത്തെ ഉണ്മയായി സ്വീകരിക്കുന്ന അജ്ഞതയിലേയ്ക്കുള്ളതുമാണ്‌. ഏതു ദിശയിലേയ്ക്കാണൊരുവന്‍ തന്റെ തിവ്ര പ്രയത്നമര്‍പ്പിക്കുന്നത്‌, ആ ദിശയ്ക്കു വിജയമുണ്ടാവുന്നു. ഒരിക്കല്‍ അവിദ്യ നീങ്ങിക്കഴിഞ്ഞാല്‍ അസത്തായ ഭ്രമദൃശ്യങ്ങള്‍ എന്നെന്നേയ്ക്കുമായി ഇല്ലാതാവുന്നു.
