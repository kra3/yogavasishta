\newpage
\section{ദിവസം 077}

\slokam{
കോധ്യയാവന്മൃതം ബ്രൂഹി ചേതനം കസ്യ കിം കഥം\\
മ്രിയന്തേ ദേഹലക്ഷാണി ചേതനം സ്ഥിതമക്ഷയം (3/54/69)\\
}

സരസ്വതി തുടര്‍ന്നു: ആദ്യത്തെ സൃഷ്ടിയിലെ ക്രമനിയമമനുസരിച്ച്‌ മനുഷ്യന്‌ നൂറ്‌,ഇരുന്നൂറ്‌,മുന്നൂറ്‌,നാനൂറ്‌, എന്നിങ്ങനെ വര്‍ഷങ്ങള്‍ ആയുസ്സുണ്ടായിരുന്നു. ജീവിച്ചിരിക്കുന്ന രാജ്യം, കാലം,കര്‍മ്മങ്ങള്‍ , ഉപയോഗിക്കുകയും ആഹരിക്കുകയും ചെയ്യുന്ന പദാര്‍ത്ഥങ്ങള്‍ എന്നീ ഘടകങ്ങളെ ആശ്രയിച്ചാണ്‌ മനുഷ്യന്റെ ആയുര്‍ദൈര്‍ഘ്യം ക്രമീകൃതമായിരുന്നത്‌. വേദഗ്രന്ഥങ്ങളിലെ ശാസ്ത്രവിധിപ്രകാരം ജീവിക്കുന്നവര്‍ക്ക്‌ ആ വേദങ്ങളില്‍പറഞ്ഞ പ്രകാരം ആയുസ്സുണ്ടായിരുന്നു.അങ്ങിനെ മനുഷ്യന്‍ ചെറുതോ നീണ്ടതോ ആയ ജീവിതം നയിച്ചശേഷം മരണപ്പെടുന്നു.

പ്രബുദ്ധയായ ലീല ചോദിച്ചു:ദേവീ മരണത്തെപറ്റി എനിക്കു കൂടുതല്‍ അറിയണമെന്നുണ്ട്‌. ആ യാത്ര സന്തോഷപ്രദമാണോ അല്ലയോ? മരണശേഷമെന്തു സംഭവിക്കുന്നു?

സരസ്വതി പറഞ്ഞു: കുഞ്ഞേ,മൂന്നു തരത്തിലാണ്‌ മനുഷ്യര്‍ . ഒന്ന്,മൂഢന്‍. രണ്ട്‌, ഏകാഗ്രതയും ധ്യാനവും സാധനയാക്കിയവന്‍. മൂന്ന്, യോഗി. ഇതില്‍ മൂന്നാമനാണ്‌ ബുദ്ധിമാന്‍. രണ്ടാമനും മൂന്നാമനും ഏകാഗ്രതയും ധ്യാനവും ശീലമാക്കി ശരീരബുദ്ധിയെ ത്യജിച്ച്‌ അവരുടെ ഇഷ്ടാനുസരണം സന്തോഷത്തോടെ ജീവിതത്തോട്‌ യാത്രപറയുന്നു. എന്നാല്‍ മൂഢന്‍ ധ്യാനാദി   പരിശീലനങ്ങളൊന്നുമില്ലാത്തതിനാല്‍ തനിക്കു പുറത്തുള്ള ശക്തികളുടെ ദയവിലാണവന്‍. അവന്‍ മരണമടുക്കുമ്പോള്‍ ഭയചകിതനാവുന്നു. അവന്റെയുള്ളില്‍ വല്ലാത്തൊരെരിച്ചില്‍ ഉണ്ടാവുന്നു. ശ്വാസം കഴിക്കാന്‍ മുദ്ധിമുട്ടനുഭവപ്പെടുകയും ശരീരം വിളര്‍ക്കുകയും ചെയ്യും. അവന്‍ കട്ടപിടിച്ച ഇരുട്ടിന്റെ ലോകത്ത്‌ പ്രവേശിച്ച്‌ പകലിലും നക്ഷത്രങ്ങളെ കാണുന്നു. അവന്റെതല ചുറ്റുന്നു.അവന്റെ കാഴ്ച്ചയില്‍ കുഴപ്പമുണ്ടാവുന്നു. ആകാശത്തെ ഭൂമിയായും ഭൂയെ ആകാശമായും അവനു തോന്നുന്നു. പലേ രീതികളില്‍ വിഭ്രമം അവനെ ബാധിക്കുന്നു. താന്‍ ഒരു കിണറിനുള്ളിലേയ്ക്കു പതിക്കുന്നതായും, ഒരു കല്ലിന്റെയുള്ളില്‍ നുഴയുന്നതായും അവന്‍ അനുഭവിക്കുന്നു. അതിവേഗമൊരു വാഹനത്തില്‍ സഞ്ചരിക്കുന്നതായും, മഞ്ഞുപോലെ ഉരുകുന്നതായും, കയറുകൊണ്ട്‌ ആരോ തന്നെ കെട്ടി വലിക്കുന്നതായും, തന്‍ കാറ്റിലൊരു പുല്‍ക്കൊടിപോലെ പാറിപ്പറക്കുന്നതായും അവനു തോന്നുന്നു. തന്റെ ദുരിതങ്ങള്‍ പറയണമെന്നവനുണ്ട്‌, എന്നാല്‍ അവനതിനു കഴിയുന്നില്ല. ക്രമത്തില്‍ അവന്റെ ഇന്ദ്രിയങ്ങള്‍ പ്രവര്‍ത്തനരഹിതമാവുന്നു. ചിന്തിക്കാന്‍ കൂടി അവനു കഴിയാതാവുന്നു. അവന്‍ അജ്ഞതയിലേയ്ക്കും അവിദ്യയിലേയ്ക്കും ആഴ്ന്നാഴ്ന്നു പോവുന്നു.

ഉദ്ബുദ്ധയായ ലീല വീണ്ടും ചോദിച്ചു: എല്ലാവര്‍ക്കും എട്ട്‌ അവയവങ്ങളുണ്ടായിരുന്നിട്ടു കൂടി എന്തുകൊണ്ടാണ്‌ അവന്‌ ഈ അജ്ഞാനവും ദു:ഖവുമെല്ലാം അനുഭവിക്കേണ്ടിവരുന്നത്‌? 

സരസ്വതി പറഞ്ഞു: അങ്ങിനെയാണ്‌ സൃഷ്ടിയുടെ സമാരംഭത്തിലുണ്ടാക്കിയ ക്രമീകരണം. പ്രാണവായുവിന്റെ സഞ്ചാരം ഇല്ലാതാവുമ്പോള്‍ മരണമായി. എന്നാല്‍ ഇതെല്ലാം സാങ്കല്‍പ്പികമാണെന്നറിയുക. അനന്തതയ്ക്ക്‌ എങ്ങിനെയാണ്‌ അവസാനമുണ്ടാവുക? വ്യക്തി എന്നത്‌ അനന്താവബോധമല്ലാതെ മറ്റൊന്നല്ല. "എപ്പോള്‍ , ആരാണു മരിക്കുന്നത്‌? ഈ അനന്താവബോധം ആരുടേതാണ്‌?എങ്ങ്‌ിനെയാണത്‌? കോടിക്കണക്കിനാളുകള്‍ മരണമടഞ്ഞിട്ടും ഈബോധം മാത്രം കുറവൊന്നും ഇല്ലാതെ നിലകൊള്ളൂന്നു."
