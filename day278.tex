\section{ദിവസം 278}

\slokam{
പ്രശാന്ത ജഗദാസ്ഥോഽന്തര്‍വീതശോകഭയൈഷണ:\\
സ്വസ്ഥോ ഭവതി യേനാത്മാ സ സമാധിരിതി സ്മൃത: (5/56/20) \\
}

വസിഷ്ഠന്‍ തുടര്‍ന്നു: ഇങ്ങിനെ ജീവിച്ച് ആത്മാവിന്റെ സഹജഭാവത്തെപ്പറ്റി അന്വേഷിച്ച് പ്രശാന്തത കൈവരിച്ചാലും. അനാസക്തി പരിശീലനം, ശാസ്ത്രപഠനം, സദ്‌ഗുരുവില്‍നിന്നും കിട്ടുന്ന മാര്‍ഗ്ഗനിര്‍ദ്ദേശങ്ങള്‍ , നിരന്തരമായ ആത്മാന്വേഷണം എന്നിവകൊണ്ട് ഈ ബോധാവസ്ഥയെ പ്രാപിക്കാം. എന്നാല്‍ ബുദ്ധിശക്തി സൂക്ഷ്മവും മൂര്‍ച്ചയേറിയതുമാണെങ്കില്‍ ഇവയൊന്നും കൂടാതെ തന്നെ നിനക്ക് ആ അഭൌമബോധതലത്തിലെത്താം.

രാമന്‍ ചോദിച്ചു: മഹര്‍ഷേ, ചിലര്‍ ആത്മജ്ഞാനത്തില്‍ പ്രബുദ്ധരായി അഭിരമിച്ചിരിക്കെതന്നെ കര്‍മ്മനിരതരായിരിക്കുന്നതായി കാണുന്നു. എന്നാല്‍ മറ്റുചിലര്‍ ഒറ്റപ്പെട്ട ജീവിതം നയിച്ച് ധ്യാനസമാധിസ്ഥരായും കഴിയുന്നു. ഈ രണ്ടു മാര്‍ഗ്ഗങ്ങളില്‍ ഏതാണ് കൂടുതല്‍ അഭികാമ്യം?

വസിഷ്ഠന്‍ പറഞ്ഞു: ധ്യാന-സമാധിയില്‍ ഒരുവന്‍ ഇന്ദ്രിയ വസ്തുക്കളെ അനാത്മാവായി തിരിച്ചറിഞ്ഞു അകമേ ശാന്തിയും സമാധാനവും എപ്പോഴും അനുഭവിക്കുന്നു. ഇങ്ങിനെ വിഷയവസ്തുക്കള്‍ മനസ്സിനെ സംബന്ധിച്ചത് മാത്രമാണെന്ന അറിവിന്റെ നിറവില്‍ , സദാ പ്രശാന്തതയോടെ ചിലര്‍ കര്‍മ്മനിരതരാവുന്നു, മറ്റു ചിലര്‍ ഏകാന്തവാസം നയിക്കുന്നു.  രണ്ടു കൂട്ടര്‍ക്കും സമാധിയുടെ ആനന്ദം ഉണ്ടാവുന്നുണ്ട്.        

സമാധിസ്ഥനായ ഒരുവന്റെ മനസ്സില്‍ ചാഞ്ചല്യമുണ്ടായി അയാളുടെ ശ്രദ്ധ വഴിതിരിഞ്ഞുപോയാല്‍ അയാള്‍ ഭ്രാന്തനാണ്. എന്നാല്‍ ഭ്രാന്തനായി കാണപ്പെടുന്ന ചിലര്‍ എല്ലാ ധാരണാസങ്കല്‍പ്പങ്ങളില്‍ നിന്നുമൊഴിഞ്ഞ് പ്രബുദ്ധതയോടെ അവിച്ഛിന്നമായ സമാധിസ്ഥിതിയില്‍ ആയിരിക്കുകയും ചെയ്യും. കര്‍മ്മങ്ങളില്‍ ഏര്‍പ്പെടുന്നുവോ ഇല്ലയോ എന്നത് പ്രബുദ്ധതയെ സംബന്ധിച്ച് പ്രാധാന്യമുള്ള കാര്യമല്ല. അത് യാതൊരു വ്യത്യാസവും ഉണ്ടാക്കുന്നില്ല.
   
മനോപാധികളില്ലെങ്കില്‍പ്പിന്നെ യാതൊരു കര്‍മ്മവും കളങ്കമുണ്ടാക്കുന്നില്ല. എന്നാല്‍ മനസ്സിലെ കര്‍മ്മരാഹിത്യമാണ് സമാധാനം. അതാണ്‌ പൂര്‍ണ്ണ സ്വാതന്ത്ര്യം. ആനന്ദാനുഗ്രഹമാണത്. ധ്യാനാവസ്ഥയും അതല്ലാത്ത അവസ്ഥയും തമ്മില്‍ ഉള്ള വ്യത്യാസമറിയുന്നത് മനസ്സില്‍ ചിന്താസഞ്ചാരങ്ങളുണ്ടോ എന്ന് നോക്കിയാണ്. അതുകൊണ്ട് മനോപാധികളെ ഇല്ലാതാക്കൂ. ഉപാധികളില്ലാത്ത മനസ്സ് ദൃഢമാണ്. അതുതന്നെ ധ്യാനാവസ്ഥയാണ്. മുക്തിയാണ്. ശാശ്വതമായ ശാന്തിയാണ്. ഉപാധികളുള്ള മനസ്സ് ശോകത്തിന് വഴിപ്പെടുന്നു. എന്നാല്‍ ഉപാധിരഹിതമാണ് മനസ്സെങ്കില്‍ അത് കര്‍മ്മരഹിതവുമാണ്. അങ്ങിനെയുള്ളയാള്‍ പരമപദമായ പ്രബുദ്ധതയെ അനായാസം പ്രാപിക്കുന്നു. അതുകൊണ്ട് ഏതുവിധേനെയും മനോപാധികളെ ഇല്ലാതാക്കാന്‍ നാം ശ്രമിക്കണം.

“ലോകത്തെക്കുറിച്ചുള്ള എല്ലാവിധ പ്രത്യാശകളും ആസക്തികളും ആഗ്രഹങ്ങളും അവസാനിച്ച്, ശോകഭയരഹിതമായി ആത്മാവ് സ്വയം തന്നില്‍ത്തന്നെ അഭിരമിക്കുന്ന അവസ്ഥയത്രേ ധ്യാനം അഥവാ സമാധി.” ആത്മാവുമായി വിഷയങ്ങള്‍ക്കുണ്ടെന്നു തെറ്റിദ്ധരിച്ചിരുന്ന എല്ലാ ബന്ധങ്ങളെയും മനസാ സംത്യജിച്ച് നിനക്കിഷ്ടമുള്ളയിടത്തു ജീവിക്കാം. അത് വീട്ടിലോ മലമുകളിലെ ഗുഹകളിലോ ആവാം. മനസ്സ് പരമപ്രശാന്തതയിലെത്തിയവന്റെ ഗൃഹമാണവന് ഏകാന്തത നല്‍കുന്ന കാട്. അഹംകാരമൊഴിഞ്ഞു മന:ശ്ശാന്തിയടഞ്ഞവന് നഗരങ്ങളും അപ്രകാരം തന്നെ. എന്നാല്‍ ആളൊഴിഞ്ഞ കാടുപോലും ഹൃദയത്തില്‍ ആസക്തികളും ദുഷ്ടതയും  വെച്ച് പുലര്‍ത്തുന്നവനു തിരക്കേറിയ നഗരമാണ്. മനസ്സിനെ ചഞ്ചലപ്പെടുത്തുന്ന കാര്യങ്ങള്‍ ഗാഢസുഷുപ്തിയില്‍ ഇല്ലാതാവുന്നു. എന്നാല്‍ പ്രബുദ്ധത സ്വയം പ്രാപ്യമാവുകയാണ്. നിനക്ക് ഇഷ്ടമുള്ള പാത തിരഞ്ഞെടുക്കാം.