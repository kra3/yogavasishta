\section{ദിവസം 232}

\slokam{
ദേശക്രമേണ ധനമല്‍പ്പ വിഗര്‍ഹണേന\\
തേനാംഗ സാധുജനമര്‍ജയ മാനപൂര്‍വ്വം\\
തത്സംഗമോത്ഥവിഷയാധൃവ ഹേളലനേന\\
സമ്യഗ്വിചാരവിഭവേന തവാത്മലാഭ:   (5/24/71)\\
}

വിരോചനന്‍ തുടര്‍ന്നു: ഉള്ളില്‍ നന്മ നിറഞ്ഞു നില്‍ക്കുമ്പോഴാണ് ഏറ്റവും ഉയര്‍ന്ന വിജ്ഞാനശ്രവണത്തിന് ഒരുവന്‍ യോഗ്യനാവുന്നത്. അതുകൊണ്ട് എപ്പോഴും മനസ്സിനെ ശുദ്ധീകരിക്കാന്‍പോന്ന അറിവുകളാണ് പഠിക്കേണ്ടത്. ശാസ്ത്രപഠനത്തിലൂടെയാണ് മനസ്സിനെ പുഷ്ടിപ്പെടുത്തി ആന്തരീകമായ പരിണാമം സാധിക്കേണ്ടത്. അങ്ങിനെ മാറ്റംവന്ന മനസ്സിന് വികലമാവാത്ത ചിന്താധാരകളിലൂടെ സത്യത്തെ അന്വേഷിക്കാന്‍ കഴിവുണ്ടാവും. താമസംവിനാ അങ്ങിനെയുള്ള സാധകന്‍ ആത്മാവിനെ ദര്‍ശിക്കാന്‍ ശ്രമമാരംഭിക്കണം. 

ആത്മസാക്ഷാത്കാരവും അനാസക്തിയും ഒരേ സമയം പുരോഗതി പ്രാപിക്കുന്നു. ശരിയായ അനാസക്തി ഒരുവനിലുണ്ടാവുന്നത് തപ:ശ്ച്ചര്യകളിലൂടെയോ തീര്‍ത്ഥാടനങ്ങളിലൂടെയോ, ദാനധര്‍മ്മങ്ങളിലൂടെയോ  അല്ല. അത് സാധിക്കുന്നത് അവനവന്റെ സ്വരൂപത്തെ, ആത്മാവിനെ നേരിട്ടറിയുന്നതിലൂടെ മാത്രമാണ്.  ഇതിനുവേണ്ടത് ശരിയായ സ്വപരിശ്രമം തന്നെയാണ്. അതിനാല്‍ സാധകന്‍ ഈശ്വരനിലും വിധിയിലും മറ്റുമുള്ള പരാധീനത ഉപേക്ഷിച്ച് ശരിയായ പരിശ്രമത്തിലൂടെ സുഖാസക്തിയെ തരണം ചെയ്യണം. നിര്‍മമത അങ്ങിനെ പക്വമാവുമ്പോള്‍ അയാളില്‍ ആത്മാന്വേഷണം താനേ ഉദിച്ചുയരും. ഈ അന്വേഷണത്വര നിര്‍മമതയെ ഊട്ടിയുറപ്പിക്കുന്നു. കടലും മേഘവുംപോലെ ഇവ രണ്ടും പരസ്പരം ബന്ധപ്പെട്ടിരിക്കുന്നു. അത്മസാക്ഷാത്കാരവും നിര്‍മമതയും, അന്വേഷണത്വരയും മൂന്നുത്തമ സുഹൃത്തുക്കളാണ്.

അതിനാല്‍ ഒരുവന്‍ ആദ്യമായി ഈശ്വരന്‍ തുടങ്ങിയ ബാഹ്യമായ എല്ലാ അവലംബനങ്ങളേയും ഉപേക്ഷിച്ച് പല്ലുഞെരിച്ചും ശരിയായ പ്രവൃത്തിയിലേര്‍പ്പെട്ട് അനാസക്തി വളര്‍ത്തിയെടുക്കണം. എന്നാല്‍ സമൂഹത്തിന്റെ നിയമത്തിനുള്ളില്‍ നിന്നുകൊണ്ട് ധനം സമ്പാദിക്കുന്നതില്‍ തെറ്റില്ല. അത് ബന്ധുക്കളെയോ മറ്റുള്ളവരേയോ വഞ്ചിച്ചാവരുത്. ഈ ധനം പാവന ചരിതന്മാരായ മഹത് വ്യക്തികളുമായുളള സത്സംഗത്തിനായി ഉപയോഗിക്കണം. അത്തരം സത്സംഗവും  അനാസക്തിയെ പ്രദാനം ചെയ്യും. അങ്ങിനെ ആത്മാന്വേഷണത്വരയും, ശാസ്ത്രപഠനവും ജ്ഞാനസമ്പാദനവും അവിടെ സഹജമാവും. ക്രമേണ, പടിപടിയായി പരമസത്യത്തിലേക്ക് സാധകന്‍ എത്തിച്ചേരുന്നു.

സുഖാനുഭവപ്രയത്നങ്ങളില്‍ നിന്നും പൂര്‍ണ്ണമായി വിട്ടുനില്‍ക്കുമ്പോള്‍ ആത്മാന്വേഷണത്താല്‍ പരമസത്യം വെളിവാകുന്നു. ആത്മാവ് തികച്ചും  നിര്‍മ്മലമാവുമ്പോള്‍ നിനക്ക് പരമശാന്തിയില്‍ അഭിരമിക്കാം. വീണ്ടുമൊരിക്കലും നിനക്ക് ദു:ഖനിദാനമായ വാസനാമാലിന്യങ്ങളിലേയ്ക്കും (തെറ്റി)ദ്ധാരണകളിലേയ്ക്കും നിപതിക്കേണ്ടാതായി വരികയില്ല. തുടര്‍ന്നും ഈ ലോകത്ത് ജീവിക്കുന്നുവെങ്കിലും നീ എല്ലാ ആശകള്‍ക്കും പ്രതീക്ഷകള്‍ക്കും അതീതനായി വര്‍ത്തിക്കുന്നതാണ്. നീ തികച്ചും ശുദ്ധനത്രേ. ശുഭോദര്‍ക്കമായ എല്ലാറ്റിന്റെയും മൂര്‍ത്തിമദ്രൂപമായ നിനക്കെന്റെ നമസ്കാരം.

“അപ്പപ്പോള്‍ നിലനില്‍ക്കുന്ന സാമൂഹ്യപരിധികള്‍ക്കുള്ളില്‍ നിന്നുകൊണ്ട് ആവശ്യത്തിനു ധനം സമ്പാദിച്ച് ആ ധനംകൊണ്ട് മഹാത്മാക്കളുമായി സത്സംഗവും സാധനയും ചെയ്ത് അവരെ ബഹുമാനിക്കുക. ആ സത്സംഗം നിന്നെ സുഖാനുഭവങ്ങളില്‍ താല്‍പ്പര്യമില്ലാത്തവനാക്കും. പിന്നെ ശരിയായ അന്വേഷണത്താല്‍ നിന്നില്‍  ആത്മജ്ഞാനം സംജാതമാകും.”   
