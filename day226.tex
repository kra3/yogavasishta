\section{ദിവസം 226}

\slokam{
മധ്യസ്ഥദൃഷ്ടയ: സ്വസ്ഥാ യഥാപ്രാപ്താർഥദർശിന:\\
തജ്ജ്ഞാസ്തു പ്രേക്ഷകാ ഏവ സാക്ഷിധർമ്മേ വ്യവസ്ഥിതാ: (5/20/40)\\
}

പുണ്യൻ തുടർന്നു: കാറ്റടിക്കുമ്പോൾ പൊടിപടലങ്ങൾ പറന്നുപോകുന്നതുപോലെ അച്ഛൻ, അമ്മ, സുഹൃത്ത്, ബന്ധു തുടങ്ങിയ മിഥ്യാ ധാരണകൾ ജ്ഞാനത്തിന്റെ വെളിച്ചത്തിൽ ഇല്ലാതെയാകും. ഈ ബന്ധുക്കളൊന്നും സത്യമല്ല. അവയെല്ലാം വെറും വാക്കുകൾ മാത്രം. ഒരാളെ സുഹൃത്തെന്നു നിനച്ചാൽ അവൻ സുഹൃത്ത്. അവനെ മറ്റൊരുവൻ എന്നുനിനച്ചാൽ അങ്ങിനെ. എന്നാൽ എല്ലാറ്റിനേയും അഖണ്ഡമായ ഒരു സർവ്വവ്യാപിയെന്നു തിരിച്ചറിഞ്ഞാൽപ്പിന്നെ സുഹൃത്ത്, ബന്ധു, ശത്രു എന്നിങ്ങനെയുള്ള തരംതിരിവിനെന്താണർത്ഥം?

അനിയാ, നിന്റെ ഉള്ളിൽത്തന്നെ അന്വേഷിക്കൂ. ദേഹം വെറും ജഢം. മാംസാസ്ഥിരക്ത നിര്‍മ്മിതമായ ഒരു കൂടാണത്. അതിൽ ‘ഞാൻ’ എവിടെ? അങ്ങിനെ നിന്റെ അന്വേഷണം തുടർന്നാൽ ‘നീ’ എന്നതും ‘ഞാൻ’ എന്നതും എല്ലാം മിഥ്യയാണെന്നറിയാം. പുണ്യൻ, പവനൻ, എന്നെല്ലാം പറയുന്നതും മിഥ്യ തന്നെ. എന്നിട്ടും നിനക്ക് ‘ഞാനുണ്ട്’ എന്ന തോന്നലുണ്ടെങ്കിൽ, നിനക്ക് പൂർവ്വജന്മങ്ങളിൽ ഉണ്ടായിരുന്ന അനേകം ബന്ധുക്കളെക്കുറിച്ച് നീയെന്തുകൊണ്ട് വിലപിക്കുന്നില്ല? നീയൊരരയന്നമായിരുന്നപ്പോൾ നിനക്ക് അനേകം ബന്ധുക്കൾ ആവർഗ്ഗത്തിൽപ്പെട്ടവരായുണ്ടായിരുന്നു. നീയൊരു മരമായിരുന്നപ്പോൾ വൃക്ഷങ്ങളായും; പിന്നീട് സിംഹജന്മത്തിൽ സിംഹബന്ധുക്കളും, മൽസ്യജന്മത്തിൽ അനേകം മൽസ്യബന്ധുക്കളും നിനക്കുണ്ടായിരുന്നു.

നീയൊരു രാജകുമാരനായിരുന്നു; കഴുതയായിരുന്നു; പേരാൽ മരമായിരുന്നു, അരയാലുമായിരുന്നു. നിനക്കൊരു ബ്രാഹ്മണജന്മമുണ്ടായിരുന്നു, നീയൊരീച്ചയായിരുന്നു, കൊതുകായിരുന്നു, ഉറുമ്പായിരുന്നു. നീയൊരു ഒരു ജന്മം തേളായിരുന്നു, പിന്നെ തേനീച്ച. ഇപ്പോള്‍ നീയെന്റെ സഹോദരൻ! ഇങ്ങിനെ അനേകദേഹങ്ങളിൽ നീ വീണ്ടും വീണ്ടും ജന്മമെടുത്തിട്ടുണ്ട്. എനിയ്ക്കും അനേകം ജന്മങ്ങളുണ്ടായിട്ടുണ്ട്. എന്നാല്‍ എന്റെ സൂക്ഷ്മദൃഷ്ടിയിൽ എനിക്കിതെല്ലാം വ്യക്തമായിക്കാണാം. ഞാനൊരു പക്ഷിയായിരുന്നു, തവള, വൃക്ഷം, ഒട്ടകം, രാജാവ്, പുലി, ഇപ്പോൾ നിന്റെ സഹോദരൻ. പത്തുകൊല്ലം ഞാനൊരു പരുന്തായിരുന്നു. അഞ്ചുമാസം ഒരു ചീങ്കണ്ണി; ഒരു നൂറുകൊല്ലം സിംഹം- ഇപ്പോള്‍ നിന്റെ സഹോദരൻ. ഇങ്ങിനെ എണ്ണമറ്റ ജന്മങ്ങൾ അജ്ഞാനത്തിന്റെയും ഭ്രമത്തിന്റെയും പിടിയിലകപ്പെട്ട് കഴിഞ്ഞുപോയതു ഞാനിപ്പോള്‍ വ്യക്തമായി ഓർക്കുന്നു.

ആജന്മങ്ങളിലെല്ലാം എനിയ്ക്ക് അനേകം ബന്ധുക്കളുണ്ടായിരുന്നു. അവരില്‍ ആരെയോർത്താണു ഞാൻ വിലപിക്കേണ്ടത്? അതുകൊണ്ട് എനിയ്ക്ക് ആരെക്കുറിച്ചും ദു:ഖമില്ല. ജീവിതമെന്ന ഈ കാട്ടുപാതയിലെ കരിയിലപോലെ ചിതറിക്കിടക്കുകയാണീ ബന്ധുക്കൾ. ഈ ലോകത്ത് ശരിയായി സന്തോഷിക്കാനോ ദു:ഖിക്കാനോ എന്താണുള്ളത്? അതുകൊണ്ട് നമുക്കീ വിലാപമെല്ലാം മതിയാക്കി പ്രശാന്തരാവാം. നിന്റെയുള്ളിൽ ‘ഞാൻ’ എന്ന പ്രതീതി ജനിപ്പിക്കുന്ന, ലോകമെന്ന ധാരണതന്നെ നമുക്കില്ലാതാക്കാം. താഴോട്ടു നിപതിക്കാതെയും മേലോട്ടു പൊങ്ങിപ്പോവാതെയും സമതയോടെ നമുക്കിരിക്കാം. നിനക്ക് ദു:ഖമില്ല; ജനനമില്ല; അച്ഛനില്ല, അമ്മയില്ല. നീ ആത്മാവാണ്‌.. മറ്റൊന്നുമല്ല.

“മഹർഷിമാർ മദ്ധ്യമാർഗ്ഗം അവലംബിക്കുന്നു. അവർ അപ്പപ്പോൾക്കാണുന്നതിനെ* അതുപോലെ ക്ഷണികമെന്നറിഞ്ഞ് പ്രശാന്തയോടെയിരിക്കുന്നു. അവർ സാക്ഷിബോധത്തിൽ സ്ഥിരപ്രതിഷ്ഠരത്രേ.” അവർ ഇരുട്ടിൽത്തിളങ്ങുന്ന വെളിച്ചമാണ്‌. വിളക്കും എണ്ണത്തിരിയുമില്ലാതെ സ്വയമുണ്ടാവുന്ന ആ വെളിച്ചത്തിലാണ്‌ എല്ലാം സംഭവിക്കുന്നത്.

*അന്നന്നു കാണ്മതിനെ വാഴ്ത്തുന്നു മാമുനികള്‍ എന്നത്രേ തോന്നി ഹരി നാരായണായ നമ:
