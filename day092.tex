\newpage
\section{ദിവസം 092}

\slokam{
അഥ വര്‍ഷസഹസ്രേണ താം പിതാമഹ ആയയൌദാരുണം ഹി തപ: സിദ്ധയൈ വിഷാഗ്നിരപി ശീതള: (3/69/1)
}

വസിഷ്ഠന്‍ തുടര്‍ന്നു: ഇതിനെ സംബന്ധിച്ച്‌ പഴയൊരൈതീഹ്യമുള്ളത്‌ ഞാന്‍ നിനക്ക്‌ പറഞ്ഞു തരാം. ഹിമാലയ പര്‍വ്വതസാനുക്കള്‍ക്കും വടക്കായി ഒരിടത്ത്‌ കാര്‍ക്കടി എന്നുപേരായ ഒരു രാക്ഷസി ജീവിച്ചിരുന്നു. കറുത്തിരുണ്ടതും വലിപ്പമേറിയതും കാണുവാന്‍ ഭയങ്കരവുമായിരുന്നു അവളുടെ രൂപം. ആവശ്യത്തിനു ഭക്ഷണം ലഭിക്കാഞ്ഞ്‌ എപ്പോഴും വിശപ്പായിരുന്നു അവള്‍ക്ക്‌..  അവള്‍ വിചാരിച്ചു: 'ജംബുദ്വീപില്‍ നിവസിക്കുന്ന എല്ലാവരേയും ഒന്നിച്ചു തിന്നാന്‍ കിട്ടിയിരുന്നുവെങ്കില്‍ എന്റെ വിശപ്പ്‌ പെരുമഴയ്ക്കുശേഷം മരീചികയെന്നപോലെ അപ്രത്യക്ഷമായേനെ! എന്നാല്‍ അത്തരം ഒരു പ്രവര്‍ത്തനം ഒരുവന്റെ ജീവന്‍ നിലനിര്‍ത്താന്‍ ഉചിതമാവുകയില്ല. മാത്രമല്ല ജംബുദ്വീപിലെ നിവാസികള്‍ ധര്‍മ്മിഷ്ഠരും സഹാനുഭൂതിയുള്ളവരും ഈശ്വരഭക്തിയുള്ളവരും പച്ചമരുന്നുകളെപ്പറ്റി അറിവുള്ളവരുമാണ്‌. ഇങ്ങിനെ സമാധാനമായി ജീവിക്കുന്നവരെ ഉപദ്രവിക്കുന്നത്‌ ശരിയല്ല. അതുകൊണ്ട്‌ ഞാന്‍ ഒരു തപസ്സിലേര്‍പ്പെടാന്‍ പോവുകയാണ്‌..  കിട്ടാന്‍ വളരെ ദുഷ്കരമായ കാര്യം സാദ്ധ്യമാക്കാന്‍ തപസ്സിനോളം ഉത്തമമായി മറ്റൊന്നുമില്ല.'

കാര്‍ക്കടി, മഞ്ഞുമലയുടെ ഉത്തുംഗശിഖരങ്ങളിലൊന്നില്‍പ്പോയി ഒറ്റക്കാലില്‍ തപസ്സു തുടങ്ങി. അവളുടെ തീരുമാനം കല്ലുപോലെ ഉറച്ചതും ശ്രദ്ധ സുദൃഢവുമായതിനാല്‍ ദിവസങ്ങളും മാസങ്ങളും കടന്നുപോയത്‌ അവളറിഞ്ഞില്ല. കാലക്രമത്തില്‍ അവളുടെ ദേഹം ശോഷിച്ച്‌ സുതാര്യമായ ചര്‍മ്മത്തില്‍ പൊതിഞ്ഞ അസ്ഥിപഞ്ചരം മാത്രമായി.അവളങ്ങിനെ ഒരായിരംകൊല്ലം തപസ്സുചെയ്തു. "ആയിരംകൊല്ലം കഴിഞ്ഞപ്പോള്‍ ബ്രഹ്മാവ്‌ അവളില്‍ സം പ്രീതനായി പ്രത്യക്ഷപ്പെട്ടു. തീവ്രതപസ്സുകൊണ്ട്‌ ഒരുവന്‌ എന്തും നേടാം. വിഷപ്പുകപോലും നിര്‍വ്വീര്യമാക്കി അണയ്ക്കാം." മനസാ ബ്രഹ്മാവിനെ വന്ദിച്ച്‌ എന്തു വരമാണു ചോദിക്കേണ്ടതെന്ന് അവള്‍ ചിന്തിച്ചു. 'ഓ കിട്ടിപ്പോയി!', അവള്‍ ആലോചിച്ചു: എനിക്കൊരു ജീവനുള്ള സൂചികയാവണം (ഇരുമ്പു സൂചി- രോഗത്തിന്റെ മൂര്‍ത്തിമദ്‌ ഭാവം) എന്ന വരം ആവശ്യപ്പെടാം. ഈ വരം ലഭിച്ചാല്‍ അതുമായി എനിക്ക്‌ എല്ലാജീവജാലങ്ങളുടേയും ഹൃദയത്തില്‍ക്കയറി എന്റെ ആഗ്രഹം നടപ്പാക്കാം, എന്റെ വിശപ്പും തീരും. 

ബ്രഹ്മാവു പറഞ്ഞു: നിന്റെ തപസ്സില്‍ നാം സംപ്രീതനാണ്‌. നിന്റെ ആഗ്രഹമെന്താണ്‌? അവള്‍ തന്റെ ആഗ്രഹമറിയിച്ചു. 'തഥാസ്തു!' ബ്രഹ്മാവു പറഞ്ഞു. അങ്ങിനെയാവട്ടെ, നിനക്ക്‌ വിഷൂചികയും ആവാം. സൂക്ഷ്മമായിനിലകൊണ്ട്‌ മോശപ്പെട്ട ആഹാരം കഴിക്കുന്നവരിലും ദുര്‍ന്നടത്ത ശീലമാക്കിയവരിലും അവരുടെ ഹൃദയത്തില്‍ക്കയറി നിനക്ക്‌ ആധിയുണ്ടാക്കാം. എന്നാല്‍ ഇനിപ്പറയുന്ന മന്ത്രം ജപിക്കുന്നവര്‍ക്ക്‌ ഈ ദുരിതത്തില്‍നിന്നു മോചനം ലഭിക്കും.

\slokam{
ഹിമാദ്രേര്‍ ഉത്തരേ പാര്‍ശ്വേ കാര്‍ക്കടീ നാമ രാക്ഷസീ\\
വിഷൂചികാഭിധാന സാനാംനാപ്യന്യയാബാധികാ\\
ഓം ഹ്രാം ഹ്രീം ശ്രീം രം വിഷ്ണുശക്തയേ  നമോ ഭഗവതീ\\
വിഷ്ണുശക്തി  ഏഹി ഏനം ഹര ഹര ദാഹ ദാഹ ഹാന ഹാന പച പച\\
മാത മാത ഉത്സദായ ഉത്സദായ ദൂരേ കുരു കുരു സ്വാഹ വിഷൂചികേ\\
ത്വാം ഹിമവന്തം ഗഛ ഗഛ ജീവസാര ചന്ദ്രമണ്ഡലം ഗതോ സി സ്വാഹ\\
}

ഈ മന്ത്രത്തില്‍ പ്രാവീണ്യം നേടിയയാള്‍ അതു തന്റെ ഇടതുകരത്തില്‍ ധരിച്ച്‌ ചന്ദ്രനെ ധ്യാനിച്ച്‌ ആ കൈ രോഗിയുടെമേല്‍ വെച്ചാല്‍ ആ ക്ഷണം അയാള്‍ സുഖപ്പെടും.

