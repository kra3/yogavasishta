 
\section{ദിവസം 016}

\slokam{
ദാനവാ അപി ദീയന്തേ ധ്രുവാപ്യധ്രുവ ജീവിതാ:\\
അമരാ അപി മാര്യന്തേ കൈവാസ്ഥാ മാദൃശേ ജനേ (1/26/26)\\
}

രാമന്‍ തുടര്‍ന്നു: ഞാന്‍ ഇതുവരെപ്പറഞ്ഞ 'കാല' ത്തിനുപുറമേ ജനന മരണങ്ങള്‍ ക്കുത്തരവാദിയായ മറ്റൊരു 'കാലം' കൂടിയുണ്ട്‌. . ആളുകള്‍ ഇതിനെ മരണദേവതയായ 'കാലന്‍' എന്നു പറയുന്നു. മറ്റൊരു കാലസങ്കല്‍പ്പം ഉള്ളതിനെ 'കൃതാന്തം' എന്നുപറയും. അതാണ്‌ അനിവാര്യമായ കര്‍മ്മഫലം. നര്‍ത്തകനായ കൃതാന്തന്‌, നിയതി (പ്രകൃതിനിയമം) ഭാര്യയാണ്‌..  ഈ ദമ്പതികള്‍ ചേര്‍ന്ന് എല്ലാ ജീവജാലങ്ങള്‍ക്കും ഉചിതമായ കര്‍മ്മഫലങ്ങള്‍ നല്‍കുന്നു. ഈ വിശ്വം നിലനില്‍ക്കുന്നിടത്തോളം അവര്‍ കര്‍ത്തവ്യനിര്‍വ്വഹണത്തില്‍ അതീവ ജാഗ്രതയുള്ളവരും, തളര്‍ച്ചയറിയാത്ത കര്‍മ്മനിരതരുമാണ്‌..  ഇങ്ങിനെ കാലം ബ്രഹ്മാണ്ഡങ്ങളില്‍ വിഹരിക്കുമ്പോള്‍ നമുക്കെന്തു പ്രതീക്ഷയാണുള്ളത്‌? അടിയുറച്ച വിശ്വാസികള്‍ പോലും കൃതാന്തന്റെ പിടിയില്‍ ആടിയുലയുന്നു. ഈ  കൃതാന്തന്‍ കാരണമാണ്‌ ലോകത്തിലെ എല്ലാം മാറ്റങ്ങള്‍ക്കു വിധേയമാവുന്നത്‌. ഒന്നും ഇവിടെ ശാശ്വതമല്ല.

എല്ലാ ജീവജാലങ്ങളിലും ദുഷ്ടതയുടെ കറ പുരണ്ടിരിക്കുന്നു. എല്ലാ ബന്ധങ്ങളും ബന്ധനങ്ങള്‍.  ആസ്വാദനങ്ങളെന്നത്‌ കൊടിയ പീഢകളും സുഖത്തിനുള്ള ആശ, മരുമരീചികയുമത്രേ. ഒരുവന്റെ ഇന്ദ്രിയങ്ങള്‍ അവനുതന്നെ ശത്രുവാകുന്നു. ശാശ്വതമായത്‌ അജ്ഞേയമാവുന്നു. മനസ്സാണ്‌ ഏറ്റവും വലിയ ശത്രു. അഹംകാരമാണ്‌ ഏറ്റവും വലിയ അനര്‍ത്ഥത്തിനു കാരണം. വിജ്ഞാന വിവേകങ്ങള്‍ തുലോം ക്ഷീണിതം. കര്‍മ്മങ്ങള്‍ അസൌഖ്യകരം. ഉല്ലാസം വിഷയസുഖത്തില്‍ മാത്രം. മനുഷ്യന്റെ ബുദ്ധിയെ അഹംകാരം വരുതിയില്‍ നിര്‍ത്തുന്നു- തിരിച്ചാണു വേണ്ടതെങ്കിലും. അതുകൊണ്ട്‌ ശാന്തിയോ മന:സുഖമോ ഉണ്ടാവുന്നില്ല. യുവത്വം ക്ഷയിക്കുന്നു. സദ്സംഗമെന്നത്‌ ദുര്‍ല്ലഭം. ഈ ദുരിതാനുഭവത്തില്‍ നിന്നും നിവൃത്തി അസാദ്ധ്യം. സത്യവസ്തുവിന്റെ സാക്ഷാത്കാരം എവിടെയും കാണുന്നില്ല. അപരന്റെ ഐശ്വര്യത്തില്‍ സന്തോഷമോ മറ്റുള്ളവരോട്‌ അനുകമ്പയോ ആരിലും കാണാനില്ല. ആളുകള്‍ അനുദിനം അധ:പ്പതിച്ചുകൊണ്ടിരിക്കുന്നു. ബലത്തെ ദൌര്‍ബ്ബല്യവും ധൈര്യത്തെ ഭീരുത്വവും കീഴടക്കിയിരിക്കുന്നു. ദുഷ്ടസഹവാസം എളുപ്പവും സദ്ജനസംസര്‍ഗ്ഗം ദുര്‍ല്ലഭവുമാവുന്നു. കാലമാണ്‌ മാനവീകതയുടെ മുഴുവന്‍ സാരഥിയെന്നെനിക്കു തോന്നുന്നു. 

"മഹാത്മന്‍, ഈ മാസ്മരീക ശക്തിതന്നെ സൃഷ്ടിയെ ഭരിക്കുന്നു, ശക്തിശാലികളായ രാക്ഷസന്മാരെപ്പോലും നശിപ്പിക്കുന്നു, ശാശ്വതമെന്നു കരുതിപ്പോന്ന എല്ലാറ്റിനേയും ഇല്ലാതാക്കുന്നു, ചിരഞ്ജീവികളെപ്പോലും കൊല്ലുന്നു. അപ്പോള്‍പ്പിന്നെ എന്നേപ്പോലുള്ള നിസ്സാരന്മാര്‍ക്ക്‌ എന്തു പ്രത്യാശ?"

ഈ നിഗൂഢജീവി എല്ലാവരിലും നിവസിക്കുന്നുവെന്നു തോന്നുന്നു. അത്‌ വ്യക്തികളില്‍ അഹംകാരമായി വര്‍ത്തിക്കുന്നു. ഒന്നിനും അതിനെ നശിപ്പിക്കാനാവില്ല. ഈ ബ്രഹ്മാണ്ഡം മുഴുവന്‍ നിയന്ത്രിക്കുന്ന അഹംകാരം മാത്രമേ കാലത്തെ അതിജീവിക്കുന്നുള്ളു. 
