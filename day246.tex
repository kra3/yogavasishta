\section{ദിവസം 246}

\slokam{
 വിചരത്യേഷ ലോകേഷു ജീവ ഏവ ജഗസ്ഥിതൌ\\
വിലസത്യേവ ഭോഗേഷു പ്രസ്ഫുരത്യേവ വസ്തുഷു (5/35/21)\\
}

പ്രഹ്ലാദന്‍ തന്റെ ധ്യാനം തുടര്‍ന്നു: യാതൊരു വികലതകളും ഇല്ലാത്ത അദ്വൈതബോധമാണ്  'ഓം' എന്ന ശബ്ദം കൊണ്ട് സൂചിപ്പിക്കപ്പെടുന്ന ആത്മാവ്. ലോകത്തിലെന്തെല്ലാമുണ്ടോ അതെല്ലാം ഇപ്പറഞ്ഞ ഏകമായ ആത്മാവ് മാത്രം. ഈ ദേഹത്തിലെ മാംസാസ്ഥിസഞ്ചയവും, എല്ലാറ്റിനെയും ഭാസുരമാക്കുന്ന പ്രജ്ഞയും പ്രകാശസ്രോതസ്സുകളായ സൂര്യചന്ദ്രാദികളും എല്ലാം ആത്മാവ് തന്നെ. അതഗ്നിക്ക് ചൂട് നല്‍കുന്നു; മധുരഫലങ്ങള്‍ക്ക് മധുരിമയേകുന്നു; ഇന്ദ്രിയങ്ങളിലൂടെ അനുഭവങ്ങള്‍ വേദ്യമാക്കുന്നു. നില്‍ക്കുന്നുവെങ്കിലും അത് സ്ഥാവരമല്ല. ചലിക്കുന്നുവെങ്കിലും അത് ജംഗമമല്ല. നിശ്ചലമെങ്കിലും അത് സദാ ചടുലവും ജാഗരൂകവുമത്രേ. കര്‍മ്മനിരതമെങ്കിലും അതിനു മാറ്റങ്ങളേതുമില്ല. ഭൂത-ഭാവി-വര്‍ത്തമാനകാലങ്ങളിലും അവിടെയുമിവിടെയുമെല്ലാടവും എല്ലാ ആപേക്ഷികമായ നാമരൂപവ്യതിയാനങ്ങളിലും അതിനു മാറ്റമൊന്നുമില്ല.

ഭയലേശമില്ലാതെയും പരിമിധികളില്ലാതെയും ഈ ബോധമാണ് ബ്രഹ്മാവ്‌ മുതല്‍ പുല്‍ക്കൊടിവരെയുള്ള അനേകമനേകം ജീവ-നിര്‍ജീവജാലങ്ങളെ പ്രത്യക്ഷപ്രകടമാക്കി അവയ്ക്ക് നിലനില്‍ക്കാനിടനല്‍കുന്നത്. അതെപ്പോഴും ചടുലവും കര്‍മ്മനിരതവുമാണ്. എന്നാല്‍ വലിയൊരു പാറപോലെ ഉറച്ചതും അനങ്ങാത്തതുമാണ് താനും. യാതൊന്നിനാലും ബാധിക്കപ്പെടാത്ത ആകാശത്തിനോളം പോലും കര്‍മ്മങ്ങള്‍ അതിനെ ബാധിക്കുന്നില്ല. 

ഈ ആത്മാവാണ് കാറ്റിലുലയുന്ന കരിയിലപോലെ മനസ്സിനെ കര്‍മ്മനിരതമാക്കുന്നത്. കുതിരയെ നയിക്കുന്ന സവാരിക്കാരനെപ്പോലെ അത് ഇന്ദ്രിയങ്ങളെ സംവേദനപ്രാപ്തമാക്കുന്നു. ആത്മാവ് ഈ ദേഹത്തിന്റെ നാഥനാണെങ്കിലും അതൊരടിമയെപ്പോലെ എണ്ണമില്ലാത്ത, വൈവിദ്ധ്യമാര്‍ന്ന കര്‍മ്മങ്ങളില്‍ ഏര്‍പ്പെട്ടിരിക്കുന്നതായി തോന്നുന്നു. ഈ ആത്മാവൊന്നുമാത്രമേ ആശ്രയിക്കാനും പൂജിക്കാനും ധ്യാനിക്കാനും യോഗ്യമായുള്ളു. 

അതിനെ സമാശ്രയിച്ചാലേ ഈ ലോകമെന്ന ജനനമരണചക്രത്തില്‍ നിന്നും ഭ്രമകല്‍പ്പനകളില്‍നിന്നും ഒരുവന് രക്ഷ കിട്ടൂ. നല്ലൊരു സുഹൃത്തിനെപ്പോലെ എളുപ്പത്തില്‍ അതിനെ വെല്ലാന്‍ കഴിയും, കാരണം അത് നിവസിക്കുന്നത് എല്ലാവരുടെയും ഹൃദയകമലത്തിലാണല്ലോ. ക്ഷണിക്കാതെ പോലും സ്വദേഹത്തില്‍ത്തന്നെ അത് ക്ഷിപ്ര സിദ്ധം. ഒരിക്കലെങ്കിലും ധ്യാനിച്ചാലത് പ്രത്യക്ഷമായി വെളിപ്പെടും. അത് പ്രപഞ്ചത്തിലെ എല്ലാറ്റിന്റെയും നാഥനാണെങ്കിലും ആത്മാവുണര്‍ന്നവന് അഹംകാരമോ ദംഭോ ഉണ്ടാവുകയില്ല. പൂക്കളിലെ സൌരഭ്യം പോലെ ആത്മാവ് ദേഹങ്ങളില്‍ സര്‍വ്വവ്യാപിയായി സ്ഥിതികൊള്ളുന്നു.  

ആരും അതെക്കുറിച്ച്, അതായത് സ്വന്തം അസ്ഥിത്വത്തെക്കുറിച്ച് ആരായാത്തതുകൊണ്ടാണ് 'അതെ'ല്ലാവര്‍ക്കും ഗമ്യമല്ലാത്തതാണെന്നു തോന്നുന്നത് . ആത്മാന്വേഷണത്തിലൂടെ മാത്രമാണത്‌ ലഭ്യമാവുക. അത് പരമാനന്ദത്തിന്റെയും സത്യത്തിന്റെയും മരണമില്ലാത്ത ഒരുത്തമദര്‍ശനമത്രേ. എല്ലാ ബന്ധനങ്ങളുമഴിഞ്ഞും  ശത്രുക്കളെല്ലാമകന്നും ആസക്തികളൊഴിഞ്ഞും നിലകൊള്ളുമ്പോള്‍ മനസ്സിനെവിടെനിന്നാണിളക്കം വരിക?  അത് കണ്ടാല്‍പ്പിന്നെ എല്ലാം കണ്ടു. അതു കേട്ടാല്‍പ്പിന്നെ എല്ലാം കേട്ടു. അത് തൊട്ടാല്‍പ്പിന്നെ എല്ലാം തൊട്ടു. ലോകം അതുള്ളതുകൊണ്ടാണുള്ളത്. ഒരുവനുറങ്ങുമ്പോള്‍ അതുണര്‍ന്നിരിക്കുന്നു. അത് മൂഢനെ ഉണര്‍ത്തുന്നു; ദു:ഖിതരുടെ ദു:ഖമകറ്റുന്നു. അഭീഷ്ടസിദ്ധിയെ പ്രദാനം ചെയ്യുന്നു.

“സൃഷ്ടിയില്‍ അത് ജീവനെന്നപോലെ (ജീവജാലമായി) നിലകൊള്ളുന്നു; അത് സുഖാനുഭവം ആസ്വദിക്കുന്നതായി തോന്നുന്നു; പ്രപഞ്ചവസ്തുക്കളില്‍ സ്വയം വിപുലമായി ലയിക്കുന്നതായും തോന്നുന്നു.” എങ്കിലും എല്ലാ ദേഹങ്ങളിലും അതാത്മാവായി വര്‍ത്തിക്കുന്നു. പരമപ്രശാന്തതയോടെ സ്വയം അനുഭവവേദ്യമാവുന്നു. വിശ്വപ്രപഞ്ചത്തിന്റെ ഒരേയൊരു പരമസത്ത അത് മാത്രമാണ്.