\section{ദിവസം 248}

\slokam{
സ്തുത്യാ പ്രണത്യാ വിജ്ഞപ്ത്യാ ശമേന നിയമേന ച\\
ലബ്ധോഽയം ഭഗവാനാത്മാ ദൃഷ്ടശ്ചാധിഗത: സ്ഫുടം  (5/35/49)\\
}

പ്രഹ്ലാദന്റെ മനനം തുടര്‍ന്നു: എന്റെ ഭാഗ്യാതിരേകമെന്നു പറയട്ടെ, ഇന്ദ്രിയ സുഖാസസക്തിയുടെ സര്‍പ്പങ്ങള്‍ എന്നെ വിട്ടുപോയിരിക്കുന്നു. എന്നില്‍ എല്ലാ പ്രത്യാശകളും ഭ്രമകല്‍പ്പനകളും അസ്തമിച്ചിരിക്കുന്നു. പരമസത്യത്തിന്റെ തലത്തില്‍ ഞാനെത്തിയിരിക്കുന്നു.

“ഭഗവദ് കീര്‍ത്തനങ്ങളിലൂടെ, നമസ്കാരങ്ങളിലൂടെ, പ്രാര്‍ത്ഥനകളിലൂടെ, അച്ചടക്കമുള്ള ജീവിതചര്യകളിലൂടെ, ആര്‍ജ്ജിച്ച മന:സമാധാനത്തിനാല്‍ എനിക്ക് ആത്മസ്വരൂപനായ ഭഗവാനെ ദര്‍ശിക്കാന്‍ കഴിഞ്ഞു.”  വിഷ്ണു ഭഗവാന്റെ അനുഗ്രഹത്താല്‍ പരംപൊരുളിനെ സാക്ഷാത്കരിക്കാനും അത് ഹൃദയത്തിലുറപ്പിക്കാനും എനിക്കായി. ഇതുവരെ എന്നെ പീഢിപ്പിച്ചുകൊണ്ടിരുന്ന അജ്ഞതയും പരിമിതികളും വിഭ്രാന്തികളും എന്നെ വിട്ടുപോയിരിക്കുന്നു.

അജ്ഞാനത്തിന്റെ കൊടുംകാട്ടില്‍ ഇന്ദ്രിയാസക്തികളാകുന്ന വിഷസര്‍പ്പങ്ങള്‍ വാഴുന്ന വലിയ മണ്‍പുറ്റുകളുണ്ട്. മരണമെന്ന ഇരുള്‍ക്കുഴികളുണ്ട്. ദു:ഖദുരിതങ്ങളുടെ കാട്ടുതീയുണ്ട്. ക്രോധലോഭങ്ങളാകുന്ന കള്ളന്മാരവിടെ മേഞ്ഞുനടക്കുന്നുണ്ട്. അതിനൊക്കെപ്പുറമേ, അഹംഭാവമെന്ന വന്‍ശത്രുവും അവിടെയാണ് വാസം.

ഇപ്പോള്‍ സ്വപ്രയത്നത്താലും വിഷ്ണുകൃപയാലും ഞാനതില്‍ നിന്നും സ്വതന്ത്രനായി. എന്റെ മേധാശക്തി പൂര്‍ണ്ണമായും ഉണര്‍ന്നിരിക്കുന്നു. ആ പ്രബുദ്ധതയില്‍ അഹംഭാവമെന്ന ഒരു വസ്തുവിനെ എനിക്ക് തിരിച്ചറിയാനേ ആവുന്നില്ല. സൂര്യപ്രഭയില്‍ ഇരുട്ടിനെ എങ്ങിനെ കാണാനാണ്? അഹംഭാവമെന്ന പിശാചിനെ ഒടുക്കിയതിനാല്‍ ഇപ്പോള്‍ ഞാന്‍ പ്രശാന്തനാണ്. സത്യം സാക്ഷാത്കരിക്കുകയും അഹംഭാവത്തെ ഇല്ലാതാക്കുകയും ചെയ്തു കഴിഞ്ഞാല്‍പ്പിന്നെ ഭ്രമാത്മകതയ്ക്കും ദു:ഖത്തിനും ആസക്തികള്‍ക്കും പ്രത്യാശകള്‍ക്കും മനോവിഭ്രാന്തികള്‍ക്കും ഇടമെവിടെ?  

സ്വര്‍ഗ്ഗനരകങ്ങള്‍ , മുക്തിയെക്കുറിച്ചുള്ള ഭ്രമചിന്തകള്‍ , എന്നിവയ്ക്കെല്ലാം അഹംഭാവമുള്ളപ്പോള്‍ മാത്രമേ നിലനില്‍പ്പുള്ളൂ. ശൂന്യാകാശത്ത് ചിത്രങ്ങള്‍ വരയ്ക്കാന്‍ കഴിയില്ലല്ലോ? അതിനൊരു മാദ്ധ്യമം വേണം. ശരത്കാലപൂര്‍ണ്ണചന്ദ്രന്‍ നഗരത്തെ ശോഭായമാനമാക്കുന്നതുപോലെ അഹംകാരത്തിന്റെ കാര്‍മേഘമെല്ലാം നീങ്ങി, ആസക്തികളാവുന്ന കൊടുങ്കാറ്റടങ്ങിയ എന്നിലെ മേധാശക്തി ആത്മജ്ഞാനത്തിന്റെ നിറവില്‍ ഭാസുരകാന്തിയോടെ നിലകൊള്ളുന്നു.

അഹംകാരവിമുക്തമായ ആത്മാവിനെ ഞാന്‍ നമസ്കരിക്കുന്നു. ഭയാനകങ്ങളായ ഇന്ദ്രിയങ്ങളും എല്ലാത്തിനെയും വിഴുങ്ങുന്ന മനസ്സും അടക്കിയ ആത്മാവിനെന്റെ നമോവാകം. പരമാനന്ദത്തിന്റെ താമരവിരിഞ്ഞുല്ലസിക്കുന്ന ആത്മാവിനെന്റെ നമസ്കാരം. ബോധവും ധ്യാനവും ഇരുചിറകുകളായ, ഹൃദയകമലനിവാസിയായ, അത്മാവിനെന്റെ കൂപ്പുകൈ. അജ്ഞാനത്തിന്റെ ഇരുട്ടകറ്റുന്ന സൂര്യനായ ആത്മാവിനെന്റെ നമസ്കാരം.  പ്രേമസ്വരൂപമായ, പ്രേമവര്‍ദ്ധനവിനു കാരണമായ, പ്രപഞ്ചത്തിലെ എല്ലാത്തിനെയും സംരക്ഷിക്കുന്ന ആത്മാവിനെന്റെ നമോവാകം.
ചൂടേറിയ ഇരുമ്പു മുറിക്കാന്‍ ഇരുമ്പുതന്നെ ഉപയോഗിക്കുന്നതുപോലെ ഞാനെന്റെ മനസ്സിനെ അടക്കി നിര്‍മ്മലമാക്കിയത് മനസ്സുകൊണ്ട് തന്നെയാണ്. ഞാന്‍ ആസക്തികളെയും, അജ്ഞതയെയും മൂഢതയെയും അവയുടെയെല്ലാം ദ്വന്ദങ്ങളെയും മനസ്സുപയോഗിച്ചു തന്നെ  അറുത്തുമുറിച്ചു കളഞ്ഞു .

അഹംകാരമില്ലാതെ എന്റെ ദേഹം അതിന്റെ നൈസര്‍ഗ്ഗികമായ  ഊര്‍ജ്ജത്താല്‍ വര്‍ത്തിക്കുന്നു. ജന്മവാസനകള്‍ , മനോപാധികള്‍ , പരിമിതികള്‍ , എല്ലാം എന്നില്‍ അവസാനിച്ചിരിക്കുന്നു. ഞാനെങ്ങിനെ ഈ മായാവലയില്‍ കുടുങ്ങി ഇത്രകാലം കഴിഞ്ഞുകൂടിയെന്നു ഞാന്‍ അത്ഭുതപ്പെടുന്നു. സര്‍വ്വസ്വാതത്ര്യത്തോടെ, ചിന്താഭാരമില്ലാതെ, ആസക്തികളോ ആശകളോ ഇല്ലാതെ, അഹം എന്ന ഭാവനാസത്വത്തിന്റെ അസ്തിത്വത്തെപ്പറ്റിയുള്ള അജ്ഞ്ഞതയില്ലാതെ, സുഖാന്വേഷണത്വരയില്‍ ചാഞ്ചാടി ഉല്ലസിക്കാതെയും വലയാതെയും,  ഞാന്‍ പരമപ്രശാന്തിയില്‍ എത്തിയിരിക്കുന്നു. എല്ലാ ദു:ഖങ്ങള്‍ക്കും അവസാനമായി. പരമാനന്ദമായി. 

