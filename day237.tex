\section{ദിവസം 237}

\slokam{
മനാക്ചലതി പര്‍ണേഽപി ദൃഷ്ടാരിഭയഭീതയ:\\
വദ്ധ്വസ്ത്രസ്യന്തി വിദ്ധ്വസ്താ മൃഗ്യോ ഗ്രാമഗതാ ഇവ. (5/31/12)\\
}

വസിഷ്ഠന്‍ തുടര്‍ന്നു: യാതൊരുവിധ തടസ്സങ്ങള്‍ക്കും ഇടയില്ലാത്ത മറ്റൊരു പാതയിലൂടെ ഒരാള്‍ ആത്മസാക്ഷാത്കാരം നേടിയതിനെപ്പറ്റി ഒരു കഥ ഇനിഞാന്‍ പറഞ്ഞു തരാം. പാതാളലോകത്തുണ്ടായിരുന്ന മറ്റൊരു അസുരരാജാവാണ് ഹിരണ്യകശിപു. ഇന്ദ്രനില്‍ നിന്നും മൂന്നുലോകങ്ങളുടെയും അധീശസ്ഥാനം അദ്ദേഹം ബലമായി നേടിയെടുത്തിരുന്നു. അദ്ദേഹത്തിന് അനേകം ആണ്‍മക്കളുമുണ്ടായിരുന്നു. അവരിലൊരാളാണ് ഭക്തശിരോമണിയായ പ്രഹ്ലാദന്‍.. മൂന്നു ലോകങ്ങളുടേയും അധികാരസ്ഥാനം, നല്ല സേനകള്‍, മിടുക്കരായ മക്കള്‍ എന്നിങ്ങനെയുള്ള സൌഭാഗ്യങ്ങള്‍ ഹിരണ്യകശിപുവിനെ ധിക്കാരിയും അഹംകാരിയുമാക്കി. അയാളുടെ അനീതിയും അക്രമവും ദേവന്മാരെ വല്ലാതെ വ്യാകുലപ്പെടുത്തി. അവര്‍ തങ്ങളുടെ പ്രശ്നപരിഹാരത്തിനായി സൃഷ്ടികര്‍ത്താവായ ബ്രഹ്മദേവനോടു സഹായമാഭ്യര്‍ത്ഥിച്ചു. എന്നാല്‍ ഭഗവാന്‍ ഹരിയാണ്  നരസിംഹരൂപമെടുത്ത്‌ അസുരനിഗ്രഹം ചെയ്ത് ദേവന്മാരെ സഹായിച്ചത്.
ഭഗവാന്‍ അവതാരമെടുത്ത നരസിംഹരൂപം ഭയാനകവും അതിശക്തവും ഭീമാകാരവുമായിരുന്നു. കൂര്‍ത്തുമൂര്‍ത്ത നഖങ്ങളും പല്ലുകളും തീ പാറുന്ന കുണ്ഡലങ്ങളും മലപോലുള്ള വയറും അഗ്നിശലാകകള്‍ പോലുള്ള രോമങ്ങളും ചാട്ടുളിപോലുള്ള അംഗങ്ങളും കൊണ്ട് കൊടുംഭീകരമായിരുന്നു നരസിംഹമൂര്‍ത്തി.  
   
നരസിംഹത്തിന്റെ ക്രോധദൃഷ്ടിയില്‍ നിന്നും രക്ഷപ്പെടാന്‍ രാക്ഷസന്മാര്‍ നാനാദിക്കുകളിലേയ്ക്കും ഓടിപ്പോയി. കൊട്ടാരത്തിലെ അന്തപ്പുരങ്ങള്‍ കത്തി ചാരമായി. പ്രഹ്ലാദന്‍ മാത്രം ഭയമേതുമില്ലാതെ അച്ഛന്റെ അന്ത്യകര്‍മ്മങ്ങള്‍ നടത്തി. അദ്ദേഹം മുറിവേറ്റവര്‍ക്ക് സാന്ത്വനമേകി. ഭീകരമായ നാശത്തിന്റെ അളവുകണ്ട് പ്രഹ്ലാദനും കൂട്ടരും സ്തബ്ധരായി നിന്നു.

പ്രഹ്ലാദന്‍ പറഞ്ഞു: ഇപ്പോള്‍ നമ്മെ സഹായിക്കാന്‍ ആരുണ്ട്? അസുരവംശത്തിന്റെ വേര് തന്നെ ഭഗവാന്‍ഹരി കടപുഴക്കിയിരിക്കുന്നു. നമ്മുടെ ശത്രുക്കള്‍ സേനയെ അപ്പാടേ കീഴടക്കി. എന്റെ അച്ഛന്റെ പാദം തൊട്ടു നമസ്കരിക്കാറുള്ള ദേവന്മാര്‍ നമ്മുടെ വസതികള്‍പോലും കീഴടക്കിയിരിക്കുന്നു. നമ്മുടെ ബന്ധുക്കളുടെ പ്രാമുഖ്യമെല്ലാം നഷ്ടപ്പെട്ടിരിക്കുന്നു. അവര്‍ തൊഴിലില്ലാതെ, ഉത്സാഹം നശിച്ച് അനാഥരായിരിക്കുന്നു. ഒരിക്കല്‍ അതീവ ബാലവാന്മാരായിരുന്ന അസുരന്മാരിപ്പോള്‍ തുലോം അവശരും ഭയചകിതരുമായിരിക്കുന്നു. നിയതിയുടെ ലീല അതിവിചിത്രം തന്നെ!

“ഭീരുവായ ഒരു മാന്‍പേട പരിചയമില്ലാത്ത ഒരു ഗ്രാമത്തിലെത്തിയാല്‍പ്പിന്നെ ഒരില നിലത്തുവീണാല്‍പ്പോലും ഭയന്ന് വിറയ്ക്കും. അതുപോലെ ഒരുകാലത്ത് വീരയോദ്ധാക്കളായിരുന്ന ഈ അസുരന്മാര്‍ എന്ത് കണ്ടാലും പരിഭ്രമിക്കുന്നു.”

നമ്മുടേതായിരുന്ന കല്‍പ്പവൃക്ഷം ദേവന്മാര്‍ കൊണ്ടു പോയി. പണ്ട് അസുരന്മാര്‍ ദേവസുന്ദരിമാരുടെ മുഖം കൊതിയോടെ നോക്കിയിരുന്നുവെങ്കില്‍ ഇപ്പോള്‍ ദേവന്മാര്‍ അസുരസുന്ദരിമാരെ നോട്ടമിട്ടിരിക്കുന്നു. അസുരന്മാരുടെ അന്തപ്പുരങ്ങളില്‍ സുഖമായിക്കഴിഞ്ഞിരുന്ന അപ്സരസ്സുകള്‍  കാട്ടിലേയ്ക്കോ മഹാമേരുവിന്റെ മുകളിലേയ്ക്കോ ഓടിപ്പോയി പക്ഷികളെപ്പോലെ ആകാശത്തിനു കീഴില്‍ കഴിയുന്നു. എന്റെ അമ്മമാര്‍ ദു:ഖമൂര്‍ത്തികളായിത്തീര്‍ന്നിരിക്കുന്നു. എന്റെ അച്ഛനെ വീശിയിരുന്ന വിശറിയിപ്പോള്‍ ഇന്ദ്രനെയാണ് സേവിക്കുന്നത്. ഭഗവാന്‍ വിഷ്ണുവിന്റെ ‘കൃപയാല്‍’ അനിതരസാധാരണവും വിവരണാതീതവുമായ കെടുതികളാണ് നാം ഇപ്പോളനുഭവിക്കുന്നത്. ആലോചിക്കുന്തോറും ഈ ദുരവസ്ഥയില്‍ നിന്ന് കരകേറാന്‍ മാര്‍ഗ്ഗമോ പ്രത്യാശയ്ക്കുള്ള വഴികളോ ഞാന്‍ കാണുന്നില്ല.

