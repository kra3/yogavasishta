\newpage
\section{ദിവസം 113}

\slokam{
കഥ്യതേ ജീവനാമ്നൈതച്ചിതം  പ്രതനുവാസനം\\
ശാന്തദേഹചമത്കാരം ജീവം വിദ്ധി ക്രമാത്‌പരം (3/91/23)\\
}

സൃഷ്ടാവായ ബ്രഹ്മാവ്‌ പറഞ്ഞു: മനസ്സ്‌ എന്ന വ്യക്തിബോധത്തിന്‌ വൈവിദ്ധ്യമേറിയ ഗുണങ്ങളും സാധ്യതകളുമുണ്ട്‌... സുഗന്ധവസ്തുക്കളില്‍ അവയ്ക്ക്  യോജിച്ച സ്വാദും അടങ്ങിയിട്ടുണ്ടല്ലോ. ബോധം തന്നെയാണ്‌ സൂക്ഷ്മശരീരമാവുന്നതും ഭൌതീകരൂപമെടുത്ത്‌ സ്ഥൂലശരീരമാവുന്നതും."ഈ വ്യക്തിബോധത്തിന്‌ ജീവന്‍ എന്നും ജീവാത്മാവ്‌ എന്നും പറയുന്നു. അതിലും അനന്തസാധ്യതകള്‍ അന്തര്‍ലീനമാണെങ്കിലും ആ സാദ്ധ്യതകള്‍ അതീവ സൂക്ഷ്മ നിലയിലാണുള്ളത്‌.. എന്നാല്‍ ജീവന്റെ മായാജാലങ്ങളെല്ലാം അവസാനിക്കുമ്പോള്‍ അതു തന്നെയാണ്‌ പരമാത്മാവായി തിളങ്ങുന്നത്‌.". ഞാനോ മറ്റാരെങ്കിലുമോ ഈ പ്രപഞ്ചത്തില്‍ ഇല്ല. എല്ലാം അനന്താവബോധം മാത്രം. ആ ചെറുപ്പക്കാരുടെ ഉദ്ദേശം സാധിതമായപോലെ ഇക്കാണപ്പെടുന്നതെല്ലാം അനന്താവബോധത്തിന്റെ അടിസ്ഥാനത്തിലാണ്‌ നിലനില്‍ക്കുന്നത്‌.. ആ ചെറുപ്പക്കാര്‍ക്ക്‌ അവരുടെ ഉദ്ദേശം സൃഷ്ടിനിര്‍വ്വഹണമായിരുന്നതുകൊണ്ട്‌ അവരില്‍ , അവരാണ്‌ സൃഷ്ടാക്കള്‍ എന്ന തോന്നലുണ്ടായി. അതുപോലെ തന്നെ എനിക്കും എന്റെ സൃഷ്ടി വൈഭവത്തെപറ്റി ഒരവബോധമുണ്ടായി.

അനന്താവബോധം തന്നെയാണ്‌ സ്വയം ജീവനായി ഭാവിച്ച്‌ മനസ്സാകുന്നതും പിന്നീട്‌ ശരീരഭാവമെടുത്ത്‌ അതു മൂര്‍ത്തീകരിക്കുന്നതും. സ്വപ്നംപോലെയുള്ള ഈ മാസ്മരീക ഭ്രമം നീണ്ടുപോവുന്തോറും യാഥാര്‍ത്ഥ്യമാണെന്ന തോന്നലുണ്ടാവുന്നു. അത്‌ സത്തും അസത്തുമാണ്‌.. മനസ്സില്‍ അതിനെ ഭാവനചെയ്തു കണ്ടതുകൊണ്ട്‌ അതു സത്താണ്‌.. പക്ഷേ സ്വതവേയുള്ള വിരോധാഭാസംകൊണ്ട്‌ അതു അസത്തുമാണ്‌.. മനസ്സ്‌ ചേതനയുള്ളതാണ്‌. കാരണം അത്‌ ബോധത്തിലധിഷ്ടിതമാണല്ലോ.  എന്നാല്‍ ബോധത്തില്‍നിന്നും വിഭിന്നമായി അതിനെക്കണ്ടാല്‍ അത്‌ ഭ്രമാത്മകമായ വെറും ജഢമാണ്‌.. മനസ്സ്‌ ഒന്നിനെപ്പറ്റി ആലോചിക്കുമ്പോള്‍ ആ വസ്തുവായി സ്വയം ഗ്രഹിക്കുകയാണ്‌.. കൈവളയുടെ രൂപത്തില്‍ മനസ്സില്‍ കാണുമ്പോള്‍ ഒരു സ്വര്‍ണ്ണവള  ആഭരണമാണ്‌.. എന്നാല്‍ മൂല്യവസ്തുവായ സ്വര്‍ണ്ണമാണല്ലോ ആഭരണത്തില്‍ സത്തായിട്ടുള്ളത്‌. .

എല്ലാം ബ്രഹ്മം തന്നെയായതുകൊണ്ട്‌ ജഢമെന്നു പറയുന്നതും അതേ അനന്താവബോധം തന്നെ. എന്നില്‍നിന്നു തുടങ്ങി വെറും കല്ലുവരെ ഒന്നിനേയും സചേതനമെന്നോ അചേതനമെന്നോ നിര്‍വ്വചിക്കുക അസാദ്ധ്യം. തികച്ചും വ്യത്യസ്ഥങ്ങളായ രണ്ടു വസ്തുക്കളെപ്പറ്റി ചിന്താക്കുഴപ്പത്തിനോ ആശങ്കക്കോ  സാദ്ധ്യതയില്ല . വിഷയവും വിഷയിയും തമ്മില്‍ സാമ്യമുണ്ടാവുമ്പോള്‍ മാത്രമേ ഏത്‌ ഏതെന്ന സംശയങ്ങള്‍ ഉണ്ടാവുകയുള്ളു. എല്ലാം നിര്‍വ്വചനാതീതമാണെന്നുള്ളതു കൊണ്ട്‌ ജഢം, ചൈതന്യം, എന്നൊക്കെയുള്ള വാക്കുകള്‍ കേവലശബ്ദങ്ങള്‍ മാത്രം.

മനസ്സിനെ സംബന്ധിച്ചിടത്തോളം വിഷയം ജഢവും, വിഷയി (അറിയുന്നയാള്‍ ചൈതന്യവുമാണ്‌. . അങ്ങിനെ ഭ്രമത്തില്‍ ഉഴന്ന് ജീവന്‍ ചുറ്റിപ്പറ്റിനില്‍ക്കുന്നു. വാസ്തവത്തില്‍ ദ്വന്ദതയെന്നതും മനസ്സിന്റെ വിഭ്രമം മാത്രം. അത്തരം വിഭ്രമം ഉള്ളതാണോ അല്ലയോ എന്നും നമുക്ക്‌ നിശ്ചയിക്കുക വയ്യ. അനന്താവബോധമാണെല്ലാം. ഭ്രമാത്മകമായ വിഭജനങ്ങളാണീ കാണപ്പെടുന്നതെല്ലാം എന്ന തിരിച്ചറിവില്ലാത്തപ്പോള്‍ അഹങ്കാരം ഉടലെടുക്കുന്നു. എന്നാല്‍ മനസ്സ്‌ സ്വരൂപത്തില്‍ ധ്യാനനിമഗ്നമാവുമ്പോള്‍ എല്ലാ വിഭജനങ്ങളും ഇല്ലാതാവുന്നു. അനന്താവബോധസാക്ഷാത്കാരം  ഉണ്ടാവുമ്പോള്‍ പരമാനന്ദവും ആവുന്നു. 

