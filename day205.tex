\section{ദിവസം 205}

\slokam{
ഉപശമസുഖമാഹരേത് പവിത്രം\\
സുശമവത: ശമമേതി സാധുചേത:\\
പ്രശമിതമനസ: സ്വകേ സ്വരൂപേ\\
ഭവതി സുഖേ സ്ഥിതിരുത്തമാ ചിരായ (5/8/18)\\
}

വസിഷ്ഠൻ തുടർന്നു: രാമാ, ജനകൻ എന്നുപേരായി വിദേഹരാജ്യം ഭരിക്കുന്ന ഒരു ചക്രവർത്തിയുണ്ട്. അനന്തമായ ആ ദിവ്യദർശനത്തില്‍ സദാ അഭിരമിക്കുന്ന ആളാണദ്ദേഹം. തന്നെ ആശ്രയിച്ചുവരുന്നവർക്കൊരക്ഷയപാത്രം തന്നെയാണ് ജനകന്‍. അദ്ദേഹത്തിന്റെ സാന്നിദ്ധ്യമാത്രയിൽ സുഹൃത്തുകളുടെ ഹൃദയകമലങ്ങൾ പൂത്തുലയുകയായി. അവർക്കദ്ദേഹം സൂര്യനാണ്‌... എല്ലാ സദ് വൃത്തരുടേയും ഉത്തമസഹചാരിയാണദ്ദേഹം.

ഒരുദിവസം തന്റെ നന്ദനോദ്യാനത്തിൽ ഉല്ലാസനടത്തത്തിലായിരുന്നപ്പോള്‍ അദ്ദേഹം എവിടെനിന്നോ അശരീരിയായി പ്രചോദനപ്രദമായ ചില ഗീതകങ്ങള്‍ പാടിക്കേട്ടു. ജ്ഞാനസമ്പന്നരായ ഏതോ മാമുനിമാരുടേതാവണം ഈ അശരീരി. അവരിങ്ങിനെ പാടി:

‘അനുഭവമെന്ന സാധകൻ സാധകവിഷയവുമായി (അനുഭവങ്ങളുമായി) ബന്ധപ്പെടുമ്പോൾ സ്വയം വെളിപ്പെടുന്ന ആനന്ദസ്വരൂപമായ, ആശയജന്യമോ ധാരണാത്മകമോ അല്ലാത്ത; ഭിന്നതകളില്ലാത്ത അത്മാവിനെ ഞങ്ങൾ ധ്യാനിക്കുന്നു.

വിഷയം, വിഷയി എന്ന വിഭിന്നാനുഭങ്ങളെല്ലാമവസാനിക്കുമ്പോൾ മേധാശ്രയമില്ലാതെ തന്നെ വിഷയങ്ങളെ പ്രതിഫലിപ്പിക്കുന്ന ആത്മാവിനെ ഞങ്ങൾ ധ്യാനിക്കുന്നു.

ദ്വിവിധ ധാരണകളാകുന്ന അസ്തിത്വം, അനസ്തിത്വം (ഉണ്ട്, ഇല്ല)എന്ന അവസ്ഥകൾക്കതീമായി വർത്തിച്ച് അവയ്ക്കു മദ്ധ്യത്തിലെന്നപോലെ നിലകൊള്ളുന്ന അത്മാവെന്ന ആ പ്രകാശവസ്തുവിനെ ഞങ്ങൾ ധ്യാനിക്കുന്നു.

എല്ലാത്തിനും കാര്യകാരണവസ്തുതയായിരിക്കുന്ന, എല്ലാറ്റിന്റേയും സൃഷ്ടി-സ്ഥിതി-സംഹാരകാരകവുമായ ആത്മാവിനെ ഞങ്ങൾ ധ്യാനിക്കുന്നു.

ആദ്യാക്ഷരമായ 'അ' യ്ക്കും അന്ത്യാക്ഷരമായ 'ക്ഷ' യ് ക്കും ഇടയ്ക്കുള്ള എല്ലാ അക്ഷരങ്ങളാലും വ്യവക്ഷിക്കപ്പെട്ട, എല്ലാ ഭാവാവിഷ്കാരങ്ങൾക്കും ഭാഷകൾക്കും അടിസ്ഥാനമായ ആത്മാവിനെ ഞങ്ങൾ ധ്യാനിക്കുന്നു.

കഷ്ടമെന്നുപറയട്ടേ, സ്വന്തം ഹൃദയഗുഹയിൽ നിവസിക്കുന്ന ഭഗവാനെ ഉപേക്ഷിച്ച് മനുഷ്യർ മറ്റു വസ്തുക്കൾക്കുപിറകേ പരക്കം പായുന്നു. ലൗകീകവസ്തുക്കളുടെ നിരർത്ഥകതയെപ്പറ്റി അറിവുണ്ടായിട്ടും ഹൃദയത്തിൽ അവയോടുള്ള മമത ഉപേക്ഷിക്കാത്തവർ മനുഷ്യരല്ല. ഹൃദയത്തിൽ ഇടം  പിടിച്ചിട്ടുള്ളതും അവിടെ വേരുറയ്ക്കാൻ സാദ്ധ്യതയുള്ളതുമായ എല്ലാ ആസക്തികളേയും നാം വിവേകമാവുന്ന ദണ്ഡുകൊണ്ട് അടിച്ചോടിക്കുകതന്നെ വേണം.

“പ്രശാന്തതയിൽ നിന്നും ഉയരുന്ന ആഹ്ളാദമാണു നാം ആസ്വദിക്കേണ്ടത്. മനോനിയന്ത്രണം വന്ന ഒരുവന്റെ മനസ്സ് പ്രശാന്തമാണല്ലോ. അങ്ങിനെ അവിടെ ശുദ്ധമായ ആനന്ദം നിറയുന്നു.” 

