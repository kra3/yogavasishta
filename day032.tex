 
\section{ദിവസം 032}

\slokam{
സന്തോഷ: പരമോ ലാഭ: സത്സംഗ: പരമാ ഗതി:
വിചാര: പരമം ജ്ഞാനം ശമോ ഹി പരമം സുഖം (2/16/19)
}

വസിഷ്ഠന്‍ തുടര്‍ന്നു: ആത്മസംതൃപ്തി, മുക്തികവാടത്തിലെ മറ്റൊരു കാവല്‍ ക്കാരനാണ്‌. സ്വയം സംതൃപ്തിയെന്ന അമൃതുമൊത്തിക്കുടിച്ചവനെ ഇന്ദ്രിയ  സുഖഭോഗങ്ങള്‍ ആസക്തനാക്കുകയില്ല. എല്ലാ പാപങ്ങളേയും ഇല്ലാതാക്കുന്ന സംതൃപ്തിയേപ്പോലെ ഈ ലോകത്തില്‍ ആഹ്ലാദപ്രദമായി മറ്റൊന്നുമില്ല. എന്താണീ ആത്മസംതൃപ്തി? കിട്ടാത്ത വസ്തുക്കളില്‍ ആസക്തിലേശമില്ലാതെയും സ്വയമേവ വന്നുചേരുന്നവയില്‍ തൃപ്തനായും, രണ്ടിലും വിഷാദം, സന്തോഷം തുടങ്ങിയ വികാരവായ്പ്പുകള്‍ കൂടാതേയും ഇരിക്കുന്ന അവസ്ഥയാണിത്‌. ആത്മസംതൃപ്തിയില്ലാത്തവന്‌ വിഷാദം സഹജം. എന്നാല്‍ ആത്മസംതൃപ്തിയുള്ളവനോ, ഒന്നും 'സ്വന്ത്‌'മായി ഇല്ലാത്തപ്പോഴും അവന്‍ ലോകത്തിന്റെ മുഴുവന്‍ അധിപനാണ്‌. അവന്റെ ഹൃദയം നിര്‍മലമായി ലോകത്തോളം വികസിതമാവുന്നു.

സത്സംഗം - സദ്ജനങ്ങളുടെ, മഹാത്മാക്കളുടെ, പ്രബുദ്ധരുമായുള്ള സംസര്‍ഗ്ഗം - മുക്തികവാടത്തിലെ മറ്റൊരു കാവലാളാണ്‌. സത്സംഗം ഒരുവന്റെ ബുദ്ധിയെ പ്രചോദിപ്പിച്ച്‌ അജ്ഞാനത്തെ നീക്കി മാനസീകാസ്വാസ്ഥ്യങ്ങളെ അകറ്റാന്‍ സഹായിക്കുന്നു. എന്തു ത്യാഗം സഹിച്ചും, എന്തു വിലകൊടുത്തും സത്സംഗത്തെ അവഗണിക്കാതിരിക്കാന്‍ പ്രത്യേകം ശ്രദ്ധിക്കണം. കാരണം അത്‌ ജീവിതപ്പാതയിലെ കെടാവിളക്കാണ്‌. ദാനധര്‍മ്മാദികള്‍, തപസ്സ്‌, തീര്‍ത്ഥാടനം, യാഗകര്‍മ്മങ്ങള്‍ എന്നിവയേക്കാളേറെ ഉയരെയാണ്‌ സത്സംഗത്തിന്റെ സ്ഥാനം.

ഹൃദയത്തിലെ അജ്ഞാനാന്ധകാരം നീങ്ങി സത്യസാക്ഷാത്കാരം നേടിയ മഹാത്മാക്കളെ കഴിവിന്റെ പരമാവധി ബഹുമാനിച്ച്‌ പൂജിക്കണം. മഹാത്മാക്കളെ ബഹുമാനിക്കാത്തവന്‌ ദുരിതമോചനം ലഭിക്കുകയില്ല. ഈ നാലുകാവല്‍ക്കാരാണ്‌ - ആത്മസംതൃപ്തി, സത്സംഗം, അത്മാന്വേഷണത്വര, ആത്മസംയമനം എന്നിവ സംസാരസാഗരത്തില്‍ മുങ്ങുന്നവന്റെ രക്ഷാമാര്‍ഗ്ഗങ്ങള്‍. 

"ആത്മസംതൃപ്തി പരമപ്രധാനമാണ്‌. സത്സംഗം ലക്ഷ്യമാര്‍ഗ്ഗത്തിലെ ഉത്തമ സഹയാത്രികരാണ്‌. ആത്മാന്വേഷണം പരമാര്‍ത്ഥലാഭം തന്നെയാണ്‌. ആത്മ നിയന്ത്രണം പരമാനന്ദപ്രദമാണ്‌." 

ഈ നാലും ഒരുപോലെ അനുഷ്ഠിക്കാനാവുന്നില്ലെങ്കില്‍ ഏതെങ്കിലുമൊന്ന് ആചരിക്കുക. ജാഗരൂകമായി ഇവയിലൊന്നനുഷ്ഠിച്ചു പക്വതവരുമ്പോള്‍ മറ്റുള്ളവ സ്വയം നിന്നെ തേടിയെത്തിക്കൊള്ളും. പരം പൊരുള്‍ സ്വയം നിന്നില്‍ ഉണരുകയും ചെയ്യും. നീ ഒരു ദേവനോ, ഉപദേവതയോ, മരമോ ആരുതന്നെ ആയിക്കൊള്ളട്ടെ, മനസ്സ്‌ എന്ന മദയാനയെ ഈ നാലു ഗുണങ്ങളാല്‍ മെരുക്കാതെ നിനക്ക്‌ പുരോഗതിയുണ്ടാവുകയില്ല. അതിനാല്‍ രാമ: ഈ ഉത്തമ സ്വഭാവങ്ങള്‍ ഏതുവിധേനേയും വളര്‍ത്തിക്കൊണ്ടുവരാന്‍ പരിശ്രമിച്ചാലും. 
