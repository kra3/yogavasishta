\newpage
\section{ദിവസം 084}

\slokam{
ദു:ഖിതസ്യ നിശാകല്‍പ: സുഖിതസൈവ ച ക്ഷണ:\\
ക്ഷണ സ്വപ്നേ ഭവേത്കല്‍പ:  കല്‍പശ്ച ഭവതി ക്ഷണ: (3/60/22)\\
}

വസിഷ്ഠന്‍ തുടര്‍ന്നു: രാജാവിനു വേണ്ട വരമെല്ലാം നല്‍കി സരസ്വതി അവിടേനിന്നും അപ്രത്യക്ഷയായി. രാജാവും രാജ്ഞിയും ആലിംഗനബദ്ധരായി. ഉറക്കമുണര്‍ന്ന സേവകര്‍ രാജാവിനു ജീവന്‍ തിരിച്ചുകിട്ടിയതറിഞ്ഞ്‌ ആഹ്ലാദചിത്തരായി. രാജ്യം മുഴുവന്‍ അഘോഷങ്ങളുണ്ടായി. ലീലരാജ്ഞിയുടെ തിരിച്ചുവരവിനെപ്പറ്റിയും രാജാവിനായി മറ്റൊരു ലീലയെ നല്‍കിയതിനെപ്പറ്റിയും രാജ്യത്തിനകത്തും പുറത്തുള്ള ആളുകള്‍ കഥകള്‍ പറഞ്ഞു നടന്നു. പ്രബുദ്ധയായ ലീലയില്‍നിന്നും തന്റെ പൂര്‍വ്വജന്മവൃത്താന്തങ്ങളെല്ലാം രാജാവ്‌ കേട്ടറിഞ്ഞു.അദ്ദേഹം ഏറെക്കാലം ത്രിലോകങ്ങളുടെ അനുഗ്രഹങ്ങളോടെ സരസ്വതീ ദേവിയുടെ കൃപയാല്‍ ആനന്ദപൂര്‍വ്വം രാജ്യം ഭരിച്ചു. എന്നാല്‍ ഇതെല്ലാം അദ്ദേഹത്തിന്റെ സ്വപരിശ്രമം കൊണ്ടുനേടിയതാണെന്നതിനു സംശയമൊന്നുമില്ല. 

രാമ: ഇതാണ്‌ ലീലോപാഖ്യാനം. ഞാനിത്‌ വിശദമായിത്തന്നെ നിനക്കു പറഞ്ഞു തന്നു. കാണപ്പെടുന്ന വിഷയങ്ങളെ സത്യമെന്നു ധരിക്കുന്ന അജ്ഞതയെ ദൂരീകരിക്കാന്‍ ഈ കഥയെകുറിച്ച്‌ ധ്യാനിക്കുന്നതിലൂടെ നിനക്കു കഴിയും. സത്തായി, ഉള്ളതിനെ മാത്രമല്ലേ നീക്കിമാറ്റാന്‍ കഴിയുകയുള്ളു? സത്തല്ലാത്തതിനെ എങ്ങിനെ മാറ്റാനാണ്‌? ഒന്നും നീക്കി മാറ്റാനായി ഇല്ല. എല്ലാം - ഭൂമിയും മറ്റും- നിന്റെ കണ്ണിലെ വെറും ഭ്രമദൃശ്യങ്ങളത്രേ. എല്ലാം അനന്താവബോധമല്ലാതെ മറ്റൊന്നല്ല. എന്തെങ്കിലും 'സൃഷ്ടിച്ചിട്ട്‌' അതിനുപകരം മറ്റൊന്ന് സ്ഥാപിച്ചാലും അത്‌ അനന്തതയില്‍ത്തന്നെയാണു നിലകൊള്ളുന്നത്‌. എല്ലാം എങ്ങിനെ എപ്രകാരമുണ്ടോ അങ്ങിനെത്തന്നെ നിലനില്‍ക്കുന്നു. ഒന്നും സൃഷ്ടിക്കപ്പെട്ടിട്ടില്ല. വിഷയലോകം സൃഷ്ടിക്കപ്പെട്ടു എന്നു നമുക്കു തോന്നുന്നത്‌ മായാശക്തിയുടെ സര്‍ഗ്ഗവൈഭവമാണെന്നു നാം പറഞ്ഞേക്കാം. എന്നാല്‍ ഈ മായയും ഉണ്മയല്ല.

രാമന്‍ പറഞ്ഞു: പരമസത്യത്തെപ്പറ്റി എത്ര ഉദാത്തമായ ഉള്‍ക്കാഴ്ച്ചയും വീക്ഷ്ണവുമാണ്‌ അങ്ങെനിക്കു തന്നത്‌! എന്നാല്‍ മഹര്‍ഷേ അങ്ങയുടെ അമൃതസമാനമായ വാക്കുകള്‍ക്കായി എന്നില്‍ ഇനിയും ദാഹമുണ്ട്‌.. . ദയവായി കാലത്തിന്റെ രഹസ്യവും ഗൂഢാര്‍ത്ഥവും എന്തെന്നു പറഞ്ഞു തരൂ. ലീലയുടെ കഥയിലൊരു ജീവിതകാലം മുഴുവന്‍ കടക്കാന്‍ ചിലപ്പോള്‍ എട്ടു ദിവസം മാത്രമെടുക്കുന്നു. മറ്റുചിലപ്പോള്‍ ഒരു മാസമെടുക്കുന്നു! ഇതെന്നെ കുഴക്കുന്നു! വിവിധ പ്രപഞ്ചങ്ങളില്‍ കാലമാപിനി വ്യത്യസ്തമാണോ? 

വസിഷ്ഠന്‍ പറഞ്ഞു: രാമ: ഒരുവന്‍ തന്റെ മേധാശക്തികൊണ്ട്‌ എന്തു ചിന്തിക്കുന്നുവോ അതാണവന്‍ അനുഭവിക്കുന്നത്‌.. അമൃതുപോലും വിഷമായിമാറും അമൃതിനെ വിഷമായി ഭാവന ചെയ്താല്‍ .   സുഹൃത്തുക്കളെ  ശത്രുക്കളാക്കാനും ശത്രുക്കളെ സുഹൃത്തുക്കളാക്കുവാനും നമ്മുടെ മനോഭാവത്തിനു കഴിയും. വിഷയത്തെ അനുഭവിക്കുന്നത്‌ അതിനോടു നമുക്കുള്ള മനോഭാവത്തെ ആശ്രയിച്ചു മാത്രമാണിരിക്കുന്നത്‌. "ദു:ഖിതന്‌ ഒരു രാത്രി ഒരു യുഗം പോലെ ദൈര്‍ഘ്യമേറിയതാണ്‌. എന്നാല്‍ ആഹ്ലാദത്തില്‍ മുങ്ങിയ ഒരു നിശ, ക്ഷണനേരമേയുള്ളൂ എന്നു തോന്നുന്നു. സ്വപ്നത്തില്‍ ക്ഷണനേരവും യുഗങ്ങളും തമ്മില്‍ ഒരു വ്യത്യാസവുമില്ല." മനുവിന്റെ ഒരു ജീവിതകാലം -ആയുസ്സ്‌)))- - ബ്രഹ്മാവിന്റെ ഒന്നര മണിക്കൂറാണ്‌. .  ബ്രഹ്മാവിന്റെ ആയുസ്സ്‌ വിഷ്ണുവിന്റെ ഒരു ദിവസമാണ്‌. വിഷ്ണുവിന്റെ ആയുസ്സോ?പരമശിവന്റെ ഒരു ദിവസം മാത്രം. എന്നാല്‍ സീമകള്‍ക്കതീതമായ ബോധമുണര്‍ന്നവന്‌ രാത്രിയോ പകലോ ഇല്ല.

