\section{ദിവസം 208}

\slokam{
സ്ഥിതേ മനസി നിഷ്കാമേ സമേ വിഗതരഞ്ജനേ\\
കായാവയവജൗ കാര്യൗ സ്പന്ദാസ്പന്ദൗ ഫലേ സമൗ (5/10/28)\\
}

വസിഷ്ഠൻ തുടർന്നു: രാജാവങ്ങിനെ തീവ്രമായ സാധനയിലിരിക്കുന്നതുകണ്ട് ഒരംഗരക്ഷകൻ ആദരപൂർവ്വം അദ്ദേഹത്തെ സമീപിച്ച് ഇങ്ങിനെ പറഞ്ഞു: ‘മഹാരാജാവേ, ഭരണകാര്യങ്ങൾ നോക്കാനുള്ള സമയമായി. പരിചാരികമാർ അങ്ങയെ പരിചരിക്കാൻ തയ്യറായി നില്‍ക്കുന്നു. നീരാട്ടിനായി സുഗന്ധതൈലമിട്ട ജലമെടുത്തുവെച്ചിരിക്കുന്നു. ഉചിതമായ മന്ത്രോച്ചാരണങ്ങൾ തുടങ്ങാനായി വന്ദ്യരായ പൂജാരിമാർ അങ്ങയെക്കാത്ത് സ്നാനഗൃഹത്തിൽ കാത്തുനിൽപ്പുണ്ട്. എഴുന്നേക്കൂ, അനുഷ്ഠിക്കേണ്ട കർമ്മങ്ങളെല്ലാം നടക്കണമല്ലോ. ഉത്തമരും പാവനചരിതന്മാരും ഒരിക്കലും സമയനിഷ്ഠ തെറ്റിക്കുകയോ അലംഭാവം കാണിക്കുകയോ ഇല്ല.’

എന്നാൽ രാജാവ് സേവകന്റെ വാക്കുകൾകേട്ടിട്ടും അങ്ങിനെ തന്നെയിരുന്ന് തന്റെ ആലോചന തുടർന്നു: 'ഈ രാജസഭയും രാജധർമ്മങ്ങളും കൊണ്ട് എനിക്കെന്തു പ്രയോജനം? ഇതെല്ലാം വെറും താൽക്കാലികമല്ലേ? അവകൊണ്ടെനിക്ക് യാതൊരു ഗുണവുമില്ല. ഞാൻ എല്ലാ കർമ്മധർമ്മങ്ങളും ഉപേക്ഷിച്ച് ആത്മാനന്ദത്തിൽ ആമഗ്നനാവാൻ പോവുന്നു. മനസ്സേ നീ ലൗകീകസുഖാനുഭവാസക്തി തീരെ ഉപേക്ഷിച്ചാലും. അങ്ങിനെ ജനന മരണചക്രത്തിന്റെ ആവർത്തനം മൂലമുണ്ടാകുന്ന ദുരിതങ്ങൾ ഒഴിവാക്കാമല്ലോ. സന്തോഷവാഹിയെന്നു കരുതുന്ന എന്തൊക്കെ നീ ആഗ്രഹിക്കുന്നുവോ അതൊക്കെ ദു:ഖമാണു നൽകുന്നതെന്ന് തെളിയിക്കപ്പെട്ടിരിക്കുന്നു. പാപപങ്കിലമായ, സുഖാസക്തമായ ഈ ജീവിതം ഇനി മതി. സഹജമായി നിന്നിൽ എന്നുമുള്ളതുമായത്  എന്താണൊ അതിനെ അന്വേഷിക്കുന്നതിൽ ആഹ്ളാദം കണ്ടെത്തൂ.' രാജാവ് മൗനമവലംബിച്ചതുകണ്ട് സേവകനും മൂകനായി അവിടെ നിന്നു.

രാജാവ് വീണ്ടും ആത്മഗതമായി ഇങ്ങിനെ പറഞ്ഞു: 'ഞാനീ ലോകത്ത് എന്താണു നേടേണ്ടത്? ഈ പ്രപഞ്ചത്തിലെ ഏതൊരു ശാശ്വതസത്യത്തെയാണ്‌ ആത്മവിശ്വാസത്തോടെ ഞാൻ ആലംബമാക്കേണ്ടത്? ഞാൻ തുടർച്ചയായി കർമ്മങ്ങൾ ചെയ്താലുമില്ലെങ്കിലും എന്തു വ്യത്യാസമാണുണ്ടാവാൻ പോവുന്നത്? ഈ ലോകത്ത് നിലനില്ക്കുന്നതായി യാതൊന്നുമില്ല. കർമ്മനിരതമായാലും അല്ലെങ്കിലും ഈ ശരീരം എപ്പോഴും മാറ്റങ്ങൾക്കു വിധേയമാണ്‌.. അതു ശാശ്വതവുമല്ല. സമത എന്ന ഭാവത്തിൽ ബുദ്ധിയുറച്ചിരിക്കുമ്പോൾ പിന്നെ എന്ത്, എങ്ങിനെ നഷ്ടപ്പെടാൻ? ഏനിക്കു കിട്ടാത്തതിനെക്കുറിച്ച് വ്യാകുലതയില്ല. സ്വയമേവ ഞാനാവശ്യപ്പെടാതെ എന്നിൽ വന്നുചേർന്നതിനെ വിട്ടുകളയാൻ ഞാൻ പരിശ്രമിക്കുന്നുമില്ല. ഞാൻ ആത്മനിഷ്ഠനാണ്‌.. എനിക്കുള്ളത് എനിക്കുതന്നെയാണ്‌.. ഒന്നിനുവേണ്ടിയും ഞാൻ പ്രയത്നിക്കേണ്ടതില്ല. എന്നാൽ കർമ്മവിമുഖനാവുന്നതുകൊണ്ടും എന്താണു പ്രയോജനം? കർമ്മംകൊണ്ടോ കർമ്മനിരാസംകൊണ്ടോ നേടുന്നതെല്ലാം വിഫലമത്രേ.'

“മനസ്സ്  ആഗ്രഹരഹിതവും സുഖാന്വേഷണാസക്തിയില്ലാതെയും ശരീരാവയവങ്ങൾ അവയവയുടെ സഹജധർമ്മങ്ങൾ അനുഷ്ഠിക്കുകയും ചെയ്യുമ്പോൾ കർമ്മത്തിനും കർമ്മരാഹിത്യത്തിനും ഒരേ സാരവും മൂല്യവുമാണ്‌.” അതുകൊണ്ട് ദേഹം അതിന്റെ കർമ്മങ്ങൾ സഹജഭാവത്തിൽത്തന്നെ ചെയ്യട്ടെ. കാരണം അതില്ലെങ്കിൽ ഈ ദേഹം നശിക്കുമല്ലോ. മനസ്സിൽ ’ഞാനിതു ചെയ്യുന്നു‘ ’ഞാൻ ഇതാസ്വദിക്കുന്നു‘ എന്നിങ്ങനെയുള്ള ധാരണകൾ വെച്ചുപുലർത്താതിരിക്കുമ്പോൾ കർമ്മം അകർമ്മമാവുന്നു. കർമ്മരാഹിത്യമല്ല ഇത്. 

