\newpage
\section{ദിവസം 116}

\slokam{
സര്‍വ്വ ഏതാ: സമായാന്തി ബ്രഹ്മണോ ഭൂതജാതയ:\\
കിഞ്ചിത്‌പ്രചലിതാ  ഭോഗാത്‌പയോരാശോരിവോര്‍മയ: (3/94/19)\\
}

വസിഷ്ഠന്‍ തുടര്‍ന്നു: ഇനി ഞാന്‍ ആദികാലം മുതലുള്ള സൃഷ്ടികളിലെ ഉത്തമം, മധ്യമം, അധമം എന്നീ  മൂന്നു തരങ്ങളെക്കുറിച്ചു പറയാം. ഉത്തമ ജീവികള്‍ സദ്കര്‍മ്മങ്ങളുടെ ഫലമായാണുണ്ടാവുന്നത്‌... അവര്‍ സ്വാഭാവികമായും നന്മനിറഞ്ഞവരും സദ്കര്‍മ്മികളുമാണ്‌... അവര്‍ കുറച്ചു ജന്മങ്ങള്‍ക്കുശേഷം മുക്തിയെ പ്രാപിക്കുന്നു. അവരില്‍ നിര്‍മ്മലഗുണങ്ങള്‍ നിറഞ്ഞു വിളങ്ങുന്നു. (സത്വഗുണങ്ങള്‍ ).  പിന്നെയൊരുകൂട്ടരുണ്ട്‌., അധമര്‍ . അവരില്‍ മലിനത നിറഞ്ഞിരിക്കുന്നു. ലൌകീകത അവരില്‍ രൂഢമൂലവും അവരുടെ സ്വഭാവം നാനാതരത്തിലുള്ളതുമാണ്. അവരുടെ മുക്തിക്ക്‌ ആയിരക്കണക്കിനു ജന്മങ്ങള്‍ വേണ്ടിവരും. ഇക്കൂട്ടരെ നന്മയുടെ പാതയില്‍ കാണുക വിരളം. അവരില്‍ ചിലരുടെ കാര്യത്തില്‍ മുക്തിലഭിക്കുകയെന്നത്‌ ഈ ലോകചക്രത്തില്‍ സംശയമാണ്‌... അവര്‍ കൂരിരുട്ടിലകപ്പെട്ട ജീവികളത്രേ. (താമസീകം).

മധ്യമരായിട്ടുള്ളവര്‍ ഊര്‍ജ്ജിതരും ഉത്സാഹശീലരും സദാ ആഗ്രഹള്‍   ഉള്ളവരുമാണ്‌. (.(രാജസീകം). അങ്ങിനെയുള്ളവര്‍ മുക്തിപദത്തിലേക്കടുക്കുമ്പോള്‍ അവരില്‍ രജോഗുണങ്ങളും സത്വഗുണങ്ങളും സമിശ്രമാവുന്നു. എന്നാല്‍ രാജസവാസനകള്‍ അതീവ തീവ്രമാണെങ്കില്‍ (ആഗ്രഹങ്ങളോട്‌ വൈകാരീകമായി ആര്‍ത്തിയുള്ളവര്‍ ) അവര്‍ക്കതു തീര്‍ന്നുകിട്ടാന്‍ കുറച്ചുകാലംകൂടിയെടുക്കും.  ഈ രാജസവാസന അതീവ ഘനരൂപത്തില്‍ താമസവാസനയുടെ ഭാവം കൈക്കൊള്ളുന്നു. മുക്തിപദലഭ്യത സംശയമായിട്ടുള്ള അവരില്‍ രാജസഭാവം തികഞ്ഞ അന്ധകാരത്തിലേക്കാണു നയിക്കുന്നത്‌.. ആയിരം ജന്മമെടുത്താലും അജ്ഞതയുടെ ഇരുട്ടില്‍ നിന്നുമുണരാത്തവരാണ്‌ താമസഗുണമുള്ളവര്‍ .  അവര്‍ മുക്തിക്കായി വളരെയേറെക്കാലം അലയേണ്ടിവരും.

അവരില്‍ മുക്തിപദം അടുക്കാറാകുമ്പോള്‍ ഈ താമസീക വാസന, സാത്വീകവാസനയുമായി സമ്മിശ്രമാകുന്നു. മുക്തിപദത്തിലെത്തും മുന്‍പ്‌ അവരില്‍ തമസ്സും രാജസും ചേര്‍ന്നു വര്‍ത്തിക്കുന്നു. നൂറു ജന്മത്തിനുശേഷവും മുക്തിപദം ഇനിയുമൊരു നൂറു ജന്മങ്ങളകലെ ആയിട്ടുള്ളവര്‍ തമസ്സിലാണുള്ളത്‌... എന്നാല്‍ മുക്തിപദം സംശയമായിട്ടുള്ളവര്‍ അന്ധകാരത്തിന്റെ ആഴങ്ങളിലാണലയുന്നത്‌. (.. (സത്വരജസ്‌തമോ ഗുണങ്ങള്‍ മുക്തിക്കു തടസ്സമല്ല എന്നാണ്‌ ഈ അദ്ധ്യായം വിവക്ഷിക്കുന്നത്‌.. എന്നാല്‍ തെറ്റായ ചിന്തകളും കര്‍മ്മങ്ങളും ആണ്‌ മുക്തിപദത്തെ നീട്ടി നീട്ടിക്കൊണ്ടുപോവുന്നത്‌.-: - സ്വാമി വെങ്കിടേശാനന്ദ)

"സമുദ്രോപരിയുണ്ടാവുന്ന അലകള്‍പോലെ, ഈ ജീവജാലങ്ങളെല്ലാം ഉത്ഭവിച്ചത്‌ പരബ്രഹ്മത്തിന്റെ സമതുലിതാവസ്ഥയിലുണ്ടായ ചെറിയൊരു സ്പന്ദനത്തെത്തുടര്‍ന്നാണ്‌." . ഒരുകുടത്തിലെ അകാശം, മുറിയിലെ ആകാശം, ചെറിയൊരു ദ്വാരത്തിലെ ആകാശം എല്ലാം ഒരേയൊരു പ്രപഞ്ചാകാശം തന്നെയാണല്ലോ." അതുപോലെ അനന്തതയാണ്‌ ജീവജാലങ്ങളടക്കം എല്ലാം. അവ ഭാഗങ്ങളല്ല. അനന്തതയില്‍നിന്നും വിക്ഷേപിക്കപ്പെട്ടതുകൊണ്ട്‌ ഈ കാഴ്ചകളെല്ലാം അടങ്ങുന്നതും അതില്‍ത്തന്നെ. അങ്ങിനെ പരബ്രഹ്മത്തിന്റെ ഇച്ഛയ്ക്കൊത്ത്‌ ഇതെല്ലാം ഉണ്ടായി മറയുന്നതായി തോന്നുന്നു. 

