 
\section{ദിവസം 095}

\slokam{
ആപതദ്ദ്ധി മനോ മോഹം പൂര്‍വ്വമാപത്പ്രയച്ഛതി\\
പാശ്ചാദനര്‍ദ്ധവിസ്ഥാരരൂപേണ പരിജൃംഭാതേ (3/71/12)\\
}

വസിഷ്ഠന്‍ തുടര്‍ന്നു: ഇപ്രകാരം ഏറെക്കാലം കഴിഞ്ഞ കാര്‍ക്കടി രാക്ഷസിക്ക്‌ മോഹഭംഗവും മന:പരിവര്‍ത്തനവും വന്നു. അവള്‍ അളുകളെ തിന്നണമെന്ന തന്റെ മൂഢമോഹത്തെപ്പറ്റിയോര്‍ത്ത്‌ പശ്ചാത്തപിച്ചു. ഈ ആര്‍ത്തി തീരാന്‍വേണ്ടി താന്‍ ആയിരംകൊല്ലം തപസ്സുചെയ്തു, വിഷൂചികയായി അധമജീവിതം നയിക്കുകയും ചെയ്തു. അവള്‍ സ്വയംകൃതാനര്‍ത്ഥത്തെച്ചൊല്ലി വേവലാതി പറഞ്ഞു: എന്റെ പര്‍വ്വതാകാര ശരീരമെവിടെ? ഈ സൂചിയെവിടെ? ചിലപ്പോള്‍ ചെളിയിലും മലത്തിലുമാണു ഞാന്‍.. ആളുകള്‍ എന്നെ ചവിട്ടി മെതിക്കുന്നു. എന്തുചെയ്യണമെന്നെനിക്കറിയില്ല. എനിക്കാരും കൂട്ടുകാരായില്ല. എന്നില്‍ ആര്‍ക്കും അനുകമ്പയുമില്ല. എനിക്കു  വീടില്ല, കാണാന്‍ തക്ക വലുപ്പത്തില്‍ ഒരു ശരീരം പോലുമില്ല. എന്റെ മനസ്സും ഇന്ദ്രിയങ്ങളും വൃഥാവിലായിരിക്കുന്നു. 

"കൊടും ദുരിതത്തിലേയ്ക്കു നടന്നടുക്കുന്നവര്‍ ആദ്യം മോഹവിഭ്രമവും ദുഷ്ടതയും സൃഷ്ടിക്കുന്നു. അവയാണ്‌ പിന്നീട്‌ ദു:ഖവും നിര്‍ഭാഗ്യവുമാവുന്നത്‌.". ഞാന്‍ ഒരിക്കലും സ്വതന്ത്രയല്ല. മറ്റുള്ളവരുടെ ദയവിലാണ്‌ എന്റെ അസ്തിത്വം. അവര്‍ എന്നേക്കൊണ്ടു ചെയ്യിക്കുന്നതൊക്കെ ഞാന്‍ ചെയ്തു കൂട്ടുകയാണ്‌.. എല്ലാവരേയും തിന്നൊടുക്കാനുള്ള അത്യാര്‍ത്തിയെന്ന പിശാചിനെ പ്രീണിപ്പിക്കാനായി ചെയ്ത മരുന്ന് രോഗത്തേക്കാള്‍ മോശപ്പെട്ട അവസ്ഥയില്‍ എന്നെ എത്തിച്ചിരിക്കുന്നു. മുന്‍പത്തേതിലും വലിയ പിശാച്‌ ഉണ്ടായിരിക്കുന്നു. തീര്‍ച്ചയായും ഞാനൊരു ബുദ്ധിയില്ലാത്ത മൂഢ. വലിയൊരു ശരീരം വേണ്ടെന്നുവച്ചിട്ട്‌ ഈ സൂചിപോലെ (വൈറസ്സ്‌)?)  വൃത്തികെട്ട രൂപമെടുത്തുവല്ലോ!. പുഴുവിനേക്കാള്‍ കഷ്ടമായുള്ള എന്റെയീ അവസ്ഥയില്‍നിന്നും ആരാണെന്നെ മോചിപ്പിക്കുക? മാമുനിമാരുടെ ഉള്ളിലൊന്നും എന്നേപ്പോലൊരു നികൃഷ്ടജീവിയെപ്പറ്റി ആലോചനകളേയുണ്ടാവില്ല. എനിക്കിനിയെന്നാണ്‌ പണ്ടത്തെപ്പോലെ ഭീമാകാരം പൂണ്ട്‌ വലിയ വലിയ ജീവികളുടെ രക്തം കുടിക്കുവാനാവുക? ഞാന്‍ വീണ്ടും തപസ്സുചെയ്യാന്‍ പോവുന്നു..

കാര്‍ക്കടി, ജീവജാലങ്ങളെ ഭക്ഷിക്കാനുള്ള കൊതിയെല്ലാമുപേക്ഷിച്ച്‌ ഹിമാലയപര്‍വ്വതത്തില്‍പ്പോയി തീവ്രതപസ്സു തുടങ്ങി. ഒറ്റക്കാലിലാണവള്‍ നിന്നത്‌.. തീവ്രതപം കൊണ്ട്‌ അവളുടെ തലയില്‍നിന്നും പുക വമിക്കാന്‍ തുടങ്ങി. അതുമൊരു സൂചികയായി, അവളുടെ സഹചാരിയായി. അവളുടെ നിഴലും മറ്റൊരു സൂചികയായി. മറ്റൊരു സുഹൃത്ത്‌. , മരങ്ങളും വള്ളിച്ചെടികളും അവളുടെ തപസ്സുകണ്ട്‌ അഭിനന്ദനപൂര്‍വ്വം അവള്‍ക്ക്‌ ഭക്ഷണമായി പാരഗവിതരണം ചെയ്തു. എന്നാല്‍ അവള്‍ ഭക്ഷണമേതും കഴിച്ചില്ല. അവള്‍ തന്റെ തീരുമാനത്തില്‍ ഉറച്ചുനിന്നു. സ്വര്‍ഗ്ഗാധിപന്‍ അവള്‍ നിന്നിടത്ത്‌ മാംസത്തിന്റെ ചെറുതുണ്ടുകള്‍ ഇട്ടുകൊടുത്തു. അതുമവള്‍ തൊട്ടില്ല. അങ്ങിനെയവള്‍ കാറ്റിലോ, കാട്ടുതീയിലോ മഴയിലോ കൂസലൊന്നും കൂടാതെ ഏഴായിരം കൊല്ലം തപസ്സുചെയ്തു. അവളിലെ പപപങ്കിലമായ എല്ലാ വാസനകളും ഇല്ലാതായി. കാര്‍ക്കടിയുടെ സമഗ്രജീവന്‍ നിര്‍മ്മലമായി ഭവിച്ചു. അവളില്‍ പരമജ്ഞാനോദയമുണ്ടായി. അവളിലെ തപോര്‍ജ്ജം ഹിമാലയപര്‍വ്വതത്തെ ജാജ്വല്യമാനമാക്കി. ദേവരാജാവായ ഇന്ദ്രന്‍ നാരദമുനിയില്‍നിന്നും കാര്‍ക്കടിയുടെ അനിതരസാധാരണമായ ഈ തപസ്സിനെപ്പറ്റി അറിഞ്ഞു.

