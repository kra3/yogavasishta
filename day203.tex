\section{ദിവസം 203}

\slokam{
യഥാ രജോഭിർഗഗനം യഥാ കമലമംബുഭി:\\
ന ലിപ്യതേ ഹി സംശ്ലിഷ്ടൈർദേഹൈരാത്മാ തഥൈവ ച (5/5/31)\\
}

വസിഷ്ഠൻ തുടർന്നു: ദേഹവും ആത്മാവും തമ്മിലുള്ള ബന്ധത്തെപ്പറ്റി നിന്നിലുള്ള ചിന്താക്കുഴപ്പത്തെ മാറ്റിയാൽപ്പിന്നെ ശാന്തിയായി. ചെളിക്കുണ്ടിൽ വീണാലും സ്വർണ്ണക്കട്ടയ്ക്ക് കളങ്കമേൽക്കുന്നില്ലല്ലോ. അതുപോലെ ആത്മാവിനെ മലീമസമാക്കുവാൻ ശരീരത്തിനാവുകയില്ല. കൈകളുയർത്തി ഞാനിതാ ആവർത്തിച്ചു പ്രസ്താവിക്കുന്നു: ‘താമരയും വെള്ളവും പോലെ ആത്മാവും ശരീരവും തികച്ചും വിഭിന്നങ്ങളാണ്‌..’ എന്നാൽ കഷ്ഠം, ആരും ഇതു ചെവിക്കൊള്ളുന്നില്ല. ജഢമായ ഈ മനസ്സ് സുഖാനുഭവങ്ങളുടെ പിറകേ പോകുന്നിടത്തോളം കാലം ലോകമെന്ന ഈ മായാപ്രപഞ്ചം ഇല്ലാതാവുകയില്ല. എന്നാൽ ഇതിൽനിന്നുണർന്നാലുടൻ, അത്മാന്വേഷണമാരംഭിച്ചാൽത്തന്നെ ഈ മായയുടെ മുഖംമൂടി ഇല്ലാതാവും. അതിനാൽ എല്ലാവരും ശരീരത്തിൽ സ്ഥിതിചെയ്യുന്ന ഈ മനസ്സിനെ ഉണർത്താൻ പരിശ്രമിക്കണം. കാരണം അങ്ങിനെയാണ്‌ നാം എല്ലാ പരിണാമപ്രക്രിയകൾക്കുമതീതരായി വർത്തിക്കുക. പരിണാമങ്ങൾ ദു:ഖപൂരിതമാണല്ലോ.

“അന്തരീക്ഷത്തിൽ പൊങ്ങിക്കിടക്കുന്ന പൊടിപടലങ്ങൾ ആകാശത്തിനെ ബാധിക്കാത്തതുപോലെ ശരീരം ആത്മാവിനെ ബാധിക്കുന്നില്ല.” സുഖദു:ഖങ്ങൾ ആത്മാവിനാണനുഭവവേദ്യമാകുന്നതെന്ന തെറ്റായ ചിന്ത, ആകാശത്തിനെ പൊടിപടലങ്ങൾ മലീമസമാക്കുന്നു എന്നു പറയും പോലെ അസംബന്ധമാണ്‌.. വാസ്തവത്തിൽ സുഖദു:ഖങ്ങൾ ശരീരത്തിന്റേതോ ആത്മാവിന്റേതോ അല്ല. അവ വെറും അജ്ഞാനത്തിന്റേതാണ്‌.. അവയുടെ ലാഭനഷ്ടങ്ങൾ ആരേയും ബാധിക്കുന്നില്ല. അവയ്ക്കാരുമായും സംബന്ധവുമില്ല.

എല്ലാമെല്ലാം ആത്മാവുമാത്രം. പരമപ്രശാന്തം, അനന്തം. രാമാ, ഈ സത്യം സാക്ഷാത്കരിക്കൂ. ആത്മാവും ലോകവും ഒന്നല്ല; അവ രണ്ടും വെവ്വേറെയാണെന്നും പറയുക വയ്യ. ഇതെല്ലാം സത്യാന്വേഷണചിന്തകൾ മാത്രം. അദ്വിതീയമായ ബ്രഹ്മം മാത്രമേ ഉണ്മയായുള്ളു. ‘ഞാൻ ഇതിൽ നിന്നു വിഭിന്നമാണ്‌’ എന്ന തോന്നൽ വെറും ഭ്രമം. രാമാ ഇതുപേക്ഷിക്കൂ. ഒരേയൊരാത്മാവ്, ആത്മാവിൽത്തന്നെ ആത്മാവിനെ അനന്താത്മാവായി അറിയുന്നു. അതിനാൽ ദു:ഖം, മോഹം, ജനനം, സൃഷ്ടി, ജീവികള്‍ എന്നുവേണ്ട ഒന്നും വാസ്തവത്തില്‍ ഇല്ല. ഇള്ളതോ സത്യവസ്തു മാത്രം. രാമാ, ദു:ഖമൊഴിവാക്കിയാലും. ദ്വന്ദത ഒഴിവാക്കിയാലും; സ്വന്തം ക്ഷേമത്തിൽ പോലും ആശങ്കയില്ലാതെ ആത്മാവിൽ ദൃഢചിത്തനാവൂ. സ്വയം ശാന്തിയെ പ്രാപിക്കൂ. മനസ്സിനെ സുദൃഢമാക്കൂ.

നിന്റെ മനസ്സിൽ ദു:ഖത്തിനിടമില്ല. ആന്തരീകമായുള്ള നിശ്ശബ്ദതയിലഭിരമിക്കൂ. ഏകാന്തനായിരുന്ന് സ്വയമിനിയും ചിന്താധാരണകള്‍  പുതുതായൊന്നുമുണ്ടാക്കാതിരിക്കൂ. ധീരതയോടെ മനസ്സിനേയും ഇന്ദ്രിയങ്ങളേയും അടക്കൂ. ആഗ്രഹലേശമില്ലാതെ, കിട്ടുന്നതുകൊണ്ട് സംതൃപ്തിയടയൂ. യാതൊന്നും കയ്യടക്കാതെയും യാതൊന്നും വിട്ടുകൊടുക്കാതെയും ആയാസരഹിതമായി ജീവിക്കൂ. മനോവൈകല്യങ്ങളെ ഒഴിവാക്കി മോഹാന്ധകാരത്തെ ഇല്ലാതാക്കൂ. ആത്മാഭിരാമനാവൂ. അങ്ങിനെ ദു:ഖനിവൃത്തനാവൂ. നിറസമുദ്രമെന്നപോലെ സർവ്വവ്യാപിത്വമാർന്ന് ആത്മാവിൽ അഭിരമിക്കൂ. ചാന്ദ്രശീതളിമയിലെന്നപോലെ ആത്മാവിനാൽ ആത്മാനന്ദമറിഞ്ഞ് ജീവിക്കൂ. 

