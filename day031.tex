\newpage
\section{ദിവസം 031}

\slokam{
വിചാരാജ്ജ്ഞായതേ തത്ത്വം തത്ത്വാദ്വിശ്രാന്തിരാത്മനി\\
അതോ മനസി ശാന്തത്വം സർവ ദു:ഖ പരീക്ഷയ: (2/14/53)\\
}

വസിഷ്ഠന്‍ തുടര്‍ന്നു:: രണ്ടാമത്തെ കാവല്‍ക്കാരനായ 'ആത്മാന്വേഷണം' നടത്തേണ്ടത്‌ വേദശാസ്ത്രങ്ങളിലെ അറിവിന്റെ വെളിച്ചത്തില്‍ ശുദ്ധമാക്കിയ ബുദ്ധികൊണ്ടാണ്‌. ഈ അന്വേഷണം ഇടമുറിയാതെ തുടരുന്നതുമൂലം ബുദ്ധി അതിസൂക്ഷ്മനിശിതമാവുകയും പരമസാക്ഷാത്കാരത്തിനു പാകപ്പെടുകയും ചെയ്യും. അതിനാല്‍ സംസാരമെന്ന ഈ മാറാരോഗത്തിനുള്ള പ്രതിവിധി ആത്മാന്വേഷണം തന്നെയാണ്‌. ജ്ഞാനിയെ സംബന്ധിച്ചിടത്തോളം പ്രാബല്യം, ബുദ്ധിസാമര്‍ത്ഥ്യം, കാര്യക്ഷമത, സമയോചിത പ്രവര്‍ത്തനങ്ങള്‍ എന്നിവ ഈ അന്വേഷണത്തിന്റെ ഫലപ്രാപ്തികളാണ്‌. രാജപദവി, ഐശ്വര്യം, സുഖഭോഗങ്ങള്‍, പിന്നെ അവസാനം മുക്തിപദപ്രാപ്തി, എല്ലാം ആത്മാന്വേഷണഫലങ്ങളത്രേ. ചിന്താരഹിതനായ വിഡ്ഢിയെ ബാധിക്കുന്നതരം ദുരിതങ്ങളില്‍ നിന്ന് സാധകന്‍ വിമുക്തനാണ്‌. അന്വേഷണത്വരിതമല്ലാത്തതുകൊണ്ട്‌ മന്ദമായ മനസ്സിന്‌ നിലാവിന്റെ ശീതളിമപോലും ഭയാനകമായ ആയുധങ്ങളായിത്തീരുന്നു. എല്ലാ ഇരുണ്ടമൂലകളിലും ബാലിശമായ ഭാവനകളാല്‍ അവന്‍ ഭൂതപിശാചുക്കളെ കാണുന്നു. അതുകൊണ്ട്‌ ആത്മാന്വേഷണമില്ലാത്തവന്‍ തീര്‍ച്ചയായും ദുരിതങ്ങളുടെ കലവറ തന്നെയാണ്‌. ആത്മാനുസന്ധാനമില്ലാത്ത ഒരുവന്‍ ചെയ്യുന്ന പ്രവൃത്തികള്‍ തനിക്കും മറ്റുള്ളവര്‍ക്കും ദോഷകരമായി പലവിധ മാനസീകവ്യാഥികള്‍ ക്കും ഇടവരുത്തുന്നു. ആയതിനാല്‍ അത്തരക്കാരുമായുള്ള സഹവാസം ഒഴിവാക്കണം.

ആരിലാണോ ആത്മാനുസന്ധാനപ്രവണത ഉണര്‍ന്നിരിക്കുന്നത്‌ അവന്‍ ലോകത്തെ ഭാസുരമാക്കുന്നു; താനുമായി ബന്ധപ്പെടുന്നവരെയെല്ലാം പ്രബുദ്ധരാക്കുന്നു; അജ്ഞാനാവരണം മൂലം മനസ്സുണ്ടാക്കുന്ന ഭൂതപിശാചുക്കളെ ഉന്മൂലനം ചെയ്യുന്നു; ഇന്ദ്രിയസുഖഭോഗങ്ങളുടേയും വസ്തുക്കളുടേയും വ്യര്‍ത്ഥതയെപ്പറ്റി ബോധ്യപ്പെടുത്തുന്നു. രാമ: ശാശ്വതവും മാറ്റങ്ങളില്ലാത്തതുമായ സത്യത്തെക്കുറിച്ച്‌ ആത്മാന്വേഷണത്തിന്റെ വെളിച്ചത്തില്‍ ഒരുവനില്‍ അറിവുണ്ടാകുന്നു. ഇതു പരമവിജ്ഞാനമാണ്‌. അവന്‌ മോഹവും ആസക്തിയുമില്ല. അവന്‍ കര്‍മ്മവിമുഖനോ കര്‍മ്മത്തില്‍ മുഴുകിയവനോ അല്ല.അവന്‍ ഈ ലോകത്തില്‍ സാധാരണക്കാരെപ്പോലെ ജീവിച്ച്‌ ആയുസ്സൊടുങ്ങി മരിക്കുന്നു. പക്ഷേ അവന്‍ ആനന്ദപൂര്‍ണ്ണമായ അവസ്ഥയിലാണ്‌ എത്തിച്ചേരുന്നത്‌. തിരക്കുപിടിച്ച പ്രവര്‍ത്തനങ്ങള്‍ക്കിടയിലും ആത്മാന്വേഷണദൃഷ്ടി ഉപേക്ഷിക്കപ്പെടുന്നില്ല. 

ഈ 'കാഴ്ച്ച'യില്ലാത്തവന്റെ കാര്യം കഷ്ടമെന്നേ പറയാവൂ. ചെളിവെള്ളത്തില്‍ കളിക്കുന്ന തവളയുടെ ജന്മവും, ചാണകത്തില്‍ പുളയുന്ന പുഴുവിന്റെ ജീവിതവും, മാളത്തിലിഴയുന്ന പാമ്പിന്റെ പൊറുതിയും ആത്മാന്വേഷണദൃഷ്ടിയില്ലാത്ത മനുഷ്യന്റെ ജീവിതത്തേക്കാള്‍ മഹത്തരമത്രേ. 

എന്താണീ അന്വേഷണം? "ഞാന്‍ ആരാണ്‌" എന്ന നിരന്തരമായ അന്വേഷണമാണിത്‌. "എങ്ങിനെയാണ്‌ ഈ സംസാരമെന്ന ദുരിതം ഉണ്ടായത്‌?". 

'അത്തരം അന്വേഷണത്തില്‍നിന്നും സത്യം തെളിഞ്ഞു കിട്ടും. ആ അറിവിന്റെ നിറവിലാണ്‌ ഒരുവനില്‍ മന:ശ്ശാന്തി ഉണ്ടാവുന്നത്‌. അങ്ങിനെ, ദുരിതങ്ങള്‍ക്കറുതിവന്ന് എല്ലാ അറിവുകള്‍ക്കുമതീതമായ പരമശാന്തി പ്രശോഭിക്കുന്നു.'

(അന്വേഷണം എന്നാല്‍ കാരണം കണ്ടെത്തുന്നതിനുള്ള അവലോകനമൊന്നുമല്ല. ഒരുവസ്തുവിനെ സമഗ്രവും സൂക്ഷ്മവുമായി നേരേ കാണുക എന്ന പ്രക്രിയയാണത്‌.) 
