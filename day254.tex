\section{ദിവസം 254}

\slokam{
സ്ഥാതവ്യമിഹ ദേഹേന കല്‍പം യാവദനേന തേ\\
വയം ഹി നിയതിം വിദ്മോ യഥാഭൂതാമനിന്ദിതാം (5/39/24)\\
}

വസിഷ്ഠന്‍ തുടര്‍ന്നു: അങ്ങിനെ തീരുമാനിച്ച് വിഷ്ണുഭഗവാന്‍ പാതാളലോകത്തെത്തി. വിഷ്ണുപ്രഭയില്‍ അസുരന്മാര്‍ക്കുണര്‍വ്വുണ്ടായി. എന്നാലാ പ്രഭയുടെ തിളക്കം താങ്ങാന്‍ കഴിയാതെ അവര്‍ ഓടിപ്പോയി. പ്രഹ്ലാദന്റെ അടുക്കല്‍പ്പോയി ഭഗവാന്‍ ഇങ്ങിനെ അലറി വിളിച്ചു: ‘മഹാത്മാവേ ഉണരൂ’, ഒപ്പം ആ തിരുശംഖനാദവും മുഴങ്ങി. ദേവന്മാരിതുകേട്ടു സന്തോഷിച്ചുവെങ്കിലും അസുരന്മാര്‍ അവിടെ കുഴഞ്ഞുവീണു. പ്രഹ്ലാദന്റെ മകുടത്തില്‍ ജീവശക്തി പ്രവഹിക്കാന്‍ തുടങ്ങി. ആ പ്രാണന്‍ പിന്നെ ദേഹം മുഴുവന്‍ വ്യാപരിച്ചു. ഇന്ദ്രിയങ്ങളില്‍ ഊര്‍ജ്ജം നിറഞ്ഞു. അവ തങ്ങളുടെ കര്‍മ്മങ്ങള്‍ ചെയ്യാന്‍ പര്യാപ്തമായി. അതതു വിഷയങ്ങളെ അവ തിരിച്ചറിയാന്‍ തുടങ്ങി. മനസ്സ് പ്രവര്‍ത്തിക്കാന്‍ തുടങ്ങി. നാഡികള്‍ കമ്പനം ചെയ്തു തുടങ്ങി. മനസ്സ് തന്റെ ദേഹരൂപകാലങ്ങളെപ്പറ്റി അറിഞ്ഞു. ഭൌതീകശരീരത്തിന്റെ അസ്തിത്വം വെളിവായി. ചുറ്റുപാടുകളിലേയ്ക്ക് പൂര്‍ണ്ണമായും ഉണര്‍ന്ന പ്രഹ്ലാദന്‍ ഭഗവാനെ കണ്ടു.  

വിഷ്ണുഭഗവാന്‍ പ്രഹ്ലാദനോടു പറഞ്ഞു: പാതാളലോകത്തെ ചക്രവര്‍ത്തിയെന്നുള്ള നിന്റെ വ്യക്തിത്വത്തിലേയ്ക്കുണര്‍ന്നാലും. നിനക്ക് ഇനിയൊന്നും നേടാനോ നഷ്ടപ്പെടാനോ ഇല്ല. എഴുന്നേല്‍ക്കൂ. “ഈ ലോകചക്രം കഴിയുംവരെ നീ ഈ ദേഹത്തില്‍ത്തന്നെ കഴിയണം. ഇതനിവാര്യമാണെന്നു ഞാനറിയുന്നു. കാരണം, എനിയ്ക്കീ ലോകത്തിന്റെ ക്രമമെന്തെന്നു നന്നായറിയാം. അതുകൊണ്ട് നീ എല്ലാവിധ ഭ്രമകല്‍പ്പനകളില്‍ നിന്നും മുക്തികൈവന്ന മാമുനിയെന്നപോലെ ഇവിടം ഭരിക്കുക. വിശ്വപ്രളയത്തിനിനിയും സമയമായിട്ടില്ല. പിന്നെയീ ശരീരത്തെ ഉപേക്ഷിക്കാന്‍ നീ ധൃതി കാണിക്കുന്നതെന്തിനാണ്? ഞാന്‍ നിലനില്‍ക്കുന്നു. ഈ ലോകവും അതിലെ ജീവികളും നിലകൊള്ളുന്നു. അതുകൊണ്ട് ഈ ദേഹമുപേക്ഷിക്കാനുള്ള ചിന്ത ഇപ്പോള്‍ വേണ്ട.

മരിക്കാന്‍ പാകമായവന്‍ അജ്ഞാനത്തിലും ദു:ഖത്തിലും മുങ്ങുന്നു. ‘ഞാന്‍ ദുര്‍ബ്ബലന്‍, ഞാനൊരു മന്ദന്‍, അല്ലെങ്കില്‍ അവശന്‍’ എന്ന് ചിന്തിക്കുന്നവനും മരിക്കാന്‍ അനുയോജ്യനാണ്. എണ്ണിയാലൊടുങ്ങാത്ത ആശകളും പ്രത്യാശകളും കൊണ്ട് മനസ്സ് ചഞ്ചലപ്പെട്ടവനും മരിക്കാന്‍ യോഗ്യന്‍ തന്നെ. സുഖ-ദു:ഖങ്ങളെന്ന ദ്വന്ദഭാവങ്ങള്‍ക്കടിമയായവര്‍ , ദേഹത്തോട് ആസക്തിയുള്ളവര്‍ , മാനസീകമായും ശാരീരികമായും വ്യാധിക്കടിമപ്പെട്ടവര്‍ , കാമക്രോധങ്ങളുടെ തീയില്‍പ്പെട്ട് ഹൃദയം വരണ്ടുണങ്ങിയവര്‍ എല്ലാം മരണയോഗ്യരാണ്. 

ഒരുവന്‍ ശരീരമുപേക്ഷിക്കുന്നതിനെ ആളുകള്‍ മരണമെന്ന് കരുതുന്നു. എന്നാല്‍ ആത്മജ്ഞാനത്താല്‍ മനോനിയന്ത്രണം വരുത്തി സത്യബോധത്തോടെ കഴിയുന്നവര്‍ക്ക് മാത്രമേ ജീവിതം പറഞ്ഞിട്ടുള്ളു.  അഹംഭാവം വെച്ചുപുലര്‍ത്താത്ത, ഒന്നിനോടും ആസക്തിയില്ലാത്ത ഇഷ്ടാനിഷ്ടവിവേചനമില്ലാത്ത, മന:ശാന്തി കൈവന്നവരാണ് (അല്ലെങ്കില്‍ മനോനിഗ്രഹം പ്രാപിച്ചവരാണ്) ജീവിക്കേണ്ടത്. സത്യത്തില്‍ അറിവോടെ അഭിരമിച്ച് ഇഹലോകത്ത് ലീലയായി ജീവിച്ച്, ബാഹ്യസംഭവങ്ങളില്‍ അമിതാഹ്ലാദമോ ദു:ഖമോ ഇല്ലാതെ, യാതൊന്നും നേടാനും ഉപേക്ഷിക്കാനുമില്ലാതെ കഴിയുന്നവര്‍ ജീവിക്കുകതന്നെ വേണം. അവരുടെ ചരിതം തന്നെ മറ്റുള്ളവര്‍ക്ക് ആനന്ദപ്രദമാണ്. അങ്ങിനെയുള്ളവരുടെ ജീവിതമാണ് അഭികാമ്യം. മരണം അവര്‍ക്ക് ചേരില്ല.