\section{ദിവസം 243}

\slokam{
ഘൃതം യഥാന്തഃ പയസോ രസശക്തിര്‍യഥാ ജലേ\\
ചിച്ഛക്തി: സര്‍വഭാവേഷു തഥാന്തരഹമാസ്ഥിത:  (5/34/56)\\
}

പ്രഹ്ലാദന്‍ തന്റെ ധ്യാനം തുടര്‍ന്നു: ജീവികളുടെ എല്ലാം അനുഭവങ്ങളിലൂടെ കടന്നുപോകുന്ന ആത്മാവ് ഒന്നാണ്. അതാണ്‌ എല്ലാം അനുഭവിക്കുന്നത്. അതിനാല്‍ ആത്മാവിന് ആയിരം കൈകളും കണ്ണുകളുമുണ്ടെന്നു പറയപ്പെടുന്നു. സൂര്യന്റെ സൌഭഗശരീരമായും വായുവായും മറ്റെല്ലാമായും ആത്മാവ് ‘ഞാന്‍’ ആയി ആകാശം മുഴുവന്‍ വ്യാപരിക്കുന്നു. ശംഖ-ചക്ര-ഗദാ പങ്കജ ധാരിയായ ദേവതയായി (വിഷ്ണു) ഈ ജഗത്തിനെ അലങ്കരിക്കുന്നതും ആത്മാവത്രേ. ഇതേ ആത്മാവാണ് താമരയില്‍ ഇരുന്നു സൃഷ്ടി ചെയ്യുന്ന ബ്രഹ്മദേവന്‍.. ഈ ലോകചക്രത്തിന്റെ അന്ത്യത്തില്‍ എല്ലാറ്റിനെയും വിലയനം ചെയ്യിക്കുന്ന സംഹാരദേവതയും (മഹേശ്വരന്‍)  ആത്മാവ്‌ തന്നെ.
  
ഇന്ദ്രനാല്‍ മൂര്‍ത്തീകരിക്കപ്പെട്ട, ‘ഞാന്‍’ എന്നറിയപ്പെടുന്ന ആത്മാവാണ് ലോകത്തെ സംരക്ഷിക്കുന്നത്. ഞാന്‍ ആണാണ്. പെണ്ണാണ്. യുവാവാണ്. ജരാനര ബാധിച്ച വൃദ്ധനാണ്. ദേഹമെടുത്തതുകൊണ്ട് ‘ഞാനി’വിടെ ജനിച്ചവനായി തോന്നുകയാണ്. ഞാന്‍ സര്‍വവ്യാപിയാണ്‌.. അനന്താവബോധത്തിന്റെ ഭൂമിയില്‍ ഞാന്‍ മരങ്ങളായും ചെടികളായും അവയിലെ അന്ത:സത്തയായും നിലകൊള്ളുന്നു. കൊച്ചുകുട്ടിയുടെ കയ്യിലെ കളിമണ്ണുപോലെ എന്റെതന്നെ പ്രാഭവത്താല്‍ പ്രത്യക്ഷലോകം മുഴുവന്‍ ഞാന്‍ നിറഞ്ഞിരിക്കുന്നു. ഈ ലോകത്തിന്റെ ഉണ്മ എന്നില്‍, ആത്മാവില്‍,  നിക്ഷിപ്തമാണ്. അതെന്നില്‍ എന്നിലൂടെ വര്‍ത്തിക്കുന്നു. എന്നാല്‍ ഞാനതിനെ നിരാകരിക്കുമ്പോള്‍ അല്ലെങ്കില്‍ ഉപേക്ഷിക്കുമ്പോള്‍ അതിനു മൂല്യമില്ലാതാവുന്നു. ലോകത്തിനുണ്ടെന്നു അതുവരെ തോന്നിയിരുന്ന ആപേക്ഷികമായ ഉണ്മ ഇല്ലാതെയാവുന്നു.
  
കണ്ണാടിയില്‍ കാണുന്ന പ്രതിഫലനമെന്നപോലെ ഈ ലോകം എന്നില്‍ , ആത്മാവില്‍ , അനന്താവബോധത്തില്‍ , നിലകൊള്ളുന്നു. പൂക്കളിലെ സൌരഭം ഞാനാണ്. പ്രകാശത്തിലെ ഭാസുരത ഞാനാണ്. അവകളിലെ അനുഭവവേദ്യതയും ഞാനാണ്. സര്‍വ്വചരാചരങ്ങളുടേയും പരമസത്ത ഞാനാണ്. എല്ലാ ഉപാധികളുമൊഴിഞ്ഞ ബോധമാണ് ഞാന്‍.. വിശ്വപ്രപഞ്ചത്തിലെ എല്ലാറ്റിന്റെയും അന്ത:സത്ത ഞാനാണ്.

“പാലില്‍ വെണ്ണപോലെ, ജലത്തിന്റെ ദ്രാവകാവസ്ഥപോലെ, ബോധത്തിലെ ചൈതന്യം പോലെ അസ്തിത്വമുള്ള എല്ലാറ്റിന്റെയും ഉണ്മ ഞാനാണ്.”
  
ഭൂത-ഭാവി-വര്‍ത്തമാന അവസ്ഥകളില്‍ ഉള്ള ഈ ലോകം അനന്താവബോധത്തില്‍ നിലകൊള്ളുന്നത് യാതൊരുവിധ വിഷയഭിന്നതകളും കൂടാതെയാണ്. സര്‍വ്വവ്യാപിയും സര്‍വ്വശക്തനുമായ വിശ്വപുരുഷനാണ് ആത്മാവ്, അല്ലെങ്കില്‍ ഞാന്‍ എന്നറിയപ്പെടുന്നത്. ഈ വിശ്വം യദൃഛയാ എന്നില്‍ വന്നു നിറഞ്ഞു. എന്നാല്‍ത്തന്നെ  പരക്കെ വ്യാപരിക്കപ്പെട്ടു. ഞാന്‍ ആത്മാവായി, പരമാവബോധമായി, പ്രളയശേഷവും  വിശ്വം മുഴുവന്‍ പ്രപഞ്ചസമുദ്രമെന്നതുപോലെ നിറഞ്ഞിരിക്കുന്നു. വികലാംഗങ്ങളോടു കൂടിയ ജലജീവികള്‍പോലും സമുദ്രത്തിന്റെ അനന്തതയെ അറിയുന്നതുപോലെ ഞാനും എന്റെ സഹജ സ്വരൂപമായ  അപാരതയെ, അളക്കാന്‍ കഴിയാത്ത അനന്തതയായി മാത്രം അറിയുന്നു.

അനന്താവബോധത്തില്‍ ഈ ലോകമെന്നത് വെറുമൊരു ധൂളീകണം മാത്രം. അതെന്നെ സംതൃപ്തനാക്കാന്‍ പോന്നതല്ല. ആനയുടെ വിശപ്പടക്കാന്‍ ചെറിയൊരു പഴത്തിനെങ്ങിനെ സാധിക്കും? ബ്രഹ്മാവിന്റെ ഗൃഹത്തില്‍ നിന്നും വികാസം തുടങ്ങിയ നാമരൂപങ്ങള്‍ ഇപ്പോഴും അനവരതം ഉണ്ടായി വളര്‍ന്നു വികാസം പ്രാപിച്ചുകൊണ്ടേയിരിക്കുന്നു.

