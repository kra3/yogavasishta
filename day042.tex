 
\section{ദിവസം 042}

\slokam{
യോ ജാഗ്രതി സുഷുപ്തസ്തോ യസ്യ ജാഗ്രൻ നവിദ്യതേ\\
യസ്യ നിർവാസനോ ബോധ: സ ജീവന്മുക്ത: ഉച്യതേ  (3/9/7)\\
}

രാമന്‍ ചോദിച്ചു: ഏതു മാര്‍ഗ്ഗത്തിലൂടെയാണ്‌ ഈ അറിവുനേടുക? എന്നില്‍ ഈ 'അറിയപ്പെടുന്നവ' (കാണപ്പെടുന്നവ) അസ്തമിക്കണമെങ്കില്‍ എന്തറിവാണു ഞാന്‍ നേടേണ്ടത്‌?

വസിഷ്ഠന്‍ പറഞ്ഞു: നിരന്തരം തെറ്റായ വഴിയില്‍ ചിന്തിച്ചുവരുന്നതുകൊണ്ടാണ്‌ ഈ ലോകം  യാഥാർത്ഥ്യമാണെനുള്ള തെറ്റിദ്ധാരണ നമ്മില്‍ രൂഢമൂലമായിരിക്കുന്നത്‌. എന്നാല്‍ മഹാത്മാക്കളുടെ സത്സംഗവും വേദഗ്രന്ഥങ്ങളുടെ പഠനവും എന്നുതുടങ്ങുന്നുവോ അന്ന് ഈ ചിന്തയെ മാറ്റാം. വേദഗ്രന്ഥങ്ങളില്‍ ഉത്തമം "മഹാരാമായണം" എന്നറിയപ്പെടുന്ന യോഗവാസിഷ്ഠം എന്ന ഈ കൃതിയാണ്‌. ഇതിലുള്ളത്‌ മറ്റുപലയിടത്തും കണ്ടെന്നുവരും എന്നാല്‍ ഇതില്‍ ഇല്ലാത്തത്‌ മറ്റൊരിടത്തും കണ്ടുകിട്ടുകയില്ല. ഈ കൃതി പഠിക്കാന്‍ താത്പ്പര്യമില്ലാത്തവര്‍ക്ക്‌ മറ്റുകൃതികളെ ആശ്രയിക്കാവുന്നതാണ്‌. നമുക്കതില്‍ യാതൊരാക്ഷേപവുമില്ല.

തെറ്റിദ്ധാരണ തീര്‍ത്തും നീങ്ങി സത്യം സാക്ഷാത്കരിച്ച്‌, ആ നിറവില്‍ സ്വയം ആണ്ടു മുങ്ങിയ ഒരുവന്‍ ചിന്തിക്കുന്നതും, പറയുന്നതും ഉല്ലസിക്കുന്നതും മറ്റുള്ളവരെ പഠിപ്പിക്കുന്നതും എല്ലാം അതു തന്നെയായിരിക്കും. അവരെ ജീവന്മുക്തരെന്നും ചിലപ്പോള്‍ വിദേഹ മുക്തരെന്നും വിളിക്കുന്നു.

രാമന്‍ ചോദിച്ചു: മഹാത്മന്‍ , ജീവന്മുക്തരുടെ (ജീവിക്കുമ്പോള്‍ ത്തന്നെ മുക്തരായവര്‍ ) ലക്ഷണങ്ങള്‍ എന്തൊക്കെയാണ്‌? വിദേഹമുക്തരുടെ (ശരീരമില്ലാത്ത മുക്തര്‍ )  ലക്ഷണങ്ങള്‍ എന്തൊക്കെയാണ്‌?

വസിഷ്ഠന്‍ മറുപടി പറഞ്ഞു: സാധാരണ ഗതിയില്‍ ജീവിതം നയിക്കുമ്പോള്‍ത്തന്നെ വിശ്വത്തെ മുഴുവന്‍ അഖണ്ഡമായ ഒരു ശൂന്യതയായി അനുഭവപ്പെടുന്നവനത്രേ ജീവന്മുക്തന്‍ . അവന്‍ ഉണര്‍ന്നിരിക്കുന്നുവെങ്കിലും ദീര്‍ഘനിദ്രയുടെ പ്രശാന്തത അനുഭവിക്കുന്നു. സുഖദു:ഖങ്ങള്‍ അവനെ അലട്ടുന്നതേയില്ല. "അവന്‍ ദീര്‍ഘനിദ്രയിലും ഉണര്‍ന്നിരിക്കുന്നു. എന്നാല്‍ അവന്റെ ഉണര്‍ച്ച ലോകത്തിലേയ്ക്കല്ല. അവന്റെ വിജ്ഞാനത്തെ ലീനവാസനകളുടെ മേഘം മൂടിമറയ്ക്കുന്നില്ല."

അവന്‍ ഇഷ്ടാനിഷ്ടങ്ങളുടേയും, ഭയത്തിന്റേയും വരുതിയിലാണെന്നപോലെ കാണപ്പെട്ടേക്കാം എന്നാല്‍ വാസ്തവത്തില്‍ അവന്‍ ആകാശം പോലെ സര്‍വ്വസ്വതന്ത്രനണ്‌. അവന്‌ അഹങ്കാരമോ മനോവൃത്തികളോ ഇല്ല. കര്‍മ്മത്തിലോ അകര്‍മ്മത്തിലോ അവന്‍ ബന്ധിതനുമല്ല. അവനെ ആര്‍ക്കും ഭയമില്ല. അവന്‍ ആരേയും ഭയക്കുന്നുമില്ല. 

അവന്‍ കാലക്രമത്തില്‍ സ്വന്തം ശരീരമുപേക്ഷിക്കുന്നതോടെ വിദേഹമുക്തനാവുന്നു. വിദേഹമുക്തന്‍ ഉണ്ടെന്നും ഇല്ലെന്നും പറയാം. അത്‌  'ഞാന്‍ ' അല്ല; ഞാന്‍ അല്ലാത്തതും അല്ല. അവന്‍ പ്രദീപ്തമായ സൂര്യനാണ്‌; സംരക്ഷകനായ വിഷ്ണുവാണ്‌; സംഹാരകനായ രുദ്രനാണ്‌; സൃഷ്ടാവായ ബ്രഹ്മാവാണ്‌. ആകാശവും, ഭൂമിയും, വായുവും, ജലവും അഗ്നിയുമാണ്‌. അവന്‍ എല്ലാ ജീവനിര്‍ജ്ജീവജാലങ്ങളുടേയും അന്ത:സത്തയായ വിശ്വാവബോധം തന്നെയാണ്‌. ഭൂത, ഭാവി, വര്‍ത്തമാനകാലങ്ങളില്‍ നിലനില്‍ക്കുന്ന എല്ലാം തീര്‍ച്ചയായും അവന്‍ മാത്രമാണ്‌.

