 
\section{ദിവസം 132}

\slokam{
യത്തു ചഞ്ചലതാഹീനം തന്‍മനോ മൃതമുച്യതേ\\
തദേവ ച തപ:ശാസ്ത്രസിദ്ധാന്തോ മോക്ഷ ഉച്യതെ (3/112/8)\\
}

വസിഷ്ഠന്‍ തുടര്‍ന്നു: മനസ്സ്‌ എതേതു പദാര്‍ത്ഥങ്ങളിലേയ്ക്കാണോ തീക്ഷ്ണമായി ശ്രദ്ധതിരിച്ച് ചലിക്കുന്നത്‌, അവിടെ ആര്‍ത്തികളെ അടക്കാനുള്ള വഴി മനസ്സ്‌ സ്വയം കണ്ടെത്തുന്നു. എന്നാല്‍  ഈ ചലനത്തിന്റെ ദിശ തികച്ചും വ്യക്തമല്ല. സമുദ്രോപരിയുള്ള ഓളങ്ങളെപ്പോലെ അവ ചിലപ്പോള്‍ ചിലയിടത്ത്‌ അപ്പപ്പോള്‍ കാണപ്പെട്ടു മറയുകയാണ്‌.. മഞ്ഞുകട്ടയില്‍ തണുപ്പ്‌ സഹജമെന്നതുപോലെ മനസ്സും ഈ ചഞ്ചലത്വവും  തമ്മില്‍ വേര്‍പിരിക്കാനാവില്ല.

രാമന്‍ ചോദിച്ചു: മഹാത്മന്‍, മനസ്സിന്റെ ഈ അസ്വസ്ഥതയെ ബലമായി നിയന്ത്രിക്കാന്‍ ശ്രമിക്കുന്നത്‌ കൂടുതല്‍ വിക്ഷോഭങ്ങള്‍ക്കിടയാക്കുകയില്ലേ?

വസിഷ്ഠന്‍ പറഞ്ഞു: ശരിയാണ്‌, അസ്വസ്ഥതകളില്ലെങ്കില്‍ മനസ്സില്ല. മനസ്സിന്റെ സഹജഭാവം തന്നെ അസ്വസ്ഥതയാണ്‌.. അനന്താവബോധത്തെ അടിസ്ഥാനമാക്കി മനസ്സിലുള്ള അസ്വസ്ഥതയാണ്‌ ഈ ലോകമായി പ്രകടമാവുന്നത്‌.. രാമ: അതാണ്‌ മനസ്സിന്റെ ശക്തി. "എന്നാല്‍ മനസ്സില്‍ നിന്ന് ഈ അസ്വസ്ഥതയെ നീക്കം ചെയ്താല്‍ അതു 'മരിച്ച' മനസ്സാണ്‌.. അതാണ്‌ തപസ്സ്‌.. വേദശാസ്ത്രങ്ങളിലൂടെ വെളിവാകുന്ന മുക്തിയും അതാണ്‌.". മനസ്സങ്ങിനെ അനന്താവബോധത്തില്‍ വിലയിക്കുമ്പോള്‍ പരമ ശാന്തിയും മനസ്സ്‌ ചിന്തകളില്‍ മുഴുകുമ്പോള്‍ ദു:ഖവുമാണ്‌ ഫലം.

മനസ്സിന്റെ ഈ അസ്വസ്ഥതയാണ്‌ അവിദ്യ, അജ്ഞാനം, എന്നൊക്കെ അറിയപ്പെടുന്നത്‌.. അതാണ്‌ വാസനകളുടെ ഇരിപ്പിടം. നിരന്തരമായ അന്വേഷണത്തിലൂടെ, സുഖദായികളായ വിഷയവസ്തുക്കളോടുള്ള ആസക്തി ഉപേക്ഷിച്ച്‌, അതിനെ ഇല്ലാതാക്കുക. രാമ: മനസ്സ്‌ ഒരു നാഴികമണിയുടെ നാക്കുപോലെ സത്തില്‍ നിന്നും അസത്തിലേയ്ക്ക്‌, ബോധത്തില്‍നിന്നും ജഢത്തിലേയ്ക്ക്‌, ചഞ്ചാടിക്കൊണ്ടേയിരിക്കുന്നു. ജഢവസ്തുവില്‍ ഏറെക്കാലം ശ്രദ്ധപതിപ്പിച്ചിരുന്നാല്‍ മനസ്സ്‌ ആ ജഢസ്വഭാവം സാത്മീകരിക്കുന്നു. അതേ മനസ്സ്‌ ആത്മാന്വേഷണത്തിന്റേയും ജ്ഞാനത്തിന്റേയും മാര്‍ഗ്ഗത്തിലേയ്ക്ക്‌ ഉന്മുഖമാവുമ്പോള്‍ എല്ലാ സങ്കല്‍പ്പങ്ങളും കുടഞ്ഞെറിഞ്ഞ്‌ നിര്‍മ്മലബോധമെന്ന സ്വരൂപത്തിലേയ്ക്ക്‌ തിരിച്ചെത്തുന്നു.

സ്വാഭാവികമായുള്ളതോ സങ്കല്‍പ്പിച്ചുണ്ടാക്കിയതോ ആയ വസ്തുവിന്റെ രൂപം മനസ്സ്‌ ആര്‍ജ്ജിക്കുന്നു. അതുകൊണ്ട്‌ ദൃഢനിശ്ചയത്തോടെ, ബുദ്ധിപൂര്‍വ്വം എല്ലാ ദു:ഖങ്ങള്‍ക്കുമതീതമായ, സംശയരഹിതമായ ആ അവസ്ഥയെ ധ്യാനിക്കൂ. മനസ്സിന്‌ സ്വയം നിയന്ത്രിക്കാന്‍ അറിയാം മറ്റൊരു മാര്‍ഗ്ഗം ഇല്ല താനും. ജ്ഞാനികള്‍ അവരുടെ ലീനവാസനകളെ (അതു മനസ്സു തന്നെ) അവ ഉയരുന്ന മാത്രയില്‍ത്തന്നെ നീക്കംചെയ്ത്‌ അവിദ്യയെ ജയിക്കുന്നു. ആദ്യം ആര്‍ത്തികളെ ത്യജിച്ച്‌ മനോപാധികളെ നശിപ്പിക്കുക. പിന്നീട്‌ ബന്ധനം, മോചനം തുടങ്ങിയ ധാരണകളെപ്പോലും മനസ്സില്‍നിന്നു നീക്കുക. അങ്ങിനെ എല്ലാ ഉപാധികളില്‍നിന്നും പൂര്‍ണ്ണമായി സ്വതന്ത്രനാവുക. 

