\section{ദിവസം 283}

\slokam{
ന നിര്‍ഘൃണോ ദയാവാന്നോ ന ദ്വന്ദീ നാഥ മത്സരീ\\
ന സുധീര്‍നാസുധീര്‍നാര്‍ധീ നാനര്‍ധീ സ ബഭൂവ  ഹ (5/60/6)\\
}

വസിഷ്ഠന്‍ തുടര്‍ന്നു: അങ്ങിനെയുള്ള അന്വേഷണസപര്യയില്‍ സുരാഗു പരമമായ ബോധതലത്തെ പ്രാപിച്ചു. അതിനുശേഷം അദ്ദേഹത്തിനു യാതൊരുവിധ ദു:ഖങ്ങളും ഉണ്ടായില്ല. എന്നാല്‍ തന്റെ ജോലികള്‍ മനസ്സില്‍ നിറഞ്ഞ സമതാദര്‍ശനത്തോടെ ഭംഗിയായി നിര്‍വ്വഹിക്കുകയും ചെയ്തു. “അദ്ദേഹം ദയവാനായിരുന്നു; എന്നാല്‍ വെറുക്കപ്പെടേണ്ടതിനെ പ്രോത്സാഹിപ്പിച്ചില്ല. എതിര്‍ചേരികളിലുള്ള ദ്വന്ദതകളെ അവഗണിക്കാതെ എന്നാല്‍ അവയുടെ സ്വാധീനത്തിലാവാതെ, ബുദ്ധിഹീനതയോ ബുദ്ധിവൈഭവപ്രകടനമോ കൂടാതെ, പ്രേരണാശക്തികളാല്‍ ബാധിക്കപ്പെടാതെ, എന്നാല്‍ പ്രചോദനത്തെ നിരാകരിക്കാതെ അദ്ദേഹം അന്തര്‍പ്രശാന്തയില്‍ അഭിരമിച്ചുകൊണ്ട് സമ്യക് ദര്‍ശനത്തോടെ തന്റെ കര്‍മ്മങ്ങളെ ഭംഗിയായി നിര്‍വഹിച്ചു.     

‘ഇതെല്ലാം അനന്താവബോധത്തിന്റെ വൈവിദ്ധ്യമാര്‍ന്ന പ്രത്യക്ഷരൂപങ്ങള്‍ മാത്രമാണെന്ന്’ അദ്ദേഹം അറിഞ്ഞു. പൂര്‍ണ്ണമായ അറിവുറച്ചതിനാല്‍ സന്തോഷദു:ഖങ്ങളില്‍ അദ്ദേഹം ഒരുപോലെ പ്രശാന്തനായിരുന്നു. അദ്ദേഹം കുറേക്കാലം ലോകം ഭരിച്ചിട്ട് സ്വാഭീഷ്ട പ്രകാരം ദേഹത്യാഗം ചെയ്ത് അനന്തതയില്‍ വിലയം പ്രാപിച്ചു. രാമാ, അദ്ദേഹത്തെപ്പോലെ  പ്രബുദ്ധതയുടെ നിറവില്‍ ലോകത്തെ ഭരിച്ചു ജീവിക്കൂ.

രാമന്‍ ചോദിച്ചു: ഭഗവന്‍, മനസ്സ്‌ ചഞ്ചലമാണല്ലോ. അപ്പോള്‍ ഒരുവന് എങ്ങിനെയാണ് പൂര്‍ണ്ണമായ അക്ഷോഭ്യതയും സമതാഭാവവും കൈവരിക?

വസിഷ്ഠന്‍ തുടര്‍ന്നു: ഇതേ വിഷയത്തെപറ്റി ഈ സുരാഗു രാജാവും പരിഘമുനിയും തമ്മിലൊരു ചര്‍ച്ച നടക്കുകയുണ്ടായി. കേട്ടാലും: സുരാഗുവിന്റെ സുഹൃത്തായി പേര്‍ഷ്യയില്‍ പരിഘന്‍ എന്ന് പേരുള്ള ഒരു രാജാവുണ്ടായിരുന്നു. ഒരിക്കലദ്ദേഹത്തിന്റെ രാജ്യത്ത് ഭയങ്കരമായ ക്ഷാമം ഉണ്ടായി. എന്തെല്ലാം ചെയ്തിട്ടും തന്റെ പ്രജകളുടെ കഷ്ടപ്പാടില്‍ കുറവൊന്നുമില്ലെന്ന് കണ്ടു മനംനൊന്ത രാജാവ് എല്ലാത്തില്‍നിന്നുമോടിയോളിക്കാന്‍, ആരും കാണാതെ  കാട്ടില്‍പ്പോയി തപസ്സാരംഭിച്ചു.

അവിടെ വെറും ഇലകള്‍ മാത്രം ആഹരിച്ച് തപസ്സു ചെയ്കയാല്‍ അദ്ദേഹത്തിനു ‘പര്‍ണാദന്‍ ’ എന്ന് പേര് കിട്ടി. ആയിരം കൊല്ലമങ്ങിനെ തപസ്സുചെയ്ത് അദ്ദേഹം ആത്മജ്ഞാനം ആര്‍ജ്ജിച്ചു. അതുകഴിഞ്ഞ് അദ്ദേഹം മൂന്നു ലോകങ്ങളും സ്വതന്ത്രമായി സഞ്ചരിച്ചു വരുമ്പോള്‍ ഒരു ദിവസം തന്റെ സുഹൃത്തായ സുരാഗുവിനെ കണ്ടുമുട്ടി. പ്രബുദ്ധരായ അവരിരുവരും പരസ്പരം വന്ദിച്ചു.

പരിഘന്‍ സുരാഗുവിനോടു ചോദിച്ചു: അങ്ങ് മാണ്ഡവ്യമുനിയുടെ സഹായത്താല്‍ ആത്മജ്ഞാനം പ്രാപിച്ചതുപോലെ ഞാനും തപസ്സനുഷ്‌ഠിച്ച് ഭഗവല്‍കൃപയാല്‍ ആത്മജ്ഞാനം കൈവരിച്ചു. എന്നാല്‍ ഞാന്‍ ഒന്ന് ചോദിക്കട്ടെ, അങ്ങയുടെ മനസ്സിപ്പോള്‍ പരിപൂര്‍ണ്ണമായും പ്രശാന്തമാണോ? അങ്ങയുടെ പ്രജകള്‍ക്ക് സൌഖ്യമാണോ? അവര്‍ ഐശ്വര്യസമ്പത്തുകള്‍ അനുഭവിക്കുന്നുണ്ടോ? അങ്ങ് നിര്‍മമതയില്‍ പൂര്‍ണ്ണമായും അഭിരമിക്കുന്നുണ്ടോ? 

സുരാഗു പറഞ്ഞു: ദിവ്യേച്ഛ എന്തെന്നാര്‍ക്ക് പറയാനാവും? അങ്ങും ഞാനും ഏറെക്കാലം അകന്നുകഴിഞ്ഞു. എന്നാലിപ്പോള്‍ കണ്ടുമുട്ടിയിരിക്കുന്നു. ദൈവത്തിന് അസാദ്ധ്യമായെന്തുണ്ട്? അങ്ങയുടെ സന്ദര്‍ശനം കൊണ്ട് ഞങ്ങള്‍ അതീവധന്യരായിരിക്കുന്നു. അങ്ങയുടെ സാന്നിദ്ധ്യം തന്നെ പാപഹരമത്രേ. മാത്രമല്ല, എല്ലാ ഐശ്വര്യങ്ങളും അങ്ങയുടെ രൂപത്തില്‍ ഞങ്ങള്‍ക്ക് മുന്നില്‍ വന്നുനില്‍ക്കുന്നതായി തോന്നുന്നു. നന്മയുടെ വിളനിലമായ സത് പുരുഷന്മാരുമായുള്ള സത്സംഗം തീര്‍ച്ചയായും മുക്തിപദത്തിനു സമമത്രേ.