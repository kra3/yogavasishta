\newpage
\section{ദിവസം 110}

\slokam{
യഥൈവ കര്‍മ്മകരണേ കാമനാ നാസ്തി ധീമതാം\\
തഥൈവ കര്‍മ്മസംത്യാഗേ കാമനാ നാസ്തി ധീമതാം (3/88/12)\\
}

സൂര്യന്‍ തുടര്‍ന്നു: പ്രഭോ,  സൃഷ്ടിവാഞ്ഛയുമായി ആ പത്തുപേര്‍ ധ്യാനനിരതരായി ഇരുന്നു. അവരുടെ ശരീരം ക്ഷയിച്ചു നശിച്ചു. ബാക്കിയായ ഭാഗങ്ങള്‍ വന്യമൃഗങ്ങള്‍ക്ക്‌ ആഹാരമായി. അവര്‍ ദേഹമില്ലാത്ത അവസ്ഥയിലും അങ്ങിനെതന്നെ ധ്യാനമഗ്നരായി നിലകൊണ്ടു.  യുഗാവസാനം വരെ  ആ നില്‍പ്പ്‌ നീണ്ടു നിന്നു. യുഗാന്ത്യത്തില്‍ പൊള്ളിക്കുന്ന സൂര്യതാപവും ഇടിവെട്ടലും എല്ലാറ്റിനേയും നശിപ്പിച്ചു. ഈ മഹാത്മാക്കളാകട്ടെ അവരുടെ ധ്യാനസപര്യ തുടര്‍ന്നു. അവരുടെ ആഗ്രഹം വിശ്വസൃഷ്ടാക്കളാകണം എന്നതായിരുന്നുവല്ലോ. പുതിയ യുഗാരംഭത്തിലും ഈ പത്തുമഹാത്മാക്കള്‍ അതേ സ്ഥലത്ത്‌ തന്നെ നിലകൊണ്ടു. അവരങ്ങിനെ സൃഷ്ടികര്‍ത്താക്കളായി. ആ പത്തുപേരുണ്ടാക്കിയ പ്രപഞ്ചങ്ങളാണ്‌ അവിടുന്ന് ദര്‍ശിച്ചത്‌. പ്രഭോ, അവരുണ്ടാക്കിയ വിശ്വങ്ങളിലൊന്നിലെ സൂര്യനാണു ഞാന്‍...

ബ്രഹ്മാവ്‌ സൂര്യനോടു ചോദിച്ചു: ഹേ സൂര്യാ, ഈ പത്തുപേര്‍ വിശ്വനിര്‍മ്മിതിയെല്ലാം ചെയ്തുകഴിഞ്ഞിരിക്കുന്നു. ഇനി എനിക്ക്‌ എന്താണു ചെയ്യാനുള്ളത്‌? എന്റെ കര്‍ത്തവ്യം എന്താണ്‌?

സൂര്യന്‍ പറഞ്ഞു: ഭഗവാനേ അവിടുത്തേയ്ക്ക്‌ ആഗ്രഹങ്ങളൊന്നും ഇല്ലല്ലോ. അങ്ങയെ പ്രചോദിപ്പിക്കാന്‍  അങ്ങില്‍ വാസനകള്‍ ഒന്നുമില്ലതാനും . അതായത്‌ അങ്ങേയ്ക്ക്‌ ചെയ്യേണ്ടതായി ഒരു കര്‍മ്മവുമില്ല. വിശ്വനിര്‍മ്മിതികൊണ്ട്‌ അങ്ങേയ്ക്കെന്തു പ്രയോജനം? വിശ്വനിര്‍മ്മിതിയെന്ന സര്‍ഗ്ഗപ്രവൃത്തി അങ്ങേയ്ക്ക്‌ കാരണമേതുമില്ലാത്ത വെറുമൊരു നേരമ്പോക്കും ലീലയുമാണ്‌.. ഭഗവാനേ അങ്ങില്‍ നിന്നും സൃഷ്ടികളുണ്ടാവുന്നത്‌ അന്തര്‍പ്രേരണകൊണ്ടോ ആഗ്രഹംകൊണ്ടോ അല്ല. സൂര്യപ്രകാശം ജലത്തില്‍ പ്രതിഫലിക്കുന്നതും അതിനെ പ്രകാശമാനമാക്കുന്നതും സൂര്യന്റെ ഉദ്ദേശലക്ഷ്യമല്ല. ജലവും എതെങ്കിലും ഉദ്ദേശലക്ഷ്യങ്ങളോടെയല്ല പ്രതിഫലനം ഏറ്റെടുക്കുന്നത്‌.. ഹേതുരഹിതമായി സൂര്യന്‍ എങ്ങിനെ ദിനരാത്രങ്ങള്‍ക്കു കാരണമാവുന്നുവോ, അപ്രകാരം അങ്ങും ഇഛാരഹിതനായി സൃഷ്ടിയിലേര്‍പ്പെട്ടാലും. അങ്ങയുടെ സ്വഭാവവത്തേയും ധര്‍മ്മത്തേയും വെടിഞ്ഞിട്ട്‌ എന്തു നേട്ടമാണങ്ങേയ്ക്കുണ്ടാവുക?

"ജ്ഞാനികള്‍ യാതൊരു പ്രവര്‍ത്തങ്ങള്‍ക്കായും ആഗ്രഹിക്കുന്നില്ല; അവര്‍ കര്‍മ്മങ്ങളെ ത്യജിക്കാനും ആഗ്രഹിക്കുന്നില്ല." അങ്ങു മനക്കണ്ണില്‍ കാണുന്നത്‌ ആ മഹാത്മാക്കള്‍ സൃഷ്ടിച്ച ലോകങ്ങളാണ്‌.. മനക്കണ്ണില്‍ ആദ്യം ദര്‍ശിച്ച (സൃഷ്ടിച്ച) വസ്തുക്കള്‍ മാത്രമേ ഭൌതീകമായി ഒരുവന്‍ പിന്നീടു കാണുന്നുള്ളു. മനസ്സില്‍ക്കണ്ട വസ്തുക്കളാകട്ടെ നശിപ്പിക്കാനരുതാത്തതാണ്‌.. മൂലപദാര്‍ത്ഥങ്ങളുടെ സംഘാതമായ വസ്തുക്കള്‍ പരിണാമത്തിനും നാശത്തിനും വിധേയമാണ്‌.. ഒരാളുടെ വ്യക്തിത്വം, അവന്റെ മനസ്സില്‍ സ്വത്വത്തെപ്പറ്റി ദൃഢമായി സ്ഥിരീകരിച്ച സത്യബോധമല്ലതെ മറ്റൊന്നുമല്ല..

