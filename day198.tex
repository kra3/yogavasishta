\section{ദിവസം 198}

\slokam{
യൈവ ചിദ് ഭുവനാഭോഗേ ഭൂഷണേ വ്യോമ്നി ഭാസ്കരേ\\
ധരാവിവരകോശസ്ഥേ  സൈവ ചിത്കീടകോദരേ (4/61/18)\\
}

വസിഷ്ഠൻ തുടർന്നു: പ്രത്യക്ഷമായ ഈ ലോകം സൃഷ്ടിക്കപ്പെട്ടുകഴിഞ്ഞപ്പോൾ അതൊരു പാത്രത്തിൽ നിറച്ച വെള്ളം പോലെയായി. അതിൽ അനേകം ജീവികൾ പൊങ്ങിയും താണും വന്നുകൊണ്ടിരുന്നു. അവകളെയെല്ലാം ബന്ധിപ്പിച്ചിരുന്നത് ജീവത്വരയാകുന്ന കയറാണ്‌.. അനന്താവബോധമെന്ന സമുദ്രത്തിൽ അലകളും ഓളങ്ങളും പോലെയുണ്ടായ ഈ ജീവജാലങ്ങൾ ഭൗതീകമണ്ഡലത്തിൽ പ്രവേശിച്ചു. വായു, ജലം, ഭൂമി തുടങ്ങിയ മൂലഘടകങ്ങൾ ഉണ്ടായപ്പോൾ അവയുമായി ഈ ജീവജാലങ്ങൾ ബന്ധപ്പെട്ടു. അപ്പോൾ ജനന മരണ ചക്രമുരുളാൻ ആരംഭിച്ചു. ജീവജാലങ്ങൾ ചന്ദ്രരംശികളിൽ കരേറി വൃക്ഷലതാദികളിൽ പ്രവേശിക്കുന്നതുപോലെയായിരുന്നു. അവ അതതു ചെടികളിൽ കായായിത്തീർന്നു. സൂര്യപ്രകാശത്തിലും ചൂടിലും കായ്കൾ പഴുത്തുപാകം വന്നു ഫലങ്ങളായി. അപ്പോൾ അവ പുനർജനിക്കാൻ പക്വമായി.

സൂക്ഷ്മരൂപത്തിൽ ആശയങ്ങളും ധാരണകളും ഇനിയും ജനിച്ചിട്ടില്ലാത്ത ജീവനുകളിൽപ്പോലും നിർലീനമത്രേ. ജനനസമയത്ത് അവയെ പൊതിഞ്ഞിരിക്കുന്ന ആവരണങ്ങൾ നീങ്ങിപ്പോകുന്നു. ചിലർ ജന്മനാ സാത്വികരും ശുദ്ധമനസ്കരുമാണ്‌.. അവർ തങ്ങളുടെ പൂർവ്വ ജന്മത്തിൽ ഇന്ദ്രിയസുഖങ്ങളുടെ ആകർഷണത്തെ വിജയകരമായി ചെറുത്തുനിന്നവരാണ്‌.. എന്നാൽ മറ്റുള്ളവർ ജനിക്കുന്നത് വീണ്ടും വീണ്ടും ജനിച്ചുമരിച്ച് ഈ ചാക്രികചലനം തുടരുന്നതിനുമാത്രമാണ്‌.. അവരിൽ ശുദ്ധാശുദ്ധങ്ങളും ആന്ധ്യവും ഇടകലർന്നിരിക്കുന്നു. ഇനിയും ചിലർ നിർമ്മലരെങ്കിലും അവരിൽ കളങ്കത്തിന്റെ ചെറിയൊരംശം ബാക്കി നിൽക്കുന്നുണ്ട്. അവർ സത്യാർത്ഥികളും സദ് വൃത്തരുമാണ്‌.. അജ്ഞാനത്തിന്റെ പിടിയിൽപ്പെടാതെ ജീവിക്കുന്ന ഇത്തരക്കാർ തുലോം വിരളമത്രേ. കല്ലും മലയും പോലെ ഉറച്ചുപോയ അജ്ഞാനത്തിന്റെ മൂർത്തിമദ്ഭാവങ്ങളായ, മന്ദതയും തമോഗുണവും നിറഞ്ഞ വേറേ ചിലരുമുണ്ട്.

അൽപ്പം കളങ്കമവശേഷിച്ചിട്ടുണ്ടെങ്കിലും കൂടുതൽ സാത്വീകഭാവമുള്ള സാത്വീക-രാജസീക സ്വഭാവമുള്ളവർ പൊതുവേ സന്തുഷ്ടരും പ്രബുദ്ധരുമാണ്. അവർ പെട്ടെന്നു ദു;ഖത്തിനടിമപ്പെട്ടു വ്യാകുലചിത്തരാകുന്നില്ല. അവർ നിസ്വാർത്ഥരായ വൃക്ഷങ്ങൾപോലെയാണ്‌.. സ്വാർജ്ജിത പൂർവ്വകർമ്മങ്ങളുടെ ഫലമനുഭവിക്കുകയാണവർ. അവർ പുതുതായി ആർജ്ജിതകർമ്മങ്ങളിൽ ഏർപ്പെടുന്നില്ല. അവർ ആഗ്രഹങ്ങൾക്കതീതരാണ്‌.. അവർ സ്വയം പ്രശാന്തരാണ്‌.. എത്ര ദുരിതമുണ്ടായാലും അവർ തങ്ങൾ കണ്ടെത്തിയ പ്രശാന്തതയെ കൈവെടിയുന്നില്ല. അവർ എല്ലാവരേയും സമതാ ഭാവത്തിൽ സ്നേഹിക്കുന്നു. അവർ ശോകസമുദ്രത്തിൽ മുങ്ങിത്താഴാനിടയാകുന്നില്ല. ഒരു കാരണവശാലും സംസാരമെന്ന ഈ ശോകസമുദ്രത്തിൽ മുങ്ങാതിരിക്കാനാണ്‌ എല്ലാവരും ശ്രമിക്കേണ്ടത്. ആത്മസ്വരൂപാന്വേഷണത്തിലൂടെ ഇതു സാദ്ധ്യമാണ്‌. “ഞാൻ ആരാണ്‌? ഈ മായാ ലോകം എങ്ങിനെയാണുണ്ടായത്?” എന്നിങ്ങനെ മനസ്സ് നിരന്തരം ധ്യാനനിരതമായിരിക്കണം.

അങ്ങിനെ ഒരുവൻ ദേഹാഭിമാനവും ലോകാസക്തിയും ഉപേക്ഷിക്കണം. ഇട്യ്ക്കൊരു മന്ദിരം നിലകൊള്ളുന്നു എന്നതുകൊണ്ട് ആകാശത്തിന്‌ വിഭജനങ്ങളോ അതിരുകളോ ഉണ്ടാകുന്നില്ലെന്ന് അപ്പോൾ അറിവുറയ്ക്കും. “ഭാസുരദീപ്തിയുള്ള സൂര്യനിൽ കുടികൊള്ളുന്ന അതേ അവബോധമാണ്‌ ഭൂമിയിലെ ഒരു ചെറുകുഴിയിൽ ഞുളയ്ക്കുന്ന പുഴുവിലും ഉള്ളത്”. 

