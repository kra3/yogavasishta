\newpage
\section{ദിവസം 010}

\begin{center}
ഭീഷയത്യപി ധീരം മാമന്ധയത്യപി സേക്ഷണം\\
ഖേദയത്യപി സാനന്ദം തൃഷ്ണാ കൃഷ്ണേവ ശർവരീ (1/17/16)\\ 
\end{center}

രാമന്‍ തുടര്‍ ന്നു: മനസ്സ്‌ ആര്‍ത്തിയോടെ ആഗ്രഹങ്ങളാല്‍ മൂടുമ്പോള്‍ അജ്ഞാനാന്ധകാരത്തില്‍ എണ്ണമില്ലാത്ത തെറ്റുകള്‍ സംഭവിക്കുന്നു. ഈ അത്യാഗ്രഹങ്ങള്‍ ഹൃദയത്തിലെ നന്മയും മഹത്വവും ഇല്ലായ്മചെയ്യുന്നു. എന്റെ മാധുര്യഭാവവും മാന്യതയ്ക്കും പകരം കഠിനതയും ക്രൂരതയും പകടമാവുന്നു. ഈ അന്ധകാരത്തില്‍ ആശകള്‍ പല രൂപത്തിലാടുന്ന പിശാചുക്കളായി മാറുന്നു. പലവിധ മാര്‍ഗ്ഗങ്ങളിലൂടെ ഞാന്‍ ഇവയെ നിയന്ത്രിച്ചാലും അവ എന്നെ നിമിഷനേരംകൊണ്ടു കീഴടക്കി കാറ്റിലാടുന്ന പുല്‍ക്കൊടിപോലെ നിസ്സഹായനാക്കുന്നു. ഞാന്‍ നിര്‍മമത തുടങ്ങിയ ഉത്തമഗുണങ്ങള്‍ എന്നിലുണ്ടാക്കാനാലോചിച്ചു തുടങ്ങുമ്പോഴേയ്ക്ക്‌ ഒരെലി ചരടുമുറിക്കുമ്പോലെ ആ പ്രത്യാശയും നഷ്ടപ്രായമാവുന്നു. ഈ ആശാചക്രത്തില്‍ ഞാന്‍ നിസ്സഹായനായി ഉഴറുകയാണ്‌. ചിറകുകള്‍ ഉണ്ടെങ്കിലും വലയില്‍ കുടുങ്ങിയ പക്ഷികള്‍ക്കു പറക്കാന്‍ കഴിയാത്തപോലെ നമുക്കും നമ്മുടെ ലക്ഷ്യത്തിലേയ്ക്ക്, ആത്മജ്ഞാനത്തിലേയ്ക്ക്‌ ഉയരാന്‍ പറ്റുന്നില്ല. അമൃതു കിട്ടിയാല്‍ പോലും ഈ ആഗ്രഹങ്ങള്‍ അടങ്ങുകയില്ല. ആശകള്‍ക്ക്‌ നിയതമായ ദിശകള്‍ ഒന്നുമില്ല. മതിഭ്രമം വന്ന കുതിരയേപ്പോലെ ഒരുനിമിഷം ഒരു ദിശയിലേക്കാണെങ്കില്‍ അടുത്ത നിമിഷം മറ്റൊരു ദിശയിലേയ്ക്ക്‌ അതു നമ്മെ കൊണ്ടുപോകുന്നു. നമ്മുടെ മുന്നില്‍ ഭാര്യ, മക്കള്‍ സുഹൃത്ത്‌ മറ്റു ബന്ധുക്കള്‍ എന്നിങ്ങനെ വിപുലമായ ശൃംഖലകളോടെ ഒരു വലവിരിച്ചിട്ടുണ്ടല്ലോ.

ഞാനൊരു വീരയോദ്ധാവാണെങ്കിലും ആശകള്‍ എന്നെ ഭീരുവാക്കുന്നു.
എനിക്കു കണ്ണുകളുണ്ടെങ്കിലും അതെന്നെ അന്ധനാക്കുന്നു.
എന്റെ സ്വത്വം ആനന്ദം നിറഞ്ഞെതെങ്കിലും ആശകള്‍ എന്നെ ദു:ഖിതനാക്കുന്നു.
അവ പിശാചുക്കളെപ്പോലെ ഭീകരങ്ങള്‍. 

നമ്മുടെ നിര്‍ഭാഗ്യങ്ങള്‍ക്കും ബന്ധനത്തിനും കാരണം ഈ ആശാപിശാചാണ്‌. മനുഷ്യന്റെ ഹൃദയം തകര്‍ത്ത്‌ അവനില്‍ മതിഭ്രമം നിറയ്ക്കുന്ന സത്വം. ഈ പിശാചിന്റെ ബാധയേറ്റ്‌ അവനവന്റെ വരുതിയിലുള്ള സുഖം പോലും മനുഷ്യന്‌ നിഷേധിക്കപ്പെടുന്നു. അഗ്രഹങ്ങള്‍ സുഖസമ്പാദനത്തിനാണെന്നു തോന്നുമെങ്കിലും അവ നമ്മെ സുഖത്തിലേയ്ക്കു നയിക്കുന്നില്ല. മറിച്ച്‌ വൃഥാ പ്രവര്‍ത്തനങ്ങളിലേയ്ക്കും അമംഗളകാര്യങ്ങളിലേയ്ക്കുമാണ്‌ നയിക്കുന്നത്‌. ജീവിതനാടകത്തിലെ പല സുഖ ദു:ഖ രംഗങ്ങളിലും, ഈ ആശാപിശാച്‌ വയസ്സുചെന്ന ഒരഭിനേത്രിയേപ്പോലെ നന്മയോ മഹത്വമോ പ്രകടിപ്പിക്കാനാവാതെ എല്ലായിടത്തും പരാജയം ഏറ്റുവാങ്ങുന്നു. എങ്കിലും അവള്‍ വേദിയിലെ നൃത്തം ഉപേക്ഷിക്കാന്‍ തയ്യാറല്ല! ശാന്തിയില്ലാതെ, ആശകള്‍ ആകാശം മുട്ടെ ഉയരത്തിലും അതീവ ആഴത്തിലും സഞ്ചരിക്കുന്നു. മനസ്സിന്റെ ശൂന്യതയെ ആശ്രയിച്ചിരിക്കുന്നു ആശകള്‍. ചിലസമയം വിജ്ഞാനത്തിന്റെ പ്രഭ നിമിഷനേരത്തേയ്ക്ക്‌ മിന്നിയാലുടന്‍ മോഹത്തിന്റെ ഇരുട്ട്‌ പടരുകയായി. മാമുനിമാര്‍ക്ക്‌ ആത്മജ്ഞാനം കൊണ്ട്‌ ഈ ആശാപാശത്തെ അറുത്തുകളയുവാന്‍ സാധിക്കുന്നു എന്നത്‌ എത്ര അത്ഭുതം!
