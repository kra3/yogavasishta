\newpage
\section{ദിവസം 033}

\slokam{
ദീപേ യഥാ വിനിദ്രസ്യ ജ്വലിതേ സം പ്രവർത്തതേ\\
അലോകോനിച്ഛതോപ്യേവം നിർവാണമനയാഭവേത് (2/17/7)\\
}

വസിഷ്ഠന്‍ പറഞ്ഞു: ഞാന്‍ വെളിപ്പെടുത്തുവാന്‍ പോകുന്ന വിജ്ഞാനമാര്‍ജ്ജിക്കാന്‍ മുന്‍പേ പറഞ്ഞ യോഗ്യതകള്‍ ഒരുവന്‍ സ്വായത്തമാക്കിയിരിക്കണം. രാമ: താങ്കള്‍ അങ്ങിനെ യോഗ്യനാണെന്നു ഞാനറിയുന്നു. മുക്തിപദപ്രാപ്തിക്കു പാകം വന്നവനേ ഇതു ശ്രവിക്കാന്‍ വെമ്പല്‍ കൊള്ളുകയുള്ളു താനും. "എന്നാല്‍ ഒരുവന്‌ ആഗ്രഹമില്ലെങ്കില്‍ കൂടി ഈ വിദ്യ അവനെ മുക്തിയിലേയ്ക്കു നയിക്കും. ഉറങ്ങിക്കിടക്കുന്നവന്റെ കണ്ണുകളേയും പ്രകാശമാനമാക്കാന്‍ വെളിച്ചത്തിനു കഴിയുമല്ലോ". കയറില്‍ പാമ്പിനെ ആരോപിച്ചു കണ്ടതിന്റെ ഭയം കയറിന്റെ സത്യാവസ്ഥ തിരിച്ചറിയുന്നതോടെ ഇല്ലാതാവുന്നു. അങ്ങിനെ ഭയജന്യമായിരുന്ന പാമ്പ്‌ അപ്രത്യക്ഷമാവുന്നതുപോലെ സംസാരദുരിതത്തില്‍ നിന്നും ഈ പഠനം കൊണ്ട്‌ മനുഷ്യന്‍ മോചിതനാവുന്നു.

ഈ വേദഗ്രന്ഥത്തില്‍ 32000 ശ്ളോകങ്ങളുണ്ട്‌. ആദ്യഭാഗത്തിന്‌  "വൈരാഗ്യപ്രകരണം" എന്നു പേര്‍. അനാസക്തിയെ പ്രതിപാദിക്കുന്ന ഈ ഭാഗം ഒരുവനില്‍ ഇഹലോകത്തിന്റെ നിജസ്ഥിതിയെന്തെന്ന ബോധമുണ്ടാക്കുന്നു. ഇതിന്റെ ശ്രദ്ധാപൂര്‍വ്വമായ പഠനം ഹൃദയത്തെ നിര്‍മ്മലമാക്കുന്നു. ഇതില്‍ 1500 ശ്ളോകങ്ങളാണുള്ളത്‌. "മുമുക്ഷു വ്യവഹാര പ്രകരണം" എന്ന രണ്ടാം ഭാഗം സാധകന്റെ സ്വഭാവത്തെയും യോഗ്യതകളേയും പ്രതിപാദിക്കുന്നു. ഇത്‌ 1000 ശ്ളോകങ്ങളാണ്‌. പിന്നീട്‌ വരുന്നത്‌ 7000 ശ്ളോകങ്ങളുള്ള "ഉത്പത്തിപ്രകരണം" ആണ്‌. യഥാര്‍ത്ഥത്തില്‍ സൃഷ്ടിക്കപ്പെട്ടിട്ടില്ലാത്ത ഈ വിശ്വത്തില്‍ ഞാന്‍ , അത്‌, ഇത്‌, എന്നിങ്ങനെയുള്ള തെറ്റിദ്ധാരണകളുടെ ലീലാവിനോദങ്ങള്‍ , സത്യത്തെ എങ്ങിനെ  മറച്ച്‌ വിശ്വത്തെ പ്രകടിതമാനമാക്കുന്നു എന്നു പഠിപ്പിക്കാനുതകുന്ന കഥകളാണ്‌ ഈ ഭാഗത്ത്‌ പറയുന്നത്‌.

ഇനി 3000 ശ്ളോകങ്ങളുള്ള "സ്ഥിതിപ്രകരണം"ആണ്‌. കഥകളുടെ സഹായത്താലാണ്‌ ഇവിടേയും വിശ്വസ്ഥിതിയെപ്പറ്റിയും അതിന്റെ അടിസ്ഥാനത്തെപ്പറ്റിയും വിവരിക്കുന്നത്‌. വിരാമത്തെക്കുറിച്ചുള്ള, 5000 ശ്ളോകങ്ങളുള്ള "ഉപശാന്തിപ്രകരണം"ആണ്‌ അടുത്തത്‌. ഇത്‌ ശ്രവിക്കുന്നതുമൂലം നേരിയ അവിദ്യാലേശം മാത്രം ബാക്കിവെച്ച്‌ വിശ്വമെന്ന മോഹപ്രതീതിക്കും അവബോധത്തിനും വിരാമമാവുന്നു.

അവസാനമായി 14,500 ശ്ളോകങ്ങളുള്ള "നിര്‍വ്വാണപ്രകരണം." മുക്തിപദപ്രാപ്തിയാണ്‌ ഇതിലെ പ്രതിപാദ്യം. ഇതിന്റെ പഠനം ഒരുവന്റെ അടിസ്ഥാനപരമായ അവിദ്യയെ നശിപ്പിക്കുന്നതത്രേ. എല്ലാ മോഹവിഭ്രാന്തികളും അടങ്ങി പരമശാന്തി കൈവരുമ്പോള്‍ തികഞ്ഞ സ്വാതന്ത്ര്യമാണ്‌. ശരീരമെന്ന വസ്ത്രമുള്ളപ്പോഴും അവന്‍ അതില്‍ ബന്ധിതനല്ല. അവന്‌ ആസക്തിയോ ആശകളോ ഇല്ല, അടുപ്പവും അകല്‍ ച്ചയുമില്ല. അവന്‍ സംസാരചക്രത്തിനു വിധേയനുമല്ല. ഇവിടെ ഇപ്പോള്‍ അവന്‍ അഹങ്കാരവിനിര്‍മുക്തനാണ്‌. അവനും അനന്തതയും ഏകമായ ഒരേ ഒരു സത്താണ്‌.
