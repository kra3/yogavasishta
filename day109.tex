 
\section{ദിവസം 109}

\slokam{
ഐശ്വര്യാണാം ഹി സര്‍വേഷാമകല്‍പ്പം ന  വിനാശി യത്‌\\
രോചതേ ഭ്രാതരസ്‌തന്‍മേ ബ്രഹ്മത്വമിഹ നേതരത്‌(31) (((3/86/31)\\
}

സൂര്യന്‍ പറഞ്ഞു: ദേവാദിദേവാ, കൈലാസപര്‍വ്വതത്തിനടുത്ത്‌ സുവര്‍ണജാതം എന്നയിടത്ത്‌ അങ്ങയുടെ പുത്രന്മാര്‍ ഉണ്ടാക്കിയ ഒരു ജനപദമുണ്ട്‌... കശ്യപമുനിപരമ്പരയില്‍പ്പെട്ട ഇന്ദു എന്നുപേരായ ഒരു മഹാത്മാവ്‌ അവിടെയുണ്ടായിരുന്നു. സന്താനഭാഗ്യമൊഴിച്ച്‌ എല്ലാ ഐശ്വര്യങ്ങളും അദ്ദേഹത്തിനും ഭാര്യക്കും ഉണ്ടായിരുന്നു. സന്താന ഭാഗ്യലാഭത്തിനായി അവര്‍ കൈലാസത്തില്‍പ്പോയി കഠിനമായ തപസ്സിലേര്‍പ്പെട്ടു. കുറച്ചു ജലം മാത്രമേ അവര്‍ ദിവസേന ആഹരിച്ചിരുന്നുള്ളു. അവര്‍ മരങ്ങളേപ്പോലെ ഒരേയിടത്ത്‌ ചലിക്കാതെ നില്‍പ്പായി. പരമശിവന്‍ അവരില്‍ സംപ്രീതരായി അവര്‍ക്കുമുന്നില്‍ പ്രത്യക്ഷമായി. വരമെന്തുവേണമെന്ന ചോദ്യത്തിനുത്തരമായി അവര്‍  ധര്‍മിഷ്ഠരും ഈശ്വരഭക്തിയുള്ളവരുമായ പത്തുപുത്രന്മാരെ വേണമെന്നാണാവശ്യപ്പെട്ടത്‌. പരമേശ്വരന്‍ അവര്‍ ആവശ്യപ്പെട്ട വരമരുളി. താമസംവിനാ അവര്‍ക്ക്‌ പ്രസരിപ്പുള്ള പത്തു സുപുത്രന്മാര്‍ ഉണ്ടായി. അവര്‍ വളര്‍ന്നു യുവക്കളായി. കേവലം ഏഴു വയസ്സായപ്പോഴേയ്ക്ക്‌ അവര്‍ക്ക്‌ വേദങ്ങള്‍ സ്വായത്തമായിരുന്നു. കുറച്ചുകഴിഞ്ഞ്‌ ആ മാതാപിതാക്കള്‍ ദിവംഗതരായി. അവര്‍ക്ക്‌ മുക്തിപദം ലഭിച്ചു.

ഈ പുത്രന്മാര്‍ക്ക്‌ മാതാപിതാക്കളുടെ ദേഹവിയോഗം ദുസ്സഹമായിരുന്നു. അവര്‍ ഒരുദിവസം ഒത്തുകൂടി ഇങ്ങിനെ ആലോചിച്ചു. 'സഹോദരന്മാരേ, നമുക്ക്‌ ഇനിയിപ്പോള്‍ ഏറ്റവും ശുഭോദര്‍ക്കമായ എന്താണു ലക്ഷ്യമാക്കേണ്ടത്‌? എന്താണു നമുക്ക്‌ നേടാന്‍ ഉചിതമായത്‌? എന്താണ്‌ നമ്മെ ദു:ഖത്തിലേയ്ക്ക്‌ നയിക്കുകില്ല എന്നുറപ്പുള്ള മാര്‍ഗ്ഗം?' രാജപദവി, ചക്രവര്‍ത്തിപദം, ദേവേന്ദ്രപദവി, എല്ലാം വെറും നിസ്സാരം. സ്വര്‍ഗ്ഗം ഭരിക്കുന്ന ഇന്ദ്രപദവി ബ്രഹ്മാവിന്റെ ഒന്നരമണിക്കൂര്‍ നേരത്തേക്കുള്ള ഒരു താല്‍ക്കാലിക തസ്തിക മാത്രമല്ലേ? "അതുകൊണ്ട്‌, സൃഷ്ടാവിന്റെ സ്ഥിതിയെ പ്രാപിക്കുക എന്നതാണ്‌ ഏറ്റവും നല്ലതെന്ന് എനിക്കു തോന്നുന്നു. ഒരു യുഗം മുഴുവന്‍ തീരാതെ സൃഷ്ടികര്‍ത്താവിന്റെ ആയുസ്സൊടുങ്ങുകയില്ലല്ലോ". മറ്റുള്ളവരും ഇതിനെ സര്‍വ്വാത്മനാ അംഗീകരിച്ചു.

അവര്‍ ആത്മഗതമായിപ്പറഞ്ഞു: എന്നാല്‍ നമുക്ക്‌ ബ്രഹ്മപദത്തിലെത്തുക തന്നെ ലക്ഷ്യം. അവിടെ ജരാനരകളോ മരണമോ ഇല്ല. ഏറ്റവും മുതിര്‍ന്ന സഹോദരന്‍ പറഞ്ഞു: എല്ലാവരും ഞാന്‍ പറയുന്നതുപോലെ ചെയ്യുക. ഇനിമുതല്‍ നമുക്ക്‌ ഇങ്ങിനെ ധ്യാനിക്കാം: "ഞാന്‍ വിടര്‍ന്ന താമരയിലിരിക്കുന്ന ബ്രഹ്മാവാണ്‌.". സഹോദരന്മാര്‍ എല്ലാവരും അങ്ങിനെ ചെയ്തു. "ഞാന്‍ വിശ്വസൃഷ്ടാവായ ബ്രഹ്മാവാണ്‌.; മാമുനിമാരും, വിദ്യാദേവതയായ സരസ്വതീദേവിയും എന്നില്‍ അവരുടെ വ്യക്തിപ്രാഭവത്തോടെ കുടികൊള്ളുന്നു. സ്വര്‍ഗ്ഗവും അതിലെ ദേവന്മാരുമെന്നിലാണ്‌.. പര്‍വ്വതങ്ങളും സമുദ്രവും ഭൂഖണ്ഡങ്ങളുമെന്നിലാണ്‌.. ഉപദേവതകളും രാക്ഷസന്മാരുമെന്നിലാണ്‌.. എന്നിലാണ്‌ സൂര്യന്‍ പ്രഭാസിക്കുന്നത്‌.. ഇപ്പോളിതാ സൃഷ്ടി സംഭവിക്കുന്നു. ഇതാ, സൃഷ്ടി, സ്ഥിതിചെയ്യുന്നു. ഇതാ, സൃഷ്ടിക്കു സ്വയമില്ലാതാവാന്‍ സമയമായി. അങ്ങിനെ ഒരു യുഗമിതാകഴിഞ്ഞു. ബ്രഹ്മാവിന്റെ ഒരു രാത്രിയായി. എനിക്ക്‌ ആത്മജ്ഞാനമുണ്ടായിരിക്കുന്നു; ഞാന്‍ മുക്തനായിരിക്കുന്നു." ഇങ്ങിനെ നിരന്തരം ധ്യാനം ചെയ്ത്‌ അവര്‍ ധ്യാനവസ്തുതന്നെ ആയിത്തീര്‍ന്നു. 
