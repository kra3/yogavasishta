\newpage
\section{ദിവസം 063}

\slokam{
ഇഹൈവാംഗുഷ്ടമാത്രാന്തേ തദ്വ്യോമ്ന്യേവ പദം സ്ഥിതം\\
മദ്ഭർതൃരാജ്യം സമവഗതം യോജനകോടിഭാക് (3/29/36)\\
}

രാമന്‍ ചോദിച്ചു: ദിവ്യഗുരോ, ഈ രണ്ടു വനിതകള്‍ ദൂരെ അങ്ങകലത്തുള്ള അകാശഗംഗയിലൂടെ എങ്ങിനെയാണ്‌ സഞ്ചരിച്ചത്‌? എങ്ങിനെയാണവര്‍ അതിനിടയിലുള്ള വിഘ്നങ്ങളെ മറികടന്നത്‌?

വസിഷ്ഠന്‍ പറഞ്ഞു: രാമ: എവിടെയാണ്‌ വിശ്വം? എവിടെയാണ്‌ താരാപഥങ്ങളെല്ലാമുള്ള ആകാശഗംഗ? എവിടെയാണ്‌ തടസ്സങ്ങള്‍ ?. രണ്ടുപേരും രാജ്ഞിയുടെ അന്തപ്പുരത്തില്‍ ത്തന്നെയായിരുന്നു. അവിടെത്തന്നെയാണ്‌ ആ ദിവ്യപുരുഷന്‍ വസിഷ്ഠന്‍ വിദുരഥന്‍ എന്ന രാജാവായി വാണതും. പദ്മന്‍ എന്ന രാജാവായി നേരത്തേ വാണിരുന്നത്‌ ഇതേയാളാണ്‌. ഇതെല്ലാം നടന്നത്‌ ശുദ്ധമായ ആകാശത്തിലാണ്‌. അവിടെ വിശ്വമില്ല, ദൂരങ്ങളില്ല, വിഘ്നങ്ങളുമില്ല. വര്‍ത്തമാനം പറഞ്ഞുകൊണ്ട്‌ ഈ രണ്ടു വനിതകള്‍ മുറിക്കുപുറത്തുവന്ന് കുന്നിന്മുകളിലുള്ള ഒരു ഗ്രാമത്തിലെത്തി. ആ മലയുടെ മഹിമയും സൌന്ദര്യവും വര്‍ണ്ണനാതീതം.! അതിലെ ഓരോ വീട്ടിനുമുകളിലും വന്‍ മരങ്ങളില്‍നിന്നുമുള്ള പുഷ്പവൃഷ്ടി അഭംഗുരം കാണപ്പെട്ടു. മേഘങ്ങള്‍ തുന്നിയ കിടക്കമേല്‍ സുന്ദരതരുണികള്‍ ഉറങ്ങി. ഇടിമിന്നല്‍ വെളിച്ചത്തില്‍ വീടുകള്‍ ദീപാലംകൃതമായി. യോഗശക്തികൊണ്ട്‌ ഭൂത, ഭാവി, വര്‍ത്തമാന കാലങ്ങളെക്കുറിച്ച്‌ ലീലയ്ക്ക്‌ സമ്പൂര്‍ണ്ണമായ അറിവുണ്ടായിരുന്നു. തന്റെ ഭൂതകാലം ഓര്‍മ്മിച്ചുകൊണ്ട്‌ ലീല സരസ്വതീ ദേവിയോടു പറഞ്ഞു: ദേവീ, കുറച്ചുമുന്‍പ്‌ ഞാന്‍ ഇവിടെയൊരു വൃദ്ധയായി ജീവിച്ചിരുന്നു. ഞാന്‍ തികച്ചും ധാര്‍മ്മീകമായ ജീവിതമാണു നയിച്ചിരുന്നതെങ്കിലും സ്വരൂപത്തെപ്പറ്റി, ഞാനാര്‌? ഈ ലോകം എന്താണ്‌ തുടങ്ങിയ അന്വേഷണമൊന്നും ചെയ്തിരുന്നില്ല. എന്റെ ഭര്‍ത്താവും ധര്‍മ്മിഷ്ഠനായിരുന്നുവെങ്കിലും ആത്മജ്ഞാനാഭിവാഞ്ഛ അദ്ദേഹത്തിലും ഉണ്ടായിരുന്നില്ല. പ്രജ്ഞ ഉണര്‍ന്നിരുന്നില്ല. ഞങ്ങള്‍ ഉത്തമജീവിതം നയിച്ചു, മാത്രമല്ല മറ്റുള്ളവരെ അങ്ങിനെ ജീവിക്കാന്‍ പ്രേരിപ്പിക്കുകയും ചെയ്തു. 

ലീല അവളുടെ പഴയ വീട്‌ ദേവിയെ കാണിച്ചുകൊടുത്തു. എന്നിട്ട്‌ തുടര്‍ന്നു: നോക്കൂ ഇതാണെനിക്കേറ്റവും പ്രിയപ്പെട്ട പശുക്കിടാവ്‌. എന്നെ പിരിഞ്ഞിരിക്കുന്നതുകൊണ്ട്‌ പുല്ലുപോലും തിന്നാതെ അവള്‍ കഴിഞ്ഞ എട്ടു ദിവസമായി കണ്ണീരൊഴുക്കുകയാണ്‌. ഇവിടെ എന്റെ ഭര്‍ത്താവാണ്‌ ഭരിച്ചിരുന്നത്‌. തീവ്രമായ ആത്മശക്തിയും വലിയൊരു ചക്രവര്‍ത്തിയാകാനുള്ള അഭിവാഞ്ഛയും നിമിത്തം കാലമേറെക്കഴിഞ്ഞതായി ഞങ്ങള്‍ക്കു തോന്നിയിരുന്നുവെങ്കിലും ഈ ചുരുങ്ങിയ എട്ടു ദിവസംകൊണ്ട്‌ അദ്ദേഹം ആഗ്രഹിച്ചതെല്ലാം സാധിച്ചു. ആകാശത്തില്‍ മന്ദഗമനംചെയ്യുന്ന കാറ്റ്‌ അദൃശ്യനാണെന്നതുപോലെ ഇവിടെയിപ്പോഴും എന്റെ ഭര്‍ത്താവ്‌ അദൃശ്യനായി ജീവിക്കുന്നു. "വിരലോളം പോന്ന ഈ ആകാശദേശത്ത്‌ എന്റെ ഭര്‍ത്താവ്‌ ഭരിക്കുന്ന രാജ്യം അനേകം കോടി ചതുരശ്രമെയിലുകള്‍ വിസ്തീര്‍ണ്ണമുള്ളതാണെന്നു ഞങ്ങള്‍ സങ്കല്‍പ്പിച്ചു." ദേവീ, ഞാനും എന്റെ പ്രിയനുമെല്ലാം ശുദ്ധ അവബോധം മാത്രമാണെന്നറിയുന്നു. എങ്കിലും മായയില്‍ ഭ്രമം മൂലം എന്റെ ഭര്‍ത്താവിന്റെ രാജ്യം നൂറുകണക്കിന്‌ മലകളാല്‍ ചുറ്റപ്പെട്ടതായി കാണപ്പെടുന്നു. അഹോ എന്തത്ഭുതം!. ഈ രാജ്യത്തിന്റെ തലസ്ഥാനത്ത്‌ പോകാന്‍ എനിക്കാഗ്രഹമുണ്ട്‌. നമുക്കങ്ങോട്ടു പോകാം. പരിശ്രമവും ഉത്സാഹമുണ്ടെങ്കില്‍ സാദ്ധ്യമല്ലാത്തതായി എന്തുണ്ട്‌?
