 
\section{ദിവസം 099}

\slokam{
സംരംഭദ്വാരമുത്സൃജ്യ സമതാ സ്വച്ഛയാ ദിയാ\\
യുക്ത്യാ ച വ്യവഹാരിണ്യാ സ്വാര്‍ത്ഥ: പ്രാജ്ഞേന സാദ്ധ്യതേ (3/78/25) \\
}


വസിഷ്ഠന്‍ തുടര്‍ന്നു: 'രാജാവിനേയും മന്ത്രിയേയും ഒന്നു പരീക്ഷിക്കാന്‍ കാര്‍ക്കടി കാതു തുളയ്ക്കുന്ന ഒരലര്‍ച്ച പുറപ്പെടുവിച്ചു. എന്നിട്ടവള്‍ ആക്രോശിച്ചു: ഹേയ്‌, നിങ്ങള്‍ രണ്ടു കീടങ്ങള്‍ ഈ നിബിഢവനത്തില്‍ എന്തിനാണു ചുറ്റിനടക്കുന്നത്‌? ആരാണു നിങ്ങള്‍ ? വേഗം മറുപടി പറഞ്ഞില്ലെങ്കില്‍ ഞാനിപ്പോള്‍ നിങ്ങളെ വിഴുങ്ങിക്കളയും.'

രാജാവ്‌ ചോദിച്ചു: 'ഹേ, പിശാചിനീ, നീയാരാണ്‌? നീയെവിടെയാണ്‌? എനിക്കു നിന്നെ കേള്‍ക്കാമെങ്കിലും കാണാനാവുന്നില്ല. നീയാരാണെന്ന് വെളിപ്പെടുത്തുക.' രാജാവിന്റെ ശാന്തവും ഉചിതവുമായ മറുപടികേട്ട്‌ തൃപ്തയായ കാര്‍ക്കടി അവളുടെ സ്വരൂപം വെളിപ്പെടുത്തി. രാജാവും മന്ത്രിയും അവളുടെ ഭീകരരൂപം കണ്ടിട്ടും ഭയലേശം പോലും പ്രകടിപ്പിച്ചില്ല. മന്ത്രി അവളോടു ചോദിച്ചു: 'നിനക്കെന്താണിത്ര ക്രോധം? ജീവികള്‍ ഇരതേടുക എന്നത്‌ സാധാരണമാണ്‌. അവരവരുടെ സഹജ കര്‍മ്മങ്ങള്‍ ചെയ്യുമ്പോള്‍ ക്രോധഭാവമെന്തിനാണ്‌?'

"സ്വാര്‍ത്ഥപരമായ കാര്യങ്ങള്‍പോലും വിവേകശാലികള്‍ നേടുന്നത്‌ ഉചിതവും ഉത്തമവുമായ കര്‍മ്മങ്ങളിലൂടെയും പെരുമാറ്റങ്ങളിലൂടെയുമാണ്‌. അവരതിനുവേണ്ടി ക്രോധവും മനസിന്റെ ചാഞ്ചല്യവും കളഞ്ഞ്‌ സമതയോടുകൂടി മനസ്സിനെ ശുദ്ധീകരിക്കുന്നു." 

നിന്നേപ്പോലുള്ള ആയിരക്കണക്കിനു കീടങ്ങളെ ഞങ്ങള്‍ ഉചിതമായി കൈകാര്യം ചെയ്തിട്ടുണ്ട്‌.. ദുഷ്ട ശിക്ഷണവും ശിഷ്ട പരിപാലനവും രാജധര്‍മ്മമാണല്ലോ. ക്രോധമെല്ലാം കളഞ്ഞ്‌ നീ നിന്റെ ലക്ഷ്യം നേടുന്നതിനായി പ്രശാന്തയാകൂ. തന്റെ ലക്ഷ്യം നേടിയാലുമില്ലെങ്കിലും ഒരുവന്‍ തന്റെ പെരുമാറ്റം സമാധാനപരവും കുറ്റമറ്റതുമാക്കണം. പറയൂ, നിനക്ക്‌ എന്തുവേണം? ഞങ്ങള്‍ ഒരു യാചകനേയും വെറുംകയ്യോടെ തിരിച്ചയച്ചിട്ടില്ല.

കാര്‍ക്കടിക്ക്‌ അവരുടെ ധൈര്യത്തേയും അറിവിനെയും കുറിച്ച്‌ മതിപ്പുണ്ടായി. അവര്‍ സാധാരണ മനുഷ്യരല്ലെന്നും അവര്‍ പ്രബുദ്ധരാണെന്നും അവളറിഞ്ഞു. അവരെ കണ്ട മാത്രയില്‍ അവളില്‍ ശാന്തി നിറഞ്ഞു. രണ്ടു ജ്ഞാനികളുടെ സദ്സംഗമുണ്ടാവുമ്പോള്‍ അവരുടെയുള്ളം ശാന്തിയും സമാധാനവും കൊണ്ടു നിറയും. മലമുകളില്‍നിന്നുള്ള രണ്ടരുവികള്‍ സംഗമിക്കുന്നതുപോലെയത്രേ ഇത്‌. മരണത്തിനുമുന്നില്‍പ്പോലും പ്രശാന്തനിര്‍ഭയനായിരിക്കാന്‍ ജ്ഞാനിക്കല്ലാതെയാര്‍ക്കു സാധിക്കും?

അവളാലോചിച്ചു: ഞാനീ അവസരം എന്റെയുള്ളിലെ സംശയനിവാരണത്തിനായി ഉപയോഗിക്കട്ടെ. മഹാത്മാക്കളുടെ സദ്സംഗമുണ്ടായിട്ടും തന്റെ സംശയങ്ങള്‍ ദൂരീകരിക്കാന്‍ ആ അവസരം ഉപയോഗിക്കാത്തവര്‍ തുലോം വിഡ്ഢികളത്രേ. അവള്‍ ചോദിച്ചതിനുത്തരമായി മന്ത്രി രാജാവിനെക്കുറിച്ചുള്ള വിവരങ്ങള്‍ അവളെ അറിയിച്ചു. 

കാര്‍ക്കടി പറഞ്ഞു: രാജാവേ നിങ്ങളുടെ മന്ത്രിക്ക്‌ വിവരമുണ്ടെന്നു തോന്നുന്നില്ല. നല്ലൊരു മന്ത്രിയാണ്‌ രാജാവിനെ വിജ്ഞാനിയാക്കുന്നത്‌.. 'യഥാ രാജാ തഥാ പ്രജാ'. പ്രഭുത്വവും സമതയും രാജവിദ്യയായ ആത്മവിദ്യയില്‍നിന്നും ലഭിക്കുന്നു. ഈ അറിവില്ലാത്തവന്‍ നല്ലൊരുമന്ത്രിയോ വിവേകമുള്ള രാജാവോ ആവില്ല. നിങ്ങള്‍ രണ്ടും ആത്മജ്ഞാനികള്‍ അല്ലെങ്കില്‍ എന്റെ സഹജസ്വഭാവമനുസരിച്ച്‌ എനിക്കു നിങ്ങള്‍ രണ്ടാളേയും ഭക്ഷിക്കാം. അതിനായി ഞാന്‍ നിങ്ങളോട്‌ ചില ചോദ്യങ്ങള്‍ ചോദിക്കാന്‍ പോവുന്നു. എനിക്കതു മാത്രം മതി. 
