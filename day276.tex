\section{ദിവസം 276}

\slokam{
ആനന്ദേ പരിണാമിത്വാദനാനന്ദപദം ഗത:\\
നാനന്ദേ ന നിരാനന്ദേ തതസ്തത്സംവിദാ ബഭൌ  (5/54/68)\\
}

വസിഷ്ഠന്‍ തുടര്‍ന്നു: അപ്പോള്‍ ഉദ്ദാലകന്റെ മനസ്സ് പൂര്‍ണ്ണപ്രശാന്തതയില്‍ അചഞ്ചലമായി പരിലസിച്ചു. തന്റെ ആത്മപ്രകാശത്തിനു തടസ്സമായി നിന്ന അവിദ്യയെന്ന ഇരുട്ടിനെ അദ്ദേഹം നേരില്‍ക്കണ്ടു. എന്നിട്ടാ ഇരുട്ടിനെ ജ്ഞാനമാകുന്ന വെളിച്ചം കൊണ്ട് ഇല്ലാതാക്കിയിട്ട് ആ പ്രഭാപൂരത്തെ ഉള്ളില്‍ ആവാഹിച്ചു നിര്‍ത്തി. ആ വെളിച്ചം മങ്ങുമ്പോഴാണ് മഹര്‍ഷി സുഷുപ്തിയിലേയ്ക്കു വീണു പോവാന്‍ തുടങ്ങിയത്. എന്നാലാ മാന്ദ്യവും ഇല്ലാതാക്കാന്‍ അദ്ദേഹത്തിനു കഴിഞ്ഞു. സുഷുപ്തിയുടെ ആലസ്യം ഇല്ലാതായതോടെ മഹര്‍ഷിയുടെ മനസ്സ് വൈവിധ്യമാര്‍ന്നതും ഉജ്വലപ്രഭ പരത്തുന്നതുമായ അനേകം രൂപങ്ങളെ വിക്ഷേപിക്കാന്‍ തുടങ്ങി. എന്നാലീ ദര്‍ശനങ്ങളെയും മുനി മറികടന്നു. അപ്പോഴേക്ക് അദ്ദേഹം ലഹരിയില്‍ മദിച്ചവനെപ്പോലെ വീണ്ടും മന്ദനായി പരിണമിച്ചു. എന്നാലാ തമസ്സും അദ്ദേഹം ക്ഷണത്തില്‍  അതിജീവിച്ചു. 

അതിനുശേഷം ഇതുവരെ പറഞ്ഞതില്‍ നിന്നുമെല്ലാം വ്യത്യസ്ഥമായ ഒരവസ്ഥയില്‍ മഹര്‍ഷിയുടെ മനസ്സ് വിശ്രമം കൊണ്ടു. അല്‍പ്പനേരം അങ്ങിനെ കഴിഞ്ഞപ്പോഴേയ്ക്ക് അസ്തിത്വത്തിന്റെ സമഷ്ടി ദര്‍ശനത്തിലേയ്ക്ക് ആ മനസ്സുണര്‍ന്നുന്മുഖമായി. പിന്നീട് ക്ഷണത്തിലദ്ദേഹത്തില്‍ ശുദ്ധാവബോധമങ്കുരിച്ചു. ഇതുവരെ മറ്റ്‌ വസ്തുക്കളുമായി ബന്ധപ്പെട്ടിരുന്ന അവബോധമിപ്പോള്‍ സര്‍വ്വമാലിന്യങ്ങളും കളഞ്ഞു സ്വതന്ത്രവും നിര്‍മ്മലവുമായിരിക്കുന്നു. മണ്‍കുടത്തിലെ കളിമണ്ണു കലങ്ങിയ ജലം മുഴുവന്‍ ബാഷ്പീകരിച്ച് കുടം വരണ്ടുണങ്ങുമ്പോള്‍ ആ ചെളിയും കുടത്തിന്റെ അവിഭാജ്യഘടകമായി മാറുമല്ലോ.  

അലകള്‍ കടലില്‍ അലിഞ്ഞുചേര്‍ന്ന്‍ ഒന്നാകുന്നതുപോലെ ബോധം വസ്തുപരിമിതികള്‍ വെടിഞ്ഞ് പരമനൈര്‍മ്മല്യം പ്രാപിക്കുന്നു. ഉദ്ദാലകന്‍ പ്രബുദ്ധനായി. ബ്രഹ്മാദി ദേവതകള്‍ അനുഭവിക്കുന്ന ആനന്ദം ഉദ്ദാലകനില്‍ വേദ്യമായി. അനിര്‍വ്വചനീയമായ ആനന്ദമാണത്. ആനന്ദസാഗരം. മഹാമുനിമാരെ ആ അനന്താവബോധത്തില്‍ ഉദ്ദാലകന്‍ കണ്ടുവെങ്കിലും അദ്ദേഹമവരെപ്പോലും അവഗണിച്ചു. ത്രിമൂര്‍ത്തികളെപ്പോലും അദ്ദേഹം അവിടെ കണ്ടു. എന്നാലദ്ദേഹം അവരില്‍നിന്നുമെല്ലാം അതീതമായ ഒരിടത്ത് പരമമായ ആനന്ദത്തോടെ ജീവന്‍മുക്തനായി വിരാജിച്ചു.  

“അദ്ദേഹം പരിപൂര്‍ണ്ണമായ ആനന്ദസ്വരൂപമായി പരിണമിച്ചു കഴിഞ്ഞിരുന്നു. അതിനാല്‍ അനുഭവവേദ്യമായ ആനന്ദത്തിനപ്പുറമായിരുന്നു ആ അവസ്ഥ. അദ്ദേഹം ആനന്ദമോ നിരാനന്ദമോ അനുഭവിച്ചില്ല.” അദ്ദേഹം സ്വയം ശുദ്ധാവബോധമായി.   

ഈയവസ്ഥ ഒരു ക്ഷണാര്‍ദ്ധമെങ്കിലും അറിഞ്ഞവര്‍ സ്വര്‍ഗ്ഗസുഖങ്ങളില്‍പ്പോലും തല്‍പ്പരരല്ല. ഇതാണ് പരമമായ ലക്ഷ്യം; ശാശ്വതമായ നിവാസസ്ഥലം. ഈ നിലയിലെത്തിവര്‍ പിന്നെ വിഷയ-വിഷയീ ബന്ധങ്ങളില്‍പ്പെട്ടുഴറുകയില്ല. പൂര്‍ണ്ണമായും അറിവുണര്‍ന്ന അവര്‍ക്ക് പിന്നീട് ധാരണകളാലും സങ്കല്‍പ്പമോഹങ്ങളാലും കഷ്ടപ്പെടേണ്ടി വരികയില്ല. ഇതൊരു ‘നേട്ടം’ അല്ല. ഒന്നും നേടേണ്ടതില്ലാത്ത ഒരവസ്ഥയാണിത്. മാനസീകമായി തന്നെ സമീപിച്ച സിദ്ധികളുടെയൊന്നും പ്രലോഭനങ്ങളില്‍പ്പെടാതെ ആറു മാസക്കാലം ഉദ്ദാലകന്‍ ആ സ്ഥിതിയില്‍ത്തന്നെയിരുന്നു. മാമുനിമാരും മഹര്‍ഷിമാരും അദ്ദേഹത്തെ വാഴ്ത്തി.

സ്വര്‍ഗ്ഗവാസത്തിനായി കിട്ടിയ ക്ഷണം അദ്ദേഹം നിരാകരിച്ചു. എല്ലാ മോഹങ്ങളുമൊഴിഞ്ഞ ജീവന്മുക്തനായ ഒരു മാമുനിയായി അദ്ദേഹം എല്ലാടവും കറങ്ങി നടന്നു. അദ്ദേഹം ദിവസങ്ങളും മാസങ്ങളും ഗുഹകളിലിരുന്നു ധ്യാനം ചെയ്തു ചിലവഴിച്ചു. മറ്റവസരങ്ങളില്‍ അദ്ദേഹം സാധാരണ ജീവിതം നയിച്ചുവെങ്കിലും സമ്പൂര്‍ണ്ണസമതയുടെ സംതുലിതാവസ്ഥയിലായിരുന്നു അദ്ദേഹം വിരാജിച്ചിരുന്നത്. അദ്ദേഹത്തിലെ ഉള്‍വെളിച്ചം എപ്പോഴും ഒരേ തരത്തില്‍ ജ്വലിച്ചു പ്രഭ പരത്തിക്കൊണ്ടിരുന്നു. ഒരിക്കലുമതില്‍ ഏറ്റക്കുറച്ചിലുകള്‍ ഉണ്ടായില്ല. ദ്വന്ദാവസ്ഥയുടെ എല്ലാ വികലതകളുമൊടുങ്ങി ദേഹാഭിമാനരഹിതനായി ശുദ്ധസ്വരൂപത്തില്‍ അദ്ദേഹം സദാ അഭിരമിച്ചു.