\section{ദിവസം 186}

\slokam{
അസദിദമഖിലം മയാ സമേതം\\
ത്വിതി വിഗണയ്യ വിഷാദിതാസ്തു മാ തേ\\
സദിഹ ഹി സകലം മയാ സമേതം\\
ത്വിതി ച വിലോക്യ വിഷാദിതാസ്തു മാ തേ (4/45/50)\\
}

വസിഷ്ഠൻ തുടർന്നു: രാമ: ഈ വിശ്വമിങ്ങനെ നിലനില്‍ക്കുന്നതായി തോന്നുന്നുണ്ടെങ്കിലും 'വിശ്വ'മായി ഒന്നും തന്നെ സത്യത്തിൽ ഇല്ല. അനന്താവബോധത്തിന്റെയൊരു വിക്ഷേപം മാത്രമാണു നാമീക്കാണുന്ന പ്രപഞ്ചം. ബോധത്തിൽ പ്രത്യക്ഷപ്പെടുന്ന വിശ്വസൃഷ്ടി എന്നത് ഒരു സ്വപ്നദൃശ്യം മാത്രം. അനന്തശൂന്യതയായ സത്യം മാത്രമാണുണ്മ. അതിലാണല്ലോ ഈ ദൃശ്യപ്രപഞ്ചം കാണപ്പെടുന്നത് . കണ്ണു മുതലായ ഇന്ദ്രിയങ്ങൾക്ക് ഗോചരമായതിനാലാണ്‌ ലോകമുണ്ടെന്നു നാം ധരിക്കുന്നത്. അതുപോലെ നാം ചിന്തിക്കുന്നതുകൊണ്ടാണ്‌ ലോകത്തിന്‌ നമ്മുടെ മനസ്സിൽ ഒരസ്തിത്വം ഉണ്ടാവുന്നത്. മനസ്സ്, അതിനു വസിക്കാനായാണ്‌ ഈ ദേഹത്തെ വിഭാവനം ചെയ്തുണ്ടാക്കിയത്. മനസ്സിന്റെ എല്ലാ കഴിവുകളും അനന്താവബോധത്തിൽ നിന്നും ഉദ്ഭൂതമായതാണ്‌. അനന്താവബോധം തന്നെയാണ്‌. മനസ്സ്. അത് സർവ്വശക്തമാണെന്ന് മഹർഷിമാർ പറയുന്നത് ഈ അർത്ഥത്തിലാണ്‌.. ദേവകളേയും, രാക്ഷസന്മാരേയും മനുഷ്യരേയുമെല്ലാം മനസ്സു തന്നെ പടച്ചുണ്ടാക്കിയതാണ്‌. എപ്പോഴാണോ മനസ്സിൽ ഈ ധാരണകൾക്ക് അവസാനമകുന്നത്, അപ്പോൾ ഇവരെല്ലാം ഇല്ലാതാവും. എണ്ണയില്ലാതെ വിളക്കു കത്തുന്നതെങ്ങിനെ?

ജ്ഞാനിയെ സംബന്ധിച്ചിടത്തോളം ലൗകീക പദാർത്ഥങ്ങൾ സുഖസമ്പാദനത്തിനുള്ളതാണ് എന്ന മിഥ്യാധാരണയില്ല. കാരണം അവയ്ക്ക് ഉണ്മയില്ലെന്ന് അയാൾക്കറിയാം. സ്വന്തം മനസ്സുണ്ടാക്കിയ വസ്തുക്കളിൽ സുഖം തിരയുന്നവനു ദു;ഖമാണനുഭവം. ആശയാലാണ്‌ ഈ ലോകം തന്നെ ഉരുത്തിരിഞ്ഞത്. അതുകൊണ്ട് തന്നെ ആശയുടെ അന്ത്യത്തിൽ മാത്രമേ ഈ ലോകമില്ലാതെയാവൂ. അതിനെതിരായി പ്രവർത്തിച്ചാലോ അതിനോട് വെറുപ്പുണ്ടായതുകൊണ്ടോ ലോകമസ്തമിക്കില്ല. ലോകമസ്തമിക്കുമ്പോൾ യാതൊന്നും തന്നെ നശിക്കുന്നില്ല താനും. അയദാർത്ഥമായ ഒരു വസ്തു ഇല്ലാതാവുമ്പോൾ എന്താണു നഷ്ടമാവുക? അത് തികച്ചും അയദാർത്ഥമാണെങ്കിൽ ഇല്ലാത്ത ഒന്നിനെ എങ്ങിനെ നശിപ്പിക്കാനാകും? മാത്രമല്ല, ഇല്ലാത്ത ഒന്ന് നഷ്ടമായാൽ എന്തിനാണു നാം വിലപിക്കുന്നത്? മറിച്ച് അത് ഉണ്മയാണെന്നിരിക്കട്ടെ. അതിനെ ആർക്കും ഇല്ലാതാക്കാനോ അസത്താക്കാനോ സാദ്ധ്യമല്ല. ഇങ്ങിനെ നോക്കുമ്പോൾ ജഗത്തും ബ്രഹ്മം തന്നെ. പരമസത്യം. അപ്പോള്‍ ദു:ഖത്തിനെവിടെയാണു സ്ഥാനം?

അതുപോലെ അയദാർത്ഥമായതിന്‌ വളരാനോ പുഷ്ടിപ്പെടാനോ ആവുന്നതെങ്ങിനെ? അപ്പോൾപ്പിന്നെ എന്താണ്‌ നാം ആഘോഷിക്കുന്നത്? എന്തിനായാണ്‌ നാം ആശിക്കുന്നത്? എല്ലാമെല്ലാം അനന്തമായ അവബോധം മാത്രമാവുമ്പോൾ നമുക്ക് സംന്യസിക്കാൻ എന്താണുള്ളത്?

തുടക്കത്തിൽ ഇല്ലാതിരുന്ന വസ്തു; ഒടുവിലും ഇല്ല. അതുകൊണ്ടുതന്നെ മദ്ധ്യത്തിൽ, (അതായത് വർത്തമാന കാലത്തിൽ) അത് സത്യമാവുക എന്നത് അസംബന്ധം. തുടക്കത്തിൽ ഉണ്മായുള്ളത്, അവസാനവും ഉണ്മ തന്നെ. അതിപ്പോഴും സത്താണ്‌., ഉണ്മയാണ്‌...  “ഞാനടക്കം എല്ലാം അയാദാർത്ഥമാണ്‌ എന്നു കണ്ടാൽപ്പിന്നെ ശോകമില്ല. അല്ലെങ്കിൽ ഞാനടക്കം എല്ലാം യാദാർത്ഥമാണ്‌ എന്നു കണ്ടാലും ശോകം നമ്മെ തീണ്ടുകയില്ല.”

വസിഷ്ഠമുനി ഇത്രയും പറഞ്ഞു തീർന്നപ്പോൾ ഒൻപതാം ദിവസത്തെ പ്രഭാഷണം അവസാനിക്കുകയും സഭ പിരിയുകയും ചെയ്തു. 

