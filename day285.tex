\section{ദിവസം 285}

\slokam{
താനി മിത്രാണി ശാസ്ത്രാണി താനി താനി ദിനാനി ച\\
വിരാഗോല്ലാസവാന്യേഭ്യ ആത്മചിത്തോദയ: സ്ഫുടം (5/64/19)\\
}

വസിഷ്ഠന്‍ തുടര്‍ന്നു: പ്രത്യക്ഷലോകത്തിന്റെ മിഥ്യാവസ്ഥയെപ്പറ്റി ചര്‍ച്ച ചെയ്തും പരസ്പരം ഉപചാരങ്ങള്‍ അര്‍പ്പിച്ചും സുരാഗുവും പരിഘനും അവരവരുടെ കര്‍മ്മ മണ്ഡലങ്ങളിലേയ്ക്ക് തിരിച്ചു പോയി. 

രാമാ, ഇപ്പറഞ്ഞ ജ്ഞാനത്തില്‍ മനസ്സുറപ്പിക്കുക. ആശുദ്ധധാരണയായ അഹത്തെ ഉള്ളില്‍നിന്നും ഓടിച്ചകറ്റുക. നമ്മുടെ ഹൃദയം, നിര്‍മ്മലവും എല്ലാ ആനന്ദത്തിന്റെയും ഇരിപ്പിടവുമായ, ഏവര്‍ക്കും പ്രാപ്യമായ, അനന്താവബോധവിഹായസ്സില്‍ ധ്യാനനിരതനാവുമ്പോള്‍ സ്വയം പരമാത്മാവില്‍ അഭിരമിക്കുകയാണ് ചെയ്യുന്നത്. അങ്ങിനെ അനന്താവബോധത്തില്‍ ഭക്ത്യാദരസംയുക്തമായ മനസ്സ് ആത്മജ്ഞാനനിരതവും അന്തര്‍മുഖത്വമുള്ളതും ആകയാല്‍ അതിനു ദു:ഖങ്ങളൊന്നും അനുഭവമാകുന്നില്ല. അപ്പോള്‍ നിനക്ക് നിത്യജീവിതത്തിലെ പ്രവര്‍ത്തനങ്ങളില്‍ ഏര്‍പ്പടേണ്ടിവന്നാലും ഇഷ്ടാനിഷ്ടദ്വന്ദങ്ങള്‍ക്ക് ഇടം നല്‍കേണ്ടതായി വന്നാലും നിന്റെ അന്തരംഗം ഒരിക്കലും മലിനപ്പെടുകയില്ല.        

ഇരുട്ടിനെ അകറ്റാന്‍ തീനാളമെത്ര ചെറുതാണെങ്കിലും അത് പരത്തുന്ന വെളിച്ചത്തിന് മാത്രമേ കഴിയൂ. അതുപോലെ ഈ ലോകം അജ്ഞാനത്തിന്റെ സൃഷ്ടിയാണെന്ന അറിവിന്‌ മാത്രമേ എല്ലാ പീഢകളില്‍ നിന്നും നമ്മെ കരകയറ്റാനാവൂ. ഈ അറിവുണര്‍ന്നു കഴിഞ്ഞാല്‍പ്പിന്നെ ലോകമെന്ന മിഥ്യാധാരണയ്ക്ക് അന്ത്യമായി. മത്സ്യത്തിന്റെ കണ്ണുകള്‍ക്ക് കടലിലെ ഉപ്പുവെള്ളം കൊണ്ട് പ്രശ്നങ്ങള്‍ ഒന്നുമുണ്ടാകാത്തതുപോലെ അനാസക്തനായി ലോകവ്യവഹാരങ്ങളില്‍ ഏര്‍പ്പെടുന്നതുകൊണ്ട് നിനക്ക് യാതൊരുവിധ കളങ്കവും സംഭവിക്കുകയില്ല. 

ഇനിയൊരിക്കലും നിന്നെ മതിഭ്രമം ബാധിക്കുകയില്ല.ആത്മജ്ഞാനത്തിന്റെ പ്രഭയുള്ളില്‍ ഭാസുരമാവുന്ന ദിനങ്ങള്‍ മാത്രമേ നാം ശരിക്കും ജീവിക്കുന്നുള്ളു എന്നറിഞ്ഞാലും. ഈ ദിനങ്ങളില്‍ അവന്റെ കര്‍മ്മങ്ങളെല്ലാം സാര്‍ത്ഥവും  ആനന്ദദായകവുമായിരിക്കും. “വേദശാസ്ത്രങ്ങളും ഇതുപോലെ സത്സംഗനിരതമായ ദിനങ്ങളുമാണ് നമ്മുടെ ഉത്തമ സുഹൃത്തുക്കള്‍.  അവ നമ്മില്‍ അനാസക്തി വളര്‍ത്തി നമുക്ക് ആത്മജ്ഞാനപ്രാപ്തി നല്‍കുന്നു.”

അല്ലയോ രാമാ, ഈ ദുരിതംപിടിച്ച പ്രത്യക്ഷലോകത്ത് നിന്നും നിന്റെ ജീവനെ രക്ഷിച്ചാലും. ഒരിക്കലീ ലോകത്തിന്റെ യഥാര്‍ത്ഥസത്യം തിരിച്ചറിഞ്ഞാല്‍പ്പിന്നെ നാമൊരിക്കലും ഈ ചെളിക്കുണ്ടില്‍ വീഴുകയില്ല. രാമാ, മഹാത്മാക്കളുമായുള്ള സത്സംഗം നിനക്ക് ആത്മജ്ഞാനമാര്‍ജ്ജിക്കുവാനുള്ള എല്ലാ തയ്യാറെടുപ്പും നല്‍കും. അതിനാല്‍ സത്സംഗം ഇല്ലാത്തിടത്ത് സത്യാന്വേഷി താമസിക്കരുത്. മഹര്‍ഷിമാരുടെ സാന്നിദ്ധ്യത്തില്‍ സാധകന്റെ മനസ്സ് ക്ഷണത്തില്‍ പ്രശാന്തമാവുന്നു. ഒരുവന്‍ എപ്പോഴും അഭിവൃദ്ധിയാണല്ലോ കാംക്ഷിക്കുന്നത്. അതിനാല്‍ ഈ അജ്ഞാനത്തില്‍ത്തന്നെ കഴിഞ്ഞ് നീയൊരിക്കലും സ്വയം വളര്‍ച്ച മുരടിപ്പിക്കരുത്. 
 
വിവേകശാലികള്‍ എപ്പോഴും ലോകത്തിന്റെ സ്വഭാവത്തെപ്പറ്റിയും ആത്മാവിനെപ്പറ്റിയും ആലോചിക്കണം. സുഹൃത്തുക്കളോ, ബന്ധുക്കളോ, ശാസ്ത്രങ്ങളോ, ഇങ്ങിനെയുള്ള മനനത്തിനു സഹായകരമല്ല. അനാസക്തമായ, ആത്മാന്വേഷണനിരതമായ ശുദ്ധമായ മനസ്സാണ് സംസാരസാഗരം തരണം ചെയ്യുവാനാവശ്യം.

ശരീരത്തെ വെറും ജഢമായി കാണാന്‍ തുടങ്ങുന്നതോടെ ഒരുവനില്‍ ആത്മജ്ഞാനം അങ്കുരിക്കുന്നു. അഹമെന്ന ഇരുട്ട് നശിക്കുമ്പോള്‍ ആത്മജ്ഞാനപ്രകാശമാണ് പ്രഭാസിക്കുന്നത്. പ്രബുദ്ധത എന്നത്  വാക്കുകള്‍കൊണ്ട് വിവരിക്കാന്‍ കഴിയാത്ത ഒരു പ്രതിഭാസമത്രേ. പഞ്ചസാരയുടെ മധുരം വിവരണാതീതമാണെന്നതുപോലെ ആത്മജ്ഞാനം നേരറിവിന്റെ അനുഭവം മാത്രമാണ്. മനസ്സും അഹംകാരവും ഒടുങ്ങുമ്പോള്‍ ആത്മജ്ഞാനം പ്രകാശിക്കുന്നു. യോഗാഭ്യാസം കൊണ്ടിത് പ്രാപിക്കാം. ദീര്‍ഘസുഷുപ്തി പോലെയാണീ അവസ്ഥയെന്ന് ചിലര്‍ പറയുമെങ്കിലും വിവരണാതീതമാണത്. താരതമ്യങ്ങള്‍ക്ക് വഴങ്ങാത്ത ഒരഭൌമതലമാണത്.

