 
\section{ദിവസം 127}

\slokam{
ആക്രുഷ്ടമുദ്ധരതരം രുദിതം വിപസ്തു\\
ഭുക്തം  കദന്നമുഷിതം  ഹതപക്കണേ ശു\\
കാലാന്തരം ബഹു മായോപഹതേന തത്ര\\
ദുര്‍വാസനാനിഗഡബന്ധഗതേന സഭ്യാ: (3/107/48)\\
}

രാജാവ്‌ തുടര്‍ന്നു: ഉടനേ ഞാനൊരു പ്രാകൃതവര്‍ഗ്ഗത്തില്‍പ്പെട്ട ഒരാളായി മാറി . എന്റെ ഭാര്യ ഒരു പെണ്‍കുട്ടിക്കു ജന്മം നല്‍കി. അതും എന്റെ ദു:ഖത്തെ വര്‍ദ്ധിപ്പിച്ചു. പിന്നെ കാലക്രമത്തില്‍ മൂന്നു കുട്ടികള്‍കൂടിയുണ്ടായി. ഞാനങ്ങിനെ ആ ആദിവാസിവര്‍ഗ്ഗത്തിലെ ഗൃഹസ്ഥനായി. അങ്ങിനെ ഭാര്യയേയും കുട്ടികളേയും ഭക്ഷണംകൊടുത്ത്‌ സംരക്ഷിക്കാനുള്ള തത്രപ്പാടില്‍ കുറേകാലം കഴിച്ചുകൂട്ടി. ഞാന്‍ വിറകുവെട്ടിയാണ്‌ കഴിഞ്ഞിരുന്നത്‌. പലപ്പോഴും മരച്ചുവടായിരുന്നു രാത്രിയില്‍ എന്റെ കിടപ്പറ. തണുപ്പുള്ളപ്പോള്‍ ഞാന്‍ പൊന്തക്കാട്ടില്‍ ചൂടുകിട്ടാന്‍ ഒളിച്ചിരുന്നു. പന്നിമാംസമായിരുന്നു എന്റെ പ്രധാന ഭക്ഷണം. കാലക്രമത്തില്‍ വയസ്സേറിയപ്പോള്‍ ഞാന്‍ മാംസക്കച്ചവടം തുടങ്ങി. വിന്ധ്യാചലത്തിലുള്ള ഗ്രാമങ്ങളില്‍പ്പോയാണ്‌ പ്രധാനമായും ഞാന്‍ മാംസം വിറ്റിരുന്നത്‌.. വില്‍ക്കാന്‍ സാധിക്കാതെവന്നിരുന്ന ബാക്കി മാംസം ഞാന്‍ മുറിച്ചുണക്കിയിരുന്നത്‌ ഒരു വൃത്തികെട്ട സ്ഥലത്തായിരുന്നു. വിശക്കുമ്പോള്‍ ചിലപ്പോള്‍  ഒരുകഷണം മാംസത്തിനായിപ്പോലും  ഞാന്‍ മറ്റുള്ളവരുമായി വഴക്കിട്ടു.

എന്റെ ശരീരം കരിക്കട്ടയുടെ നിറമായി. പാപപങ്കിലമായ കര്‍മ്മങ്ങളിലേര്‍പ്പെട്ട്‌ എന്റെ മനസ്സും പാപവൃത്തികളിലേയ്ക്ക്‌ ഉന്മുഖമായിത്തീര്‍ന്നു. സദ്ച്ചിന്തകളും സഹാനുഭൂതിയും എന്നില്‍ അപ്രത്യക്ഷമായി. പാമ്പ്‌ പടം പൊഴിക്കുമ്പോലെ എന്നില്‍ ദയാവായ്പ്പ്‌ ഇല്ലാതെയായി. പക്ഷിമൃഗാദികളെ കെണിവച്ചും വലവച്ചും പിടിച്ച്‌ പറയാനരുതാത്തത്ര ക്രൂരമായി ഞാനവരെ ദ്രോഹിച്ചു. വെറുമൊരുകോണകം മാത്രമുടുത്ത്‌ ഋതുമാറ്റങ്ങളെല്ലാം ഞാന്‍ സഹിച്ചു. അങ്ങിനെ ഏഴുവര്‍ഷങ്ങള്‍ ഞാന്‍ കഴിഞ്ഞു. "തിന്മ നിറഞ്ഞ ദുര്‍വാസനകളാകുന്ന കയറാല്‍ ബന്ധിതനായി ഞാന്‍ ക്രുദ്ധനായി മാറി. അസഭ്യം പറയുകയും, ചിലപ്പോള്‍ ദുര്‍ഗ്ഗതിയില്‍ വിലപിച്ച്‌ കണ്ണീരൊഴുക്കുകയും ജീര്‍ണ്ണിച്ച ഭക്ഷണം കഴിക്കുകയും ചെയ്തു. അങ്ങിനെ ഞാന്‍ ഏറെക്കാലം അവിടെക്കഴിഞ്ഞു."

കാറ്റില്‍പ്പറക്കുന്ന കരിയില പോലെ ഞാന്‍ അലഞ്ഞു. ഭക്ഷണം കഴിക്കുക എന്നതുമാത്രമായി എന്റെ ജീവിത ലക്ഷ്യം. നാട്ടില്‍ അപ്പോള്‍ വരള്‍ച്ചയുണ്ടായി. തീനാളം പോലെ ചൂടുള്ളതായിരുന്നു അവിടുത്തെ കാറ്റ്‌.. കാടിനു തീപിടിച്ചു, ചാരം മാത്രം ബാക്കിയായി. ആളുകള്‍ പട്ടിണികൊണ്ട്‌ മരിക്കാന്‍ തുടങ്ങി. മരുപ്പച്ചക്കുപിറകേ ദാഹജലത്തിനായി ആളുകള്‍ ഓട്ടമായി. കല്‍ത്തുണ്ടുകള്‍ മാംസങ്ങളാണെന്നു തെറ്റിദ്ധരിച്ച്‌ ആളുകള്‍ അവയെടുത്ത്‌ ചവച്ചുനോക്കി. ചിലര്‍ മരിച്ചുപോയവരുടെ ശവം ഭക്ഷണമാക്കി. ചിലര്‍ സ്വന്തം വിരല്‍ മരിച്ചവരുടെ ദേഹത്തിലെ ചോരയില്‍ മുക്കി ആഹരിച്ചു. അതായിരുന്നു പട്ടിണിയുടെ ഭ്രാന്തമായ അവസ്ഥ. ഒരുകാലത്ത്‌ ഫലഭൂയിഷ്ടമായിരുന്ന വനപ്രദേശം ഇന്ന് ചുടുകാടായിരിക്കുന്നു. ആഹ്ലാദനാദങ്ങള്‍ക്കുപകരം ഇന്നു കേള്‍ക്കുന്നത്‌ വേദനാവിലാപങ്ങളാണ്‌.

(കുറിപ്പ്‌:: ഈ രണ്ടു അദ്ധ്യായങ്ങള്‍ മുഴുവന്‍ വളരെ വിശദമായി പ്രദിപാദിച്ചിരിക്കുന്ന ഉചിതമായ ഉപമകള്‍കൊണ്ടു സമ്പന്നമാണ്‌.)

