 
\section{ദിവസം 091}

\slokam{
യഥാ സംപധ്യതേ ബ്രഹ്മാ കീട: സംപധ്യതേ തദാ\\
കീടസ്തു രൂഠഭൂതൌഘവലനാത്തുച്ഛകര്‍മക: (3/67/69)\\
}

വസിഷ്ഠന്‍ തുടര്‍ന്നു: പലേവിധ കാരണങ്ങള്‍ കൊണ്ട്‌ ചിന്തകളുടെ ചലനം ഉണ്ടാവുന്നു. ചിലര്‍ ഒരേയൊരുജന്മം കൊണ്ട്‌ ഇതില്‍നിന്നും മോചിതരാവുന്നു. മറ്റുചിലര്‍ ആയിരം ജന്മംകഴിഞ്ഞ്‌ മുക്തരാവുന്നു. ചിന്തകള്‍ ചലിച്ചുകൊണ്ടിരിക്കുമ്പോള്‍ സത്യദര്‍ശനം സാദ്ധ്യമല്ല,കാരണം അപ്പോളവിടെ 'ഞാന്‍' ഉണ്ട്‌, 'ഇതെന്റേതാണ്‌' തുടങ്ങിയ ധാരണകള്‍ അപ്പോള്‍  നിലവിലുണ്ടല്ലോ. 

കാണപ്പെടുന്ന ലോകം അവബോധത്തിന്റെ ജാഗ്രതാവസ്ഥയാണ്‌.. അഹംകാരം അതിന്റെ സ്വപ്നാവസ്ഥയാണ്‌.. മനസ്സെന്ന വസ്തുവാണതിന്റെ സുഷുപ്തി അവസ്ഥ. ശുദ്ധബോധമാണ്‌ നാലാമത്തെ അവസ്ഥ. അത്‌ വൈരുദ്ധ്യമേതുമില്ലാത്ത, നിഷേധിക്കാനരുതാത്ത സത്യമത്രേ. ഈ നാലാമത്തെ അവസ്ഥയ്ക്കുമപ്പുറം ബോധത്തിന്റെ പരമനൈര്‍മ്മല്യമുണ്ട്‌.. ഈ സത്യത്തില്‍ ദൃഢീകരിച്ചവനെ ദു:ഖം തീണ്ടുകയില്ല. കാണപ്പെടുന്ന ഈ ലോകത്തിന്റെ കാരണമായി പരബ്രഹ്മത്തിനെ പറയുന്നു. ഒരു വൃക്ഷം വളര്‍ന്നു വലുതാവാനുള്ള കാരണം ആകാശമാണെന്നു പറയുന്നതുപോലെയാണത്‌.. (അകാശം മരത്തിന്റെ വളര്‍ച്ചയെ തടയുന്നില്ല;അതുകൊണ്ട്‌ ആകാശം വളര്‍ച്ചക്കു ഹേതുവാകുന്നു). സമഗ്രമായ അന്വേഷണത്തിലൂടെ, വാസ്തവത്തില്‍ ബ്രഹ്മം സജീവമായി യതൊന്നിനും ഹേതുവായി വര്‍ത്തിക്കുന്നില്ല എന്നറിയാന്‍ കഴിയും. മണ്ണുകുഴിച്ചുനീങ്ങുന്നവന്‍ അവസാനം കണ്ടെത്തുന്നത്‌ ശൂന്യമായ ആകാശമാണെന്നതുപോലെ അന്വേഷണം തുടരുമ്പോള്‍ ഇതെല്ലാം അനന്താവബോധമല്ലാതെ മറ്റൊന്നുമല്ല എന്നറിയുന്നു. 

രാമന്‍ ചോദിച്ചു: ഈ സൃഷ്ടിജാലങ്ങള്‍ ഇത്ര ബ്രഹത്തായതെങ്ങിനെയെന്ന് ദയവായി പറഞ്ഞു തരൂ. 

വസിഷ്ഠന്‍ തുടര്‍ന്നു: ബോധത്തിലെ ആന്ദോളനവും അനന്താവബോധത്തിലെ കമ്പനങ്ങളും വ്യത്യസ്ഥമല്ല. ആന്ദോളനങ്ങളില്‍നിന്നും ജീവന്‍ പ്രത്യക്ഷമാവുന്നതുപോലെ ജീവനില്‍ മനസ്സുണ്ടാവുന്നത്‌ ചിന്തകളുള്ളതുകൊണ്ടാണ്‌.. മനസ്സ്‌ സ്വയം പഞ്ചഭൂതങ്ങളാണെന്ന  ധാരണ പുലര്‍ത്തി അവയായിത്തീരുന്നു. മനസ്സ്‌ എന്തു ചിന്തിക്കുന്നുവോ അതുമാത്രം കാണുന്നു. അതുകഴിഞ്ഞ്‌ ഓരോന്നോരോന്നായി ജീവന്‍ - നാവ്‌, കണ്ണുകള്‍ , മൂക്ക്‌, സ്പര്‍ശം തുടങ്ങിയ ഇന്ദ്രിയങ്ങളെ ആര്‍ജ്ജിക്കുന്നു. ഇതില്‍ ഇന്ദ്രിയങ്ങളും മനസ്സും തമ്മില്‍ കാര്യ കാരണബന്ധമില്ല. എന്നാല്‍ കാക്കയും പനമ്പഴവും പോലുള്ള ആകസ്മികതയാണ്‌ ചിന്തകളും ഇന്ദ്രിയങ്ങളുടെ ആവിര്‍ഭാവവും. ഒരു കാക്ക പറന്നുവന്ന് പനയില്‍ ഇരിക്കുന്ന നിമിഷം നിലത്തുവീഴുന്ന പനമ്പഴം കാക്ക കൊത്തിയിട്ടതാണെന്നു കാണികള്‍ക്കു തോന്നാം, അത്രമാത്രം. അങ്ങിനെയാണ്‌ ആദ്യത്തെ വിശ്വജീവന്‍ ഉണ്ടായത്‌.

രാമന്‍ ചോദിച്ചു: അവിദ്യയെന്നത്‌ സത്യത്തില്‍ നിലനില്‍പ്പില്ലാത്ത ഒന്നാണല്ലോ. അപ്പോള്‍പ്പിന്നെ എന്തിനാണു നാം മുക്തിയെപ്പറ്റിയും ആത്മാന്വേഷണത്തെപ്പറ്റിയും ആലോചിച്ച്  വിഷമിക്കുന്നത്‌? 

വസിഷ്ഠന്‍ പറഞ്ഞു: രാമ: ആ ചോദ്യം അതുയരേണ്ട സമയത്തുയരണം, ഇപ്പോഴല്ല. പൂക്കള്‍ വിടരുന്നതും കായ്കള്‍ പഴുക്കുന്നതും അതതിന്റെ സമയത്താണല്ലോ. ഈ വിശ്വജീവന്‍ 'ഓം' എന്നുരുവിട്ട്‌ ശുദ്ധമായ ഇഛകൊണ്ട്‌ ഇക്കാണായതെല്ലാം സൃഷ്ടിക്കുന്നു. "ബ്രഹ്മാവിനെ ഇഛയാല്‍ എങ്ങിനെ സൃഷ്ടിച്ചുവോ അങ്ങിനെതന്നെയാണ്‌ ഒരു പുഴുവിനെ സൃഷ്ടിച്ചതും. പുഴുവില്‍ മാലിന്യമുള്ളതുകൊണ്ട്‌ അതിനു തുഛമായ കര്‍മ്മമേ അനുഷ്ഠിക്കാനുള്ളു." ഇവ തമ്മിലുള്ള അന്തരം വെറും ഭാവനമാത്രം. സത്യത്തില്‍ സൃഷ്ടിയേ ഇല്ല, അതിനാല്‍ ഭിന്നതയും ഇല്ല. 
