\newpage
\section{ദിവസം 014}

\slokam{
ന ജിതാ: ശത്രുഭി: സംഖ്യേ പ്രവിഷ്ടാ യേദ്രികോടരേ\\
തേ ജരാജീർണരാക്ഷസ്യാ പശ്യാശു വിജിതാ മുനേ (1/22/31)\
}

രാമന്‍ തുടര്‍ന്നു: യൌവ്വനത്തില്‍ മനുഷ്യന്‍ ലൈംഗീകാകര്‍ഷണത്തിനടിമയാണ്‌..  രക്തമാംസാസ്ഥി രോമചര്‍മ്മങ്ങളുടെ കൂടിച്ചേരലായ ശരീരത്തില്‍ സൌന്ദര്യവും ആകര്‍ഷണീയതയും അവന്‍ കണ്ടെത്തുന്നു. ഈ സൌന്ദര്യമാകട്ടെ സുസ്ഥിരമായിരുന്നുവെങ്കില്‍ ഈ ആസക്തിയെ നമുക്ക്‌ ന്യായീകരിക്കാം, പക്ഷേ ഈ ശരീരസൌന്ദര്യം അധികം നീണ്ടുനില്‍ക്കുന്നില്ലല്ലോ. മറിച്ച്‌ ആകര്‍ഷണം നല്‍കിയിരുന്ന തന്റെ പ്രിയപ്പെട്ടവള്‍ , ഈ മാംസസഞ്ചയമായ സൌന്ദര്യധാമം, താമസംവിനാ വാര്‍ദ്ധക്യത്തിന്റെ ജരാനരകള്‍ ബാധിച്ച്‌ ഒടുവില്‍ അഗ്നിക്കോ മണ്ണിലെ കീടങ്ങള്‍ക്കോ കഴുകനോ ഇരയായിത്തീരുന്നു. എങ്കിലും നിലനില്‍ക്കുന്നിടത്തോളം കാലം ലൈംഗീകാകര്‍ഷണം മനുഷ്യന്റെ ഹൃദയവും വിവേകവും കവരുന്നു. ഈ ആസക്തികൊണ്ടാണ്‌ സൃഷ്ടി നിലനില്‍ക്കുന്നത്‌. ഈ ആകര്‍ഷണം നിലച്ചാല്‍ ജനനമരണമെന്ന സംസാരചക്രത്തിനും അവസാനമായി.

ബാല്യം അസംതൃപ്തമാവുമ്പോള്‍ യൌവ്വനം അവനെ കൈവശപ്പെടുത്തുന്നു. യൌവ്വനകാലം ദുരിതത്തിലും അസംതൃപ്തിയിലും വിഫലമായിത്തീരുമ്പോഴേയ്ക്ക്‌ വാര്‍ദ്ധക്യം പിടികൂടുന്നു. ജീവിതം എത്ര ക്രൂരം! കാറ്റ്‌ ഇലയില്‍പ്പറ്റിയിരിക്കുന്ന ജലകണത്തെ തെറിപ്പിച്ചു കളയുമ്പോലെ വാര്‍ദ്ധക്യം ശരീരത്തെ ഇല്ലായ്മചെയ്യുന്നു. ഒരു വിഷത്തുള്ളിയെപ്പോലെ ജരാതുരത്വം ശരീരത്തെമുഴുവന്‍ ബാധിച്ച്‌ ഓരോ അവയവങ്ങളുടെയും പ്രാപ്തി കുറച്ച്‌ അവനെ എല്ലാവരുടെയും പരിഹാസപാത്രമാക്കുന്നു. ശാരീരികമായി വയ്യെങ്കിലും വാര്‍ദ്ധക്യത്തിലും അവന്റെ തൃഷ്ണയ്ക്കു കുറവൊന്നുമില്ല. അവ വര്‍ദ്ധിച്ചുകൊണ്ടേയിരിക്കുന്നു. "ഞാന്‍ ആരാണ്‌? ഞാന്‍ എന്താണു ചെയ്യേണ്ടത്‌?" തുടങ്ങിയ ചോദ്യങ്ങള്‍ മനസ്സിനുള്ളില്‍ ഉണരുമ്പോഴേയ്ക്കും ജീവിതപ്പാതയെ നേര്‍വഴിക്കു കൊണ്ടുപോവാനും ജീവിതശെയിലിയില്‍ മാറ്റം വരുത്താനും ജീവിതം കൂടുതല്‍ അര്‍ത്ഥവത്താക്കാനും തുലോം വൈകിയിരിക്കും. ജരാതുരതയ്ക്കൊപ്പം ചുമ പോലുള്ള ശാരീരികാസ്വാസ്ഥ്യങ്ങള്‍ , നര, ശ്വാസതടസ്സം, അജീര്‍ണ്ണം, ശരീരശോഷണം എന്നിവയെല്ലാം സ്വഭാവീകമായി ഉണ്ടാവും. ഒരുപക്ഷേ മരണദേവന്‍ മുകളില്‍ നിന്നു നോക്കുമ്പോള്‍ ഈ വെളുത്ത തല കണ്ടിട്ട്‌ ഉപ്പിട്ടുവച്ച മത്തനാണെന്നു കരുതി അതെടുക്കാന്‍ തിരക്കിട്ടു വരാനും മതി!

നദീ തീരത്തുനില്‍ക്കുന്ന വൃക്ഷങ്ങളെ വേരോടെ പുഴക്കിയെറിയുന്ന വെള്ളപ്പൊക്കം പോലെ വാര്‍ദ്ധക്യം ജീവിതവൃക്ഷത്തിന്റെ വേര്‌ ആവേശത്തോടെ മുറിച്ചെറിയുന്നു. മരണം അതു കൊണ്ടുപോവുകയും ചെയ്യുന്നു. വാര്‍ദ്ധക്യപീഢകള്‍ മരണദേവനുമുന്‍പേ നടക്കുന്ന യമദൂതരത്രേ. അഹോ എത്ര നിഗൂഢം! വിസ്മയകരം!

"ശത്രുക്കളില്‍ നിന്ന് ഒരിക്കലും പരാജയമേറ്റുവാങ്ങാത്തവരും മറ്റാര്‍ക്കും എത്തിപ്പെടാനാവാത്ത മലമുകളില്‍ താമസമാക്കിയവരും എല്ലാം വാര്‍ദ്ധക്യത്തിലെ ജരാതുരത്വം എന്ന രാക്ഷസിയുടെ പിടിയില്‍ നിന്നു രക്ഷപ്പെടുന്നില്ല."
