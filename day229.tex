\section{ദിവസം 229}

\slokam{
ഏക ഏവാസ്മി സുമഹാംസ്തത്ര രാജാ മഹാദ്യുതി:\\
സർവകൃത് സർവഗ: സർവ: സ ച തൂഷ്ണീം വ്യവസ്ഥിത: (5/23/6)\\
}

വിരോചനൻ ബലിയോടായി തുടർന്നു പറഞ്ഞു: മകനേ മൂന്നു ലോകങ്ങളേയും ഉൾക്കൊള്ളാൻ കഴിയുന്ന ഒരു മണ്ഡലമുണ്ട്. അതിൽ തടാകങ്ങളോ സമുദ്രങ്ങളോ മലകളോ കാടുകളോ നദികളോ ഭൂമിയോ ആകാശമോ കാറ്റോ ഞാനോ വിഷ്ണു ഭഗവാൻ തുടങ്ങിയ ദേവന്മാരോ ഇല്ല. “അവിടെയുള്ള ഒരേയൊരു വസ്തു പരമപ്രകാശം മാത്രം. സർവ്വശക്തനും സർവ്വവ്യാപിയും ആയി വിളങ്ങുന്ന ആ സത്ത എല്ലാറ്റിന്റേയും എല്ലാമാണ്‌.. ശാന്തമായി കർമ്മരഹിതമെന്നപോലെ അതു നിലകൊള്ളുന്നു.” 'മഹാരാജാവായ' അദ്ദേഹത്തിന്റെ ആജ്ഞയ്ക്കനുസരിച്ച് മന്ത്രിയാണ്‌ കാര്യങ്ങളെല്ലാം നടത്തുന്നത്. ഇല്ലാത്തതിനെ അദ്ദേഹം ഉണ്ടാക്കുന്നു. ഉള്ളതിനെ മാറ്റിമറിക്കുനു. ഈ മന്ത്രിയാകട്ടെ യതൊന്നും അനുഭവിക്കാനോ ആസ്വദിക്കാനോ കഴിയുന്നവനല്ല. അയാൾക്ക് ഒന്നും അറിയുകയുമില്ല.

അജ്ഞനും ജഢനുമാണെങ്കിലും തന്റെ രാജാവിനുവെണ്ടി ഈ മന്ത്രിപുംഗവൻ എല്ലാക്കാര്യങ്ങളും ചെയ്യും. രാജാവാകട്ടെ ഏകാന്തതയിൽ പ്രശാന്തനായി കഴിയുന്നു.

ബലി ചോദിച്ചു: അച്ഛാ, ശാരീരികവും മാനസീകവുമായ പീഢകൾ ഒന്നുമേൽക്കാത്ത ആ മണ്ഡലം ഏതാണ്‌? ആരൊക്കെയാണീ മന്ത്രിയും രാജാവും? ഇക്കഥ വിചിത്രമായിരിക്കുന്നു, ഞാനിതുവരെ കേട്ടിട്ടുമില്ലിത്. ഇതിനെക്കുറിച്ച് വിശദമായി പറഞ്ഞു തന്നാലും.

വിരോചനൻ പറഞ്ഞു: എല്ലാ ദേവന്മാരും അസുരന്മാരും ചേർന്നൊരു പടയൊരുക്കിയാലും ഈ മന്ത്രിയെ വെല്ലാനാവില്ല. അതു ദേവരാജാവായ ഇന്ദ്രനല്ല. മരണദേവനായ യമനല്ല. സമ്പത്തിന്റെ അധിദേവനായ കുബേരനുമല്ല. നിനക്ക് വേഗത്തിൽ കീഴടക്കാനാവുന്ന ഒരു ദേവനോ അസുരനോ അല്ല അത്. ഭഗവാൻ വിഷ്ണു അസുരന്മാരെ കൊന്നൊടുക്കിയതായി കേട്ടിട്ടുണ്ടല്ലോ?. വാസ്തവത്തിൽ ഈ മന്ത്രിയാണതെല്ലാം ചെയ്തത്. വിഷ്ണുഭഗവാൻ പോലും ഈ മന്ത്രിയുടെ വലയിൽ വീണ്‌ ജന്മങ്ങളെടുക്കേണ്ടി വനിട്ടുണ്ട്. കാമദേവനു ശക്തി നല്‍കുന്നതിദ്ദേഹമാണ്‌. ക്രോധത്തിന്റെ ശക്തിസ്രോതസ്സും ഇദ്ദേഹമാണ്‌..

ഈ മന്ത്രിയുടെ ഇച്ഛയ്ക്കൊത്താണ്‌ നന്മതിന്മകളുടെ പോരാട്ടങ്ങൾ നടക്കുന്നത്. ഇദ്ദേഹത്തെ ജയിക്കാൻ രാജാവിനല്ലാതെ മറ്റാർക്കും കഴിയില്ല. കാലക്രമത്തിൽ രാജാവിന്റെയുള്ളിൽ മന്ത്രിയെ തോല്‍പ്പിക്കണമെന്നു തോന്നിയാൽ അതു ക്ഷിപ്രസാധ്യമാണ്‌.. പക്ഷേ ത്രിലോകങ്ങളിൽ ഏറ്റവും ശക്തിമാനാണിദ്ദേഹം. വാസ്തവത്തിൽ ഈ ത്രിലോകങ്ങൾ അദ്ദേഹത്തിന്റെ ഉഛ്വാസം മാത്രമാണ്‌.. നിനക്കദ്ദേഹത്തെ വെല്ലാൻ കഴിയുമെങ്കിൽ നീ തികച്ചുമൊരു വീരനായകൻ തന്നെ. ഈ മന്ത്രി ഉണരുമ്പോൾ മൂന്നു ലോകങ്ങളും പ്രകടമാവുന്നു. താമര വിടരുന്നത് സൂര്യനുദിക്കുമ്പോഴാണല്ലോ. അദ്ദേഹമുറങ്ങുമ്പോൾ ത്രിലോകങ്ങളും നിദ്രാവസ്ഥയെ പുൽകുന്നു.

നിന്റെ മനസ്സ് ഏകാഗ്രമാക്കി എല്ലാ വിഭ്രാന്തികളിൽ നിന്നും വിട്ടകന്ന് അജ്ഞാനലേശമില്ലാതെ നിനക്കദ്ദേഹത്തെ കീഴടക്കാമെങ്കിൽ നീയൊരു വീരൻ തന്നെ. അദ്ദേഹം കീഴടങ്ങിയാൽപ്പിന്നെ മൂന്നുലോകങ്ങളും അവയിലുള്ളതുമെല്ലാം നിനക്കു സ്വന്തം. അയാളെ കീഴടക്കിയില്ലെങ്കിൽ നിനക്ക് ഒന്നിനേയും വെല്ലാൻ കഴിയില്ല. അദ്ദേഹത്തെ നിന്റെ വരുതിയിലാക്കിയാലല്ലാതെ ഈ ലോകമോ മറ്റെന്തൊക്കെയോ  നേടിയെന്നു നീ കരുതിയാലും അവയ്ക്കൊന്നും യാതൊരു മൂല്യവുമില്ല.

അതുകൊണ്ട് മകനേ, പരിപൂർണ്ണതയും ശാശ്വതാനന്ദവും വേണമെന്നുണ്ടെങ്കിൽ നിന്റെ കഴിവിന്റെ പരമാവധി ഉപയോഗിച്ച് എല്ലാ തടസ്സങ്ങളേയും തരണം ചെയ്ത് ഈ മന്ത്രിയെ കീഴടക്കി നിന്റെ വരുതിയിലാക്കിയാലും.
