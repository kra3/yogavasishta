\newpage
\secstar{പ്രാര്‍ത്ഥന}

ഓം തത്‌ സത്‌
ഓം നമ: ശിവാനന്ദായ
ഓം നമോ നാരായണായ
ഓം നമോ വെങ്കിടേശായ

യത: സര്‍വ്വാണി ഭൂതാനി പ്രതിഭാന്തി സ്ഥിതാനി ച
യത്രൈവോപശമം യാന്തി തസ്മൈ സത്യാത്മനേ നമ: (1)
ജ്ഞാതാ ജ്ഞാനം തഥാ ജ്ഞേയം ദൃഷ്ടാ ദര്‍ശന ദൃശ്യഭു:
കര്‍ത്താ ഹേതു: ക്രിയാ യസ്മാത്‌ തസ്മൈ ജ്ഞപ്ത്യാത്മനേ നമ: (2)
സ്പുരന്തി സീകരാ യസ്മാദ്‌ അനന്ദാസ്യാംബരേ വനൌ
സര്‍വേശാം ജീവനം തസ്മൈ ബ്രഹ്മാനന്ദാത്മനേ നമ: (3)

എല്ലാ സജീവ-നിര്‍ജ്ജീവ ജാലങ്ങള്‍ക്കും പ്രഭയേകി അവയ്ക്ക്‌ സ്വതന്ത്രമായ ഒരസ്തിത്വമുണ്ടെന്നപോലെ  കുറച്ചുകാലം നിലനില്‍ക്കാന്‍ ഇടയാക്കി അവസാനം തിരിയെ സ്വത്വത്തിലേയ്ക്ക്‌ നിര്‍ലീനമാക്കിച്ചേര്‍ക്കുന്ന ആ ഉണ്മയ്ക്ക്‌ നമോവാകം. വ്യതിരിക്തമായി കാണപ്പെടുന്ന ത്രിപുടികള്‍ (അറിയുന്നയാള്‍ , അറിയപ്പെടുന്ന വസ്തു, അറിവ്‌; കാണുന്നയാള്‍ , കാഴ്ച്ച, കാണല്‍ ; കര്‍ത്താവ്‌ , കര്‍മ്മം, ക്രിയ) ഏതൊന്നിന്റെ പ്രഭാവത്താല്‍ ഉദ്ഭൂതമാവുന്നുവോ ആ പരമബോധത്തിനു നമോവാകം. ഏതൊരാനന്ദവാരിധിയില്‍ നിന്നും തെറിച്ചുവീഴുന്ന ആനന്ദകണമാണ്‌ ജീവജാലങ്ങളുടെ ആഹ്ലാദത്തിനും ആത്മവികാസത്തിനും ഹേതുവായത്‌, ആ പരമാനന്ദബോധത്തിനു സമസ്കാരം.

ഓം തത്‌ സത്‌
ഓം നമ: ശിവാനന്ദായ
ഓം നമോ നാരായണായ
ഓം നമോ വെങ്കിടേശായ
