 
\section{ദിവസം 044}

\slokam{
പൂർണ്ണാത് പൂർണ്ണം പ്രസരതി സംസ്ഥിതം പൂർണ്ണമേവ തത്\\
അതോ വിശ്വമനുത്പന്നം യച്ചോത്പന്നം തദേവ തത് (3/10/29)\\
}

രാമന്‍ പറഞ്ഞു: മഹാത്മന്‍ , അതിനെ എങ്ങിനെയാണ്‌ ശൂന്യമല്ലെന്നും, പ്രകാശമാനമല്ലെന്നും, ഇരുട്ടല്ലെന്നും മറ്റും പറയാന്‍ കഴിയുക? അങ്ങു പറയുന്ന പരസ്പരവിരുദ്ധമായ കാര്യങ്ങള്‍ എന്നില്‍ ചിന്താക്കുഴപ്പമുണ്ടാക്കുന്നു.

വസിഷ്ഠന്‍ പറഞ്ഞു: രാമ: നീയിപ്പോള്‍ പക്വതയില്ലാത്തവര്‍ ചോദിക്കുന്ന തരം ചോദ്യമാണു ചോദിച്ചത്‌. എങ്കിലും ഞാന്‍ അതിനു ശരിയായ ഉത്തരം വിശദീകരിച്ചുതന്നെ നല്‍കാം. ഇതുവരെ കൊത്തിയെടുത്തിട്ടില്ലെങ്കില്‍പ്പോലും ഒരു ശിലയില്‍ ശില്‍പ്പം എന്നും ഉണ്ടായിരുന്നു. അതുപോലെ ലോകമെന്നത്‌ നീ സത്തായോ അസത്തായോ കണക്കാക്കിയാലുമില്ലെങ്കിലും അതു പരബ്രഹ്മത്തില്‍ സഹജമായി ലീനമത്രേ. അതുകൊണ്ട്‌ അതിനെ 'ഇല്ലാത്തത്‌' എന്നു പറയാന്‍ കഴിയില്ല. ശാന്തസമുദ്രത്തില്‍ തിരകള്‍ കാണുന്നില്ല എന്നതുകൊണ്ട്‌ അവ ഇല്ല എന്നു പറയാന്‍ കഴിയാത്തതുപോലെ പരബ്രഹ്മം ലോകശൂന്യം എന്നു പറയുക വയ്യ. ഈ ഉദാഹരണങ്ങള്‍ക്ക്‌ പരിമിതമായ ഉപയോഗമേയുള്ളു എന്നറിയുക. 

സത്യത്തില്‍ ഈ ലോകം പരബ്രഹ്മത്തില്‍ നിന്നും ഉദ്ഭൂതമായി തിരികെ അതില്‍ ത്തന്നെ വിലീനമാവുന്നു എന്നുപറഞ്ഞാല്‍ അതും ശരിയല്ല. കാരണം പരബ്രഹ്മം മാത്രമേ എക്കാലവും നിലനില്‍ക്കുന്നുള്ളു. ഒരുവന്‍ അതിനെ ശൂന്യമെന്നു ചിന്തിക്കുന്നത്‌ അത്‌ ശൂന്യമല്ല എന്ന ധാരണ അവനുള്ളതുകൊണ്ടാണ്‌. അതുപോലെ അതിനെ ശൂന്യമായ ഒന്നല്ല എന്നൊരുവന്‍ ചിന്തിക്കുന്നത്‌ അതു ശൂന്യമാണെന്നൊരു ധാരണ അവനുള്ളതുകൊണ്ടാണ്‌. പരം പൊരുളിനെ ഭാസുരമാക്കാന്‍ സൂര്യപ്രഭപോലെയുള്ള വസ്തുശ്രോതസ്സുകള്‍ക്കൊന്നും കഴിയില്ല. കാരണം പരം പൊരുള്‍ അവസ്തുവത്രേ. അതു സ്വയം പ്രഭമാണ്‌. അതിനാല്‍ അത്‌ ജഢമോ ഇരുട്ടോ അല്ല. പരം പൊരുളിനെ സാക്ഷാത്കരിക്കാന്‍ 'മറ്റൊരു' വസ്തുവിനാവില്ല. അതിനു സ്വയം സാക്ഷാത്കരിക്കാനേ കഴിയൂ. അനന്തമായ അവബോധം അനന്തമായ അകാശത്തേക്കാള്‍ പവിത്രം. ലോകം ഈ അനന്തതപോലെയത്രേ. 

കപ്പല്‍ മുളകിന്റെ സ്വാദുനോക്കാത്തവന്‌ അതറിയാത്തപോലെ വസ്തുനിഷ്ഠമായ പഠനം കൂടാതെ ഒരുവന്‌ അനന്തതാവബോധം അനുഭവവേദ്യമാവുകയില്ല. അങ്ങിനെ ഈ ബോധം ജഢമായും ചേതനാഹീനമായും തോന്നുന്നു. ലോകങ്ങളും അതുപോലെയത്രേ. പ്രത്യക്ഷമായ സമുദ്രത്തില്‍ അലകള്‍ പ്രകടമെന്നതുപോലെ രൂപഭാവരഹിതമായ ബ്രഹ്മത്തില്‍ ലോകങ്ങളും രൂപഭാവങ്ങളില്ലാതെ നിലകൊള്ളുന്നു.

"അനന്തതയില്‍ നിന്നും അനന്തത ഉയര്‍ന്നുദ്ഭവിച്ചു അനന്തമായിത്തന്നെ നിലനില്‍ക്കുന്നു. അതിനാല്‍ ലോകം സൃഷ്ടിക്കപ്പെട്ടിട്ടില്ല. അത്‌ എന്തില്‍നിന്നുദ്ഭവിച്ചുവോ അതായിത്തന്നെ നിലനില്‍ക്കുകയാണ്‌." (പൂര്‍ണ്ണത്തില്‍ നിന്നും പൂര്‍ണ്ണം). ആശയങ്ങളാകുന്ന ഇന്ധനം മനസ്സില്‍ ചിന്തകളുണ്ടാക്കുന്നു. അവയെ പിന്‍വലിച്ചുകഴിഞ്ഞാല്‍ 'ആത്മാവ്‌' എന്ന ധാരണയ്ക്ക്‌ അന്ത്യമായി. അപ്പോള്‍ എന്തുണ്ടോ അതാണ്‌ അനന്തം. ഏതാണോ ഉറക്കമോ ജഢമോ അല്ലാത്തത്‌ അതാണനന്തം. ഈ അനന്തതയിലാണ്‌ അറിവ്‌, അറിയുന്നവന്‍ , അറിയപ്പെടുന്നവസ്തു എന്നീ ത്രിപുടികള്‍ ധിഷണയുടെ അഭാവത്തില്‍ ഒന്നായി നിലനില്‍ക്കുന്നത്‌.
