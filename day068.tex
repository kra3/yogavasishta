\newpage
\section{ദിവസം 068}

\slokam{
പശ്യസീവൈതദഖിലം ന ച പശ്യസി കിഞ്ചന\\
സർവാത്മകതയാ നിത്യം പ്രകചസ്യാത്മനാത്മനി (3/41/55)\\
}

വസിഷ്ഠന്‍ തുടര്‍ന്നു: പൂര്‍ണ്ണചന്ദ്രന്മാരേപ്പോലെ പ്രഭാവതികളായ ലീലയും സരസ്വതീ ദേവിയും രാജാവിന്റെ പള്ളിയറയില്‍ കടന്നു. കാവല്‍ക്കാര്‍ ഉറക്കമായിരുന്നു. അവര്‍ വന്നിരുന്നപ്പോഴേയ്ക്കും രാജാവ്‌ ഉറക്കമുണര്‍ന്നു. അദ്ദേഹം എഴുന്നേറ്റിരുന്ന് അവരുടെ താമരപ്പാദങ്ങള്‍ പൂക്കള്‍കൊണ്ട്‌ അര്‍ച്ചിച്ചു പൂജ ചെയ്തു. രാജാവിന്റെ പൈതൃകത്തെപ്പറ്റി ലീലയെ മനസ്സിലാക്കാന്‍ മന്ത്രിയോടാവശ്യപ്പെടണമെന്ന് സരസ്വതി ചിന്തിച്ചമാത്രയില്‍ മന്ത്രി ഉറക്കമേണീറ്റു വന്നു. രാജാവിന്റെ പൈതൃകത്തെപറ്റി ആരായവേ മന്ത്രി പറഞ്ഞു: ഇഷ്വാകു വംശത്തിലെ നഭോരഥന്‍ എന്നുപേരായ ഒരു രാജാവ്‌ തന്റെ മകന്‌ പത്തു വയസ്സുള്ളപ്പോള്‍ രാജ്യം അവനെ ഏല്‍പ്പിച്ച്‌ വാനപ്രസ്ഥത്തിനു പോയി. ആ കുമാരനാണ്‌ ഈ രാജാവ്‌ - വിദുരഥന്‍. ദേവി രാജാവിന്റെ മൂര്‍ദ്ധാവില്‍ കൈവച്ച്‌ അനുഗ്രഹിക്കവേ അദ്ദേഹത്തിന്‌ തന്റെ പഴയ ജന്മങ്ങളിലെ കാര്യങ്ങള്‍ ഓര്‍മ്മയുണ്ടായി. 

എല്ലാം ഓര്‍മ്മയില്‍ തെളിഞ്ഞ രാജാവ്‌ സരസ്വതിയോട്‌ ചോദിച്ചു: എന്താണു ദേവീ ഇങ്ങിനെ? ഞാന്‍ മരിച്ചിട്ട്‌ ഒരുദിനമാവുന്നതല്ലേയുള്ളു? എന്നാല്‍ എന്റെ ഈ ശരീരം എഴുപതുകൊല്ലം ജീവിച്ചു. എന്റെ യൌവ്വനകാല സംഭവങ്ങളെല്ലാം എനിക്കു നല്ല ഓര്‍മ്മയുമുണ്ട്‌.

ദേവി പറഞ്ഞു: അല്ലയോ രാജാവേ, അങ്ങ്‌ മരിച്ച അതേ നിമിഷത്തില്‍ , അതേയിടത്ത്‌ അങ്ങ്‌ കാണുന്നതെല്ലാം മൂര്‍ത്തീകരിച്ചു. ഇവിടെയാണ്‌ ദിവ്യപുരുഷനായ വസിഷ്ഠന്‍, ആ മലമുകളിലെ ഗ്രാമത്തില്‍ ജീവിച്ചിരുന്നത്‌. അദ്ദേഹത്തിന്റെ ലോകത്തിലാണ്‌ പദ്മ രാജാവുണ്ടായിരുന്നത്‌. ആ രാജാവിന്റെ ലോകത്താണ്‌ താങ്കള്‍ ഇപ്പോള്‍ ഉള്ളത്‌. ആ സങ്കല്‍പ്പലോകത്തിരുന്നുകൊണ്ട്‌ 'ഇതെല്ലാം എന്റെ ബന്ധുജനങ്ങളാണ്‌, ഇതെല്ലാം എന്റെ പ്രജകളാണ്‌, ഇതെന്റെ മന്ത്രിമാരാണ്‌, ഇവരെന്റെ ശത്രുക്കളാണ്‌' എന്നെല്ലാം അങ്ങു വിചാരിക്കുന്നു. താങ്കളാണു ഭരിക്കുന്നതെന്നും താങ്കള്‍ യാഗാദികള്‍ ചെയ്യുന്നുവെന്നും അഭിമാനിക്കുന്നു. ശത്രുക്കളോട്‌ യുദ്ധം ചെയ്യുന്നുവെന്നും ചിലപ്പോള്‍ താങ്കള്‍ യുദ്ധത്തില്‍ തോല്‍വിയടയുന്നുവെന്നും വിശ്വസിക്കുന്നു. ഞങ്ങളെ ഇപ്പോള്‍ കാണുന്നുവെന്നും പൂജിക്കുന്നുവെന്നും ഞങ്ങളില്‍നിന്നും കാര്യങ്ങള്‍ മനസ്സിലാക്കുന്നുവെന്നും താങ്കള്‍ കരുതുന്നു. 'ഞാന്‍ ദു:ഖനിവൃത്തനായി, ഞാന്‍ പരമാനന്ദം അനുഭവിക്കുന്നു, പരമാര്‍ത്ഥപ്രകാശത്തില്‍ ഞാന്‍ ദൃഢപ്രജ്ഞനായിരിക്കുന്നു' എന്നെല്ലാം താങ്കള്‍ ചിന്തിക്കുന്നു. ഇതെല്ലാം സംഭവിക്കുന്നതിന്‌ സമയമൊന്നും എടുക്കുന്നില്ല. ഒരു ജീവിതകാലം മുഴുവനും, സ്വപ്നത്തില്‍ ഒരു ക്ഷണത്തിലാണല്ലോ അവതരിക്കപ്പെടുന്നത്‌. സത്യത്തില്‍ നീ ജനിക്കുന്നില്ല, മരിക്കുന്നുമില്ല.

"നീ ഇതെല്ലാം കാണുന്നു; എന്നാല്‍ കാണുന്നുമില്ല. ഇതെല്ലാം അനന്തബോധം മാത്രമാകുമ്പോള്‍ ആര്‌ എന്തു കാണാന്‍?"

വിദുരഥന്‍ ചോദിച്ചു: അപ്പോള്‍ എന്റെയീ മന്ത്രിമാര്‍ സ്വതന്ത്ര വ്യക്തികള്‍ അല്ലെന്നുണ്ടോ?

പ്രബുദ്ധനായ ഒരുവന്റെ ദൃഷ്ടിയില്‍ ഒരേയൊരു അനന്താവബോധം മാത്രമേയുള്ളു. 'ഞാന്‍', 'ഇത്‌', തുടങ്ങിയ ധാരണ   കള്‍ ഇല്ല.

വസിഷ്ഠമുനി ഇത്രയും പറഞ്ഞു നിര്‍ത്തവേ മറ്റൊരു ദിനം കൂടി അവസാനിച്ചു.

