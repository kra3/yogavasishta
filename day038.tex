\newpage
\section{ദിവസം 038}

\slokam{
ആതിവാഹികമേവാന്തർവിസ്മൃത്യാ ദൃഢരൂപയാ\\
ആധിഭൗതിക ബോധേന മുദ്ധാ ഭാതി പിശാചവത് (3/3/22)\\
}

വസിഷ്ഠന്‍ തുടര്‍ന്നു: സൃഷ്ടാവില്‍ ഓര്‍മ്മകള്‍ ഒന്നും അവശേഷിക്കുന്നില്ല, കാരണം അദ്ദേഹത്തിന്‌ 'കര്‍മ്മം' ഒന്നും ഉണ്ടായിരുന്നില്ല. അദ്ദേഹത്തിന്‌ ഭൌതീകമായി ഒരു ശരീരം പോലും ഉണ്ടായിരുന്നില്ല. അതൊരു ആത്മീയവസ്തു മാത്രമായിരുന്നു. മര്‍ത്ത്യന്‌ (മൃത്യുവശഗന്‌) രണ്ടു ശരീരങ്ങളാണ്‌- ഭൌതീകശരീരവും ആത്മീയശരീരവും. ഇനിയും 'ജനിച്ചിട്ടില്ലാത്ത്‌' സൃഷ്ടാവിന്‌ ആത്മീയശരീരം മാത്രമേയുള്ളു എന്നതുകൊണ്ട്‌ ഭൌതീകശരീരമുണ്ടാവാനുള്ള കാരണങ്ങള്‍ ഒന്നും അവനില്‍ ഇല്ല. ഇനിയും സൃഷ്ടിക്കപ്പെട്ടിട്ടില്ലാത്ത അവനാണ്‌ സര്‍വ്വ ജീവജാലങ്ങളുടേയും സൃഷ്ടാവ്‌. 

തീര്‍ച്ചയായും സ്വര്‍ണ്ണാഭരണങ്ങളുടെ സ്വത്ത്വം സ്വര്‍ണ്ണമാണെന്നതുപോലെ സൃഷ്ടിക്കപ്പെട്ടവയുടെയെല്ലാം സത്ത സൃഷ്ടാവിന്റേതു തന്നെയത്രേ. സ്വയം ശരീരമില്ലാതെ വെറും ചിന്തകള്‍ കൊണ്ട്‌ ഈ വിവിധങ്ങളായ സൃഷ്ടികളെ സൃഷ്ടിച്ചുവെങ്കില്‍ ഈ സൃഷ്ടികളും ചിന്തകളുടെ രൂപത്തിലാവണമല്ലോ. അവയില്‍ ഭൌതികവസ്തു ഉണ്ടാവുക വയ്യ. സൃഷ്ടാവില്‍ ഉണ്ടായ ഒരു സ്പന്ദനം ചിന്തകളായി വിശ്വമാകെ വികസിച്ചു. ഈ ബോധസ്വരൂപമായ സ്പന്ദനമാണ്‌ ജീവികളിലെ സൂക്ഷ്മശരീരം. ചിന്താനിര്‍മ്മിതികളായതുകൊണ്ട്‌ ഇവകള്‍ക്കെല്ലാം പ്രകടിതമായ ഒരു  താല്‍ക്കാലിക നിലനില്‍പ്പ്‌ മാത്രമേയുള്ളു. പക്ഷേ അവയിലെല്ലാം സ്വയം സത്താണെന്നുള്ള ചിന്തയുറച്ചിരിക്കുന്നു. ഈ പ്രകടിതാവസ്ഥ ഭാവനയില്‍ മാത്രമാണെങ്കില്‍ ക്കൂടി അവ അനുഭവപ്രതികരണങ്ങള്‍ക്കു കാരണമായി. സ്വപ്നത്തിലെ ലൈംഗീകാസ്വാദനം പോലെ ഭാവനാസൃഷ്ടിയുടെ പരിണിതഫലമാണിത്‌.

ഇക്കഥയിലെ മഹാത്മാവ്‌ - സൃഷ്ടാവ്‌- സ്വയം ശരീരമില്ലെങ്കിലും ശരീരമുള്ളതുപോലെ കാണപ്പെടുകയാണ്‌. ഇദ്ദേഹത്തിനും രണ്ടു പ്രകൃതികളുണ്ട്‌. ഒന്ന് ബോധസ്വരൂപം, രണ്ട്‌ ചിന്താസ്വരൂപം. ബോധം നിര്‍മ്മലം; ചിന്തകളോ അവ്യക്തം. അതുകൊണ്ട്‌ അയാള്‍ ജീവാത്മാവായി ഉയിര്‍ക്കുന്നതായി കാണപ്പെടുന്നു. ലോകത്തെ മുഴുവന്‍ നയിക്കുന്ന ബോധത്തില്‍ - വിശ്വബോധത്തില്‍ ഉദിച്ചുയരുന്ന ഓരോ ചിന്തകളും ഓരോ സൃഷ്ടികളാവുന്നു.

"ഈ സൃഷ്ടിക്കപ്പെട്ട രൂപങ്ങളെല്ലാം ശുദ്ധബോധസ്വരൂപമാണെങ്കിലും സ്വരൂപത്തെ മറന്നതിനാലും ഭൌതീകശരീരത്തെപ്പറ്റി ചിന്തിച്ചതിനാലും അവര്‍ അതതു ശരീരരൂപഭാവങ്ങളില്‍ മൂര്‍ത്തീകരിക്കുന്നു. കാണുന്നവന്റെ വിഭ്രാന്തിക്കനുസരിച്ച്‌ ശരീരരഹിതരായ ഭൂതപിശാചുക്കള്‍ക്ക്‌ ഒരുവന്റെ മനസ്സില്‍ രൂപമുണ്ടാവുന്നതുപോലെയത്രെ ഇത്‌."

എന്നാല്‍ സൃഷ്ടാവിന്‌ അത്തരം വിഭ്രാന്തികളൊന്നുമില്ല. അദ്ദേഹത്തിന്റെ അസ്തിത്വം ആത്മനിഷ്ഠമാണ്‌; വസ്തുനിഷ്ഠമല്ല. സൃഷ്ടാവിന്റെ സത്ത ആത്മനിഷ്ഠമാണെങ്കില്‍ സൃഷ്ടികള്‍ക്കും അതപ്രകാരമായിരിക്കണം. സൃഷ്ടിക്ക്‌ കാരണമേതുമില്ല. പരമാത്മാവായ ബ്രഹ്മത്തിനെന്ന പോലെ സൃഷ്ടിയുടേയും സത്ത ആത്മാവത്രേ. ആകാശത്തിലെ കോട്ടപോലെ, സ്വമനസ്സില്‍ , സങ്കല്‍പ്പിച്ചു, കെട്ടിപ്പൊക്കുന്ന ഭാവനകളാണ്‌ സൃഷ്ടികളായി സാക്ഷാത്കരിക്കപ്പെടുന്നത്‌. 

സൃഷ്ടാവ്‌ മനസ്സാണ്‌; ശുദ്ധബോധമാണദ്ദേഹത്തിന്റെ ശരീരം. മനസ്സിനു മനനം സഹജം. കാണുന്നവനില്‍ കാണപ്പെടുന്ന വസ്തു അന്തര്‍ലീനം. ഇവകള്‍ തമ്മിലുള്ള അന്തരം ആരെങ്കിലും കണ്ടുപിടിച്ചിട്ടുണ്ടോ?
