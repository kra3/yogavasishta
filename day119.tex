 
\section{ദിവസം 119}

\slokam{
മന ഏവ വിചാരേണ മന്യേ വിലയമേഷ്യതി\\
മനോവിലയമാത്രേണ തത: ശ്രേയോ ഭവിഷ്യതി (3/97/10)\\
}

വസിഷ്ഠന്‍ തുടര്‍ന്നു: മനസ്സുണ്ടെന്നുള്ള ദൃഢബുദ്ധിയാല്‍ മറയ്ക്കപ്പെട്ട ബോധപ്രകാശമാണ്‌ മനസ്സ്‌. . ഈ മനസ്സ്‌, മനുഷ്യന്‍, ദൈവങ്ങള്‍ , രാക്ഷസര്‍ , ഉപദേവതമാര്‍ , സ്വര്‍ഗ്ഗവാസികള്‍ എന്നിങ്ങനെ വൈവിധ്യമാര്‍ന്ന നാമരൂപങ്ങള്‍ സ്വീകരിക്കുന്നു. പിന്നീടത്‌ പല സ്വഭാവസവിശേഷതകളായും പട്ടണങ്ങളായും നഗരങ്ങളായും വികസ്വരമാവുന്നു. സത്യം അങ്ങിനെയായിരിക്കുമ്പോള്‍ ബാഹ്യമായി സത്യത്തെ തേടുന്നതില്‍ എന്താണര്‍ത്ഥം? മനസ്സിനെ മാത്രം നിരീക്ഷിച്ചാല്‍ മതിയല്ലോ. മനസ്സിനെ ആഴത്തിലറിയുമ്പോള്‍ സൃഷ്ടിക്കപ്പെട്ട എല്ലാ വസ്തുക്കളും, പ്രകടമായ എല്ലാ കാഴ്ചകളും മനസ്സുണ്ടാക്കിയതാണെന്നറിയുന്നു. അനന്താവബോധം മാത്രമേ മനസ്സിന്റെ സൃഷ്ടിയല്ലാതെയുള്ളു.

"ആഴത്തില്‍ നിരീക്ഷിക്കുമ്പോള്‍ മനസ്സ്‌ അതിന്റെ അടിസ്ഥാനത്തിലേയ്ക്ക്‌ ഉള്‍വലിയുന്നു. അങ്ങിനെ പൂര്‍ണ്ണമായും മഗ്നമായ മനസ്സില്‍ പരമസുഖം കളിയാടുന്നു." മനസ്സങ്ങിനെ ഇല്ലാതാവുമ്പോള്‍ മുക്തിയായി. പുനര്‍ജന്മങ്ങള്‍ ഇനിയില്ല. മനസ്സുതന്നെയാണല്ലോ ജനനമരണങ്ങളായി പ്രകടമായിക്കൊണ്ടിരുന്നത്‌!.

രാമന്‍ വീണ്ടും ചോദിച്ചു: ഭഗവാനേ അനന്താവബോധത്തില്‍ ഇതെല്ലാം എങ്ങി നെ സംഭവിച്ചു? എങ്ങിനെയാണ്‌ സത്തും അസത്തും ചേര്‍ന്നൊരു മിശ്രിതമായ മനസ്സ്‌ ബോധമണ്ഡലത്തില്‍നിന്നും ഉദ്ഭൂതമാവാനിടയായത്‌?

വസിഷ്ഠന്‍ പറഞ്ഞു: ആകാശം മൂന്നുവിധത്തിലാണ്‌. അവിച്ഛിന്ന ആകാശം, അനന്താവബോധം; പരിമിതമായ ആകാശം, ഭിന്നബോധം; പിന്നെ പദാര്‍ത്ഥങ്ങളുടെ ലോകമായ ഭൌതീക ആകാശം. ആദ്യത്തേത്‌ 'ചിദാകാശം'. അത്‌ അകത്തും പുറത്തും, സത്തിനും ജഢത്തിനും, ശുദ്ധ സാക്ഷിയായി സ്ഥിതിചെയ്യുന്നു. രണ്ടാമത്തേത്‌ 'ചിത്താകാശം'. പരിമിതമായ ആകാശമാണിത്‌.. അത്‌ സമയമെന്ന പരിധിയുണ്ടാക്കുന്നതും സര്‍വ്വവ്യാപിയും ജീവജാലങ്ങളുടെ ഗുണകാംഷിയുമത്രേ. മൂന്നാമത്തെ ആകാശമായ ഭൌതീകാകാശത്ത്‌ വായുമുതലായ മൂല ഘടകങ്ങള്‍ നിലനില്‍ക്കുന്നു. രണ്ടാമത്തേതും മൂന്നാമതേതും ഒന്നാമത്തെ ആകാശത്തില്‍നിന്ന്, അതായത് അനന്താവബോധത്തില്‍ നിന്ന് സ്വതന്ത്രമല്ല. വാസ്തവത്തില്‍ രണ്ടും മൂന്നും ആകാശങ്ങള്‍ ഇല്ലതന്നെ. അജ്ഞാനികള്‍ക്ക്‌ വിവേകമുദിക്കാനുള്ള മാര്‍ഗ്ഗമെന്നതില്‍ക്കവിഞ്ഞ്‌ ഈ 'ത്രിവിധ ആകാശം' എന്ന ധാരണയ്ക്ക്‌ യാതൊരു പ്രസക്തിയുമില്ല. പ്രബുദ്ധരായവര്‍ക്കറിയാം ഒരേയൊരുണ്മയേ ഉള്ളു എന്ന്. അത്‌ അനന്താവബോധമാണ്‌.. ആ ബോധം 'ഞാന്‍ ബുദ്ധിമാന്‍' എന്നോ; ഞാന്‍ ജഢം' എന്നോ ചിന്തിക്കുമ്പോള്‍ അതാണ്‌ മനസ്സ്‌.. ഈ തെറ്റിദ്ധാരണ മൂലമാണ് ഭൌതീകവും മാനസീകവുമായ എല്ലാ ഘടകങ്ങളും കല്‍പ്പനാസൃഷ്ടമായി പ്രകടമായത്‌.

