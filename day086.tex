 
\section{ദിവസം 086}

\slokam{
സമസ്ഥാ: സമതൈവാന്താ: സംവിദോ ബുദ്ധ്യതേ യത:\\
സര്‍വഥാ സര്‍വദാ സര്‍വം സര്‍വാത്മകമജസ്തത: (3,61,2)\\
}

രാമന്‍ ചോദിച്ചു: മഹാത്മന്‍, കാരണമൊന്നും കൂടാതെ, ഈ 'ഞാന്‍', 'ലോകം', എന്നീ ഭ്രമധാരണകള്‍ എങ്ങിനെയുണ്ടായിയെന്ന് ഒരിക്കല്‍ക്കൂടി ചെറുതായൊന്നു വിവരിച്ചു തന്നാലും.

വസിഷ്ഠന്‍ പറഞ്ഞു: "എല്ലാ വസ്തുക്കളുടേയും (ജീവജാലങ്ങളടക്കം) അന്തര്യാമിയായ പ്രജ്ഞ ഏവരിലും ഒരേരീതിയില്‍ കുടികൊള്ളുന്നു. അതുകൊണ്ട്‌ എല്ലായ്പ്പോഴും അസ്പഷ്ടമായ അതുതന്നെയാണ്‌ എല്ലാറ്റിന്റേയും ആത്മസത്ത." എല്ലാ വസ്തുക്കളും എന്ന് ആലങ്കാരികമായി പറഞ്ഞുവെന്നേയുള്ളു. വാസ്തവത്തില്‍ ബ്രഹ്മം മാത്രമേ ഉള്ളു. അത്‌ അനന്തമായ ബോധമാണ്‌. സ്വര്‍ണ്ണത്തിനും സ്വര്‍ണ്ണാഭരണത്തിനും തമ്മില്‍ വ്യത്യാസമില്ലാത്തതുപോലെ, ജലത്തിനും അലകള്‍ ക്കും തമ്മില്‍ വ്യത്യാസമില്ലാത്തതുപോലെ, വസ്തുപ്രപഞ്ചവും അനന്താവബോധവും വിഭിന്നങ്ങളല്ല. ഒരുമനുഷ്യനും അവന്റെ അവയവങ്ങളും തമ്മില്‍ വ്യത്യാസമില്ലെന്നു പറയുമ്പോലെ എല്ലാജീവജാലങ്ങളേയും അനന്തബോധത്തിന്റെ സാന്നിദ്ധ്യം എന്നു വിളിക്കാം. അവ തമ്മില്‍ അന്തരമില്ല എന്നര്‍ത്ഥം. 

അനന്തതയെ സ്വയം തിരിച്ചറിയാതിരിക്കുന്ന അവസ്ഥ അനന്തബോധത്തില്‍ സഹജമായി വരുമ്പോള്‍ അത്‌ 'ഞാന്‍' ആയും 'ലോകം' ആയും പ്രകടമാവുന്നതായി തോന്നുന്നു. ഒരു വെണ്ണക്കല്ലില്‍ ഇനിയുംകൊത്തിയിട്ടില്ലാത്ത ശില്‍പ്പത്തിനേപ്പോലെ അനന്താവബോധത്തില്‍ 'ഞാനും' 'ലോകവും' നിര്‍ലീനമാണ്‌. പ്രശാന്തമായ സമുദ്രത്തിലും തിരകള്‍ ഒരു സാദ്ധ്യതാ സാന്നിദ്ധ്യമായിനിലകൊള്ളുന്നപോലെ ലോകം ഒരു സാദ്ധ്യതയായി അനന്തതയില്‍ എപ്പോഴുമുണ്ട്‌. അതിനെയാണ്‌ സൃഷ്ടിയെന്നു പറയുന്നത്‌. 'സൃഷ്ടി' എന്ന പദത്തിന്‌ മറ്റ്‌ യാതൊരര്‍ത്ഥവും കല്‍പ്പിക്കേണ്ടതില്ല. പരം പൊരുളില്‍ സൃഷ്ടിയൊന്നും സംഭവിക്കുന്നില്ല; അനന്താവബോധം സൃഷ്ടികര്‍മ്മത്തില്‍ പങ്കാളിയുമല്ല. അവ തമ്മില്‍ വിഭജിച്ചുള്ള ഒരു നിലനില്‍പ്പുമില്ല. തമ്മില്‍ വ്യത്യാസമൊന്നുമില്ലെങ്കിലും, അനന്തബോധം, സ്വപ്രജ്ഞയെ സ്വന്തം ഹൃദയത്തില്‍ സ്വാംശീകരിക്കുന്നു; കാറ്റും അതിന്റെ ഗതിവിഗതിയും തമ്മില്‍ അന്തരമേതുമില്ലാത്തതുപോലെയത്രേ ഇത്‌. 

യാഥാര്‍ത്ഥ്യമല്ലാത്ത ആ 'വിഭജനം' നടക്കുന്ന ക്ഷണത്തില്‍ത്തന്നെ, ബോധത്തില്‍ 'ആകാശം' എന്ന ധാരണ ഉദിക്കുന്നു. അത്‌ ബോധത്തിന്റെ തന്നെ ബലം കൊണ്ട്‌ സൂക്ഷ്മാകാശം (വിയത്ത്‌) എന്ന മൂല ഘടകമായി പ്രകടമാവുന്നു. അതു പിന്നീട്‌ സ്വയം വായുവായും അഗ്നിയായും ഭാവിക്കുന്നു. ഈ ധാരണയില്‍നിന്നാണ്‌ അഗ്നി, ജ്വാലയായും വെളിച്ചമായും പ്രത്യക്ഷമാവുന്നത്‌. അതുതന്നെയാണ്‌ പിന്നീട്‌ സഹജസ്വഭാവമായ സ്വാദോടുകൂടിയ ജലമായും ദൃഢതയും ഗന്ധവും സഹജമായ ഭൂമിയായും ഭാവന ചെയ്യുന്നത്‌. അങ്ങിനെ ഭാവനയിലെ ജലതത്വവും ഭൂ തത്വവും അതതു മൂലഘടകങ്ങളായി പ്രകടിതമാവുന്നു.

