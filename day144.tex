\newpage
\section{ദിവസം 144}

\slokam{
അപരിജ്ഞായമാനൈഷാ മഹാമോഹപ്രദായിനീ\\
പരിജ്ഞാതാ ത്വനന്താഖ്യാ സുഖദാ ബ്രഹ്മദായിനീ (3/122/29)\\
}

വസിഷ്ഠൻ തുടർന്നു: മഹാത്മാക്കളുമായുള്ള സത്സംഗംകൊണ്ടു മാത്രമേ ഇക്കാണപ്പെടുന്ന അന്തമില്ലാത്ത അവിദ്യാപ്രവാഹത്തെ  മറികടക്കുവാനാകൂ. അത്തരം സത്സംഗങ്ങളിൽ നിന്നും ഏതു വസ്തുവാണ്‌ അന്വേഷണത്തിനനുയോജ്യമായത് എന്നും ഏതൊക്കെയാണ്‌ വർജ്ജിക്കേണ്ടതെന്നുമുള്ള വിജ്ഞാനവിവേകം നമുക്ക് ലഭിക്കും. അങ്ങിനെ മുക്തിപദപ്രാപ്തിക്കായുള്ള ശുദ്ധമായ ആഗ്രഹം നമ്മിലുദിക്കും. ഇത് ഗൗരവതരമായ അന്വേഷണത്തിലേയ്ക്ക് വഴിതെളിക്കും. അപ്പോൾ മനസ്സ് സൂക്ഷ്മമാവും, കാരണം മനോപാധികൾക്ക് അന്വേഷണം മൂലം ക്ഷയം സംഭവിക്കുമല്ലോ. അങ്ങിനെ ശുദ്ധജ്ഞാനത്തിൽ ജീവന്റെ ബോധമണ്ഡലം സത്തിലേയ്ക്കുന്മുഖമാവുന്നു. അപ്പോൾ മനോപാധികളെല്ലാമൊഴിഞ്ഞ് അനാസക്തി സഹജമാവുന്നു. കർമ്മങ്ങളോടുള്ള ബന്ധവും കർമ്മഫലങ്ങളോടുള്ള ആശകളും ആശങ്കകളും നിലയ്ക്കുന്നു.

സത്യത്തിൽ ദൃഢീകരിച്ച ദർശനം, തനിക്ക് അയഥാർത്ഥ്യമായതിനോടുണ്ടെന്നു തെറ്റിദ്ധരിച്ചിരുന്ന ബന്ധത്തെ ക്ഷയിപ്പിക്കുന്നു. ഇഹലോകത്തിൽ ജീവിച്ച് പ്രവർത്തിക്കുമ്പോഴും അപരിമേയമായ ദർശനമുള്ളവൻ ചെയ്യേണ്ട കാര്യങ്ങളെല്ലാം ഭംഗിയായിത്തന്നെ നിർവ്വഹിക്കുന്നു. ഉറക്കത്തിലെ കർമ്മങ്ങളിൽ (ശരീര ധർമ്മങ്ങളുടെ നടത്തിപ്പ്) എന്നപോലെ ഈ കർമ്മങ്ങൾ അനുഷ്ഠിക്കുമ്പോൾ അവൻ ലോകത്തെപ്പറ്റിയോ അതിലെ സുഖാനുഭവങ്ങളെപ്പറ്റിയോ വ്യകുലപ്പെടുന്നില്ല. ഇങ്ങിനെ കുറേക്കഴിഞ്ഞാൽ അവൻ എല്ലാ അവസ്ഥകൾക്കുമതീതമായ മുക്തിപദം പ്രാപിക്കുന്നു. അവൻ ജീവൻ മുക്തനത്രേ.

ജീവിച്ചിരിക്കേ മുക്തനായ ഒരുവൻ ലാഭങ്ങളിൽ സന്തോഷവാനോ നഷ്ടങ്ങളിൽ ദു:ഖിതനോ അല്ല. രാമ: നിന്നിലും മനോപാധികൾ നിന്നെ പരിക്ഷീണനാക്കിയിരിക്കുന്നു. സത്യത്തെ അറിയാനായി പരിശ്രമിക്കൂ. ആത്മാവിനെ, അനന്താവബോധത്തെ അറിയുന്നതിലൂടെ നിനക്ക് ദുരിതങ്ങളുടെ, മോഹത്തിന്റെ, ജനനമരണങ്ങളുടെ, സുഖദു:ഖങ്ങളുടെ അപ്പുറം പോകാൻ കഴിയും.

ആത്മാവ് ഏകവും അവിച്ഛിന്നവുമാകയാൽ നിനക്ക് ബന്ധുക്കളാരുമില്ല. അതുകൊണ്ടുതന്നെ അയാഥാർഥ്യമായ ബന്ധുത്വം മൂലമുള്ള ദു:ഖങ്ങളുമില്ല. അത്മാവ് ഏകവും അദ്വിതീയവുമാകയാൽ മറ്റൊന്നും ആഗ്രഹിക്കാനോ കിട്ടാനോ ഇല്ല. അത്മാവിന്‌ മാറ്റങ്ങളില്ല, മരണമില്ല. മൺ കുടമുടയുമ്പോൾ കുടത്തിനകത്തെ ആകാശം ഉടയുന്നില്ല. മനോപാധികളെ മറികടന്ന്‌ മന:ശ്ശാന്തി കൈവരുമ്പോൾ അജ്ഞാനിയെ മോഹിപ്പിക്കുന്ന മായക്കാഴ്ചകൾക്ക് അവസാനമായി.

ഈ മായയെ ശരിയായി മനസ്സിലാക്കാതിരിക്കുമ്പോൾ മാത്രമാണ്‌ അത് വലിയ വലിയ വിഭ്രാന്തികൾക്ക് കാരണമാവുന്നത്. എന്നാൽ മായയെ ശരിയായി അറിഞ്ഞാൽ, അത് അനന്തമാണെന്നുറച്ചാൽ, പിന്നെ ആഹ്ളാദത്തിനും പരബ്രഹ്മ സാക്ഷാത്കാരത്തിനും അത് കാരണമാവുകയും ചെയ്യും. വേദപഠനത്തിനായി മാത്രമാണ്‌ ആത്മാവ്, ബ്രഹ്മം എന്നൊക്കെ വാക്കുകളുണ്ടാക്കിപ്പറയുന്നത്. വാസ്തവത്തിൽ ഒരേയൊരു സത്യവസ്തുവായ ആ 'ഒന്നു' മാത്രമാണുണ്മ. അതു ശുദ്ധ അവബോധമത്രേ. ശരീരഭാവത്തിലുള്ള ഒന്നല്ല. അറിഞ്ഞാലുമില്ലെങ്കിലും, ശരീരമെടുത്താലുമില്ലെങ്കിലും, അതാണുണ്മ.

നാം കാണുന്ന ദുരിതാനുഭവങ്ങളെല്ലാം ശരീരത്തിനുമാത്രമുള്ളതാണ്‌..  അത്മാവാകട്ടെ ഇന്ദ്രിയസംവേദനങ്ങളുടെ പരിധിയിലും പരിമിതിയിലും പെടാത്തതുകൊണ്ട് അതിനെ ദു:ഖം ബാധിക്കയില്ല. അത്മാവിൽ ആഗ്രഹങ്ങൾ ഒന്നുമില്ല. ലോകം അതിൽ പ്രതിഫലിച്ചുകാണുന്നത് ആഗ്രഹംകൊണ്ടോ അത്മാവ് അതിനായി ഇച്ഛിച്ചതുകൊണ്ടോ അല്ല.

രാമ: എന്റെ പ്രഭാഷണംകൊണ്ട് നിന്നിൽ സൃഷ്ടിസ്ഥിതികളെക്കുറിച്ചുണ്ടായിരുന്ന തെറ്റിധാരണകൾ നീങ്ങിയിരിക്കുന്നു. നിന്റെ ബോധമണ്ഡലം ദ്വൈതഭാവം വെടിഞ്ഞ് നിർമ്മലമായിരിക്കുന്നു.

(ഉത്പത്തി പ്രകരണം എന്ന മൂന്നാം ഭാഗം അവസാനിച്ചു)

