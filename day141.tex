 
\section{ദിവസം 141}

\slokam{
ഇത്യേവം രാഘവാവിദ്യാ മഹതി ഭ്രമദായനി\\
അസത്സത്താം നയത്യാശു സച്ചാസത്താം നയത്യലം (3/121/10)\\
}

വസിഷ്ഠൻ തുടർന്നു: തനിക്ക് മോഹവിഭ്രാന്തിയുണ്ടായതിന്റെ പിറ്റേന്ന് ലവണ രാജാവ് ഇങ്ങിനെ ചിന്തിച്ചു: “ഞാൻ ഇന്നലെയാ മോഹദർശനത്തിൽ കണ്ട സ്ഥലങ്ങളെല്ലാം എനിക്കൊന്നു സന്ദർശിക്കണം. അവ യഥാർത്ഥത്തിൽ ഉണ്ടെങ്കിലോ?” ഉടനേ തന്നെ പരിവാരസമേതം രാജാവ് തെക്കുദേശത്തേക്ക് യാത്ര പുറപ്പെട്ടു. താമസംവിനാ അദ്ദേഹം തന്റെ മോഹദർശനത്തിൽ കണ്ടതരം ആളുകളേയും ദേശവും കണ്ടു. ഗോത്രവർഗ്ഗക്കാരനായിരുന്നപ്പോൾ കണ്ടുമുട്ടിയ അതേ ആൾക്കാരെ അദ്ദേഹമവിടെ കണ്ടു. തന്റെ അഗതികളായ മക്കളും അവിടെയുണ്ടായിരുന്നു. അവിടെ ഒരു വൃദ്ധ ദു:ഖിച്ചു വിലപിക്കുന്നു: എന്റെ പ്രിയതമാ, ഞങ്ങളെ ഇവിടെയിട്ട് അങ്ങ് പോയതെങ്ങോട്ടാണ്‌? എന്റെ മകൾ ഭാഗ്യശാലിയായതുകൊണ്ട് സുന്ദരനും വീരനുമായൊരു രാജാവിനെ അവൾക്ക് ഭർത്താവായി ലഭിച്ചു. അവളും ഇപ്പോഴെനിക്കു നഷ്ടമായിരിക്കുന്നു. അവരെല്ലാം എവിടെപ്പോയി? എല്ലാവരും  എന്നെ വിട്ടു പോയിരിക്കുന്നു!.

രാജാവ് അവരെ സമീപിച്ച് ആശ്വാസവചനങ്ങൾ പറഞ്ഞു. അവർ തന്റെ ഗോത്രവർഗ്ഗഭാര്യയുടെ മാതാവാണെന്നു തിരിച്ചറിഞ്ഞു. താൻ തലേന്ന് ദർശനത്തിൽ കണ്ടിട്ടുള്ളതും ഇപ്പോൾ അവർ അനുഭവിക്കുന്നതുമായ വരൾച്ചയിൽ കഷ്ടപ്പെടുന്ന അവർക്കുവേണ്ടി രാജാവ് ദയാപൂർവ്വം ആവശ്യമുള്ള ധനം നൽ കി. അവരുമൊത്ത് കുറച്ചു സമയം ചിലവഴിച്ചശേഷം രാജാവ് കൊട്ടാരത്തിൽ തിരിച്ചെത്തി. പിറ്റേന്ന് രാജാവ് എന്നെ വിളിച്ച് ഇതിന്റെയെല്ലാം അർത്ഥം മനസ്സിലാക്കിക്കൊടുക്കാൻ ആവശ്യപ്പെട്ടു. എന്റെ വിശദീകരണത്തിൽ രാജാവ് സന്തുഷ്ടനായി. രാമാ, സത്തിനേയും അസത്തിനേയും തമ്മിൽ വേർതിരിക്കാനാവാത്തവണ്ണം സങ്കീർണ്ണമാക്കാൻ അവിദ്യയ്ക്കുള്ള ശക്തി അദമ്യമാണ്‌. .

രാമൻ ചോദിച്ചു: മഹർഷേ തികച്ചും വിസ്മയകരമാണത്. സ്വപ്നത്തിൽ അല്ലെങ്കിൽ വിഭ്രാന്തിയിൽ കണ്ട കാര്യങ്ങൾ ജാഗ്രദാവസ്ഥയിൽ യാഥാർത്ഥ്യമായി കാണാൻ സാധിക്കുന്നതെങ്ങിനെ?

വസിഷ്ഠൻ പറഞ്ഞു: രാമാ, ഇതെല്ലാം അവിദ്യയാണ്‌.. അകലെ, അരികെ, ക്ഷണനേരം, യുഗപര്യന്തം, ഭ്രമകൽപ്പനകൾ, എല്ലാം - അവിദ്യയുടെ പ്രഭാവത്താൽ സത്തായത് അസത്തായും, അസത്തായത് സത്തായും കാണപ്പെടുന്നു. മനോപാധികൾ ഉള്ളതിനാൽ ജീവബോധത്തിൽ ഏതേതു ചിന്തകളുണ്ടാവുന്നുവോ അവ ധാരണകളാവുന്നു. അജ്ഞാനംമൂലം അഹംകാരഭാവം ഉദിക്കുമ്പോൾ ആദിമദ്ധ്യാന്തങ്ങളുള്ള ഒരു മോഹവിഭ്രാന്തികൂടി അവിടെ ഉദയം ചെയ്യുന്നു. അങ്ങിനെ മോഹത്തിനടിമയായവൻ സ്വയം ഒരു മൃഗമെന്നുകരുതി ഇതെല്ലാം അനുഭവിക്കുന്നു.

ഇതെല്ലാം സംഭവിക്കുന്നത് തികച്ചും ആകസ്മികമായാണ്‌.. കാക്കയും പനമ്പഴവും പോലെ. കാക്ക ഒരു മരക്കൊമ്പിൽ പറന്നുവന്നിരിക്കുമ്പോഴേക്കും പഴം വീഴുന്നത് കാക്കയുടെ ചെയ്തിയൊന്നും അല്ല. അങ്ങിനെ കാണുന്നവർക്കു തോന്നുമെങ്കിലും. അപ്രകാരം വെറും ആകസ്മികതകൊണ്ടു മാത്രം അയാഥാർത്ഥ്യമായത് യാഥാർത്ഥ്യമായി തോന്നുകയാണ്‌.

