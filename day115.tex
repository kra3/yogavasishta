 
\section{ദിവസം 115}

\slokam{
ഏഷ ജഗജ്ജാംഗല ജീര്‍ണവല്ലീ\\
സമ്യക്‌സമാലോക കുഠാരകൃത്താ\\
വല്ലീവ വിക്ഷുബ്ധ മന:ശരീരാ\\
ഭൂയോ ന സംരോഹതി രാമഭദ്ര (3/93/24)\\
}

വസിഷ്ഠന്‍ തുടര്‍ന്നു: ഇത്‌ പണ്ടുകാലത്ത്‌ ബ്രഹ്മാവെനിക്കു പറഞ്ഞു തന്നതാണ്‌. രാമ: അതു ഞാന്‍ നിനക്കായി പറഞ്ഞു തന്നു. പറഞ്ഞുവന്നത്,  പരബ്രഹ്മം അതിന്റെ അവിച്ഛിന്നമായ അവസ്ഥയില്‍ സര്‍വ്വവ്യാപിയായതിനാല്‍ എല്ലാം അതേ അവിച്ഛിന്നസ്ഥിതിയിലാണു നിലകൊള്ളുന്നത്‌. അത്‌ സ്വാഭീഷ്ടപ്രകാരം ഘനീഭവിക്കുമ്പോള്‍ വിശ്വമനസ്സുണ്ടാവുന്നു. ആ മനസ്സില്‍ വിവിധങ്ങളായ മൂലഘടകങ്ങള്‍ അവയുടെ അതിസൂക്ഷ്മഭാവത്തില്‍ നിലകൊള്ളുന്നതായി ഭാവനയുണ്ടാവുന്നു. ഇതിന്റെയെല്ലാം ആകെത്തുകയായ (സമഷ്ടി) പ്രഭാവാനായ വിശ്വപുരുഷനാണ്‌. ബ്രഹ്മാവ്‌. ബോധസ്വഭാവമായതിനാല്‍ സ്വയം എന്തൊക്കെ കാണാനിച്ഛിക്കുന്നുവോ അവ സ്വന്തം മനസ്സില്‍ ദര്‍ശിക്കുന്നു. ഈ ബ്രഹ്മാവാണ്‌ പ്രപഞ്ചത്തിലെ നാനാത്വഭാവത്തിനു കാരണമായ അവിദ്യയെ ഇച്ഛിച്ചുണ്ടാക്കിയത്‌.. അവിദ്യയാണ്‌ ആത്മ-അനാത്മ വസ്തുക്കള്‍ എന്ന തരംതിരിവിനു കാരണം. ഈ അവിദ്യയാലാണ്‌ ബ്രഹ്മാവ്‌ മലകളും പുല്‍ച്ചെടികളും ജലം മുതലായ പഞ്ചഭൂതങ്ങളും അടങ്ങുന്ന വിശ്വത്തെ നാനാത്വങ്ങളുടെ സംഘാതമായി പ്രത്യക്ഷപ്പെടുത്തുന്നത്‌. ഇതിനാലാണ്‌ ഈ സമഷ്ടിപ്രപഞ്ചമാകെ അനന്താവബോധമാണെന്ന സത്യം നിലനില്‍ക്കുമ്പോഴും മാത്രാ-തന്മാത്രാ അണുക്കളില്‍ നിന്നും ജീവജാലങ്ങള്‍ ഉദ്ഭൂതമാവുന്നതായി തോന്നുന്നത്‌..

അതുകൊണ്ട്‌ രാമാ, പ്രപഞ്ചത്തിലെ എല്ലാ വസ്തുക്കളും പദാര്‍ത്ഥങ്ങളും, സമുദ്രജലത്തില്‍ നിന്നും തിരമാലകളെന്നപോലെ, പരബ്രഹ്മത്തില്‍നിന്നും ഉണ്ടായതാണ്‌.. ഈ 'സൃഷ്ടിക്കപ്പെട്ടിട്ടില്ലാത്ത' വിശ്വത്തില്‍ , ബ്രഹ്മാവിന്റെ മനസ്സില്‍ സ്വയം അഹംകാരമുദിച്ച്‌ 'വിശ്വമനസ്സായ' ബ്രഹ്മാവ്‌ 'വിശ്വസൃഷ്ടാവായ' ബ്രഹ്മാവാകുന്നു. ആ വിശ്വമനസ്സിന്റെ ശക്തിയാണ്‌ നനാത്വം സഹജ സ്വഭാവമായ പ്രപഞ്ചമായിത്തീരുന്നത്‌.. അനന്തകോടി ജീവജാലങ്ങള്‍ ഈ വിശ്വമനസ്സില്‍ പ്രകടമായി, ജീവനുകളാവുന്നു. ഈ വിവിധ ജീവനുകള്‍ അനന്താകാശത്ത്‌ മൂലഘടകങ്ങളുടെ സംഘാതമായ ശരീരങ്ങളായി അതില്‍ ബോധം പ്രാണശക്തിയായി കടന്ന് ജീവ-നിര്‍ജീവ ജാലങ്ങളുടെ ശരീരങ്ങള്‍ക്ക്‌ ബീജമാവുന്നു. അങ്ങിനെ വ്യക്തിഗത ജീവാത്മാക്കള്‍ ജന്മമെടുക്കുന്നു. ഇവ ഓരോന്നും അകസ്മീകമായി (കാകതാലീയം എന്ന് ന്യായേന) വൈവിദ്ധ്യമാര്‍ന്ന സാദ്ധ്യതകള്‍ ആയിത്തീരുന്നു. ഈ സാധ്യതകളുടെ പ്രത്യക്ഷ പ്രകടനമാണ്‌ കാര്യ-കാരണ നിയമങ്ങളും കര്‍മ്മഫല സിദ്ധാന്തമനുസരിച്ചുള്ള ഉയര്‍ച്ച-താഴ്ച്ചകളും.

"രാമാ, അങ്ങിനെയൊക്കെയാണ്‌ ഈ പ്രത്യക്ഷലോകമെന്ന കാനനം. ആരൊരുവനാണോ ഈ വനത്തിന്റെ വേരുകളെ അന്വേഷണം എന്ന കോടാലികൊണ്ട്‌ വെട്ടി നീക്കുന്നത്‌, അവനതില്‍നിന്നു സ്വതന്ത്രനാണ്‌.". ചിലരില്‍ ഈ അറിവ്‌ വേഗത്തില്‍ ഉണരുന്നു; മറ്റുചിലരില്‍ ഈ അറിവുണ്ടാവാന്‍ വളരെയേറെ സമയമെടുത്തേക്കാം. 
