\section{ദിവസം 204}

\slokam{
കേചിത്വകർമണി രതാ വിരതാ അപി കർമണ:\\
നരകാന്നരകം യാന്തി ദു:ഖാദ്ദു:ഖം ഭയാദ്ഭയം (5/6/3)\\
}

വസിഷ്ഠൻ തുടർന്നു: സ്ഫടികത്തിനു സമീപം വെച്ചിട്ടുള്ള വസ്തുക്കൾ അതിൽ പ്രതിഫലിച്ചുകാണുന്നപോലെ ബോധത്തിന്റെ അസ്തിത്വം ഹേതുവായാണ്‌ എല്ലാ കർമ്മങ്ങളും സംഭവിക്കുന്നതെന്നറിയുന്നവൻ മുക്തനത്രേ. എന്നാൽ മനുഷ്യജന്മം നേടിയിട്ടും ഈദൃശകാര്യങ്ങളിൽ ശ്രദ്ധാലുക്കളല്ലാത്തവർ സ്വർഗ്ഗങ്ങളിൽനിന്നു നരകങ്ങളിലേയ്ക്കും തിരിച്ചും പൊയ്ക്കൊണ്ടിരിക്കുന്നു. “കർമ്മവിമുഖരായവരായി ചിലരുണ്ട്. അവർ കർമ്മങ്ങളെയെല്ലാം ബലമായടക്കി അല്ലെങ്കിൽ അതിൽനിന്നു പിന്തിരിഞ്ഞ് നരകങ്ങളിൽ നിന്നു നരകങ്ങളിലേയ്ക്ക്, ദു:ഖത്തിൽ നിന്നു ദു:ഖത്തിലേക്ക്, നരകത്തിൽനിന്നു നരകത്തിലേക്ക് പോയ്ക്കൊണ്ടേയിരിക്കുന്നു.”

ചിലർ അവരുടെ സ്വകർമ്മഫലങ്ങളുമായും തൽജന്യങ്ങളായ വാസനകളുമായും ഗാഢബന്ധത്തിലാണ്‌..  അവർ കൃമികീടങ്ങളായും പുഴുക്കളായും, ചെടികളും മരങ്ങളുമായും, വീണ്ടും കൃമികീടങ്ങളായും ജനനമരണങ്ങള്‍ ആവര്‍ത്തിച്ചു കൊണ്ടിരിക്കുന്നു.

ഇനിയും ചിലർ ആത്മജ്ഞാനമാർജ്ജിച്ചവരാണ്. അനുഗൃഹീതരാണവർ. അവർ മനസ്സിന്റെ സ്വഭാവങ്ങളെപ്പറ്റി ആരാഞ്ഞറിഞ്ഞ്, ആസക്തികളുപേക്ഷിച്ചവരത്രേ. അവർ ബോധമണ്ഡലത്തിന്റെ ഉയർന്ന തലങ്ങളിലേയ്ക്കു പോകുന്നു. ജനനമരണചക്രത്തിൽ അവസാനജന്മമെടുത്ത ജീവനിൽ ഏറിയ പങ്കും സത്വപ്രകാശവും ചെറിയൊരംശം രജസ്സും (മാലിന്യം) ഉണ്ടായിരിക്കും. ജനനസമയം മുതൽ അയാൾ പവിത്രത പ്രകടമാക്കും. ജ്ഞാനവിജ്ഞാനങ്ങൾ അയാൾക്ക് എളുപ്പം ഹൃദിസ്ഥമാവും. സൗഹൃദം, കാരുണ്യം, വിവേകം, നന്മ, മഹാമനസ്കത തുടങ്ങിയ ഗുണങ്ങൾ അയാളെത്തേടിയെത്തും. അയാൾ ഉചിതമായ  കർമ്മങ്ങളിലേർപ്പെടുന്നു, എന്നാൽ അവയുടെ ഫലങ്ങളെപ്പറ്റി അയാള്‍ ചിന്താകുലനല്ല. ലാഭനഷ്ടങ്ങൾ അയാൾക്ക് സന്തോഷസന്താപങ്ങൾക്കു കാരണമാവുന്നില്ല. അയാളുടെ ഹൃദയം നിർമ്മലം. അയാൾ എല്ലാവർക്കും അഭിമതൻ. സദ്സ്വഭാവനിരതനായ അയാൾ അനുയോജ്യനായ ഒരു ഗുരുവിനെ കണ്ടെത്തുന്നു. ഗുരുവിൽ നിന്നും ആത്മജ്ഞാനമാർഗ്ഗം അറിയുന്നു. അങ്ങിനെ അയാൾ ആത്മാവിനെ അനന്താവബോധമായി സാക്ഷാത്കരിക്കുന്നു; ഏകവും അദ്വയവുമായ വിശ്വസത്വമായി ലോകത്തെ അറിയുന്നു. അതുവരെ നിദ്രാവസ്ഥയിലിരുന്ന മേധാശക്തി അയാളിൽ ഉണർന്നുവിടരുകയായി. എല്ലാമെല്ലാം അനന്താവബോധം തന്നെയാണെന്ന് അയാളറിയുന്നു. ആന്തരീകമായ ഈ പ്രകാശശ്രോതസ്സുമായി നിരന്തരസമ്പർക്കത്തിലായതുകൊണ്ട് അയാൾ അതിനിർമ്മലമായ ഒരു പാവന തലത്തിൽ വിരാജിക്കുന്നു. ഇതാണ്‌ സാധാരണയായി കണ്ടുവരുന്ന പരിണാമപ്രക്രിയ. എന്നാൽ ഇതിനപവാദങ്ങളുമുണ്ട്.

ഈ ലോകത്തുജന്മമെടുത്തവരുടെ കാര്യത്തിൽ മുക്തിപദപ്രാപ്തിക്കായി രണ്ടുമാർഗ്ഗങ്ങളാണുള്ളത്.  ഒന്ന്, ഗുരുവിന്റെ പാതയെ ശ്രദ്ധയോടെ പിന്തുടരുക- ശിഷ്യൻ കാലക്രമത്തിൽ മുക്തിപദമെന്ന ലക്ഷ്യത്തിലെത്തിച്ചേരും. രണ്ടാമത്, ആത്മജ്ഞാനമാർഗ്ഗം. അക്ഷരാർത്ഥത്തിൽ ആരോ മടിയിലിട്ടു കൊടുത്തതുപോലെ ആത്മജ്ഞാനം ഒരുവനിൽ പൊടുന്നനേ അങ്കുരിക്കുന്നു. നിർവ്വാണപദത്തിലേയ്ക്ക് സാധകൻ താനേ വിലയനംചെയ്യപ്പെടുന്നു. രണ്ടാമത്തെ തരം പ്രബുദ്ധതയെക്കുറിക്കുന്ന പുരാതനമായ ഒരു കഥ ഞാൻ പറയാം. ശ്രദ്ധിച്ചു കേട്ടാലും.. 

