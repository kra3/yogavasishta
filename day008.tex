\newpage
\section{ദിവസം 008}

\begin{center}
ഭാരോവിവേകിന: ശാസ്ത്രം ഭാരോ ജ്ഞാനം ച രാഗിണ:\\
അശാന്തസ്യ മനോ ഭാരോ ഭാരോനാത്മവിദോ വപു:  (1/14/13)\\
\end{center}


രാമന്‍ തുടര്‍ന്നു: അജ്ഞാനിയെ മോഹിപ്പിക്കുന്ന സമ്പത്ത്‌ തികച്ചും ഉപയോഗശൂന്യമാണ്‌ മഹര്‍ഷേ. ധനം പലേവിധത്തിലുള്ള ദുരിതങ്ങള്‍ക്കും കാരണമാവുന്നു. മാത്രമല്ല അത്‌ ഒരിക്കലും അടങ്ങാത്ത ആര്‍ത്തിക്കു ഹേതുവുമായിത്തീരുന്നു. സദ്ഗുണസമ്പന്നര്‍ക്കും ദുഷ്ടജനത്തിനും സമ്പല്‍സമൃദ്ധി ഉണ്ടാവാം അതുകൊണ്ട്‌ സമ്പത്ത്‌ ഒരുവന്റെ മാന്യതയ്ക്ക്‌ നിദാനമൊന്നുമല്ല എന്നു തെളിഞ്ഞു. എന്തൊക്കെയായാലും ധനസമ്പാദനത്തിനുള്ള ഉള്‍ക്കടവാഞ്ഛയുടെ പിടിയില്‍ പ്പെട്ട്‌ ഹൃദയകാഠിന്യം വന്നാല്‍പ്പിന്നെ ആളുകളുടെ സദ്സ്വഭാവത്തിനു മാറ്റമുണ്ടാവുന്നുണ്ട്‌. സ്നേഹഭാവം അപ്രത്യക്ഷമാവുന്നു. ജ്ഞാനസമ്പന്നനായ വിദ്വാന്റെയും, കൃതജ്ഞന്റേയും, രണവീരന്റേയും മൃദുഭാഷിയുടേയും നൈപുണ്യമുള്ളവന്റേയും എന്നുവേണ്ട എല്ലാവരുടേയും ഹൃദയത്തെ ധനം കളങ്കപ്പെടുത്തുന്നു. സമ്പത്തും സന്തോഷവും ഒരുമിച്ചു നിലനില്‍ ക്കുക അസാദ്ധ്യം. ധനികന്‌ ശത്രുക്കളോ തനിക്കെതിരേ ഉപജാപം നടത്തുന്ന എതിരാളികളോ ഇല്ലാതിരിക്കുക എന്നത്‌ ദുര്‍ലഭം. 

സദ്കര്‍മ്മങ്ങളാകുന്ന താമരയ്ക്ക്‌ ധനം രാത്രിയാണ്‌. ദു:ഖമാവുന്ന വെള്ളത്താമരയ്ക്ക്‌ ധനം നിലാവാണ്‌. തെളിഞ്ഞ ഉള്‍ക്കാഴ്ച്ചയാകുന്ന ദീപനാളത്തിനത്‌ കാറ്റാണ്‌. ശത്രുതയാകുന്ന തിരകള്‍ക്കത്‌ ജലപ്രളയം. ചിന്താക്കുഴപ്പത്തെ വര്‍ദ്ധിപ്പിക്കുന്ന കാറ്റാണത്‌. വിഷാദമെന്ന വിഷത്തിനത്‌ ശക്തി കൂട്ടുന്നു. ധനം ദുഷ്ചിന്തകളാകുന്ന സര്‍പ്പമാകുന്നു. അത്‌ ഒരുവന്റെ ദുരിതത്തിനുമേല്‍ ഭയം കൂട്ടിച്ചേര്‍ക്കുന്നു. അനാസക്തിപ്രവണതയ്ക്കുമേല്‍ പ്പതിക്കുന്ന നാശോന്മുഖമായ മഞ്ഞാണത്‌. ദുഷ്ചിന്തകള്‍ക്കത്‌ ഊമനു (നത്ത്‌) രാത്രിപോലെ ഹിതം. ജ്ഞാനചന്ദ്രന്റെ ഗ്രഹണമാണത്‌. ധനത്തിന്റെ സാന്നിദ്ധ്യത്തില്‍ ഒരുവന്റെ സല്‍സ്വഭാവങ്ങള്‍ സങ്കുചിതമാവുന്നു. തീര്‍ച്ചയായും മരണം നേരത്തേതന്നെ നോട്ടമിട്ടു തിരഞ്ഞെടുത്തവനെയാണ്‌ ധനം തേടിത്തിരഞ്ഞെടുക്കുന്നത്‌. എങ്കിലും മഹാമുനേ ഈ ജീവിതകാലം എന്നത്‌ ഒരിലയില്‍ വീണ ജലകണത്തിന്റേതുപോലെ ക്ഷണികം. 


ആത്മജ്ഞാനിക്കുമാത്രമേ ഈ ജീവിതകാലം സഫലമാവുകയുള്ളു. നാം കാറ്റിനെ കീഴടക്കിയേക്കാം, ആകാശത്തെ ഛിന്നഭിന്നമാക്കിയേക്കാം, തിരമാലകള്‍കൊണ്ട്‌ മാലകള്‍ കൊരുത്തേക്കാം. എന്നാല്‍ നമ്മുടെ ആയുസ്സിനെപ്പറ്റി നമുക്ക്‌ യാതൊരുറപ്പും ഇല്ല. മനുഷ്യന്‍ വെറുതേ തന്റെ ആയുസ്സു നീട്ടിക്കിട്ടാന്‍ പരിശ്രമിക്കുന്നു. അതിലൂടെ കൂടുതല്‍ ദു:ഖങ്ങളെ ക്ഷണിച്ചുവരുത്തി സഹനത്തിന്റെ കാലയളവു കൂട്ടുന്നു.

ആത്മജ്ഞാനം ആര്‍ജ്ജിക്കാന്‍ പരിശ്രമിക്കുന്നവന്‍ മാത്രമേ 'ജീവിക്കുന്നുള്ളു'. അതുമാത്രമേ ഈ ജിവിതത്തില്‍ നേടാന്‍ കൊള്ളാവുന്നതായി ഉള്ളു. കാരണം അത്‌ ഭാവിയിലെ ജനനനമരണങ്ങളെ ഇല്ലായ്മ ചെയ്യുന്നു. മറ്റുള്ളവര്‍ കഴുതകളേപ്പോലെ ജീവിതം തുടരുന്നു. 

അജ്ഞാനിക്ക്‌ വേദഗ്രന്ഥങ്ങളിലെ അറിവ്‌ ഭാരം.
ആഗ്രഹങ്ങള്‍കൊണ്ടുള്ളു നിറഞ്ഞവന്‌ ജ്ഞാനം പോലും ഭാരം.
മന:ശാന്തിയില്ലാത്തവന്‌ സ്വന്തം മനസ്സ്‌ ഭാരം.
ആത്മജ്ഞാനമില്ലാത്തവന്‌ ഈ ശരീരവും ആയുസ്സും പോലും ഭാരം.

സമയം എന്ന എലി ആയുസ്സു കരണ്ടു തിന്നുകൊണ്ടേയിരിക്കുന്നു. എല്ലാ ജീവജാലങ്ങളുടേയും പ്രാണനെ അസുഖമാവുന്ന ചിതലരിക്കുന്നു. എലിയെപ്പിടിക്കാന്‍ ജാഗരൂകനായിരിക്കുന്ന ഒരു പൂച്ചയേപ്പോലെ മരണം ജീവന്റെ ആയുസ്സിനുമേല്‍ ദൃഷ്ടിപതിപ്പിച്ച്‌ എപ്പോഴും തയ്യാറായി കാത്തിരിക്കുന്നു. 
