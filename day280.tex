\section{ദിവസം 280}

\slokam{
പരമാത്മമണേശ്ചിത്വാധ്യദന്ത: കചനം സ്വയം\\
ചേതനാത്മപദേ ചാന്തരഹമിത്യാദി വേത്യസൌ    (5/57/15)\\
}

വസിഷ്ഠന്‍ തുടര്‍ന്നു: അല്ലയോ രാമാ, അനന്താവബോധം മുളകിന്റെ എരിവറിയുന്നു; അത് എല്ലാ വൈവിദ്ധ്യങ്ങളോടും കൂടിയ, കാലദേശനിബദ്ധമായ  അഹംബോധത്തിനു കാരണമാകുന്നു. അനന്താവബോധം ഉപ്പിന്റെ സ്വാദറിയുന്നു; അത് എല്ലാ വൈവിദ്ധ്യങ്ങളോടും കൂടിയ, കാലദേശാനുസാരിയെന്ന് തോന്നിപ്പിക്കുന്ന ‘അഹ’മാവുന്നു. അനന്താവബോധം കരിമ്പിന്‍തുണ്ടിലെ മാധുര്യമറിയുന്നു; അതിനതിന്റെ തനത് ഗുണങ്ങളെപ്പറ്റി അവബോധമുണ്ടാവുന്നു. 

അതുപോലെ അനന്താവബോധം അന്തര്യാമിയും സര്‍വ്വവ്യാപിയുമാകയാല്‍ അതിന് പാറക്കല്ലിന്റെ ഗുണം, മലയുടെ പ്രത്യേകത, മരത്തിന്റെ, ജലത്തിന്റെ, ആകാശത്തിന്റെ എല്ലാം പ്രത്യേകതകള്‍ ഭിന്നമായി തിരിച്ചറിയുന്നതോടെ ആത്മബോധം, അല്ലെങ്കില്‍ വ്യക്തിത്വം ഉരുവാകുന്നു. ബോധംതന്നെ അന്തര്യാമിയായ ആണവമൂലകങ്ങള്‍ സ്വാഭാവികമായി സങ്കലനത്തിലേര്‍പ്പെട്ട് സ്വയം ഉണ്ടാക്കുന്ന തടവറയുടെ  മതിലുകളെന്നപോലെ ഞാന്‍, നീ തുടങ്ങിയ തരംതിരിവുകള്‍ ഉണ്ടാക്കുന്നു. അവ ബോധത്തിനു ബാഹ്യമായ വസ്തുക്കളാണെന്ന ധാരണയും അവയില്‍ സംജാതമാവുന്നു. എന്നാല്‍ ഇതെല്ലാം ബോധത്തിന്റെ പ്രതിഫലനങ്ങള്‍ മാത്രമാണ്. വൈവിദ്ധ്യതയെപ്പറ്റി അവബോധമുണ്ടാവുമ്പോഴാണ് വ്യക്തിത്വമെന്ന ധാരണയുണ്ടാകുന്നത്. ബോധം സ്വയമറിയുന്നു. വസ്തു-അവബോധമെന്നത് ബോധം തന്നെയാണ്. അതുതന്നെയാണ് അഹംബോധം മുതലായ ധാരണകളെ ഉണ്ടാക്കുന്നതും.         

“അനന്താവബോധമെന്ന ഈ സ്ഫടികം സ്വന്തം പ്രകാശത്തെയാണ് പ്രതിഫലിപ്പിക്കുന്നത്. അത് എല്ലാ അണുസംഘാതങ്ങളിലും വസ്തുക്കളിലും കുടികൊള്ളുന്നു. ആ വസ്തുക്കള്‍ വ്യതിരിക്തമായ ആത്മബോധം ഉണ്ടെന്നമട്ടില്‍ ‘ഞാന്‍’ തുടങ്ങിയ ചിന്തകളെ ഉണ്ടാക്കുന്നു.” 

വാസ്തവത്തില്‍ അനന്താവബോധവും ഈ വസ്തുസംഘാതങ്ങളും തമ്മില്‍ വ്യത്യാസമേതുമില്ലാത്തതിനാല്‍ അവ തമ്മില്‍ വിഷയ-വിഷയീ ബന്ധത്തിനോ, ‘ഇത്-അത്‌’ തുടങ്ങിയ ഭേദചിന്തകള്‍ക്കോ യാതൊരു സാംഗത്യവുമില്ല. ഒന്ന് മറ്റൊന്നിനെ നേടുകയോ, വെല്ലുകയോ, മാറ്റിമറിക്കുകയോ ചെയ്യുന്നുമില്ല.  രാമാ, ഞാനീ പറഞ്ഞതെല്ലാം നിനക്ക് മനസ്സിലാക്കാനായി ഉണ്ടാക്കിയ വാക്കുകളുടെ വെറും കളിമാത്രം. സത്യത്തില്‍ ‘ഞാന്‍’, ‘ലോകം’ (അണുസംഘാതങ്ങള്‍)  എന്നിവ ഉള്ളതല്ല. മനസ്സോ, അറിവിനുള്ള വിഷയങ്ങളോ ലോകമെന്ന ഭ്രമക്കാഴ്ചയോ ഒന്നും ‘ഉള്ളതല്ല’.

നിയതമായ രൂപമൊന്നുമില്ലാത്ത ജലം ഒരുചുഴിയായി തല്‍ക്കാലത്തേയ്ക്ക് സ്വയമൊരു വ്യക്തിത്വം ആര്‍ജ്ജിക്കുന്നതുപോലെ ബോധം സ്വയമേവ ‘ഞാന്‍’ തുടങ്ങിയ പ്രകടനങ്ങള്‍ നടത്തുന്നു. എന്നാല്‍ ബോധം തന്നെ സ്വയം പരമശിവനെന്നു ചിന്തിച്ചാലും ചെറിയൊരു പരിമിതജീവനെന്നു കരുതിയാലും, അത് ബോധം മാത്രമാണ്. ‘ഞാന്‍’, ‘നീ’, ‘പദാര്‍ത്ഥങ്ങള്‍ ’,  തുടങ്ങിയ ഭിന്നരൂപങ്ങള്‍ അജ്ഞാനിയുടെ സംതൃപ്തിക്കായിമാത്രം ഉണ്ടായതാണ്. അജ്ഞാനി അനന്താവബോധത്തില്‍ സങ്കല്‍പ്പിക്കുന്നതെന്തോ അതാണവന് ദൃശ്യമാവുന്നത്. ജ്ഞാനത്തിന്റെ വെളിച്ചത്തില്‍ ജീവിതം എന്നത് ബോധമാണ്. എന്നാലതിനെ ജീവിതമായി കണ്ടാല്‍ അതിന് അങ്ങിനെയൊരു പരിമിത മൂല്യമേയുള്ളു. വാസ്തവത്തില്‍ ജീവിതവും ബോധവും വിഭിന്നമല്ല. അതുപോലെ വ്യക്തിജീവനും, സമഷ്ടിജീവനും, (ശിവന്‍) തമ്മിലും യാതൊരു ഭിന്നതയുമില്ല. എല്ലാം ഒരവിച്ഛിന്നഅനന്താവബോധമാണെന്നറിയുക.