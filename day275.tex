\section{ദിവസം 275}

\slokam{
അന്ത:കുണ്ഡലിനീം പ്രാണ: പൂരയാമാസുരാധൃതാ:\\
ചക്രാനുവര്‍ത്തപ്രസൃതാം പയാംസീവ സരിദ്വരാം (5/54/26)\\
}

വസിഷ്ഠന്‍ തുടര്‍ന്നു: എന്നിട്ട് മഹര്‍ഷി ഉദ്ദാലകന്‍ പത്മാസനത്തിലിരുന്നു. പാതി അടഞ്ഞ കണ്ണുകളോടെ ധ്യാനത്തിലാണ്ടു. അദ്ദേഹം പരമോന്നതാവസ്ഥയെ പ്രദാനം ചെയ്യുന്ന പ്രണവമന്ത്രമായ ഓംകാരം ജപിച്ചു. അദ്ദേഹത്തില്‍ മുഖരിതമായ ഓംകാരം ആ ദേഹത്തില്‍ ആപാദചൂഡം ചെറുപ്രമ്പനങ്ങളുണ്ടാക്കി. ആദ്യമായി അദ്ദേഹം ശ്വാസത്തെ പൂര്‍ണ്ണമായും ഉഛ്വസിച്ചു കളഞ്ഞു. പ്രാണശക്തി, ശരീരമുപേക്ഷിച്ച് ആകാശസീമകളില്‍ , അനന്താവബോധത്തില്‍ , ചുറ്റിത്തിരിയുന്നതു പോലെയായിരുന്നു അത്. ഹൃദയത്തില്‍ നിന്നുയര്‍ന്ന അഗ്നി അദ്ദേഹത്തിന്‍റെ ദേഹത്തെ ആകമാനം ചൂടാക്കി. ഹഠയോഗാഭ്യാസത്തിന്റെ ഭാഗമായി ഉദ്ദാലകന്‍ ഇതെല്ലാം നേരത്തെ പഠിച്ചിരുന്നു.

രണ്ടാമത്തെ ഓംകാരജപത്തോടെ അദ്ദേഹം പൂര്‍ണ്ണമായ സമസംതുലിതാവസ്ഥയെ പ്രാപിച്ചു. പെട്ടെന്ന് പ്രാണശക്തി പ്രകമ്പനമോ പ്രകോപനമോ ഇല്ലാതെ, ചലനമേതുമില്ലാതെ പിടിച്ചു നിന്നു. അകത്തും പുറത്തുമല്ലാതെ താഴെയോ മുകളിലോ അല്ലാതെ പ്രാണന്‍ അനങ്ങാതെ നിന്നിടത്തുറച്ചു നിന്നു. ശരീരം എരിഞ്ഞു ഭസ്മമായിക്കഴിഞ്ഞപ്പോള്‍ അഗ്നി സ്വയമടങ്ങി. ശുദ്ധശുഭ്രമായ ഭസ്മം മാത്രം അവശേഷിച്ചു കാണായി.  

അസ്ഥികള്‍ ഭംഗിയില്‍ സ്വയം കര്‍പ്പൂരമെന്നതുപോലെ ജാജ്വല്യമാനമായി എരിയുന്നുണ്ടായിരുന്നു.Iഅവശേഷിച്ച ചാരം പ്രബലമായ ഒരു കാറ്റില്‍ പറന്നുപോയി ആകാശത്തില്‍ വിലയിച്ചു. ഹഠയോഗത്തിന്റെ സഹജമായ വേദനകളൊന്നുമില്ലാതെയാണദ്ദേഹമിത് സാധിച്ചത്. മൂന്നാമത്തെ ഓംകാരം പ്രശാന്തതയുടെ ഉത്തുംഗസീമയില്‍ എത്തവേ പ്രാണവായു അകത്തേക്കെടുക്കാന്‍ സമയമായി. ഈയവസരത്തില്‍ ബോധമെന്ന അമൃതിന്റെ ഉള്ളിലായി നിലകൊണ്ടിരുന്ന പ്രാണന്‍ ശാന്തശീതളമായ മാരുതനായി ആകാശം നിറഞ്ഞു. അത് ചാന്ദ്രസീമയിലെത്തി. അവിടെ നിന്നത് പവിത്രരശ്മികളായി ശരീരത്തില്‍ ബാക്കിനിന്ന ഭസ്മത്തില്‍ വര്‍ഷിക്കപ്പെട്ടു. പൊടുന്നനെ ആ ചാരത്തില്‍ നിന്നും വിഷ്ണുരൂപത്തില്‍ നാല് കൈകളോടെ ഒരു ദിവ്യസത്വം ഉയര്‍ന്നു വന്നു. ഉദ്ദാലകന്‍ ദിവ്യതയുടെ മൂര്‍ത്തിമദ്ഭാവമായ ദേവനായി പരിണമിച്ചു.  

“കുണ്ഡലിനിയില്‍ നിറഞ്ഞു നിന്ന പ്രാണശക്തി സര്‍പ്പിളമായി എല്ലാടവും പരന്നു.” അങ്ങിനെ ഉദ്ദാലകന്റെ ശരീരം പരിപൂര്‍ണ്ണമായും നിര്‍മലതയെ പ്രാപിച്ചു. പിന്നീടദ്ദേഹം തന്റെ പത്മാസനത്തെ പൂര്‍വ്വാധികം ഉറപ്പിച്ചു നിര്‍ത്തി, ഇന്ദ്രിയങ്ങളെ മുറുകെ കെട്ടി, തന്റെ ബോധമണ്ഡലത്തെ ചിന്തകളുടെ കണിക പോലുമില്ലാതെയാക്കാനുള്ള ശ്രമം തുടങ്ങി. മനസ്സിനെ എല്ലാ ശക്തികളുമുപയോഗിച്ച് നിയന്ത്രിച്ചു. അപ്പോഴും അദ്ദേഹത്തിന്‍റെ അര്‍ദ്ധനിമീലിതങ്ങളായ നേത്രങ്ങള്‍ അനക്കമൊന്നുമില്ലാതെ നിശ്ചലവും പ്രശാന്തവുമായിരുന്നു. മനസ്സ് നിശ്ശബ്ദമായിരിക്കെ അദ്ദേഹം പ്രാണനെയും അപാനനെയും സംതുലമാക്കി നിലനിര്‍ത്തി.   

വിത്തില്‍നിന്നും എണ്ണയെടുക്കുന്നതുപോലെ അന്തരേന്ദ്രിയങ്ങളെ അവയുടെ സഹജമായ ബാഹ്യവസ്തുക്കളിലേയ്ക്ക് ആകര്‍ഷിക്കപ്പെടുന്നതില്‍ നിന്നും അദ്ദേഹം ശ്രദ്ധയോടെ പിന്‍വലിച്ചു. എന്നിട്ടദ്ദേഹം മുന്‍കാല അനുഭവങ്ങളുടെ ഫലമായുണ്ടായ മനോപാധികളെപ്പറ്റി മനപ്പൂര്‍വ്വം ബോധവാനായി. പിന്നീട് ഈ ഉപാധികളില്‍ നിന്നും തന്റെ അവബോധത്തെ സ്വതന്ത്രവും അങ്ങിനെ ശുദ്ധവുമാക്കി. അദ്ദേഹമെന്നിട്ട് തന്റെ ഗുദദ്വാരം തുടങ്ങിയ നവദ്വാരങ്ങളേയും അടച്ചുപിടിച്ചു. അങ്ങിനെ സമഗ്രമായ യോഗവിദ്യയിലൂടെ പ്രാണനെയും അവബോധത്തെയും ബാഹ്യാഭിമുഖമാവുന്നതില്‍ നിന്നും വിലക്കിയിട്ട് അദ്ദേഹം തന്റെ മനസ്സ് ഹൃദയത്തില്‍ അഭിരമിപ്പിച്ചു.