\section{ദിവസം 219}

\slokam{
ആത്മനോ ജഗതശ്ചാന്തർ ദൃഷ്ടൃദൃശ്യദശാന്തരേ\\
ദർശനാഖ്യേ സ്വമാത്മാനം സർവദാ ഭാവയൻഭവ (5/14/50)\\
}

വസിഷ്ഠൻ തുടർന്നു: ബുദ്ധിവൈകല്യമുള്ളവരെ ആത്മീയതയിലേയ്ക്കുണര്‍ത്താന്‍ ശ്രമിക്കുന്നവർ ഒരു ചെറിയ കുടകൊണ്ട് ആകാശത്തെ മറയ്ക്കാൻ ശ്രമിക്കുന്നവരാണ്‌.. കാരണം, മൃഗങ്ങളെപ്പോലെ ജീവിക്കുന്നവരെ ഉന്നതനിലവാരത്തിലുള്ള തത്വചിന്താപാഠങ്ങള്‍ മനസ്സിലാക്കുക അസാദ്ധ്യം. സ്വന്തം മനസ്സാകുന്ന കയറാൽ ബന്ധിക്കപ്പെട്ട് അവർ മേയ്ക്കപ്പെടുകയാണ്‌.. അവർ സ്വന്തം മനസ്സിന്റെ മായാബന്ധത്തിൽപ്പെട്ടുഴന്ന് അജ്ഞാനത്തിന്റെ ചെളിക്കുണ്ടിലേയ്ക്കാഴ്ന്നു പോകുന്നതു കണ്ടാൽ കല്ലിനുപോലും കണ്ണീരു വരും. അതുകൊണ്ട് ജ്ഞാനികൾ അങ്ങിനെയുള്ള, മനോനിയന്ത്രണമില്ലാതെ വ്യാകുലചിത്തരായവരെ ആത്മവിദ്യ പഠിപ്പിക്കാൻ തുനിയുകയില്ല.

എന്നാൽ മനോനിയന്ത്രണം വന്ന് ആത്മാന്വേഷണത്തിനായി പക്വതയാർജ്ജിച്ചവരുടെ ദു:ഖം നീക്കാൻ മഹാത്മാക്കൾ മുൻകയ്യെടുക്കുകയും ചെയ്യും. രാമാ, മനസ്സിന്‌ അസ്തിത്വമില്ല. ആവശ്യമില്ലാതെ അതിന്റെ അസ്തിത്വത്തെ സങ്കൽപ്പിച്ചുണ്ടാക്കാതിരിക്കൂ. കാരണം സങ്കൽപ്പിച്ചുണ്ടാക്കിയാല്‍ മനസ്സൊരു ഭൂതത്തെപ്പോലെയാണ്‌.. നമ്മെയത് നശിപ്പിച്ചുകളയും. ആത്മാവിനെ നാം മറന്നുവെന്നാൽ ഈ സാങ്കൽപ്പിക വസ്തു- മനസ്സുണ്ട് എന്നർത്ഥം. മനസ്സ് വലുതാവുന്നത് തുടർച്ചയായി അതിനെക്കുറിച്ചാലോചിച്ചുറപ്പിച്ചുകൊണ്ടിരിക്കുന്നതിനാലാണ്‌.. അത്തരം ചിന്തകളുപേക്ഷിക്കൂ.

നിന്നിൽ വസ്തുതാബോധം അങ്കുരിക്കുമ്പോൾ നിന്റെ ബോധമണ്ഡലം പരിമിതപ്പെടുന്നു. നീ ഉപാധികൾക്കടിമയാവുന്നു. അതാണു ബന്ധനം. ഈ ഉപാധികളാകുന്ന  പരിമിതികളില്ലാതാവുന്നതാണു മുക്തി. മനസ്സില്ലാത്ത അവസ്ഥ. പ്രകൃതിയുടെ ഗുണഗണങ്ങളുമായി സമ്പർക്കമുണ്ടാവുന്നത് ബന്ധനത്തിനു കാരണമാകുന്നു. ആ സമ്പർക്കമുപേക്ഷിച്ചാൽ അതു മുക്തിയിലേയ്ക്കാണു നയിക്കുക. ഇതറിഞ്ഞ് നിനക്കിഷ്ടമുള്ള പാത സ്വീകരിക്കാം.

‘ഞാൻ അല്ല’, ‘ഇതല്ല’ (നേതി, നേതി) എന്നിങ്ങനെ ഉറപ്പിച്ച് അനന്താകാശം പോലെ അചഞ്ചലനായി വർത്തിക്കുക. ആത്മാവ്, ലോകം എന്നിങ്ങനെ 'രണ്ടായി പിരിഞ്ഞിരിക്കുന്ന' ലോകത്തെ സൃഷ്ടിക്കുന്ന മലിനചിന്തകളെ ഉപേക്ഷിക്കൂ. “ആത്മാവെന്ന ദൃഷ്ടാവിനും ലോകമെന്ന ദൃശ്യത്തിനും മദ്ധ്യേ നീ ദൃഷ്ടിയായി നിലകൊള്ളുന്നു. ഈ ദർശനത്തിന്റെ സാക്ഷാത്കാരം നിന്നുള്ളിലെപ്പോഴും നിറഞ്ഞുവിളങ്ങട്ടെ”. അനുഭവിക്കുന്നയാളിനും അനുഭവത്തിനും ഇടയിൽ നീയാണ്‌ ‘അനുഭവിക്കൽ’ എന്ന പ്രതിഭാസം എന്നറിഞ്ഞ് ആ ആത്മജ്ഞാനത്തിൽ ഉറച്ചു നിൽക്കുക.

ഈ ആത്മാവിനെവിട്ട് നീയൊരു വസ്തുവിനെക്കുറിച്ചു (വിഷയം) ചിന്തിക്കുമ്പോൾ നീ മനസ്സാകുന്നു (വിഷയി). അങ്ങിനെ ദു:ഖത്തിനു പാത്രമാകുന്നു. ആത്മവിദ്യയിൽ നിന്നു വേറിട്ട ബുദ്ധിശക്തിയാണു മനസ്സ്. ദു:ഖത്തിന്റെ മൂലഹേതുവാണത്. ‘എല്ലാമെല്ലാം ആത്മാവുമാത്രം’ എന്ന സാക്ഷാത്കാരമുണ്ടായാൽപ്പിന്നെ മനസ്സില്ല. വിഷയമില്ല, വിഷയി ഇല്ല, ചിന്തകളുമില്ല. ‘ഞാൻ ജീവനാണ്‌’ എന്നു ചിന്തിച്ചാലോ ദു:ഖസങ്കുലമായ മനസ്സുണ്ടാവുന്നു.

‘ഞാൻ ആത്മാവാണ്‌; ജീവനോ അതുപോലെ മറ്റെന്തെങ്കിലും വസ്തുക്കളോ സത്യത്തിൽ ഇല്ല’ എന്ന അറിവുണ്ടായാൽപ്പിന്നെ മനസ്സു നിലച്ചു. പരമാനന്ദമാണു പിന്നെ. “ഈ പ്രപഞ്ചം മുഴുവനും ആത്മാവു മാത്രം‘ എന്ന അറിവിന്റെ വെളിച്ചത്തിൽ മനസ്സെന്ന ഇരുട്ടിനു നിലനിൽപ്പില്ല. മനസ്സെന്ന സർപ്പം ദേഹത്തു കുടിയിരിക്കുമ്പോൾ മാത്രമേ ഭയമുള്ളു. യോഗസാധനകൾ കൊണ്ട് മനസ്സിനെ നീക്കംചെയ്താൽപ്പിന്നെ ഭയത്തിനു സ്ഥനമെവിടെ? 


