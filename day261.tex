\section{ദിവസം 261}

\slokam{
ഏവം സ ശ്വപചോ രാജ്യം പ്രാപ കീരപുരാന്തരേ\\
ആരണ്യം ഹരിണം പുഷ്ടമപ്രാണമിവ വായസ:   (5/45/44)\\
}

വസിഷ്ഠന്‍ തുടര്‍ന്നു: ജലത്തില്‍ അപ്പോഴും മുങ്ങിയിരുന്ന ഗാധി താന്‍ ഭൂതമണ്ഡലം എന്ന ഒരിടത്തെ ഒരു ഗോത്രവനിതയുടെ ഗര്‍ഭത്തില്‍ ഭ്രൂണഭാവത്തില്‍ കിടക്കുന്നതായിക്കണ്ടു. ആ സ്ത്രീശരീരത്തിനകത്തെ ഗര്‍ഭത്തിനുള്ളില്‍ മാംസവും ചോരയും മറ്റു വൃത്തികേടുകളും  തന്നെ വലയം ചെയ്തിരിക്കുന്നതായി ഗാധിയറിഞ്ഞു. കാലക്രമത്തില്‍ ആ സ്ത്രീ ഒരു ബാലനെ പ്രസവിച്ചു. കുറച്ചുസമയം ആ കുട്ടി സ്വന്തം മലത്തില്‍ കിടന്നുരുണ്ടു! അച്ഛനമ്മമാരെപ്പോലെ കറുത്ത ദേഹം. കുട്ടി വീട്ടിലെ ഓമനയായി വളര്‍ന്നു. അങ്ങിനെ അയാളൊരു സുന്ദര തരുണനായിത്തീര്‍ന്നു. നല്ലൊരു വേട്ടക്കാരനായിരുന്നു അയാള്‍ . ഒരു ഗോത്രവര്‍ഗ്ഗ പെണ്‍കുട്ടിയെ അയാള്‍ വിവാഹവും ചെയ്തു. കാട്ടിലയാള്‍ സ്വതന്ത്രനായി വിഹരിച്ചു ജീവിച്ചു .

അയാളുടെ ജീവിതം ഊര് ചുറ്റുന്ന നാടുതെണ്ടികളുടേതായിരുന്നു. ചിലപ്പോള്‍ കുറ്റിച്ചെടികള്‍ക്കിടയിലയാള്‍ ഒളിച്ചു. ചിലപ്പോള്‍ ഗുഹകളില്‍ അഭയം തേടി. അയാള്‍ താമസംവിനാ ഒരു പിതാവായി. അയാളെപ്പോലെതന്നെ ദുഷ്ടസ്വഭാവികളും ക്രൂരവിനോദികളുമായിരുന്നു അയാളുടെ മക്കള്‍ .

അയാള്‍ക്ക്‌ വലിയൊരു കുടുംബമുണ്ടായിരുന്നു. അനേകം ബന്ധുക്കളും സുഹൃത്തുക്കളും. അയാള്‍ക്ക്‌ വയസ്സായി. ബന്ധുമിത്രാദികള്‍ ഓരോരുത്തരായി മരിച്ചു. അയാളെ മരണം കൊണ്ടുപോവാത്തതുകൊണ്ടയാള്‍ അന്യരാജ്യങ്ങളില്‍ അലഞ്ഞു തിരിഞ്ഞു നടന്നു. പലയിടങ്ങളിലും സഞ്ചരിച്ചു. അങ്ങിനെ നടക്കുമ്പോള്‍ ഐശ്വര്യസമ്പന്ന സമ്പൂര്‍ണ്ണമായ ഒരു വലിയ നഗരത്തിലെത്തിച്ചേര്‍ന്നു. നെറ്റിപ്പട്ടം കെട്ടിയ ഒരാന തലസ്ഥാന നഗരവാതില്‍ക്കല്‍ നില്‍ക്കുന്നു. ഈ ആനയ്ക്കൊരു പ്രത്യേക ദൗത്യമുണ്ടായിരുന്നു. നഗരത്തിലെ രാജാവ് അവകാശികളായി മക്കളില്ലാതെ അടുത്തയിടയ്ക്ക് മരണപ്പെട്ടിരുന്നു. ആ രാജ്യത്തിലെ രീതിയനുസരിച്ച് ഈ രാജകീയനായ ആനയാണ് അടുത്ത രാജാവിനെ തിരഞ്ഞെടുക്കേണ്ടത്.

ആഭരണവ്യാപാരി രത്നക്കല്ല് തേടുംപോലെ ആന  രാജപദവിക്കു യോജിച്ച ഒരാളെ തിരഞ്ഞു കൊണ്ടിരിക്കുന്ന സമയത്താണ് കാട്ടുജാതിക്കാരനായ അയാളവിടെയെത്തുന്നത്. വേട്ടക്കാരനായ ഗാധി ആനയെ സാകൂതം സൂക്ഷിച്ചു നോക്കി. പെട്ടെന്ന് ആന ഗാധിയെ തുമ്പിക്കൈകൊണ്ട് തൂക്കിയെടുത്ത് അതിന്റെ പുറത്തിരുത്തി. ഉടനെ തന്നെ നഗരത്തില്‍ പെരുംപറയും ശബ്ദകോലാഹലങ്ങളും തുടങ്ങി. ‘മഹാരാജാവ് നീണാള്‍ വാഴട്ടെ’ എന്ന് വിളിച്ചുപറഞ്ഞു ജനം ഓടിക്കൂടി. ആനയങ്ങിനെ  രാജാവിനെ തിരഞ്ഞെടുത്തു കഴിഞ്ഞു.

കേവലം കാട്ടുജാതിക്കാരനും നായാട്ടുകാരനുമായ ഗാധി ഇപ്പോള്‍ കൊട്ടാരത്തിലാണ്. സുന്ദര തരുണികള്‍ അദ്ദേഹത്തെ കുളിപ്പിച്ച് അണിയിച്ചൊരുക്കി. വിലയേറിയ ആഭാരണങ്ങള്‍ അണിയിച്ചു. മാലയിട്ടു. സുഗന്ധം പൂശി. വെറുമൊരു വേടനായിരുന്നവന്‍ പൊടുന്നനവേ ഒരു രാജാവായി തിളങ്ങി. രാജകീയ ഛിഹ്നങ്ങളും കിരീടവും അണിഞ്ഞു. സിംഹാസനത്തില്‍ അദ്ദേഹം ആസനസ്ഥനായി.

“അങ്ങിനെ ഗോത്രവര്‍ഗ്ഗക്കാരനായ ഒരു വേടന്‍ കിരാപുരമെന്ന രാജ്യത്തിന്റെ രാജാവായി.” രാജഭോഗങ്ങളെല്ലാം അദ്ദേഹം ആസ്വദിച്ചു തുടങ്ങി. കാലക്രമത്തില്‍ രാജഭരണം അയാള്‍ക്ക്‌ വശമായി. അദ്ദേഹം ഗാവലന്‍ എന്ന പേരില്‍ പ്രശസ്തനായ ഒരു രാജാവായി ഏറെക്കാലം ഭരണം നടത്തി.