\section{ദിവസം 192}

\slokam{
അസത്സ ത്സദ സത്സർവം സങ്കൽപ്പാദേവ നാന്യത:\\
സങ്കൽപ്പം സദസച്ചൈവമിഹ സത്യം കിമുച്യതാം (4/53/45)\\
}

ദാസുരമുനി തുടർന്നു: അങ്ങിനെയാണ്‌ ഈ വിശ്വത്തിന്റെയും മനുഷ്യന്റെയും സൃഷ്ടിയെപ്പറ്റി വിശദീകരിക്കാനാവുക. ആ മഹത്തായ ശൂന്യതയിൽ നിന്നുണ്ടായ ഖൊത്തൻ എന്നത് ഒരാളല്ല. അതൊരു സങ്കൽപ്പമോ ധാരണയോ പ്രതീതിയോ ആണ്‌.. നിശ്ശൂന്യതയിൽ ഈ ധാരണയുണ്ടാവുന്നത് സ്വയമേവയാണ്‌.. അതില്ലാതാവുന്നതും അപ്രകാരം തന്നെ. ഈ വിശ്വം മുഴുവനും, അതിലുള്ള എല്ലാ വസ്തുക്കളും ഈ ധാരണയുടെ സൃഷ്ടിയാണ്‌.. ബ്രഹ്മാവിഷ്ണുമഹേശ്വരന്മാർ ആകുന്ന ത്രിമൂർത്തികൾപോലും ഈ ധാരണയുടെ അവയവങ്ങളത്രേ. അതേ ധാരണയാണ്‌ മൂന്നു ലോകങ്ങളും പതിന്നാലു സ്വർഗ്ഗനരകങ്ങളും ഏഴു സമുദ്രങ്ങൾക്കും കാരണം. രാജാവുണ്ടാക്കിയ നഗരം വിവിധ സ്വഭാവസവിശേഷതകളോടുകൂടിയ അംഗങ്ങളും ശരീരവുമുള്ള ജീവനല്ലാതെ മറ്റൊന്നുമല്ല. ഇങ്ങിനെ സൃഷ്ടിക്കപ്പെട്ട സത്വങ്ങളിൽ ചിലവ ഉയർന്ന തലങ്ങളിൽ ദേവന്മാരായി വിരാജിക്കുന്നു. മറ്റു ചിലവ അധമതലങ്ങളിലാണ്‌.. ഇങ്ങിനെയൊരു സാങ്കൽപ്പിക നഗരം നിർമ്മിച്ചതിനുശേഷം രാജാവ് ഭൂതങ്ങളെയാണതിനു കാവലേൽപ്പിച്ചത്. അഹംകാരമാണീ ഭൂതങ്ങൾ.

രാജാവ് ഈ ദേഹത്തിലിരുന്ന് ലോകമെന്ന വേദിയിൽ ലീലയാടി രസിക്കുന്നു. ഒരുനിമിഷം അദ്ദേഹം ഈ ലോകത്തെ ജാഗ്രദവസ്ഥയിൽ ദർശിക്കുന്നു. എന്നാൽ കുറച്ചുകഴിഞ്ഞ് അയാൾ തന്റെ ദൃഷ്ടി ലോകത്തിൽപ്പതിപ്പിക്കുന്നു. അതയാൾ ആസ്വദിക്കുന്നത് സ്വപ്നാവസ്ഥയിലാണ്‌.. അയാള്‍ ഒരു നഗരത്തിൽ നിന്നും മറ്റൊന്നിലേയ്ക്ക്- ഒരു ദേഹത്തിൽ നിന്നും മറ്റൊന്നിലേയ്ക്ക് വാസം മാറ്റുന്നു. ഒരു തലത്തിൽ നിന്നും മറ്റൊന്നിലേയ്ക്ക് സംക്രമിക്കുന്നു. ഇങ്ങിനെയുള്ള അനവധി സഞ്ചാരങ്ങൾക്കുശേഷം അയാൾക്ക് വിവേകമുദിക്കുന്നു. ലൗകീക ജീവിതത്തിലും സുഖ-ദു:ഖാനുഭവങ്ങളിലും പെട്ടുഴറിവലഞ്ഞ് അയാൾ അലച്ചിലെല്ലാം അവസാനിപ്പിക്കുന്നു. ധാരണകളെല്ലാം അങ്ങിനെ ഒടുങ്ങുന്നു. ഒരുനിമിഷം അയാൾ ജ്ഞാനത്തിലാമഗ്നനായും ഉടനേ തന്നെ ആ ജ്ഞാനമെല്ലാം നഷ്ടപ്പെട്ട് വീണ്ടും സുഖാസക്തനായും ഒരു ശിശുവിനേപ്പോലെ ചഞ്ചലപ്പെടുന്നു. ഈ ധാരണകൾ ഒന്നുകിൽ അതീവ തമസ്സിലേയ്ക്ക്, അതായത് അജ്ഞാനത്തിലേയ്ക്കും അധമയോനികളിലെ ജനനത്തിലേയ്ക്കും നയിക്കുന്നു. അല്ലെങ്കിൽ, അവ ശുദ്ധപ്രഭയാർന്ന് സുതാര്യമായ ലോകത്തേയ്ക്ക്, അതായത്, ജ്ഞാനത്തിലേയ്ക്ക്, സത്യസവിധത്തിലേയ്ക്ക്, നയിക്കുന്നു. മൂന്നാമത്തെ കൂട്ടർ ധാരണാമാലിന്യം മൂലം ലൗകീകതയിൽ മുങ്ങുന്നു. ഇപ്രകാരമുള്ള എല്ലാ ധാരണകളുടേയും അവസാനമാണ് മുക്തി.

ഒരുവൻ ഏതൊരാത്മീയ സാധനയിലേർപ്പെട്ടുവെന്നാലും, അവനു ഗുരുവായി ദേവതമാർ തന്നെ ഉണ്ടായിരുന്നുവെന്നാലും, ഈ ധാരണകളുടെ അന്ത്യംകൊണ്ടുമാത്രമേ മുക്തി സാദ്ധ്യമാവൂ. “സത്ത്, അസത്ത്, സത്താസത്തുകളുടെ സമ്മിശ്രം എല്ലാം വെറും ധാരണകളല്ലാതെ മറ്റൊന്നുമല്ല. ഈ ധാരണകൾ സ്വയം സത്തോ അസത്തോ അല്ല. അങ്ങിനെയിരിക്കേ ഈ പ്രപഞ്ചത്തിൽ നാമെന്തിനെയാണ്‌ ഉണ്മയെന്ന്, സത്തെന്നു പറയുന്നത്?” അതുകൊണ്ട് പ്രിയപ്പെട്ട മകനേ, എല്ലാ ചിന്തകളും, സങ്കൽപ്പങ്ങളും, ധാരണകളും ഉപേക്ഷിക്കൂ. അവയെല്ലാമൊടുങ്ങുമ്പോൾ മനസ്സ് സ്വയമേവ മനസ്സിനതീതമായുള്ള അനന്താവബോധസീമയിലേയ്ക്ക് ഉന്മുഖമാവുകതന്നെ ചെയ്യും. 

