\newpage
\section{ദിവസം 098}

\slokam{
അകൃത്രിമം സുഖം കീര്‍ത്തിമായുശ്ചൈവാഭിവാഞ്ഛതാ\\
സര്‍വാഭിമതദാനേന പൂജനീയാ ഗുണാന്വിതാ:  (3/77/26)\\
}

വസിഷ്ഠന്‍ തുടര്‍ന്നു: കാര്‍ക്കടിക്ക്‌ പഴയ ഭീകരരൂപം തിരിച്ചു കിട്ടിയിരുന്നുവെങ്കിലും അവളിലെ രാക്ഷസീയത തീരെ ഇല്ലാതായിപ്പോയിരുന്നു. അവളുടെ അവബോധം അതിഭൌതീകമായി നിലകൊണ്ടു. അവള്‍ അതേയിടത്ത്‌ പത്മാസനത്തിലിരുന്ന് ധ്യാനനിരതയായിക്കഴിഞ്ഞു. ആറുമാസം കഴിഞ്ഞപ്പോഴെയ്ക്ക്‌ അവള്‍ക്ക്‌ പുറം ലോകത്തെക്കുറിച്ചും ശരീരത്തെക്കുറിച്ചും ബോധമുണ്ടായി. ഉടനേ അവള്‍ക്ക്‌ വിശപ്പനുഭവപ്പെട്ടു. ശരീരമുള്ളിടത്തോളം അതിന്റെ നിയത നിയമങ്ങള്‍ - വിശപ്പ്‌, ദാഹം എന്നിവയടക്കം- ബാധകമാകാതെ വയ്യല്ലോ. കാര്‍ക്കടി ആലോചിച്ചു: ഞാന്‍ എന്താണു തിന്നേണ്ടത്‌? ആരെയാണു ഞാന്‍ വിഴുങ്ങേണ്ടത്‌? ജീവികളെ സ്വജീവസന്ധാരണത്തിനായി കൊല്ലുന്നത്‌ വിവേകശാലികളായ മാമുനിമാര്‍ വിലക്കിയിട്ടുണ്ട്‌. . അപ്പോള്‍ ഈ നിഷിദ്ധഭക്ഷണം കഴിക്കാതിരുന്നാല്‍ ഞാന്‍ എന്റെ ശരീരം ഉപേക്ഷിക്കുന്നു എന്നര്‍ത്ഥം. ഞാനതില്‍ തെറ്റൊന്നും കാണുന്നില്ല. ഉത്തമമല്ലാത്ത ഭക്ഷണം വിഷത്തിനു സമം. മാത്രമല്ല, പ്രബുദ്ധതയിലഭിരമിക്കുന്ന എന്നേപ്പോലുള്ളവര്‍ക്ക്‌ ജീവിതവും മരണവും തമ്മില്‍ വ്യത്യാസമൊന്നുമില്ല.'

കാര്‍ക്കടി ഇപ്രകാരം ചിന്തിക്കേ ഒരശരീരി ആകാശത്തില്‍നിന്നു കേള്‍ക്കായി. "കാര്‍ക്കടീ, നീ അജ്ഞാനികളേയും മോഹവിഭ്രാന്തരേയും സമീപിച്ച്‌ അവരില്‍ വിവേകമുണ്ടാക്കിയാലും. പ്രബുദ്ധതയിലേക്കുണര്‍ന്നവര്‍ക്ക്‌ ചെയ്യാന്‍ ഇതല്ലാതെ മറ്റ്‌ കര്‍മ്മങ്ങളൊന്നുമില്ല. അങ്ങിനെ നീ സമീപിച്ചുനോക്കിയിട്ടും സത്യത്തിലേയ്ക്കുണരാന്‍ കൂട്ടാക്കാത്തവരെ നിനക്കു ഭക്ഷണമാക്കാം. അങ്ങിനെയുള്ളവരെ പാപഭീതിയൊന്നും കൂടാതെ നിനക്ക്‌ വിഴുങ്ങാം." 

ഇതുകേട്ട്‌ കാര്‍ക്കടി മലയിറങ്ങിവന്ന് ഒരു നിബിഢവനത്തില്‍ പ്രവേശിച്ചു. അവിടെ ആദിവാസികളായ നായാട്ടുകാരാണു വസിച്ചിരുന്നത്‌.. രാത്രിയായി. അവരുടെ രാജാവാണ്‌ വിക്രമന്‍.. രാത്രി തന്റെ മന്ത്രിയോടൊപ്പം കാട്ടില്‍ , ഇരുട്ടിന്റെമറവില്‍പ്പോയി തന്റെ പ്രജകളെ ദ്രോഹിക്കുന്ന കൊള്ളക്കാരേയും പിടിച്ചുപറിക്കാരേയും കീഴടക്കുന്നത്‌ വിക്രമന്റെ പതിവായിരുന്നു. ഈ രണ്ടു ധീരയോധാക്കള്‍ -രാജാവും മന്ത്രിയും- വനദേവതമാര്‍ക്കു പൂജ ചെയ്യുന്നത്‌ കാര്‍ക്കടി കണ്ടു. അവളാലോചിച്ചു: എന്റെ വിശപ്പുമാറ്റാനാണ്‌ ഇവരിപ്പോള്‍ ഇവിടെയെത്തിയിരിക്കുന്നത്‌.. അവര്‍ അജ്ഞരും ഭൂമിക്കൊരു ഭാരവുമാണ്‌.. അവര്‍ ഇഹജന്മത്തിലും പരജന്മത്തിലും ദുരിതമനുഭവിക്കുന്നു. ദുരിതം മാത്രമാണവരുടെ ജന്മലക്ഷ്യമെന്നു തോന്നുന്നു. മരണം അവര്‍ക്കൊരനുഗ്രഹമായിരിക്കും, കാരണം അവര്‍ക്കീ ദുരിതത്തില്‍നിന്നു മോചനമാകുമല്ലോ. മാത്രമല്ല മരണശേഷം ചിലപ്പോള്‍ അവര്‍ക്ക്‌ വിവേകബോധമുദിക്കാനും മുക്തിപദം പ്രാപിക്കാനും അവസരം ലഭിച്ചേക്കാം. പക്ഷേ അവര്‍ രണ്ടാളും നല്ല ജ്ഞാനികളാണെങ്കിലോ? ജ്ഞാനികളെ കൊല്ലുന്നത്‌ എനിക്കിഷ്ടമല്ല. "കറപുരളാത്ത ആനന്ദം ആഗ്രഹിക്കുന്നവര്‍ ഏതുവിധേനെയും സദ്ജനങ്ങളെ വന്ദിച്ചു ബഹുമാനിക്കണം. അവര്‍ക്ക്‌ ഹിതമായതെല്ലാം ആവുന്നത്ര കൊടുക്കുകയുംവേണം". അവരുടെ ജ്ഞാനത്തെ ഒന്നു പരീക്ഷിച്ചുകളയാം. അവര്‍ ജയിച്ചാല്‍ ഞാനവരെ ഉപദ്രവിക്കുകയില്ല. വിവരമുള്ള സദ്ജങ്ങള്‍ മാനവീകതയ്ക്കൊരനുഗ്രമാണല്ലോ.

