\newpage
\section{ദിവസം 053}

\slokam{
തപോജപയാമൈർദേവി സമസ്താ: സിദ്ധസിദ്ധയ:\\
സം പ്രാപ്യന്തേമരത്വം തു ന കദാചന ലഭ്യതേ (3/16/24)\\
}

വസിഷ്ഠന്‍ തുടര്‍ന്നു: പദ്മരാജാവും ലീലരാജ്ഞിയും ഉത്തമമായ ജീവിതം നയിച്ചു വന്നു. അവര്‍ ധാര്‍മ്മീകമായ എല്ലാവിധ സുഖസൌകര്യങ്ങളും ആസ്വദിച്ച്‌ ജീവിതം നയിച്ചു. അവര്‍ യൌവ്വനയുക്തരും ദേവതമാരേപ്പോലെ താരുണ്യം കൈവിടാത്തവരുമായിരുന്നു. അവരുടെ പ്രേമം കാപട്യലേശമില്ലാത്തതും, കളങ്കരഹിതവും തീവ്രവുമായിരുന്നു. 

ഒരു ദിവസം രാജ്ഞി ആലോചിച്ചു: "എന്റെ ജീവനേക്കാള്‍ എനിക്കു പ്രിയപ്പെട്ടതാണ്‌ സുന്ദരനായ എന്റെ പ്രിയതമന്‍ . ഞാനും അദ്ദേഹവും എന്നെന്നും ഒരുമിച്ചു  ആഹ്ളാദവും    സന്തോഷവും പങ്കുവെച്ച്‌ കഴിയണമെങ്കില്‍ എന്താണു ഞാന്‍ ചെയ്യേണ്ടത്‌? അതിനായി, മഹാന്മാരായ മുനിമാരോട്‌ ചോദിച്ച്‌ അവര്‍ ഉപദേശിക്കുന്ന തപ:ശ്ചര്യകള്‍ ഞാന്‍ അനുഷ്ഠിക്കും" രാജ്ഞി മഹര്‍ഷിമാരോട്‌ ആരാഞ്ഞതിനു മറുപടിയായി അവര്‍ പറഞ്ഞു: "അല്ലയോ രാജ്ഞി, തപശ്ചര്യകള്‍ , വ്രതം, മന്ത്രജപം, അച്ചടക്കമുള്ള ജീവിതം എന്നിവകൊണ്ട്‌ ഈ ലോകത്തില്‍ ലഭിക്കാവുന്ന എല്ലാം നിനക്കു നേടാന്‍ കഴിയും. എന്നാല്‍ ഭൌതികമായി അമര്‍ത്ത്യത എന്ന അവസ്ഥ, ഈ ലോകത്തില്‍ അസാദ്ധ്യമാണ്‌."

രാജ്ഞി അവരുടെ ഉപദേശത്തെക്കുറിച്ച്‌ ഇങ്ങിനെ ചിന്തിച്ചു: 'ഞാന്‍ എന്റെ പ്രിയന്‍ മരിക്കുന്നതിനുമുന്‍പ്‌ മരിക്കുകയാണെങ്കില്‍ എനിക്ക്‌ ആത്മജ്ഞാനവും അങ്ങിനെ ദു:ഖനിവൃത്തിയും ഉണ്ടാവണം. അദ്ദേഹം എനിയ്ക്കു മുന്‍പ്‌ മരിക്കുകയാണെങ്കില്‍ അദ്ദേഹത്തിന്റെ ജീവന്‍ കൊട്ടാരം വിട്ട്‌ പോവാതിരിക്കാനായി ദേവതകളില്‍ നിന്ന് എനിക്കൊരു വരം നേടണം. അദ്ദേഹം കൊട്ടാരത്തില്‍ എന്റെയടുത്തുണ്ടെന്ന സന്തോഷത്തില്‍ എനിക്കങ്ങനെ ജീവിക്കാമല്ലൊ.' അങ്ങിനെ തീരുമാനിച്ച്‌ രാജ്ഞി സരസ്വതി ദേവിയെ പ്രസാദിപ്പിക്കാന്‍ തപസ്സാരംഭിച്ചു. ഇക്കാര്യം രാജാവിനോട്‌ പറഞ്ഞില്ല. ഭഗവദ്‌ പൂജ കഴിഞ്ഞ്‌ മഹര്‍ഷിമാരേയും ഗുരുക്കന്മാരേയും പ്രസാദിപ്പിച്ചശേഷം മൂന്നു ദിവസത്തിലൊരിക്കല്‍ മാത്രം ഭക്ഷണം കഴിച്ച്‌ അവള്‍ തന്റെ വ്രതം തുടര്‍ന്നു. ഈ തപസ്സ്‌ അത്യന്തം ഫലപ്രദമാവുമെന്ന് അവള്‍ ക്കുറപ്പുണ്ടായിരുന്നു. അതുകൊണ്ടുതന്നെ അവളുടെ തപശ്ചര്യകള്‍ പ്രബലമായിത്തീര്‍ന്നിരുന്നു. രജാവിനോട്‌ തപസ്സിന്റെ കാര്യം പറഞ്ഞിരുന്നില്ലെങ്കിലും തന്റെ തപസ്സുകൊണ്ട്‌ അദ്ദേഹത്തിന്‌ യാതൊരു ബുദ്ധിമുട്ടും സൌകര്യക്കുറവും വരുത്താതെ രാജ്ഞി ശ്രദ്ധയോടെ പെരുമാറി. അങ്ങിനെ നൂറ്‌ ത്രിദിന പൂജകള്‍ കഴിഞ്ഞപ്പോള്‍ സരസ്വതീ ദേവി അവള്‍ ക്കുമുന്നില്‍ പ്രത്യക്ഷയായി അവള്‍ക്കാവശ്യമുള്ള വരങ്ങള്‍ നല്‍ കി.

ലീല പറഞ്ഞു: 'ദിവ്യ ജനനീ എനിയ്ക്കു രണ്ടു വരങ്ങള്‍ തന്നാലും: ഒന്ന്- എന്റെ പ്രിയതമന്‍ ശരീരമുപേക്ഷിക്കുമ്പോള്‍ അദ്ദേഹത്തിന്റെ ജീവന്‍ കൊട്ടാരത്തില്‍ത്തന്നെയുണ്ടാവണം. രണ്ട്‌- ഞാന്‍ പ്രാര്‍ത്ഥിച്ചാലുടന്‍ എനിക്കവിടുത്തെ ദര്‍ശനം കിട്ടണം' ദേവി, വരങ്ങള്‍ നല്‍കിയിട്ട് അപ്രത്യക്ഷയായി. കാലമേറെ കടന്നുപോയി. യുദ്ധക്കളത്തില്‍ പദ്മരാജാവിനു മുറിവേറ്റു. താമസിയാതെ കൊട്ടാരത്തിലെത്തി മരിക്കുകയുംചെയ്തു. രാജ്ഞി ദു:ഖാകുലയായി സഹിക്കവയ്യാത്ത സങ്കടത്തിലിരിക്കുമ്പോള്‍ ഒരശരീരി തന്നോടു സംസാരിക്കുന്നതായിക്കേട്ടു.
