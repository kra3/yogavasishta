 
\section{ദിവസം 103}

\slokam{
ആത്മാ യത്നശതപ്രാപ്യോ ലബ്ധേ ഽസ്മിന്ന ച കിംചന\\
ലബ്ധം ഭവതി തച്ചൈതത്പരം വാ ന കിംചന (3/81/9)\\
}

കാര്‍ക്കടി പറഞ്ഞു: രാജാവേ അങ്ങയുടെ മന്ത്രിയുടെ ഉത്തരങ്ങളില്‍ ഞാന്‍ സംപ്രീതയാണ്‌. ഇനി അങ്ങയുടെ ഉത്തരങ്ങള്‍ എന്തെന്നറിയണം. 

രാജാവ്‌ പറഞ്ഞു: ഭവതിയുടെ ചോദ്യങ്ങള്‍ ശാശ്വതമായ ബ്രഹ്മത്തെക്കുറിച്ചാണ്‌.. അത്‌ ശുദ്ധ അസ്തിത്വമാണ്‌.. ജാഗ്രത്ത്‌, സ്വപ്നം, സുഷുപ്തി എന്നീ മൂന്നവസ്ഥകള്‍ അവസാനിച്ച്‌ മാനസീകവ്യാപാരങ്ങള്‍ എല്ലാം നിലയ്ക്കുമ്പോള്‍ അതിനെ അറിയാന്‍ കഴിയും. ബ്രഹ്മത്തിന്റെ പ്രത്യക്ഷഭാവമായ വിക്ഷേപവും ആവരണവുമാണ്‌ സൃഷ്ടിയും പ്രളയവുമായി പൊതുവേ പറഞ്ഞുവരുന്നത്‌.. നിശ്ശബ്ദതയുടെ ആഴത്തില്‍ 'അറിവിന്റെ' ഭാരമെല്ലാമൊഴിഞ്ഞ്‌, എല്ലാ ആവിഷ്കാരങ്ങള്‍ക്കും അതീതമായി അതു നിലകൊള്ളുന്നു. അതിന്റെ മദ്ധ്യഭാഗം അതീവ സൂക്ഷ്മമത്രേ. ഈ മദ്ധ്യത്തിനു രണ്ടു വശങ്ങളുണ്ട്‌.  ഇക്കാണുന്ന വിശ്വം അതിന്റെ ലീലാവിലാസമായ ബോധവിക്ഷേപം മാത്രമാണ്‌.. കാണപ്പെടുന്നതായ നാനാത്വമായി അറിയുന്നുവെങ്കിലും അത്‌ അവിച്ഛിന്നമാണ്‌.. ഈ ബ്രഹ്മം ഇച്ഛിക്കുമ്പോള്‍ വായു ഉണ്ടാവുന്നു. ഈ വായുവോ, അതും ശുദ്ധബോധം തന്നെ. അതുപോലെ ശബ്ദത്തെപറ്റി ആലോചിക്കുമ്പോള്‍ ശബ്ദമുണ്ടാവുന്നു. എന്നാല്‍ സത്യത്തില്‍ ശബ്ദമായോ ശബ്ദമെന്നു വിവക്ഷിക്കുന്നതിന്റെ അര്‍ത്ഥമായോ വസ്തുവായോ ബ്രഹ്മത്തിനു ബന്ധമൊന്നുമില്ല.

ആ പരമാണു എല്ലാമാണ്‌; എന്നാല്‍ ഒന്നുമല്ല. ഞാന്‍ അതാണ്‌; എന്നാല്‍ ഞാനല്ല. അതു മാത്രമേ ഉണ്മയായുള്ളു. അത്‌ സര്‍വ്വശക്തമാകയാല്‍ ഇക്കാണായതെല്ലാം നമുക്ക്‌ പ്രത്യക്ഷമാണ്‌.. "ഈ ആത്മാവിനെ പ്രാപിക്കാന്‍ നൂറുകണക്കിനു മാര്‍ഗ്ഗങ്ങളുണ്ട്‌.. എന്നാലാ സാക്ഷാത്കാരമായിക്കഴിഞ്ഞാല്‍ യാതൊന്നും 'നേടിയിട്ടില്ല'. അതു പരമാത്മാവാണ്‌.. എങ്കിലും അത്‌ 'ഒന്നും' അല്ല". സംസാരമെന്ന ഈ കാനനത്തില്‍ അലയാന്‍, ചാക്രികമായ ഈ ചരിതം ആവര്‍ത്തിക്കാന്‍ ഇടയാക്കുന്നത്‌ ഇല്ലാത്തതിനെ ഉള്ളതാണെന്നു തോന്നുന്ന അവിദ്യകൊണ്ടാണ്‌. . അറിവിന്റെ സൂര്യോദയം മാത്രമേ ഈ രൂഢമൂലമായ അവിദ്യയെ ഇല്ലാതാക്കാന്‍ ഉതകൂ. ജലദൃശ്യംകാരണം മരുപ്പച്ചയിലേക്ക്‌ ആകര്‍ഷിക്കപെടുന്ന അജ്ഞാനിക്കെന്നപോലെ അജ്ഞാനിക്ക്‌ ലോകദൃശ്യം സത്യമായിത്തോന്നുന്നു. അനന്താവബോധം ഈ വിശ്വത്തെ ഭാവനചെയ്ത്‌ സ്വന്തം മായാശക്തിയാല്‍ പ്രത്യക്ഷമാകുകയാണ്‌.. ബോധമണ്ഡലത്തിനകമേ കാണുന്ന ലോകം പുറത്തു പ്രതിബിംബിക്കുകയാണ്‌.. കാമഭ്രാന്തന്റെ വിഭ്രമം പോലെയത്രേ ഇത്‌..

അത്മാവ്‌ അണുപോലെ സൂക്ഷ്മമായ ശുദ്ധ അവബോധമായതിനാല്‍ വിശ്വം മുഴുവന്‍ വ്യാപരിച്ചിരിക്കുന്നു. സര്‍വ്വവ്യാപിയായതിനാല്‍ ദൃശ്യപ്രപഞ്ചം മുഴുവന്‍ അതിന്റെ താളത്തിനു തുള്ളാന്‍ പ്രചോദിതമാവുന്നു. അതിനു മുടിനാരിന്റെ നൂറിലൊരംശം വലുപ്പമില്ലെങ്കിലും സഹജമായ സര്‍വ്വവ്യാപിത്വം കാരണം അതിനേക്കാള്‍ വലുതായി ഒന്നുമില്ല.

