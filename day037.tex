\newpage
\section{ദിവസം 037}

\slokam{
പ്രാണസ്പന്ദോ സ്യ യത്കർമ ലക്ഷ്യതേ ചാസ്മദാദിഭി:\\
ദൃശ്യതേസ്മാഭിരേവം തത്ര ത്വസ്യാസ്ത്യത്ര കർമ്മധീ: (3/2/25)\\
}

വസിഷ്ഠന്‍ തുടര്‍ന്നു: ആകാശജന്‍ എന്നുപേരായ ഒരു മഹാത്മാവുണ്ടായിരുന്നു. അദ്ദേഹം സര്‍വ്വരുടേയും ക്ഷേമത്തിനായി നിരന്തരം ധ്യാനനിരതനായിരുന്നു. ദീര്‍ഘകാലം ജീവിച്ചശേഷം മരണം അദ്ദേഹത്തെ കൊണ്ടുപോവാന്‍ വന്നടുത്തു. എന്നാല്‍ ഒരു തീവ്രമായ അഗ്നിവലയം അദ്ദേഹത്തിനു ചുറ്റും ഉണ്ടായിരുന്നതുകൊണ്ട്‌ മരണത്തിന്‌ അദ്ദേഹത്തെ സമീപിക്കാന്‍ കഴിയാതെവന്നു. തീവലയത്തെ പ്രതിരോധിച്ചുവെങ്കിലും അദ്ദേഹത്തെ തൊടാന്‍ പോലും മരണത്തിനു സാധിച്ചില്ല. നാണംകെട്ട്‌ , മരണം ഈ അസാധാരണ പ്രതിഭാസത്തെപ്പറ്റി പരാതിയുമായി മനുഷ്യന്റെ ഭാഗധേയം നിശ്ചയിക്കുന്ന യമരാജന്റെയടുക്കല്‍ ചെന്നു. "എനിക്കെന്തുകൊണ്ടാണ്‌ അദ്ദേഹത്തെ പിടികൂടാന്‍ കഴിയാതെ വന്നത്‌?" യമന്‍ പറഞ്ഞു: മരണമേ, നീ ആരേയും കൊല്ലുന്നില്ല. മരണം ഒരുവന്റെ കര്‍മ്മഫലങ്ങളെ ആശ്രയിച്ചിരിക്കുന്നു. അതുകൊണ്ട്‌ ഈ മനുഷ്യന്റെ മരണം ഉറപ്പാക്കുന്ന കര്‍മ്മം എന്തെന്നു കണ്ടുപിടിക്കുക." വന്ധ്യ പ്രസവിച്ച മകനെവിടെയെന്നു തിരയും പോലെ അസംബന്ധമായതിനാല്‍ അയാളുടെ കര്‍മ്മത്തെപ്പറ്റി മരണത്തിന്‌ വിവരമൊന്നും കിട്ടിയില്ല. മരണം ആ വിവരം യമനെ അറിയിച്ചു.

യമന്‍ പറഞ്ഞു: "ഈ ആകാശജന്‍ എന്ന മഹാന്‍ ആകാശത്തില്‍ നിന്നു ജനിച്ചതുകൊണ്ട്‌ അവന്‌ കര്‍മ്മങ്ങള്‍ ഒന്നുമില്ല. അദ്ദേഹം ആകാശം പൊലെ നിര്‍മ്മലനാണ്‌. യാതൊരു കര്‍മ്മങ്ങളും ആര്‍ജ്ജിക്കാത്തതുമൂലം അദ്ദേഹത്തിനെ പിടിക്കാനോ ആഹരിക്കാനോ നിനക്കു സാധിക്കുകയില്ല. വന്ധ്യയുടെ പുത്രനെന്നപോലെ അവന്‍ ഇനിയും ജനിച്ചിട്ടില്ല. അവന്‌ മു  മുന്‍ജന്മകര്‍മ്മങ്ങളില്ല. അതുകൊണ്ട്‌ മനസ്സുമില്ല. അവന്‍ യാതൊരു കര്‍മ്മങ്ങളും മനസാ പോലും ചെയ്തിട്ടില്ല. അവന്‍ ഒരു ബോധഘനമാണ്‌. 

"അവന്‍ ജീവിയേപ്പോലെ തോന്നുന്നത്‌ നമ്മുടെ ദൃഷ്ടിയില്‍ മാതമാണ്‌. അവനില്‍ കര്‍മ്മബന്ധിതമാവുന്ന അത്തരം ധാരണകള്‍ യാതൊന്നുമില്ല." ബോധം ബോധത്താല്‍ പ്രതിഫലിക്കുന്നു. ആ നിഴലാകട്ടെ താന്‍ സ്വതന്ത്രമായ ഉണ്മയാണെന്നനുമാനിക്കുന്നു. ഇത്‌ തെറ്റിദ്ധാരണയാണ്‌, സത്യാവസ്ഥയല്ല. ഈ മഹാത്മാവിന്‌ അതറിയാം. എപ്രകാരം ജലം ദ്രാവകാവസ്ഥയിലാണോ, ആകാശം ശൂന്യമാണോ, അതുപോലെ ഈ മഹാത്മാവ്‌ പരമാത്മാവില്‍ വിരാജിക്കുന്നു. അദ്ദേഹത്തിന്റേത്‌ അഹൈതുകമായ ഒരു സാക്ഷാത്കാരമാണ്‌. അതുകൊണ്ട്‌ അദ്ദേഹം സ്വയംഭൂ (സ്വയംകൃതന്‍) എന്നും അറിയപ്പെടുന്നു. എന്നാല്‍ ഏതൊരുവന്‍ 'ഞാന്‍ ഈ ഭൂമി ധാതുവിനാല്‍ നിര്‍മ്മിതമായ ശരീരമാണ്‌' എന്ന വസ്തു ധാരണയില്‍ കുടുങ്ങിയിരിക്കുന്നുവോ അവനെ നിനക്ക്‌ കീഴ്പ്പെടുത്താം. 

ഈ മഹാത്മാവിന്‌ അത്തരം ധാരണകള്‍ ഒന്നുമില്ലാത്തതിനാല്‍ (ശരീരം തന്നെ ഇല്ലാത്തതിനാല്‍ ) അദ്ദേഹം നിനക്കു കീഴടങ്ങുകയില്ല. അദ്ദേഹം ജനിച്ചിട്ടുകൂടിയില്ല. അദ്ദേഹം മാറ്റങ്ങള്‍ ക്കു വിധേയമാകാത്ത ശുദ്ധബോധമാണ്‌. അന്തര്‍ലീനമായ അവിദ്യ കാരണം യുഗാരംഭത്തില്‍ അനന്തതയില്‍ ഒരു  സ്പന്ദനം  (പ്രകമ്പനം) ഉണ്ടാവുന്നു. ഒരു പ്രപഞ്ച സ്വപ്നത്തിലെന്നവണ്ണം ഈ സ്പന്ദനം  അനേകം വൈവിദ്ധ്യമാര്‍ന്ന ജീവജാലങ്ങളെ പ്രകടമാക്കുന്നു. എങ്കിലും ഇവയിലൊന്നും ഇടപെടാതെ ഈ മഹാത്മാവ്‌ ശുദ്ധബോധമായി നിലകൊള്ളുന്നു.

വസിഷ്ഠന്‍ പറഞ്ഞു: സൃഷ്ടാവില്‍ ദൃഷ്ടാവോ ദൃഷ്ടിയോ ഇല്ല. എങ്കിലും അദ്ദേഹം സ്വയംഭൂ - സ്വയം സൃഷ്ടിക്കപ്പെട്ടവന്‍ - എന്നറിയപ്പെടുന്നു. വിശ്വബോധത്തില്‍ ഒരു ചിത്രകാരന്റെ മനസ്സിലെ ചിത്രമെന്നപോലെ അദ്ദേഹം ദീപ്തിയോടെ വിളങ്ങുന്നു.

