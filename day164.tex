\section{ദിവസം 164}

\slokam{
മയി സർവ്വമിദം പ്രോതം സൂത്രേ മണിഗണാ ഇവ\\
ചിത്തം തു നാഹമേവേതി യ: പശ്യതി സ പശ്യതി(4/22/31)\\
}

വസിഷ്ഠൻ തുടർന്നു: ശരീരത്തെ ഭ്രമാത്മകമായ ഒരു ധാരണയുടെ സന്താനവും ദൗർഭാഗ്യങ്ങളുടെ ശ്രോതസ്സും മത്രമാണെന്നു തിരിച്ചറിയുന്നവൻ സത്യദർശിയാണ്‌..  ശരീരമല്ല ആത്മാവെന്ന തിരിച്ചറിവും അയാൾ ക്കുണ്ടല്ലോ. ഈ ശരീരത്തിൽ ഉളവാകുന്ന സുഖവും ദു:ഖവും സമയനിബദ്ധമായി, ചുറ്റുപാടുകൾക്കധീനമായി സംഭവിക്കുന്നതാണെന്നും അവയ്ക്ക് താനുമായി സംഗമൊന്നുമില്ലെന്നും അറിഞ്ഞവൻ സത്യദർശിയാണ്‌..  എല്ലായിടത്തും സംഭവിക്കുന്ന എല്ലാക്കാര്യവും സമഗ്രമായി ഉൾക്കൊള്ളുന്ന സർവ്വവ്യാപിയായ അന്താവബോധമാണ്‌ താനെന്നറിഞ്ഞവനും സത്യദർശിയത്രേ. മുടിനാരിഴയുടെ കോടിയിലൊരംശത്തോളം സൂക്ഷ്മമായ അത്മാവിന്റെ സർവ്വവ്യാപിത്വം തിരിച്ചറിഞ്ഞവനും സത്യമറിഞ്ഞവനാണ്‌..

സ്വാത്മാവും മറ്റുള്ളവയും തമ്മിൽ അന്തരമേതുമില്ലെന്ന് തിരിച്ചറിയുകയും ഉണ്മയായുള്ളത് അനന്തമായ അവബോധത്തിന്റെ പ്രഭയൊന്നുമാത്രമാണെന്നുറപ്പിക്കുകയും ചെയ്തവൻ സത്യദർശിയാണ്‌.. എല്ലാ ജീവ നിർജ്ജീവ ജാലങ്ങളിലും സർവ്വാന്തര്യാമിയായി, സർവ്വവ്യാപിയായി, സർവ്വശക്തനായി വർത്തിക്കുന്നത് അദ്വൈതമായ അനന്താവബോധം മാത്രമാണെന്നറിഞ്ഞവൻ സത്യദർശി. രോഗപീഢകൾ, ഭയം, വിക്ഷോഭങ്ങൾ, വാർദ്ധക്യം, മരണം എന്നിവയാൽ നിരന്തരം വേട്ടയാടിക്കൊണ്ടിരിക്കുന്ന ശരീരമാണു താനെന്ന മോഹവിഭ്രമത്തിനടിപ്പെടാത്തവൻ സത്യദർശി തന്നെ. “ഞാൻ മനസ്സല്ല എന്ന തിരിച്ചറിവോടെ സർവ്വവും ഒരു മാലയുടെ ചരടിൽ കോർത്ത മുത്തുമണികൾപോലെ പരസ്പരം ബന്ധപ്പെട്ടിരിക്കുന്നു എന്നറിഞ്ഞവൻ സത്യമറിയുന്നു.”

ഞാനോ, നീയോ പരമസത്യമല്ലെന്നും സര്‍വ്വവും ബ്രഹ്മമാണെന്നുമറിഞ്ഞവൻ സത്യദർശിയാകുന്നു. മൂന്നുലോകത്തിലെ സർവ്വ ജീവജാലങ്ങളും തന്റെ കുടുംബാംഗങ്ങളാണെന്നും അവർക്കെല്ലാം വേണ്ട സേവനങ്ങൾ ചെയ്തു കൊടുക്കേണ്ടത് തന്റെ കർത്തവ്യമാണെന്നും അവയ്ക്കെല്ലാം തന്റെ അനുകമ്പയ്ക്കും സംരക്ഷണത്തിനും അർഹതയുണ്ടെന്നും അറിഞ്ഞവൻ സത്യമറിഞ്ഞവനത്രേ. അത്മാവുമാത്രമേ ഉണ്മയായുള്ളു എന്നും വസ്തു-വിഷയങ്ങൾ അയാഥാർത്ഥ്യമാണെന്നും അവനറിയാം. സുഖം, ദു:ഖം, ജനനം, മരണം എന്നിവയെല്ലാം ആത്മാവു തന്നെയെന്നും അവനറിയുന്നതുകൊണ്ട് ഈ അവസ്ഥകൾ അവനെ ബാധിക്കുന്നില്ല. “ഞാനടക്കം ഈ കാണുന്നതെല്ലാം ആത്മാവു മാത്രമാകയാൽ ഞാൻ എന്തു സമ്പാദിക്കാനാണ്‌? ഞാൻ എന്തു ത്യജിക്കുവാനാണ്‌?” എന്ന തോന്നലുള്ളവൻ സത്യത്തിൽ ഉറച്ചിരിക്കുന്നു. സൃഷ്ടി സ്ഥിതി സംഹാരങ്ങൾ എന്ന പ്രകടമായ മാറ്റങ്ങളിലെല്ലാം മാറ്റമില്ലാതെ അചഞ്ചലമായി നിലകൊള്ളുന്ന ബ്രഹ്മമാണ്‌ വിശ്വത്തിനടിസ്ഥാനമെന്ന അറിവു സാക്ഷാത്കരിച്ചവർക്കെല്ലാം നമസ്കാരം. 

