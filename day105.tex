\newpage
\section{ദിവസം 105}

\slokam{
ദൃഷ്ടാ ദൃശ്യതയാ തിഷ്ഠന്ത്രഷ്ടൃതാമുപജീവതിസത്യം കടകസംവിത്തൌ ഹേമ കാഞ്ചനതാമിവ  (3/81/80)\\
}

രാജാവു തുടര്‍ന്നു: മാമുനിമാര്‍ അകത്തേതും പുറത്തേതും എന്നൊക്കെപ്പറയുന്നത്‌ വെറും വാക്കുകള്‍ മാത്രം. അവയില്‍ 'വസ്തു' ഇല്ല. അത്തരം വിവരണങ്ങളെല്ലാം അജ്ഞാനികള്‍ക്കു മനസ്സിലാക്കാന്‍ വേണ്ടിയാണ്‌... ദൃഷ്ടാവ്‌, അദൃശ്യനായിരുന്നുകൊണ്ട്‌ സ്വയം കാണുന്നു. അയാള്‍ സ്വയം ബോധത്തിലെ ഒരു വിഷയമായിത്തീരുന്നില്ലതാനും. ദൃഷ്ടാവും ദൃശ്യവും ഒന്ന്. ലീനമായ മനോവാസനകള്‍ അവസാനിക്കുമ്പോള്‍ ദൃഷ്ടാവിനു ശുദ്ധവസ്തുപ്രാപ്തിയുണ്ടാവുന്നു. പുറംവസ്തുക്കളെ ഭാവന ചെയ്യുമ്പോള്‍ ദൃഷ്ടാവിന്റെ ജനനമായി. വിഷയി ഇല്ലെങ്കില്‍ , വിഷയം ഇല്ല. ദൃഷ്ടാവില്ലെങ്കില്‍ ദൃശ്യവുമില്ല. മകനാണല്ലോ ഒരാളെ അച്ഛനാക്കുന്നത്‌. ദൃഷ്ടാവാണ്‌ സ്വയം ദൃശ്യമാവുന്നത്‌. കാരണം, കാണുന്നവനില്ലെങ്കില്‍ കാണപ്പെടുന്ന വസ്തുവും ഇല്ല. അച്ഛനില്ലാതെ മകനുണ്ടാവുക സാദ്ധ്യമല്ലല്ലോ. ദൃഷ്ടാവെന്നത്‌ ശുദ്ധബോധമായതിനാല്‍ അതിന്‌ പദാര്‍ത്ഥങ്ങളെ ഇന്ദ്രജാലത്തില്‍ എന്നപോലെ ഉണ്ടാക്കിയെടുക്കാന്‍ ആവുന്നു. ഇതു തിരിച്ചാവുക വയ്യ. ദൃശ്യത്തിന്‌ ദൃഷ്ടാവിനെ സൃഷ്ടിക്കുക വയ്യ. അതുകൊണ്ട്‌ ദൃഷ്ടാവു മാത്രമാണുണ്മ. പദാര്‍ത്ഥങ്ങള്‍ വെറും ഭ്രമം. സ്വര്‍ണ്ണം മാത്രം സത്യം. കൈവള എന്നത്‌ ഒരു നാമവും രൂപവും. "കൈവളയിലെ ബോധസാന്നിദ്ധ്യം മൂലം സ്വര്‍ണ്ണം അതിന്റെ സ്വര്‍ണ്ണ സ്വരൂപം തിരിച്ചറിയുന്നു. ദൃഷ്ടാവ്‌ ദൃശ്യമായി പ്രകടിപ്പിക്കപ്പെടുമ്പോള്‍ സ്വയം ബോധസാക്ഷാത്കാരമുണ്ടാവുന്നു."

ഒന്ന് മറ്റേതിന്റെ പ്രതിഫലനമാണ്‌. അത്‌ ശരിക്കും ദ്വന്ദതയല്ല. ദൃഷ്ടാവ്‌ ദൃശ്യത്തിനെ കണ്ടുകൊണ്ടിരിക്കുമ്പോള്‍ സ്വയം (കാണുന്നില്ല) അറിയുന്നില്ല. അതിനാല്‍ സ്വയം ഉണ്മയാണെന്നറിയാതെ അയാഥാര്‍ത്ഥ്യമായി കാണപ്പെടുന്നു. എന്നാല്‍ ആത്മജ്ഞാനലാഭമുണ്ടായാല്‍ കാഴ്ച്ചകള്‍ ഇല്ലാതായി ദൃഷ്ടാവ്‌ സ്വയം പരമ്പൊരുളായി സാക്ഷാത്കാരം നേടുന്നു.

വിഷയം നിലനില്‍ക്കുന്നത്‌ വിഷയി ഉള്ളതുകൊണ്ടാണ്‌.. വിഷയം വിഷയിയിലെ ഒരു പ്രതിഫലനം മത്രം. രണ്ടില്ലാത്തയിടത്ത്‌ ദ്വന്ദത ഉണ്ടാവുകവയ്യ. ഒന്നേയുള്ളൂ എങ്കില്‍ ഏകാത്മകത എന്ന ആശയത്തിന്‌ എന്താണു പ്രസക്തി? ഇങ്ങിനെ വിചാരം ചെയ്തു ശരിയായ ആറിവുണര്‍ന്നാല്‍ , പരമവിജ്ഞാനമായി. അത്‌ വാക്കുകള്‍കൊണ്ട്‌ വിവരിക്കാവുന്നതല്ല. അത്‌ ഒന്നെന്നോ പലതെന്നോ പറയുക വയ്യ. അത്‌ ദൃശ്യമോ ദൃഷ്ടാവോ അല്ല. വിഷയമോ വിഷയിയോ അല്ല. അതും ഇതുമൊന്നും അല്ല. ഏകത്വമൊ അനേകത്വമോ സത്യമായി നിര്‍വ്വചിക്കാനാവില്ല. ഏതു നിലപാടെടുത്താലും അതിനൊരെതിര്‍ നിലപാടുണ്ടാവും. എന്നാല്‍ ഒന്ന് മറ്റൊന്നില്‍ നിന്നും വിഭിന്നമല്ല. അലകളില്‍നിന്നും ജലം വിഭിന്നമല്ലാത്തതുപോലെ. കൈവള സ്വര്‍ണ്ണത്തില്‍ നിന്നും വിഭിന്നമല്ലാത്തതുപോലെ. ഈ വിഭജനം എന്നതും ഏകാത്മകത എന്നതും തമ്മില്‍ വിരോധാഭാസമേതുമില്ല. നാനാത്വത്തേയും ഏകത്വത്തേയും കുറിച്ചുള്ള ഈ അനുമാനങ്ങളെല്ലാം ദു:ഖ നിവാരണത്തിനുള്ളതാണ്‌. ഇവകള്‍ക്കെല്ലാമപ്പുറമാണ്‌ പരം പൊരുളായ സത്യം.

