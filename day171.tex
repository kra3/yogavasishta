\section{ദിവസം 171}

\slokam{
ആചാരചാരുചിരതസ്യ വിവിക്തവൃത്തേ:\\
സംസാരസൗഖ്യഫലദു:ഖദശാസ്ത്രഗൃധ്നോ:\\
ആയുര്യശാംസി ച ഗുണാശ്ച സഹൈവ ലക്ഷ്മ്യാ\\
ഫുല്ലന്തി മാധവലതാ ഇവ സത്ഫലായ (4/32/60)\\
}

വസിഷ്ഠൻ തുടർന്നു: ശാസ്ത്രവിധിപ്രകാരം മുക്തിപദത്തിൽ ശ്രദ്ധയുറപ്പിച്ചവർ ഈ സംസാരമെന്ന പ്രത്യക്ഷലോകം തരണം ചെയ്യുന്നത് അവരുടെ ബോധം ആത്മാവിലേയ്ക്ക് ഒഴുകുമ്പോഴാണ്‌.. എന്നാൽ ദുരിതങ്ങൾക്കും ചിന്താക്കുഴപ്പങ്ങൾക്കും ഇടയാക്കുന്ന വിവാദവിഷയങ്ങളിൽ ആമഗ്നരായിരിക്കുന്നവർ അവരുടെതന്നെ ഏറ്റവും ഉന്നതമായ നന്മകളെ ഉപേക്ഷിക്കുന്നവരത്രേ. ശാസ്ത്രോക്തമായ പാതകൾ അനവധിയുണ്ടെങ്കിലും ഒരുവന്റെ നേരറിവും അനുഭവവും മാത്രമേ അവനെ പരമപദത്തിലേയ്ക്ക് നയിക്കുകയുള്ളു. എത്ര ദുരാഗ്രഹിയായ മനുഷ്യനും അവസാനം ബാക്കിവെയ്ക്കാൻ ഉള്ളത് ഒരു പിടി ചാരം മാത്രമാണല്ലോ. എന്നാൽ ഈ ലോകവിഷയങ്ങളെ തൃണവൽഗണിക്കുന്നവനെ ദു:ഖം തീണ്ടുകയില്ല. അനന്താവബോധം സാക്ഷാത്കരിച്ചവനു സംരക്ഷയേകാൻ വിശ്വദേവതകളുണ്ട്. അതുകൊണ്ട് അത്യാപത്തിന്റെ സമയത്തും ഒരുവൻ തെറ്റായ വഴികൾ തേടി പോകരുത്. പുണ്യകർമ്മങ്ങൾകൊണ്ടു സമ്പന്നമായ ജീവിതം നയിച്ച് സൽപ്പേരുണ്ടാക്കിയ ആൾക്ക് നേടാൻ കഴിയാതിരുന്നവ അചിരേണ നേടുവാനാകും. ദുർവിധിയെ വെല്ലാനുമാവും.

അവനവന്റെ സ്ഥിതിയിൽ അലംഭാവത്തോടെ തൃപ്തനായിരിക്കുന്നവൻ മനുഷ്യനാമത്തിനർഹനല്ല. സ്വന്തം അഭിവൃദ്ധിയിൽ എപ്പോഴും ജാഗരൂകനായിരിക്കുന്നവനേയും, താൻ പഠിച്ചത് മറ്റുള്ളവർക്കു പകർന്നു നൽകുന്നവനേയും, സത്യത്തിന്റെ പാത പിന്തുടരുന്നവനേയും മാത്രമേ മനുഷ്യനെന്ന് വിളിക്കാനാകൂ. മറ്റുള്ളവരെല്ലാം വേഷപ്രച്ഛന്നരായ മൃഗങ്ങൾ മാത്രം. മാനുഷീകതയുടെ ദുഗ്ദ്ധം ഹൃദയത്തിൽ നിറഞ്ഞവൻ ശ്രീഹരിയുടെ ഇരിപ്പിടമത്രേ. (ശ്രീഹരിയുടെ വാസം പാൽ ക്കടലിൽ ആണല്ലോ). വിജ്ഞാനിയായ ഒരാൾ ആസ്വദിക്കേണ്ടതെല്ലാം ആസ്വദിച്ചു കഴിഞ്ഞു; കാണേണ്ടതെല്ലാം കണ്ടും കഴിഞ്ഞു. അയാൾക്ക് ഈ ലോകത്തിൽ നിന്നും ഇനിയെന്തുണ്ട് നേടാൻ? അതുകൊണ്ട് എല്ലാ വിഷയസുഖാസക്തികളേയും വെടിഞ്ഞ് ശാസ്ത്രോക്തമായ കർമ്മങ്ങളിൽ ഏർപ്പെട്ടു കഴിയുന്നതാണുചിതം. മഹാത്മാക്കളെ പൂജിക്കൂ. അതുകൊണ്ട് നിനക്ക് മൃത്യുവിൽ നിന്നുപോലും രക്ഷ നേടാം. ശാസ്ത്രാനുസാരിയായ ജീവിത രീതിയും സാധനയും അനുഷ്ഠിച്ച്, ക്ഷമയോടെ പരിപൂർണ്ണതയ്ക്കായി കാത്തിരിക്കുക. സമയമാവുമ്പോൾ അതു സംഭവിക്കും. അധ:പ്പതനത്തിലേയ്ക്ക് നീങ്ങാതെയിരിക്കാൻ മഹത്തായ ദിവ്യഗ്രന്ഥങ്ങളുടെ പഠനം സഹായിക്കും.

ഇക്കാണുന്നതെല്ലാം വെറും പ്രതിഫലനം മാത്രം എന്ന അറിവോടെ സത്യത്തിന്റെ സ്വഭാവത്തെ അനവരതം ഉപാസിക്കുക. മറ്റുള്ളവരാൽ നയിക്കപ്പെടരുത്- അത് മൃഗങ്ങളുടെ രീതിയാണ്‌..  അജ്ഞതയുടെ നിദ്രയിൽ നിന്നും ഉണർന്നെണീറ്റാലും. ആ ഉണർച്ചയിൽ വാർദ്ധക്യത്തെയും മരണത്തെയും ജയിച്ചാലും. സമ്പത്താണ്‌ ദുഷ്ടതയുടെ മാതാവ്. ഇന്ദ്രിയസുഖങ്ങളാണ്‌ വേദനയുടെ ഉറവിടം. നിർഭാഗ്യം ഏറ്റവും വലിയ ഭാഗ്യം. പരിത്യജിക്കൽ ഏറ്റവും വലിയ വിജയം. “ജീവിതം, ബഹുമാനം, സദ്ഗുണങ്ങൾ എന്നിവ ഫലപ്രാപ്തിയെത്തുന്നത് ഒരുവന്റെ സ്വഭാവവും പ്രവർത്തനവും നന്മനിറഞ്ഞതാവുമ്പോഴാണ്‌..  അയാള്‍ ഏകാന്തനായി വസിക്കുന്നു. ദു:ഖ ദുരിതങ്ങൾക്ക് കാരണമാകുന്ന ലോകസുഖങ്ങളിൽ അവന്‌ ആസക്തിയേതുമില്ല.” 

