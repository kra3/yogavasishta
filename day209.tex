\section{ദിവസം 209}

\slokam{
ചിത്ത ചഞ്ചല സംസാര ആത്മനോ ന സുഖായ തേ\\
ശമമേഹി ശമാച്ഛാന്തം സുഖം സാരമവാപ്യതേ (5/11/5)\\
}

വസിഷ്ഠൻ തുടർന്നു: ഇങ്ങിനെ ചിന്തിച്ച് ജനകരാജാവ് അവിടെ നിന്നെഴുന്നേറ്റു. ദിവസവും സൂര്യൻ ചക്രവാളത്തിലുദിക്കുമ്പോള്‍ മുതൽ തന്റെ രാജധർമ്മങ്ങൾ യാതൊരാസക്തിയുമില്ലാതെ അദ്ദേഹം  നിർവ്വഹിച്ചു വന്നു. എല്ലാ മനോപാധികളിൽ നിന്നും, ധാരണാസങ്കൽപ്പ-വികൽപ്പങ്ങളിൽ നിന്നും അദ്ദേഹം മുക്തനായിരുന്നു. തികച്ചും ജാഗ്രദാവസ്ഥയിലാണെങ്കിലും ദീർഘനിദ്രയിലെ ശരീരകർമ്മങ്ങളെന്നപ്പോലെ നിസ്സംഗതയോടെ ഉത്തമവും ഉചിതവുമായ കർമ്മങ്ങളിലേർപ്പെട്ട് അദ്ദേഹം രാജധർമ്മം പരിപാലിച്ചു. മാമുനിമാരെ ആദരിക്കുന്നതുൾപ്പടെയുള്ള നിത്യകർമ്മങ്ങളനുഷ്ഠിച്ച് ദിനാന്ത്യത്തിൽ തന്റെ പള്ളിയറയിലെ ഏകാന്തതയിൽ ധ്യാനനിരതനായിക്കഴിഞ്ഞു. അതദ്ദേഹത്തിനു സഹജമായും ക്ഷിപ്രസാദ്ധ്യമായിരുന്നു. അദ്ദേഹത്തിന്റെ മനസ്സ് സ്വാഭാവികമായും കാലുഷ്യമകന്ന്, ഭ്രമങ്ങളൊഴിഞ്ഞ് സമതയിൽ അഭിരമിച്ചിരുന്നുവല്ലോ.

പ്രഭാതത്തിലുണർന്നപ്പോൾ അദ്ദേഹം ഇങ്ങിനെ ചിന്തിച്ചു: “ചഞ്ചലമായ മനസ്സേ! ഈ ലൗകീകജീവിതം നിനക്കു ശരിയായ സന്തുഷ്ടിയേകാൻ പര്യാപ്തമല്ല. അതുകൊണ്ട് നീ സമതാഭാവമെന്ന ആ ഉന്നത തലത്തിലേയ്ക്കെത്തുക. കാരണം ഈ സമതയുടെ തലത്തിൽ എത്തിയാൽ മാത്രമേ നിനക്ക് സമാധാനവും, ആനന്ദവും സത്യവും അനുഭവിക്കാനാവൂ.” നിന്നിലെ അച്ചടക്കമില്ലായ്മ കാരണം ഉണ്ടാവുന്ന വിചിത്രവും വികലവുമായ ചിന്തകൾ കാരണമാണ് ലോകമെന്ന  ഭ്രമക്കാഴ്ച്ച ഇങ്ങിനെ നീണ്ടുപരന്നു വികസിക്കുന്നത്. നിന്നിൽ സുഖാസക്തിയുണ്ടാകുമ്പോഴാണ്‌ ഈ മായക്കാഴ്ച്ചയ്ക്ക് എണ്ണമില്ലാത്തത്ര ശാഖകൾ മുളപൊട്ടുന്നത്. ചിന്തകളാണ്‌ ലോകമെന്ന ഈ കാഴ്ച്ചാശൃംഖലയ്ക്കു നിദാനം. അതുകൊണ്ട് ഈ ഭ്രമകൽപ്പനകളെ ഉപേക്ഷിച്ച് സമതയിലെത്തൂ. നിന്റെ ജ്ഞാനത്തെ ഒരു തുലാസ്സിൽ വെച്ചു തൂക്കിനോക്കുക. ഒരുതട്ടിൽ ഇന്ദ്രിയ സുഖങ്ങൾ, മറ്റേ തട്ടിൽ പ്രശാന്തിയുടെ ആനന്ദം. ഇവയിൽ സത്യമേതെന്നു നീ കരുതുന്നുവോ അതിന്റെ പാത സ്വീകരിക്കുക.

എല്ലാ പ്രതീക്ഷകളും ആശകളും ഉപേക്ഷിച്ച്, അന്വേഷണത്വരയും സന്യാസത്വരപോലും  കളഞ്ഞ് സ്വാതന്ത്ര്യത്തോടെ നടക്കൂ. ഈ ലോകമെന്നത് സത്തോ അസത്തോ ആകട്ടെ. അതുദിക്കുകയോ അസ്തമിക്കുകയോ ചെയ്യട്ടേ! അവയുടെ ഗുണാഗുണങ്ങൾ നിന്റെ സമതാ ഭാവത്തെ ബാധിക്കാതിരിക്കട്ടെ. നിനക്ക് ഈ ലോകവുമായി ഒരിക്കലും ശരിയായ ഒരു ബന്ധവുമുണ്ടായിട്ടില്ല. നിന്റെ അജ്ഞാനം കൊണ്ടാണ്‌ നിന്നിൽ അങ്ങിനെയൊരു ബന്ധുത്വം ഉണ്ടെന്ന തോന്നലുദിച്ചത്. മനസ്സേ നീ വെറും മിഥ്യയാണ്‌.. ലോകമെന്ന ഈ കെട്ടുകാഴ്ച്ചയും മിഥ്യ. അതിനാൽ നിങ്ങൾതമ്മിലുള്ള ബന്ധം വളരെ വിചിത്രമാണ്‌.. വന്ധ്യയുടെ പുത്രൻ എന്നു പറയും പോലെ അസംബന്ധം!

നീ സത്തും, ലോകം അസത്തും ആണ് എന്നു നീ കരുതുന്നതെങ്കിൽ നിങ്ങൾ തമ്മിൽ എങ്ങിനെയാണൊരു ബന്ധുതയുണ്ടാവുക? മറിച്ച് രണ്ടും സത്യമാണെങ്കിൽ സുഖദു:ഖങ്ങളെന്ന അനുഭവങ്ങൾക്ക് എന്താണൊരു ന്യായീകരണം? അതുകൊണ്ട് എല്ലാ ദു:ഖചിന്തകളുമുപേക്ഷിക്കൂ, ധ്യാനത്തിൽ ആമഗ്നമാകൂ. ഇഹലോകത്തിൽ നിന്നെ പരിപൂർണ്ണതയിലേയ്ക്കു നയിക്കാൻ തക്കതായി യാതൊന്നുമില്ല. അതുകൊണ്ട് സധീരം, ദൃഢതയോടെ നിന്റെ മനശ്ചാഞ്ചല്യത്തെ മറികടന്ന് ധ്യാനത്തിലൂടെ സമതയിൽ അഭിരമിച്ചാലും. 

